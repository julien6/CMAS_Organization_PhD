\documentclass[12pt,a4paper]{article}

% --- Encoding & language
\usepackage[utf8]{inputenc}
\usepackage[T1]{fontenc}
\usepackage[english]{babel}

% --- Margins
\usepackage[margin=1.5cm]{geometry}

% --- Common packages
\usepackage{graphicx}
\usepackage{ragged2e}
\usepackage{hyperref}
\usepackage[inkscapeformat=png]{svg}
\usepackage{setspace}
\usepackage{fontawesome5}

% --- Hyperlinks
\hypersetup{
    colorlinks=true,
    linkcolor=blue,
    urlcolor=blue
}

\begin{document}

%========================================
% Centered title and author with superscripts
%========================================
\begin{center}
    \Huge\bfseries On the Organization of a Cyberdefense Multi-Agent System\\[0.8cm]
    \LARGE Julien Soulé$^{1,2,3}$\\[0.6cm]
    \normalsize
\end{center}

%========================================
% Logos
%========================================
\begin{figure}[ht!]
    \centering
    \begin{minipage}{0.25\textwidth}
        \centering
        \includesvg[height=0.4cm]{logos/thales_logo.svg}
    \end{minipage}\hfill
    \begin{minipage}{0.25\textwidth}
        \centering
        \includegraphics[height=1.2cm]{logos/la-ruche_logo.png}
    \end{minipage}\hfill
    \begin{minipage}{0.25\textwidth}
        \centering
        \includegraphics[height=0.6cm]{logos/lcis_logo.png}
    \end{minipage}\hfill
    \begin{minipage}{0.25\textwidth}
        \centering
        \includegraphics[height=1cm]{logos/uga_logo.jpg}
    \end{minipage}
\end{figure}

%========================================
% Updated Supervision
%========================================

\begin{center}
    \textbf{Jean-Paul Jamont}$^{1}$ — Thesis Director\\
    \textbf{Michel Occello}$^{1}$ — Co-supervisor\\
    \textbf{Louis-Marie Traonouez}$^{2}$ — Supervisor\\
    \textbf{Paul Theron}$^{2,3}$ — Supervisor\\
\end{center}

%========================================
% Full affiliations
%========================================
\begin{center}
    {\footnotesize $^{1}$ \textit{Université Grenoble Alpes, Grenoble INP, LCIS, 26000, Valence, France}}\\[0.1cm]
    {\footnotesize $^{2}$ \textit{Thales Land and Air Systems, BU IAS, 35000, Rennes, France}}\\[0.1cm]
    {\footnotesize $^{3}$ \textit{AICA IWG, 06600, Antibes, France}}\\[1cm]
\end{center}


%========================================
% Abstract
%========================================
\noindent\textbf{Abstract}\\[0.3cm]
\justifying
In the face of growing complexity in cybersecurity threats, centralized approaches are showing their limitations in effectively protecting distributed and dynamic systems.
This thesis explores a distributed approach based on Multi-Agent Systems (MASs) capable of collectively detecting, responding to, and adapting to autonomous and evolving attacks.
The central objective is to enable the design of a cyber defense MAS by identifying an organizational mechanism suited to the constraints of both the designers and the environment.
The literature highlights symbolic approaches that favor control and connectionist approaches that emphasize performance.
To overcome this tension, the thesis proposes a hybrid method combining a symbolic organizational model with Multi-Agent Reinforcement Learning (MARL).

\medskip
The key idea of this method is to view the design of an MAS as a constrained optimization problem,
in which agents' joint policy is learned while respecting organizational constraints reflecting the designer’s requirements.
This approach requires both accurate modeling of the environment and the ability to analyze and control the resulting behaviors.
The method integrates four activities:
(i) modeling (World Models or manual modeling);
(ii) constrained training (MOISE+MARL);
(iii) analysis of learned policies (extraction of implicit roles and objectives via unsupervised methods);
(iv) transfer to the real environment (continuous updating of models and policies).

\medskip
A software tool was developed to implement this method and applied to three use cases:
a drone swarm, an enterprise infrastructure, and a microservice architecture.
The results show improvements in resilience, adaptability, and autonomy compared to centralized approaches.
Future work includes: improving modeling through expert knowledge integration,
making learning more robust in dynamic environments,
and exploring latent representations to facilitate organizational analysis.

\vfill

%========================================
% Contact (optional)
%========================================
\noindent\textbf{Contact:} \href{mailto:julien.soule@univ-grenoble-alpes.fr}{julien.soule@univ-grenoble-alpes.fr}
\hfill
\href{https://julien6.github.io/home/}{\textrm{\faGlobe}\;https://julien6.github.io/home/}

\end{document}
