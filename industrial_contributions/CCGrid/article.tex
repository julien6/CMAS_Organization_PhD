\documentclass[conference]{IEEEtran}
\IEEEoverridecommandlockouts
% The preceding line is only needed to identify funding in the first footnote. If that is unneeded, please comment it out.
\usepackage{amsmath,amssymb,amsfonts}
\usepackage{algorithmic}
\usepackage{graphicx}
\usepackage[inline, shortlabels]{enumitem}
\usepackage{tabularx}
\usepackage{caption}
\usepackage{titlesec}
\usepackage[T2A,T1]{fontenc}
\usepackage[english]{babel}
\captionsetup{font=it}
\usepackage{ragged2e}
\usepackage{hyperref}
\addto\extrasenglish{%
  \renewcommand{\sectionautorefname}{Section}%
  \renewcommand{\subsectionautorefname}{Subsection}%
  \renewcommand{\subsubsectionautorefname}{Subsubsection}%
  \renewcommand{\tableautorefname}{Table}%
  \renewcommand{\figureautorefname}{Figure}%
}
\usepackage{pifont}
\newcommand{\cmark}{\ding{51}}%
\newcommand{\xmark}{\ding{55}}%
\usepackage{footmisc}
\usepackage{multirow}

% --- Tickz
\usepackage{physics}
\usepackage{amsmath}
\usepackage{tikz}
\usepackage{mathdots}
\usepackage{yhmath}
\usepackage{cancel}
\usepackage{color}
\usepackage{siunitx}
\usepackage{array}
\usepackage{multirow}
\usepackage{amssymb}
\usepackage{gensymb}
\usepackage{tabularx}
\usepackage{extarrows}
\usepackage{booktabs}
\usetikzlibrary{fadings}
\usetikzlibrary{patterns}
\usetikzlibrary{shadows.blur}
\usetikzlibrary{shapes}

% ---------

\usepackage{pdfpages}
\usepackage{booktabs}
\usepackage{csquotes}
\usepackage{lipsum}  
\usepackage{arydshln}
\usepackage{smartdiagram}
\usepackage[inkscapeformat=png]{svg}
\usepackage{textcomp}
\usepackage{tabularray}\UseTblrLibrary{varwidth}
\usepackage{xcolor}
\def\BibTeX{{\rm B\kern-.05em{\sc i\kern-.025em b}\kern-.08em
    T\kern-.1667em\lower.7ex\hbox{E}\kern-.125emX}}
\usepackage{cite}
\usepackage{amsmath}
\newcommand{\probP}{\text{I\kern-0.15em P}}
\usepackage{etoolbox}
\patchcmd{\thebibliography}{\section*{\refname}}{}{}{}

\setlength{\extrarowheight}{2.5pt}

% \renewcommand{\arraystretch}{1.7}

% \setlength{\extrarowheight}{2.5pt}
% \renewcommand{\arraystretch}{0.2}
% \renewcommand{\arraystretch}{1.7}

% --------------
\titleclass{\subsubsubsection}{straight}[\subsection]

\newcounter{subsubsubsection}[subsubsection]
\renewcommand\thesubsubsubsection{\thesubsubsection.\arabic{subsubsubsection}}
\renewcommand\theparagraph{\thesubsubsubsection.\arabic{paragraph}} % optional; useful if paragraphs are to be numbered

\titleformat{\subsubsubsection}
  {\normalfont\normalsize\bfseries}{\thesubsubsubsection}{1em}{}
\titlespacing*{\subsubsubsection}
{0pt}{3.25ex plus 1ex minus .2ex}{1.5ex plus .2ex}

\makeatletter
\renewcommand\paragraph{\@startsection{paragraph}{5}{\z@}%
  {3.25ex \@plus1ex \@minus.2ex}%
  {-1em}%
  {\normalfont\normalsize\bfseries}}
\renewcommand\subparagraph{\@startsection{subparagraph}{6}{\parindent}%
  {3.25ex \@plus1ex \@minus .2ex}%
  {-1em}%
  {\normalfont\normalsize\bfseries}}
\def\toclevel@subsubsubsection{4}
\def\toclevel@paragraph{5}
\def\toclevel@paragraph{6}
\def\l@subsubsubsection{\@dottedtocline{4}{7em}{4em}}
\def\l@paragraph{\@dottedtocline{5}{10em}{5em}}
\def\l@subparagraph{\@dottedtocline{6}{14em}{6em}}
\makeatother

\setcounter{secnumdepth}{4}
\setcounter{tocdepth}{4}
% --------------


\newcommand{\before}[1]{\textcolor{red}{#1}}
\newcommand{\after}[1]{\textcolor{green}{#1}}

\newcommand{\old}[1]{\textcolor{orange}{#1}}
\newcommand{\rem}[1]{\textcolor{red}{#1}}
\newcommand{\todo}[1]{\textcolor{orange}{\newline \textit{\textbf{TODO:} #1}} \newline \newline }

\makeatletter
\newcommand{\linebreakand}{%
  \end{@IEEEauthorhalign}
  \hfill\mbox{}\par
  \mbox{}\hfill\begin{@IEEEauthorhalign}
}
\makeatother




% ---------------------------


\begin{document}

\title{Streamlining Resilient Kubernetes Autoscaling with Multi-Agent Systems via an Automated Online Design Framework\\
    % {\footnotesize \textsuperscript{Note}}
    % \thanks{Identify applicable funding agency here. If none, delete this.}
}

% \IEEEaftertitletext{\vspace{-1\baselineskip}}

\author{

    \IEEEauthorblockN{Julien Soulé}
    \IEEEauthorblockA{\textit{Thales Land and Air Systems, BU IAS}}
    %Rennes, France \\
    \IEEEauthorblockA{\textit{Univ. Grenoble Alpes,} \\
        \textit{Grenoble INP, LCIS, 26000,}\\
        Valence, France \\
        julien.soule@lcis.grenoble-inp.fr}

    \and

    \IEEEauthorblockN{Jean-Paul Jamont\IEEEauthorrefmark{1}, Michel Occello\IEEEauthorrefmark{2}}
    \IEEEauthorblockA{\textit{Univ. Grenoble Alpes,} \\
        \textit{Grenoble INP, LCIS, 26000,}\\
        Valence, France \\
        \{\IEEEauthorrefmark{1}jean-paul.jamont,\IEEEauthorrefmark{2}michel.occello\}@lcis.grenoble-inp.fr
    }

    % \and

    % \IEEEauthorblockN{Michel Occello}
    % \IEEEauthorblockA{\textit{Univ. Grenoble Alpes,} \\
    % \textit{Grenoble INP, LCIS, 26000,}\\
    % Valence, France \\
    % michel.occello@lcis.grenoble-inp.fr}

    % \and

    \linebreakand

    \hspace{-0.5cm}
    \IEEEauthorblockN{Paul Théron}
    \IEEEauthorblockA{
        \hspace{-0.5cm}
        \textit{AICA IWG} \\
        \hspace{-0.5cm}
        La Guillermie, France \\
        \hspace{-0.5cm}
        %lieu-dit Le Bourg, France \\
        paul.theron@orange.fr}

    \and

    \hspace{0.5cm}
    \IEEEauthorblockN{Louis-Marie Traonouez}
    \IEEEauthorblockA{
        \hspace{0.5cm}
        \textit{Thales Land and Air Systems, BU IAS} \\
        \hspace{0.5cm}
        Rennes, France \\
        \hspace{0.5cm}
        louis-marie.traonouez@thalesgroup.com}}


\maketitle

\begin{abstract}
    In cloud-native critical systems relying on complex Kubernetes clusters composed of interdependent services, poor workload management can compromise cluster availability through various failures such as resource blocking, bottlenecks, or continuous pod crashes. Conventional Horizontal Pod Autoscaling (HPA) approaches often fall short in such dynamic environments, while reinforcement learning-based ones, though more adaptable, typically focus on a single latency or resource minimization objective without explicitly addressing all of the known failures.
    A Multi-Agent System (MAS) enables resilient Kubernetes HPA by decomposing the availability maximization objective into failure-related sub-objectives delegated to agents. We streamline the generation of such MASs through an online automated cycle in four phases: (1) modeling the cluster as a simulation from collected real cluster traces; (2) training agents in simulation, partially guided by roles and missions incorporating knowledge of failures; (3) optionally validating the trained agents' behaviors and extracting design insights; and (4) transferring the learned behaviors to the real cluster.
    Experimental results show that the generated MASs are original and outperform four cutting-edge HPA approaches in two different scenarios.
\end{abstract}

\begin{IEEEkeywords}
    cyberdefense, MARL, Digital Twins, formal
\end{IEEEkeywords}

\section{Introduction}
\label{sec:introduction}

% Contexte and Motivation : Présenter les systèmes critiques cloud-native, Kubernetes, et les défis de l'autoscaling

% Problème : pourquoi il est crucial de traiter les échecs spécifiques connus.

% Contribution : Résumé des contributions principales.

% Organisation du papier

\section{Background and Related Work}
\label{sec:related_work}
\subsection{Kubernetes Autoscaling and Its Challenges}
% Présentez le fonctionnement de l'HPA et ses limites.
\subsection{Reinforcement Learning-Based Approaches}
% Discutez des méthodes d’apprentissage par renforcement.
\subsection{Multi-Agent Systems in Distributed Environments}
% Introduisez les travaux liés aux systèmes multi-agents.
\subsection{Our Contribution vs State-of-the-Art}
% Comparez votre approche avec l'état de l'art.

\section{Proposed Approach: Multi-Agent System for Resilient HPA}
\label{sec:proposed_approach}
\subsection{Overview of the System}
% Description générale de votre approche et de son architecture.
\subsection{Phase 1: Cluster Simulation (Digital Twin)}
% Comment le jumeau numérique est modélisé à partir des traces.
\subsection{Phase 2: Agent Training with Failure Knowledge}
% Méthodologie d'entraînement des agents guidés par des missions.
\subsection{Phase 3: Agent Validation and Design Insights}
% Validation et interprétation des comportements des agents.
\subsection{Phase 4: Behavior Transfer to Real Cluster}
% Processus de transfert des comportements appris au cluster réel.

\section{Experimental Setup}
\label{sec:experiments}
\subsection{Experimental Configuration}
% Environnement expérimental et outils utilisés.
\subsection{Test Scenarios}
% Présentation des scénarios utilisés pour l'évaluation.
\subsection{Evaluation Metrics}
% Protocole d'experimentation reproductible
\subsection{Experimental Protocol}

\section{Results and Discussion}
\label{sec:results}
% Explication des métriques utilisées (ex : disponibilité, latence).
\subsection{Results and Comparisons}
% Résultats obtenus et comparaison avec les approches existantes.
\subsection{Discussion of Results}
% Analyse des performances et des points clés des résultats.

\section{Discussion}
\label{sec:discussion}
\subsection{Practical Implications}
% Comment votre approche peut être utilisée dans des environnements réels.
\subsection{Limitations}
% Limites actuelles de votre méthode.
\subsection{Future Directions}
% Perspectives pour des travaux futurs.

\section*{References}

\nocite{alDhuraibi2017elasticDocker}

% \bibliographystyle{abbrv}
\bibliographystyle{IEEEtran}

\bibliography{references}

\end{document}
