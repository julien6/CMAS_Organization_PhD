\documentclass[9pt, aspectratio=169]{beamer}
% \documentclass[10pt]{beamer}
\usepackage[utf8]{inputenc}
\usepackage[T1]{fontenc}
\usepackage[english]{babel}
\usetheme{Frankfurt}

\usepackage[backend=biber, style=authoryear]{biblatex}
\addbibresource{local_references.bib}

%\usepackage{lmodern}
\usepackage{amsfonts,amssymb,amsmath}
\usepackage[english]{babel}
\usetheme{Frankfurt}

\usepackage{csquotes}
\usepackage{setspace}

\usepackage{colortbl}
\usepackage{tabularx}
\renewcommand\tabularxcolumn[1]{m{#1}}

% --- Tickz
\usepackage{physics}
\usepackage{amsmath}
\usepackage{tikz}
\usepackage{mathdots}
\usepackage{yhmath}
\usepackage{cancel}
\usepackage{color}
\usepackage{siunitx}
\usepackage{array}
\usepackage{multirow}
\usepackage{amssymb}
\usepackage{gensymb}
\usepackage{tabularx}
\usepackage{extarrows}
\usepackage{booktabs}
\usetikzlibrary{fadings}
\usetikzlibrary{patterns}
\usetikzlibrary{shadows.blur}
\usetikzlibrary{shapes}

% ---------

\usepackage{booktabs}
\usepackage{setspace}
\usepackage{amssymb}
\usepackage{adjustbox}
\usepackage{pifont}
\usepackage[inkscapeformat=png]{svg}
\usepackage{graphicx}
\usepackage{times}
\setbeamertemplate{caption}[numbered]
% % \setbeamertemplate{bibliography item}{[\theenumiv]}

\setbeamerfont{bibliography item}{size=\tiny}
\setbeamerfont{bibliography entry author}{size=\tiny}
\setbeamerfont{bibliography entry title}{size=\tiny}
\setbeamerfont{bibliography entry location}{size=\tiny}
\setbeamerfont{bibliography entry note}{size=\tiny}

\setbeamerfont{frametitle}{size=\large}

\usepackage{caption}
\usepackage{float}
\usepackage{xcolor}
\usepackage{listings}
\usepackage{animate}

\definecolor{codegreen}{rgb}{0,0.6,0}
\definecolor{codegray}{rgb}{0.5,0.5,0.5}
\definecolor{codepurple}{rgb}{0.58,0,0.82}
\definecolor{backcolour}{rgb}{0.95,0.95,0.92}
 
\lstdefinestyle{mystyle}{
    backgroundcolor=\color{backcolour},   
    commentstyle=\color{codegreen},
    keywordstyle=\color{magenta},
    numberstyle=\tiny\color{codegray},
    stringstyle=\color{codepurple},
    basicstyle=\footnotesize,
    breakatwhitespace=false,         
    breaklines=true,                 
    captionpos=b,                    
    keepspaces=true,                 
    numbers=left,                    
    numbersep=5pt,                  
    showspaces=false,                
    showstringspaces=false,
    showtabs=false,                  
    tabsize=2
}
 
\lstset{style=mystyle}

\usepackage{ragged2e}
\setbeamercolor{section in foot}{fg=white,bg=darkorange}
\setbeamercolor{subsection in foot}{fg=white,bg=darkorange}
\setbeamercolor{frametitle}{fg=white, bg=darkorange}
\setbeamercolor{title}{fg=white, bg=darkorange}
\setbeamercolor{frame}{bg=darkorange}
\setbeamercolor{block title}{bg=darkorange,fg=white}

\setbeamercolor{item}{fg=darkorange}

% \definecolor{darkorange}{rgb}{0.81, 0.52, 0.05}
\definecolor{darkorange}{rgb}{1,0.5,0}
\definecolor{darkorange2}{rgb}{1, 0.64, 0.2}
\definecolor{honeydew}{rgb}{1, 0.85, 0.45}


\newenvironment{variableblock}[3]{%
  \setbeamercolor{block body}{#2}
  \setbeamercolor{block title}{#3}
  \begin{block}{#1}}{\end{block}}

\newenvironment{prosblock}[1]{%
  % \setbeamercolor{block body}{bg=blue,fg=white}
  \setbeamercolor{block title}{bg=blue,fg=white}
  \begin{block}{#1}}{\end{block}}

\newenvironment{consblock}[1]{%
  % \setbeamercolor{block body}{bg=red,fg=white}
  \setbeamercolor{block title}{bg=red,fg=white}
  \begin{block}{#1}}{\end{block}}

\newcommand{\cmark}{\ding{51}}%
\newcommand{\xmark}{\ding{55}}%

\renewcommand{\arraystretch}{1.5}

% Please add the following required packages to your document preamble:
\usepackage{booktabs}
\usepackage{multirow}
\usepackage{colortbl}
% Beamer presentation requires \usepackage{colortbl} instead of \usepackage[table,xcdraw]{xcolor}

\usepackage{tabularray}\UseTblrLibrary{varwidth}
\usepackage{xcolor}
\def\BibTeX{{\rm B\kern-.05em{\sc i\kern-.025em b}\kern-.08em
    T\kern-.1667em\lower.7ex\hbox{E}\kern-.125emX}}
% \usepackage{cite}
\usepackage{amsmath}
\newcommand{\probP}{\text{I\kern-0.15em P}}
\usepackage{etoolbox}
\patchcmd{\thebibliography}{\section*{\refname}}{}{}{}

\setlength\tabcolsep{0.5pt}

\renewcommand{\arraystretch}{0.9}
\setlength{\tabcolsep}{2pt}

\usepackage{pgffor}

\setbeamerfont{bibliography item}{size=\tiny}
\setbeamerfont{bibliography entry author}{size=\tiny}
\setbeamerfont{bibliography entry title}{size=\tiny}
\setbeamerfont{bibliography entry location}{size=\tiny}
\setbeamerfont{bibliography entry note}{size=\tiny}

\setbeamerfont{bibliography entry author}{shape=\upshape,series=\mdseries,size=\footnotesize}
\setbeamerfont{bibliography entry title}{shape=\slshape,series=\mdseries,size=\footnotesize}
\setbeamerfont{bibliography entry journal}{shape=\upshape,series=\mdseries,size=\footnotesize}
\setbeamerfont{bibliography entry note}{shape=\upshape,series=\mdseries,size=\footnotesize}

\renewcommand*{\bibfont}{\scriptsize}

\newenvironment<>{varblock}[2][.9\textwidth]{%
  \setlength{\textwidth}{#1}
  \begin{actionenv}#3%
    \def\insertblocktitle{#2}%
    \par%
    \usebeamertemplate{block begin}}
  {\par%
    \usebeamertemplate{block end}%
  \end{actionenv}}

% \setbeamertemplate{footline}[frame number]

\setbeamertemplate{footline}{
  \leavevmode%
  \hfill
  \usebeamercolor[fg]{page number in head/foot}%
  \scriptsize%
  \ifnum\value{framenumber}>16%
    Appendix \number\numexpr\value{framenumber}-13\relax/32%
  \else%
    \ifnum\value{framenumber}>13%
      %
    \else
      \number\numexpr\value{framenumber}\relax/13%
    \fi

  \fi%
  \hspace{1em}
}

\begin{document}

\author{\textbf{Julien Soulé$^{1,2}$}, Jean-Paul Jamont$^1$, Michel Occello$^1$, Louis-Marie Traonouez$^2$, Paul Théron$^3$}

\title{\textbf{Towards Assisted MAS Design: A Library for
Explainable MARL with Organizational Model}}

\subtitle{ECAI 2024 Demo Presentation}

% \logo{\includegraphics[scale=0.01]{figures/grenoble-inp_logo.png}}

\institute{\footnotesize \textit{University Grenoble Alpes, Grenoble INP, LCIS, 26000, Valence, France \\
$^1$\{julien.soule, jean-paul.jamont, michel.occello\}@lcis.grenoble-inp.fr \\ \phantom{U} \\
Thales Land and Air Systems, BL IAS, 35000, Rennes, France \\
$^2$\{julien.soule, louis-marie.traonouez\}@thalesgroup.com \\ \phantom{U} \\
AICA IWG, La Guillermie, France \\
$^3$paul.theron@orange.fr}}


\date{\textit{\footnotesize May 9, 2024}}

%\subject{}
\setbeamercovered{transparent}
%\setbeamertemplate{navigation symbols}{}
\begin{frame}[plain]
	\maketitle\vspace{-0.8cm}
	\begin{figure}[ht!]
		\centering
            \includegraphics[height=0.8cm]{figures/la-ruche_logo.png}
            \hspace{0.8cm}
            \includegraphics[height=0.8cm]{figures/lcis_logo.png}
            \hspace{0.8cm}
		\includegraphics[height=0.8cm]{figures/grenoble-inp_logo.png}
            \hspace{0.8cm}
            \includegraphics[height=0.8cm]{figures/uga_logo.jpg}
	\end{figure}
\end{frame}

% \begin{frame}{Content}
%   \tableofcontents
% \end{frame}

\addtocounter{framenumber}{-1}

\section{Introduction}

\begin{frame}{Context and Problem}
  \begin{columns}
    \column{0.4\textwidth}
    \begin{itemize}
      \item Kubernetes clusters often fail under dynamic conditions (e.g., DDoS, bottlenecks).
      \item Traditional HPA is reactive and struggles with complex failure contexts.
      \item RL-based autoscalers optimize single metrics (e.g., latency), ignoring failure diversity.
      \item \textbf{Need}: A resilient autoscaling system addressing \textit{multiple failure types} collaboratively.
    \end{itemize}

    \column{0.7\textwidth}
    \includegraphics[width=\linewidth]{figures/scenario_introduction.pdf}
  \end{columns}
\end{frame}

\begin{frame}{Related Works}
  \begin{columns}
    \column{0.5\textwidth}
    \begin{itemize}
      \item \textbf{Gym-HPA}~\parencite{gymhpa2022}, \textbf{QoS-RL}~\parencite{QoSRL}: learning-based, but limited to simulated settings and single-goal.
      \item \textbf{AWARE}~\parencite{aware2023}: partial adversarial support, no multi-agent reasoning.
    \end{itemize}

    \column{0.5\textwidth}
    \begin{itemize}
      \item \textbf{RLad-core}~\parencite{Rossi2019}: adaptive mono-agent scaling, lacks explainability.
      \item \textbf{AHPA}~\parencite{Zhou2024}, \textbf{KOSMOS}~\parencite{KOSMOS}, \textbf{COPA}~\parencite{COPA}: hybrid approaches, rule-based or reactive, non-learning.
    \end{itemize}

  \end{columns}

  \vspace{0.5cm}

  \begin{center}

    \centering
    \vfill
    \footnotesize
    \renewcommand{\arraystretch}{1.2}
    \begin{tabular}{lccccccc}
      \hline
      \textbf{Criterion}   & \textbf{Gym-HPA} & \textbf{AWARE} & \textbf{RLad} & \textbf{QoS-RL} & \textbf{AHPA} & \textbf{KOSMOS} & \textbf{COPA} \\
      \hline
      Adversarial Handling & \xmark           & $\sim$         & \xmark        & \xmark          & \xmark        & \xmark          & $\sim$        \\
      Multi-Goal Support   & \xmark           & \cmark         & $\sim$        & \cmark          & \xmark        & \cmark          & \xmark        \\
      Learning-Based       & \cmark           & \cmark         & \cmark        & \cmark          & \xmark        & \xmark          & \xmark        \\
      MAS Support          & \xmark           & \xmark         & \xmark        & \xmark          & \xmark        & \xmark          & \xmark        \\
      Real Env. Ready      & \xmark           & \cmark         & \cmark        & \cmark          & \cmark        & \cmark          & \cmark        \\
      Explainable          & \xmark           & \xmark         & \xmark        & \xmark          & \xmark        & \xmark          & \xmark        \\
      Adaptation Capacity  & \cmark           & $\sim$         & $\sim$        & $\sim$          & \cmark        & \cmark          & $\sim$        \\
      \hline
    \end{tabular}
  \end{center}

  \vspace{0.5cm}

  $\rightarrow$ \textbf{Most systems lack}: MAS coordination, explainability, and resilience to adversarial conditions.

\end{frame}

\begin{frame}{Our Proposition: KARMA Framework}
  \begin{columns}
    \column{0.3\textwidth}
    \begin{itemize}
      \item We introduce \textbf{KARMA}: a 4-phase framework for resilient autoscaling.
      \item Decomposes autoscaling into specialized agent roles and missions.
      \item Uses \textbf{Multi-Agent Reinforcement Learning} (MARL) guided by organizational constraints.
      \item Operates in a closed-loop with the real Kubernetes cluster.
    \end{itemize}

    \column{0.75\textwidth}
    \centering
    


\tikzset{every picture/.style={line width=0.75pt}} %set default line width to 0.75pt        

\begin{tikzpicture}[x=0.75pt,y=0.75pt,yscale=-1.2,xscale=1.2]
%uncomment if require: \path (0,1414); %set diagram left start at 0, and has height of 1414

%Straight Lines [id:da5609883377896374] 
\draw [color={rgb, 255:red, 74; green, 144; blue, 226 }  ,draw opacity=1 ][line width=2.25]    (317.22,111.13) -- (360.07,111.13) ;
\draw [shift={(365.07,111.13)}, rotate = 180] [fill={rgb, 255:red, 74; green, 144; blue, 226 }  ,fill opacity=1 ][line width=0.08]  [draw opacity=0] (5.72,-2.75) -- (0,0) -- (5.72,2.75) -- cycle    ;
%Image [id:dp9396292457736715] 
\draw (106.77,60.95) node  {\includegraphics[width=18.66pt,height=18.36pt]{figures/karma_architecture/pod.png}};
%Image [id:dp3874378335758297] 
\draw (145.86,60.95) node  {\includegraphics[width=18.66pt,height=18.36pt]{figures/karma_architecture/pod.png}};
%Shape: Rectangle [id:dp4562827234223257] 
\draw  [color={rgb, 255:red, 74; green, 144; blue, 226 }  ,draw opacity=1 ][line width=1.5]  (89,28.36) .. controls (89,25.6) and (91.24,23.36) .. (94,23.36) -- (255,23.36) .. controls (257.76,23.36) and (260,25.6) .. (260,28.36) -- (260,132) .. controls (260,134.76) and (257.76,137) .. (255,137) -- (94,137) .. controls (91.24,137) and (89,134.76) .. (89,132) -- cycle ;
%Image [id:dp9455935833751838] 
\draw (172.5,16.24) node  {\includegraphics[width=18.66pt,height=18.36pt]{figures/karma_architecture/kubernetes.png}};
%Shape: Rectangle [id:dp9564725691593288] 
\draw  [color={rgb, 255:red, 74; green, 144; blue, 226 }  ,draw opacity=1 ][line width=1.5]  (92.55,50.21) .. controls (92.55,47.45) and (94.79,45.21) .. (97.55,45.21) -- (155.08,45.21) .. controls (157.84,45.21) and (160.08,47.45) .. (160.08,50.21) -- (160.08,70.81) .. controls (160.08,73.57) and (157.84,75.81) .. (155.08,75.81) -- (97.55,75.81) .. controls (94.79,75.81) and (92.55,73.57) .. (92.55,70.81) -- cycle ;
%Image [id:dp9120856447688912] 
\draw (126.32,39.97) node  {\includegraphics[width=18.66pt,height=18.36pt]{figures/karma_architecture/node.png}};
%Image [id:dp5738167237736518] 
\draw (106.77,119.52) node  {\includegraphics[width=18.66pt,height=18.36pt]{figures/karma_architecture/pod.png}};
%Image [id:dp2199681121060142] 
\draw (145.86,119.52) node  {\includegraphics[width=18.66pt,height=18.36pt]{figures/karma_architecture/pod.png}};
%Shape: Rectangle [id:dp12159705904547402] 
\draw  [color={rgb, 255:red, 74; green, 144; blue, 226 }  ,draw opacity=1 ][line width=1.5]  (92.55,108.78) .. controls (92.55,106.02) and (94.79,103.78) .. (97.55,103.78) -- (155.08,103.78) .. controls (157.84,103.78) and (160.08,106.02) .. (160.08,108.78) -- (160.08,129.38) .. controls (160.08,132.14) and (157.84,134.38) .. (155.08,134.38) -- (97.55,134.38) .. controls (94.79,134.38) and (92.55,132.14) .. (92.55,129.38) -- cycle ;
%Image [id:dp37768653229718074] 
\draw (126.32,98.54) node  {\includegraphics[width=18.66pt,height=18.36pt]{figures/karma_architecture/node.png}};
%Shape: Rectangle [id:dp20840815212238661] 
\draw  [color={rgb, 255:red, 74; green, 144; blue, 226 }  ,draw opacity=1 ][line width=1.5]  (264,28.36) .. controls (264,25.6) and (266.24,23.36) .. (269,23.36) -- (411.78,23.36) .. controls (414.54,23.36) and (416.78,25.6) .. (416.78,28.36) -- (416.78,132) .. controls (416.78,134.76) and (414.54,137) .. (411.78,137) -- (269,137) .. controls (266.24,137) and (264,134.76) .. (264,132) -- cycle ;
%Straight Lines [id:da9232180983272227] 
\draw [color={rgb, 255:red, 74; green, 144; blue, 226 }  ,draw opacity=1 ][line width=2.25]    (164,112) -- (201,112) ;
\draw [shift={(206,112)}, rotate = 180] [fill={rgb, 255:red, 74; green, 144; blue, 226 }  ,fill opacity=1 ][line width=0.08]  [draw opacity=0] (5.72,-2.75) -- (0,0) -- (5.72,2.75) -- cycle    ;
%Straight Lines [id:da6082715106712999] 
\draw [color={rgb, 255:red, 74; green, 144; blue, 226 }  ,draw opacity=1 ][line width=2.25]    (180,90.22) -- (167,90.22) ;
\draw [shift={(162,90.22)}, rotate = 360] [fill={rgb, 255:red, 74; green, 144; blue, 226 }  ,fill opacity=1 ][line width=0.08]  [draw opacity=0] (5.72,-2.75) -- (0,0) -- (5.72,2.75) -- cycle    ;
%Straight Lines [id:da30764510910060716] 
\draw [color={rgb, 255:red, 74; green, 144; blue, 226 }  ,draw opacity=1 ][line width=2.25]    (210,72) -- (199,72) ;
\draw [shift={(194,72)}, rotate = 360] [fill={rgb, 255:red, 74; green, 144; blue, 226 }  ,fill opacity=1 ][line width=0.08]  [draw opacity=0] (5.72,-2.75) -- (0,0) -- (5.72,2.75) -- cycle    ;
%Straight Lines [id:da5394403186779959] 
\draw [color={rgb, 255:red, 74; green, 144; blue, 226 }  ,draw opacity=1 ][line width=2.25]    (178,56.22) -- (167,56.22) ;
\draw [shift={(162,56.22)}, rotate = 360] [fill={rgb, 255:red, 74; green, 144; blue, 226 }  ,fill opacity=1 ][line width=0.08]  [draw opacity=0] (5.72,-2.75) -- (0,0) -- (5.72,2.75) -- cycle    ;
%Straight Lines [id:da6399475815904785] 
\draw [color={rgb, 255:red, 74; green, 144; blue, 226 }  ,draw opacity=1 ][line width=2.25]    (272,72) -- (235,72) ;
\draw [shift={(230,72)}, rotate = 360] [fill={rgb, 255:red, 74; green, 144; blue, 226 }  ,fill opacity=1 ][line width=0.08]  [draw opacity=0] (5.72,-2.75) -- (0,0) -- (5.72,2.75) -- cycle    ;
%Straight Lines [id:da24320102833940327] 
\draw [color={rgb, 255:red, 74; green, 144; blue, 226 }  ,draw opacity=1 ][line width=2.25]    (267,112) -- (230,112) ;
\draw [shift={(272,112)}, rotate = 180] [fill={rgb, 255:red, 74; green, 144; blue, 226 }  ,fill opacity=1 ][line width=0.08]  [draw opacity=0] (5.72,-2.75) -- (0,0) -- (5.72,2.75) -- cycle    ;
%Shape: Rectangle [id:dp9149123987409296] 
\draw  [color={rgb, 255:red, 255; green, 255; blue, 255 }  ,draw opacity=1 ][fill={rgb, 255:red, 255; green, 255; blue, 255 }  ,fill opacity=1 ] (328.83,106.67) -- (353.11,106.67) -- (353.11,114) -- (328.83,114) -- cycle ;
%Image [id:dp7127891043436136] 
\draw (341.68,109.47) node  {\includegraphics[width=14.57pt,height=15.21pt]{figures/karma_architecture/pettingzoo.png}};
%Straight Lines [id:da5757027637146572] 
\draw [color={rgb, 255:red, 74; green, 144; blue, 226 }  ,draw opacity=1 ][line width=2.25]    (357.9,41) -- (364.9,41) ;
\draw [shift={(352.9,41)}, rotate = 0] [fill={rgb, 255:red, 74; green, 144; blue, 226 }  ,fill opacity=1 ][line width=0.08]  [draw opacity=0] (5.72,-2.75) -- (0,0) -- (5.72,2.75) -- cycle    ;
%Straight Lines [id:da29364722138505184] 
\draw [color={rgb, 255:red, 74; green, 144; blue, 226 }  ,draw opacity=1 ][line width=2.25]    (390.99,97.77) -- (390.99,57.25) ;
\draw [shift={(390.99,52.25)}, rotate = 90] [fill={rgb, 255:red, 74; green, 144; blue, 226 }  ,fill opacity=1 ][line width=0.08]  [draw opacity=0] (5.72,-2.75) -- (0,0) -- (5.72,2.75) -- cycle    ;
%Straight Lines [id:da5470457469462804] 
\draw [color={rgb, 255:red, 74; green, 144; blue, 226 }  ,draw opacity=1 ][line width=2.25]    (390.9,71) -- (325.9,71) ;
\draw [shift={(320.9,71)}, rotate = 360] [fill={rgb, 255:red, 74; green, 144; blue, 226 }  ,fill opacity=1 ][line width=0.08]  [draw opacity=0] (5.72,-2.75) -- (0,0) -- (5.72,2.75) -- cycle    ;
%Shape: Rectangle [id:dp8476965567779329] 
\draw  [color={rgb, 255:red, 255; green, 255; blue, 255 }  ,draw opacity=1 ][fill={rgb, 255:red, 255; green, 255; blue, 255 }  ,fill opacity=1 ] (378.41,65.11) -- (397.83,65.11) -- (397.83,92) -- (378.41,92) -- cycle ;
%Shape: Smiley Face [id:dp8656850497140396] 
\draw  [fill={rgb, 255:red, 255; green, 255; blue, 255 }  ,fill opacity=1 ][line width=0.75]  (380.35,69.4) .. controls (380.35,67.3) and (382.09,65.6) .. (384.24,65.6) .. controls (386.38,65.6) and (388.12,67.3) .. (388.12,69.4) .. controls (388.12,71.5) and (386.38,73.2) .. (384.24,73.2) .. controls (382.09,73.2) and (380.35,71.5) .. (380.35,69.4) -- cycle ; \draw  [fill={rgb, 255:red, 255; green, 255; blue, 255 }  ,fill opacity=1 ][line width=0.75]  (382.53,68.11) .. controls (382.53,67.9) and (382.7,67.73) .. (382.92,67.73) .. controls (383.13,67.73) and (383.31,67.9) .. (383.31,68.11) .. controls (383.31,68.32) and (383.13,68.49) .. (382.92,68.49) .. controls (382.7,68.49) and (382.53,68.32) .. (382.53,68.11) -- cycle ; \draw  [fill={rgb, 255:red, 255; green, 255; blue, 255 }  ,fill opacity=1 ][line width=0.75]  (385.17,68.11) .. controls (385.17,67.9) and (385.34,67.73) .. (385.56,67.73) .. controls (385.77,67.73) and (385.95,67.9) .. (385.95,68.11) .. controls (385.95,68.32) and (385.77,68.49) .. (385.56,68.49) .. controls (385.34,68.49) and (385.17,68.32) .. (385.17,68.11) -- cycle ; \draw  [line width=0.75]  (382.29,70.92) .. controls (383.59,71.94) and (384.88,71.94) .. (386.18,70.92) ;
%Shape: Smiley Face [id:dp9163740789669144] 
\draw  [fill={rgb, 255:red, 255; green, 255; blue, 255 }  ,fill opacity=1 ][line width=0.75]  (392.01,69.4) .. controls (392.01,67.3) and (393.75,65.6) .. (395.89,65.6) .. controls (398.04,65.6) and (399.78,67.3) .. (399.78,69.4) .. controls (399.78,71.5) and (398.04,73.2) .. (395.89,73.2) .. controls (393.75,73.2) and (392.01,71.5) .. (392.01,69.4) -- cycle ; \draw  [fill={rgb, 255:red, 255; green, 255; blue, 255 }  ,fill opacity=1 ][line width=0.75]  (394.18,68.11) .. controls (394.18,67.9) and (394.36,67.73) .. (394.57,67.73) .. controls (394.79,67.73) and (394.96,67.9) .. (394.96,68.11) .. controls (394.96,68.32) and (394.79,68.49) .. (394.57,68.49) .. controls (394.36,68.49) and (394.18,68.32) .. (394.18,68.11) -- cycle ; \draw  [fill={rgb, 255:red, 255; green, 255; blue, 255 }  ,fill opacity=1 ][line width=0.75]  (396.82,68.11) .. controls (396.82,67.9) and (397,67.73) .. (397.21,67.73) .. controls (397.43,67.73) and (397.6,67.9) .. (397.6,68.11) .. controls (397.6,68.32) and (397.43,68.49) .. (397.21,68.49) .. controls (397,68.49) and (396.82,68.32) .. (396.82,68.11) -- cycle ; \draw  [line width=0.75]  (393.95,70.92) .. controls (395.24,71.94) and (396.54,71.94) .. (397.83,70.92) ;
%Shape: Smiley Face [id:dp8186451078369623] 
\draw  [fill={rgb, 255:red, 255; green, 255; blue, 255 }  ,fill opacity=1 ][line width=0.75]  (386.18,77.44) .. controls (386.18,75.34) and (387.92,73.64) .. (390.06,73.64) .. controls (392.21,73.64) and (393.95,75.34) .. (393.95,77.44) .. controls (393.95,79.54) and (392.21,81.24) .. (390.06,81.24) .. controls (387.92,81.24) and (386.18,79.54) .. (386.18,77.44) -- cycle ; \draw  [fill={rgb, 255:red, 255; green, 255; blue, 255 }  ,fill opacity=1 ][line width=0.75]  (388.36,76.15) .. controls (388.36,75.94) and (388.53,75.77) .. (388.74,75.77) .. controls (388.96,75.77) and (389.13,75.94) .. (389.13,76.15) .. controls (389.13,76.36) and (388.96,76.53) .. (388.74,76.53) .. controls (388.53,76.53) and (388.36,76.36) .. (388.36,76.15) -- cycle ; \draw  [fill={rgb, 255:red, 255; green, 255; blue, 255 }  ,fill opacity=1 ][line width=0.75]  (391,76.15) .. controls (391,75.94) and (391.17,75.77) .. (391.39,75.77) .. controls (391.6,75.77) and (391.77,75.94) .. (391.77,76.15) .. controls (391.77,76.36) and (391.6,76.53) .. (391.39,76.53) .. controls (391.17,76.53) and (391,76.36) .. (391,76.15) -- cycle ; \draw  [line width=0.75]  (388.12,78.96) .. controls (389.42,79.98) and (390.71,79.98) .. (392.01,78.96) ;
%Flowchart: Punched Tape [id:dp6020643269389074] 
\draw  [fill={rgb, 255:red, 255; green, 255; blue, 255 }  ,fill opacity=1 ] (313.9,33.81) .. controls (313.9,35.03) and (318.18,36.02) .. (323.45,36.02) .. controls (328.73,36.02) and (333,35.03) .. (333,33.81) .. controls (333,32.58) and (337.28,31.59) .. (342.55,31.59) .. controls (347.83,31.59) and (352.1,32.58) .. (352.1,33.81) -- (352.1,51.52) .. controls (352.1,50.3) and (347.83,49.31) .. (342.55,49.31) .. controls (337.28,49.31) and (333,50.3) .. (333,51.52) .. controls (333,52.75) and (328.73,53.74) .. (323.45,53.74) .. controls (318.18,53.74) and (313.9,52.75) .. (313.9,51.52) -- cycle ;
%Straight Lines [id:da950307097731951] 
\draw [line width=0.75]    (324.14,41.04) -- (341.9,41) ;
%Shape: Smiley Face [id:dp8914579811118104] 
\draw  [line width=0.75]  (320.58,40.88) .. controls (320.58,39.7) and (321.59,38.73) .. (322.85,38.73) .. controls (324.1,38.73) and (325.11,39.7) .. (325.11,40.88) .. controls (325.11,42.07) and (324.1,43.03) .. (322.85,43.03) .. controls (321.59,43.03) and (320.58,42.07) .. (320.58,40.88) -- cycle ; \draw  [line width=0.75]  (321.85,40.15) .. controls (321.85,40.03) and (321.95,39.94) .. (322.08,39.94) .. controls (322.2,39.94) and (322.3,40.03) .. (322.3,40.15) .. controls (322.3,40.27) and (322.2,40.37) .. (322.08,40.37) .. controls (321.95,40.37) and (321.85,40.27) .. (321.85,40.15) -- cycle ; \draw  [line width=0.75]  (323.39,40.15) .. controls (323.39,40.03) and (323.49,39.94) .. (323.62,39.94) .. controls (323.74,39.94) and (323.84,40.03) .. (323.84,40.15) .. controls (323.84,40.27) and (323.74,40.37) .. (323.62,40.37) .. controls (323.49,40.37) and (323.39,40.27) .. (323.39,40.15) -- cycle ; \draw  [line width=0.75]  (321.71,41.74) .. controls (322.47,42.31) and (323.22,42.31) .. (323.98,41.74) ;
%Shape: Smiley Face [id:dp07941198495535606] 
\draw  [line width=0.75]  (329.9,45.15) .. controls (329.9,43.96) and (330.92,43) .. (332.17,43) .. controls (333.42,43) and (334.44,43.96) .. (334.44,45.15) .. controls (334.44,46.33) and (333.42,47.29) .. (332.17,47.29) .. controls (330.92,47.29) and (329.9,46.33) .. (329.9,45.15) -- cycle ; \draw  [line width=0.75]  (331.17,44.42) .. controls (331.17,44.3) and (331.28,44.2) .. (331.4,44.2) .. controls (331.53,44.2) and (331.63,44.3) .. (331.63,44.42) .. controls (331.63,44.54) and (331.53,44.63) .. (331.4,44.63) .. controls (331.28,44.63) and (331.17,44.54) .. (331.17,44.42) -- cycle ; \draw  [line width=0.75]  (332.72,44.42) .. controls (332.72,44.3) and (332.82,44.2) .. (332.94,44.2) .. controls (333.07,44.2) and (333.17,44.3) .. (333.17,44.42) .. controls (333.17,44.54) and (333.07,44.63) .. (332.94,44.63) .. controls (332.82,44.63) and (332.72,44.54) .. (332.72,44.42) -- cycle ; \draw  [line width=0.75]  (331.04,46.01) .. controls (331.79,46.58) and (332.55,46.58) .. (333.3,46.01) ;
%Shape: Smiley Face [id:dp8353415903298282] 
\draw  [line width=0.75]  (341.9,40.85) .. controls (341.9,39.67) and (342.92,38.71) .. (344.17,38.71) .. controls (345.42,38.71) and (346.44,39.67) .. (346.44,40.85) .. controls (346.44,42.04) and (345.42,43) .. (344.17,43) .. controls (342.92,43) and (341.9,42.04) .. (341.9,40.85) -- cycle ; \draw  [line width=0.75]  (343.17,40.12) .. controls (343.17,40) and (343.28,39.91) .. (343.4,39.91) .. controls (343.53,39.91) and (343.63,40) .. (343.63,40.12) .. controls (343.63,40.24) and (343.53,40.34) .. (343.4,40.34) .. controls (343.28,40.34) and (343.17,40.24) .. (343.17,40.12) -- cycle ; \draw  [line width=0.75]  (344.72,40.12) .. controls (344.72,40) and (344.82,39.91) .. (344.94,39.91) .. controls (345.07,39.91) and (345.17,40) .. (345.17,40.12) .. controls (345.17,40.24) and (345.07,40.34) .. (344.94,40.34) .. controls (344.82,40.34) and (344.72,40.24) .. (344.72,40.12) -- cycle ; \draw  [line width=0.75]  (343.04,41.71) .. controls (343.79,42.28) and (344.55,42.28) .. (345.3,41.71) ;
%Straight Lines [id:da21936285199788075] 
\draw [line width=0.75]    (324.19,41.87) -- (329.9,45) ;
%Image [id:dp05694376090002984] 
\draw (218.44,70.24) node  {\includegraphics[width=18.66pt,height=18.36pt]{figures/karma_architecture/api.png}};
%Image [id:dp7747210194064744] 
\draw (186.74,54.54) node  {\includegraphics[width=18.66pt,height=18.36pt]{figures/karma_architecture/deploy.png}};
%Image [id:dp5268588430037433] 
\draw (186.74,87.76) node  {\includegraphics[width=18.66pt,height=18.36pt]{figures/karma_architecture/deploy.png}};
%Image [id:dp7447308292951857] 
\draw (218.44,111.76) node  {\includegraphics[width=18.66pt,height=18.36pt]{figures/karma_architecture/prometheus.png}};
%Shape: Rectangle [id:dp7837974954754439] 
\draw  [color={rgb, 255:red, 75; green, 101; blue, 225 }  ,draw opacity=1 ][fill={rgb, 255:red, 74; green, 144; blue, 226 }  ,fill opacity=1 ] (202.37,94.64) -- (208.29,94.64) -- (208.29,101.67) -- (202.37,101.67) -- cycle ;
%Shape: Rectangle [id:dp780870970882084] 
\draw  [color={rgb, 255:red, 75; green, 101; blue, 225 }  ,draw opacity=1 ][fill={rgb, 255:red, 74; green, 144; blue, 226 }  ,fill opacity=1 ] (286.37,126) -- (292.29,126) -- (292.29,133.03) -- (286.37,133.03) -- cycle ;

%Shape: Rectangle [id:dp7051683429553395] 
\draw  [color={rgb, 255:red, 75; green, 101; blue, 225 }  ,draw opacity=1 ][fill={rgb, 255:red, 74; green, 144; blue, 226 }  ,fill opacity=1 ] (397.37,124.64) -- (403.29,124.64) -- (403.29,131.67) -- (397.37,131.67) -- cycle ;

%Shape: Rectangle [id:dp5578959475333973] 
\draw  [color={rgb, 255:red, 75; green, 101; blue, 225 }  ,draw opacity=1 ][fill={rgb, 255:red, 74; green, 144; blue, 226 }  ,fill opacity=1 ] (368.37,54.64) -- (374.29,54.64) -- (374.29,61.67) -- (368.37,61.67) -- cycle ;

%Shape: Rectangle [id:dp2822949836407178] 
\draw  [color={rgb, 255:red, 75; green, 101; blue, 225 }  ,draw opacity=1 ][fill={rgb, 255:red, 74; green, 144; blue, 226 }  ,fill opacity=1 ] (324.37,78) -- (330.29,78) -- (330.29,85.03) -- (324.37,85.03) -- cycle ;

%Shape: Rectangle [id:dp9339299822588341] 
\draw  [color={rgb, 255:red, 75; green, 101; blue, 225 }  ,draw opacity=1 ][fill={rgb, 255:red, 74; green, 144; blue, 226 }  ,fill opacity=1 ] (205.37,48) -- (211.29,48) -- (211.29,55.03) -- (205.37,55.03) -- cycle ;



% Text Node
\draw (205.5,98.5) node  [font=\fontsize{0.33em}{0.4em}\selectfont,color={rgb, 255:red, 255; green, 255; blue, 255 }  ,opacity=1 ] [align=left] {1};
% Text Node
\draw (244,58.5) node  [font=\normalsize] [align=left] {{\tiny Scaling}};
\draw (244,64.5) node  [font=\normalsize] [align=left] {{\tiny actions}};
% Text Node
\draw (244,99.5) node  [font=\normalsize] [align=left] {{\tiny Metrics}};
\draw (244,105.5) node  [font=\normalsize] [align=left] {{\tiny data}};
% Text Node
\draw (344.5,36) node  [font=\fontsize{0.33em}{0.4em}\selectfont] [align=left] {\begin{minipage}[lt]{8.66pt}\setlength\topsep{0pt}
\begin{center}
{\fontsize{0.33em}{0.4em}\selectfont $\displaystyle \mathbf{\textcolor[rgb]{0.82,0.01,0.11}{\pi }\textcolor[rgb]{0.82,0.01,0.11}{_{3}}}$}
\end{center}

\end{minipage}};
% Text Node
\draw (341,46.5) node  [font=\fontsize{0.33em}{0.4em}\selectfont] [align=left] {\begin{minipage}[lt]{8.66pt}\setlength\topsep{0pt}
\begin{center}
{\fontsize{0.33em}{0.4em}\selectfont $\displaystyle \mathbf{\textcolor[rgb]{0.82,0.01,0.11}{\pi }\textcolor[rgb]{0.82,0.01,0.11}{_{2}}}$}
\end{center}

\end{minipage}};
% Text Node
\draw (320.9,48) node  [font=\fontsize{0.33em}{0.4em}\selectfont] [align=left] {\begin{minipage}[lt]{8.66pt}\setlength\topsep{0pt}
\begin{center}
{\fontsize{0.33em}{0.4em}\selectfont $\displaystyle \mathbf{\textcolor[rgb]{0.82,0.01,0.11}{\pi }\textcolor[rgb]{0.82,0.01,0.11}{_{1}}}$}
\end{center}

\end{minipage}};
% Text Node
\draw  [color={rgb, 255:red, 75; green, 101; blue, 225 }  ,draw opacity=1 ][fill={rgb, 255:red, 136; green, 197; blue, 246 }  ,fill opacity=1 ][line width=1.5]   (322.77,14.89) .. controls (322.77,13.78) and (323.67,12.89) .. (324.77,12.89) -- (355.77,12.89) .. controls (356.88,12.89) and (357.77,13.78) .. (357.77,14.89) -- (357.77,26.89) .. controls (357.77,27.99) and (356.88,28.89) .. (355.77,28.89) -- (324.77,28.89) .. controls (323.67,28.89) and (322.77,27.99) .. (322.77,26.89) -- cycle  ;
\draw (340.27,20.89) node  [font=\tiny] [align=left] {\begin{minipage}[lt]{21.5pt}\setlength\topsep{0pt}
\begin{center}
KARMA
\end{center}

\end{minipage}};
% Text Node
\draw (290,40.5) node  [font=\tiny] [align=left] {\begin{minipage}[lt]{27.24pt}\setlength\topsep{0pt}
\begin{center}
Organizational\\Analysis
\end{center}

\end{minipage}};
% Text Node
\draw (388,86.39) node  [font=\tiny] [align=left] {\begin{minipage}[lt]{43.42pt}\setlength\topsep{0pt}
\begin{center}
Trained policies
\end{center}

\end{minipage}};
% Text Node
\draw (344.13,127.35) node  [font=\tiny] [align=left] {\begin{minipage}[lt]{60.78pt}\setlength\topsep{0pt}
\begin{center}
PettingZoo environment
\end{center}

\end{minipage}};
% Text Node
\draw (218,127) node  [font=\tiny] [align=left] {\begin{minipage}[lt]{30.31pt}\setlength\topsep{0pt}
\begin{center}
Prometheus
\end{center}

\end{minipage}};
% Text Node
\draw  [color={rgb, 255:red, 75; green, 101; blue, 225 }  ,draw opacity=1 ][fill={rgb, 255:red, 136; green, 197; blue, 246 }  ,fill opacity=1 ][line width=1.5]   (272.9,62) .. controls (272.9,60.9) and (273.8,60) .. (274.9,60) -- (317.9,60) .. controls (319.01,60) and (319.9,60.9) .. (319.9,62) -- (319.9,83) .. controls (319.9,84.1) and (319.01,85) .. (317.9,85) -- (274.9,85) .. controls (273.8,85) and (272.9,84.1) .. (272.9,83) -- cycle  ;
\draw (296.4,72.5) node  [font=\tiny,color={rgb, 255:red, 0; green, 0; blue, 0 }  ,opacity=1 ] [align=left] {Transfer\\Component};
% Text Node
\draw  [color={rgb, 255:red, 75; green, 101; blue, 225 }  ,draw opacity=1 ][fill={rgb, 255:red, 136; green, 197; blue, 246 }  ,fill opacity=1 ][line width=1.5]   (365.88,29.46) .. controls (365.88,28.35) and (366.78,27.46) .. (367.88,27.46) -- (410.88,27.46) .. controls (411.99,27.46) and (412.88,28.35) .. (412.88,29.46) -- (412.88,50.46) .. controls (412.88,51.56) and (411.99,52.46) .. (410.88,52.46) -- (367.88,52.46) .. controls (366.78,52.46) and (365.88,51.56) .. (365.88,50.46) -- cycle  ;
\draw (389.38,39.96) node  [font=\tiny,color={rgb, 255:red, 0; green, 0; blue, 0 }  ,opacity=1 ] [align=left] {Analyzing\\Component};
% Text Node
\draw  [color={rgb, 255:red, 75; green, 101; blue, 225 }  ,draw opacity=1 ][fill={rgb, 255:red, 136; green, 197; blue, 246 }  ,fill opacity=1 ][line width=1.5]   (365.88,98.24) .. controls (365.88,97.13) and (366.78,96.24) .. (367.88,96.24) -- (410.88,96.24) .. controls (411.99,96.24) and (412.88,97.13) .. (412.88,98.24) -- (412.88,119.24) .. controls (412.88,120.34) and (411.99,121.24) .. (410.88,121.24) -- (367.88,121.24) .. controls (366.78,121.24) and (365.88,120.34) .. (365.88,119.24) -- cycle  ;
\draw (389.38,108.74) node  [font=\tiny,color={rgb, 255:red, 0; green, 0; blue, 0 }  ,opacity=1 ] [align=left] {Training\\Component};
% Text Node
\draw (172.5,33.36) node  [font=\tiny] [align=left] {\begin{minipage}[lt]{16.92pt}\setlength\topsep{0pt}
\begin{center}
Cluster
\end{center}

\end{minipage}};
% Text Node
\draw  [color={rgb, 255:red, 75; green, 101; blue, 225 }  ,draw opacity=1 ][fill={rgb, 255:red, 136; green, 197; blue, 246 }  ,fill opacity=1 ][line width=1.5]   (272.9,99) .. controls (272.9,97.9) and (273.8,97) .. (274.9,97) -- (317.9,97) .. controls (319.01,97) and (319.9,97.9) .. (319.9,99) -- (319.9,120) .. controls (319.9,121.1) and (319.01,122) .. (317.9,122) -- (274.9,122) .. controls (273.8,122) and (272.9,121.1) .. (272.9,120) -- cycle  ;
\draw (296.4,109.5) node  [font=\tiny,color={rgb, 255:red, 0; green, 0; blue, 0 }  ,opacity=1 ] [align=left] {Modeling\\Component};
% Text Node
\draw (173,73.72) node  [font=\tiny,rotate=-90] [align=left] {{\LARGE {\fontfamily{helvet}\selectfont \textcolor[rgb]{0.29,0.56,0.89}{...}}}};
% Text Node
\draw (125.61,118.47) node  [font=\tiny] [align=left] {{\LARGE {\fontfamily{helvet}\selectfont \textcolor[rgb]{0.29,0.56,0.89}{...}}}};
% Text Node
\draw (147,89.5) node  [font=\tiny,rotate=-90] [align=left] {{\LARGE {\fontfamily{helvet}\selectfont \textcolor[rgb]{0.29,0.56,0.89}{...}}}};
% Text Node
\draw (125.61,59.9) node  [font=\tiny] [align=left] {{\LARGE {\fontfamily{helvet}\selectfont \textcolor[rgb]{0.29,0.56,0.89}{...}}}};
% Text Node
\draw (208.5,51.86) node  [font=\fontsize{0.33em}{0.4em}\selectfont,color={rgb, 255:red, 255; green, 255; blue, 255 }  ,opacity=1 ] [align=left] {6};
% Text Node
\draw (327.5,81.86) node  [font=\fontsize{0.33em}{0.4em}\selectfont,color={rgb, 255:red, 255; green, 255; blue, 255 }  ,opacity=1 ] [align=left] {5};
% Text Node
\draw (371.5,58.5) node  [font=\fontsize{0.33em}{0.4em}\selectfont,color={rgb, 255:red, 255; green, 255; blue, 255 }  ,opacity=1 ] [align=left] {4};
% Text Node
\draw (400.5,128.5) node  [font=\fontsize{0.33em}{0.4em}\selectfont,color={rgb, 255:red, 255; green, 255; blue, 255 }  ,opacity=1 ] [align=left] {3};
% Text Node
\draw (289.5,129.86) node  [font=\fontsize{0.33em}{0.4em}\selectfont,color={rgb, 255:red, 255; green, 255; blue, 255 }  ,opacity=1 ] [align=left] {2};


\end{tikzpicture}
  \end{columns}
\end{frame}

\section{The KARMA framework}

\begin{frame}{Phase 1: Modeling (Digital Twin)}
  \begin{columns}
    \column{0.4\textwidth}
    \begin{itemize}
      \item Collected traces from real cluster \(\Rightarrow\) state-action transitions.
      \item Modeled as a \textbf{zero-sum Stochastic Game} with attacker and defenders.
      \item Uses an MLP to approximate unknown transitions: \( \hat{T}(s, a) \approx s' \).
      \item Enables near-realistic training without impacting production.
    \end{itemize}

    \column{0.7\textwidth}
    \centering
    \includegraphics[trim=0cm 3.3cm 0cm 3.5cm, clip, width=\linewidth]{figures/k8s_cluster_graph.pdf}
  \end{columns}
\end{frame}

\begin{frame}{Phase 2: Training (Constrained/Guided MARL)}
  \begin{columns}
    \column{0.55\textwidth}
    \begin{itemize}
      \item Agents are trained using Multi-Agent PPO (MAPPO).
      \item Roles constrain actions via \textbf{Role Action Guides (RAG)}.
      \item Missions guide learning via \textbf{Goal Reward Guides (GRG)}.
      \item Constraints can be hard (enforced) or soft (shaping rewards).
      \item Ensures specialization and coordination across agents.
      \item Based on the \textbf{MOISE+MARL}~\parencite{soule2024moise_marl}~\footnote{\tiny J. Soule, J.-P. Jamont, M. Occello, L.-M. Traonouez, and P. Théron. An organizationally-oriented approach to enhancing explainability and control in multi-agent reinforcement learning. Proc. of the 24th Int. Conf. on Autonomous Agents and Multiagent Systems (AAMAS), 2025.}.
    \end{itemize}

    \hspace{-1cm}

    \column{0.55\textwidth}
    \includegraphics[width=\linewidth]{figures/mm_simple_representation.png}
  \end{columns}
\end{frame}

\begin{frame}{Phase 3: Analyzing Agent Behaviors}
  \begin{columns}
    \column{0.55\textwidth}
    \begin{itemize}
      \item Analyze trained policies to interpret \textbf{implicit roles and missions}.
      \item Clustering of trajectories:
            \begin{itemize}
              \item Action sequences $\Rightarrow$ roles
              \item State visitation patterns $\Rightarrow$ goals
            \end{itemize}
      \item Inspired by the \textbf{TEMM} method from \textbf{MOISE+MARL}.
      \item Improves explainability and supports system design refinement.
      \item Enables better understanding of coordination and interactions.
    \end{itemize}

    \column{0.4\textwidth}
    \centering
    


\tikzset{every picture/.style={line width=0.75pt}} %set default line width to 0.75pt        

\begin{tikzpicture}[x=0.75pt,y=0.75pt,yscale=-1,xscale=1]
    %uncomment if require: \path (0,1974); %set diagram left start at 0, and has height of 1974

    %Shape: Rectangle [id:dp9996076613305621] 
    \draw  [fill={rgb, 255:red, 255; green, 255; blue, 255 }  ,fill opacity=1 ] (24,1558.11) -- (176.1,1558.11) -- (176.1,1644) -- (24,1644) -- cycle ;
    %Straight Lines [id:da05824332013205091] 
    \draw [color={rgb, 255:red, 208; green, 2; blue, 27 }  ,draw opacity=1 ]   (142.67,1570.84) -- (124.28,1577.49) -- (87.26,1592.41) -- (108.68,1604.12) -- (93.53,1601.36) -- (86.58,1603.01) -- (86.58,1612.77) -- (82.05,1616.07) -- (81.22,1616.67) -- (78.65,1614.8) -- (70.51,1608.86) -- (54.44,1608.86) -- (57,1610.73) -- (49.09,1612.77) -- (51.85,1616.79) -- (38.38,1628.38) ;
    \draw [shift={(145.49,1569.82)}, rotate = 160.12] [fill={rgb, 255:red, 208; green, 2; blue, 27 }  ,fill opacity=1 ][line width=0.08]  [draw opacity=0] (3.57,-1.72) -- (0,0) -- (3.57,1.72) -- cycle    ;
    %Straight Lines [id:da9249559779542824] 
    \draw [color={rgb, 255:red, 80; green, 227; blue, 194 }  ,draw opacity=1 ]   (143.47,1568.13) -- (134.78,1577.63) -- (113.36,1577.63) -- (113.36,1585.44) -- (86.58,1593.25) -- (91.93,1597.15) -- (97.29,1604.96) -- (81.22,1601.06) -- (86.58,1608.86) -- (81.22,1608.86) -- (86.58,1616.67) -- (75.87,1616.67) -- (67.94,1608.94) -- (65.16,1614.72) -- (43.73,1603.01) -- (59.8,1616.67) -- (43.73,1608.86) -- (49.09,1616.67) -- (43.73,1632.29) ;
    \draw [shift={(145.49,1565.92)}, rotate = 132.45] [fill={rgb, 255:red, 80; green, 227; blue, 194 }  ,fill opacity=1 ][line width=0.08]  [draw opacity=0] (3.57,-1.72) -- (0,0) -- (3.57,1.72) -- cycle    ;
    %Straight Lines [id:da17118391857757054] 
    \draw [color={rgb, 255:red, 248; green, 231; blue, 28 }  ,draw opacity=1 ]   (153.23,1574.14) -- (126.83,1577.75) -- (124.07,1581.54) -- (105.41,1585.56) -- (91.93,1593.25) -- (93.53,1601.36) -- (89.34,1605.08) -- (81.22,1597.15) -- (91.93,1612.77) -- (91.93,1616.67) -- (81.22,1616.67) -- (59.99,1609.06) -- (57.21,1614.84) -- (41.14,1616.79) -- (35.78,1632.41) ;
    \draw [shift={(156.2,1573.73)}, rotate = 172.2] [fill={rgb, 255:red, 248; green, 231; blue, 28 }  ,fill opacity=1 ][line width=0.08]  [draw opacity=0] (3.57,-1.72) -- (0,0) -- (3.57,1.72) -- cycle    ;
    %Straight Lines [id:da6427777277243145] 
    \draw [color={rgb, 255:red, 144; green, 19; blue, 254 }  ,draw opacity=1 ]   (160.23,1577.39) -- (165.84,1578.41) -- (161.56,1573.73) -- (157.27,1570.61) -- (164.86,1573.73) -- (170.13,1578.41) -- (161.56,1583.1) -- (166.91,1589.34) -- (161.56,1597.15) -- (166.91,1601.06) -- (161.56,1612.77) -- (161.56,1628.38) -- (145.49,1624.48) -- (134.78,1624.48) -- (128.99,1623.07) -- (125.92,1622.33) -- (121.57,1621.27) -- (118.71,1620.58) -- (107.96,1619.71) -- (99.43,1619.01) -- (95.23,1618.25) -- (86.58,1616.67) -- (75.87,1616.67) -- (70.51,1620.58) -- (59.8,1624.48) -- (59.8,1632.29) ;
    \draw [shift={(157.27,1576.85)}, rotate = 10.33] [fill={rgb, 255:red, 144; green, 19; blue, 254 }  ,fill opacity=1 ][line width=0.08]  [draw opacity=0] (3.57,-1.72) -- (0,0) -- (3.57,1.72) -- cycle    ;
    %Straight Lines [id:da7390021320622445] 
    \draw [color={rgb, 255:red, 65; green, 117; blue, 5 }  ,draw opacity=1 ]   (159.15,1578.17) -- (164.77,1579.19) -- (160.49,1574.51) -- (156.2,1571.39) -- (163.79,1574.51) -- (169.06,1579.19) -- (161.56,1587.78) -- (165.84,1590.13) -- (163.7,1601.84) -- (150.85,1597.15) -- (161.56,1603.4) -- (174.41,1615.89) -- (157.27,1606.52) -- (160.49,1613.55) -- (163.7,1625.26) -- (152.99,1628.38) -- (135.85,1622.14) -- (123,1622.14) -- (116.57,1622.14) -- (110.14,1620.58) -- (108,1623.7) -- (103.72,1620.58) -- (105.86,1625.26) -- (94.16,1619.03) -- (85.51,1617.45) -- (74.8,1617.45) -- (69.44,1621.36) -- (58.73,1625.26) -- (58.73,1633.07) ;
    \draw [shift={(156.2,1577.63)}, rotate = 10.33] [fill={rgb, 255:red, 65; green, 117; blue, 5 }  ,fill opacity=1 ][line width=0.08]  [draw opacity=0] (3.57,-1.72) -- (0,0) -- (3.57,1.72) -- cycle    ;
    %Shape: Ellipse [id:dp30050508180239144] 
    \draw  [draw opacity=0][fill={rgb, 255:red, 208; green, 2; blue, 27 }  ,fill opacity=0.62 ] (46.49,1615.89) .. controls (46.49,1614.6) and (47.93,1613.55) .. (49.71,1613.55) .. controls (51.48,1613.55) and (52.92,1614.6) .. (52.92,1615.89) .. controls (52.92,1617.19) and (51.48,1618.23) .. (49.71,1618.23) .. controls (47.93,1618.23) and (46.49,1617.19) .. (46.49,1615.89) -- cycle ;
    %Shape: Ellipse [id:dp15311501498248647] 
    \draw  [draw opacity=0][fill={rgb, 255:red, 208; green, 2; blue, 27 }  ,fill opacity=0.62 ] (90.49,1619.03) .. controls (90.49,1617.74) and (91.93,1616.69) .. (93.71,1616.69) .. controls (95.48,1616.69) and (96.92,1617.74) .. (96.92,1619.03) .. controls (96.92,1620.32) and (95.48,1621.37) .. (93.71,1621.37) .. controls (91.93,1621.37) and (90.49,1620.32) .. (90.49,1619.03) -- cycle ;
    %Shape: Ellipse [id:dp19167487081496637] 
    \draw  [draw opacity=0][fill={rgb, 255:red, 208; green, 2; blue, 27 }  ,fill opacity=0.62 ] (161.11,1606.52) .. controls (161.11,1605.23) and (162.54,1604.18) .. (164.32,1604.18) .. controls (166.09,1604.18) and (167.53,1605.23) .. (167.53,1606.52) .. controls (167.53,1607.82) and (166.09,1608.86) .. (164.32,1608.86) .. controls (162.54,1608.86) and (161.11,1607.82) .. (161.11,1606.52) -- cycle ;
    %Shape: Ellipse [id:dp9201279867822619] 
    \draw  [draw opacity=0][fill={rgb, 255:red, 208; green, 2; blue, 27 }  ,fill opacity=0.62 ] (120.4,1622.92) .. controls (120.4,1621.62) and (121.84,1620.58) .. (123.62,1620.58) .. controls (125.39,1620.58) and (126.83,1621.62) .. (126.83,1622.92) .. controls (126.83,1624.21) and (125.39,1625.26) .. (123.62,1625.26) .. controls (121.84,1625.26) and (120.4,1624.21) .. (120.4,1622.92) -- cycle ;
    %Shape: Ellipse [id:dp3048334813609519] 
    \draw  [draw opacity=0][fill={rgb, 255:red, 208; green, 2; blue, 27 }  ,fill opacity=0.62 ] (161.11,1590.91) .. controls (161.11,1589.61) and (162.54,1588.56) .. (164.32,1588.56) .. controls (166.09,1588.56) and (167.53,1589.61) .. (167.53,1590.91) .. controls (167.53,1592.2) and (166.09,1593.25) .. (164.32,1593.25) .. controls (162.54,1593.25) and (161.11,1592.2) .. (161.11,1590.91) -- cycle ;
    %Shape: Ellipse [id:dp7290465976812913] 
    \draw  [draw opacity=0][fill={rgb, 255:red, 208; green, 2; blue, 27 }  ,fill opacity=0.62 ] (86.13,1603.4) .. controls (86.13,1602.11) and (87.56,1601.06) .. (89.34,1601.06) .. controls (91.11,1601.06) and (92.55,1602.11) .. (92.55,1603.4) .. controls (92.55,1604.69) and (91.11,1605.74) .. (89.34,1605.74) .. controls (87.56,1605.74) and (86.13,1604.69) .. (86.13,1603.4) -- cycle ;
    %Shape: Ellipse [id:dp6154487622646608] 
    \draw  [draw opacity=0][fill={rgb, 255:red, 208; green, 2; blue, 27 }  ,fill opacity=0.62 ] (109.69,1583.1) .. controls (109.69,1581.8) and (111.13,1580.76) .. (112.9,1580.76) .. controls (114.68,1580.76) and (116.12,1581.8) .. (116.12,1583.1) .. controls (116.12,1584.39) and (114.68,1585.44) .. (112.9,1585.44) .. controls (111.13,1585.44) and (109.69,1584.39) .. (109.69,1583.1) -- cycle ;
    %Shape: Ellipse [id:dp6108483574180856] 
    \draw  [draw opacity=0][fill={rgb, 255:red, 189; green, 16; blue, 224 }  ,fill opacity=0.8 ] (77.56,1615.89) .. controls (77.56,1614.6) and (79,1613.55) .. (80.77,1613.55) .. controls (82.55,1613.55) and (83.98,1614.6) .. (83.98,1615.89) .. controls (83.98,1617.19) and (82.55,1618.23) .. (80.77,1618.23) .. controls (79,1618.23) and (77.56,1617.19) .. (77.56,1615.89) -- cycle ;
    %Shape: Ellipse [id:dp08863924891219843] 
    \draw  [draw opacity=0][fill={rgb, 255:red, 208; green, 2; blue, 27 }  ,fill opacity=0.62 ] (84.52,1609.06) .. controls (84.52,1607.77) and (85.96,1606.72) .. (87.73,1606.72) .. controls (89.51,1606.72) and (90.95,1607.77) .. (90.95,1609.06) .. controls (90.95,1610.35) and (89.51,1611.4) .. (87.73,1611.4) .. controls (85.96,1611.4) and (84.52,1610.35) .. (84.52,1609.06) -- cycle ;
    %Shape: Ellipse [id:dp49807154634681794] 
    \draw  [draw opacity=0][fill={rgb, 255:red, 208; green, 2; blue, 27 }  ,fill opacity=0.62 ] (91.21,1601.64) .. controls (91.21,1600.35) and (92.65,1599.3) .. (94.43,1599.3) .. controls (96.2,1599.3) and (97.64,1600.35) .. (97.64,1601.64) .. controls (97.64,1602.94) and (96.2,1603.98) .. (94.43,1603.98) .. controls (92.65,1603.98) and (91.21,1602.94) .. (91.21,1601.64) -- cycle ;
    %Shape: Ellipse [id:dp17062416794692736] 
    \draw  [draw opacity=0][fill={rgb, 255:red, 208; green, 2; blue, 27 }  ,fill opacity=0.62 ] (100.59,1619.41) .. controls (100.59,1618.11) and (102.03,1617.06) .. (103.8,1617.06) .. controls (105.57,1617.06) and (107.01,1618.11) .. (107.01,1619.41) .. controls (107.01,1620.7) and (105.57,1621.75) .. (103.8,1621.75) .. controls (102.03,1621.75) and (100.59,1620.7) .. (100.59,1619.41) -- cycle ;
    %Shape: Ellipse [id:dp05427293190477478] 
    \draw  [draw opacity=0][fill={rgb, 255:red, 189; green, 16; blue, 224 }  ,fill opacity=0.8 ] (152.54,1626.82) .. controls (152.54,1625.53) and (153.98,1624.48) .. (155.75,1624.48) .. controls (157.52,1624.48) and (158.96,1625.53) .. (158.96,1626.82) .. controls (158.96,1628.12) and (157.52,1629.16) .. (155.75,1629.16) .. controls (153.98,1629.16) and (152.54,1628.12) .. (152.54,1626.82) -- cycle ;
    %Shape: Ellipse [id:dp0565658994925915] 
    \draw  [draw opacity=0][fill={rgb, 255:red, 208; green, 2; blue, 27 }  ,fill opacity=0.62 ] (158.96,1616.67) .. controls (158.96,1615.38) and (160.4,1614.33) .. (162.18,1614.33) .. controls (163.95,1614.33) and (165.39,1615.38) .. (165.39,1616.67) .. controls (165.39,1617.97) and (163.95,1619.01) .. (162.18,1619.01) .. controls (160.4,1619.01) and (158.96,1617.97) .. (158.96,1616.67) -- cycle ;
    %Shape: Ellipse [id:dp5007110255270828] 
    \draw  [draw opacity=0][fill={rgb, 255:red, 189; green, 16; blue, 224 }  ,fill opacity=0.8 ] (57,1610.73) .. controls (57,1609.43) and (58.44,1608.38) .. (60.21,1608.38) .. controls (61.99,1608.38) and (63.43,1609.43) .. (63.43,1610.73) .. controls (63.43,1612.02) and (61.99,1613.07) .. (60.21,1613.07) .. controls (58.44,1613.07) and (57,1612.02) .. (57,1610.73) -- cycle ;
    %Shape: Ellipse [id:dp22598728144573377] 
    \draw  [draw opacity=0][fill={rgb, 255:red, 208; green, 2; blue, 27 }  ,fill opacity=0.62 ] (88.72,1595.59) .. controls (88.72,1594.3) and (90.16,1593.25) .. (91.93,1593.25) .. controls (93.71,1593.25) and (95.15,1594.3) .. (95.15,1595.59) .. controls (95.15,1596.88) and (93.71,1597.93) .. (91.93,1597.93) .. controls (90.16,1597.93) and (88.72,1596.88) .. (88.72,1595.59) -- cycle ;
    %Shape: Ellipse [id:dp14749486568088088] 
    \draw  [draw opacity=0][fill={rgb, 255:red, 189; green, 16; blue, 224 }  ,fill opacity=0.8 ] (93.54,1589.34) .. controls (93.54,1588.05) and (94.98,1587) .. (96.75,1587) .. controls (98.53,1587) and (99.97,1588.05) .. (99.97,1589.34) .. controls (99.97,1590.64) and (98.53,1591.69) .. (96.75,1591.69) .. controls (94.98,1591.69) and (93.54,1590.64) .. (93.54,1589.34) -- cycle ;
    %Shape: Polygon Curved [id:ds6643267525526769] 
    \draw  [color={rgb, 255:red, 74; green, 144; blue, 226 }  ,draw opacity=1 ][fill={rgb, 255:red, 74; green, 144; blue, 226 }  ,fill opacity=0.5 ] (27.21,1628.38) .. controls (30.38,1623.39) and (36.63,1621.99) .. (42.56,1622.18) .. controls (47.05,1622.33) and (51.36,1623.39) .. (53.99,1624.48) .. controls (60.1,1627.02) and (65.56,1626.63) .. (64.7,1632.29) .. controls (63.85,1637.95) and (56.88,1637.17) .. (48.64,1636.19) .. controls (40.39,1635.22) and (21.64,1637.17) .. (27.21,1628.38) -- cycle ;
    %Shape: Polygon Curved [id:ds9461514343962948] 
    \draw  [color={rgb, 255:red, 208; green, 2; blue, 27 }  ,draw opacity=1 ][fill={rgb, 255:red, 208; green, 2; blue, 27 }  ,fill opacity=0.5 ] (139.68,1569.82) .. controls (145.25,1561.04) and (148.01,1561.59) .. (145.04,1565.92) .. controls (142.07,1570.25) and (154.68,1563.87) .. (153.82,1569.53) .. controls (152.97,1575.19) and (169.35,1578.61) .. (161.11,1577.63) .. controls (152.86,1576.66) and (134.11,1578.61) .. (139.68,1569.82) -- cycle ;
    %Shape: Ellipse [id:dp06406072166611776] 
    \draw  [draw opacity=0][fill={rgb, 255:red, 208; green, 2; blue, 27 }  ,fill opacity=0.62 ] (71.13,1617.45) .. controls (71.13,1616.16) and (72.57,1615.11) .. (74.34,1615.11) .. controls (76.12,1615.11) and (77.56,1616.16) .. (77.56,1617.45) .. controls (77.56,1618.75) and (76.12,1619.8) .. (74.34,1619.8) .. controls (72.57,1619.8) and (71.13,1618.75) .. (71.13,1617.45) -- cycle ;
    %Shape: Ellipse [id:dp049221150011381054] 
    \draw  [draw opacity=0][fill={rgb, 255:red, 189; green, 16; blue, 224 }  ,fill opacity=0.8 ] (56.59,1624.48) .. controls (56.59,1623.19) and (58.03,1622.14) .. (59.8,1622.14) .. controls (61.58,1622.14) and (63.01,1623.19) .. (63.01,1624.48) .. controls (63.01,1625.77) and (61.58,1626.82) .. (59.8,1626.82) .. controls (58.03,1626.82) and (56.59,1625.77) .. (56.59,1624.48) -- cycle ;


    % Text Node
    \draw (68.95,1583.75) node  [font=\tiny,color={rgb, 255:red, 189; green, 16; blue, 224 }  ,opacity=1 ] [align=left] {$\displaystyle g_{5} =\{\omega _{21} \dotsc \}$};
    % Text Node
    \draw (82.63,1627.35) node  [font=\tiny,color={rgb, 255:red, 189; green, 16; blue, 224 }  ,opacity=1 ] [align=left] {$\displaystyle g_{2} =\{\omega _{5}\}$};
    % Text Node
    \draw (152.91,1586.86) node  [font=\tiny,color={rgb, 255:red, 189; green, 16; blue, 224 }  ,opacity=1 ] [align=left] {$\displaystyle ...$};
    % Text Node
    \draw (101.5,1575.93) node  [font=\tiny,color={rgb, 255:red, 189; green, 16; blue, 224 }  ,opacity=1 ] [align=left] {$\displaystyle ...$};
    % Text Node
    \draw (136.45,1636.35) node  [font=\tiny,color={rgb, 255:red, 189; green, 16; blue, 224 }  ,opacity=1 ] [align=left] {$\displaystyle g_{4} =\{\omega _{301} ,\omega _{302}\}$};
    % Text Node
    \draw (113.58,1612.35) node  [font=\tiny,color={rgb, 255:red, 189; green, 16; blue, 224 }  ,opacity=1 ] [align=left] {$\displaystyle g_{3} =\{\omega _{10}\}$};
    % Text Node
    \draw (55.11,1600.35) node  [font=\tiny,color={rgb, 255:red, 189; green, 16; blue, 224 }  ,opacity=1 ] [align=left] {$\displaystyle g_{1} =\{\omega _{1}\}$};
    % Text Node
    \draw (105.58,1568.35) node  [font=\tiny,color={rgb, 255:red, 202; green, 52; blue, 69 }  ,opacity=1 ] [align=left] {$\displaystyle g_{*} =\Omega _{goal}$};
    % Text Node
    \draw (73.43,1637.35) node  [font=\tiny,color={rgb, 255:red, 74; green, 144; blue, 226 }  ,opacity=1 ] [align=left] {$\displaystyle \Omega _{init}$};
    % Text Node
    \draw (32.91,1567.84) node  [font=\scriptsize] [align=left] {$\displaystyle \Omega $};
    % Text Node
    \draw (93.61,1653) node   [align=left] {{\tiny \textit{Une visualisation abstraite des}}};
    \draw (93.61,1662) node   [align=left] {{\tiny \textit{observations dans les trajectoires}}};

\end{tikzpicture}

    \vspace{0.5cm}

    


\tikzset{every picture/.style={line width=0.75pt}} %set default line width to 0.75pt        

\begin{tikzpicture}[x=0.75pt,y=0.75pt,yscale=-1,xscale=1]
%uncomment if require: \path (0,1974); %set diagram left start at 0, and has height of 1974

%Shape: Rectangle [id:dp5335676631264512] 
\draw  [fill={rgb, 255:red, 255; green, 255; blue, 255 }  ,fill opacity=1 ] (190,1560.11) -- (342.1,1560.11) -- (342.1,1646) -- (190,1646) -- cycle ;
%Straight Lines [id:da6623576988919416] 
\draw [color={rgb, 255:red, 208; green, 2; blue, 27 }  ,draw opacity=1 ]   (308.67,1572.84) -- (290.28,1579.49) -- (253.26,1594.41) -- (292.1,1604) -- (274.1,1612) -- (262.1,1616) -- (252.58,1614.77) -- (246.1,1604) -- (240.1,1604) -- (236.1,1606) -- (236.51,1610.86) -- (220.44,1610.86) -- (223,1612.73) -- (215.09,1614.77) -- (217.85,1618.79) -- (204.38,1630.38) ;
\draw [shift={(311.49,1571.82)}, rotate = 160.12] [fill={rgb, 255:red, 208; green, 2; blue, 27 }  ,fill opacity=1 ][line width=0.08]  [draw opacity=0] (3.57,-1.72) -- (0,0) -- (3.57,1.72) -- cycle    ;
%Straight Lines [id:da5424854363807742] 
\draw [color={rgb, 255:red, 80; green, 227; blue, 194 }  ,draw opacity=1 ]   (309.47,1570.13) -- (300.78,1579.63) -- (279.36,1579.63) -- (279.36,1587.44) -- (252.58,1595.25) -- (250.1,1598) -- (248.1,1600) -- (247.22,1603.06) -- (252.58,1610.86) -- (247.22,1610.86) -- (242.1,1608) -- (242.1,1610) -- (233.94,1610.94) -- (231.16,1616.72) -- (209.73,1605.01) -- (225.8,1618.67) -- (209.73,1610.86) -- (215.09,1618.67) -- (209.73,1634.29) ;
\draw [shift={(311.49,1567.92)}, rotate = 132.45] [fill={rgb, 255:red, 80; green, 227; blue, 194 }  ,fill opacity=1 ][line width=0.08]  [draw opacity=0] (3.57,-1.72) -- (0,0) -- (3.57,1.72) -- cycle    ;
%Straight Lines [id:da21186841526109945] 
\draw [color={rgb, 255:red, 248; green, 231; blue, 28 }  ,draw opacity=1 ]   (319.23,1576.14) -- (292.83,1579.75) -- (290.07,1583.54) -- (271.41,1587.56) -- (257.93,1595.25) -- (280.1,1592) -- (284.1,1594) -- (290.1,1604) -- (257.93,1614.77) -- (257.93,1618.67) -- (247.22,1618.67) -- (225.99,1611.06) -- (223.21,1616.84) -- (207.14,1618.79) -- (201.78,1634.41) ;
\draw [shift={(322.2,1575.73)}, rotate = 172.2] [fill={rgb, 255:red, 248; green, 231; blue, 28 }  ,fill opacity=1 ][line width=0.08]  [draw opacity=0] (3.57,-1.72) -- (0,0) -- (3.57,1.72) -- cycle    ;
%Straight Lines [id:da6313290732282947] 
\draw [color={rgb, 255:red, 144; green, 19; blue, 254 }  ,draw opacity=1 ]   (326.23,1579.39) -- (331.84,1580.41) -- (327.56,1575.73) -- (323.27,1572.61) -- (330.86,1575.73) -- (336.13,1580.41) -- (327.56,1585.1) -- (332.91,1591.34) -- (324.1,1600) -- (330.1,1606) -- (312.1,1636) -- (320.1,1638) -- (306.1,1644) -- (306.1,1606) -- (300.1,1614) -- (311.49,1626.48) -- (300.78,1626.48) -- (294.99,1625.07) -- (291.92,1624.33) -- (287.57,1623.27) -- (284.71,1622.58) -- (273.96,1621.71) -- (265.43,1621.01) -- (261.23,1620.25) -- (250.1,1622) -- (241.87,1618.67) -- (236.51,1622.58) -- (225.8,1626.48) -- (225.8,1634.29) ;
\draw [shift={(323.27,1578.85)}, rotate = 10.33] [fill={rgb, 255:red, 144; green, 19; blue, 254 }  ,fill opacity=1 ][line width=0.08]  [draw opacity=0] (3.57,-1.72) -- (0,0) -- (3.57,1.72) -- cycle    ;
%Straight Lines [id:da1305524961942589] 
\draw [color={rgb, 255:red, 65; green, 117; blue, 5 }  ,draw opacity=1 ]   (325.15,1580.17) -- (330.77,1581.19) -- (326.49,1576.51) -- (322.2,1573.39) -- (329.79,1576.51) -- (335.06,1581.19) -- (327.56,1589.78) -- (331.84,1592.13) -- (329.7,1603.84) -- (316.85,1599.15) -- (327.56,1605.4) -- (314.1,1636) -- (318.1,1640) -- (310.1,1642) -- (304.1,1608) -- (298.1,1612) -- (301.85,1624.14) -- (289,1624.14) -- (282.57,1624.14) -- (276.14,1622.58) -- (274,1625.7) -- (269.72,1622.58) -- (271.86,1627.26) -- (260.16,1621.03) -- (251.51,1619.45) -- (240.8,1619.45) -- (235.44,1623.36) -- (224.73,1627.26) -- (224.73,1635.07) ;
\draw [shift={(322.2,1579.63)}, rotate = 10.33] [fill={rgb, 255:red, 65; green, 117; blue, 5 }  ,fill opacity=1 ][line width=0.08]  [draw opacity=0] (3.57,-1.72) -- (0,0) -- (3.57,1.72) -- cycle    ;
%Shape: Polygon Curved [id:ds29559681347985167] 
\draw  [color={rgb, 255:red, 184; green, 233; blue, 134 }  ,draw opacity=0 ][fill={rgb, 255:red, 74; green, 144; blue, 226 }  ,fill opacity=0.75 ] (203.31,1627.26) .. controls (208.88,1618.48) and (203.46,1612.19) .. (210,1608) .. controls (216.54,1603.81) and (229.5,1600.53) .. (248,1604) .. controls (266.5,1607.47) and (269.83,1605.44) .. (280,1604) .. controls (290.17,1602.56) and (250.44,1601.87) .. (252,1594) .. controls (253.56,1586.13) and (258.5,1591.77) .. (262,1588) .. controls (265.5,1584.23) and (314.2,1570.33) .. (316,1570) .. controls (317.8,1569.67) and (320.91,1572.43) .. (314,1576) .. controls (307.09,1579.57) and (294.2,1581.95) .. (288,1584) .. controls (281.8,1586.05) and (277.13,1589.32) .. (276,1590) .. controls (274.87,1590.68) and (280.1,1589.85) .. (284,1592) .. controls (287.9,1594.15) and (295.82,1601.53) .. (296,1602) .. controls (296.18,1602.47) and (264.78,1615.69) .. (264,1616) .. controls (263.22,1616.31) and (249.54,1612.38) .. (244,1612) .. controls (238.46,1611.62) and (217.32,1621.29) .. (214,1626) .. controls (210.68,1630.71) and (210.87,1632.35) .. (210,1636) .. controls (209.13,1639.65) and (197.74,1636.04) .. (203.31,1627.26) -- cycle ;
%Shape: Polygon Curved [id:ds2928272635642186] 
\draw  [color={rgb, 255:red, 208; green, 2; blue, 27 }  ,draw opacity=0 ][fill={rgb, 255:red, 208; green, 2; blue, 27 }  ,fill opacity=0.5 ] (304,1644) .. controls (302.69,1641.5) and (306.85,1644.5) .. (304,1634) .. controls (301.15,1623.5) and (270.08,1628.81) .. (266,1628) .. controls (261.92,1627.19) and (254.9,1623.42) .. (244,1624) .. controls (233.1,1624.58) and (230.1,1635.78) .. (226,1638) .. controls (221.9,1640.22) and (223.11,1630.38) .. (222,1630) .. controls (220.89,1629.62) and (222.67,1626.54) .. (226,1624) .. controls (229.33,1621.46) and (236.24,1618.61) .. (236,1618) .. controls (235.76,1617.39) and (243.31,1617.27) .. (250,1618) .. controls (256.69,1618.73) and (262.53,1620.62) .. (264,1620) .. controls (265.47,1619.38) and (296.11,1616.11) .. (296,1614) .. controls (295.89,1611.89) and (301.99,1602.86) .. (304,1602) .. controls (306.01,1601.14) and (317.99,1622.88) .. (320,1616) .. controls (322.01,1609.12) and (321.73,1568.73) .. (326,1570) .. controls (330.27,1571.27) and (339.87,1563.9) .. (338,1584) .. controls (336.13,1604.1) and (324.06,1639.29) .. (320,1642) .. controls (315.94,1644.71) and (305.31,1646.5) .. (304,1644) -- cycle ;


% Text Node
\draw (268.2,1637.75) node  [font=\tiny,color={rgb, 255:red, 189; green, 16; blue, 224 }  ,opacity=1 ] [align=left] {$\displaystyle \rho _{2} =\{( \omega _{11} ,a_{11}) \dotsc \}$};
% Text Node
\draw (231.2,1581.75) node  [font=\tiny,color={rgb, 255:red, 189; green, 16; blue, 224 }  ,opacity=1 ] [align=left] {$\displaystyle \rho _{1} =\{( \omega _{21} ,a_{21}) \dotsc \}$};
% Text Node
\draw (209.5,1570) node  [font=\scriptsize] [align=left] {$\displaystyle \Omega \times A$};
% Text Node
\draw (267.61,1655) node   [align=left] {{\tiny \textit{An abstract visualization of}}};
\draw (267.61,1665) node   [align=left] {{\tiny \textit{transitions in trajectories}}};

\end{tikzpicture}

  \end{columns}
\end{frame}

\begin{frame}{Phase 4: Transfer to Real Cluster}
  \begin{columns}
    \column{0.35\textwidth}
    \begin{itemize}
      \item Deploy trained agents to control the real Kubernetes cluster.
      \item Agents interact via the Kubernetes API to scale pods.
      \item A fallback mechanism ensures safety in production.
      \item Digital twin remains updated in parallel with real-time data.
      \item Supports iterative retraining and adaptation.
    \end{itemize}

    \column{0.75\textwidth}
    \centering
    


\tikzset{every picture/.style={line width=0.75pt}} %set default line width to 0.75pt        

\begin{tikzpicture}[x=0.75pt,y=0.75pt,yscale=-1.2,xscale=1.2]
%uncomment if require: \path (0,1414); %set diagram left start at 0, and has height of 1414

%Straight Lines [id:da5609883377896374] 
\draw [color={rgb, 255:red, 74; green, 144; blue, 226 }  ,draw opacity=1 ][line width=2.25]    (317.22,111.13) -- (360.07,111.13) ;
\draw [shift={(365.07,111.13)}, rotate = 180] [fill={rgb, 255:red, 74; green, 144; blue, 226 }  ,fill opacity=1 ][line width=0.08]  [draw opacity=0] (5.72,-2.75) -- (0,0) -- (5.72,2.75) -- cycle    ;
%Image [id:dp9396292457736715] 
\draw (106.77,60.95) node  {\includegraphics[width=18.66pt,height=18.36pt]{figures/karma_architecture/pod.png}};
%Image [id:dp3874378335758297] 
\draw (145.86,60.95) node  {\includegraphics[width=18.66pt,height=18.36pt]{figures/karma_architecture/pod.png}};
%Shape: Rectangle [id:dp4562827234223257] 
\draw  [color={rgb, 255:red, 74; green, 144; blue, 226 }  ,draw opacity=1 ][line width=1.5]  (89,28.36) .. controls (89,25.6) and (91.24,23.36) .. (94,23.36) -- (255,23.36) .. controls (257.76,23.36) and (260,25.6) .. (260,28.36) -- (260,132) .. controls (260,134.76) and (257.76,137) .. (255,137) -- (94,137) .. controls (91.24,137) and (89,134.76) .. (89,132) -- cycle ;
%Image [id:dp9455935833751838] 
\draw (172.5,16.24) node  {\includegraphics[width=18.66pt,height=18.36pt]{figures/karma_architecture/kubernetes.png}};
%Shape: Rectangle [id:dp9564725691593288] 
\draw  [color={rgb, 255:red, 74; green, 144; blue, 226 }  ,draw opacity=1 ][line width=1.5]  (92.55,50.21) .. controls (92.55,47.45) and (94.79,45.21) .. (97.55,45.21) -- (155.08,45.21) .. controls (157.84,45.21) and (160.08,47.45) .. (160.08,50.21) -- (160.08,70.81) .. controls (160.08,73.57) and (157.84,75.81) .. (155.08,75.81) -- (97.55,75.81) .. controls (94.79,75.81) and (92.55,73.57) .. (92.55,70.81) -- cycle ;
%Image [id:dp9120856447688912] 
\draw (126.32,39.97) node  {\includegraphics[width=18.66pt,height=18.36pt]{figures/karma_architecture/node.png}};
%Image [id:dp5738167237736518] 
\draw (106.77,119.52) node  {\includegraphics[width=18.66pt,height=18.36pt]{figures/karma_architecture/pod.png}};
%Image [id:dp2199681121060142] 
\draw (145.86,119.52) node  {\includegraphics[width=18.66pt,height=18.36pt]{figures/karma_architecture/pod.png}};
%Shape: Rectangle [id:dp12159705904547402] 
\draw  [color={rgb, 255:red, 74; green, 144; blue, 226 }  ,draw opacity=1 ][line width=1.5]  (92.55,108.78) .. controls (92.55,106.02) and (94.79,103.78) .. (97.55,103.78) -- (155.08,103.78) .. controls (157.84,103.78) and (160.08,106.02) .. (160.08,108.78) -- (160.08,129.38) .. controls (160.08,132.14) and (157.84,134.38) .. (155.08,134.38) -- (97.55,134.38) .. controls (94.79,134.38) and (92.55,132.14) .. (92.55,129.38) -- cycle ;
%Image [id:dp37768653229718074] 
\draw (126.32,98.54) node  {\includegraphics[width=18.66pt,height=18.36pt]{figures/karma_architecture/node.png}};
%Shape: Rectangle [id:dp20840815212238661] 
\draw  [color={rgb, 255:red, 74; green, 144; blue, 226 }  ,draw opacity=1 ][line width=1.5]  (264,28.36) .. controls (264,25.6) and (266.24,23.36) .. (269,23.36) -- (411.78,23.36) .. controls (414.54,23.36) and (416.78,25.6) .. (416.78,28.36) -- (416.78,132) .. controls (416.78,134.76) and (414.54,137) .. (411.78,137) -- (269,137) .. controls (266.24,137) and (264,134.76) .. (264,132) -- cycle ;
%Straight Lines [id:da9232180983272227] 
\draw [color={rgb, 255:red, 74; green, 144; blue, 226 }  ,draw opacity=1 ][line width=2.25]    (164,112) -- (201,112) ;
\draw [shift={(206,112)}, rotate = 180] [fill={rgb, 255:red, 74; green, 144; blue, 226 }  ,fill opacity=1 ][line width=0.08]  [draw opacity=0] (5.72,-2.75) -- (0,0) -- (5.72,2.75) -- cycle    ;
%Straight Lines [id:da6082715106712999] 
\draw [color={rgb, 255:red, 74; green, 144; blue, 226 }  ,draw opacity=1 ][line width=2.25]    (180,90.22) -- (167,90.22) ;
\draw [shift={(162,90.22)}, rotate = 360] [fill={rgb, 255:red, 74; green, 144; blue, 226 }  ,fill opacity=1 ][line width=0.08]  [draw opacity=0] (5.72,-2.75) -- (0,0) -- (5.72,2.75) -- cycle    ;
%Straight Lines [id:da30764510910060716] 
\draw [color={rgb, 255:red, 74; green, 144; blue, 226 }  ,draw opacity=1 ][line width=2.25]    (210,72) -- (199,72) ;
\draw [shift={(194,72)}, rotate = 360] [fill={rgb, 255:red, 74; green, 144; blue, 226 }  ,fill opacity=1 ][line width=0.08]  [draw opacity=0] (5.72,-2.75) -- (0,0) -- (5.72,2.75) -- cycle    ;
%Straight Lines [id:da5394403186779959] 
\draw [color={rgb, 255:red, 74; green, 144; blue, 226 }  ,draw opacity=1 ][line width=2.25]    (178,56.22) -- (167,56.22) ;
\draw [shift={(162,56.22)}, rotate = 360] [fill={rgb, 255:red, 74; green, 144; blue, 226 }  ,fill opacity=1 ][line width=0.08]  [draw opacity=0] (5.72,-2.75) -- (0,0) -- (5.72,2.75) -- cycle    ;
%Straight Lines [id:da6399475815904785] 
\draw [color={rgb, 255:red, 74; green, 144; blue, 226 }  ,draw opacity=1 ][line width=2.25]    (272,72) -- (235,72) ;
\draw [shift={(230,72)}, rotate = 360] [fill={rgb, 255:red, 74; green, 144; blue, 226 }  ,fill opacity=1 ][line width=0.08]  [draw opacity=0] (5.72,-2.75) -- (0,0) -- (5.72,2.75) -- cycle    ;
%Straight Lines [id:da24320102833940327] 
\draw [color={rgb, 255:red, 74; green, 144; blue, 226 }  ,draw opacity=1 ][line width=2.25]    (267,112) -- (230,112) ;
\draw [shift={(272,112)}, rotate = 180] [fill={rgb, 255:red, 74; green, 144; blue, 226 }  ,fill opacity=1 ][line width=0.08]  [draw opacity=0] (5.72,-2.75) -- (0,0) -- (5.72,2.75) -- cycle    ;
%Shape: Rectangle [id:dp9149123987409296] 
\draw  [color={rgb, 255:red, 255; green, 255; blue, 255 }  ,draw opacity=1 ][fill={rgb, 255:red, 255; green, 255; blue, 255 }  ,fill opacity=1 ] (328.83,106.67) -- (353.11,106.67) -- (353.11,114) -- (328.83,114) -- cycle ;
%Image [id:dp7127891043436136] 
\draw (341.68,109.47) node  {\includegraphics[width=14.57pt,height=15.21pt]{figures/karma_architecture/pettingzoo.png}};
%Straight Lines [id:da5757027637146572] 
\draw [color={rgb, 255:red, 74; green, 144; blue, 226 }  ,draw opacity=1 ][line width=2.25]    (357.9,41) -- (364.9,41) ;
\draw [shift={(352.9,41)}, rotate = 0] [fill={rgb, 255:red, 74; green, 144; blue, 226 }  ,fill opacity=1 ][line width=0.08]  [draw opacity=0] (5.72,-2.75) -- (0,0) -- (5.72,2.75) -- cycle    ;
%Straight Lines [id:da29364722138505184] 
\draw [color={rgb, 255:red, 74; green, 144; blue, 226 }  ,draw opacity=1 ][line width=2.25]    (390.99,97.77) -- (390.99,57.25) ;
\draw [shift={(390.99,52.25)}, rotate = 90] [fill={rgb, 255:red, 74; green, 144; blue, 226 }  ,fill opacity=1 ][line width=0.08]  [draw opacity=0] (5.72,-2.75) -- (0,0) -- (5.72,2.75) -- cycle    ;
%Straight Lines [id:da5470457469462804] 
\draw [color={rgb, 255:red, 74; green, 144; blue, 226 }  ,draw opacity=1 ][line width=2.25]    (390.9,71) -- (325.9,71) ;
\draw [shift={(320.9,71)}, rotate = 360] [fill={rgb, 255:red, 74; green, 144; blue, 226 }  ,fill opacity=1 ][line width=0.08]  [draw opacity=0] (5.72,-2.75) -- (0,0) -- (5.72,2.75) -- cycle    ;
%Shape: Rectangle [id:dp8476965567779329] 
\draw  [color={rgb, 255:red, 255; green, 255; blue, 255 }  ,draw opacity=1 ][fill={rgb, 255:red, 255; green, 255; blue, 255 }  ,fill opacity=1 ] (378.41,65.11) -- (397.83,65.11) -- (397.83,92) -- (378.41,92) -- cycle ;
%Shape: Smiley Face [id:dp8656850497140396] 
\draw  [fill={rgb, 255:red, 255; green, 255; blue, 255 }  ,fill opacity=1 ][line width=0.75]  (380.35,69.4) .. controls (380.35,67.3) and (382.09,65.6) .. (384.24,65.6) .. controls (386.38,65.6) and (388.12,67.3) .. (388.12,69.4) .. controls (388.12,71.5) and (386.38,73.2) .. (384.24,73.2) .. controls (382.09,73.2) and (380.35,71.5) .. (380.35,69.4) -- cycle ; \draw  [fill={rgb, 255:red, 255; green, 255; blue, 255 }  ,fill opacity=1 ][line width=0.75]  (382.53,68.11) .. controls (382.53,67.9) and (382.7,67.73) .. (382.92,67.73) .. controls (383.13,67.73) and (383.31,67.9) .. (383.31,68.11) .. controls (383.31,68.32) and (383.13,68.49) .. (382.92,68.49) .. controls (382.7,68.49) and (382.53,68.32) .. (382.53,68.11) -- cycle ; \draw  [fill={rgb, 255:red, 255; green, 255; blue, 255 }  ,fill opacity=1 ][line width=0.75]  (385.17,68.11) .. controls (385.17,67.9) and (385.34,67.73) .. (385.56,67.73) .. controls (385.77,67.73) and (385.95,67.9) .. (385.95,68.11) .. controls (385.95,68.32) and (385.77,68.49) .. (385.56,68.49) .. controls (385.34,68.49) and (385.17,68.32) .. (385.17,68.11) -- cycle ; \draw  [line width=0.75]  (382.29,70.92) .. controls (383.59,71.94) and (384.88,71.94) .. (386.18,70.92) ;
%Shape: Smiley Face [id:dp9163740789669144] 
\draw  [fill={rgb, 255:red, 255; green, 255; blue, 255 }  ,fill opacity=1 ][line width=0.75]  (392.01,69.4) .. controls (392.01,67.3) and (393.75,65.6) .. (395.89,65.6) .. controls (398.04,65.6) and (399.78,67.3) .. (399.78,69.4) .. controls (399.78,71.5) and (398.04,73.2) .. (395.89,73.2) .. controls (393.75,73.2) and (392.01,71.5) .. (392.01,69.4) -- cycle ; \draw  [fill={rgb, 255:red, 255; green, 255; blue, 255 }  ,fill opacity=1 ][line width=0.75]  (394.18,68.11) .. controls (394.18,67.9) and (394.36,67.73) .. (394.57,67.73) .. controls (394.79,67.73) and (394.96,67.9) .. (394.96,68.11) .. controls (394.96,68.32) and (394.79,68.49) .. (394.57,68.49) .. controls (394.36,68.49) and (394.18,68.32) .. (394.18,68.11) -- cycle ; \draw  [fill={rgb, 255:red, 255; green, 255; blue, 255 }  ,fill opacity=1 ][line width=0.75]  (396.82,68.11) .. controls (396.82,67.9) and (397,67.73) .. (397.21,67.73) .. controls (397.43,67.73) and (397.6,67.9) .. (397.6,68.11) .. controls (397.6,68.32) and (397.43,68.49) .. (397.21,68.49) .. controls (397,68.49) and (396.82,68.32) .. (396.82,68.11) -- cycle ; \draw  [line width=0.75]  (393.95,70.92) .. controls (395.24,71.94) and (396.54,71.94) .. (397.83,70.92) ;
%Shape: Smiley Face [id:dp8186451078369623] 
\draw  [fill={rgb, 255:red, 255; green, 255; blue, 255 }  ,fill opacity=1 ][line width=0.75]  (386.18,77.44) .. controls (386.18,75.34) and (387.92,73.64) .. (390.06,73.64) .. controls (392.21,73.64) and (393.95,75.34) .. (393.95,77.44) .. controls (393.95,79.54) and (392.21,81.24) .. (390.06,81.24) .. controls (387.92,81.24) and (386.18,79.54) .. (386.18,77.44) -- cycle ; \draw  [fill={rgb, 255:red, 255; green, 255; blue, 255 }  ,fill opacity=1 ][line width=0.75]  (388.36,76.15) .. controls (388.36,75.94) and (388.53,75.77) .. (388.74,75.77) .. controls (388.96,75.77) and (389.13,75.94) .. (389.13,76.15) .. controls (389.13,76.36) and (388.96,76.53) .. (388.74,76.53) .. controls (388.53,76.53) and (388.36,76.36) .. (388.36,76.15) -- cycle ; \draw  [fill={rgb, 255:red, 255; green, 255; blue, 255 }  ,fill opacity=1 ][line width=0.75]  (391,76.15) .. controls (391,75.94) and (391.17,75.77) .. (391.39,75.77) .. controls (391.6,75.77) and (391.77,75.94) .. (391.77,76.15) .. controls (391.77,76.36) and (391.6,76.53) .. (391.39,76.53) .. controls (391.17,76.53) and (391,76.36) .. (391,76.15) -- cycle ; \draw  [line width=0.75]  (388.12,78.96) .. controls (389.42,79.98) and (390.71,79.98) .. (392.01,78.96) ;
%Flowchart: Punched Tape [id:dp6020643269389074] 
\draw  [fill={rgb, 255:red, 255; green, 255; blue, 255 }  ,fill opacity=1 ] (313.9,33.81) .. controls (313.9,35.03) and (318.18,36.02) .. (323.45,36.02) .. controls (328.73,36.02) and (333,35.03) .. (333,33.81) .. controls (333,32.58) and (337.28,31.59) .. (342.55,31.59) .. controls (347.83,31.59) and (352.1,32.58) .. (352.1,33.81) -- (352.1,51.52) .. controls (352.1,50.3) and (347.83,49.31) .. (342.55,49.31) .. controls (337.28,49.31) and (333,50.3) .. (333,51.52) .. controls (333,52.75) and (328.73,53.74) .. (323.45,53.74) .. controls (318.18,53.74) and (313.9,52.75) .. (313.9,51.52) -- cycle ;
%Straight Lines [id:da950307097731951] 
\draw [line width=0.75]    (324.14,41.04) -- (341.9,41) ;
%Shape: Smiley Face [id:dp8914579811118104] 
\draw  [line width=0.75]  (320.58,40.88) .. controls (320.58,39.7) and (321.59,38.73) .. (322.85,38.73) .. controls (324.1,38.73) and (325.11,39.7) .. (325.11,40.88) .. controls (325.11,42.07) and (324.1,43.03) .. (322.85,43.03) .. controls (321.59,43.03) and (320.58,42.07) .. (320.58,40.88) -- cycle ; \draw  [line width=0.75]  (321.85,40.15) .. controls (321.85,40.03) and (321.95,39.94) .. (322.08,39.94) .. controls (322.2,39.94) and (322.3,40.03) .. (322.3,40.15) .. controls (322.3,40.27) and (322.2,40.37) .. (322.08,40.37) .. controls (321.95,40.37) and (321.85,40.27) .. (321.85,40.15) -- cycle ; \draw  [line width=0.75]  (323.39,40.15) .. controls (323.39,40.03) and (323.49,39.94) .. (323.62,39.94) .. controls (323.74,39.94) and (323.84,40.03) .. (323.84,40.15) .. controls (323.84,40.27) and (323.74,40.37) .. (323.62,40.37) .. controls (323.49,40.37) and (323.39,40.27) .. (323.39,40.15) -- cycle ; \draw  [line width=0.75]  (321.71,41.74) .. controls (322.47,42.31) and (323.22,42.31) .. (323.98,41.74) ;
%Shape: Smiley Face [id:dp07941198495535606] 
\draw  [line width=0.75]  (329.9,45.15) .. controls (329.9,43.96) and (330.92,43) .. (332.17,43) .. controls (333.42,43) and (334.44,43.96) .. (334.44,45.15) .. controls (334.44,46.33) and (333.42,47.29) .. (332.17,47.29) .. controls (330.92,47.29) and (329.9,46.33) .. (329.9,45.15) -- cycle ; \draw  [line width=0.75]  (331.17,44.42) .. controls (331.17,44.3) and (331.28,44.2) .. (331.4,44.2) .. controls (331.53,44.2) and (331.63,44.3) .. (331.63,44.42) .. controls (331.63,44.54) and (331.53,44.63) .. (331.4,44.63) .. controls (331.28,44.63) and (331.17,44.54) .. (331.17,44.42) -- cycle ; \draw  [line width=0.75]  (332.72,44.42) .. controls (332.72,44.3) and (332.82,44.2) .. (332.94,44.2) .. controls (333.07,44.2) and (333.17,44.3) .. (333.17,44.42) .. controls (333.17,44.54) and (333.07,44.63) .. (332.94,44.63) .. controls (332.82,44.63) and (332.72,44.54) .. (332.72,44.42) -- cycle ; \draw  [line width=0.75]  (331.04,46.01) .. controls (331.79,46.58) and (332.55,46.58) .. (333.3,46.01) ;
%Shape: Smiley Face [id:dp8353415903298282] 
\draw  [line width=0.75]  (341.9,40.85) .. controls (341.9,39.67) and (342.92,38.71) .. (344.17,38.71) .. controls (345.42,38.71) and (346.44,39.67) .. (346.44,40.85) .. controls (346.44,42.04) and (345.42,43) .. (344.17,43) .. controls (342.92,43) and (341.9,42.04) .. (341.9,40.85) -- cycle ; \draw  [line width=0.75]  (343.17,40.12) .. controls (343.17,40) and (343.28,39.91) .. (343.4,39.91) .. controls (343.53,39.91) and (343.63,40) .. (343.63,40.12) .. controls (343.63,40.24) and (343.53,40.34) .. (343.4,40.34) .. controls (343.28,40.34) and (343.17,40.24) .. (343.17,40.12) -- cycle ; \draw  [line width=0.75]  (344.72,40.12) .. controls (344.72,40) and (344.82,39.91) .. (344.94,39.91) .. controls (345.07,39.91) and (345.17,40) .. (345.17,40.12) .. controls (345.17,40.24) and (345.07,40.34) .. (344.94,40.34) .. controls (344.82,40.34) and (344.72,40.24) .. (344.72,40.12) -- cycle ; \draw  [line width=0.75]  (343.04,41.71) .. controls (343.79,42.28) and (344.55,42.28) .. (345.3,41.71) ;
%Straight Lines [id:da21936285199788075] 
\draw [line width=0.75]    (324.19,41.87) -- (329.9,45) ;
%Image [id:dp05694376090002984] 
\draw (218.44,70.24) node  {\includegraphics[width=18.66pt,height=18.36pt]{figures/karma_architecture/api.png}};
%Image [id:dp7747210194064744] 
\draw (186.74,54.54) node  {\includegraphics[width=18.66pt,height=18.36pt]{figures/karma_architecture/deploy.png}};
%Image [id:dp5268588430037433] 
\draw (186.74,87.76) node  {\includegraphics[width=18.66pt,height=18.36pt]{figures/karma_architecture/deploy.png}};
%Image [id:dp7447308292951857] 
\draw (218.44,111.76) node  {\includegraphics[width=18.66pt,height=18.36pt]{figures/karma_architecture/prometheus.png}};
%Shape: Rectangle [id:dp7837974954754439] 
\draw  [color={rgb, 255:red, 75; green, 101; blue, 225 }  ,draw opacity=1 ][fill={rgb, 255:red, 74; green, 144; blue, 226 }  ,fill opacity=1 ] (202.37,94.64) -- (208.29,94.64) -- (208.29,101.67) -- (202.37,101.67) -- cycle ;
%Shape: Rectangle [id:dp780870970882084] 
\draw  [color={rgb, 255:red, 75; green, 101; blue, 225 }  ,draw opacity=1 ][fill={rgb, 255:red, 74; green, 144; blue, 226 }  ,fill opacity=1 ] (286.37,126) -- (292.29,126) -- (292.29,133.03) -- (286.37,133.03) -- cycle ;

%Shape: Rectangle [id:dp7051683429553395] 
\draw  [color={rgb, 255:red, 75; green, 101; blue, 225 }  ,draw opacity=1 ][fill={rgb, 255:red, 74; green, 144; blue, 226 }  ,fill opacity=1 ] (397.37,124.64) -- (403.29,124.64) -- (403.29,131.67) -- (397.37,131.67) -- cycle ;

%Shape: Rectangle [id:dp5578959475333973] 
\draw  [color={rgb, 255:red, 75; green, 101; blue, 225 }  ,draw opacity=1 ][fill={rgb, 255:red, 74; green, 144; blue, 226 }  ,fill opacity=1 ] (368.37,54.64) -- (374.29,54.64) -- (374.29,61.67) -- (368.37,61.67) -- cycle ;

%Shape: Rectangle [id:dp2822949836407178] 
\draw  [color={rgb, 255:red, 75; green, 101; blue, 225 }  ,draw opacity=1 ][fill={rgb, 255:red, 74; green, 144; blue, 226 }  ,fill opacity=1 ] (324.37,78) -- (330.29,78) -- (330.29,85.03) -- (324.37,85.03) -- cycle ;

%Shape: Rectangle [id:dp9339299822588341] 
\draw  [color={rgb, 255:red, 75; green, 101; blue, 225 }  ,draw opacity=1 ][fill={rgb, 255:red, 74; green, 144; blue, 226 }  ,fill opacity=1 ] (205.37,48) -- (211.29,48) -- (211.29,55.03) -- (205.37,55.03) -- cycle ;



% Text Node
\draw (205.5,98.5) node  [font=\fontsize{0.33em}{0.4em}\selectfont,color={rgb, 255:red, 255; green, 255; blue, 255 }  ,opacity=1 ] [align=left] {1};
% Text Node
\draw (244,58.5) node  [font=\normalsize] [align=left] {{\tiny Scaling}};
\draw (244,64.5) node  [font=\normalsize] [align=left] {{\tiny actions}};
% Text Node
\draw (244,99.5) node  [font=\normalsize] [align=left] {{\tiny Metrics}};
\draw (244,105.5) node  [font=\normalsize] [align=left] {{\tiny data}};
% Text Node
\draw (344.5,36) node  [font=\fontsize{0.33em}{0.4em}\selectfont] [align=left] {\begin{minipage}[lt]{8.66pt}\setlength\topsep{0pt}
\begin{center}
{\fontsize{0.33em}{0.4em}\selectfont $\displaystyle \mathbf{\textcolor[rgb]{0.82,0.01,0.11}{\pi }\textcolor[rgb]{0.82,0.01,0.11}{_{3}}}$}
\end{center}

\end{minipage}};
% Text Node
\draw (341,46.5) node  [font=\fontsize{0.33em}{0.4em}\selectfont] [align=left] {\begin{minipage}[lt]{8.66pt}\setlength\topsep{0pt}
\begin{center}
{\fontsize{0.33em}{0.4em}\selectfont $\displaystyle \mathbf{\textcolor[rgb]{0.82,0.01,0.11}{\pi }\textcolor[rgb]{0.82,0.01,0.11}{_{2}}}$}
\end{center}

\end{minipage}};
% Text Node
\draw (320.9,48) node  [font=\fontsize{0.33em}{0.4em}\selectfont] [align=left] {\begin{minipage}[lt]{8.66pt}\setlength\topsep{0pt}
\begin{center}
{\fontsize{0.33em}{0.4em}\selectfont $\displaystyle \mathbf{\textcolor[rgb]{0.82,0.01,0.11}{\pi }\textcolor[rgb]{0.82,0.01,0.11}{_{1}}}$}
\end{center}

\end{minipage}};
% Text Node
\draw  [color={rgb, 255:red, 75; green, 101; blue, 225 }  ,draw opacity=1 ][fill={rgb, 255:red, 136; green, 197; blue, 246 }  ,fill opacity=1 ][line width=1.5]   (322.77,14.89) .. controls (322.77,13.78) and (323.67,12.89) .. (324.77,12.89) -- (355.77,12.89) .. controls (356.88,12.89) and (357.77,13.78) .. (357.77,14.89) -- (357.77,26.89) .. controls (357.77,27.99) and (356.88,28.89) .. (355.77,28.89) -- (324.77,28.89) .. controls (323.67,28.89) and (322.77,27.99) .. (322.77,26.89) -- cycle  ;
\draw (340.27,20.89) node  [font=\tiny] [align=left] {\begin{minipage}[lt]{21.5pt}\setlength\topsep{0pt}
\begin{center}
KARMA
\end{center}

\end{minipage}};
% Text Node
\draw (290,40.5) node  [font=\tiny] [align=left] {\begin{minipage}[lt]{27.24pt}\setlength\topsep{0pt}
\begin{center}
Organizational\\Analysis
\end{center}

\end{minipage}};
% Text Node
\draw (388,86.39) node  [font=\tiny] [align=left] {\begin{minipage}[lt]{43.42pt}\setlength\topsep{0pt}
\begin{center}
Trained policies
\end{center}

\end{minipage}};
% Text Node
\draw (344.13,127.35) node  [font=\tiny] [align=left] {\begin{minipage}[lt]{60.78pt}\setlength\topsep{0pt}
\begin{center}
PettingZoo environment
\end{center}

\end{minipage}};
% Text Node
\draw (218,127) node  [font=\tiny] [align=left] {\begin{minipage}[lt]{30.31pt}\setlength\topsep{0pt}
\begin{center}
Prometheus
\end{center}

\end{minipage}};
% Text Node
\draw  [color={rgb, 255:red, 75; green, 101; blue, 225 }  ,draw opacity=1 ][fill={rgb, 255:red, 136; green, 197; blue, 246 }  ,fill opacity=1 ][line width=1.5]   (272.9,62) .. controls (272.9,60.9) and (273.8,60) .. (274.9,60) -- (317.9,60) .. controls (319.01,60) and (319.9,60.9) .. (319.9,62) -- (319.9,83) .. controls (319.9,84.1) and (319.01,85) .. (317.9,85) -- (274.9,85) .. controls (273.8,85) and (272.9,84.1) .. (272.9,83) -- cycle  ;
\draw (296.4,72.5) node  [font=\tiny,color={rgb, 255:red, 0; green, 0; blue, 0 }  ,opacity=1 ] [align=left] {Transfer\\Component};
% Text Node
\draw  [color={rgb, 255:red, 75; green, 101; blue, 225 }  ,draw opacity=1 ][fill={rgb, 255:red, 136; green, 197; blue, 246 }  ,fill opacity=1 ][line width=1.5]   (365.88,29.46) .. controls (365.88,28.35) and (366.78,27.46) .. (367.88,27.46) -- (410.88,27.46) .. controls (411.99,27.46) and (412.88,28.35) .. (412.88,29.46) -- (412.88,50.46) .. controls (412.88,51.56) and (411.99,52.46) .. (410.88,52.46) -- (367.88,52.46) .. controls (366.78,52.46) and (365.88,51.56) .. (365.88,50.46) -- cycle  ;
\draw (389.38,39.96) node  [font=\tiny,color={rgb, 255:red, 0; green, 0; blue, 0 }  ,opacity=1 ] [align=left] {Analyzing\\Component};
% Text Node
\draw  [color={rgb, 255:red, 75; green, 101; blue, 225 }  ,draw opacity=1 ][fill={rgb, 255:red, 136; green, 197; blue, 246 }  ,fill opacity=1 ][line width=1.5]   (365.88,98.24) .. controls (365.88,97.13) and (366.78,96.24) .. (367.88,96.24) -- (410.88,96.24) .. controls (411.99,96.24) and (412.88,97.13) .. (412.88,98.24) -- (412.88,119.24) .. controls (412.88,120.34) and (411.99,121.24) .. (410.88,121.24) -- (367.88,121.24) .. controls (366.78,121.24) and (365.88,120.34) .. (365.88,119.24) -- cycle  ;
\draw (389.38,108.74) node  [font=\tiny,color={rgb, 255:red, 0; green, 0; blue, 0 }  ,opacity=1 ] [align=left] {Training\\Component};
% Text Node
\draw (172.5,33.36) node  [font=\tiny] [align=left] {\begin{minipage}[lt]{16.92pt}\setlength\topsep{0pt}
\begin{center}
Cluster
\end{center}

\end{minipage}};
% Text Node
\draw  [color={rgb, 255:red, 75; green, 101; blue, 225 }  ,draw opacity=1 ][fill={rgb, 255:red, 136; green, 197; blue, 246 }  ,fill opacity=1 ][line width=1.5]   (272.9,99) .. controls (272.9,97.9) and (273.8,97) .. (274.9,97) -- (317.9,97) .. controls (319.01,97) and (319.9,97.9) .. (319.9,99) -- (319.9,120) .. controls (319.9,121.1) and (319.01,122) .. (317.9,122) -- (274.9,122) .. controls (273.8,122) and (272.9,121.1) .. (272.9,120) -- cycle  ;
\draw (296.4,109.5) node  [font=\tiny,color={rgb, 255:red, 0; green, 0; blue, 0 }  ,opacity=1 ] [align=left] {Modeling\\Component};
% Text Node
\draw (173,73.72) node  [font=\tiny,rotate=-90] [align=left] {{\LARGE {\fontfamily{helvet}\selectfont \textcolor[rgb]{0.29,0.56,0.89}{...}}}};
% Text Node
\draw (125.61,118.47) node  [font=\tiny] [align=left] {{\LARGE {\fontfamily{helvet}\selectfont \textcolor[rgb]{0.29,0.56,0.89}{...}}}};
% Text Node
\draw (147,89.5) node  [font=\tiny,rotate=-90] [align=left] {{\LARGE {\fontfamily{helvet}\selectfont \textcolor[rgb]{0.29,0.56,0.89}{...}}}};
% Text Node
\draw (125.61,59.9) node  [font=\tiny] [align=left] {{\LARGE {\fontfamily{helvet}\selectfont \textcolor[rgb]{0.29,0.56,0.89}{...}}}};
% Text Node
\draw (208.5,51.86) node  [font=\fontsize{0.33em}{0.4em}\selectfont,color={rgb, 255:red, 255; green, 255; blue, 255 }  ,opacity=1 ] [align=left] {6};
% Text Node
\draw (327.5,81.86) node  [font=\fontsize{0.33em}{0.4em}\selectfont,color={rgb, 255:red, 255; green, 255; blue, 255 }  ,opacity=1 ] [align=left] {5};
% Text Node
\draw (371.5,58.5) node  [font=\fontsize{0.33em}{0.4em}\selectfont,color={rgb, 255:red, 255; green, 255; blue, 255 }  ,opacity=1 ] [align=left] {4};
% Text Node
\draw (400.5,128.5) node  [font=\fontsize{0.33em}{0.4em}\selectfont,color={rgb, 255:red, 255; green, 255; blue, 255 }  ,opacity=1 ] [align=left] {3};
% Text Node
\draw (289.5,129.86) node  [font=\fontsize{0.33em}{0.4em}\selectfont,color={rgb, 255:red, 255; green, 255; blue, 255 }  ,opacity=1 ] [align=left] {2};


\end{tikzpicture}
  \end{columns}
\end{frame}

\section{Experiments and discussion}

\begin{frame}{Experimental Setup}
  \begin{columns}
    \column{0.4\textwidth}
    \begin{itemize}
      \item Simulated Kubernetes cluster with 4 microservices in cascade.
      \item Services connected as a directed chain: A $\rightarrow$ B $\rightarrow$ C $\rightarrow$ D.
      \item Injected dynamic failures:
            \begin{itemize}
              \item DDoS attacks on random services
              \item Resource bottlenecks (CPU/memory)
              \item Crashes and restarts
            \end{itemize}
      \item Agents act on replicas of each pod type to adapt flow.
      \item Training and evaluation performed on a high-performance GPU cluster.
    \end{itemize}

    \column{0.7\textwidth}
    \centering
    \includegraphics[trim=0cm 3.3cm 0cm 3.5cm, clip, width=\linewidth]{figures/k8s_cluster_graph.pdf}

  \end{columns}
\end{frame}

\begin{frame}{Comparative Evaluation (1/2)}
  \begin{columns}
    \column{0.4\textwidth}
    \begin{itemize}
      \item We compare KARMA against:
            \begin{itemize}
              \item AWARE (baseline MARL)
              \item Gym-HPA (standard RL autoscaler)
              \item Rlad-core (mono-agent approach)
            \end{itemize}
      \item Metrics evaluated:
            \begin{itemize}
              \item Success rate (QoS satisfaction)
              \item Average latency
              \item Pending request queue length
            \end{itemize}
      \item KARMA achieves higher resilience and efficiency across all metrics.
    \end{itemize}

    \column{0.6\textwidth}

    \begin{table}[h!]
      \centering
      \caption{\small Comparative results on operational resilience and adversarial recovery.}
      \label{tab:combined_evaluation}
      \small
      \renewcommand{\arraystretch}{1.5}
      \setlength{\tabcolsep}{6pt}
      \begin{tabular}{>{\raggedright\arraybackslash}m{1.0cm}
        >{\centering\arraybackslash}m{1.0cm}
        >{\centering\arraybackslash}m{1.1cm}
        >{\centering\arraybackslash}m{1.1cm}
        >{\centering\arraybackslash}m{1.0cm}
        >{\centering\arraybackslash}m{1.0cm}}
        \hline
        \textbf{Baseline} & \textbf{Succ.} & \textbf{Latency} & \textbf{Pending} & \textbf{Recov.} & \textbf{Avail.} \\
                          & \textbf{(\%)}  & \textbf{(\%)}    & \textbf{(\%)}    & \textbf{(s)}    & \textbf{(\%)}   \\
        \hline
        KHPA              & 64.8           & 58.1             & 20.7             & 80.7            & 65.6            \\
        Gym-HPA           & 73.1           & 65.7             & 20.8             & 66.2            & 72.6            \\
        Rlad-core         & 77.4           & 70.1             & 15.9             & 37.4            & 78.3            \\
        AWARE             & 80.6           & 73.8             & 13.3             & 49.5            & 83.6            \\
        \textbf{KARMA}    & \textbf{90.9}  & \textbf{85.7}    & \textbf{5.9}     & \textbf{33.0}   & \textbf{90.7}   \\
        \hline
      \end{tabular}
    \end{table}


  \end{columns}
\end{frame}

\begin{frame}{Comparative Evaluation (2/2)}
  \begin{columns}

    \column{0.4\textwidth}
    \begin{itemize}
      \item Learning curves show:
            \begin{itemize}
              \item Faster convergence for KARMA
              \item More stable reward evolution over time
              \item Less variance across seeds
            \end{itemize}
      \item Role constraints reduce the policy search space.
      \item Multi-agent structure encourages coordinated exploration.
      \item KARMA generalizes better under diverse failure conditions.
    \end{itemize}

    \column{0.65\textwidth}
    \includegraphics[width=0.95\linewidth]{figures/learning_curves.pdf}

  \end{columns}
\end{frame}

\begin{frame}{Explainability \& Organizational Fit}
  \begin{columns}
    \column{0.5\textwidth}
    \begin{itemize}
      \item Post-hoc analysis of agent behaviors:
            \begin{itemize}
              \item Role identification via trajectory clustering
              \item Mission recognition via observation patterns
            \end{itemize}
      \item Computation of \textbf{Organizational Fit} metrics:
            \begin{itemize}
              \item Structural Fit (alignment to roles)
              \item Functional Fit (goal achievement patterns)
            \end{itemize}
      \item Helps validate whether emergent behavior matches organizational design.
    \end{itemize}

    \column{0.5\textwidth}

    \includegraphics[width=0.95\linewidth]{figures/role_hierarchical_clustering.pdf}

    \

    \includegraphics[width=0.95\linewidth]{figures/roles_graph.pdf}

  \end{columns}
\end{frame}

\begin{frame}{Ablation Studies}
  \begin{columns}
    \column{0.4\textwidth}
    \begin{itemize}
      \item We conducted ablations to assess component impact:
            \begin{itemize}
              \item \textbf{No digital twin}: -11\% performance
              \item \textbf{Mono-agent}: degraded coordination, higher latency
              \item \textbf{No organizational constraints}: higher variance, slower convergence
            \end{itemize}
      \item Each component (Twin, MARL, MOISE+) contributes significantly.
      \item The full KARMA stack yields best trade-off: performance + explainability + robustness.
    \end{itemize}

    \column{0.6\textwidth}
    \centering
    \begin{table}[h!]
      \centering
      \caption{\scriptsize Ablation study: impact of organizational structure and multi-agent design.}
      \label{tab:ablation_study}
      \scriptsize
      \renewcommand{\arraystretch}{1.5}
      \setlength{\tabcolsep}{6pt}
      \begin{tabular}{>{\raggedright\arraybackslash}m{1.5cm}
        >{\centering\arraybackslash}m{0.8cm}
        >{\centering\arraybackslash}m{0.7cm}
        >{\centering\arraybackslash}m{0.7cm}
        >{\centering\arraybackslash}m{0.7cm}
        >{\centering\arraybackslash}m{0.7cm}}
        \hline
        \textbf{Configuration}                   & \textbf{Succ. (\%)} & \textbf{Latency (\%)} & \textbf{Pending (\%)} & \textbf{Recov. (s)} & \textbf{Avail. (\%)} \\
        \hline
        Single-Agent \textit{w/o} Org. Spec.     & 72.6                & 65.4                  & 17.0                  & 60.3                & 72.4                 \\
        Single-Agent \textit{w/} Hard Org. Spec. & 80.8                & 72.5                  & 15.4                  & 48.5                & 77.5                 \\
        Multi-Agent \textit{w/o} Org. Spec.      & 87.7                & 81.5                  & 9.3                   & 43.5                & 82.0                 \\
        Multi-Agent \textit{w/} Soft Org. Spec.  & 82.0                & 74.7                  & 15.0                  & 38.8                & 86.0                 \\
        \textbf{KARMA (Multi-Agent + Hard)}      & \textbf{90.9}       & \textbf{85.7}         & \textbf{5.9}          & \textbf{33.0}       & \textbf{90.7}        \\
        \hline
      \end{tabular}
    \end{table}


  \end{columns}
\end{frame}

\section{Conclusion}

\begin{frame}{Conclusion \& Perspectives}
  \begin{itemize}
    \item \textbf{KARMA} framework addresses 6 key gaps in resilient autoscaling:
          \begin{itemize}
            \item Online MAS design
            \item Failure-aware training
            \item Organizational constraints
            \item Explainability of agents
            \item Safe deployment
            \item Continuous adaptation
          \end{itemize}
    \item Combines Digital Twin + Guided MARL + Org. Fit Analysis.
    \item \textbf{Perspectives}:
          \begin{itemize}
            \item Real multi-node Kubernetes cluster
            \item Extend to more service types (e.g. streaming)
            \item Use LLMs to document behaviors or missions
          \end{itemize}
  \end{itemize}

\end{frame}


\appendix
%\setbeamertemplate{headline}{}
\setbeamertemplate{mini frames}{}

% \AtBeginSection[]{
% 	\begin{frame}
% 		\frametitle{}
% 		\tableofcontents[currentsection]
% 	\end{frame}
% }

% %%%%%%%%%%%%%%%%%%%%%%%%%%%%%%%%%%%%

\section*{\phantom{Thanks}}

\begin{frame}{}

  \vspace{6ex}

  \centering
  {
    \Huge
    \emph{Thank You}
  }

  \vspace{6ex}

  \begin{columns}

    \hspace{-27ex}

    \begin{column}{0.5\textwidth}
      \raggedleft
      {\Large Demo video $\Longrightarrow$}
    \end{column}

    \hspace{-12ex}

    \begin{column}{0.5\textwidth}
      \includegraphics[width=0.5\linewidth]{figures/demo_qr_code.png}
    \end{column}

  \end{columns}

  \vspace{3ex}

  \centering
  {\Large
    \url{https://t.ly/4JBxr}
  }

\end{frame}


\section*{\phantom{References}}
\begin{frame}[allowframebreaks]{References}{}
  \printbibliography
\end{frame}

\newcounter{mainframenumber}
\setcounter{mainframenumber}{\value{framenumber}}

% % \begin{frame}{Annexes}
    {Context}

    \begin{block}{Multi-Agent Systems (MAS) paradigm for complex \& distributed problems}
        \begin{itemize}
            \item \textbf{task decomposition}: missions delegated to agents achieved through cooperation~\cite{Raileanu2023};
            \item \textbf{benefits}: handle conflicting goals, parallel computation, system robustness, scalability\dots
        \end{itemize}
    \end{block}

    \begin{block}{\textbf{Organization}: key for MAS designing}
        \begin{itemize}
            \item \textbf{coordination}: how to collaboratively achieve a common goal~\cite{Hubner2007};
            \item \textbf{dynamic \& uncertain environments}: flexible runtime behavior to adapt~\cite{Kathleen2020};
        \end{itemize}
    \end{block}

    \begin{block}{Methods and practice for MAS design}
        \begin{itemize}
            \item \textbf{approach + organizational model}: methods rely on designers' experience to hand-craft agents' \textbf{policies} so resulting MAS achieve goals;
                  %   \begin{itemize}
                  %       \item Examples: \emph{GAIA}~\cite{Wooldridge2000,Cernuzzi2014}, \emph{ADELFE}~\cite{Mefteh2015}, or \emph{DIAMOND}~\cite{Jamont2015}, \emph{KB-ORG}~\cite{Sims2008}
                  %   \end{itemize}
            \item \textbf{simulation to reality}: 1) safe \& efficient MAS design in high fidelity simulated environment; \quad 2) transfer to real environment to perform adequately~\cite{Schon2021}.
        \end{itemize}
        \vspace{1ex}
        \quad $\Longrightarrow$ \textbf{Iterative process proceeding by trial and error}

    \end{block}

\end{frame}

\begin{frame}{Annexes}
    {MAS basics}

    \begin{block}{Keywords}
        \begin{itemize}
            \item \textbf{Agent}: entity immersed in an environment perceiving observation and making decision autonomously to achieve some goals;
            \item \textbf{MAS}: a set of agents collaborating with self/re-organizing mechanisms to achieve their goal;
            \item \textbf{Organization}: the agents' interactions even though it may be implicit;
            \item \textbf{Organizational Model (OM)}: medium to formally describe an explicit/implicit organization;
            \item \textbf{Organizational Specifications (OS)}: components of an OM to characterize an organization
        \end{itemize}
    \end{block}

    \begin{block}{Organizational model: $\mathcal{M}OISE^+$}
        \begin{itemize}
            \item more complex than \emph{Agent Group Roles} (integration of standards);
            \item takes into account the social aspects between agents explicitly;
            \item possible to link agents' policies to organizational specifications.
        \end{itemize}
    \end{block}

\end{frame}

\begin{frame}{Annexes}
    {MARL basics}

    \begin{block}{Keywords}
        \begin{itemize}
            \item \textbf{Policy}: the \textquote{logic} to choose next action according to observation for an agent;
            \item \textbf{History/trajectory}: the tuple of (observation, action) couples over an episode;
            \item \textbf{Joint-policy / Joint-history}: all of the agents' policies / histories as tuples;
            \item \textbf{Reinforcement learning}: an agent updates its policy to maximize a cumulative reward;
            \item \textbf{Multi-Agent Reinforcement Learning (MARL)}: extends to multiple agents that learn while considering the actions of other agents;
        \end{itemize}
    \end{block}

\end{frame}



\end{document}
