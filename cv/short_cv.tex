% !TEX TS-program = xelatex
% Compile with: xelatex or lualatex

\documentclass[11pt,a4paper,sans]{moderncv}

% ModernCV settings
\moderncvstyle{classic}
\moderncvcolor{blue}

% Packages
\usepackage[utf8]{inputenc}
\usepackage[TU]{fontenc}
\usepackage[scale=0.85]{geometry}
\usepackage{setspace}
\setstretch{1.05}
\usepackage{parskip}

% \usepackage[hidelinks]{hyperref}
\usepackage{biblatex}
\moderncvicons{awesome}
\usepackage{academicons}

\newcommand{\dblp}[1]{\collectionadd[twitter]{socials}{\protect\httplink[#1]{dblp.uni-trier.de/search?q=#1}}}

\renewcommand*{\twittersocialsymbol}{%
    \raisebox{-0.08em}{%
    \scalebox{0.6}{%
        \tikz{\draw node[rectangle, rounded corners, draw=color2, inner sep=1pt] (0,0) {\textsf{dblp}};}~% 
    }}}

% Personal Information
\name{Julien}{Soulé}
\title{Doctor in Computer Science \\ AI \& Multi-Agent Systems \\ \ \\}
\address{35 Rue Mathieu-de-la-Drôme}{26000 Valence, France}{}

\email{julien.soule@hotmail.fr}
\phone[mobile]{+33~(0)~6~77~63~12~13}
\homepage{julien6.github.io/home/}

\extrainfo{
  \faLinkedin\ 
  \href{https://www.linkedin.com/in/julien-soul\%C3\%A9-6b2b27173}{julien-soulé-6b2b27173}\\
  \noindent
  \twittersocialsymbol~\href{https://dblp.org/pid/367/9947.html}{homepages/367/9947}\\
  \noindent
  \aiOrcid~\href{https://orcid.org/0009-0002-3218-2614}{0009-0002-3218-2614}\\
  \noindent
  \faGithub~\href{https://github.com/julien6}{julien6}
}

\photo[110pt][0pt]{photo.jpeg}

% === Load your .bib file ===
\addbibresource{references.bib}

% === Sorting by year (descending), then author ===
\ExecuteBibliographyOptions{sorting=ydnt}

\setlength{\hintscolumnwidth}{2.5cm}

\AfterPreamble{
  \hypersetup{
    colorlinks=false,
    pdfborder={0 0 1},
    pdfborderstyle={/S/U/W 1},
    linkbordercolor={0 0 1},
    urlbordercolor={0 0 1},
    citebordercolor={0 0 1}
  }
}

\begin{document}
\makecvtitle

\vspace{-0.8cm}
\begin{center}
    {\tiny \textcolor{gray}{[Updated on \today]}}
\end{center}

%----------------------------------------------------------------------------------------
% PROFILE
%----------------------------------------------------------------------------------------

\section{Professional Profile}

I recently obtained a PhD in Computer Science, specializing in Multi-Agent Reinforcement Learning (MARL) and Multi-Agent Systems, with a strong focus on autonomous decision-making, simulation, and safety-critical environments.
I have experience bridging research-grade AI methods with deployable architectures in defense and cyber-resilient systems through R\&D work at \textbf{Thales Land \& Air Systems (LAS)}.
My work emphasizes robustness, interpretability, and coordination in MARL under operational constraints.

\section{Technical and R\&D Interests}

MARL; Multi-Agent System Organizations; Cyberdefense \& Intelligent Autonomous Agents; Simulation \& Digital Twins; Safety-critical and Adversarial environments; World Models and Simulation-based development ; Software Architecture for MAS; Explainability in MARL.


%----------------------------------------------------------------------------------------
% EXPERIENCE
%----------------------------------------------------------------------------------------

\section{Appointments}

\cventry{2022--2025}{Doctoral Researcher}{Thales Land \& Air Systems + Université Grenoble Alpes}{France}{
    Design of multi-agent architectures for cyber-resilient and safety-critical defense systems (\href{https://link.springer.com/book/10.1007/978-3-031-29269-9}{\textbf{Autonomous Intelligent Cyberdefense Agents -- AICA}}).
    Application of MARL under organizational constraint.
    Development of simulation platforms for adversarial cyber scenarios.
}{}

\cventry{2021--2022}{Research Engineer}{Thales Land \& Air Systems}{Rennes}{France}{
    Multi-agent modeling of cyber environments and autonomous anomaly detection in Thales LAS context.
}{}

\cventry{2020--2021}{Software \& Research Engineer}{Atos}{France}{
    Software engineering for satellite command and control systems for \textbf{\textit{Centre National d'Etudes Spatiales}} and \textbf{European Spatial Agency}.
}{}

\cventry{2020}{Software Engineering Internship}{Atos}{Toulouse}{France}{Development of the \textbf{ISIS} (\textbf{CNES Initiative for Space Innovative Standards}) system for satellite command \& control (Python, Bash, KVM, Grafana, Django).}{}

\cventry{2019}{Software Engineering Internship}{SQLI}{Toulouse}{France}{Cybersecurity, maintenance, and software engineering tasks for \textbf{Airbus Helicopters} projects.}{}


%----------------------------------------------------------------------------------------
% EDUCATION
%----------------------------------------------------------------------------------------

\section{Education}

\cventry{2022--2025}{PhD in Computer Science}{Université Grenoble Alpes (UGA)}{France}{}{
    Dissertation defended on \textbf{17 November 2025}.
    Title: \textit{``On the Organization of a Cyberdefence Multi-Agent System''}.
    Status: \textbf{Degree conferred}.
    Supervisors: Jean-Paul Jamont, Michel Occello, Louis-Marie Traonouez, Paul Théron.
}

\cventry{2019}{Exchange Semester in Computer Engineering}{École de Technologie Supérieure (ETS)}{Montréal, Canada}{}{}

\cventry{2015--2021}{Integrated Bachelor's and Master's Degree}{INSA Rennes}{France}{}{
    Specialization in Computer Science and Cybersecurity.}


% \cventry{2015--2018}{Integrated Preparatory Cycle}{INSA Rennes}{France}{}{}

% \cventry{2012--2015}{High School}{Lycée Charles Renouvier}{Prades, France}{}{}


%----------------------------------------------------------------------------------------
% RESEARCH PROJECTS
%----------------------------------------------------------------------------------------

\section{Research \& Industrial Projects}

\cvitem{\textcolor{blue}{\href{https://github.com/julien6/CybMASDE}{CybMASDE}}}{\qquad A research platform combining MARL, organizational modeling, and cyberdefense simulation to support the design and analysis of intelligent multi-agent architectures (with application for \textbf{Thales}-related projects).}

\bigskip

\cvitem{\textcolor{blue}{\href{https://github.com/julien6/MOISE-MARL}{MOISE+MARL}}}{\qquad A proof-of-concept framework integrating organizational roles, missions, and goals into MARL environments to guide learning and improve agent coordination.}

%----------------------------------------------------------------------------------------
% PUBLICATIONS
%----------------------------------------------------------------------------------------

\section{Selected Publications}

 {\vspace{-0.5em} \hfill \tiny \textit{Full publication list available on request or via \href{https://dblp.org/pid/367/9947.html}{DBLP}.}}

\vspace{-1em}

\subsection*{Journal Articles}
\begin{itemize}
    \item \href{https://assets-eu.researchsquare.com/files/rs-7166037/v1_covered_908e23dd-6fb8-4efc-9ef3-a78c4d539bac.pdf?c=1753863562}{\textbf{JAAMAS (rev., 2025)}} -- \emph{Assisting Multi-Agent System Design with MOISE+ and MARL: The MAMAD Method} (Soulé et al.).
\end{itemize}

\subsection*{International Conferences}
\begin{itemize}
    \item \href{https://dl.acm.org/doi/10.5555/3709347.3743834}{\textbf{AAMAS 2025}} -- \emph{An Organizationally-Oriented Approach to Enhancing Explainability and Control in Multi-Agent Reinforcement Learning} (Soulé et al.).
    \item \href{https://ieeexplore.ieee.org/document/11120591}{\textbf{IEEE CLOUD 2025}} -- \emph{Streamlining Resilient Kubernetes Autoscaling with Multi-Agent Systems via an Automated Online Design Framework} (Soulé et al.). \textit{DOI: 10.1109/CLOUD67622.2025.00015}.
    \item \href{https://link.springer.com/chapter/10.1007/978-3-031-63223-5_24}{\textbf{AIAI 2024}} -- \emph{A MARL-Based Approach for Easing MAS Organization Engineering} (Soulé et al.).
    \item \href{https://ieeexplore.ieee.org/document/10394564}{\textbf{IEEE SMC 2023}} -- \emph{Towards a Multi-Agent Simulation of Cyber-attackers and Cyber-defenders Battles} (Soulé et al.). \textit{DOI: 10.1109/SMC53992.2023.10394564}.
\end{itemize}
\smallskip

\subsection*{Invited Talks}

\begin{itemize}
    \item \textbf{Mar. 2023} — \emph{Invited talk for the NATO \href{https://gdr-securite.irisa.fr/wp-content/uploads/RESSI2019-Projet-ChaireCyberResilienceAerospatiale.pdf}{\textbf{Cyb’Air}} chair}. \textbf{Thales}, \textbf{Dassault}, \textbf{\textit{École de l’air et de l’espace}}.
\end{itemize}

\section{Awards \& Distinctions}

\cvitem{2025}{\href{https://hal.science/hal-05151654}{Best Paper Award -- JFSMA 2025, for work on explainability and control in multi-agent reinforcement learning.}}


%----------------------------------------------------------------------------------------
% ADDITIONAL ACTIVITIES
%----------------------------------------------------------------------------------------

\section{Additional Activities}

\cvitem{2023 -- Now}{\textbf{Treasurer} of the \href{https://www.aica-iwg.org/}{\textbf{AICA International Work Group}}.}

\cvitem{2023 -- Now}{Conference reviewing in particular for \textbf{AAMAS}, \textbf{ICAART}, \textbf{IEEE SMC}, \textbf{ICDL}.}

%----------------------------------------------------------------------------------------
% SKILLS
%----------------------------------------------------------------------------------------

\section{Skills}

\cvitem{AI \& Learning}{
    RL/MARL, Learning under Constraints,
    Behavior Analysis and Interpretability
}

\cvitem{Multi-Agent Systems}{
    Agent Architectures, Coordination and Cooperation,
    Organizational Models, Distributed Decision-Making,
    Simulation and Digital Twins
}

\cvitem{Systems \& Engineering}{
    Python, Gym/PettingZoo, JAX / JaxMARL, PyTorch, Optuna, RLlib, MARLlib, JS/HTML/CSS, Flask, Docker, Linux, Kubernetes, HPC Platforms
}

\cvitem{Applied Domains}{
    Embedded Autonomous Systems, Drone Swarms (IoT/IoBT), Cyberdefense, Adversarial and Multi-Agent Environments
}

\cvitem{Languages}{
    French (native), English (professional),
    Spanish (intermediate), Catalan (intermediate), Japanese (basic)
}



\end{document}
