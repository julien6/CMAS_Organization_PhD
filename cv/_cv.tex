
%-----------------------------------------------------------------------------------------------------------------------------------------------%
%	The MIT License (MIT)
%
%	Copyright (c) 2021 Philip Empl
%
%	Permission is hereby granted, free of charge, to any person obtaining a copy
%	of this software and associated documentation files (the "Software"), to deal
%	in the Software without restriction, including without limitation the rights
%	to use, copy, modify, merge, publish, distribute, sublicense, and/or sell
%	copies of the Software, and to permit persons to whom the Software is
%	furnished to do so, subject to the following conditions:
%	
%	THE SOFTWARE IS PROVIDED "AS IS", WITHOUT WARRANTY OF ANY KIND, EXPRESS OR
%	IMPLIED, INCLUDING BUT NOT LIMITED TO THE WARRANTIES OF MERCHANTABILITY,
%	FITNESS FOR A PARTICULAR PURPOSE AND NONINFRINGEMENT. IN NO EVENT SHALL THE
%	AUTHORS OR COPYRIGHT HOLDERS BE LIABLE FOR ANY CLAIM, DAMAGES OR OTHER
%	LIABILITY, WHETHER IN AN ACTION OF CONTRACT, TORT OR OTHERWISE, ARISING FROM,
%	OUT OF OR IN CONNECTION WITH THE SOFTWARE OR THE USE OR OTHER DEALINGS IN
%	THE SOFTWARE.
%	
%
%-----------------------------------------------------------------------------------------------------------------------------------------------%


%============================================================================%
%
%	DOCUMENT DEFINITION
%
%============================================================================%

\documentclass[10pt,A4,english]{article}	


%----------------------------------------------------------------------------------------
%	ENCODING
%----------------------------------------------------------------------------------------

% we use utf8 since we want to build from any machine
\usepackage[utf8]{inputenc}		
\usepackage[USenglish]{isodate}
\usepackage[english]{babel}
\usepackage{fancyhdr}
\usepackage[numbers]{natbib}

%----------------------------------------------------------------------------------------
%	LOGIC
%----------------------------------------------------------------------------------------

% provides \isempty test
\usepackage{xstring, xifthen}
\usepackage{enumitem}
\usepackage{blindtext}
\usepackage{pdfpages}
\usepackage{changepage}
%----------------------------------------------------------------------------------------
%	FONT BASICS
%----------------------------------------------------------------------------------------

% some tex-live fonts - choose your own

\usepackage[defaultsans]{droidsans}
\usepackage[default]{comfortaa}
\usepackage{cmbright}
\usepackage[default]{raleway}
\usepackage{fetamont}
\usepackage[default]{gillius}
\usepackage[light,math]{iwona}
\usepackage[thin]{roboto} 

% set font default
\renewcommand*\familydefault{\sfdefault} 	
\usepackage[T1]{fontenc}

% more font size definitions
\usepackage{moresize}

%----------------------------------------------------------------------------------------
%	FONT AWESOME ICONS
%---------------------------------------------------------------------------------------- 

% include the fontawesome icon set
\usepackage{fontawesome}

% use to vertically center content
% credits to: http://tex.stackexchange.com/questions/7219/how-to-vertically-center-two-images-next-to-each-other
\newcommand{\vcenteredinclude}[1]{\begingroup
\setbox0=\hbox{\includegraphics{#1}}%
\parbox{\wd0}{\box0}\endgroup}
\newcommand{\tab}[1]{\hspace{.2\textwidth}\rlap{#1}}
% use to vertically center content
% credits to: http://tex.stackexchange.com/questions/7219/how-to-vertically-center-two-images-next-to-each-other
\newcommand*{\vcenteredhbox}[1]{\begingroup
\setbox0=\hbox{#1}\parbox{\wd0}{\box0}\endgroup}

% icon shortcut
\newcommand{\icon}[3] { 							
	\makebox(#2, #2){\textcolor{maincol}{\csname fa#1\endcsname}}
}	


% icon with text shortcut
\newcommand{\icontext}[4]{ 						
	\vcenteredhbox{\icon{#1}{#2}{#3}}  \hspace{2pt}  \parbox{0.9\mpwidth}{\textcolor{#4}{#3}}
}

% icon with website url
\newcommand{\iconhref}[5]{ 						
    \vcenteredhbox{\icon{#1}{#2}{#5}}  \hspace{2pt} \href{#4}{\textcolor{#5}{#3}}
}

% icon with email link
\newcommand{\iconemail}[5]{ 						
    \vcenteredhbox{\icon{#1}{#2}{#5}}  \hspace{2pt} \href{mailto:#4}{\textcolor{#5}{#3}}
}

%----------------------------------------------------------------------------------------
%	PAGE LAYOUT  DEFINITIONS
%----------------------------------------------------------------------------------------

% page outer frames (debug-only)
% \usepackage{showframe}		

% we use paracol to display breakable two columns
\usepackage{paracol}
\usepackage{tikzpagenodes}
\usetikzlibrary{calc}
\usepackage{lmodern}
\usepackage{multicol}
\usepackage{lipsum}
\usepackage{atbegshi}
% define page styles using geometry
\usepackage[a4paper]{geometry}

% remove all possible margins
\geometry{top=1cm, bottom=1cm, left=0.5cm, right=1cm}

\usepackage{fancyhdr}
\pagestyle{empty}

% space between header and content
% \setlength{\headheight}{0pt}

% indentation is zero
\setlength{\parindent}{0mm}

%----------------------------------------------------------------------------------------
%	TABLE /ARRAY DEFINITIONS
%---------------------------------------------------------------------------------------- 

% extended aligning of tabular cells
\usepackage{array}

% custom column right-align with fixed width
% use like p{size} but via x{size}
\newcolumntype{x}[1]{%
>{\raggedleft\hspace{0pt}}p{#1}}%


%----------------------------------------------------------------------------------------
%	GRAPHICS DEFINITIONS
%---------------------------------------------------------------------------------------- 

%for header image
\usepackage{graphicx}

% use this for floating figures
\usepackage{wrapfig}
\usepackage{float}
\floatstyle{boxed} 
\restylefloat{figure}

%for drawing graphics		
\usepackage{tikz}			
\usepackage{ragged2e}	
%\usetikzlibrary{shapes, backgrounds,mindmap, trees}

%----------------------------------------------------------------------------------------
%	Color DEFINITIONS
%---------------------------------------------------------------------------------------- 
\usepackage{transparent}
\usepackage{color}

% primary color
\definecolor{maincol}{RGB}{ 64,64,64}

% accent color, secondary
% \definecolor{accentcol}{RGB}{ 250, 150, 10 }

% dark color
\definecolor{darkcol}{RGB}{ 70, 70, 70 }

% light color
\definecolor{lightcol}{RGB}{245,245,245}

\definecolor{accentcol}{RGB}{59,77,97}



% Package for links, must be the last package used
\usepackage[hidelinks]{hyperref}

% returns minipage width minus two times \fboxsep
% to keep padding included in width calculations
% can also be used for other boxes / environments
\newcommand{\mpwidth}{\linewidth-\fboxsep-\fboxsep}
	


%============================================================================%
%
%	CV COMMANDS
%
%============================================================================%

%----------------------------------------------------------------------------------------
%	 CV LIST
%----------------------------------------------------------------------------------------

% renders a standard latex list but abstracts away the environment definition (begin/end)
\newcommand{\cvlist}[1] {
	\begin{itemize}{#1}\end{itemize}
}

%----------------------------------------------------------------------------------------
%	 CV TEXT
%----------------------------------------------------------------------------------------

% base class to wrap any text based stuff here. Renders like a paragraph.
% Allows complex commands to be passed, too.
% param 1: *any
\newcommand{\cvtext}[1] {
	\begin{tabular*}{1\mpwidth}{p{0.98\mpwidth}}
		\parbox{1\mpwidth}{#1}
	\end{tabular*}
}
\newcommand{\cvtextsmall}[1] {
	\begin{tabular*}{0.8\mpwidth}{p{0.8\mpwidth}}
		\parbox{0.8\mpwidth}{#1}
	\end{tabular*}
}
%----------------------------------------------------------------------------------------
%	CV SECTION
%----------------------------------------------------------------------------------------

% Renders a a CV section headline with a nice underline in main color.
% param 1: section title
\newcommand{\cvsection}[1] {
	\vspace{14pt}
	\cvtext{
		\textbf{\LARGE{\textcolor{darkcol}{#1}}}\\[-4pt]
		\textcolor{orange}{ \rule{0.2\textwidth}{1.5pt} } \\
	}
}

\newcommand{\cvsectionsmall}[1] {
	\vspace{14pt}
	\cvtext{
		\textbf{\Large{\textcolor{darkcol}{#1}}}\\[-4pt]
		\textcolor{accentcol}{ \rule{0.2\textwidth}{1.5pt} } \\
	}
}

\newcommand{\cvheadline}[1] {
	\vspace{16pt}
	\cvtext{
		\textbf{\Huge{\textcolor{accentcol}{#1}}}\\[-4pt]
		 
	}
}

\newcommand{\cvsubheadline}[1] {
	\vspace{16pt}
	\cvtext{
		\textbf{\huge{\textcolor{darkcol}{#1}}}\\[-4pt]
		 
	}
}
%----------------------------------------------------------------------------------------
%	META SKILL
%----------------------------------------------------------------------------------------

% Renders a progress-bar to indicate a certain skill in percent.
% param 1: name of the skill / tech / etc.
% param 2: level (for example in years)
% param 3: percent, values range from 0 to 1
\newcommand{\cvskill}[3] {
	\begin{tabular*}{1\mpwidth}{p{0.37\mpwidth} r }
 		\textcolor{black}{\textbf{#1}} & \textcolor{maincol}{#2}\\
	\end{tabular*}%
	
	\hspace{4pt}
	\begin{tikzpicture}[scale=1,rounded corners=2pt,very thin]
		\fill [lightcol] (0,0) rectangle (1\mpwidth, 0.15);
		\fill [accentcol] (0,0) rectangle (#3\mpwidth, 0.15);
  	\end{tikzpicture}%
}


%----------------------------------------------------------------------------------------
%	 CV EVENT
%----------------------------------------------------------------------------------------

% Renders a table and a paragraph (cvtext) wrapped in a parbox (to ensure minimum content
% is glued together when a pagebreak appears).
% Additional Information can be passed in text or list form (or other environments).
% the work you did
% param 1: time-frame i.e. Sep 14 - Jan 15 etc.
% param 2:	 event name (job position etc.)
% param 3: Customer, Employer, Industry
% param 4: Short description
% param 5: work done (optional)
% param 6: technologies include (optional)
% param 7: achievements (optional)
\newcommand{\cvevent}[7] {
	
	% we wrap this part in a parbox, so title and description are not separated on a pagebreak
	% if you need more control on page breaks, remove the parbox
	\parbox{\mpwidth}{
		\begin{tabular*}{1\mpwidth}{p{0.66\mpwidth}  r}
	 		\textcolor{black}{\textbf{#2}} & \colorbox{accentcol}{\makebox[0.32\mpwidth]{\textcolor{white}{\textbf{#1}}}} \\
			\textcolor{maincol}{#3} & \\
		\end{tabular*}\\[8pt]
	
		\ifthenelse{\isempty{#4}}{}{
			\cvtext{#4}\\
		}
	}
	\vspace{14pt}
}


%----------------------------------------------------------------------------------------
%	 CV META EVENT
%----------------------------------------------------------------------------------------

% Renders a CV event on the sidebar
% param 1: title
% param 2: subtitle (optional)
% param 3: customer, employer, etc,. (optional)
% param 4: info text (optional)
\newcommand{\cvmetaevent}[4] {
	\textcolor{maincol} { \cvtext{\textbf{\begin{flushleft}#1\end{flushleft}}}}

	\ifthenelse{\isempty{#2}}{}{
	\textcolor{black} {\cvtext{\textbf{#2}} }
	}

	\ifthenelse{\isempty{#3}}{}{
		\cvtext{{ \textcolor{maincol} {#3} }}\\
	}

	\cvtext{#4}\\[14pt]
}



% HEADER AND FOOOTER 
%====================================
\newcommand\Header[1]{%
\begin{tikzpicture}[remember picture,overlay]
\fill[accentcol]
  (current page.north west) -- (current page.north east) --
  ([yshift=50pt]current page.north east|-current page text area.north east) --
  ([yshift=50pt,xshift=-3cm]current page.north|-current page text area.north) --
  ([yshift=10pt,xshift=-5cm]current page.north|-current page text area.north) --
  ([yshift=10pt]current page.north west|-current page text area.north west) -- cycle;
\node[font=\sffamily\bfseries\color{white},anchor=west,
  xshift=0.7cm,yshift=-0.32cm] at (current page.north west)
  {\fontsize{12}{12}\selectfont {#1}};
\end{tikzpicture}%
}

\newcommand\Footer[1]{%
\begin{tikzpicture}[remember picture,overlay]
\fill[lightcol]
  (current page.south east) -- (current page.south west) --
  ([yshift=-80pt]current page.south east|-current page text area.south east) --
  ([yshift=-80pt,xshift=-6cm]current page.south|-current page text area.south) --
  ([xshift=-2.5cm,yshift=-10pt]current page.south|-current page text area.south) --	
  ([yshift=-10pt]current page.south east|-current page text area.south east) -- cycle;
\node[yshift=0.32cm,xshift=9cm] at (current page.south) {\fontsize{10}{10}\selectfont \textbf{\thepage}};
\end{tikzpicture}%
}


%=====================================
%============================================================================%
%
%
%
%	DOCUMENT CONTENT
%
%
%
%============================================================================%
\begin{document}


% LEBENSLAUF HIERE
\AtBeginShipoutFirst{\Header{CV}\Footer{1}}
\AtBeginShipout{\AtBeginShipoutAddToBox{\Header{CV}\Footer{2}}}

\newpage

\colseprulecolor{orange}
\columnratio{0.28}
\setlength{\columnsep}{2.2em}
\setlength{\columnseprule}{3pt}
\begin{paracol}{2}


	\begin{leftcolumn}
		%---------------------------------------------------------------------------------------
		%	META IMAGE
		%----------------------------------------------------------------------------------------
		\includegraphics[width=\linewidth]{photo_Vincent.jpg}	%trimming relative to image size

		%---------------------------------------------------------------------------------------
		%	META SKILLS
		%----------------------------------------------------------------------------------------
		\fcolorbox{white}{white}{\begin{minipage}[c][1.5cm][c]{1\mpwidth}
				\LARGE{\textbf{\textcolor{maincol}{Vincent Marguet}}} \\[2pt]
				\normalsize{ \textcolor{maincol} {Doctor in Automation and Production Engineering} }
			\end{minipage}} \\
		\icontext{CaretRight}{12}{Born on 14.09.1998 in Arnas}{black}\\[6pt]
		\icontext{CaretRight}{12}{French, Romanian}{black}\\[6pt]
		\icontext{CaretRight}{12}{Driving licence}{black}\\[6pt]
		\icontext{CaretRight}{12}{Unmarried}{black}\\[6pt]
		\newline\newline\newline\newline
		{I enjoyed my research on motion planning, I have already written my thesis and defended on December 3rd 2024.}
		\newline\newline\newline\newline\newline\newline\newline
		{I followed the 3 last years of the ESISAR-Grenoble INP engineering cycle. I greatly enjoyed the school's interactions with industry through \\projects and interesting courses that always went from theory to practice.}
		\newline\newline\newline\newline\newline
		{Only first ranked students had the chance to be admitted for an exchange semester. I chose one of the best universities in my field not only to expand my scientific knowledge in control engineering but also because I am fascinated by the Portuguese language and the Brazilian culture.}
		\newline\newline\newline\newline
		{Working hard for the exams required rigor and good organization. My interest in exact sciences was paramount for my future career.}
		\newpage




		{~\vspace{2.5cm}
			\newline\newline\newline\newline\newline\newline\newline
			I had the opportunity to teach the practical works in automation to students in my former engineering school.}
		\newline\newline\newline\newline
		{I enjoyed working in a company, with young motivated colleagues who\\ shared with me their experience but also asked for my help. I felt part of the team and I enjoyed the environment.}
		\newline\newline\newline\newline
		{%\vspace{0.3cm} 

			I was fascinated by the topic of my innovation project and I felt like a true engineer when handling different tasks. I also enjoyed working in a team.}
		\newline\newline\newline\newline
		{This project gave me the chance to improve my programming skills and share my knowledge of communication protocols with an excellent student.}
		\newline\newline\newline\newline\newline\newline\newline\newline\newline\newline
		{I am keen on discovering new courses that apply theory to real systems. I used the motion capture system Qualisys Track Manager for my experiments involving Crazyflie\\ nanodrones.}
		\newline\newline\newline\newline\newline\newline\newline\newline\newline\newline\newline\newline
		{Playing chess enabled me to anticipate everything I do in my life and to keep me concentrated and calm.}
		\newline\newline\newline\newline\newline\newline\newline\newline\newline\newline\newline\newline\newline\newline\newline
		{I started to learn Portuguese with the application \emph{Duolingo} before coming to Brazil and I then mastered this language by practicing with local people.}
		\newline\newline\newline\newline\newline\newline\newline
		{I am fond of practicing intellectual but also physical sports, alone and as part of a team. The most important is to enjoy what I do.}
		\newline\newline\newline\newline\newline\newline\newline\newline\newline
		{Some official diplomas showing my interests and abilities.}
		\newline\newline\newline\newline\newline\newline\newline\newline\newline\newline\newline\newline\newline\newline\newline\newline\newline\newline\newline\newline\newline\newline\newline\newline\newline\newline\newline\newline\newline\newline\newline\newline\newline\newline\newline\newline
		{During my PhD, I was involved in many international collaborations that lead to publications in top journals and conferences in the field.}
		\newline\newline\newline\newline\newline\newline\newline\newline\newline\newline\newline\newline\newline\newline\newline\newline\newline\newline\newline\newline\newline\newline\newline
		{Do not hesitate to contact them! They will respond to your questions with pleasure!}
	\end{leftcolumn}

	\begin{rightcolumn}
		%---------------------------------------------------------------------------------------
		%	TITLE  HEADER
		%----------------------------------------------------------------------------------------
		\cvsection{Contact}

		\icontext{MapMarker}{16}{51, Rue Antoine Martin, 69400 Villefranche sur Saône, FRANCE}{black}\\[6pt]
		\icontext{MobilePhone}{16}{ +33 (0) 6 44 24 74 30}{black}\\[6pt]
		\iconemail{Envelope}{16}{marguetvincent7@gmail.com, vincent.marguet@lcis.grenoble-inp.fr}{}{black}{black}\\[6pt]
		\hspace*{0.2cm}\includegraphics[width=0.4cm]{458-linkedin.pdf}\hspace{0.35cm}
		{www.linkedin.com/in/vincent-marguet}

		%---------------------------------------------------------------------------------------
		%	PROFILE
		%----------------------------------------------------------------------------------------
		\cvsection{Summary}
		\vspace{4pt}

		\cvtext{
			I consider myself an organized, highly motivated and enthusiastic student in Control Engineering with experience in Unmanned Aerial Vehicles (UAVs) and Systems control. I enjoyed my PhD in which I have used my skills in programming,  system identification, modeling and control. I am intellectually curious, able to integrate into a team with excellent communication and interpersonal skills.
		}


		%---------------------------------------------------------------------------------------
		%	EDUCATION
		%----------------------------------------------------------------------------------------
		\vspace{10pt}
		\cvsection{Education}
		\vspace{4pt}

		\cvevent
		{10/2021 - 12/2024}
		{PhD "Motion planning for multi-agent dynamical systems in a B-spline
			framework"}
		{LCIS (Laboratoire de Conception et d'Intégration des Systèmes), team CO4SYS (Coordination, Coopération, Contrôle des SYStèmes complexes),
			LabEx PERSYVAL-Lab,
			Univ. Grenoble Alpes, Valence, France.
		}
		{}
		\vfill\null

		\cvevent
		{09/2018 - 07/2021}
		{Engineer diploma}
		{ESISAR (\'Ecole Nationale Supérieure en Systèmes Avancés et Réseaux), Grenoble INP (Institut National Polytechnique de Grenoble), Univ. Grenoble Alpes, Valence, France.
		} %Major in Computer science and Automation
		{Major in ISC (Engineering of Complex Systems) specialized in Automation (Decentralized control of complex systems, Diagnostics and robust control, Modeling and control of non-linear systems, Control optimization, Complex systems: dynamic processes defined on large-scale networks.)
			\\Grade Point Average in 5th year: 15.508 out of 20 (2nd out of 61 students).\\Grade Point Average in 4th year: 13.238 out of 20 (13th out of 68 students).\\Grade Point Average in 3rd year: 14.403 out of 20 (5th out of 109 students).}
		\vfill\null

		\cvevent
		{02/2020 – 07/2020}
		{Exchange semester in  Brazil during 4th year at ESISAR}
		{Escola Politecnica, Universidade de São Paulo (USP)}
		{Courses in Portuguese at the best University of Latin America: Introduction to Robust Control Systems Design, Multi-variable Control, Probabilistic Models, Project Management. Adaptation to a different culture.\\Grade Point Average: 8.59 out of 10.}
		\vfill\null

		\cvevent
		{09/2016 – 07/2018}
		{CPGE (Intensive general program for top-ranking higher education establishments in sciences)}
		{Lycée du Parc, Lyon\\Physical, Chemistry and Engineering sciences (PCSI, PC*)}
		{Dedicated to study hard for preparing the admission exams for an Engineering School. Interest in Mathematics, Physics and Chemistry. Succeeded to enter in PC* (Physics Chemistry star) class with the best students of this famous establishment.}
		\vfill\null
		\newpage

		\cvevent
		{09/2009 – 07/2016}
		{High School Diploma}
		{Notre-Dame de Mongré, Villefranche-sur-Saône}
		{Brevet des collèges (General Certificate of Secondary Education) with Highest Honours and Baccalauréat général scientifique (A Levels) with Highest Honours overall score $17.65$ out of $20$.}
		\vfill\null

		%---------------------------------------------------------------------------------------
		%	PROFESSIONAL EXPERIENCE
		%----------------------------------------------------------------------------------------

		\cvsection{Professional experience}
		\cvevent
		{09/2021-07/2024}
		{Teacher in practical works in automation}
		{ESISAR - INP Grenoble, Valence}
		{Animated practical sessions using Matlab/Simulink, manipulated real systems (a motor and a thermal engine), evaluated reports of students in 2nd year and 4th year (Apprentice) after the Baccalauréat général scientifique.}
		\vfill\null

		\newline

		\cvevent
		{1/02/2021-2/07/2021}
		{Diploma project}
		{ Tronics Microsystems, Crolles, France}
		{Modeled and simulated an accelerometer and a gyrometer produced by the company in Matlab/Simulink. Compared the results obtained in Simulink with the results measured on real systems. Presented my work to my French team and our German partners.}
		\vfill\null

		\newline

		\cvevent
		{09/2020-12/2020}
		{Innovation project: Sense and avoidance strategies for collision avoidance of drones implemented in ROS}
		{Team leader of 4 students\newline ESISAR - INP Grenoble, Valence}
		{Programmed an algorithm in Python for Crazyflie 2.1 drones to detect cylindrical obstacles with a Lidar and avoid them. Installed and discovered the Ubuntu, ROS and Gazebo environments. Collaborated with classmates of the team to share the tasks, write the final report and prepare the oral presentation, which was elected the best presentation by the class.}
		\vfill\null
		\newline\newline

		\cvevent
		{17/06/2019-9/08/2019}
		{Technician Internship in Brazil}
		{Universidade Federal de Itajubá (UNIFEI)}
		{Helped a student for her final project on a conveyor.
			Improved my knowledge in Java and software communications.
			Programmed an applet in Java communicating with the software CPN Tools.}
		\vfill\null


		\cvsection{Technical Skills}

		\icontext{CaretRight}{12}{Automation: (Non) linear dynamical systems, Modeling and identification, System simulation, Nonlinear control, Numerical methods, Distributed systems, Optimal control.}{black}\\[6pt]
		\icontext{CaretRight}{12}{Programming languages: Matlab/Simulink, Python, Java}{black}\\[6pt]
		\icontext{CaretRight}{12}{Operating systems: Windows, Linux, Ubuntu}{black}\\[6pt]
		\icontext{CaretRight}{12}{Database: SQL}{black}\\[6pt]
		\icontext{CaretRight}{12}{ CPN Tools,  ROS (Robotic Operating System), Gazebo}{black}\\[6pt]
		\icontext{CaretRight}{12}{Text editing and processing: Microsoft Office Suite, \LaTeX}{black}\\[6pt]
		\icontext{CaretRight}{12}{Motion Capture System: Qualisys Track Manager with markers on Crazyflie nanodrones}{black}\\[6pt]
		\newline

		\cvsection{Personal Skills}
		\cvevent
		{2005-today}
		{Volunteering in my chess clubs during events}
		{Calade Echecs, Villefranche sur Saône\newline Valence Echecs, Valence}
		{Animated a stand in order to share my passion, teach the rules and play against several players simultaneously.
			Installed and tidied up the tables, chairs and chess games during tournaments. Refereed a friendly tournament.}
		\icontext{CaretRight}{12}{}{}\\[6pt]
		\icontext{CaretRight}{12}{Analytical thinking}{black}\\[6pt]
		\icontext{CaretRight}{12}{Perfectionist}{black}\\[6pt]
		\icontext{CaretRight}{12}{Strong motivational and leadership skills}{black}\\[6pt]
		\icontext{CaretRight}{12}{Ability to work under pressure}{black}\\[6pt]
		\icontext{CaretRight}{12}{Ability to work individually as well as in a team}{black}

		\cvsection{Languages}

		\icontext{CaretRight}{12}{French (Mother Tongue)}{black}\\[6pt]
		\icontext{CaretRight}{12}{English (Fluent, TOEIC 830/990)}{black}\\[6pt]
		\icontext{CaretRight}{12}{Portuguese (Fluent)}{black}\\[6pt]
		\icontext{CaretRight}{12}{German (Working Knowledge)}{black}\\[6pt]
		\icontext{CaretRight}{12}{Romanian (Intermediate Level)}{black}

		\cvsection{Interests}

		\icontext{CaretRight}{12}{Chess player affiliated at International Chess Federation (FIDE): ELO ratings: 1807 (classical), 1751 (rapid)}{black}\\[6pt]
		%\icontext{CaretRight}{12}{Pétanque}{black}\\[6pt]
		\icontext{CaretRight}{12}{Volley-ball}{black}\\[6pt]
		\icontext{CaretRight}{12}{Ski in the Alpes}{black}\\[6pt]
		\icontext{CaretRight}{12}{Travel to discover new landscapes and cultures}{black}

		\cvsection{Certifications}

		\icontext{CaretRight}{12}{Master Diploma in Science Engineering\newline (Institut Polytechnique de Grenoble)}{black}\\[6pt]
		\newline
		\icontext{CaretRight}{12}{BIA (Brevet d'Initiation à l'Aéronautique) \newline (Introduction to aeronautics certificate)}{black}\\[6pt]
		\newline
		\icontext{CaretRight}{12}{DIFFE (Diplôme d'Initiateur de la Fédération Française des Echecs)\newline  (Initiator Diploma of the French Chess Federation)}{black}\\[6pt]


		\cvsection{Publications}

		\textbf{Journal papers:}

		\begin{itemize}
			\item \label{journal_cep} \textbf{V. Marguet}, C. K. Dinh, F. Stoican, I. Prodan: \textcolor{blue}{Indoor formation motion planning using B-splines parametrization and evolutionary optimization}, \emph{Control Engineering Practice}, \emph{https://doi.org/10.1016/j.conengprac.2024.106066}, Volume 152, pp 106066, Elsevier, 2024.
		\end{itemize}

		\begin{itemize}
			\item \label{journal_2} \textbf{V. Marguet}, C. K. Dinh, J. Barreiro-Gomez, I. Prodan, C. Ocampo-Martinez, N. Quijano: \textcolor{blue}{Reliable Reconfiguration of Multi-Agent Formations using B-Splines and Replicator Dynamics}, in preparation, to be submitted in Dec. 2024 to IEEE Transations in Control Systems Technology.
		\end{itemize}

		\textbf{Accepted international conference papers:}
		\begin{itemize}
			\item \label{conf_cdc} \textbf{V. Marguet}, F. Stoican, I. Prodan: \textcolor{blue}{Energy-Efficient Trajectory Planning with B-Splines and the Schoenberg Quasi-Interpolant}. \emph{The 63rd IEEE Conference on Decision and Control}, CDC'24, Milan, Italy, 16-19 December 2024.
			\item \label{conf_med} F. Stoican, \textbf{V. Marguet}, D. Popescu, I. Prodan, L. Ichim: \textcolor{blue}{On the Energy Consumption of a Quadcopter Navigating in an Orchard Environment}. \emph{The 32nd Mediterranean Conference on Control and Automation}, MED'24, Chania, Crete, Greece, 11-14 June 2024. DOI: 10.1109/MED61351.2024.10566251
			\item \label{conf_icuas} \textbf{V. Marguet}, C. K. Dinh, I. Prodan, F. Stoican: \textcolor{blue}{Constrained PSO-splines trajectory generation for an indoor nanodrone}. \emph{2024 International Conference on Unmanned Aircraft Systems}, ICUAS'24, Chania, Crete, Greece, 4-7 June 2024.\\ DOI: 10.1109/ICUAS60882.2024.10556977
			\item \label{conf_ecc} \textbf{V. Marguet}, F. Stoican, I. Prodan: \textcolor{blue}{On the application of the Schoenberg quasi-interpolant for complexity reduction in trajectory generation}. \emph{The 21st European Control Conference}, ECC'23, Bucharest, Romania, 13-16 June 2023.\\ DOI: 10.23919/ECC57647.2023.10178175
		\end{itemize}

		\textbf{Accepted national workshop papers:}
		\begin{itemize}
			\item \label{workshop_sagip2024_cep} \textbf{V. Marguet}, C. K. Dinh, I. Prodan, F. Stoican: \textcolor{blue}{Formation motion planning for indoor applications using B-splines parametrization and evolutionary computation}. \emph{The 2nd annual congress of Automation, Industrial Engineering and Production Society (Société d’Automatique, de Génie Industriel et de Productique)}, SAGIP'23, Villeurbanne, France, 29-31 May 2024.
			\item \label{workshop_sagip2024_gt} C. K. Dinh, \textbf{V. Marguet}, I. Prodan, C. Ocampo-Martinez: \textcolor{blue}{Evolutionary games for the distributed control of multiple nanodrones switching formation with experimental validation}. \emph{The 2nd annual congress of Automation, Industrial Engineering and Production Society (Société d’Automatique, de Génie Industriel et de Productique)}, SAGIP'23, Villeurbanne, France, 29-31 May 2024.
			\item \label{workshop_sagip2023} \textbf{V. Marguet}, V. Casagrande, F. Boem, I. Prodan: \textcolor{blue}{Trajectory planning for multicopters connectivity maintenance through distributed optimization}. \emph{The 1st annual congress of Automation, Industrial Engineering and Production Society (Société d’Automatique, de Génie Industriel et de Productique)}, SAGIP'23, Marseille, France, 7-9 June 2023.
		\end{itemize}

		\cvsection{References}

		\iffalse Dr. Luiz Edival de Souza, Internship tutor at UNIFEI \\
			\iconemail{Envelope}{16}{edival@unifei.edu.br}{ }{black}\\[6pt]
			\\ \fi Dr. Guillaume Papin, Diploma Project tutor at Tronics Microsystems\\
		\iconemail{Envelope}{16}{guillaume.papin@tronicsgroup.com}{ }{black}{black}\\[6pt]
		\iffalse \\Dr. Laurent Lefevre, Diploma Project tutor at Esisar \\
			\iconemail{Envelope}{16}{laurent.lefevre@lcis.grenoble-inp.fr}{ }{black}\\ \fi
		\\Dr. Ionela Prodan, Drone Project tutor at Esisar and PhD Supervisor \\
		\iconemail{Envelope}{16}{ionela.prodan@lcis.grenoble-inp.fr}{ }{black}{black}\\[6pt]

		\hrulefill\newline
		\raggedleft \LARGE{\textbf{\textcolor{maincol}{Vincent Marguet}}} \\[2pt]

	\end{rightcolumn}
\end{paracol}


\end{document}