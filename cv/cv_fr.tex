\documentclass[11pt,a4paper]{moderncv}

% ModernCV settings
\moderncvstyle{classic}
\moderncvcolor{blue}

% Packages
\usepackage[utf8]{inputenc}
\usepackage[T1]{fontenc}
\usepackage[scale=0.85]{geometry}
\usepackage{setspace}
\setstretch{1.05}
\usepackage{parskip}
\usepackage{csquotes}

% \usepackage[hidelinks]{hyperref}
\usepackage{biblatex}
\moderncvicons{awesome}
\usepackage{academicons}

\newcommand{\dblp}[1]{\collectionadd[twitter]{socials}{\protect\httplink[#1]{dblp.uni-trier.de/search?q=#1}}}

\renewcommand*{\twittersocialsymbol}{%
    \raisebox{-0.08em}{%
    \scalebox{0.6}{%
        \tikz{\draw node[rectangle, rounded corners, draw=color2, inner sep=1pt] (0,0) {\textsf{dblp}};}~% 
    }}}

% Personal Information
\name{Julien}{Soulé}
\title{Docteur en informatique \\ IA \& Systèmes Multi-Agents \\ \ \\}
\address{35 Rue Mathieu-de-la-Drôme}{26000 Valence, France}{}

\email{julien.soule@hotmail.fr}
\phone[mobile]{+33~(0)~6~77~63~12~13}
\homepage{julien6.github.io/home/}

\extrainfo{
  \faLinkedin\ 
  \href{https://www.linkedin.com/in/julien-soul\%C3\%A9-6b2b27173}{julien-soulé-6b2b27173}\\
  \noindent
  \twittersocialsymbol~\href{https://dblp.org/pid/367/9947.html}{homepages/367/9947}\\
  \noindent
  \aiOrcid~\href{https://orcid.org/0009-0002-3218-2614}{0009-0002-3218-2614}\\
  \noindent
  \faGithub~\href{https://github.com/julien6}{julien6}
}

\photo[110pt][0pt]{photo.jpeg}

% === Load your .bib file ===
\addbibresource{references.bib}

% === Sorting by year (descending), then author ===
\ExecuteBibliographyOptions{sorting=ydnt}

\setlength{\hintscolumnwidth}{2.5cm}

\begin{document}
\makecvtitle

\vspace{-1cm}
\begin{center}
    {\tiny
        \noindent \textcolor{gray}{[Mise à jour le 12/09/2025]}}
\end{center}


%----------------------------------------------------------------------------------------
% RESEARCH PROFILE
%----------------------------------------------------------------------------------------

\section{Profil de recherche}

Je suis docteur en informatique, spécialisé en systèmes multi-agents et en cyberdéfense.
Mes travaux de recherche portent sur la manière de combiner des modèles organisationnels
et l'apprentissage par renforcement multi-agent afin de concevoir des agents de cyberdéfense
intelligents, explicables et résilients.
Je développe des méthodes exploitant des contraintes organisationnelles pour guider
l'apprentissage, améliorer la coordination au niveau du système et renforcer la
contrôlabilité et l'interprétabilité dans des environnements adversariaux complexes.


%----------------------------------------------------------------------------------------
% RESEARCH INTERESTS
%----------------------------------------------------------------------------------------

\section{Axes de recherche}
Apprentissage par renforcement multi-agent (MARL) ; organisations de systèmes multi-agents ;\\
Cyberdéfense \& agents cyber intelligents ; simulation \& jumeaux numériques ; architecture logicielle SMA.

%----------------------------------------------------------------------------------------
% EDUCATION
%----------------------------------------------------------------------------------------

\section{Formation}

\cventry{2022--2025}{Doctorat en informatique}{Université Grenoble Alpes (UGA)}{France}{}{
    Thèse soutenue le \textbf{17 novembre 2025}.
    Titre : \textit{« On the Organization of a Cyberdefence Multi-Agent System »}.
    Statut : \textbf{diplôme délivré}.
    Directeurs de thèse : Jean-Paul Jamont, Michel Occello, Louis-Marie Traonouez, Paul Théron.
}

\cventry{2018--2021}{Diplôme d'ingénieur en informatique}{INSA Rennes}{France}{}{
    Spécialisation en informatique et cybersécurité.}

\cventry{2019}{Semestre d'échange en informatique}{École de Technologie Supérieure (ETS)}{Montréal, Canada}{}{}

\cventry{2015--2018}{Cycle préparatoire intégré}{INSA Rennes}{France}{}{}

\cventry{2012--2015}{Lycée}{Lycée Charles Renouvier}{Prades, France}{}{}


%----------------------------------------------------------------------------------------
% APPOINTMENTS / EXPERIENCE
%----------------------------------------------------------------------------------------

\section{Postes et expériences}

\cventry{2022--2025}{Doctorant}{Thales Land \& Air Systems + Université Grenoble Alpes}{}{France}{
    Travaux de recherche doctorale sur les systèmes multi-agents organisationnels,
    l'apprentissage par renforcement multi-agent et les architectures de cyberdéfense.
    Thèse soutenue le 17 novembre 2025.
}

\cventry{2021--2022}{Ingénieur de recherche}{Thales Land \& Air Systems}{Rennes}{France}{
    Étude sur la modélisation multi-agent des environnements cyber et la détection d'anomalies
    dans le contexte de Thales LAS.}

\cventry{2020--2021}{Ingénieur logiciel et recherche}{Atos}{Toulouse}{France}{
    Ingénierie logicielle pour des systèmes de gestion de l'espace aérien développés avec le CNES et l'ESA.}

\cventry{2020}{Stage en ingénierie logicielle}{Atos}{Toulouse}{France}{
    Développement du système ISIS pour le commandement et le contrôle de satellites
    (Python, Bash, KVM, Grafana, Django).}

\cventry{2019}{Stage en ingénierie logicielle}{SQLI}{Toulouse}{France}{
    Cybersécurité, maintenance et ingénierie logicielle pour des projets Airbus Helicopters.
}

%----------------------------------------------------------------------------------------
% RESEARCH PROJECTS
%----------------------------------------------------------------------------------------

\section{Projets de recherche}

\cvitem{\textcolor{blue}{\href{https://github.com/julien6/CybMASDE}{CybMASDE}}}{\qquad
    Plateforme de recherche combinant apprentissage par renforcement multi-agent,
    modélisation organisationnelle et simulation de cyberdéfense pour soutenir
    la conception et l'analyse d'architectures multi-agents intelligentes.}

\bigskip

\cvitem{\textcolor{blue}{\href{https://github.com/julien6/MOISE-MARL}{MOISE+MARL}}}{\qquad
    Cadre de preuve de concept intégrant rôles, missions et objectifs organisationnels
    dans des environnements MARL afin de guider l'apprentissage et améliorer
    la coordination des agents.}


%----------------------------------------------------------------------------------------
% PUBLICATIONS — FULL LIST
%----------------------------------------------------------------------------------------

\section{Publications}

\smallskip

\subsection*{Articles de journal}
\nocite{soulej2025jaamas}
\printbibliography[keyword=journal, heading=none, sorting=ydnt]

\smallskip

\subsection*{Conférences internationales}
\nocite{soulej2025cloud}
\nocite{soule2024moise_marl}
\nocite{soule2024marl}
\nocite{soulej2023sim}
\printbibliography[keyword=international, heading=none, sorting=ydnt]

\smallskip

\subsection*{Conférences nationales (France)}
\nocite{soule2025jfsma}
\nocite{soule2024outil}
\nocite{soule2024approche}
\nocite{soule2023jfsmathese}
\nocite{soule2023ressithese}
\nocite{soule2023rjciathese}
\printbibliography[keyword=national, heading=none, sorting=ydnt]

\smallskip

\subsection*{Conférences invitées}
\nocite{soule2023cybairtalk}
\printbibliography[keyword=talk, heading=none, sorting=ydnt]

%----------------------------------------------------------------------------------------
% TEACHING EXPERIENCE
%----------------------------------------------------------------------------------------

\section{Expérience d'enseignement}

\cventry{2023--2025}{Chargé de travaux dirigés}{Valence (École d'ingénieurs UGA \& IUT)}{}{}{
    Travaux dirigés et travaux pratiques en systèmes d'exploitation,
    programmation système et gestion des processus.
    Encadrement de groupes d'étudiants travaillant sur des projets de cyberdéfense
    issus de problématiques industrielles.}


%----------------------------------------------------------------------------------------
% SERVICE
%----------------------------------------------------------------------------------------

\section{Responsabilités collectives}

\cvitem{2023--Présent}{Trésorier du groupe de travail international sur les agents cyber autonomes et intelligents (AICA).}


%----------------------------------------------------------------------------------------
% REVIEWING ACTIVITIES
%----------------------------------------------------------------------------------------

\section{Activités de relecture scientifique}

\cvitem{2025}{Relecteur pour la 17e Conférence internationale sur les agents et l'intelligence artificielle (\textbf{ICAART} 2025).}

\cvitem{2025}{Relecteur pour la 25e Conférence internationale sur les agents autonomes et les systèmes multi-agents (\textbf{AAMAS} 2026).}

\cvitem{2025}{Relecteur pour la Conférence internationale IEEE sur le développement et l'apprentissage (\textbf{ICDL} 2025).}

\cvitem{2024}{Relecteur pour la 16e Conférence internationale sur les agents et l'intelligence artificielle (\textbf{ICAART} 2024).}

\cvitem{2024}{Relecteur pour la Conférence internationale IEEE sur les systèmes, l'homme et la cybernétique (\textbf{SMC} 2025).}



%----------------------------------------------------------------------------------------
% AWARDS AND DISTINCTIONS
%----------------------------------------------------------------------------------------

\section{Prix et distinctions}
\cvitem{2025}{Prix du meilleur article à JFSMA 2025 pour \textquote{Une approche organisationnelle pour améliorer l'explicabilité et le contrôle dans l'apprentissage par renforcement multi-agent}.}


%----------------------------------------------------------------------------------------
% SKILLS
%----------------------------------------------------------------------------------------

\section{Compétences}

\cvitem{Langues}{Français (langue maternelle), anglais (professionnel, TOEIC 910), japonais (notions).}



\end{document}
