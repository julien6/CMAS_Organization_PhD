\documentclass[11pt,a4paper,sans]{moderncv}

% Style & Colors
\moderncvstyle{classic}
\moderncvcolor{blue}

% Packages
\usepackage[utf8]{inputenc}
\usepackage[scale=0.85]{geometry}
\usepackage{setspace}
\setstretch{1.05}

% Personal Information
\name{Julien}{Soulé}
\title{PhD in Computer Science --- Multi-Agent Systems \& Cyberdefense}
\email{julien.soule@univ-grenoble-alpes.fr}
\phone[mobile]{+33~6~77~63~12~13}
\address{35 Rue Mathieu-de-la-Drôme}{26000 Valence}{France}
\social[linkedin]{}
\homepage{}
\extrainfo{French nationality}

\begin{document}
\makecvtitle

%----------------------------------------------------------------------------------------
% RESEARCH PROFILE
%----------------------------------------------------------------------------------------

\section{Research Profile}

I am a PhD researcher in Multi-Agent Systems and Cyberdefense, focusing on the integration of
organizational models and multi-agent reinforcement learning to design intelligent, explainable,
and resilient cyberdefence agents. My work explores how organizational constraints can guide learning,
improve system-level coordination, and support controllability and explainability of autonomous agents
in complex cyber environments.

%----------------------------------------------------------------------------------------
% RESEARCH INTERESTS
%----------------------------------------------------------------------------------------

\section{Research Interests}

Multi-Agent Reinforcement Learning (MARL); Multi-Agent System Organizations;\\
Cyberdefense \& Intelligent Cyber Agents; Software Architecture for MAS; Simulation \& Digital Twins.

%----------------------------------------------------------------------------------------
% EDUCATION
%----------------------------------------------------------------------------------------

\section{Education}

\cventry{2022--2025}{PhD in Computer Science}{Université Grenoble Alpes (UGA)}{France}{}{
    Dissertation: \textit{``On the Organization of a Cyberdefence Multi-Agent System''}.\\
    Supervisors: Jean-Paul Jamont, Michel Occello, Louis-Marie Traonouez, Paul Théron.
}

\cventry{2018--2021}{Master's Degree in Computer Engineering}{INSA Rennes}{France}{}{
    Department of Computer Science and Cybersecurity.}

\cventry{2019}{Exchange Semester in Computer Engineering}{École de Technologie Supérieure (ETS)}{Montreal, Canada}{}{}

\cventry{2015--2018}{Integrated Preparatory Cycle}{INSA Rennes}{France}{}{}

\cventry{2012--2015}{High School}{Lycée Charles Renouvier}{Prades, France}{}{}

%----------------------------------------------------------------------------------------
% APPOINTMENTS / EXPERIENCE
%----------------------------------------------------------------------------------------

\section{Appointments}

\cventry{2022--2025}{PhD Researcher}{Thales Land \& Air Systems + Université Grenoble Alpes}{}{France}{
    Research on organizational multi-agent systems, MARL, and cyberdefense architectures.}

\cventry{2021--2022}{Research Engineer}{Thales Land \& Air Systems}{Rennes}{France}{
    Study on multi-agent modeling of cyber environments and anomaly detection.}

\cventry{2020--2021}{Software and Research Engineer}{Atos}{Toulouse}{France}{
    Software engineering for airspace systems with CNES and ESA.}

\cventry{2020}{Software Engineering Internship}{Atos}{Toulouse}{France}{
    Development of the ISIS system for satellite command \& control.}

\cventry{2019}{Software Engineering Internship}{SQLI}{Toulouse}{France}{
    Cybersecurity and development tasks for Airbus Helicopter projects.}

%----------------------------------------------------------------------------------------
% RESEARCH PROJECTS
%----------------------------------------------------------------------------------------

\section{Research Projects}

\cvitem{CybMASDE}{A research platform combining MARL, organizational modeling, and cyberdefence simulation to support the design and analysis of intelligent multi-agent architectures.}

\cvitem{MOISE+MARL}{A proof-of-concept framework integrating organizational roles, missions, and goals directly into MARL environments to guide learning and coordination.}

%----------------------------------------------------------------------------------------
% PUBLICATIONS — FULL LIST
%----------------------------------------------------------------------------------------

\section{Publications}

\subsection{Journal Articles}
\cvitem{}{J.~Soulé, J.-P.~Jamont, M.~Occello, L.-M.~Traonouez, P.~Théron.
    \textit{Assisting Multi-Agent System Design with MOISE+ and MARL: The MAMAD Method}.
    \textbf{Journal of Autonomous Agents and Multi-Agent Systems (JAAMAS)}, under revision, 2025.}

\subsection{International Conferences}

\cvitem{}{J.~Soulé, J.-P.~Jamont, M.~Occello, L.-M.~Traonouez, P.~Théron.
    \textit{Streamlining Resilient Kubernetes Autoscaling with Multi-Agent Systems via an Automated Online Design Framework}.
    Proceedings of the 18th IEEE International Conference on Cloud Computing (CLOUD 2025), Helsinki, 2025.}

\cvitem{}{J.~Soulé, J.-P.~Jamont, M.~Occello, L.-M.~Traonouez, P.~Théron.
    \textit{An Organizationally-Oriented Approach to Enhancing Explainability and Control in Multi-Agent Reinforcement Learning}.
    Proceedings of the 24th International Conference on Autonomous Agents and Multiagent Systems (AAMAS 2025).}

\cvitem{}{J.~Soulé, J.-P.~Jamont, M.~Occello, P.~Théron, L.-M.~Traonouez.
    \textit{A MARL-based Approach for Easing MAS Organization Engineering}.
    Proceedings of the 20th International Conference on Artificial Intelligence Applications and Innovations (AIAI 2024).}

\cvitem{}{J.~Soulé, J.-P.~Jamont, M.~Occello, P.~Théron, L.-M.~Traonouez.
    \textit{Towards a Multi-Agent Simulation of Cyber-Attackers and Cyber-Defenders Battles}.
    Proceedings of IEEE Systems, Man, and Cybernetics (SMC 2023).}

\subsection{National Conferences}

\cvitem{}{J.~Soulé, J.-P.~Jamont, M.~Occello, L.-M.~Traonouez, P.~Théron.
    \textit{Une approche organisationnelle pour améliorer l’explicabilité et le contrôle dans l’apprentissage par renforcement multi-agent}.
    JFSMA~2025. \textbf{Best Paper Award}.}

\cvitem{}{J.~Soulé et al.
    \textit{Une Approche basée sur l’Apprentissage par Renforcement pour l’Ingénierie Organisationnelle d’un SMA}.
    JFSMA~2024.}

\cvitem{}{J.~Soulé et al.
    \textit{Un Outil pour la Conception de SMA par Apprentissage par Renforcement et Modélisation Organisationnelle}.
    JFSMA~2024.}

\cvitem{}{J.~Soulé et al.
    \textit{De l’Organisation des Systèmes Multi-Agents de Cyberdéfense}.
    RJCIA~2023, RESSI~2023.}

\subsection{Posters}

\cvitem{}{J.~Soulé et al.
    \textit{De l’Organisation des Systèmes Multi-Agents de Cyberdéfense}.
    Poster presented at JFSMA~2023.}

\subsection{Invited Talks}

\cvitem{}{Talk for the CybAIR NATO Chair, École de l’air et de l’espace, March 2023.}

%----------------------------------------------------------------------------------------
% TEACHING
%----------------------------------------------------------------------------------------

\section{Teaching Experience}

\cventry{2023--2025}{Teaching Assistant}{Valence (UGA Engineering School \& IUT)}{}{}{
    Tutorials and labs in Operating Systems, Systems Programming, Process Management.\\
    Supervision of student projects in cyberdefense.}

%----------------------------------------------------------------------------------------
% SERVICE AND RESPONSIBILITIES
%----------------------------------------------------------------------------------------

\section{Service}

\cvitem{2023--Present}{Treasurer of the Autonomous Intelligent Cyberdefence Agent (AICA) International Work Group.}

%----------------------------------------------------------------------------------------
% SKILLS
%----------------------------------------------------------------------------------------

\section{Skills}

\cvitem{Languages}{French (native), English (professional, TOEIC 910), Japanese (basic).}

%----------------------------------------------------------------------------------------
% REFERENCES
%----------------------------------------------------------------------------------------

\section{References}

\cvitem{}{Prof.~Jean-Paul Jamont --- Université Grenoble Alpes}
\cvitem{}{Prof.~Michel Occello --- Université Grenoble Alpes}
\cvitem{}{Louis-Marie Traonouez --- Thales Land \& Air Systems}
\cvitem{}{Dr.~Paul Théron --- AICA International Working Group}

\end{document}
