\documentclass[11pt,a4paper,sans]{moderncv}

% ModernCV settings
\moderncvstyle{classic}
\moderncvcolor{blue}

% Packages
\usepackage[utf8]{inputenc}
\usepackage[scale=0.85]{geometry}
\usepackage{setspace}
\setstretch{1.05}

% Personal Information
\name{Julien}{Soulé}
\title{Doctor in Computer Science --- Multi-Agent Systems \& Cyberdefense}
\email{julien.soule@univ-grenoble-alpes.fr}
\phone[mobile]{+33~6~77~63~12~13}
\address{35 Rue Mathieu-de-la-Drôme}{26000 Valence}{France}
\extrainfo{French nationality}

\begin{document}
\makecvtitle

%----------------------------------------------------------------------------------------
% RESEARCH PROFILE
%----------------------------------------------------------------------------------------

\section{Research Profile}

I am a doctor in Computer Science specializing in Multi-Agent Systems and Cyberdefense.
My research investigates how organizational models and multi-agent reinforcement learning
can be combined to design intelligent, explainable, and resilient cyberdefence agents.
I develop methods that exploit organizational constraints to guide learning, improve system-level
coordination, and enhance controllability and interpretability in complex adversarial environments.

%----------------------------------------------------------------------------------------
% RESEARCH INTERESTS
%----------------------------------------------------------------------------------------

\section{Research Interests}
Multi-Agent Reinforcement Learning (MARL); Multi-Agent System Organizations;\\
Cyberdefense \& Intelligent Cyber Agents; Simulation \& Digital Twins; Software Architecture for MAS.

%----------------------------------------------------------------------------------------
% EDUCATION
%----------------------------------------------------------------------------------------

\section{Education}

\cventry{2022--2025}{PhD in Computer Science}{Université Grenoble Alpes (UGA)}{France}{}{
    Dissertation defended on \textbf{17 November 2025}.
    Title: \textit{``On the Organization of a Cyberdefence Multi-Agent System''}.
    Status: \textbf{Degree conferred}.
    Supervisors: Jean-Paul Jamont, Michel Occello, Louis-Marie Traonouez, Paul Théron.
}

\cventry{2018--2021}{Master's Degree in Computer Engineering}{INSA Rennes}{France}{}{
    Specialization in Computer Science and Cybersecurity.}

\cventry{2019}{Exchange Semester in Computer Engineering}{École de Technologie Supérieure (ETS)}{Montréal, Canada}{}{}

\cventry{2015--2018}{Integrated Preparatory Cycle}{INSA Rennes}{France}{}{}

\cventry{2012--2015}{High School}{Lycée Charles Renouvier}{Prades, France}{}{}

%----------------------------------------------------------------------------------------
% APPOINTMENTS / EXPERIENCE
%----------------------------------------------------------------------------------------

\section{Appointments}

\cventry{2022--2025}{Doctoral Researcher}{Thales Land \& Air Systems + Université Grenoble Alpes}{}{France}{
    Conducted doctoral research on organizational multi-agent systems, MARL, and cyberdefense architectures.
    PhD defended on 17 November 2025.
}

\cventry{2021--2022}{Research Engineer}{Thales Land \& Air Systems}{Rennes}{France}{
    Study on multi-agent modeling of cyber environments and anomaly detection in Thales LAS context.}

\cventry{2020--2021}{Software and Research Engineer}{Atos}{Toulouse}{France}{
    Software engineering for airspace systems developed with CNES and ESA.}

\cventry{2020}{Software Engineering Internship}{Atos}{Toulouse}{France}{
    Development of the ISIS system for satellite command \& control (Python, Bash, KVM, Grafana, Django).}

\cventry{2019}{Software Engineering Internship}{SQLI}{Toulouse}{France}{
    Cybersecurity, maintenance, and software engineering tasks for Airbus Helicopters projects.
}

%----------------------------------------------------------------------------------------
% RESEARCH PROJECTS
%----------------------------------------------------------------------------------------

\section{Research Projects}

\cvitem{CybMASDE}{A research platform combining MARL, organizational modeling, and cyberdefense simulation to support the design and analysis of intelligent multi-agent architectures.}

\cvitem{MOISE+MARL}{A proof-of-concept framework integrating organizational roles, missions, and goals into MARL environments to guide learning and improve agent coordination.}

%----------------------------------------------------------------------------------------
% PUBLICATIONS — FULL LIST
%----------------------------------------------------------------------------------------

\section{Publications}

\subsection{Journal Articles}
\cvitem{}{J.~Soulé, J.-P.~Jamont, M.~Occello, L.-M.~Traonouez, P.~Théron.
    \textit{Assisting Multi-Agent System Design with MOISE+ and MARL: The MAMAD Method}.
    \textbf{Journal of Autonomous Agents and Multi-Agent Systems (JAAMAS)}, under revision, 2025.}

\subsection{International Conferences}

\cvitem{}{J.~Soulé, J.-P.~Jamont, M.~Occello, L.-M.~Traonouez, P.~Théron.
    \textit{Streamlining Resilient Kubernetes Autoscaling with Multi-Agent Systems via an Automated Online Design Framework}.
    IEEE CLOUD 2025, Helsinki, Finland.}

\cvitem{}{J.~Soulé, J.-P.~Jamont, M.~Occello, L.-M.~Traonouez, P.~Théron.
    \textit{An Organizationally-Oriented Approach to Enhancing Explainability and Control in Multi-Agent Reinforcement Learning}.
    AAMAS 2025.}

\cvitem{}{J.~Soulé, J.-P.~Jamont, M.~Occello, P.~Théron, L.-M.~Traonouez.
    \textit{A MARL-based Approach for Easing MAS Organization Engineering}.
    AIAI 2024.}

\cvitem{}{J.~Soulé, J.-P.~Jamont, M.~Occello, P.~Théron, L.-M.~Traonouez.
    \textit{Towards a Multi-Agent Simulation of Cyber-Attackers and Cyber-Defenders Battles}.
    IEEE SMC 2023.}

\subsection{National Conferences}

\cvitem{}{J.~Soulé, J.-P.~Jamont, M.~Occello, L.-M.~Traonouez, P.~Théron.
    \textit{Une approche organisationnelle pour améliorer l’explicabilité et le contrôle dans l’apprentissage par renforcement multi-agent}.
    JFSMA 2025. \textbf{Best Paper Award}.}

\cvitem{}{J.~Soulé et al.
    \textit{Une Approche basée sur l’Apprentissage par Renforcement pour l’Ingénierie Organisationnelle d’un SMA}.
    JFSMA 2024.}

\cvitem{}{J.~Soulé et al.
    \textit{Un Outil pour la Conception de SMA par Apprentissage par Renforcement et Modélisation Organisationnelle}.
    JFSMA 2024.}

\cvitem{}{J.~Soulé et al.
    \textit{De l’Organisation des Systèmes Multi-Agents de Cyberdéfense}.
    RJCIA 2023, RESSI 2023.}

\subsection{Posters}

\cvitem{}{J.~Soulé et al.
    \textit{De l’Organisation des Systèmes Multi-Agents de Cyberdéfense}.
    Poster presented at JFSMA 2023.}

\subsection{Invited Talks}

\cvitem{}{Talk for the CybAIR NATO Chair, École de l’air et de l’espace, March 2023.}

%----------------------------------------------------------------------------------------
% TEACHING EXPERIENCE
%----------------------------------------------------------------------------------------

\section{Teaching Experience}

\cventry{2023--2025}{Teaching Assistant}{Valence (UGA Engineering School \& IUT)}{}{}{
    Tutorials and labs in Operating Systems, Systems Programming, and Process Management.
    Supervision of student groups working on industry-driven cyberdefense projects.}

%----------------------------------------------------------------------------------------
% SERVICE
%----------------------------------------------------------------------------------------

\section{Service}

\cvitem{2023--Present}{Treasurer of the Autonomous Intelligent Cyberdefence Agent (AICA) International Work Group.}

%----------------------------------------------------------------------------------------
% REVIEWING ACTIVITIES
%----------------------------------------------------------------------------------------

\section{Committees and Reviewing Activities}

\cvitem{2025}{Reviewer for the 14th International Conference on Agents and Artificial Intelligence (ICAART 2022).}

%----------------------------------------------------------------------------------------
% AWARDS AND DISTINCTIONS
%----------------------------------------------------------------------------------------

\section{Awards and Distinctions}

\cvitem{2025}{Best Paper Award at JFSMA 2025 for \textit{``Une approche organisationnelle pour améliorer l’explicabilité et le contrôle dans l’apprentissage par renforcement multi-agent''}.}

%----------------------------------------------------------------------------------------
% SKILLS
%----------------------------------------------------------------------------------------

\section{Skills}

\cvitem{Languages}{French (native), English (professional, TOEIC 910), Japanese (basic).}

%----------------------------------------------------------------------------------------
% REFERENCES
%----------------------------------------------------------------------------------------

\section{References}

\cvitem{}{Prof.~Jean-Paul Jamont --- Université Grenoble Alpes}
\cvitem{}{Prof.~Michel Occello --- Université Grenoble Alpes}
\cvitem{}{Louis-Marie Traonouez --- Thales Land \& Air Systems}
\cvitem{}{Dr.~Paul Théron --- AICA International Working Group}

\end{document}
