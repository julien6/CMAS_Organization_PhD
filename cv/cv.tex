% !TEX TS-program = xelatex
% Compile with: xelatex or lualatex

\documentclass[11pt,a4paper,sans]{moderncv}

% ModernCV settings
\moderncvstyle{classic}
\moderncvcolor{blue}

% Packages
\usepackage[utf8]{inputenc}
\usepackage[TU]{fontenc}
\usepackage[scale=0.85]{geometry}
\usepackage{setspace}
\setstretch{1.05}
\usepackage{parskip}

% \usepackage[hidelinks]{hyperref}
\usepackage{biblatex}
\moderncvicons{awesome}
\usepackage{academicons}

\newcommand{\dblp}[1]{\collectionadd[twitter]{socials}{\protect\httplink[#1]{dblp.uni-trier.de/search?q=#1}}}

\renewcommand*{\twittersocialsymbol}{%
    \raisebox{-0.08em}{%
    \scalebox{0.6}{%
        \tikz{\draw node[rectangle, rounded corners, draw=color2, inner sep=1pt] (0,0) {\textsf{dblp}};}~% 
    }}}

% Personal Information
\name{Julien}{Soulé}
\title{Doctor in Computer Science \\ AI \& Multi-Agent Systems \\ \ \\}
\address{35 Rue Mathieu-de-la-Drôme}{26000 Valence, France}{}

\email{julien.soule@hotmail.fr}
\phone[mobile]{+33~(0)~6~77~63~12~13}
\homepage{julien6.github.io/home/}

\extrainfo{
  \faLinkedin\ 
  \href{https://www.linkedin.com/in/julien-soul\%C3\%A9-6b2b27173}{julien-soulé-6b2b27173}\\
  \noindent
  \twittersocialsymbol~\href{https://dblp.org/pid/367/9947.html}{homepages/367/9947}\\
  \noindent
  \aiOrcid~\href{https://orcid.org/0009-0002-3218-2614}{0009-0002-3218-2614}\\
  \noindent
  \faGithub~\href{https://github.com/julien6}{julien6}
}

\photo[110pt][0pt]{photo.jpeg}

% === Load your .bib file ===
\addbibresource{references.bib}

% === Sorting by year (descending), then author ===
\ExecuteBibliographyOptions{sorting=ydnt}

\setlength{\hintscolumnwidth}{2.5cm}

\begin{document}
\makecvtitle

\vspace{-1cm}
\begin{center}
    {\tiny
        \noindent \textcolor{gray}{[Updated on \today]}}
\end{center}


%----------------------------------------------------------------------------------------
% RESEARCH PROFILE
%----------------------------------------------------------------------------------------

\section{Research Profile}

I am a doctor in Computer Science specializing in Multi-Agent Systems and Cyberdefense.
My research investigates how organizational models and multi-agent reinforcement learning
can be combined to design intelligent, explainable, and resilient cyberdefence agents.
I develop methods that exploit organizational constraints to guide learning, improve system-level
coordination, and enhance controllability and interpretability in complex adversarial environments.

%----------------------------------------------------------------------------------------
% RESEARCH INTERESTS
%----------------------------------------------------------------------------------------

\section{Research Interests}
Multi-Agent Reinforcement Learning (MARL); Multi-Agent System Organizations;\\
Cyberdefense \& Intelligent Cyber Agents; Simulation \& Digital Twins; Software Architecture for MAS.

%----------------------------------------------------------------------------------------
% EDUCATION
%----------------------------------------------------------------------------------------

\section{Education}

\cventry{2022--2025}{PhD in Computer Science}{Université Grenoble Alpes (UGA)}{France}{}{
    Dissertation defended on \textbf{17 November 2025}.
    Title: \textit{``On the Organization of a Cyberdefence Multi-Agent System''}.
    Status: \textbf{Degree conferred}.
    Supervisors: Jean-Paul Jamont, Michel Occello, Louis-Marie Traonouez, Paul Théron.
}

\cventry{2018--2021}{Master's Degree in Computer Engineering}{INSA Rennes}{France}{}{
    Specialization in Computer Science and Cybersecurity.}

\cventry{2019}{Exchange Semester in Computer Engineering}{École de Technologie Supérieure (ETS)}{Montréal, Canada}{}{}

\cventry{2015--2018}{Integrated Preparatory Cycle}{INSA Rennes}{France}{}{}

\cventry{2012--2015}{High School}{Lycée Charles Renouvier}{Prades, France}{}{}

%----------------------------------------------------------------------------------------
% APPOINTMENTS / EXPERIENCE
%----------------------------------------------------------------------------------------

\section{Appointments}

\cventry{2022--2025}{Doctoral Researcher}{Thales Land \& Air Systems + Université Grenoble Alpes}{}{France}{
    Conducted doctoral research on organizational multi-agent systems, MARL, and cyberdefense architectures.
    PhD defended on 17 November 2025.
}

\cventry{2021--2022}{Research Engineer}{Thales Land \& Air Systems}{Rennes}{France}{
    Study on multi-agent modeling of cyber environments and anomaly detection in Thales LAS context.}

\cventry{2020--2021}{Software and Research Engineer}{Atos}{Toulouse}{France}{
    Software engineering for airspace systems developed with CNES and ESA.}

\cventry{2020}{Software Engineering Internship}{Atos}{Toulouse}{France}{
    Development of the ISIS system for satellite command \& control (Python, Bash, KVM, Grafana, Django).}

\cventry{2019}{Software Engineering Internship}{SQLI}{Toulouse}{France}{
    Cybersecurity, maintenance, and software engineering tasks for Airbus Helicopters projects.
}

\clearpage

%----------------------------------------------------------------------------------------
% RESEARCH PROJECTS
%----------------------------------------------------------------------------------------

\section{Research Projects}


\cvitem{\textcolor{blue}{\href{https://github.com/julien6/CybMASDE}{CybMASDE}}}{\qquad A research platform combining MARL, organizational modeling, and cyberdefense simulation to support the design and analysis of intelligent multi-agent architectures.}

\bigskip

\cvitem{\textcolor{blue}{\href{https://github.com/julien6/MOISE-MARL}{MOISE+MARL}}}{\qquad A proof-of-concept framework integrating organizational roles, missions, and goals into MARL environments to guide learning and improve agent coordination.}

%----------------------------------------------------------------------------------------
% PUBLICATIONS — FULL LIST
%----------------------------------------------------------------------------------------

\section{Publications}

\smallskip

\subsection*{Journal Articles}
\nocite{soulej2025jaamas}
\nocite{soule2025roia}
\printbibliography[keyword=journal, heading=none, sorting=ydnt]

\smallskip

\subsection*{International Conferences}
\nocite{soulej2025cloud}
\nocite{soule2024moise_marl}
\nocite{soule2024marl}
\nocite{soulej2023sim}
\printbibliography[keyword=international, heading=none, sorting=ydnt]

\smallskip

\subsection*{National Conferences (France)}
\nocite{soule2025jfsma}
\nocite{soule2024outil}
\nocite{soule2024approche}
\nocite{soule2023jfsmathese}
\nocite{soule2023ressithese}
\nocite{soule2023rjciathese}
\printbibliography[keyword=national, heading=none, sorting=ydnt]

\smallskip

\subsection*{Invited Talks}
\nocite{soule2023cybairtalk}
\printbibliography[keyword=talk, heading=none, sorting=ydnt]

%----------------------------------------------------------------------------------------
% TEACHING EXPERIENCE
%----------------------------------------------------------------------------------------

\section{Teaching Experience}

\cventry{2023--2025}{Teaching Assistant}{Valence (UGA Engineering School \& IUT)}{}{}{
    Tutorials and labs in Operating Systems, Systems Programming, and Process Management.
    Supervision of student groups working on industry-driven cyberdefense projects.}

%----------------------------------------------------------------------------------------
% SERVICE
%----------------------------------------------------------------------------------------

\section{Service}

\cvitem{2023--Present}{Treasurer of the Autonomous Intelligent Cyberdefence Agent (AICA) International Work Group.}

%----------------------------------------------------------------------------------------
% REVIEWING ACTIVITIES
%----------------------------------------------------------------------------------------

\section{Reviewing Activities}

\cvitem{2025}{Reviewer for the 17th International Conference on Agents and Artificial Intelligence (\textbf{ICAART} 2025).}

\cvitem{2025}{Reviewer for the 25th International Conference on Autonomous Agents and Multiagent Systems (\textbf{AAMAS} 2026).}

\cvitem{2025}{2025 IEEE International Conference on Development and Learning (\textbf{ICDL} 2025).}

\cvitem{2024}{Reviewer for the 16th International Conference on Agents and Artificial Intelligence (\textbf{ICAART} 2024).}

\cvitem{2024}{Reviewer for the 2025 IEEE International Conference on Systems, Man, and Cybernetics (\textbf{SMC} 2025).}


%----------------------------------------------------------------------------------------
% AWARDS AND DISTINCTIONS
%----------------------------------------------------------------------------------------

\section{Awards and Distinctions}

\cvitem{2025}{Best Paper Award at JFSMA 2025 for \textit{``Une approche organisationnelle pour améliorer l’explicabilité et le contrôle dans l’apprentissage par renforcement multi-agent''}.}

%----------------------------------------------------------------------------------------
% SKILLS
%----------------------------------------------------------------------------------------

\section{Skills}

\cvitem{Languages}{French (native), English (professional, TOEIC 910), Spanish (intermediate),, Japanese (basic).}


\end{document}
