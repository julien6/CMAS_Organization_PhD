% vim: set textwidth=120:

% Example CV based on the 1.5-column-cv template. Main features:
% * uses the Roboto font family and IcoMoon icon set;
% * doesn't use colours, different font weights are used instead for styling;
% * because the CV fits on one page, header and footer is empty, since there isn't much useful info to put there;
% * includes a photo.
\documentclass[a4paper,10pt]{article}


% package imports
% ---------------

\usepackage[romanian, french, british]{babel} % for correct language and hyphenation and stuff
\usepackage{calc}           % for easier length calculations (infix notation)
\usepackage{enumitem}       % for configuring list environments
\usepackage{fancyhdr}       % for setting header and footer
\usepackage{fontspec}       % for fonts
\usepackage{geometry}       % for setting margins (\newgeometry)
\usepackage{graphicx}       % for pictures
\usepackage{microtype}      % for microtypography stuff
\usepackage{xcolor}         % for colours


% margin and column widths
% ------------------------

% margins
\newgeometry{left=0mm,right=15mm,top=15mm,bottom=15mm}
% \newgeometry{left=15mm,right=15mm,top=15mm,bottom=15mm} % original

% width of the gap between left and right column
\newlength{\cvcolumngapwidth}
\setlength{\cvcolumngapwidth}{3.5mm}

% left column width
\newlength{\cvleftcolumnwidth}
\setlength{\cvleftcolumnwidth}{36mm}

% right column width
\newlength{\cvrightcolumnwidth}
\setlength{\cvrightcolumnwidth}{\textwidth-\cvleftcolumnwidth-\cvcolumngapwidth}

% set paragraph indentation to 0, because it screws up the whole layout otherwise
\setlength{\parindent}{0mm}


% style definitions
% -----------------
% style categories explanation:
% * \cvnameXXX is used for the name; 

% * \cvsectionXXX is used for section names (left column, accompanied by a horizontal rule);
% * \cvtitleXXX is used for job/education titles (right column);
% * \cvdurationXXX is used for job/education durations (left column);
% * \cvheadingXXX is used for headings (left column);
% * \cvmainXXX (and \setmainfont) is used for main text;
% * \cvruleXXX is used for the horizontal rules denoting sections.

% font families
\defaultfontfeatures{Ligatures=TeX} % reportedly a good idea, see https://tex.stackexchange.com/a/37251

\newfontfamily{\cvnamefont}{Liberation Serif}
\newfontfamily{\cvsectionfont}{Liberation Serif}
\newfontfamily{\cvtitlefont}{Liberation Serif}
\newfontfamily{\cvdurationfont}{Liberation Serif}
\newfontfamily{\cvheadingfont}{Liberation Serif}
\setmainfont{Liberation Serif}

% colours
\definecolor{cvnamecolor}{RGB}{0,0,255}
\definecolor{cvsectioncolor}{RGB}{0,0,255}
\definecolor{cvtitlecolor}{HTML}{000000}
\definecolor{cvdurationcolor}{HTML}{000000}
\definecolor{cvheadingcolor}{HTML}{000000}
\definecolor{cvmaincolor}{HTML}{000000}
\definecolor{cvrulecolor}{HTML}{000000}

\color{cvmaincolor}

% styles
\newcommand{\cvnamestyle}[1]{{\huge\cvnamefont\textcolor{cvnamecolor}{#1}}}
\newcommand{\cvsectionstyle}[1]{{\large\cvsectionfont\textcolor{cvsectioncolor}{#1}}}
\newcommand{\cvtitlestyle}[1]{{\large\cvtitlefont\textcolor{cvtitlecolor}{#1}}}
\newcommand{\cvdurationstyle}[1]{{\cvdurationfont\textcolor{cvdurationcolor}{#1}}}
\newcommand{\cvheadingstyle}[1]{{\cvheadingfont\textcolor{cvheadingcolor}{#1}}}


% inter-item spacing
% ------------------

% vertical space after personal info and standard CV items
\newlength{\cvafteritemskipamount}
\setlength{\cvafteritemskipamount}{5mm plus 1.25mm minus 1.25mm}

% vertical space after sections
\newlength{\cvaftersectionskipamount}
\setlength{\cvaftersectionskipamount}{2mm plus 0.5mm minus 0.5mm}

% extra vertical space to be used when a section starts with an item with a heading (e.g. in the skills section),
% so that the heading does not follow the section name too closely
\newlength{\cvbetweensectionandheadingextraskipamount}
\setlength{\cvbetweensectionandheadingextraskipamount}{1mm plus 0.25mm minus 0.25mm}


% intra-item spacing
% ------------------

% vertical space after name
\newlength{\cvafternameskipamount}
\setlength{\cvafternameskipamount}{3mm plus 0.75mm minus 0.75mm}

% vertical space after personal info lines
\newlength{\cvafterpersonalinfolineskipamount}
\setlength{\cvafterpersonalinfolineskipamount}{2mm plus 0.5mm minus 0.5mm}

% vertical space after titles
\newlength{\cvaftertitleskipamount}
\setlength{\cvaftertitleskipamount}{1mm plus 0.25mm minus 0.25mm}

% value to be used as parskip in right column of CV items and itemsep in lists (same for both, for consistency)
\newlength{\cvparskip}
\setlength{\cvparskip}{0.5mm plus 0.125mm minus 0.125mm}

% set global list configuration (use parskip as itemsep, and no separation otherwise)
\setlist{parsep=0mm,topsep=0mm,partopsep=0mm,itemsep=\cvparskip}


% CV commands
% -----------

% creates a "personal info" CV item with the given left and right column contents, with appropriate vertical space after
% @param #1 left column content (should be the CV photo)
% @param #2 right column content (should be the name and personal info)
\newcommand{\cvpersonalinfo}[2]{
    % left and right column
    \begin{minipage}[t]{\cvleftcolumnwidth}
        \vspace{0mm} % XXX hack to align to top, see https://tex.stackexchange.com/a/11632
        \raggedleft #1
    \end{minipage}% XXX necessary comment to avoid unwanted space
    \hspace{\cvcolumngapwidth}% XXX necessary comment to avoid unwanted space
    \begin{minipage}[t]{\cvrightcolumnwidth}
        \vspace{0mm} % XXX hack to align to top, see https://tex.stackexchange.com/a/11632
        #2
    \end{minipage}

    % space after
    \vspace{\cvafteritemskipamount}
}

% typesets a name, with appropriate vertical space after
% @param #1 name text
\newcommand{\cvname}[1]{
    % name
    \cvnamestyle{#1}

    % space after
    \vspace{\cvafternameskipamount}
}

% typesets a line of personal info beginning with an icon, with appropriate vertical space after
% @param #1 parameters for the \includegraphics command used to include the icon
% @param #2 icon filename
% @param #3 line text
\newcommand{\cvpersonalinfolinewithicon}[3]{
    % icon, vertically aligned with text (see https://tex.stackexchange.com/a/129463)
    \raisebox{.5\fontcharht\font`E-.5\height}{\includegraphics[#1]{#2}}
    % text
    #3

    % space after
    \vspace{\cvafterpersonalinfolineskipamount}
}

% creates a "section" CV item with the given left column content, a horizontal rule in the right column, and with
% appropriate vertical space after
% @param #1 left column content (should be the section name)
\newcommand{\cvsection}[1]{
    % left and right column
    \begin{minipage}[t]{\cvleftcolumnwidth}
        \raggedleft\cvsectionstyle{#1}
    \end{minipage}% XXX necessary comment to avoid unwanted space
    \hspace{\cvcolumngapwidth}% XXX necessary comment to avoid unwanted space
    \begin{minipage}[t]{\cvrightcolumnwidth}
        \textcolor{cvrulecolor}{\rule{\cvrightcolumnwidth}{0.3mm}}
    \end{minipage}

    % space after
    \vspace{\cvaftersectionskipamount}
}

% creates a standard, multi-purpose CV item with the given left and right column contents, parskip set to cvparskip
% in the right column, and with appropriate vertical space after
% @param #1 left column content
% @param #2 right column content
\newcommand{\cvitem}[2]{
    % left and right column
    \begin{minipage}[t]{\cvleftcolumnwidth}
        \raggedleft #1
    \end{minipage}% XXX necessary comment to avoid unwanted space
    \hspace{\cvcolumngapwidth}% XXX necessary comment to avoid unwanted space
    \begin{minipage}[t]{\cvrightcolumnwidth}
        \setlength{\parskip}{\cvparskip} #2
    \end{minipage}

    % space after
    \vspace{\cvafteritemskipamount}
}

% typesets a title, with appropriate vertical space after
% @param #1 title text
\newcommand{\cvtitle}[1]{
    % title
    \cvtitlestyle{#1}

    % space after
    \vspace{\cvaftertitleskipamount}
    % XXX need to subtract cvparskip here, because it is automatically inserted after the title "paragraph"
    \vspace{-\cvparskip}
}


% header and footer
% -----------------

% set empty header and footer
\pagestyle{empty}


\hyphenation{Romania Uncertainty Engineering surface Informatics}

% preamble end/document start
% ===========================

\begin{document}


% personal info
% -------------

\cvpersonalinfo{
    % photo
    \includegraphics[height=36mm]{photo_Vincent.jpg}
}{
    % name
    \cvname{\textbf{Vincent MARGUET}}

    % address
    \cvpersonalinfolinewithicon{height=4mm}{072-location.pdf}{
        10, Rue Edouard Iung, 26000 Valence, FRANCE
    }

    % phone number
    \cvpersonalinfolinewithicon{height=4mm}{067-phone.pdf}{
        +33 (0) 6 44 24 74 30
    }

    % email address
    \cvpersonalinfolinewithicon{height=4mm}{070-envelop.pdf}{
        vincent.marguet@lcis.grenoble-inp.fr
    }

    % LinkedIn account
    \cvpersonalinfolinewithicon{height=4mm}{458-linkedin.pdf}{
        www.linkedin.com/in/vincent-marguet
    }

    % date of birth
    Born on 14th September 1998, Arnas, FRANCE
}



% education
% ---------

\cvsection{EDUCATION}

% master's
\cvitem{
    \cvdurationstyle{\textbf{Oct. 2021 -- today}}
}{
    \cvtitle{\textbf{PhD "Coordination of drones for reliable data collection from a ground-based sensor network"} at \textbf{LCIS},
        Univ. Grenoble Alpes, Valence, France.}


    \begin{itemize}[leftmargin=*]
        \item \textbf{Supervisers:}  \textbf{Ionela Prodan}, LCIS (Laboratoire de Conception et d'Intégration des Systèmes), team CO4SYS (Coordination, Coopération, Contrôle des SYStèmes complexes)\\
              \textbf{Francesca Boem}, UCL (University College London)
              \iffalse\item \textbf{Topics of interest:} Scientific Computing, Numerical Programming, Parallel Programming, Uncertainty Quantification, Computational Fluid Dynamics.\fi
        \item \textbf{Fundings:} LabEx PERSYVAL-Lab,
        \item \textbf{Teaching responsibilities:} Supervised teaching in Automation, 2022/2023, given to approx. 16 students.
    \end{itemize}
}

% master's
\cvitem{
    \cvdurationstyle{\textbf{Sept. 2018 -- Jul. 2021}}
}{
    \cvtitle{\textbf{Engineer diploma} at \textbf{ESISAR (\'Ecole Nationale Supérieure en Systèmes Avancés et Réseaux)}, Grenoble INP (Institut National Polytechnique de Grenoble), Univ. Grenoble Alpes, Valence, France.}


    \textbf{Major in ISC (Engineering of Complex Systems):} specialised in Automation (Decentralised control of complex systems, Diagnostics and robust control, Modelling and control of non-linear systems, Control optimisation, Complex systems : dynamic processes defined on large-scale networks.)

}

% bachelor's
\cvitem{
    \cvdurationstyle{\textbf{Feb. 2020 -- Jul. 2020}}
}{
    \cvtitle{\textbf{Exchange semester in  Brazil during 4th year at ESISAR} at \textbf{Escola Politecnica, Universidade de São Paulo (USP)}, São Paulo, Brazil}


    Courses in Portuguese at the best University of Latin America: Introduction to Robust Control Systems Design, Multi-variable Control, Probabilistic Models, Project Management.

}

% master's
\cvitem{
    \cvdurationstyle{\textbf{Sep. 2016 -- Jul. 2018}}
}{
    \cvtitle{\textbf{CPGE (Intensive general program for top-ranking higher education establishments in sciences)} from the \textbf{Lycée du Parc}, Lyon, France}

    Physical, Chemistry and Engineering sciences (PCSI, PC*). Dedicated to study hard for preparing the admission exams for an Engineering School. Interest in Mathematics, Physics and Chemistry. Succeeded to enter in PC* (Physics Chemistry star) class with the best students of this famous establishment.
}

\cvitem{
    \cvdurationstyle{\textbf{Sep. 2009 -- Jul. 2016}}
}{
    \cvtitle{\textbf{High School Diploma} from \textbf{Notre-Dame de Mongré}, Villefranche-sur-Saône, France}

    Physical, Chemistry and Engineering sciences (PCSI, PC*). Dedicated to study hard for preparing the admission exams for an Engineering School. Interest in Mathematics, Physics and Chemistry. Succeeded to enter in PC* (Physics Chemistry star) class with the best students of this famous establishment.
}

\vspace{0.2cm}

% education
% ---------

\cvsection{Certifications}
\cvitem{
    \cvdurationstyle{\textbf{2016}}
}{  %Certification for the teaching profession
    \textbf{Teaching Certification}
    \begin{itemize}[leftmargin=*]
        \item Graduated the Psycho-Pedagogical Training Module at Univ. ``Politehnica'' of Bucharest, Romania.
    \end{itemize}
}
\vspace{-0.3cm}
\cvitem{
    \cvdurationstyle{\textbf{2011}}
}{
    % master's
    \textbf{CISCO IT Essentials - PC Hardware and Software}
    \begin{itemize}[leftmargin=*]
        \item Skills in installing, maintaining and troubleshooting the hardware and software components.
    \end{itemize}
}
\vspace{-0.3cm}
\cvitem{
    \cvdurationstyle{\textbf{2008}}
}{
    \textbf{European Computer Driving Licence (ECDL)}
    \begin{itemize}[leftmargin=*]
        \item Very good command of Office suite.
    \end{itemize}
}

\vspace{2cm}

% work experience
% ---------------

\cvsection{WORK EXPERIENCE}


\cvitem{
    \cvdurationstyle{\textbf{May 2019 -- present}}
}{
    \cvtitle{\textbf{Intern student} within a team project at \textbf{TU Munich} and \textbf{Audi AG}}
    %{\color{red}{Nu stiu exact ce sa scriu aici: suntem 4 studenti in echipa, coordonati de cineva de la Audi AG }}

    \begin{itemize}[leftmargin=*]
        \item \textbf{Topic}: A High Performance Vehicle Dynamics Model for Autonomous Driving.
        \item \textbf{Purpose:} Developing an open source tool to be used internally by Audi AG. \\
              Team of four students coordinated by Dr. Stefan Sicklinger, technical leader at  Audi AG.
        \item \textbf{Brief description:} Creating a high-performance C++ implementation for simulating the two-track model of a car by using appropriate numerical schemes and optimized mathematical-related libraries (Intel MKL, Eigen). This model is used for simulating various driving scenarios and performing uncertainty quantification.

    \end{itemize}
}

\cvitem{
    \cvdurationstyle{\textbf{Apr. 2019 -- present}}
}{
    \cvtitle{\textbf{Scientific Research Assistant} at \textbf{TU Munich}, Munich, Germany.}



    \begin{itemize}[leftmargin=*]
        \item Participating within the work of the PhD student, Ivana Jovanovic, in applying Uncertainty Quantification on Flood Simulation in Bavaria.
        \item Implementing Python scripts for parsing hydrological related data and preparing the background functions necessary for performing uncertainty quantification simulations on them.
    \end{itemize}
}

\cvitem{
    \cvdurationstyle{\textbf{Oct. 2017 -- Jun. 2018}}
}{
    \cvtitle{\textbf{Research Assistant} at \textbf{Univ. ``Politehnica'' of Bucharest}, Bucharest, Romania.}


    \begin{itemize}[leftmargin=*]
        \item Team member in the \textbf{ESA-funded project PROBA-3} Non-Cooperative RV Experiment Phases C/D/E1,
              \begin{itemize}[leftmargin=*]
                  \item with attributions in Matlab simulations and technical reports for the trajectory of a satellite formation;
                  \item in collaboration with the Romanian Institute of Space Science.
              \end{itemize}
        \item Team member in the \textbf{ESA-funded project ACTFL} (Advanced Control Techniques for Future Launchers),
              \begin{itemize}[leftmargin=*]
                  \item with attributions in developing control algorithms (LQR/LTR and H-infinity, PID) for a satellite launcher;
                  \item in collaboration with the Aerospace Faculty of UPB (Univ. ``Politehnica'' of Bucharest).
              \end{itemize}
    \end{itemize}
}

\cvitem{
    \cvdurationstyle{\textbf{Oct. 2015 -- Jun. 2018}}
}{
    \cvtitle{\textbf{Teaching Assistant} at \textbf{Faculty of Automatic Control and Computer Science}, Univ. ``Politehnica'' of Bucharest, Romania.}


    \begin{itemize}[leftmargin=*]
        \item Providing assistance in laboratories and tutorial classes for approx. 100 students in the following subjects:
              \begin{itemize}[leftmargin=*]
                  \item Signals and Systems (2nd year undergraduate class).
                  \item Control System Theory  (2nd year undergraduate class).
                  \item Numerical Techniques in Automatics and Informatics  (2nd year undergraduate class).
              \end{itemize}
    \end{itemize}
}

\cvitem{
    \cvdurationstyle{\textbf{Jul. 2016 -- Sep. 2016}}
}{
    \cvtitle{\textbf{Internship - Software Engineer} at \textbf{DEIMOS Space Ltd.}, Bucharest, Romania.}
    \begin{itemize}[leftmargin=*]
        \item Numerical algorithm analysis, namely:
              \begin{itemize}[leftmargin=*]
                  \item Studying scientific references on radiative transfer models (RTM).
                  \item Implementing numerical algorithms specified in the IDS design (RTM module).
              \end{itemize}
        \item C++ coding and software engineering, namely:
              \begin{itemize}[leftmargin=*]
                  \item Unit testing
                  \item Studying and implementing coding standards (Google C++ coding style, cpplint).
                  \item Studying and using version control (git).
              \end{itemize}
    \end{itemize}
}

% skills
% ------

\cvsection{Technical skills}

\vspace{\cvbetweensectionandheadingextraskipamount}

\cvitem{
    \cvheadingstyle{\textbf{Automation}}
    % \cvheadingstyle{Programming Languages} % original
}{

    (Non)linear dynamical systems, Modeling and identification, System simulation, Nonlinear control (Adaptive control, Lyapunov control), Signal processing, Optimization techniques, Numerical methods, Distributed systems, Optimal control.
    %... put some courses/topics which you like/master the most, Robotics should appear also. {\color{red}{In facultate am avut un singur curs la care am vazut roboti. Cum nici nu ma pasioneaza in mod explicit, nu stiu ce as putea scrie de \textbf{Robotics}}}
}

\cvitem{
    \cvheadingstyle{\textbf{Informatics}}
    % \cvheadingstyle{Programming Languages} % original
}{

    \begin{itemize}[leftmargin=*]
        \item Programming languages: C, C++, Python, Matlab.
        \item Operating systems: Windows, Linux.
        \item Databases: SQL.
        \item Text editing and processing: \LaTeX, Office.
    \end{itemize}
}

% \cvitem{
%     \cvheadingstyle{Databases}
% }{
%     \begin{itemize}[leftmargin=*]
%         \item SQL
%     \end{itemize}
% }

% \cvitem{
%     \cvheadingstyle{Text editing}
% }{
%     \begin{itemize}[leftmargin=*]
%         \item LaTeX
%     \end{itemize}
% }

\cvsection{Personal and linguistic skills}

\vspace{\cvbetweensectionandheadingextraskipamount}

\cvitem{
    \cvheadingstyle{\textbf{Strengths}}
}{Analytical thinking, Perfectionist, Strong motivational and leadership skills, Ability to work under pressure, Ability to work individually as well as in a team, Positive attitude.
}


% languages
\cvitem{
    \cvheadingstyle{\textbf{Languages}}
    % \cvheadingstyle{Languages Known} % original
}{ English: C1 (professional level), French: A2 (intermediate level), German: B1 (upper-intermediate level), Romanian (native).
}

% additional info
% ---------------

\cvsection{Honors/Awards}

\vspace{\cvbetweensectionandheadingextraskipamount}
\cvitem{
    \cvheadingstyle{\textbf{Scholarships}}
}{
    \begin{itemize}[leftmargin=*]
        \item \textbf{DAAD (German Academic Exchange Service) Study Scholarship - Master Studies for All Academic Disciplines}, awarded to 9 out of 53 Romanian applicants studying in Germany, Oct 2019 - Oct 2020.
        \item \textbf{Merit Scholarship}, awarded to MS students at UPB with the GPA greater than 9.5 out of 10, my GPA 10/10.
        \item \textbf{Performance Scholarship}, awarded to students with the GPA greater than 9.7 out of 10 and excellent academic results, every semester of my bachelor studies at UPB, my GPA 10/10.
    \end{itemize}
}
% driving licence
\cvitem{
    \cvheadingstyle{\textbf{University level contests}}
}{
    \begin{itemize}[leftmargin=*]
        \item \textbf{International Mathematics Competition (IMC)}, Blagoevgrad, Bulgaria (org. by Univ. College London): \\
              % {\color{red}{Fun fact: Organizata chiar de UCL (Prof. John Jayne), insa in Bulgaria}} \\
              \textbf{2017} (Bronze Medal), \textbf{2016} (Bronze Medal), \textbf{2015} (Bronze Medal), \textbf{2014} (Bronze Medal)

        \item \textbf{South Eastern European Mathematical Olympiad (SEEMOUS)}: \\
              \textbf{2015} (Third Place, Gold Medal), \textbf{2014} (Silver Medal).

        \item \textbf{``Ion I. Agârbiceanu" General Physics National Competition}, Bucharest, Romania:\\
              \textbf{2016} (Second Place), \textbf{2014} (First Place).

        \item Participated to \textbf{International Physics Contest PLANCKS 2015}, Leiden, Netherlands.

        \item \textbf{``Traian Lalescu'' Romanian National Mathematics Contest}, Timisoara, Romania:\\
              \textbf{2014} (Third Place).

        \item \textbf{Students Scientific Session Contest} at UPB, Faculty of Automatic Control and Computer Science, Romania: \\
              \textbf{2015} (Second Place), team leader of the project: \emph{``Trajectory controller for utility mobile robots'' (involved writing a covering algorithm for a definite area and implementing it in Vivado)}. \\
              \textbf{2014} (Second Place), team leader of the project: \emph{``Determination of the parameters of magnetically coupled resonators used in wireless transfer of electromagnetic energy using Q3D Extractor and Matlab''}.

    \end{itemize}
}
\cvitem{
    \cvheadingstyle{\textbf{Pre-university level Olympiads}}
}{
    \begin{itemize}[leftmargin=*]
        \item \textbf{Romanian National Physics Olympiad}: \\
              \textbf{2013} (Honour Mention), \textbf{2010} (Silver Medal), \textbf{2009} (Silver Medal), \textbf{2008} (Bronze Medal).
        \item \textbf{Romanian National Mathematics Olympiad}: \\
              \textbf{2010} (Silver Medal), \textbf{2009} (Silver Medal), \textbf{2008} (Bronze Medal), \textbf{2007} (Gold Medal).

        \item \textbf{Romanian National Earth Science Olympiad}: \\
              \textbf{2013} (Honour Mention).
    \end{itemize}


}

\cvsection{Summer Schools (study and training)}

\vspace{\cvbetweensectionandheadingextraskipamount}



\cvitem{
    \cvdurationstyle{\textbf{Sep. 2019}}
}{
    \textbf{Ferienakademie 2019}, Italy, organised by TU Munich, University of Stuttgart and FAU Erlangen-Nuremberg
    \begin{itemize}[leftmargin=*]
        \item Project: Autonomous Drones for Sustainability - deploy a system of drones to scan the surface of a polluted lake and take water samples to be tested in a mobile laboratory.
              % {\color{red}{Din nou, a fost un team project de vreo 20 de persoane}}
        \item Two weeks project (22nd Sep - 4th Oct) with a 20-student team.
        \item I was part of the mobile laboratory team and worked on design and implementation of an underwater temperature sensor that reads real-time data, send it to an Arduino Uno that keeps the communication with the Raspberry PI that is the core of the drone.
    \end{itemize}

}
\cvitem{
    \cvdurationstyle{\textbf{May 2017}}
}{  \textbf{``\'Ecole d’\'et\'e franco-roumaine}: Commande Avanc\'ee des Systèmes et Nouvelles Technologies
    Informatiques (CA'NTI) '', one week Control Engineering courses attended in French at the Faculty of Automatic Control and Computer Science, UPB, Romania (24th-29th May).

}

% \cvitem{
%     \cvheadingstyle{Scholarships}
% }{
% \begin{itemize}[leftmargin=*]
%     \item DAAD fellowship (German Academic Exchange Service), Oct 2019 - Oct 2020
% \end{itemize}
% }

% Extra

\cvsection{Miscellaneous}
\cvitem{
    \cvheadingstyle{Extracurricular}
}{

    Chess player affiliated at International Chess Federation (FIDE):
    \begin{itemize}[leftmargin=*]
        \item ELO ratings: 1855 (classical), 1922 (rapid), 1951 (blitz).
    \end{itemize}
}
\cvitem{
    \cvheadingstyle{Others}
}{
    %Robotics (algorithms implementation - {\color{red}{nu am facut asta niciodata...}}), Programming ({\color{red}{programare fac in mod uzual la master, dar nu si extra}}..)...\\
    Hiking, Traveling (cultural discovery), Reading (F. Dostoevsky), Music (classical music).


}

\clearpage

\newpage

\newgeometry{left=1.5cm,right=1.5cm,top=1.5cm,bottom=1.5cm}

\begin{center}
    \Large{\textbf{Brief description of most relevant projects}}
\end{center}
%\maketitle
%\date{}
\section{Final internship at Tronics Microsystems, Crolles, France}

\begin{description}
    \item[Time period:] February 2021- July 2021
    \item[Title:] Model and simulate in Matlab-simulink a high performance MEMS gyrometer sensor to predict its mechanical and electrical behavior
    \item[Purpose:] Developing 2 models of poducts made by Tronics Microsystems to simulate their behavior.
    \item[Keywords:] Accelerometer, Gyrometer, Matlab, Simulink, Model, Simulation, MEMS sensors, ASIC
    \item[Subject:] %Creating a high-performance C++ implementation for simulating the two-track model of car by using appropriate numerical schemes and optimized mathematical-related libraries (Intel MKL, Eigen). This model is used for simulating various driving scenarios and performing uncertainty quantification.
    \item[Details:] %For the design of autonomous vehicles, high fidelity simulation in combination with uncertainty quantification is one of the key ingredients, as machine learning algorithms need to be exposed to many relevant driving situations. Moreover, a high-fidelity simulation framework is important to achieve a high level of safety. The focus of this project is on the technology pillar of vehicle dynamics. Because of their generality, the usually multibody dynamics software packages might not have an acceptable runtime for a simulation framework for autonomous driving. The designed vehicle dynamics model is able to handle displacement or velocity dependent input parameters (damping, stiffness). Two models with 14 DOF (using Lagrangian-Eulerian approach) and 30 DOF (using multibody dynamics), respectively, were firstly designed in Matlab and then implemented in C++.

        The work on this project is in progress and will be finished in July when I will provide a technical report and defend.
\end{description}

\section{Drone project: Sense and avoidance strategies for collision avoidance implemented in ROS}
\begin{description}
    \item[Time period:] September 2020 - January 2021
    \item[Title:] SenseROS: Sense and avoidance strategies for collision
        avoidance implemented in ROS
    \item[Team:] With 3 other classmates from Esisar, INP Grenoble
    \item[Keywords:] Collision avoidance, Unmanned Aerial Vehicles (UAVs), ROS, Gazebo, Python, obstacles, LIDAR.
    \item[Abstract:] Inspired by some practical applications concerning collision avoidance topics, this project focuses on the
        implementation of sense and avoidance strategies for collision avoidance coded in Python, simulated in ROS (Robot Operating System) with a Gazebo environment. The Unmanned Aerial Vehicle (UAV) uses a LIDAR to detect the obstacles, calculates a trajectory to reach the destination without collision with the static and dynamic obstacles in his path. By static obstacle, we mean cylindrical obstacles which can be set in the initial map and a dynamic obstacle can be another drone in the environment. In order to illustrate the benefits of the proposed method, typical applications involving the control of multi-agent systems are considered.

        This project extended my knowledge in computer programming and trajectory planning acquired at Esisar.

        The results of this project were gathered in a report and a presentation. The defense was appreciated by the jury members and the class who elected this project as the best one.
\end{description}

\section{Technician Internship at UNIFEI, Itajuba, BRAZIL}
\begin{description}
    \item[Time period:] June 2019 - August 2019
    \item[Title:]  First proposal of a communication between Petri nets and an applet programmed in Java
    \item[Keywords:] CPN Tools, Petri nets, conveyor, Java, applet, communication
    \item[Abstract:] %The main purpose of this thesis was studying the Matrix-Sign Function technique applied in the context of solving Continuous Algebraic Riccati Equation (CARE), that is ubiquitous in optimal and robust control design. It is one of the three main methods for solving the CAREs and I studied its conditioning and several approaches in order to assure a faster convergence and avoiding the required explicit matrix inversions that appears invariable at the beginning of the MSF iterations. My application consists in designing an LQR for a better poles' placement for the longitudinal motion of a civil aircraft, as well as using LQR to track a step reference.

        The results were gathered in a report and the student I helped succeeded her final project.
\end{description}



% write at each project around 5 to 10 lines describing the results

% ---- give some keywords at each project\\
% ---- mention if you were the leader of the team \\
% ---- organize the projects in chronological order

% \section{Exemplu pentru descrierea unui proiect, poti sa faci mai scurt dc doresti si cu itemize}

% Research conducted within my third year of university studies at UPB (Mar. - Apr. 2014) \\

% 2-axis solar panel tracking system \\

% Keywords : solar panel, mobile robot, Arduino microcontroller. \\

% The purpose of this project was to design a solar panel which tries to obtain maximal illumination
% for a maximal time. In this project, we worked in a team of 3 students, for which I was the leader
% taking responsibility for time and task scheduling. To fulfill the tasks of the project we first decided
% to use a microprocessor trainer system model Z3/EV, which we programmed using Assembly language
% programming. Due to the communication delays and time-variations between Z3/EV and the other
% components (sensors, solar panel and two servomotors), we choose an Arduino microcontroller unit for
% the implementation of the control laws. After an adequate research on both, conventional linear control
% (such as the usual PID controller) and modern control (such as fuzzy control), we decided to use a PI
% controller. A reason for this choice was that this control algorithm is very easy to implement in the
% Arduino microcontroller unit. We have obtained good practical results with a negligible tracking error.
% Also, we have tackled the case of temporarily cloudy weather to minimize the energy consumption, by
% considering it as perturbations and trying to reject them. This work resulted in a prototype solar panel,
% a technical report and a final presentation which was rewarded with the First Prize at the Scientific
% Research Conference of Students, Faculty of Automatic Control and Computer Science, UPB.

% \clearpage

% \newpage



% \title{Motivation Guideline}
% \maketitle

% \vspace{1cm}

% I am writing this motivation letter in order to apply to the 2020 DGA doctoral grant FR-UK collaboration ``Reliable and energy-efficient data collection in a
% complex environment through a team of autonomous aerial
% vehicles''. 

% Assoc. Prof.. Ionela PRODAN from the LCIS laboratory of Grenoble INP has apprised me about this opportunity. As I
% will detail below, I strongly believe that the proposal’s topic is a close fit to my research interests and
% academic dddistrajectory up to now. \\


% My undergraduate studies have been dedicated to the pursue of knowledge in ..... \\



% The four years of Bachelor studies .... \\

% Master... \\

% As extracurricular activities, I can mention my multiple participations and awards obtained at national and international level competitions.

% ... \\

% ---- ar trebui un paragraf in cae se vorbesti despre proiect, de ce e interesant ptr tine. De ce ai vrea sa vii in Franta, la Grenoble INP, Valence si UCL London..., Ce directii ai vrea sa iti deschida acesta teza, ce perspective ai... \\


% The DGA thesis proposal for which I am applying mentions goals like multiple autonomous vehicles with applications for area coverage using..... I believe that my scientific background is well suited to these topics, ...

% Consequently,
% I believe that I can start the work quickly and efficiently.... 


% .... \\


% Finally, I firmly believe that I can give my contribution to the development of research activities
% through the “DGA doctoral grant 2020” and I look forward to join the LCIS (Grenoble INP) and UCL
% (Univ. College London) laboratories, enjoy the environment, find interesting ideas and open problems.....

\newpage

\normalsize{
    \begin{center}
        Motivation Letter
    \end{center}

    \vspace{0.5cm}
    \par Dear Madam / Sir,

    \vspace{0.5cm}

    \par This motivation letter is meant to support my application to the 2021 DGA doctoral grant within the FR-UK collaboration titled ``Reliable and energy-efficient data collection in a complex environment through a team of autonomous aerial vehicles''.

    \vspace{0.5cm}

    ... if it is simple for you, you can write the letter in French.




}

\end{document}

