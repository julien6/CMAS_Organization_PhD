%*******************************************************
% Abstract
%*******************************************************
\renewcommand{\abstractname}{Abstract}
\pdfbookmark[1]{Résumé}{Résumé}
\begingroup
\let\clearpage\relax
\let\cleardoublepage\relax
\let\cleardoublepage\relax

\begin{otherlanguage}{ngerman}
  % \pdfbookmark[1]{Résumé}{Résumé}
  \chapter*{Résumé}

  Face à la complexité croissante des menaces en Cybersécurité, les approches centralisées montrent leurs limites pour protéger efficacement des systèmes distribués et dynamiques. Cette thèse explore une approche distribuée fondée sur des \acplu{SMA} capables de détecter, répondre et s'adapter collectivement à des attaques autonomes et évolutives.
  %
  L'objectif central est de permettre la conception d'un SMA de cyberdéfense en trouvant un mécanisme d'organisation adapté aux contraintes des concepteurs et de l'environnement. La littérature sur le sujet met en lumière l'approche symbolique favorisant le contrôle et l'approche connexionniste favorisant la performance. Pour dépasser cette tension, la thèse propose une méthode hybride combinant un modèle organisationnel symbolique et \acn{MARL}.

  La clé de cette méthode consiste à voir la conception d'un SMA au travers d'un \textit{problème d'optimisation sous contraintes}, dans lequel la politique conjointe des agents est apprise tout en respectant des contraintes organisationnelles exprimant les exigences du concepteur. Cette approche requiert à la fois une modélisation fidèle de l'environn\-ement et une capacité à analyser et contrôler les comportements obtenus.
  %
  La méthode intègre les différents travaux selon quatre activités : (i) \textbf{modélisation} de l'environnement cible à l'aide de techniques manuelles ou de type \textit{World Models}, pour obtenir une version simulée de l'environnement cible~; (ii) \textbf{entraînement} des agents via MARL, avec intégration de contraintes issues du modèle organisationnel $\mathcal{M}OISE^+$~; (iii) \textbf{analyse} des politiques apprises, en extrayant rôles et objectifs implicites via des méthodes non supervisées sur les trajectoires~; (iv) \textbf{transfert} des résultats dans l'environnement réel, avec mise à jour continue des modèles et politiques.

  Un outil logiciel a été développé pour mettre en œuvre cette méthode, et appliqué à trois cas d'usage : un essaim de drones, une infrastructure d'entreprise et une architecture de micro-services. Les résultats montrent une amélioration en termes de résilience, d'adaptabilité et d'autonomie par rapport aux approches centralisées.
  %
  Enfin, cette thèse ouvre plusieurs perspectives de recherche : l'amélioration de la modélisation de l'environnement grâce à l'intégration de connaissances expertes, le renforcement de la robustesse de l'apprentissage dans des environnements dynamiques, ainsi que l'exploration des représentations latentes pour faciliter l'analyse organisationnelle.

  \medskip

  \

  \noindent MOTS-CLEFS :
  Système Multi-Agent \raisebox{0.25ex}{\tiny$\bullet$} Cyberdéfense \raisebox{0.25ex}{\tiny$\bullet$} Apprentissage par Reinforcement Multi-Agent \raisebox{0.25ex}{\tiny$\bullet$} Autonomous Intelligent Cyberdefense Agent \raisebox{0.25ex}{\tiny$\bullet$} Conception assistée

\end{otherlanguage}

\clearpage
\thispagestyle{empty}
\null
\newpage

\chapter*{Abstract}

Faced with the growing complexity of cybersecurity threats, centralized approaches are showing their limitations in effectively protecting distributed and dynamic systems. This thesis explores a distributed approach based on \acplu{MAS}, capable of collectively detecting, responding to, and adapting to autonomous and evolving attacks.
%
The central objective is to enable the design of a MAS for cyber defense by finding an organizational mechanism suited to the constraints of both the designers and the environment. The literature highlights the symbolic approach, which favors control, and the connectionist approach, which favors performance. To overcome this tension, the thesis proposes a hybrid method combining a symbolic organizational model with \acn{MARL}.

The key idea of this method is to view the design of a MAS as a \textit{constraint optimization problem}, in which the agents' joint policy is learned while respecting organizational constraints expressing the designer's requirements. This approach requires both a faithful modeling of the environment and the ability to analyze and control the resulting behaviors.
%
The method integrates the various contributions into four activities: (i) \textbf{modeling} the target environment using manual techniques or \textit{World Models}, to obtain a simulated version of the target environment; (ii) \textbf{training} the agents via MARL, incorporating constraints derived from the organizational model $\mathcal{M}OISE^+$; (iii) \textbf{analyzing} the learned policies by extracting implicit roles and objectives through unsupervised methods applied to trajectories; (iv) \textbf{transferring} the results to the real environment, with continuous updates of the models and policies.

A software tool was developed to implement this method and applied to three use cases: a drone swarm, an enterprise infrastructure, and a microservices architecture. The results show improvements in terms of resilience, adaptability, and autonomy compared to centralized approaches.
%
Finally, the thesis opens several research directions: improving environment modeling through the integration of expert knowledge, strengthening the robustness of learning in dynamic environments, and exploring latent representations to facilitate organizational analysis.

\medskip

\

\noindent KEYWORDS :
Multi-Agent Systems \raisebox{0.25ex}{\tiny$\bullet$} Cyberdefense \raisebox{0.25ex}{\tiny$\bullet$} Multi-Agent \raisebox{0.25ex}{\tiny$\bullet$} Autonomous Intelligent Cyberdefense Agent Reinforcement Learning {\tiny$\bullet$} Assisted-Design

\endgroup

\vfill