\begin{tikzpicture}[
        chapter/.style={draw, fill=blue!10, thick, minimum width=9cm, minimum height=1.2cm, text centered, font=\bfseries},
        section/.style={draw, fill=blue!5, thick, minimum width=8cm, minimum height=1cm, text centered, font=\small},
        arrow/.style={-Latex, thick},
        node distance=0.4cm,
        annotated/.style={above,font=\small\itshape, inner sep=1pt, yshift=0.8cm, xshift=-8cm}
    ]

    % Chapitre 6 : MAMAD comme réponse
    \node[chapter] (ch6) {\parbox{15cm}{\centering Chapitre 6 : Présentation globale de la méthode}};
    \node[section, below=1cm of ch6, xshift=-2cm] (ch6s1) {Application flexible de la méthode};
    \draw[arrow] ($ (ch6.south) + (4.0,0) $) -- ++(0,0) |- (ch6s1.east) node[annotated] {Détérmination des modalités d'application de la méthode.};

    % Chapitre 7 : Activité 1 — Modélisation
    \node[chapter, below=1cm of ch6s1, xshift=2cm] (ch7) {Chapitre 7 : Modéliser l'environnement en simulation};
    \node[section, below=1cm of ch7, xshift=-2cm] (ch7s1) {Travaux mobilisés et verrous identifiés};
    \node[section, below=1cm of ch7s1] (ch7s2) {Positionnement et contribution proposée};
    \node[section, below=1cm of ch7s2] (ch7s3) {Description et mise en oeuvre dans l'activité};
    \node[section, below=1cm of ch7s3] (ch7s4) {Synthèse};

    \draw[arrow] ($ (ch7.south) + (4.0,0) $) -- ++(0,0) |- (ch7s1.east) node[annotated] {};
    \draw[arrow] ($ (ch7.south) + (4.0,0) $) -- ++(0,0) |- (ch7s2.east) node[annotated] {};
    \draw[arrow] ($ (ch7.south) + (4.0,0) $) -- ++(0,0) |- (ch7s3.east) node[annotated] {};
    \draw[arrow] ($ (ch7.south) + (4.0,0) $) -- ++(0,0) |- (ch7s4.east) node[annotated] {};

    \draw[arrow] ($ (ch6.south) + (4.5,0) $) -- ($ (ch7.north) + (4.5,0) $) node[annotated, yshift=-0.5cm] {La méthode débute par la création d'un environnement simulé réaliste intégrant objectifs et contraintes};

    % Chapitre 8 : Activité 2 — Apprentissage guidé
    \node[chapter, below=1cm of ch7s4, xshift=2cm] (ch8) {Chapitre 8 : Entraîner des politiques sous contraintes};
    \node[section, below=1cm of ch8, xshift=-2cm] (ch8s1) {Travaux mobilisés et verrous identifiés};
    \node[section, below=1cm of ch8s1] (ch8s2) {Positionnement et contributions proposés};
    \node[section, below=1cm of ch8s2] (ch8s3) {Description et mise en oeuvre dans l'activité};
    \node[section, below=1cm of ch8s3] (ch8s4) {Synthèse};

    \draw[arrow] ($ (ch8.south) + (4.0,0) $) -- ++(0,0) |- (ch8s1.east) node[annotated] {};
    \draw[arrow] ($ (ch8.south) + (4.0,0) $) -- ++(0,0) |- (ch8s2.east) node[annotated] {};
    \draw[arrow] ($ (ch8.south) + (4.0,0) $) -- ++(0,0) |- (ch8s3.east) node[annotated] {};
    \draw[arrow] ($ (ch8.south) + (4.0,0) $) -- ++(0,0) |- (ch8s4.east) node[annotated] {};

    \draw[arrow] ($ (ch7.south) + (4.5,0) $) -- ($ (ch8.north) + (4.5,0) $) node[annotated, yshift=-0.5cm] {Une fois l'environnement modélisé, les agents peuvent être entraînés à agir sous contraintes organisationnelles};

    % Chapitre 9 : Activité 3 — Analyse
    \node[chapter, below=1cm of ch8s4, xshift=2cm] (ch9) {Chapitre 9 : Analyser les comportements émergents};
    \node[section, below=1cm of ch9, xshift=-2cm] (ch9s1) {Travaux mobilisés et verrous identifiés};
    \node[section, below=1cm of ch9s1] (ch9s2) {Positionnement et contributions proposés};
    \node[section, below=1cm of ch9s2] (ch9s3) {Description et mise en oeuvre dans l'activité};
    \node[section, below=1cm of ch9s3] (ch9s4) {Synthèse};

    \draw[arrow] ($ (ch9.south) + (4.0,0) $) -- ++(0,0) |- (ch9s1.east) node[annotated] {};
    \draw[arrow] ($ (ch9.south) + (4.0,0) $) -- ++(0,0) |- (ch9s2.east) node[annotated] {};
    \draw[arrow] ($ (ch9.south) + (4.0,0) $) -- ++(0,0) |- (ch9s3.east) node[annotated] {};
    \draw[arrow] ($ (ch9.south) + (4.0,0) $) -- ++(0,0) |- (ch9s4.east) node[annotated] {};

    \draw[arrow] ($ (ch8.south) + (4.5,0) $) -- ($ (ch9.north) + (4.5,0) $) node[annotated, yshift=-0.5cm] {Les politiques apprises sont ensuite analysées pour révéler les structures organisationnelles émergentes};

    % Chapitre 10 : Activité 4 — Transfert
    \node[chapter, below=1cm of ch9s4, xshift=2cm] (ch10) {Chapitre 10 : Transférer et superviser en environnement réel};

    \node[section, below=1cm of ch10, xshift=-2cm] (ch10s1) {Travaux mobilisés et verrous identifiés};
    \node[section, below=1cm of ch10s1] (ch10s2) {Positionnement et contributions proposés};
    \node[section, below=1cm of ch10s2] (ch10s3) {Description et mise en oeuvre dans l'activité};
    \node[section, below=1cm of ch10s3] (ch10s4) {Synthèse};

    \draw[arrow] ($ (ch10.south) + (4.0,0) $) -- ++(0,0) |- (ch10s1.east) node[annotated] {};
    \draw[arrow] ($ (ch10.south) + (4.0,0) $) -- ++(0,0) |- (ch10s2.east) node[annotated] {};
    \draw[arrow] ($ (ch10.south) + (4.0,0) $) -- ++(0,0) |- (ch10s3.east) node[annotated] {};
    \draw[arrow] ($ (ch10.south) + (4.0,0) $) -- ++(0,0) |- (ch10s4.east) node[annotated] {};

    \draw[arrow] ($ (ch9.south) + (4.5,0) $) -- ($ (ch10.north) + (4.5,0) $) node[annotated, yshift=-0.5cm] {Les politiques validées peuvent être transférées dans l'environnement réel avec supervision adaptative};

\end{tikzpicture}
