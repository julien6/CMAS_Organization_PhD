\begin{tikzpicture}[
        chapter/.style={draw, fill=blue!10, thick, minimum width=9cm, minimum height=1.2cm, text centered, font=\bfseries},
        section/.style={draw, fill=blue!5, thick, minimum width=8cm, minimum height=1cm, text centered, font=\small},
        arrow/.style={-Latex, thick},
        node distance=0.4cm,
        annotated/.style={above,font=\small\itshape, inner sep=1pt, yshift=0.8cm, xshift=-8cm}
    ]

    % Chapitre 15 : Bilan de la thèse
    \node[chapter] (ch15) {Chapitre 15 : Bilan de la thèse};
    \node[section, below=1cm of ch15] (ch15s1) {Synthèse de la démarche et des résultats};
    \node[section, below=1cm of ch15s1] (ch15s2) {Limitations techniques, théoriques, expérimentales};

    \draw[arrow] (ch15) -- (ch15s1) node[annotated] {Retour global sur la méthode proposée introduit l'analyse détaillée de ses résultats.};
    \draw[arrow] (ch15s1) -- (ch15s2) node[annotated] {Identification des limites actuelles, aussi bien techniques que conceptuelles.};

    % Chapitre 16 : Perspectives et ouvertures
    \node[chapter, below=1cm of ch15s2] (ch16) {Chapitre 16 : Perspectives et ouvertures};
    \node[section, below=1cm of ch16, xshift=-2cm] (ch16s1) {Perspectives scientifiques à court et long terme};
    \node[section, below=1cm of ch16s1] (ch16s2) {Ouvertures industrielles, valorisation};

    \draw[arrow] ($ (ch16.south) + (4.0,0) $) -- ++(0,0) |- (ch16s1.east) node[annotated] {Limites identifiées ouvrant sur plusieurs pistes scientifiques de prolongement.};
    \draw[arrow] ($ (ch16.south) + (4.0,0) $) -- ++(0,0) |- (ch16s2.east) node[annotated] {Des opportunités concrètes d'application et de valorisation.};

    \draw[arrow] (ch15s2) -- (ch16) node[annotated] {Identification des limites actuelles pour motiver l'ouverture vers des perspectives futures.};

\end{tikzpicture}
