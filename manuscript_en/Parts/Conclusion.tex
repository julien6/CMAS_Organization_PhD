\clearpage
\thispagestyle{empty}
\null
\newpage

\cleardoublepage
\phantomsection
% \pdfbookmark[1]{GENERAL CONCLUSION}{GENERAL CONCLUSION}
\markboth{\spacedlowsmallcaps{GENERAL CONCLUSION}}{\spacedlowsmallcaps{GENERAL CONCLUSION}}
\part*{GENERAL CONCLUSION}
\label{part:conclusion}

\clearpage
\thispagestyle{empty}
\null
\newpage


% === INTRODUCTION TO THE CONCLUSION ===
% Objective: briefly recap the main theme, introduce the research question, and outline the structure of the conclusion.
\noindent
This conclusion revisits the initial research question, provides a summary of the contributions and a critical assessment of the \acn{MAMAD} method, before opening up academic and industrial perspectives.
% ==============================
\section*{Summary of contributions and assessment}
\label{sec:summary_assessment}
\noindent
This section aims to present a structured summary of the main contributions of the thesis, highlighting the specific contributions made for each key activity in the design and evaluation process of \acplu{SMA}. It also offers a cross-sectional analysis of the results obtained, as well as a critical assessment of the limitations identified and possible avenues for improvement.
\subsection*{Reminder of the research question}
\noindent
The issue that guided this entire thesis can be formulated as follows:
\begin{quote}
  \emph{How can we design a cyberdefense \acn{SMA} capable of satisfactorily achieving its defense objectives, while self-organizing to dynamically adapt to environmental constraints and design requirements?}
\end{quote}
\noindent
This question, posed in the introduction, has made it possible to link two complementary dimensions:
(i) the need for systematic and rigorous design engineering of \acplu{SMAs}, and
(ii) the opportunity offered by \acn{MARL} to bring about effective collective behaviors.
All of the contributions presented in this manuscript have been developed and validated with a view to providing a reasoned answer to this research question. The following section provides a structured summary, criterion by criterion and activity by activity, in order to highlight the concrete contributions, areas of partial coverage, and prospects for improvement.

% === CONTRIBUTIONS BY ACTIVITY ===
\subsection*{Specific contributions by activity}
This subsection provides a detailed analysis of the specific contributions made for each key activity identified in the thesis. For each of these activities (modeling, constrained training, organizational analysis, and transfer/supervision), the initial scientific challenge is recalled, the solution developed is explained, and its effectiveness is illustrated through validations or representative experimental environments.
\paragraph{MOD – Modeling}
The first challenge identified in this thesis concerned the difficulty of having a simulation environment that was accurate, modular, and usable by \acn{MARL} type algorithms.
Indeed, the multi-agent environments in the literature (such as \textit{Overcooked-AI} or \textit{Predator-Prey}) have two major limitations:
(i) a difficulty in realistically representing the complexity of a distributed cyberdefense system, where technical and organizational processes interact simultaneously;
(ii) a lack of systematic link to an explicit theoretical formalism, guaranteeing the reproducibility and comparability of experiments.
To overcome this obstacle, two main contributions have been proposed.
\medskip
\noindent
\textbf{(i) Extension of \textit{World Models} to the multi-agent context.}
We introduced a method for automatically learning a model of the environment (transition and observation dynamics) from collected traces, by generalizing classic \textit{World Models} to the multi-agent case.
This extension takes into account both simultaneous interactions between agents and the organizational constraints on their actions.
It thus offers the ability to automatically generate simulation environments adapted to various scenarios, while reducing dependence on exhaustive expert modeling.
\medskip
\noindent
\textbf{(ii) Proposal of the \acn{MCAS} model.}
In addition, we have developed a formal model \acn{Dec-POMDP} pre-specialized for cyber defense, called \textbf{MCAS}.
This constitutes a guided manual modeling base, in which the designer has a ready-to-use template to quickly instantiate an environment that respects the expected organizational structures (agents, roles, missions, constraints).
\acn{MCAS} thus makes it possible to combine expert modeling and machine learning, while maintaining strong traceability between modeling choices and simulated dynamics.
\medskip
\noindent
\textbf{Integration and validation.}
These two contributions have been integrated into the \acn{CybMASDE} platform, which provides a modular and reproducible execution environment for experiments.
They have been validated on several experimental scenarios:
\begin{itemize}
\item \textit{Predator-Prey}, to test the robustness of the environment model and its ability to handle simple competitive interactions.

\item \textit{Company Infrastructure}, to simulate a cyberdefense organizational infrastructure and verify that the modeling makes it possible to integrate realistic security constraints.
\item \textit{Drone Swarm}, to evaluate the model's ability to represent a dynamic multi-agent communication and coordination graph.
\end {itemize}
\noindent
The results showed that the proposed modeling meets two major objectives:
(i) providing a unified formal framework for the simulation of \acn{SMA} for cyber defense;
(ii) offering flexibility between automation and human expertise, ensuring a compromise between realism, generality, and reproducibility.
Thus, the contributions of the \textbf{MOD} phase lay the foundations for the \acn{MAMAD} pipeline, ensuring that multi-agent learning is based on credible and exploitable environments.

\paragraph{TRN – Training under constraints}
The second bottleneck identified concerned the difficulty of orienting multi-agent learning in such a way as to guarantee, beyond simple cumulative performance, essential properties such as safety, stability, and organizational compliance.
Indeed, classical \acn{MARL} approaches favor the optimization of global reward, but offer little guarantee of compliance with critical constraints (non-interference, role consistency, coordination according to a defined mission).
In the field of cyber defense, unsupervised learning can thus lead to behaviors that are locally effective but dangerous or inconsistent from a global perspective.
\medskip
\noindent
\textbf{(i) Integration of the organizational model $\mathcal{M}OISE^+$ into \acn{MARL}.}
To overcome this obstacle, the thesis proposes a novel integration between the organizational formalism \textbf{MOISE} and \acn{MARL} algorithms, giving rise to the \textbf {MOISE+MARL}.
This coupling makes it possible to explicitly represent roles, missions, and permissions in an organizational graph, then translate these specifications into \textit{constraint guides} applied during learning.
The actions explored by agents are thus filtered or weighted according to their compatibility with the global objectives defined by the organization.
\medskip
\noindent
\textbf{(ii) Safety and stability mechanisms.}
This organizational guidance has two main effects:
(i) it reduces the space of policies explored, limiting inconsistent or dangerous behaviors,
(ii) it improves the stability of learning by more quickly directing trajectories toward behaviors compatible with the overall mission.
The results showed a clear improvement in \textit{convergence} (higher average cumulative reward, reduced variance $\sigma_R$) and better \textit{robustness to disturbances} (resilience to the injection of unexpected events) .
In addition, this constrained filtering makes it possible to maintain organizational safety guarantees, a particularly critical point for cyberdefense systems.
\medskip
\noindent
\textbf{(iii) Multi-environment experimental validation.}
The MOISE+MARL framework has been validated in several test environments:
\begin{itemize}
  \item \textit{Company Infrastructure}, to represent a critical cyberdefense infrastructure and show that the integration of security constraints improves the consistency of defense policies.

  \item \textit{Drone Swarm}, to demonstrate the ability to stabilize learning in distributed and dynamic scenarios, with role reconfiguration when nodes are lost.
  \item \textit{Predator-Prey}, as a simplified reference environment, to measure the impact of constrained guidance on convergence speed and robustness in the face of random variations.
\end{itemize}
\medskip
\noindent
\textbf{(iv) Methodological impact.}
The contribution of \textbf{TRN} illustrates the value of an \textit{organizational consciousness} approach in multi-agent learning.
By integrating formal constraints, learning is no longer optimized solely for raw performance, but also for organizational compliance, safety, and decision transparency.
This result constitutes a new contribution, in that it demonstrates the possibility of linking a symbolic formalism ($\mathcal{M}OISE^+$) and a numerical optimization technique (\acn{MARL}) in a unified framework.
This is an important step towards Cyberdefense \acn{SMA}s that are not only capable of learning, but also of complying with explicit safety and coordination rules.
\paragraph{ANL – Organizational Analysis}
The third obstacle identified was the lack of \textbf{explainability} and \textbf{organizational analysis capability} in \acplu {SMA} trained by reinforcement.
In classical \acn{MARL} approaches, evaluation is based primarily on global numerical metrics (cumulative reward, success rate, stability), which reflect performance but not the internal mechanisms that led to the observed behaviors.
This opacity limits the confidence of designers, hinders validation by human experts, and makes it difficult to transfer to real-world environments subject to strong constraints (such as cybersecurity).
\medskip
\noindent
\textbf{(i) Proposal of the \acn{TEMM} method.}
To overcome this limitation, we have introduced \textbf{TEMM}, an approach that analyzes learned behaviors through their trajectories.
The central idea is to consider that a set of trajectories reflects an implicit emerging organization, which can be highlighted by identifying:
\begin{itemize}

  \item \textbf{roles} (groups of agents adopting similar or complementary behaviors),
  \item \textbf{objectives} (sequences of actions converging towards a common goal),
  \item \textbf{missions} (coordinated sets of roles and objectives observed in collective dynamics).
\end{itemize}
\acn{TEMM} combines \textit{temporal clustering}, sequence detection, and graph analysis techniques to reconstruct this implicit organization and compare it to expected organizational specifications.
\medskip
\noindent
\textbf{(ii) \acn{Auto-TEMM} extension: automation and parameter optimization.}
To enhance the robustness and generality of the approach, we have developed an extension called \textbf{\acn{Auto-TEMM}}.
This variant automates the choice of critical hyperparameters (number of clusters, temporal granularity, similarity thresholds) using Bayesian optimization and active learning techniques.
It reduces dependence on human expertise in the analysis and allows the method to be applied in a reproducible manner to a wide range of scenarios.
\medskip
\noindent
\textbf{(iii) Organizational explainability versus traditional \acn{XAI}.}
Unlike \textit{Explainable AI} (\acn{XAI}) approaches focused on the local interpretation of neural model decisions (e.g., feature importance, gradient attribution), \acn{TEMM} and \acn{Auto-TEMM} propose \textbf {organizational explainability}.
This approach aims not to explain an isolated action, but to reconstruct and interpret the \textit{collective interaction patterns} and the emerging structure of an \acn{SMA}.
This perspective is particularly relevant for distributed cyberdefense systems, where the challenge is not only to justify an individual decision, but to understand the overall organizational logic that led to the success (or failure) of the mission.
\medskip
\noindent
\textbf{(iv) Experimental validation and results obtained.}
\acn{TEMM} and \acn{Auto-TEMM} were integrated into the \acn{CybMASDE} platform and tested on several experimental scenarios:
\begin{itemize}
  \item In the \textit{Company Infrastructure} environment, the analysis made it possible to reconstruct roles such as \textit{proactive defender} and \textit{flow supervisor}, highlighting the compliance of the learned policies with the initial security specifications.

  \item In the \textit{Drone Swarm} environment, \acn{TEMM} revealed emerging missions corresponding to coverage and distributed communication patterns that were not explicitly programmed by the designer but were consistent with organizational constraints.
  \item In the \textit {Predator-Prey}, \acn{Auto-TEMM} demonstrated the approach's ability to automatically detect \textit{pursuit} and \textit{blocking} roles, while optimizing analysis parameters to obtain a clear and reproducible representation.
\end{itemize}
\medskip
\noindent
\textbf{(v) Methodological and scientific impact.}
The \textbf{ANL} contribution demonstrates the feasibility of a systematic and partially automated a posteriori organizational analysis of reinforcement-trained \acn{SMA} behaviors.
It introduces a new form of explainability, focused on \textbf{collective and organizational structure} rather than individual decision-making.
This constitutes a contribution to the field, bringing \acn{MARL} methods closer to model-driven engineering practices and paving the way for interactive validation with human experts.
\paragraph{TRF – Transfer and supervision}
The fourth obstacle identified concerned the difficulty of ensuring a \textbf{reliable transfer} of policies learned in a simulated environment to a real system, while guaranteeing consistency between the two worlds.
In the \acn{MARL} literature, this issue is often addressed from the perspective of \textit{sim-to-real transfer}, but the solutions proposed remain limited:
they mainly aim to reduce the distribution gap between simulation and reality, without taking into account the organizational dimension and the constraints specific to critical systems.
However, in the context of cyberdefense or cloud orchestration, it is essential to have mechanisms in place that not only transfer behaviors, but also continuously monitor their adequacy in relation to constraints and objectives.
\medskip
\noindent
\textbf{(i) Introduction of the adaptive digital twin.}
To address this bottleneck, we have proposed the concept of the \textbf{adaptive digital twin}.
This is an intermediate model that maintains dynamic synchronization between the simulated environment (where learning takes place) and the real environment (where agents are deployed).
The digital twin is continuously updated with data flows from the real system (logs, performance metrics, security events), and this information can be fed back into the simulation to adapt policies or revise organizational constraints.
This mechanism ensures \textbf{structural and behavioral consistency} between simulation and reality, thereby reducing the risk of drift between the two contexts.
\medskip
\noindent
\textbf{(ii) Supervision and reconfiguration mechanisms.}
The adaptive digital twin is not limited to the initial transfer.
It also provides an \textbf{online organizational supervision} capability:
the policies deployed in the real system are continuously evaluated against defined organizational constraints and compliance metrics (\acn{SOF}, \acn{FOF}, \acn {OF}).
If non-compliance is detected, the system can trigger \textbf{dynamic reconfiguration} (adjustment of roles, partial redefinition of missions, or resumption of learning in simulation with revised constraints).
This process ensures continuity of safety and robustness, even in the face of unforeseen or changing conditions.
\medskip
\noindent
\textbf{(iii) Experimental validation.}
The concept of adaptive digital twins has been tested in several representative contexts:
\begin{itemize}
  \item In the \textit{Kubernetes} environment, the digital twin made it possible to model the distribution and migration of services within a cluster and to re-inject observed load or failure events into the simulation. This demonstrated the feasibility of \textit{self-adaptive} orchestration, guided by organizational constraints.

  \item In the \textit{Company Infrastructure} environment, the mechanism was used to test defense policies in a simulator enriched by real security logs, demonstrating the system's ability to adjust to emerging threats and infrastructure reconfigurations.
        \ item In a simplified \textit{Drone Swarm} scenario, the digital twin ensured continuity between an ad hoc communication simulator and a realistic network emulator, demonstrating that learned roles could be maintained and adjusted despite dynamic connectivity losses.
\end{itemize}
\medskip
\noindent
\textbf{(iv) Methodological and applicative impact.}
The \textbf{TRF} contribution highlights an integrated approach to transfer and supervision that goes beyond simple sim-to-real adaptation.
It proposes an \textbf{adaptive organizational framework}, where simulation is no longer an isolated environment, but a component continuously coupled to the real system.
This result constitutes a contribution of the thesis: it demonstrates the possibility of designing \acn{SMA} capable not only of learning and explaining themselves, but also of \textbf{transferring and supervising themselves dynamically} in critical and distributed environments.
% === \acn{TRANSVERSAL} CONTRIBUTIONS ===
\subsection* {Technical contributions}
Beyond the contributions specific to each activity (\textbf{MOD}, \textbf{TRN}, \textbf{ANL}, \textbf{TRF}), the thesis makes a major cross-cutting contribution on a technical level: the development of the \textbf {CybMASDE} platform.
This platform is the integrated implementation of the \acn{MAMAD} pipeline and fulfills several essential functions:
\begin{itemize}

  \item \textbf{Modular and reproducible environment}: CybMASDE offers a modular architecture that allows the steps of modeling, learning, analysis, and transfer to be linked together in a coherent manner. Each component (simulator, \acn{MARL} algorithm, organizational analysis, supervision) can be used independently or in combination.
  \item \textbf {Generic experimental framework}: the platform allows a variety of environments to be instantiated (cooperative game, Predator-Prey, Company Infrastructure, Drone Swarm, Kubernetes), demonstrating the generality of the method beyond the context of cyberdefense alone.
  \item \textbf{Traceability and reproducibility}: all experiments conducted in CybMASDE can be configured using parameter files. This approach facilitates comparison between methodological variants.
  \item \textbf{Foundation for valorization}: by offering a concrete implementation of the \acn{MAMAD} framework, CybMASDE provides a technical foundation for future academic work (open source, reuse of modules), but also for possible industrialization (integration into DevOps pipelines or distributed systems).
\end{itemize}
Thus, CybMASDE is not only a validation tool, but a \textbf{true design and experimentation framework}, putting into practice all the methodological contributions of the thesis.

\noindent
Although these initial results confirm the feasibility of the principle underlying our structured approach to the design of an \acn{SMA}, they also open up directly onto the academic and industrial prospects discussed in the following section.
% ==============================
\section*{Prospects and openings}
\label {sec:perspectives}
% === \acn{ASSESSMENT} \acn{CRITIQUE} ===
\subsection*{Assessment and areas for improvement}
The evaluation of the contributions shows that the \acn{MAMAD} method provides a largely positive response to the research question, combining formal modeling, constrained learning, organizational analysis, and adaptive supervision.
However, several areas for improvement have been identified. Far from constituting insurmountable limitations, they represent \textbf{levers for improvement} that naturally open up academic and industrial perspectives.
\begin{itemize}
  \item \textbf{Automation of modeling}: the initial creation of a simulator remains costly in terms of human expertise (definition of rules, organizational patterns). The World Models approach has been a significant advance in the modeling of dynamics. However, integrating explicit expert knowledge into World Model training or adding interpretable rules would be a major area for improvement in terms of increasing the accuracy and generality of modeling.
  \item \textbf{Adaptability of constraints}: The organizational constraints integrated into learning are currently static. This observation opens up the prospect of \textbf{organizational meta-learning}, where constraints could evolve dynamically according to context, user feedback, or operational conditions.
  \item \textbf{Validation in real conditions}: The experiments conducted are mainly based on simulated environments. This point calls for extending validation to real or hybrid systems (e.g., operational cloud infrastructures, robotic prototypes, distributed cybersecurity systems).
  \item \textbf{Computational cost}: certain phases (\acn{MARL} learning, \acn{Auto-TEMM} organizational analysis) remain resource-intensive. This point opens up prospects for parallel optimization and scaling (\acn{GPU} multi-nodes, distributed execution).
  \item \textbf{User-centered evaluation}: The proposed organizational explainability is still evaluated using internal metrics. A natural extension would be to involve human experts to judge the clarity, usefulness, and relevance of the inferred roles and missions. This evaluation work has already begun, but remains limited by the complexity of certain scenarios and the lack of clarity of the elements produced.
\end{itemize}
% === ACADEMIC PERSPECTIVES ===
\subsection* {Academic perspectives}
These areas for improvement point to several avenues for future research, which can be organized according to time horizons.
\medskip
\noindent
\textbf{(i) Short term: towards more automated modeling.}
A necessary improvement is to reduce the engineering effort involved in creating environments.
This involves exploring the combination of \textbf{multi-agent World Models} with approaches to \textbf{generating interpretable rules} (LLMs, symbolic inference), in order to automatically learn dynamics while maintaining traceability that is understandable to an expert.
\medskip
\noindent
\textbf{(ii) Medium term: towards dynamic adaptation of constraints.}
The current static nature of constraints calls for the implementation of \textbf{organizational meta-learning}.
Constraint guides could thus adapt according to:
\begin{itemize}
  \item errors observed during execution,
  \item feedback provided by a human expert,
  \item changes in environmental conditions.
\end{itemize}
This perspective paves the way for greater human integration into the loop and improved organizational resilience.
\medskip
\noindent
\textbf{(iii) Long term: towards formal organizational explainability.}
While \acn{TEMM} and \acn{Auto-TEMM} have enabled an initial form of empirical organizational explainability, a more rigorous formalization is needed.
In the long term, this will involve defining a \textbf{logical and theoretical framework}, explicitly linking observed behaviors, inferred organizational structures, and causal justifications.
This would strengthen the scientific robustness of explainability in a multi-agent context.
\medskip
\noindent
\textbf{(iv) Cross-cutting: user-centered validation.}
Finally, a cross-cutting theme concerns the systematic involvement of human experts in the evaluation of inferred roles and missions.
Such qualitative validation would make it possible to:
\begin{itemize}

  \item assess the readability and relevance of the structures explained,
  \item test the usability of the CybMASDE platform as a co-design tool,
  \item strengthen the acceptability and transfer of the method to operational environments.
\end{itemize}
\medskip
\noindent
Thus, these academic perspectives are not mere extensions, but true \textbf{steps in the evolution} of the \acn{MAMAD} framework.
They aim to consolidate the method scientifically (formalization, automation, scaling) while bringing it closer to the actual practices of \acn{SMA} design and evaluation.
\subsection*{Industrial applications}
Beyond its academic contributions, the \acn{MAMAD} method and its implementation in the \acn{CybMASDE} platform offer several opportunities for application in concrete industrial contexts.
Critical and distributed environments, whether physical or software-based, pose similar challenges in terms of safety, coordination, and explainability, to which this thesis provides appropriate answers.
\medskip
\noindent
\textbf{(i) Cyber-physical systems and autonomous fleets.}
Constraint-driven multi-agent architectures can be deployed for the supervision and coordination of \textbf{drone fleets}, mobile robots, or connected vehicles.
The \acn{MAMAD} approach allows:
\begin{itemize}
  \item to ensure \textbf{resilience to local failures} through organizational redundancy and dynamic reconfigurations,
  \item to facilitate \textbf{adaptive reconfiguration} of the mission in the event of an unforeseen event (node loss, obstacle, failure),

  \item to integrate organizational supervision directly into robotic middleware (\acn{ROS}, \acn{DDS}).
\end{itemize}
These properties are particularly well suited to surveillance, logistics, and defense applications.
\medskip
\noindent
\textbf{(ii) Adaptive orchestration in the cloud (Kubernetes).}
In modern cloud environments, resource management increasingly relies on distributed architectures (microservices, containers).
The integration of \acn{MAMAD} into orchestrators such as \textbf{Kubernetes} opens up several possibilities:
\begin{itemize}

  \item guiding service elasticity and migration strategies based on explicit organizational constraints,
  \item organizing control policies into specialized roles (planner, dispatcher, supervisor),
  \item moving towards \textbf{organizational self-management} of infrastructures, where decisions are made locally but remain consistent globally.
\end{itemize}
\medskip
\noindent
\textbf{(iii) Distributed IT security.}
Cybersecurity environments are a natural field for the promotion of \acn{MAMAD}.
Further work on \acn{AICA} agents capable of playing an even more important role in \textbf{proactive detection and response}, with organizational supervision.
The major contributions are:
\begin{itemize}
  \item the ability to learn robust and distributed defense strategies,
  \item the ability to orchestrate different types of agents (filtering, surveillance, countermeasures) according to an organizational logic,
  \item the traceability and explainability of decisions, facilitating certification and integration into critical systems.
\end{itemize}
This responds to a growing industrial need: to have intelligent, adaptive, and auditable defense systems.
\medskip
\noindent
\textbf{(iv) Software promotion and open source availability.}
The \acn{CybMASDE} platform could be promoted as an \textbf{open source prototyping framework} for research and engineering of constrained \acn{SMA}.
Several avenues are possible:
\begin{itemize}
  \item publication of a stable, documented, and modular version,
  \item provision of reusable components (modules \acn{TEMM}, \acn{HPO}),
  \item extension to other simulation frameworks such as Gymnasium~\cite{kwiatkowski2024}, Isaac Gym~\cite{Makoviychuk2021}, \acn{JAX}~\cite{Frostig2019} with JaxMARL~\cite{Rutherford2024}.
\end{itemize}
An open source license (\acparen{MIT}, \acparen{LGPL}) would both accelerate academic collaborations and pave the way for a dual strategy of academic and industrial exploitation.
\medskip
\noindent
\textbf{(v) Transfer to other industrial sectors.}
Finally, the principles of \acn{MAMAD} apply beyond cyberdefense and the cloud, in any field requiring decentralized intelligent systems:
\begin{itemize}

  \item \textbf{Industry 4.0}: coordination between production lines and distributed logistics,
  \item \textbf{Smart grids}: organizational regulation of energy flows,
  \item \textbf{Environmental monitoring}: cooperation between sensors and drones for monitoring sensitive environments.
\end{itemize}
These sectors share a common need: to guarantee the robustness, explainability, and safety of collective decisions in dynamic and constrained environments.
\medskip
\noindent
Thus, the \acn{MAMAD} method offers a \textbf{strong match between industrial needs and scientific contributions}:
it provides a formal theoretical framework, an operational software pipeline, and guarantees of safety and explainability that are essential in critical environments.
% ==============================
% === CLOSING \acn{OF} \acn{THE} CONCLUSION ===
% Objective: to end on a positive and open note.
\medskip
\noindent
Ultimately, this thesis proposed a method for designing \acplu{SMA} guided by organizational constraints.
It has demonstrated the feasibility and value of such an approach by exploring the possibility of combining formal modeling, reinforcement learning, organizational analysis, and supervised transfer within a single coherent framework.
\medskip
\noindent
The contributions are multi-level: in theoretical structuring (the \acn{MAMAD} framework, integration of MOISE+MARL, organizational explainability), methodological innovations (multi-agent World Models, \acn{TEMM} and \acn{Auto-TEMM}, adaptive digital twin), and technical development (\acn{CybMASDE} platform).
These elements provide answers to the research question and constitute a foundation that can be mobilized and enriched in future work.
\medskip
\noindent
In this work, beyond the academic contributions, the aim was also to outline avenues for exploitation in critical environments such as cybersecurity, distributed cloud, and cyber-physical systems.
This orientation illustrates the potential value of an \textit{organizational consciousness} approach in contexts where autonomy, safety, and explainability are strong requirements.
\medskip
\noindent
The perspectives identified, such as modeling automation, organizational learning, formalization of explainability, and user-centered validation, as well as openings to various fields (autonomous systems, cloud computing, cybersecurity, Industry 4.0, smart grids), outline numerous avenues for further research.
Future work will explore these avenues and measure their impact in different application contexts.
\medskip
\noindent
Finally, this work has sought to bring together two fields that are often studied separately: formal systems engineering and multi-agent learning.
The result is a proposal for an integrated and scalable framework for the design of \acn{SMA}, whose relevance remains to be tested and consolidated, but which could contribute to the emergence of intelligent systems that are robust, explainable, and adapted to critical environments.

\clearpage
\thispagestyle{empty}
\null
\newpage
