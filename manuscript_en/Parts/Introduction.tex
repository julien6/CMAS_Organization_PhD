\clearpage
\thispagestyle{empty}
\null
\newpage
\cleardoublepage
\pagenumbering{arabic}
\cleardoublepage
\phantomsection
% \pdfbookmark[1]{GENERAL INTRODUCTION}{GENERAL INTRODUCTION}
\addcontentsline{toc}{part}{GENERAL INTRODUCTION}
\markboth{\spacedlowsmallcaps{GENERAL INTRODUCTION}}{\spacedlowsmallcaps{GENERAL INTRODUCTION}}
\noindent {\LARGE\textbf{GENERAL INTRODUCTION}}
\
\vspace{1em}
\bigskip
\noindent {\Large\textbf{Motivations}}\\
\noindent
For more than a decade, the complexity, automation, and speed of cyberattacks have challenged traditional approaches to cyberdefense, which are centralized and primarily reactive or based on fixed rules. In a context where the systems to be defended are themselves becoming distributed and dynamic (microservices architectures, drone swarms, industrial IoT, etc.), defense mechanisms must also become more autonomous, adaptable, and resilient.
The idea of an intelligent cyber defense agent, capable of autonomously detecting and countering threats, has been embodied by the concept of the \acn{AICA}~\cite{Kott2023} agent. The first generation of \acn{AICA} took the form of a \textit{monolithic} agent, equipped with perception, decision-making, and action capabilities within a localized perimeter. The complexity of the situations handled quickly justified a move towards a more explicit and modular architecture, illustrated by the \acn{MASCARA}~\cite{Kott2023} architecture, which proposes a cognitive-type symbolic modular architecture for the \acn{AICA} agent.
In line with this work, this distributed approach evolved into a multi-agent version of the \acn{AICA} concept, in which several agents collaborate to defend a complex system, paving the way for the concept of \acn{SMA} for cyberdefense. This evolution raises many questions: how can we specify an organization that is adapted to environmental constraints and attacks? What capabilities should these agents have and at what cost? How can we ensure that their behavior meets functional (performance, adaptation) and non-functional (safety, explainability) requirements?
\
\bigskip
\noindent {\Large\textbf{Objectives}}\\
\noindent
The overall objective of this thesis is to obtain a Cyber Defense \acn{SMA} capable of adapting to the dynamic constraints of the environment to be defended, while maximizing its ability to detect, prevent, and counter threats.
%
Rather than seeking a single Cyberdefense \acn{SMA} solution to address this complex problem, we aim to establish a design method to guide or assist the process of creating such an \acn{SMA}. This method is intended to be generic and capable of integrating, organizing, and orchestrating the various contributions in a coherent manner.
\noindent
The idea of implementing such a method involves taking several challenges into account:
\begin{itemize}
  \item \textbf{Modeling the design problem} of the \acn{SMA} to take into account both the cyberdefense objectives and the constraints imposed by the environment or the designer;

  \item \textbf{Reducing the design cost} by partially freeing oneself from dependence on expert knowledge~;
  \item \textbf{Quantitatively evaluating the performance} of the SMA on criteria such as resilience, autonomy, or adaptability~;

  \item \textbf{Ensure explainability} of the overall behavior of the \acn{SMA}, in particular by identifying emerging roles, objectives, and interactions~;
  \item \textbf{Guarantee operational safety and compliance with critical constraints} through organizational control mechanisms~;

  \item \textbf{Enable dynamic adaptation} of the \acn{SMA} in response to changing constraints and situations in the real environment.
\end{itemize}
\noindent
These challenges are guiding research toward a method that covers the design of the \acn{SMA} for cyberdefense by seeking to benefit from the advantages of both connectionist and symbolic approaches.
\
\bigskip
\noindent {\Large\textbf{Targeted contributions}}\\
\noindent
We seek to establish a method for designing \acn{SMA} for \acplu{SMA} in cyber defense. Assuming the necessary simulation, our method can be approached in four activities~:
\begin{itemize}
  \item \textbf{Modeling}~: creation of a simulated environment either manually following a generic Markovian framework and expert knowledge, or automatically using \acn{ML} techniques that capture the dynamics of the environment from its traces~;
  \item \textbf{Resolution}: obtaining joint multi-agent policies manually or using \acn{ML} techniques under constraints expressed as organizational specifications in particular;
  \item \textbf{Analysis}~: inference of emerging organizational specifications, such as roles or objectives, from the trajectories of agents, using unsupervised learning techniques in particular~;
  \item \textbf{Transfer}: coupling and minimizing the gap between the real environment and the simulated environment in the manner of digital twins by updating the policies of real agents and the simulated environment.
\end{itemize}


\noindent These activities require several contributions:
\begin{itemize}
  \item Formalization of the design problem as an optimization problem under the constraints of the environment and designers. On this basis, a formalization of the method can be proposed to formally model the entire processing chain orchestrating all contributions;

  \item A multi-agent extension of \acn{ML} techniques such as \textit{World Models} will capture the dynamics of observational transition, thus enabling the automatic generation of a model simulating the environment as perceived by the agents;
  \item A framework that will provide a generic framework to guide the manual modeling of a real environment as a simulation using expert knowledge;

  \item A framework that will allow the integration of designers' requirements into the policy acquisition process, such as in a \acn{MARL} process where policy learning can be guided or constrained~;
  \item A method for analyzing emerging behaviors, capable of inferring organizational specifications from the trajectories of trained agents~;
  \item A tool implementing the entire method, including the various contributions and its application in several environments for some or all of the activities.
\end{itemize}
\noindent The method will need to be evaluated in representative cyber defense environments. The method may also be evaluated in non-cyber defense environments to demonstrate its generalizability in other multi-agent contexts.
\
\bigskip
\noindent {\Large\textbf{Manuscript outline}}\\
\noindent
The question of organizing a cyber defense \acn{SMA} first led us to conduct a literature review, the analysis of which highlighted the value of specifying the research question within the framework of an optimization problem. This specified question allows us to structure our approach around a series of hypotheses, combining symbolic and connectionist approaches, which form the basis of the various contributions of the thesis. These contributions are then evaluated experimentally in order to test the validity of the hypotheses and, ultimately, to answer the research question.
\noindent
The manuscript is structured in five parts, each consisting of three to five chapters, as presented in \autoref{fig:manuscript_organization}. This organization follows a progressive reasoning, detailed in \autoref{fig:manuscript_logic}.
\begin{figure}[h!]
  \centering
  \resizebox{0.8\textwidth}{!}{%
    


\tikzset{every picture/.style={line width=0.75pt}} %set default line width to 0.75pt        

\begin{tikzpicture}[x=0.75pt,y=0.75pt,yscale=-1,xscale=1]
    %uncomment if require: \path (0,1743); %set diagram left start at 0, and has height of 1743

    %Shape: Rectangle [id:dp18574584963494045] 
    \draw  [line width=1.5]  (50,150) -- (250,150) -- (250,210) -- (50,210) -- cycle ;
    %Straight Lines [id:da12196343525546238] 
    \draw    (151.5,210) -- (151.5,262)(148.5,210) -- (148.5,262) ;
    \draw [shift={(150,270)}, rotate = 270] [color={rgb, 255:red, 0; green, 0; blue, 0 }  ][line width=0.75]    (10.93,-3.29) .. controls (6.95,-1.4) and (3.31,-0.3) .. (0,0) .. controls (3.31,0.3) and (6.95,1.4) .. (10.93,3.29)   ;
    %Shape: Rectangle [id:dp4847066759253956] 
    \draw   (370,400) -- (570,400) -- (570,440) -- (370,440) -- cycle ;
    %Shape: Rectangle [id:dp3090415206682712] 
    \draw  [line width=1.5]  (50,390) -- (250,390) -- (250,450) -- (50,450) -- cycle ;
    %Straight Lines [id:da08726836991221709] 
    \draw  [dash pattern={on 0.84pt off 2.51pt}]  (450,212) -- (450,400) ;
    \draw [shift={(450,210)}, rotate = 90] [color={rgb, 255:red, 0; green, 0; blue, 0 }  ][line width=0.75]    (10.93,-3.29) .. controls (6.95,-1.4) and (3.31,-0.3) .. (0,0) .. controls (3.31,0.3) and (6.95,1.4) .. (10.93,3.29)   ;
    %Shape: Rectangle [id:dp5698076228521041] 
    \draw  [line width=1.5]  (20,750) -- (280,750) -- (280,810) -- (20,810) -- cycle ;
    %Shape: Rectangle [id:dp4753743401883225] 
    \draw  [line width=1.5]  (0,270) -- (310,270) -- (310,330) -- (0,330) -- cycle ;
    %Straight Lines [id:da04102324085536324] 
    \draw    (151.5,330) -- (151.5,382)(148.5,330) -- (148.5,382) ;
    \draw [shift={(150,390)}, rotate = 270] [color={rgb, 255:red, 0; green, 0; blue, 0 }  ][line width=0.75]    (10.93,-3.29) .. controls (6.95,-1.4) and (3.31,-0.3) .. (0,0) .. controls (3.31,0.3) and (6.95,1.4) .. (10.93,3.29)   ;
    %Shape: Rectangle [id:dp22707656448590674] 
    \draw  [line width=1.5]  (10,630) -- (280,630) -- (280,690) -- (10,690) -- cycle ;
    %Straight Lines [id:da981655106922525] 
    \draw  [dash pattern={on 0.84pt off 2.51pt}]  (250,190) -- (366,190) ;
    \draw [shift={(368,190)}, rotate = 180] [color={rgb, 255:red, 0; green, 0; blue, 0 }  ][line width=0.75]    (10.93,-3.29) .. controls (6.95,-1.4) and (3.31,-0.3) .. (0,0) .. controls (3.31,0.3) and (6.95,1.4) .. (10.93,3.29)   ;
    %Straight Lines [id:da008901650084719437] 
    \draw  [dash pattern={on 0.84pt off 2.51pt}]  (250,420) -- (368,420) ;
    \draw [shift={(370,420)}, rotate = 180] [color={rgb, 255:red, 0; green, 0; blue, 0 }  ][line width=0.75]    (10.93,-3.29) .. controls (6.95,-1.4) and (3.31,-0.3) .. (0,0) .. controls (3.31,0.3) and (6.95,1.4) .. (10.93,3.29)   ;
    %Straight Lines [id:da7631246123173825] 
    \draw    (151.5,690) -- (151.5,742)(148.5,690) -- (148.5,742) ;
    \draw [shift={(150,750)}, rotate = 270] [color={rgb, 255:red, 0; green, 0; blue, 0 }  ][line width=0.75]    (10.93,-3.29) .. controls (6.95,-1.4) and (3.31,-0.3) .. (0,0) .. controls (3.31,0.3) and (6.95,1.4) .. (10.93,3.29)   ;
    %Straight Lines [id:da4329171999280237] 
    \draw    (151.5,570) -- (151.5,622)(148.5,570) -- (148.5,622) ;
    \draw [shift={(150,630)}, rotate = 270] [color={rgb, 255:red, 0; green, 0; blue, 0 }  ][line width=0.75]    (10.93,-3.29) .. controls (6.95,-1.4) and (3.31,-0.3) .. (0,0) .. controls (3.31,0.3) and (6.95,1.4) .. (10.93,3.29)   ;
    %Straight Lines [id:da4945277958586467] 
    \draw    (150.5,449.98) -- (151.37,501.98)(147.5,450.02) -- (148.37,502.03) ;
    \draw [shift={(150,510)}, rotate = 269.05] [color={rgb, 255:red, 0; green, 0; blue, 0 }  ][line width=0.75]    (10.93,-3.29) .. controls (6.95,-1.4) and (3.31,-0.3) .. (0,0) .. controls (3.31,0.3) and (6.95,1.4) .. (10.93,3.29)   ;
    %Shape: Rectangle [id:dp4662158969205311] 
    \draw  [line width=1.5]  (50,510) -- (250,510) -- (250,570) -- (50,570) -- cycle ;
    %Straight Lines [id:da5672568795108123] 
    \draw  [dash pattern={on 0.84pt off 2.51pt}]  (490,442) -- (490,750) ;
    \draw [shift={(490,440)}, rotate = 90] [color={rgb, 255:red, 0; green, 0; blue, 0 }  ][line width=0.75]    (10.93,-3.29) .. controls (6.95,-1.4) and (3.31,-0.3) .. (0,0) .. controls (3.31,0.3) and (6.95,1.4) .. (10.93,3.29)   ;
    %Straight Lines [id:da6449350432746315] 
    \draw  [dash pattern={on 0.84pt off 2.51pt}]  (252,550) -- (450,550) -- (450,750) ;
    \draw [shift={(250,550)}, rotate = 0] [color={rgb, 255:red, 0; green, 0; blue, 0 }  ][line width=0.75]    (10.93,-3.29) .. controls (6.95,-1.4) and (3.31,-0.3) .. (0,0) .. controls (3.31,0.3) and (6.95,1.4) .. (10.93,3.29)   ;
    %Shape: Rectangle [id:dp9145794948784826] 
    \draw   (400,750) -- (630,750) -- (630,810) -- (400,810) -- cycle ;
    %Straight Lines [id:da5051851041653137] 
    \draw    (280,778.5) -- (392,778.5)(280,781.5) -- (392,781.5) ;
    \draw [shift={(400,780)}, rotate = 180] [color={rgb, 255:red, 0; green, 0; blue, 0 }  ][line width=0.75]    (10.93,-3.29) .. controls (6.95,-1.4) and (3.31,-0.3) .. (0,0) .. controls (3.31,0.3) and (6.95,1.4) .. (10.93,3.29)   ;
    %Straight Lines [id:da9009018604767407] 
    \draw    (468.5,810) -- (468.5,862)(465.5,810) -- (465.5,862) ;
    \draw [shift={(467,870)}, rotate = 270] [color={rgb, 255:red, 0; green, 0; blue, 0 }  ][line width=0.75]    (10.93,-3.29) .. controls (6.95,-1.4) and (3.31,-0.3) .. (0,0) .. controls (3.31,0.3) and (6.95,1.4) .. (10.93,3.29)   ;
    %Shape: Rectangle [id:dp9696498661224131] 
    \draw   (400,870) -- (590,870) -- (590,910) -- (400,910) -- cycle ;
    %Straight Lines [id:da7339727457125783] 
    \draw    (151.5,810) -- (151.5,862)(148.5,810) -- (148.5,862) ;
    \draw [shift={(150,870)}, rotate = 270] [color={rgb, 255:red, 0; green, 0; blue, 0 }  ][line width=0.75]    (10.93,-3.29) .. controls (6.95,-1.4) and (3.31,-0.3) .. (0,0) .. controls (3.31,0.3) and (6.95,1.4) .. (10.93,3.29)   ;
    %Shape: Rectangle [id:dp10500637872389273] 
    \draw  [line width=1.5]  (20,870) -- (280,870) -- (280,910) -- (20,910) -- cycle ;
    %Straight Lines [id:da22911770438659773] 
    \draw  [dash pattern={on 0.84pt off 2.51pt}]  (280,890) -- (398,890) ;
    \draw [shift={(400,890)}, rotate = 180] [color={rgb, 255:red, 0; green, 0; blue, 0 }  ][line width=0.75]    (10.93,-3.29) .. controls (6.95,-1.4) and (3.31,-0.3) .. (0,0) .. controls (3.31,0.3) and (6.95,1.4) .. (10.93,3.29)   ;
    %Straight Lines [id:da12093630372596798] 
    \draw    (151.5,910) -- (151.5,962)(148.5,910) -- (148.5,962) ;
    \draw [shift={(150,970)}, rotate = 270] [color={rgb, 255:red, 0; green, 0; blue, 0 }  ][line width=0.75]    (10.93,-3.29) .. controls (6.95,-1.4) and (3.31,-0.3) .. (0,0) .. controls (3.31,0.3) and (6.95,1.4) .. (10.93,3.29)   ;
    %Shape: Rectangle [id:dp13160181319541253] 
    \draw  [line width=1.5]  (26.88,970) -- (286.88,970) -- (286.88,1010) -- (26.88,1010) -- cycle ;
    %Straight Lines [id:da5189191976441709] 
    \draw    (151.5,1010) -- (151.5,1062)(148.5,1010) -- (148.5,1062) ;
    \draw [shift={(150,1070)}, rotate = 270] [color={rgb, 255:red, 0; green, 0; blue, 0 }  ][line width=0.75]    (10.93,-3.29) .. controls (6.95,-1.4) and (3.31,-0.3) .. (0,0) .. controls (3.31,0.3) and (6.95,1.4) .. (10.93,3.29)   ;
    %Shape: Rectangle [id:dp29716526412104993] 
    \draw  [line width=1.5]  (40,1070) -- (260,1070) -- (260,1110) -- (40,1110) -- cycle ;
    %Straight Lines [id:da1456519279417069] 
    \draw  [dash pattern={on 0.84pt off 2.51pt}]  (286.88,990) -- (398,990) ;
    \draw [shift={(400,990)}, rotate = 180] [color={rgb, 255:red, 0; green, 0; blue, 0 }  ][line width=0.75]    (10.93,-3.29) .. controls (6.95,-1.4) and (3.31,-0.3) .. (0,0) .. controls (3.31,0.3) and (6.95,1.4) .. (10.93,3.29)   ;
    %Shape: Rectangle [id:dp04492823912399335] 
    \draw   (400,950) -- (630,950) -- (630,1030) -- (400,1030) -- cycle ;
    %Shape: Rectangle [id:dp07847633718854785] 
    \draw   (370,170) -- (570,170) -- (570,210) -- (370,210) -- cycle ;
    %Shape: Rectangle [id:dp1355817602636692] 
    \draw  [line width=1.5]  (90,1140) -- (110,1140) -- (110,1160) -- (90,1160) -- cycle ;
    %Shape: Rectangle [id:dp07143868735089931] 
    \draw  [line width=0.75]  (350,1139) -- (370,1139) -- (370,1159) -- (350,1159) -- cycle ;
    %Straight Lines [id:da8433905723884118] 
    \draw  [dash pattern={on 0.84pt off 2.51pt}]  (360,1179) -- (360,1197) ;
    \draw [shift={(360,1199)}, rotate = 270] [color={rgb, 255:red, 0; green, 0; blue, 0 }  ][line width=0.75]    (6.56,-1.97) .. controls (4.17,-0.84) and (1.99,-0.18) .. (0,0) .. controls (1.99,0.18) and (4.17,0.84) .. (6.56,1.97)   ;
    %Straight Lines [id:da473045317791972] 
    \draw    (151.5,90) -- (151.5,142)(148.5,90) -- (148.5,142) ;
    \draw [shift={(150,150)}, rotate = 270] [color={rgb, 255:red, 0; green, 0; blue, 0 }  ][line width=0.75]    (10.93,-3.29) .. controls (6.95,-1.4) and (3.31,-0.3) .. (0,0) .. controls (3.31,0.3) and (6.95,1.4) .. (10.93,3.29)   ;
    %Shape: Rectangle [id:dp28718160440236395] 
    \draw  [line width=1.5]  (50,30) -- (250,30) -- (250,90) -- (50,90) -- cycle ;
    %Straight Lines [id:da002924452234624897] 
    \draw    (98.5,1180) -- (98.5,1194) ;
    %Straight Lines [id:da10403872552724802] 
    \draw    (100.5,1180) -- (100.5,1194) ;
    \draw   (101.58,1194) .. controls (100.47,1196) and (99.8,1198) .. (99.58,1200) .. controls (99.36,1198) and (98.69,1196) .. (97.58,1194) ;
    %Shape: Rectangle [id:dp964138966540305] 
    \draw  [fill={rgb, 255:red, 184; green, 233; blue, 134 }  ,fill opacity=1 ] (96.5,1184) -- (102.5,1184) -- (102.5,1190) -- (96.5,1190) -- cycle ;



    % Text Node
    \draw (486,1149.5) node   [align=left] {\begin{minipage}[lt]{174.47pt}\setlength\topsep{0pt}
            \begin{center}
                Données / résultats spécifiques
            \end{center}

        \end{minipage}};
    % Text Node
    \draw (211.5,1149.5) node   [align=left] {\begin{minipage}[lt]{169.17pt}\setlength\topsep{0pt}
            \begin{center}
                Données de jalons globaux
            \end{center}

        \end{minipage}};
    % Text Node
    \draw (217.5,1188.5) node   [align=left] {Développement de chapitres};
    % Text Node
    \draw (481,1189.5) node   [align=left] {Lien entre données / résultats};
    % Text Node
    \draw (150,60) node   [align=left, name=sujet] {\begin{minipage}[lt]{190.21pt}\setlength\topsep{0pt}
            \begin{center}
                Sujet "De l'Organisation\\d'un SMA de Cyberdéfense"
            \end{center}

        \end{minipage}};
    % Text Node
    \draw (238,120.5) node   [align=left] {\textit{amène à une...}};
    % Text Node
    \draw  [fill={rgb, 255:red, 184; green, 233; blue, 134 }  ,fill opacity=1 ]  (137.5,107.5) -- (162.5,107.5) -- (162.5,132.5) -- (137.5,132.5) -- cycle  ;
    \draw (150,120) node   [align=left] {\begin{minipage}[lt]{14.07pt}\setlength\topsep{0pt}
            \begin{center}
                1
            \end{center}

        \end{minipage}};
    % Text Node
    \draw  [fill={rgb, 255:red, 184; green, 233; blue, 134 }  ,fill opacity=1 ]  (297,768) -- (372,768) -- (372,793) -- (297,793) -- cycle  ;
    \draw (334.5,780.5) node   [align=left] {\begin{minipage}[lt]{48.1pt}\setlength\topsep{0pt}
            \begin{center}
                6,7,8,9,10
            \end{center}

        \end{minipage}};
    % Text Node
    \draw (513.67,990.54) node   [align=left] {\begin{minipage}[lt]{153.12pt}\setlength\topsep{0pt}
            \begin{center}
                Validation des hypothèses\\vis-à-vis de la question de\\recherche \& sous-problèmes
            \end{center}

        \end{minipage}};
    % Text Node
    \draw (338,970.5) node   [align=left] {\textit{pour une...}};
    % Text Node
    \draw (150,1090) node   [align=left, name=conclusion] {\begin{minipage}[lt]{53.19pt}\setlength\topsep{0pt}
            \begin{center}
                Conclusion
            \end{center}

        \end{minipage}};
    % Text Node
    \draw (242.5,1040.5) node   [align=left] {\textit{ce qui permet une...}};
    % Text Node
    \draw  [fill={rgb, 255:red, 184; green, 233; blue, 134 }  ,fill opacity=1 ]  (124.5,1028) -- (174.5,1028) -- (174.5,1053) -- (124.5,1053) -- cycle  ;
    \draw (149.5,1040.5) node   [align=left] {\begin{minipage}[lt]{28.25pt}\setlength\topsep{0pt}
            \begin{center}
                15,16
            \end{center}
        \end{minipage}};
    % Text Node
    \draw (156.88,990) node   [align=left, name=resultats] {\begin{minipage}[lt]{142.7pt}\setlength\topsep{0pt}
            \begin{center}
                Discussion des résultats
            \end{center}

        \end{minipage}};
    % Text Node
    \draw (228.5,940.5) node   [align=left] {\textit{permettant une...}};
    % Text Node
    \draw  [fill={rgb, 255:red, 184; green, 233; blue, 134 }  ,fill opacity=1 ]  (137,928) -- (162,928) -- (162,953) -- (137,953) -- cycle  ;
    \draw (149.5,940.5) node   [align=left] {\begin{minipage}[lt]{14.07pt}\setlength\topsep{0pt}
            \begin{center}
                14
            \end{center}

        \end{minipage}};
    % Text Node
    \draw (336,870.5) node   [align=left] {\textit{en utilisant...}};
    % Text Node
    \draw (150,890) node   [align=left] {\begin{minipage}[lt]{180.03pt}\setlength\topsep{0pt}
            \begin{center}
                Cadre experimental et d'évaluation
            \end{center}

        \end{minipage}};
    % Text Node
    \draw (239.5,840.5) node   [align=left] {\textit{évaluées via un...}};
    % Text Node
    \draw  [fill={rgb, 255:red, 184; green, 233; blue, 134 }  ,fill opacity=1 ]  (127,827.5) -- (173,827.5) -- (173,852.5) -- (127,852.5) -- cycle  ;
    \draw (150,840) node   [align=left] {\begin{minipage}[lt]{28.25pt}\setlength\topsep{0pt}
            \begin{center}
                12,13
            \end{center}

        \end{minipage}};
    % Text Node
    \draw (492,890.5) node   [align=left] {\begin{minipage}[lt]{57.14pt}\setlength\topsep{0pt}
            \begin{center}
                CybMASDE
            \end{center}

        \end{minipage}};
    % Text Node
    \draw (563.5,841) node   [align=left] {\textit{implémentées et}\\\textit{intégrées dans l'outil...}};
    % Text Node
    \draw  [fill={rgb, 255:red, 184; green, 233; blue, 134 }  ,fill opacity=1 ]  (455,827.5) -- (479,827.5) -- (479,852.5) -- (455,852.5) -- cycle  ;
    \draw (467,840) node   [align=left] {\begin{minipage}[lt]{13.31pt}\setlength\topsep{0pt}
            \begin{center}
                11
            \end{center}

        \end{minipage}};
    % Text Node
    \draw (155.02,178.08) node   [align=left] {\begin{minipage}[lt]{113.81pt}\setlength\topsep{0pt}
            \begin{center}
                Question de\\recherche globale
            \end{center}

        \end{minipage}};
    % Text Node
    \draw (155,300) node   [align=left, name=revue] {\begin{minipage}[lt]{221.17pt}\setlength\topsep{0pt}
            \begin{center}
                Revue de littérature sur les SMA de\\Cyberdéfense et ses moyens de conception
            \end{center}

        \end{minipage}};
    % Text Node
    \draw (288.5,359.5) node   [align=left] {\textit{qui préconise de pivoter vers une...}};
    % Text Node
    \draw  [fill={rgb, 255:red, 184; green, 233; blue, 134 }  ,fill opacity=1 ]  (137.5,347.5) -- (162.5,347.5) -- (162.5,372.5) -- (137.5,372.5) -- cycle  ;
    \draw (150,360) node   [align=left] {\begin{minipage}[lt]{14.07pt}\setlength\topsep{0pt}
            \begin{center}
                3
            \end{center}

        \end{minipage}};
    % Text Node
    \draw (229.5,240.5) node   [align=left] {\textit{qui initie une...}};
    % Text Node
    \draw (337.5,750.5) node   [align=left] {\textit{menant aux...}};
    % Text Node
    \draw (515.5,779.5) node   [align=left] {\begin{minipage}[lt]{143.35pt}\setlength\topsep{0pt}
            \begin{center}
                Contributions méthodologiques
            \end{center}

        \end{minipage}};
    % Text Node
    \draw (153.5,781) node   [align=left, name=hypothese_ssp] {\begin{minipage}[lt]{174.65pt}\setlength\topsep{0pt}
            \begin{center}
                Hypothèses comblant les lacunes\\pour chaque sous-problème
            \end{center}

        \end{minipage}};
    % Text Node
    \draw (380.5,530.5) node   [align=left] {\textit{orientées vers l'...}};
    % Text Node
    \draw (561,551) node   [align=left] {\textit{qui permettent}\\\textit{d'addresser les...}};
    % Text Node
    \draw (151.73,539.5) node   [align=left, name=hypothese] {\begin{minipage}[lt]{148.77pt}\setlength\topsep{0pt}
            \begin{center}
                Hypothèse d'une\\méthode de conception
            \end{center}

        \end{minipage}};
    % Text Node
    \draw (227,480.5) node   [align=left] {\textit{addressé par l'...}};
    % Text Node
    \draw  [fill={rgb, 255:red, 184; green, 233; blue, 134 }  ,fill opacity=1 ]  (137,467.5) -- (162,467.5) -- (162,492.5) -- (137,492.5) -- cycle  ;
    \draw (149.5,480) node   [align=left] {\begin{minipage}[lt]{14.07pt}\setlength\topsep{0pt}
            \begin{center}
                3
            \end{center}

        \end{minipage}};
    % Text Node
    \draw (254,599) node   [align=left] {\textit{qui structure}\\\textit{les travaux pour une...}};
    % Text Node
    \draw  [fill={rgb, 255:red, 184; green, 233; blue, 134 }  ,fill opacity=1 ]  (135.5,587.5) -- (164.5,587.5) -- (164.5,612.5) -- (135.5,612.5) -- cycle  ;
    \draw (150,600) node   [align=left] {\begin{minipage}[lt]{16.9pt}\setlength\topsep{0pt}
            \begin{center}
                4,5
            \end{center}

        \end{minipage}};
    % Text Node
    \draw (239.5,721) node   [align=left] {\textit{servant à}\\\textit{établir les...}};
    % Text Node
    \draw  [fill={rgb, 255:red, 184; green, 233; blue, 134 }  ,fill opacity=1 ]  (112.5,707.5) -- (187.5,707.5) -- (187.5,732.5) -- (112.5,732.5) -- cycle  ;
    \draw (150,720) node   [align=left] {\begin{minipage}[lt]{48.1pt}\setlength\topsep{0pt}
            \begin{center}
                6,7,8,9,10
            \end{center}

        \end{minipage}};
    % Text Node
    \draw (310.5,400.5) node   [align=left] {\textit{impliquant des}};
    % Text Node
    \draw (307.37,170.5) node   [align=left] {\textit{impliquant des}};
    % Text Node
    \draw (146.5,661) node   [align=left, name=revue_ssp] {\begin{minipage}[lt]{192.26pt}\setlength\topsep{0pt}
            \begin{center}
                Revue de littérature orientée\\méthode pour chaque sous-problème
            \end{center}

        \end{minipage}};
    % Text Node
    \draw  [fill={rgb, 255:red, 184; green, 233; blue, 134 }  ,fill opacity=1 ]  (137.5,227.5) -- (162.5,227.5) -- (162.5,252.5) -- (137.5,252.5) -- cycle  ;
    \draw (150,240) node   [align=left] {\begin{minipage}[lt]{14.07pt}\setlength\topsep{0pt}
            \begin{center}
                2
            \end{center}

        \end{minipage}};
    % Text Node
    \draw (515.5,281) node   [align=left] {\textit{qui permettent}\\\textit{d'englober les...}};
    % Text Node
    \draw (145.8,419) node   [align=left] {Question spécifiée via un\\problème d'optimisation};
    % Text Node
    \draw (471.5,189.5) node   [align=left] {Critères};
    % Text Node
    \draw (466.5,419.5) node   [align=left] {Sous-problèmes};
    % Text Node
    \draw (99.5,1187) node  [font=\tiny] [align=left] {\begin{minipage}[lt]{8.67pt}\setlength\topsep{0pt}
            \begin{center}
                {\tiny x}
            \end{center}

        \end{minipage}};

    \draw[decorate, decoration={brace, amplitude=20pt}, thick]
    ($([yshift=-5pt]revue.west |- hypothese.west)+(-0.3,-0.4)$) -- ($([yshift=5pt]revue.west |- sujet.west)+(-0.3,-0.4)$)
    node[midway,xshift=-2cm,rotate=0]{\textbf{Partie I}};

    \draw[decorate, decoration={brace, amplitude=20pt}, thick]
    ($([yshift=-5pt]revue.west |- revue_ssp.west)+(-0.3,-0.4)$) -- ($([yshift=5pt]revue.west |- hypothese.west)+(-0.3,-0.4)$)
    node[midway,xshift=-2cm,rotate=0]{\textbf{Partie II}};

    \draw[decorate, decoration={brace, amplitude=20pt}, thick]
    ($([yshift=-5pt]revue.west |- hypothese_ssp.west)+(-0.3,-0.4)$) -- ($([yshift=5pt]revue.west |- revue_ssp.west)+(-0.3,-0.4)$)
    node[midway,xshift=-2cm,rotate=0]{\textbf{Partie III}};

    \draw[decorate, decoration={brace, amplitude=20pt}, thick]
    ($([yshift=-5pt]revue.west |- resultats.west)+(-0.3,-0.4)$) -- ($([yshift=5pt]revue.west |- hypothese_ssp.west)+(-0.3,-0.4)$)
    node[midway,xshift=-2cm,rotate=0]{\textbf{Partie IV}};

    \draw[decorate, decoration={brace, amplitude=20pt}, thick]
    ($([yshift=-5pt]revue.west |- conclusion.west)+(-0.3,-0.4)$) -- ($([yshift=5pt]revue.west |- resultats.west)+(-3,-0.4)$)
    node[midway,xshift=-2cm,rotate=0]{\textbf{Partie V}};

\end{tikzpicture}
  }
  \caption{Diagram of the underlying logic of our reasoning behind the organization of the manuscript}
  \label{fig:manuscript_logic}
\end{figure}
\noindent
The \fullpartref{part:context}{Working Context} introduces the general research question by highlighting the limitations of existing approaches to designing MAS dedicated to cyber defense. It thus motivates the use of a hybrid approach, combining symbolic models and connectionist techniques, and specifies the research question within a constrained optimization framework. This section concludes with a presentation of the hypotheses that will structure the entire study.
%
Based on these hypotheses, the \fullpartref{part:state_of_the_art}{State of the Art} establishes the necessary theoretical foundations and identifies, for each of them, the scientific obstacles that will need to be overcome. It thus constitutes the conceptual basis on which the proposed method will be based.
%
Continuing on from this, the \fullpartref{part:methode}{MAMAD method} describes the \acn{MAMAD} method, designed to answer the research question. It details the four key activities and explains the specific contributions associated with each one.
%
To evaluate the method and validate the underlying hypotheses, the \fullpartref{part:experimentation}{Experimental validation of the method} presents the experimental protocol that was implemented. It describes the environments tested, the evaluation objectives, the metrics used, as well as the results obtained and their comparative analysis.
%
Finally, the \fullpartref{part:conclusion}{Conclusion} provides a summary of the thesis's contributions. It discusses the limitations encountered and outlines several perspectives for extending this work.
\begin{figure}[h!]
  \centering
  \resizebox{\textwidth}{!}{%
    \resizebox{\textwidth}{!}{%
  \begin{tikzpicture}[
    chapter/.style = {draw, thick, fill=blue!10, minimum width=8cm, align=left, font=\normalsize},
    arrow/.style = {-{Latex[round]}, thick},
    partlabel/.style = {rotate=90, font=\bfseries, align=center},
    dashedline/.style = {red, thick, dashed},
    node distance=0.9cm and 2cm,
    annotated/.style={above,font=\small\itshape, inner sep=1pt, yshift=5.5mm, xshift=-4cm}
    ]

    % Chapitres principaux (colonne verticale)
    \node[chapter] (c1) {Chapitre 1 : Repenser la Cyberdéfense pour de nouveaux enjeux};
    \node[chapter, below=of c1] (c2) {Chapitre 2 : Vers des SMA de Cyberdéfense et leur conception};
    \node[chapter, below=of c2] (c3) {Chapitre 3 : Un problème d'optimisation pour structurer une méthode};

    \node[chapter, below=of c3] (c4) {Chapitre 4 : Les concepts théoriques mobilisés};
    \node[chapter, below=of c4] (c5) {Chapitre 5 : Les verrous d'une méthode de conception};

    \node[chapter, below=of c5] (c6) {Chapitre 6 : Présentation globale de la méthode};

    % Flèches verticales principales avec annotations
    \draw[arrow] (c1) -- (c2) node[annotated, xshift=-1.5cm] {Limites des approches de Cyberdéfense et question globale à adresser};
    \draw[arrow] (c2) -- (c3) node[annotated] {Connaissance des verrous et hypothèses};
    \draw[arrow] (c3) -- (c4) node[annotated] {Préciser les fondations théoriques nécessaires};
    \draw[arrow] (c4) -- (c5) node[annotated] {Identifier les verrous pour chaque hypothèse};
    \draw[arrow] (c5) -- (c6) node[annotated] {Assembler une méthode pour répondre aux verrous};

    % Branche de droite (activités)
    \node[chapter, below= of c6, xshift=-7cm] (c7) {Chapitre 7 : Modéliser l'environnement simulé};
    \node[chapter, below=of c7] (c8) {Chapitre 8 : Entraînement des politiques sous contraintes};
    \node[chapter, below=of c8] (c9) {Chapitre 9 : Analyser et interpréter les comportements émergents};
    \node[chapter, below=of c9] (c10) {Chapitre 10 : Transférer et superviser en environnement réel};

    % Suite verticale
    \node[chapter, below=7cm of c6, xshift=-7cm] (c11) {Chapitre 11 : CybMASDE : un framework supportant MAMAD};
    \node[chapter, below=of c11, xshift=7cm] (c12) {Chapitre 12 : Protocole expérimental};
    \node[chapter, below=of c12] (c13) {Chapitre 13 : Études de cas};
    \node[chapter, below=of c13] (c14) {Chapitre 14 : Résultats expérimentaux et analyse comparative};
    \node[chapter, below=of c14] (c15) {Chapitre 15 : Synthèse des apports et évaluation de la méthode};

    \node[chapter, below=of c15] (c16) {Chapitre 16 : Perspectives et ouvertures};

    % Arrows
    \foreach \i/\j in {c1/c2, c2/c3, c3/c4, c4/c5, c5/c6, c12/c13, c13/c14, c14/c15, c15/c16}
    \draw[arrow] (\i) -- (\j);

    % Flèches horizontales depuis c6
    \foreach \dest in {c7, c8, c9, c10}
    \draw[arrow] ($ (c6.south) + (-0.35,0) $) -- ++(0,0) |- (\dest.east);

    \draw[arrow] (c10.east) -- ++(1.325,0) -- ($ (c6.south) + (-0.35,0) $);

    \draw[arrow] (c6.south) -- ++(0,0) |- (c11.east);

    \draw[arrow] ($ (c11.south) + (1.5, 0) $) -- ++(0,0) |- (c12.west) node[annotated] {Support protocole expérimental};

    \draw[arrow] (c6.south) -- (c12.north);


    % Flèches suite expérimentale
    \draw[arrow] (c12) -- (c13) node[annotated] {Application sur des scénarios variés};
    \draw[arrow] (c13) -- (c14) node[annotated] {Consolidation et comparaison des résultats};
    \draw[arrow] (c14) -- (c15) node[annotated] {Synthèse des acquis et limites restantes};
    \draw[arrow] (c15) -- (c16) node[annotated] {Perspectives scientifiques et industrielles};


    % Partie labels à gauche
    \draw[decorate, decoration={brace, amplitude=15pt}, thick]
    ($(c9.west |- c3.west)+(-0.3,-0.4)$) -- ($(c9.west |- c1.west)+(-0.3,0.4)$)
    node[midway,xshift=-1.4cm,rotate=0]{\textbf{Partie I}};

    \draw[decorate, decoration={brace, amplitude=15pt}, thick]
    ($(c9.west |- c5.west)+(-0.3,-0.4)$) -- ($(c9.west |- c4.west)+(-0.3,0.4)$)
    node[midway,xshift=-1.4cm,rotate=0]{\textbf{Partie II}};

    \draw[decorate, decoration={brace, amplitude=15pt}, thick]
    ($(c9.west |- c10.west)+(-0.3,-0.4)$) -- ($(c9.west |- c6.west)+(-0.3,0.4)$)
    node[midway,xshift=-1.4cm,rotate=0]{\textbf{Partie III}};

    \draw[decorate, decoration={brace, amplitude=15pt}, thick]
    ($(c9.west |- c14.west)+(-0.3,-0.4)$) -- ($(c9.west |- c11.west)+(-0.3,0.4)$)
    node[midway,xshift=-1.4cm,rotate=0]{\textbf{Partie IV}};

    \draw[decorate, decoration={brace, amplitude=15pt}, thick]
    ($(c9.west |- c16.west)+(-0.3,-0.4)$) -- ($(c9.west |- c15.west)+(-0.3,0.4)$)
    node[midway,xshift=-1.4cm,rotate=0]{\textbf{Partie V}};

  \end{tikzpicture}
}

  }
  \caption{Diagram of the manuscript's organization}
  \label{fig:manuscript_organization}
\end{figure}
\cleardoublepage
