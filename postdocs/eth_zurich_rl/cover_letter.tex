\documentclass[11pt,a4paper,sans]{moderncv}        

\moderncvstyle{banking}
\moderncvcolor{red}                                
\usepackage{dirtytalk}
\usepackage[utf8]{inputenc}
\usepackage[scale=0.85]{geometry}
\usepackage{graphicx}
\usepackage{ragged2e}
\usepackage{setspace}
\usepackage{parskip}
\setstretch{1.05}

\AfterPreamble{
  \hypersetup{
    colorlinks=false,        % désactive les liens colorés (sinon pas de cadre)
    pdfborder={0 0 1},       % active un cadre autour des liens
    pdfborderstyle={/S/U/W 1}, % facultatif : style souligné
    linkbordercolor={0 0 1}, % cadre bleu pour liens internes
    urlbordercolor={0 0 1},  % cadre bleu pour les URL
    citebordercolor={0 0 1}  % cadre bleu pour citations (si utilisées)
  }
}

\name{Julien}{Soulé}
\title{PhD Student -- \textit{Thales/UGA}}
\address{35 Rue Mathieu de la Drôme, Valence}{26000}{France}

\phone[mobile]{06 77 63 12 13}
\email{julien.soule@hotmail.fr}
\homepage{julien6.github.io/home/}

%----------------------------------------------------------------------------------
%            content
%----------------------------------------------------------------------------------
\begin{document}

\recipient{ODI Research Group}{Institute of Machine Learning, ETH Zurich\\Zurich, Switzerland}
\date{\today}
\opening{Dear Members of the ODI Research Group,}
\closing{{I look forward to the opportunity to further discuss my background and research plans in an interview.}\\[0.6cm]Sincerely,\vspace{-0.6cm}}

\makelettertitle

\justifying

\noindent

\noindent
I am writing to express my strong interest in the Postdoctoral Researcher position in Reinforcement Learning at \textit{ETH Zurich}, within the \textit{ODI Research Group}.
My research focuses on the intersection of \textbf{Multi-Agent Systems (MAS)}, \textbf{Multi-Agent Reinforcement Learning (MARL)}, and \textbf{organizational modeling}, with the overarching goal of developing principled frameworks that make multi-agent learning \textbf{interpretable}, \textbf{controllable}, and \textbf{systematically designable}.
Throughout my PhD, I have treated the design of MAS as an optimization problem over a joint policy space constrained by sub-symbolic structures encoding environmental and organizational requirements.
This formal yet practical approach connects with several of the ODI group's research themes, particularly \textbf{theory of RL}, \textbf{learning in games}, and \textbf{representation learning in RL}.

\section*{Background and main contributions}

During my doctoral studies at \textit{Université Grenoble Alpes} (UGA) and \textit{Thales~LAS} (France), I developed a \textbf{methodological framework for assisting the MAS design} we implemented in the \href{https://github.com/julien6/CybMASDE}{\textbf{CybMASDE platform}}.
This framework assists the full lifecycle of MAS design by combining \textit{Dec-POMDP}-based formal modeling, MARL (using \textit{Gym}/\textit{PettingZoo}, \textit{MARLlib}, and \textit{JAX}/\textit{JaxMARL}), and unsupervised behavior analysis.
The aim is to assist designers in specifying, training, and interpreting MASs under structured constraints, bridging symbolic organizational design and MARL.

My work could be organized around the following themes:

\begin{itemize}
      \item \textbf{MARL-assisted MAS design.}
            I proposed a methodological framework that formulates MAS design as a constrained policy-optimization problem and supports iterative design--train--analyze--transfer cycles \href{https://link.springer.com/chapter/10.1007/978-3-031-63223-5_24}{\textbf{(Soulé \textit{et al.}, AIAI~2024)}}; \href{https://sciety-labs.elifesciences.org/articles/by?article_doi=10.21203/rs.3.rs-7166037/v1}{\textbf{(JAAMAS~2025, under revision)}}.

      \item \textbf{Organization-guided MARL.}
            I designed an \href{https://github.com/julien6/MOISE-MARL}{\textbf{organizational-oriented MARL framework}} integrating symbolic constraints such as roles and goals into MARL.
            This improves convergence, addresses non-stationarity, and enhances safety and explainability in complex environments \href{https://arxiv.org/abs/2503.23615}{\textbf{(AAMAS~2025)}}; \href{https://arxiv.org/abs/2505.21559}{\textbf{(IEEE CLOUD~2025)}}.

      \item \textbf{World-model-based multi-agent learning.}
            I proposed \textbf{World Models}~\href{https://link.springer.com/chapter/10.1007/978-3-031-63223-5_24}{(Ha \textit{et al.}, 2018)} for multi-agent settings to be used as digital twins that approximate environmental dynamics (using \textit{PyTorch} and \textit{scikit-learn}).
            These models enable efficient model-based MARL and representation learning while exposing open challenges such as model brittleness and interpretability, which I began to formalize theoretically \href{https://sciety-labs.elifesciences.org/articles/by?article_doi=10.21203/rs.3.rs-7166037/v1}{\textbf{(JAAMAS~2025, under revision)}}; \href{https://arxiv.org/abs/2505.21559}{\textbf{(IEEE CLOUD~2025)}}.

      \item \textbf{Behavior analysis and explainability.}
            I proposed an unsupervised analysis method that extracts implicit roles and objectives from agent trajectories, offering organizational insight into emergent behaviors and contributing to \textbf{explainable MARL} \href{https://arxiv.org/abs/2503.23615}{\textbf{(AAMAS~2025)}}.
\end{itemize}

Together, these contributions form a methodological continuum (from \textbf{modeling} to \textbf{training}, \textbf{analysis} and \textbf{transferring}) providing a structured process for developing, training, and interpreting MASs in complex environments.

\section*{Research vision and future directions}


Building on these foundations, I intend my postdoctoral research to pursue a methodological goal with a focus on \textbf{learning in games}, \textbf{MARL}, and \textbf{representation learning in RL}, possibly leveraging \textbf{LLM-based approaches}.


\begin{enumerate}
      \item \textbf{Theoretical RL/MARL with organizational constraints.}
            I aim to establish a formal understanding of how organizational structures can stabilize and guide multi-agent learning.
            My work will study convergence properties and stability guarantees of organization-guided MARL under \emph{Dec-POMDP} settings, contributing to the broader \emph{theory of online and RL}.
            I will analyze how structured priors (such as hierarchical roles or shared subgoals) affect equilibrium selection, exploration efficiency, and non-stationarity.
            I also plan to explore \emph{Lagrangian-based} and constrained policy-optimization methods to ensure consistent satisfaction of symbolic and safety constraints, possibly enabling deriving theoretical guarantees for coordination and fairness.

      \item \textbf{Representation learning for multi-agent World Models.}
            I intend to develop \textbf{neuro-symbolic World Models} that combine learned dynamics with structured, interpretable representations.
            These models will factorize multi-agent environments into entities, relations, and causal dependencies using \emph{graph-based encodings} or symbolic reasoning.
            The objective is to obtain semantically grounded models that generalize across tasks and support reasoning, transfer, and verification.
            Such representations can act as abstractions linking low-level observations to high-level organizational constraints, paving the way for explainable and robust decision-making.
            This aligns with the ODI group’s work on \emph{representation learning} and \emph{data-driven control}.

      \item \textbf{Bridging simulation and physical environments.}
            Finally, I plan to extend these frameworks to real-world multi-robot and industrial scenarios, where agents must coordinate under uncertainty to optimize shared processes.
            These experiments will serve both as empirical validation and as a testbed for deploying safe, adaptive, and interpretable MARL in embodied systems.
\end{enumerate}


\section*{Collaboration and mentoring}

My research has always been conducted in a collaborative, interdisciplinary environment.
The industrial dimension of my PhD at \textit{Thales LAS} exposed me to applied research in Cyberdefense and network security (e.g., ad-hoc drone networks, enterprise infrastructures, microservice architectures).
These collaborations strengthened my ability to translate theoretical advances into practical, high-impact applications.
I also participated with \textit{Thales Germany} researchers on behavior-analysis methods for MARL agents, leading to an ongoing joint publication.

In parallel, I have supervised several Master's and engineering students at \emph{Grenoble INP--Esisar} on exploratory topics such as blockchain and anomaly detection with machine learning, which strengthened my mentoring experience and my motivation to contribute to team-based academic projects.

\section*{Long-term perspective}

My long-term objective is to pursue an academic career combining \textbf{theoretical RL/MARL} and \textbf{methodological development for MASs}, while maintaining strong connections with industrial and interdisciplinary applications.
I am particularly motivated by the prospect of working in an international environment like ETH Zurich, where theoretical research meets methodological innovation and real-world impact.

\vspace{0.5em}
\noindent
I believe that my background in MARL, formal modeling, and explainability provides a perspective that aligns with the ODI group's objectives.
I am eager to contribute to the ongoing research on the theory and foundations of MARL/RL, while developing new directions that connect symbolic structure, learning dynamics, and interpretability in complex systems.


\makeletterclosing

\end{document}
