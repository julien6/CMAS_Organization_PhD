\documentclass[11pt,a4paper]{article}

% ------------------------------------------------------------
% Packages and Formatting
% ------------------------------------------------------------
\usepackage[margin=2.2cm]{geometry}
\usepackage{setspace}
\usepackage{parskip}
\usepackage{hyperref}
\usepackage{enumitem}
\usepackage{titlesec}
\usepackage{url}
\usepackage{lmodern}

\setstretch{1.05}

\titleformat{\section}
  {\large\bfseries}
  {}{0pt}{}

\titleformat{\subsection}
  {\normalsize\bfseries}
  {}{0pt}{}

\hypersetup{
    colorlinks=true,
    urlcolor=blue,
    linkcolor=black,
    citecolor=black
}

% ------------------------------------------------------------
% DOCUMENT
% ------------------------------------------------------------
\begin{document}

% ============================================================
% Header
% ============================================================
{\LARGE \textbf{Julien Soulé}}\\[6pt]
PhD Candidate --- Reinforcement Learning, Multi-Agent Systems \\
Université Grenoble Alpes (UGA) \& Thales LAS (France) \\[4pt]

\href{mailto:julien.soule@univ-grenoble-alpes.fr}{julien.soule@univ-grenoble-alpes.fr} \\
+33 6 77 63 12 13 \\
Valence, France \\
\href{https://julien6.github.io/home/}{https://julien6.github.io/home/}

\vspace{0.3cm}

% ============================================================
\section*{Research Interests}
% ============================================================
Reinforcement Learning (RL), Multi-Agent Reinforcement Learning (MARL),
Digital Twins and Model-Based Simulation, Multi-Robot Coordination,
Explainable AI for Multi-Agent Systems, Organizational Modeling.

% ============================================================
\section*{Education}
% ============================================================

\textbf{PhD in Computer Science} \hfill \textit{2022 -- present} \\
Université Grenoble Alpes (UGA), France \\
\emph{“On the Organization of a Cyberdefence Multi-Agent System”} \\
Supervisors: J.-P. Jamont, M. Occello, L.-M. Traonouez (Thales LAS), P. Théron (AICA IWG)

\vspace{0.15cm}

\textbf{Master’s Degree in Computer Engineering} \hfill \textit{2015 -- 2020} \\
INSA Rennes, France

\vspace{0.15cm}

\textbf{Exchange Semester in Computer Engineering} \hfill \textit{Jan 2019 -- May 2019} \\
ÉTS Montréal, Canada

% ============================================================
\section*{Research Experience}
% ============================================================

\textbf{Research Engineer} --- Thales LAS, Rennes \hfill \textit{Dec 2021 -- Jun 2022} \\
Developed multi-agent modeling and simulation tools for cyber-physical systems.
Explored RL-based anomaly detection and coordination strategies for distributed agents.

\vspace{0.12cm}

\textbf{Software Engineer} --- Atos, Toulouse \hfill \textit{Aug 2020 -- Sep 2021} \\
Worked on the ISIS satellite command \& control system for CNES.
Technologies: Python, Bash, KVM, Grafana, Django.

\vspace{0.12cm}

\textbf{Software Engineering Intern} --- Atos, Toulouse \hfill \textit{Jan 2020 -- Jul 2020} \\
Developed modules for satellite launch supervision and distributed control pipelines.

\vspace{0.12cm}

\textbf{Software Engineering Intern} --- SQLI, Toulouse \hfill \textit{May 2019 -- Jul 2019} \\
Engineering contributions to Airbus Helicopters workflows (security, maintenance, software development).

% ============================================================
\section*{Selected Research Projects}
% ============================================================

\textbf{CybMASDE Platform} (UGA \& Thales LAS) \\
Framework combining Dec-POMDP modeling, multi-agent RL, digital-twin simulation,
and behavior analysis for structured multi-agent system design.

\vspace{0.1cm}

\textbf{MOISE+MARL Framework} \\
Integration of symbolic roles, missions, and goals into MARL to improve safety,
explainability, and stability of learned behaviors.

% ============================================================
\section*{Publications}
% ============================================================

\subsection*{Journal Articles}
Soulé, J., Jamont, J.-P., Occello, M., Traonouez, L.-M., \& Théron, P. (2025).
\emph{Assisting Multi-Agent System Design with MOISE+ and MARL: The MAMAD Method}.
Journal of Autonomous Agents and Multi-Agent Systems (JAAMAS), under revision.
\href{https://sciety-labs.elifesciences.org/articles/by?article_doi=10.21203/rs.3.rs-7166037/v1}{Link}

\subsection*{International Conferences}
Soulé, J., et al. (2025). \emph{Streamlining Resilient Kubernetes Autoscaling with Multi-Agent Systems via an Automated Online Design Framework}.
IEEE CLOUD 2025. \href{https://arxiv.org/abs/2505.21559}{Link}

Soulé, J., et al. (2025). \emph{An Organizationally-Oriented Approach to Enhancing Explainability and Control in MARL}.
AAMAS 2025. \href{https://arxiv.org/abs/2503.23615}{Link}

Soulé, J., et al. (2024). \emph{A MARL-based Approach for Easing MAS Organization Engineering}.
AIAI 2024. \href{https://link.springer.com/chapter/10.1007/978-3-031-63223-5_24}{Link}

Soulé, J., et al. (2023). \emph{Towards a Multi-Agent Simulation of Cyber-Attackers and Cyber-Defenders Battles}.
IEEE SMC 2023.

\subsection*{National Conferences (Selected)}
(*Best Paper Award — JFSMA 2025*)
Multiple contributions on MARL, MAS organization modeling, and explainability.

% ============================================================
\section*{Talks \& Presentations}
% ============================================================
\begin{itemize}[itemsep=2pt]
    \item Poster Presentation, JFSMA 2023.
    \item Invited Talk: CybAIR NATO Chair, École de l’Air et de l’Espace (Mar 2023).
\end{itemize}

% ============================================================
\section*{Academic Service}
% ============================================================

\textbf{Treasurer}, Autonomous Intelligent Cyberdefence Agent (AICA) International Work Group
\hfill \textit{May 2023 -- present}

% ============================================================
\section*{Teaching Experience}
% ============================================================

\textbf{Teaching Assistant}, Grenoble INP / IUT Valence \hfill \textit{2023--2025} \\
Supervised team-based cyberdefense projects.
Taught Operating Systems, System Programming, Administration, Process Management.

% ============================================================
\section*{Technical Skills}
% ============================================================

\textbf{Machine Learning / RL:} PyTorch, JAX, JaxMARL, MARLlib, Gym, PettingZoo \\
\textbf{Simulation / Modeling:} Dec-POMDPs, World Models, digital twins \\
\textbf{Software Engineering:} Python, Linux, Docker, Git, KVM, distributed systems \\
\textbf{Languages:} French (native), English (professional, TOEIC 910), Japanese (basic)

% ============================================================
\section*{References}
Available upon request.

\end{document}
