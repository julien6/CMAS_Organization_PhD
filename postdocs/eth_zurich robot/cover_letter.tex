\documentclass[11pt,a4paper,sans]{moderncv}        

\moderncvstyle{banking}
\moderncvcolor{red}                                
\usepackage{dirtytalk}
\usepackage[utf8]{inputenc}
\usepackage[scale=0.85]{geometry}
\usepackage{graphicx}
\usepackage{ragged2e}
\usepackage{setspace}
\usepackage{parskip}
\setstretch{1.05}

\AfterPreamble{
  \hypersetup{
    colorlinks=false,
    pdfborder={0 0 1},
    pdfborderstyle={/S/U/W 1},
    linkbordercolor={0 0 1},
    urlbordercolor={0 0 1},
    citebordercolor={0 0 1}
  }
}

\name{Julien}{Soulé}
\title{Postdoctoral Researcher Application -- RobotX, ETH Zürich}
\address{35 Rue Mathieu de la Drôme, Valence}{26000}{France}

\phone[mobile]{+33 6 77 63 12 13}
\email{julien.soule@hotmail.fr}
\homepage{julien6.github.io/home/}

\begin{document}

\recipient{RobotX Research Center}{ETH Zurich\\Zurich, Switzerland}
\date{\today}
\opening{Dear RobotX Selection Committee,}
\closing{{I would be delighted to further discuss how my background and research vision connect with the activities of RobotX.}\\[0.6cm]Sincerely,\vspace{-0.6cm}}

\makelettertitle

\justifying

I am writing to express my strong interest in the Postdoctoral Researcher position at the \emph{RobotX Research Center}, ETH Zurich.
My research bridges \textbf{Reinforcement Learning (RL)}, \textbf{Multi-Agent Systems (MAS)}, and \textbf{simulation-based modeling}, with the broader goal of designing learning systems that are robust, interpretable, and suitable for deployment in real-world robotics.
During my PhD at \textit{Université Grenoble Alpes} (UGA) and \textit{Thales LAS}, I developed methodological and algorithmic tools that unify symbolic structures with learning-based control, enabling scalable and explainable adaptive behaviours.
RobotX’s mission (advancing embodied intelligence through RL, simulation, and predictive models) resonates with my expertise. In particular, the ongoing projects on \textbf{humanoid locomotion}, \textbf{digital twins}, and \textbf{data-driven control pipelines} align with my research vision.

\section*{Background and main contributions}

My doctoral research proposes a structured workflow for designing, training, and analyzing multi-agent behaviors under uncertainty and partial observability.
This methodology materialized in the \href{https://github.com/julien6/CybMASDE}{\textbf{CybMASDE platform}}, which integrates \textbf{Dec-POMDP} modeling, multi-agent RL (via \textit{Gym}, \textit{PettingZoo}, \textit{MARLlib}, and \textit{JaxMARL}), digital-twin-style simulation, and behavior-analysis tools.
These contributions are directly relevant to multi-robot coordination, locomotion learning, and simulation-driven control pipelines.

My main research contributions include:

\begin{itemize}
      \item \textbf{MARL-assisted system design.}
            I formulated MAS design as a constrained policy-optimization problem, enabling structured design-train-analyze-transfer cycles.
            This work appears in \href{https://link.springer.com/chapter/10.1007/978-3-031-63223-5_24}{\textbf{AIAI~2024}} and is expanded in \href{https://sciety-labs.elifesciences.org/articles/by?article_doi=10.21203/rs.3.rs-7166037/v1}{\textbf{JAAMAS~2025 (under revision)}}.
            The methodology scales to robotics scenarios where coordination, safety, and structure must be embedded into policy learning.

      \item \textbf{Organization-guided MARL.}
            I developed an \href{https://github.com/julien6/MOISE-MARL}{\textbf{organization-oriented MARL framework}} that incorporates symbolic constraints (roles, sub-goals, norms) into decentralized learning.
            This reduces non-stationarity, improves learning stability, and yields interpretable coordination patterns.
            Results are presented in \href{https://arxiv.org/abs/2503.23615}{\textbf{AAMAS~2025}} and \href{https://arxiv.org/abs/2505.21559}{\textbf{IEEE CLOUD~2025}}.
            Such structured priors can naturally encode robot morphology constraints, locomotion phases, or manipulation synergies.

      \item \textbf{World-model-based multi-agent learning and digital twins.}
            I extended \textbf{World Models} to multi-agent settings by learning latent generative models for both environmental and inter-agent dynamics.
            Implemented with \textit{PyTorch} and \textit{scikit-learn}, these models support long-horizon prediction, uncertainty-aware planning, and sample-efficient model-based RL.
            Their analysis appears in \href{https://sciety-labs.elifesciences.org/articles/by?article_doi=10.21203/rs.3.rs-7166037/v1}{\textbf{JAAMAS~2025}} and \href{https://arxiv.org/abs/2505.21559}{\textbf{IEEE CLOUD~2025}}.
            These digital twins are directly applicable to sim-to-real locomotion, motion planning, and adaptive control.

      \item \textbf{Behavior analysis and explainability.}
            I designed unsupervised models to extract implicit roles, coordination patterns, and subgoals from agent trajectories.
            Presented in \href{https://arxiv.org/abs/2503.23615}{\textbf{AAMAS~2025}}, these tools reinforce trust, safety, and interpretability that are core requirements in robotics.
\end{itemize}

\section*{Research vision within RobotX}

RobotX’s priorities (learning locomotion and manipulation, sim-to-real transfer, digital-twin simulation, and adaptive multi-robot behavior) align closely with my research trajectory.

\textbf{I am particularly motivated by the Advanced Humanoid Locomotion (AHL) project}, which aims to endow bipedal robots with robust locomotion across complex terrains using RL, MPC-guided learning, and perceptual feedback.
My work on world-model-based learning and structured MARL offers a complementary perspective:
\begin{itemize}
      \item latent-dynamics models can act as \textbf{predictive simulators} for foothold selection, stability evaluation, and adaptive gait generation;
      \item structure-guided learning can encode \textbf{coordination constraints}, safety rules, and multi-limb synergies directly into the policy space;
      \item unsupervised behavior analysis can serve as a \textbf{diagnostic tool} for understanding failure modes, emergent locomotion strategies, and policy robustness.
\end{itemize}
I would be enthusiastic to collaborate with RSL and CRL teams to extend these ideas and contribute to next-generation locomotion controllers.

Beyond AHL, my broader research vision includes:

\begin{enumerate}
      \item \textbf{Reinforcement Learning for cooperative and adaptive robotics.}
            Designing scalable RL algorithms that handle partial observability, noisy sensing, heterogeneous robot morphologies, and dynamic coordination.

      \item \textbf{Digital twins and simulation-driven robotics.}
            Building predictive models tailored to robotic embodiment to accelerate learning, planning, and sim-to-real transfer.

      \item \textbf{Embodied experimentation for reliable robot learning.}
            Deploying controllers on legged robots, manipulators, or aerial platforms to evaluate robustness and adaptivity in real-world conditions.
\end{enumerate}

\section*{Collaboration and mentoring}

My PhD involved extensive collaboration with industrial research teams at \textit{Thales France} and \textit{Thales Germany}, where I contributed to cooperative decision-making in drone networks and cyber-physical infrastructures.
These collaborations strengthened my ability to build prototypes under real-world constraints and coordinate with interdisciplinary teams.
I have also supervised several Master's and engineering students, mentoring them in machine learning, multi-agent systems, and robotics-inspired projects, an experience I am eager to expand in an academic setting.

\section*{Long-term perspective}

My long-term goal is to pursue an academic career focused on \textbf{adaptive, explainable, and safe multi-agent robotic systems}, unifying RL, predictive modeling, and structured coordination through a \textbf{neuro-symbolic} approach.
ETH Zurich’s robotics ecosystem offers an exceptional environment for advancing such research and for bridging the gap between theory, simulation, and embodied intelligence.

\vspace{0.5em}
\noindent
I would be excited to bring my background in RL, digital twins, and multi-agent learning to RobotX and to contribute to its ambitious research agenda.

\makeletterclosing

\end{document}
