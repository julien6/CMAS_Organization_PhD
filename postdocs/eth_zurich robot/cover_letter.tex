\documentclass[11pt,a4paper,sans]{moderncv}        

\moderncvstyle{banking}
\moderncvcolor{red}                                
\usepackage{dirtytalk}
\usepackage[utf8]{inputenc}
\usepackage[scale=0.85]{geometry}
\usepackage{graphicx}
\usepackage{ragged2e}
\usepackage{setspace}
\usepackage{parskip}
\setstretch{1.05}

\AfterPreamble{
  \hypersetup{
    colorlinks=false,        % désactive les liens colorés (sinon pas de cadre)
    pdfborder={0 0 1},       % active un cadre autour des liens
    pdfborderstyle={/S/U/W 1}, % facultatif : style souligné
    linkbordercolor={0 0 1}, % cadre bleu pour liens internes
    urlbordercolor={0 0 1},  % cadre bleu pour les URL
    citebordercolor={0 0 1}  % cadre bleu pour citations (si utilisées)
  }
}

\name{Julien}{Soulé}
\title{PhD Student -- \textit{Thales/UGA}}
\address{35 Rue Mathieu de la Drôme, Valence}{26000}{France}

\phone[mobile]{06 77 63 12 13}
\email{julien.soule@hotmail.fr}
\homepage{julien6.github.io/home/}

%----------------------------------------------------------------------------------
%            content
%----------------------------------------------------------------------------------
\begin{document}

\recipient{RobotX Research Center}{ETH Zurich\\Zurich, Switzerland}
\date{\today}
\opening{Dear RobotX Selection Committee,}
\closing{{I would be delighted to further discuss how my background and research vision connect with the activities of RobotX.}\\[0.6cm]Sincerely,\vspace{-0.6cm}}

\makelettertitle

\justifying

I am writing to express my strong interest in the RobotX Postdoctoral Position at \textit{ETH Zurich}.
My research focuses on the intersection of \textbf{Reinforcement Learning (RL)}, \textbf{multi-agent systems}, and \textbf{simulation-based modeling}.
During my PhD at \emph{Université Grenoble Alpes} (UGA) and \emph{Thales LAS}, I developed methodological and algorithmic tools to make multi-agent learning \textbf{more reliable, interpretable, and deployable in real-world systems}.
The scientific challenges central to modern robotics—coordination, uncertainty management, safety, and the integration of learning with physical constraints—closely match the problems I have been addressing throughout my doctoral work.
I am highly motivated to contribute these perspectives to the multidisciplinary environment of RobotX.

\section*{Background and main contributions}

My PhD introduces a structured workflow for designing, training, and analyzing multi-agent behaviors using RL, formal modeling, and unsupervised analysis.
We implemented this workflow in the \href{https://github.com/julien6/CybMASDE}{\textbf{CybMASDE platform}}, which supports \textit{Dec-POMDP} modeling, multi-agent RL (based on \textit{Gym}/\textit{PettingZoo}, \textit{MARLlib}, and \textit{JaxMARL}), and post-hoc behavior analysis.

My main contributions—each of which connects naturally to robotics—are organized as follows:

\begin{itemize}

      \item \textbf{MARL-assisted MAS design.}
            I proposed a methodological framework that formulates MAS design as a constrained policy-optimization problem and supports iterative design--train--analyze--transfer cycles.
            This work provides a principled foundation for scalable multi-robot coordination and appears in
            \href{https://link.springer.com/chapter/10.1007/978-3-031-63223-5_24}{\textbf{(Soulé \textit{et al.}, AIAI~2024)}}
            and
            \href{https://sciety-labs.elifesciences.org/articles/by?article_doi=10.21203/rs.3.rs-7166037/v1}{\textbf{(JAAMAS~2025, under revision)}}.

      \item \textbf{Organization-guided MARL for safer and more predictable learning.}
            I designed an \href{https://github.com/julien6/MOISE-MARL}{\textbf{organizational-oriented MARL framework}} that integrates symbolic constraints such as roles and shared goals.
            These constraints improve convergence, structure coordination patterns, and enhance safety and explainability in complex environments—properties essential for embodied multi-robot systems.
            This line of work appears in
            \href{https://arxiv.org/abs/2503.23615}{\textbf{(AAMAS~2025)}}
            and
            \href{https://arxiv.org/abs/2505.21559}{\textbf{(IEEE CLOUD~2025)}}.

      \item \textbf{Digital twins via multi-agent World Models.}
            I extended \textbf{World Models} \href{https://link.springer.com/chapter/10.1007/978-3-031-63223-5_24}{(Ha \textit{et al.}, 2018)} to multi-agent settings as learned simulators of environmental dynamics, implemented using \textit{PyTorch} and \textit{scikit-learn}.
            These digital twins support model-based RL, fast experimentation, and representation learning, while also revealing challenges such as brittleness and interpretability.
            This research is further detailed in
            \href{https://sciety-labs.elifesciences.org/articles/by?article_doi=10.21203/rs.3.rs-7166037/v1}{\textbf{(JAAMAS~2025, under revision)}}
            and
            \href{https://arxiv.org/abs/2505.21559}{\textbf{(IEEE CLOUD~2025)}}.

      \item \textbf{Behavior analysis and explainability in MARL.}
            I proposed unsupervised methods that extract implicit roles and coordination patterns from agent trajectories, supporting debugging, validation, and interpretability—critical components when deploying RL-driven behaviors to real robots.
            This contribution is presented in
            \href{https://arxiv.org/abs/2503.23615}{\textbf{(AAMAS~2025)}}.

\end{itemize}

Together, these contributions outline a unified methodology for structured multi-agent learning, from \textbf{modeling} to \textbf{training}, \textbf{analysis}, and \textbf{transfer}.
They are directly relevant to the design of safe, adaptive, and scalable multi-robot systems.

\section*{Research vision within RobotX}

I am enthusiastic about conducting postdoctoral research that advances the design and deployment of learning-enabled robotic systems.
Within RobotX, I aim to pursue three complementary research directions:

\begin{enumerate}

      \item \textbf{Reinforcement Learning for multi-robot coordination.}
            I intend to study RL and MARL algorithms for cooperative control under constraints such as limited communication, uncertainty, and the need for safe interactions.
            My background in integrating symbolic structure into learning provides a foundation for improving stability and predictability in multi-robot behaviors.

      \item \textbf{Simulation and digital twins for robotics.}
            Building on my work with multi-agent World Models, I plan to develop learned simulators that accelerate policy training, transfer, and validation.
            These digital twins would serve as efficient testbeds for real-robot experiments and complement physics-based simulation pipelines.

      \item \textbf{Deployment of learning systems in embodied platforms.}
            I am particularly motivated by applying RL-driven coordination tools to physical robots—whether aerial, legged, or manipulator-based.
            Such deployments would provide both empirical grounding for algorithmic research and insight into the challenges of robustness, uncertainty, and adaptation in real-world settings.

\end{enumerate}

These themes resonate strongly with RobotX's missions in \textbf{RL/ML for robotics}, \textbf{multi-robot coordination}, \textbf{computational design}, and \textbf{simulation-driven development}.

\section*{Collaboration and mentoring}

My doctoral research was conducted in a collaborative environment involving \textit{Thales LAS}, \textit{Thales Germany}, and several academic partners.
This experience strengthened my ability to work across disciplinary boundaries and to translate theoretical work into applied prototypes—ranging from cyber-physical infrastructures to networked drone simulations.
I also supervised multiple Master's and engineering students, which reinforced my motivation to contribute to collaborative research initiatives and mentoring activities.

\section*{Long-term perspective}

My long-term objective is to pursue a research career centered on \textbf{learning, coordination, and adaptive behavior in multi-agent and robotic systems}.
The interdisciplinary nature of RobotX—combining robotics, simulation, control, and machine learning—makes it an ideal environment to pursue this vision.
I am highly motivated by the prospect of contributing to ETH Zurich’s efforts in building the next generation of intelligent, safe, and adaptive robotic systems.

\vspace{0.5em}
\noindent
I would be excited to bring my experience in multi-agent RL, digital twins, and structured coordination to RobotX and to collaborate on innovative research at the interface of robotics and learning.

\makeletterclosing

\end{document}