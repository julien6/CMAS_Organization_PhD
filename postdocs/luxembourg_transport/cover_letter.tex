\documentclass[11pt,a4paper,sans]{moderncv}        

\moderncvstyle{banking}
\moderncvcolor{red}                                
\usepackage{dirtytalk}
\usepackage[utf8]{inputenc}
\usepackage[scale=0.85]{geometry}
\usepackage{graphicx}
\usepackage{ragged2e}
\usepackage{setspace}
\usepackage{parskip}
\setstretch{1.05}

\AfterPreamble{
  \hypersetup{
    colorlinks=false,        % désactive les liens colorés (sinon pas de cadre)
    pdfborder={0 0 1},       % active un cadre autour des liens
    pdfborderstyle={/S/U/W 1}, % facultatif : style souligné
    linkbordercolor={0 0 1}, % cadre bleu pour liens internes
    urlbordercolor={0 0 1},  % cadre bleu pour les URL
    citebordercolor={0 0 1}  % cadre bleu pour citations (si utilisées)
  }
}

\name{Julien}{Soulé}
\title{PhD Student -- \textit{Thales/UGA}}
\address{35 Rue Mathieu de la Drôme, Valence}{26000}{France}

\phone[mobile]{06 77 63 12 13}
\email{julien.soule@hotmail.fr}
\homepage{julien6.github.io/home/}

%----------------------------------------------------------------------------------
%            content
%----------------------------------------------------------------------------------
\begin{document}

\recipient{Mobilab Transport Research Group}{Faculty of Science, Technology and Medicine (FSTM)\\University of Luxembourg\\Belval Campus, Luxembourg}
\date{\today}
\opening{Dear Members of the Mobilab Transport Research group,}
\closing{{I look forward to the opportunity to further discuss my background and research ideas in an interview.}\\[0.6cm]Sincerely,\\[0.2cm]Julien Soulé\vspace{-0.6cm}}

\makelettertitle

\justifying

\noindent
I am writing to express my strong interest in the \textit{Postdoctoral Researcher in Transport Modelling and Simulation} position (Ref. UOL07796) within the \textit{Mobilab Transport Research} group at the University of Luxembourg. My research lies at the intersection of \textbf{multi-agent systems (MAS)}, \textbf{multi-agent reinforcement learning (MARL)}, \textbf{organizational modelling}, and \textbf{simulation-based decision support}. Throughout my PhD, I have treated the design of complex multi-agent systems as an optimization problem over a joint policy space constrained by structural requirements, with the aim of making multi-agent learning \textbf{interpretable}, \textbf{controllable}, and \textbf{systematically designable}. I am particularly enthusiastic about bringing these ideas to the domains of \textbf{transport modelling}, \textbf{multi-scale and multi-modal mobility}, and \textbf{data-driven transport simulation} developed at Mobilab.

\section*{Background and main contributions}

During my doctoral studies at \textit{Université Grenoble Alpes} in collaboration with \textit{Thales LAS} (France, CIFRE contract), I developed the \textbf{MAMAD} (\textit{MOISE+MARL Assisted Multi-Agent system Development}) methodology and its implementation in the \href{https://github.com/julien6/CybMASDE}{\textbf{CybMASDE platform}}. This framework supports the full lifecycle of MAS design by combining:
\begin{itemize}
      \item \textit{Dec-POMDP}-based formal modelling of multi-agent environments;
      \item MARL algorithms (via \textit{Gym}/\textit{PettingZoo}, \textit{MARLlib}, \textit{JAX}/\textit{JaxMARL});
      \item organizational specifications (roles, missions, goals) inspired by the MOISE+ model;
      \item and unsupervised behaviour analysis for explainability.
\end{itemize}
The key idea is to explore policy spaces under \textbf{symbolic and sub-symbolic constraints}, thereby structuring learning processes and providing designers with meaningful levers to guide emergent behaviour.

From a methodological standpoint, my work has led to three main contributions that I believe are directly relevant to transport modelling and simulation:

\begin{itemize}
      \item \textbf{Organization-guided MARL.}
            I have proposed an \href{https://github.com/julien6/MOISE-MARL}{\textbf{organization-oriented MARL framework}} in which agents learn under hard constraints (roles, admissible actions) and soft constraints (subgoals, priorities, reward shaping). This improves convergence and coordination in complex, non-stationary environments. In a transport context, such structures can naturally encode \textit{traffic regulations}, \textit{service policies}, or \textit{operational constraints} for different stakeholders (travellers, operators, infrastructure managers).

      \item \textbf{World-model-based multi-agent learning and digital twins.}
            I have explored \textbf{World Models} for multi-agent settings, using neural and statistical models to approximate the dynamics of complex systems and to build \textit{digital twins} that support model-based MARL and counterfactual analysis. This perspective is highly compatible with \textbf{multi-scale transport simulations}, where data-driven surrogate models can accelerate scenario evaluation, sensitivity analysis, and optimisation.

      \item \textbf{Behaviour analysis and explainability.}
            I have developed unsupervised methods that extract implicit roles, strategies, and objectives from the trajectories of trained agents, providing organizational and behavioural insight into emergent dynamics. For transport applications, such tools could help to analyse \textit{route choice}, \textit{activity patterns}, or \textit{network usage} emerging from agent-based models and to relate them back to economic or policy assumptions.
\end{itemize}

Together, these contributions form a coherent methodological continuum—from \textbf{formal modelling}, through \textbf{learning and simulation}, to \textbf{behaviour analysis and interpretation}. I am convinced that this continuum can be fruitfully transposed to transport systems, where agent-based and activity-based models must reconcile behavioural realism, computational tractability, and policy relevance.

\section*{Relevance to transport modelling and Mobilab}

The Postdoctoral position at Mobilab emphasises the development of \textbf{multi-scale and multi-modal models}, the integration of \textbf{transport economics} with \textbf{network modelling} and \textbf{activity-based modelling}, and the exploitation of heterogeneous (big) data sources for \textbf{dynamic transport modelling}, \textbf{demand forecasting}, \textbf{agent-based simulation}, and \textbf{transportation planning}. In this context, I see several natural intersections with my expertise:

\begin{itemize}
      \item \textbf{Agent-based and activity-based modelling.}
            My experience with MAS organizations provides a principled way to structure agent roles (e.g., travellers, operators, logistics providers) and to represent activity chains, service schedules, and operational rules as organizational constraints. This can complement activity-based transport models by offering explicit coordination structures and mechanisms for representing multi-actor interactions.

      \item \textbf{Dynamic transport simulation and network design.}
            The tools I developed for MARL and simulation can be adapted to model congestion dynamics, routing decisions, and control policies (e.g., traffic signal control, fleet management, shared mobility services). Learning-based approaches can be used either to design new control policies or to approximate equilibrium conditions under complex behavioural assumptions.

      \item \textbf{Data-driven surrogate modelling and digital twins of mobility systems.}
            My work on world models and digital twins can support the construction of data-driven surrogates for computationally expensive transport simulations, using observational and sensor data to learn approximate dynamics at multiple scales (from individual trajectories to network flows). These surrogates can be embedded in optimisation and scenario-analysis pipelines, enabling faster iteration on policy and network design questions.
\end{itemize}

I am particularly motivated by the possibility of combining \textbf{agent-based models}, \textbf{activity-based frameworks}, and \textbf{data-driven world models} to develop transport simulations that are both behaviourally grounded and computationally efficient, in line with Mobilab’s objectives.

\section*{Research vision for the postdoctoral project}

Building on my doctoral work, my postdoctoral research vision within Mobilab could be structured around three complementary directions:

\begin{enumerate}
      \item \textbf{Organization-guided, multi-scale agent-based models for mobility.}
            Develop MAS-based transport models in which different actor types (individual travellers, transport operators, infrastructure managers) are organised through explicit roles, responsibilities, and coordination protocols. This would provide a structured way to represent institutional and regulatory aspects of transport systems, and to study how they interact with individual behaviour and network performance.

      \item \textbf{World-model-based digital twins for dynamic transport networks.}
            Design and validate world models that approximate the dynamics of complex, multi-modal transport networks by leveraging mobility data (e.g., trajectories, counts, sensor streams). These models would act as digital twins capable of supporting demand forecasting, scenario testing, and control design, in conjunction with existing microscopic or mesoscopic simulators.

      \item \textbf{Learning-based decision support for transport planning and operations.}
            Explore MARL and related learning methods to tackle problems such as traffic management, multimodal assignment, or the coordination of on-demand mobility services, while ensuring interpretability and robustness through organizational and structural priors. The long-term goal is to build tools that can be trusted by planners and operators, and that integrate naturally with existing modelling workflows.
\end{enumerate}

These axes are intended to be flexible and to integrate with ongoing projects and PhD theses within the group. I am keen to adapt them to Mobilab’s priorities and to co-design research directions with the team.

\section*{Collaboration, supervision, and long-term perspective}

My PhD has been conducted in a strongly interdisciplinary and collaborative environment, at the interface between \textit{academia} (Université Grenoble Alpes, LCIS laboratory) and \textit{industry} (Thales LAS “La Ruche”). This context has given me experience in:
\begin{itemize}
      \item translating methodological advances into applied settings (e.g., cyberdefence, networked systems);
      \item working with heterogeneous teams (researchers, engineers, stakeholders);
      \item and contributing to research projects involving multiple institutional partners.
\end{itemize}

In parallel, I have been involved in teaching activities and in the supervision of Master’s and engineering students, which has strengthened my mentoring skills and my motivation to contribute to the supervision of doctoral researchers, as foreseen in the postdoctoral position.

In the longer term, I aim to pursue an academic career that combines \textbf{methodological research in MARL and MAS} with \textbf{application-driven work} in domains such as transport, logistics, and cyber-physical systems. I see the Mobilab Transport Research group, and more broadly the University of Luxembourg, as an ideal environment to develop this agenda, given its strong emphasis on multidisciplinarity, international collaborations, and societally relevant research.

\vspace{0.5em}
\noindent
I would be delighted to contribute my expertise in multi-agent systems, reinforcement learning, and simulation to Mobilab’s ongoing projects in transport modelling and mobility research. Thank you very much for considering my application.

\makeletterclosing

\end{document}