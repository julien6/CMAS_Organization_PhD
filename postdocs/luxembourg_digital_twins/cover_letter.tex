\documentclass[11pt,a4paper,sans]{moderncv}        

\moderncvstyle{banking}
\moderncvcolor{red}                                
\usepackage{dirtytalk}
\usepackage[utf8]{inputenc}
\usepackage[scale=0.85]{geometry}
\usepackage{graphicx}
\usepackage{ragged2e}
\usepackage{setspace}
\usepackage{parskip}
% \usepackage[hidelinks]{hyperref}
% \usepackage{biblatex}

% === Load your .bib file ===
% \addbibresource{references.bib}

% \setstretch{1.05}

\AfterPreamble{
  \hypersetup{
    colorlinks=false,        % désactive les liens colorés (sinon pas de cadre)
    pdfborder={0 0 1},       % active un cadre autour des liens
    pdfborderstyle={/S/U/W 1}, % facultatif : style souligné
    linkbordercolor={0 0 1}, % cadre bleu pour liens internes
    urlbordercolor={0 0 1},  % cadre bleu pour les URL
    citebordercolor={0 0 1}  % cadre bleu pour citations (si utilisées)
  }
}

\name{Julien}{Soulé}
\title{Postdoctoral Researcher Application -- DC-26064}
\address{35 Rue Mathieu de la Drôme, Valence}{26000}{France}

\phone[mobile]{+33~6~77~63~12~13}
\email{julien.soule@hotmail.fr}
\homepage{julien6.github.io/home/}

%----------------------------------------------------------------------------------
\begin{document}

\recipient{Selection Committee}{Luxembourg Institute of Science and Technology (LIST)\\Belval, Luxembourg}
\date{\today}
\opening{Dear Members of the Selection Committee,}
\closing{{I would be delighted to further discuss my application and ideas in an interview.}\\[0.5cm]Sincerely,\vspace{-0.5cm}}

\makelettertitle
\justifying

\noindent
I am writing to apply for the position \textit{Postdoctoral Researcher in Model-Based Digital Twin (DC-26064)} at the Luxembourg Institute of Science and Technology (LIST).
One of my key research interests is focusing on the \textbf{model-based simulation of complex systems} using an hybrid approach combining data-driven models such as \textbf{World Models}~\href{https://link.springer.com/chapter/10.1007/978-3-031-63223-5_24}{(Ha \textit{et al.}, 2018)} and abstract frameworks such as \href{https://link.springer.com/book/10.1007/978-3-319-28929-8}{\textbf{Dec-POMDP}} to support \textbf{analysis, scenario exploration, and decision-making} especially for \textbf{Multi-Agent Systems (MASs)}.
This perspective closely aligns with LIST’s goals of developing \textbf{incremental, model-driven Digital Twins} to support system design and operational decision support.

I recently defended \href{https://julien6.github.io/home/#thesis}{\textbf{my PhD}} in Computer Science at \textit{Université Grenoble Alpes} (UGA), conducted in close collaboration with \textit{Thales Land \& Air Systems}.
My doctoral work addressed the modeling, simulation, and analysis of  \textbf{MASs}, formal modeling, and learning-based components.
I approached MAS design in evolving, partially specified environments, where executable models are used from early design stages to explore behaviors, assess trade-offs, and support informed decision-making.

\section*{Background and main contributions}

During \href{https://julien6.github.io/home/#thesis}{\textbf{my PhD}}, I developed a \textbf{methodological framework for the design and analysis of MAS}, implemented in the open-source \href{https://github.com/julien6/CybMASDE}{\textbf{CybMASDE platform}}.
This framework combines formal modeling, \textbf{Multi-Agent Reinforcement Learning (MARL)}, and data-driven analysis to support the full lifecycle of system design, from early conceptualization to executable prototypes.
A central goal of this work is to enable \textbf{incremental Digital Twin development}, even when limited information is available about the final system.

My main contributions can be summarized as follows:

\begin{itemize}
      \item \textbf{MARL-assisted MAS design.}
            I proposed a methodological framework that formulates MAS design as a constrained policy-optimization problem through iterative \textit{model--train--analyze--transfer} cycles.
            In a Digital Twin context, this approach enables the exploration and comparison of operational strategies within an model-based simulation
            \href{https://link.springer.com/chapter/10.1007/978-3-031-63223-5_24}{\textbf{(Soulé \textit{et al.}, AIAI~2024)}} (\textit{PettingZoo/Gym}); \href{https://assets-eu.researchsquare.com/files/rs-7166037/v1_covered_908e23dd-6fb8-4efc-9ef3-a78c4d539bac.pdf?c=1753863562}{\textbf{(JAAMAS~2025, under revision)}}.

      \item \textbf{World-model-based Digital Twin.}
            I proposed \textbf{World Models} \href{https://link.springer.com/chapter/10.1007/978-3-031-63223-5_24}{(Ha \textit{et al.}, 2018)} for multi-agent settings, using them as \textit{Digital Twins} that approximate environmental dynamics through learned representations (\textit{PyTorch, scikit-learn}).
            These models support model-based simulation and scenario exploration while raising key questions on robustness and interpretability, which I began to address theoretically through a \textbf{neuro-symbolic} approach by combining learned and formal models
            \href{https://assets-eu.researchsquare.com/files/rs-7166037/v1_covered_908e23dd-6fb8-4efc-9ef3-a78c4d539bac.pdf?c=1753863562}{\textbf{(JAAMAS~2025, under revision)}}; \href{https://arxiv.org/abs/2505.21559}{\textbf{(IEEE CLOUD~2025)}}.

      \item \textbf{Organization-guided MARL.}
            I designed an \href{https://github.com/julien6/MOISE-MARL}{\textbf{organizational-oriented MARL framework}} integrating symbolic constraints such as roles and goals into MARL (\textit{MARLlib/JaxMARL}).
            Beyond learning performance, this provides explicit structural and organizational abstractions that can be leveraged to model, constrain, and interpret system behaviors within a generated/handcrafted Digital Twin
            \href{https://arxiv.org/abs/2503.23615}{\textbf{(AAMAS~2025)}}; \href{https://arxiv.org/abs/2505.21559}{\textbf{(IEEE CLOUD~2025)}}.

      \item \textbf{Behavior analysis and explainability.}
            I proposed an unsupervised analysis method that extracts implicit roles and goals from agent trajectories.
            This supports the interpretation of simulated behaviors, the identification of emergent patterns, and the production of explainable indicators for decision-support
            \href{https://arxiv.org/abs/2503.23615}{\textbf{(AAMAS~2025)}}.
\end{itemize}

\section*{Fit with the model-based Digital Twin project}

The goals of the Digital Twin activities at LIST, and in particular within the context of the LuxHyVal initiative, strongly resonate with my research trajectory.
They align with a \textbf{central research goal} structuring my work: the design of \textbf{incremental, model-driven Digital Twins that remain executable, coherent, and decision-oriented throughout their lifecycle}, even when information about the final system is incomplete.

Within this goal, I envision my contributions along the following complementary axes:

\begin{itemize}
      \item \textbf{Incremental and executable Digital Twin framework.}
            Contribute to a theoretical framework in which executable Digital Twins are available from early design stages and progressively refined as knowledge increases, while preserving consistency across abstraction levels.

      \item \textbf{Integration of heterogeneous concerns.}
            Support the coordination of technical, organizational, and economic viewpoints within the Digital Twin, enabling coherent representations of the hydrogen value chain.

      \item \textbf{Scenario exploration and decision support.}
            Strengthen simulation- and analysis-based capabilities for feasibility assessment, and what-if reasoning, transforming simulation outputs into interpretable decision-support indicators.
            MARL approaches could assist the enhancement of operational strategies within the Digital Twin.

      \item \textbf{Model-based foundations with hybrid data-driven approaches.}
            Investigate and strengthen data-driven techniques for Digital Twins to approximate unknown dynamics, while preserving interpretability and lifecycle consistency through a \textbf{neuro-symbolic} approach combining a model-based framework and \textbf{World Models}.

      \item \textbf{Accessible and interactive Digital Twin interfaces.}
            Contribute to visual and interactive tools (dashboards, visual analytics, simulation interfaces) that make Digital Twin outputs accessible and support collective understanding and decision-making.
\end{itemize}




\section*{Collaboration and motivation}

My doctoral research was conducted in close collaboration with industrial partners, giving me a strong appreciation for \textbf{research-to-industry transfer} and the practical constraints associated with deploying research prototypes.
I am accustomed to working at the interface between researchers, engineers, and domain experts.

In this context, I see my contribution to collaborative research projects structured around the following principles:

\begin{itemize}
      \item \textbf{Expert knowledge elicitation and formalization.}
            Capture and formalize domain expertise into executable and analyzable models that can be shared, discussed, and refined collectively.

      \item \textbf{Iterative and participatory development.}
            Develop models and prototypes iteratively, incorporating stakeholder feedback to ensure their relevance for decision-making.

      \item \textbf{Bridging research and operational needs.}
            Translate research concepts into tools, representations, and workflows usable by non-academic stakeholders.

      \item \textbf{Scientific and technical dissemination.}
            Contribute to scientific publications, technical reports, and project deliverables with both academic rigor and practical impact.
\end{itemize}

More broadly, my motivation is to pursue a research agenda centered on \textbf{model-based Digital Twins as decision-support}, particularly under uncertainty, incomplete information, and long-term societal implications.

For me, the \textit{LuxHyVal} European initiative, involving an interdisciplinary research environment at LIST and its strong commitment to the energy transition, provides inspiring motivation to consolidate my expertise while contributing to research supporting this crucial societal challenge.


\vspace{0.4em}
\noindent
I believe that my background in model-based system design, simulation, and decision-support would allow me to make a meaningful contribution to the Digital Twin activities at LIST.

\makeletterclosing
\end{document}
