\documentclass[10pt]{beamer}

% =========================
% Theme (sobre / académique)
% =========================
\usetheme{Madrid}
\usecolortheme{default}
\usefonttheme{professionalfonts}
\setbeamertemplate{navigation symbols}{}
\setbeamertemplate{footline}[frame number]

% =========================
% Packages
% =========================
\usepackage[utf8]{inputenc}
\usepackage[T1]{fontenc}
\usepackage{lmodern}
\usepackage{graphicx}
\usepackage{amsmath}
\usepackage{tikz}
\usetikzlibrary{shapes}
\usepackage[french]{babel}

% =========================
% Meta
% =========================
\title[Model-Based Digital Twin]{Proposition de démarrage du post-doctorat\\Model-Based Digital Twin}
\author{Julien Soulé}
\institute{Luxembourg Institute of Science and Technology (LIST)}
\date{\today}


% =========================
\begin{document}
% =========================

% -------------------------
\begin{frame}
    \titlepage
\end{frame}

% =====================================================
% AXE 1 — COMPRÉHENSION DU SUJET ET DES OBJECTIFS
% =====================================================

% -------------------------
\begin{frame}{Contexte général : LuxHyVal}

    \begin{tikzpicture}[remember picture,overlay]
        \node[anchor=north east] at (current page.north east |- 0cm, 1.1cm)
        {\includegraphics[trim={0cm 3cm 0 3cm},clip,width=0.3\textwidth]{luxhyval_icon.png}};
    \end{tikzpicture}

    \textbf{LuxHyVal — Luxembourg Hydrogen Valley}

    \vspace{0.2cm}
    \begin{itemize}
        \item Initiative européenne visant à déployer une \textbf{chaîne de valeur hydrogène vert intégrée}
        \item Couverture complète :
              \begin{itemize}
                  \item production, stockage, transport,
                  \item usages industriels et mobilité,
                  \item dimensions économiques, réglementaires et territoriales
              \end{itemize}
        \item Projet intrinsèquement \textbf{systémique} et \textbf{multi-acteurs}
    \end{itemize}

    \vspace{0.3cm}
    \textbf{Enjeu central}
    \begin{itemize}
        \item Aider à \textbf{concevoir, analyser et piloter} un système complexe en évolution
        \item Dans un contexte d’incertitudes techniques, économiques et temporelles
    \end{itemize}
\end{frame}

% -------------------------
\begin{frame}{Objectifs du postdoc}
    \textbf{Lecture des objectifs du post-doctorat}

    \vspace{0.2cm}
    \begin{itemize}
        \item Contribuer au développement d’un \textbf{model-based Digital Twin}
        \item Structurer la connaissance et les hypothèses sous forme de \textbf{modèles explicites}
        \item Rendre ces modèles :
              \begin{itemize}
                  \item exécutables,
                  \item analysables,
                  \item exploitables pour l’aide à la décision
              \end{itemize}
    \end{itemize}

    \vspace{0.3cm}
    \textbf{Posture adoptée}
    \begin{itemize}
        \item Le Digital Twin comme \textbf{objet d’ingénierie}
        \item Les outils numériques comme \textbf{supports}, non comme finalité
    \end{itemize}
\end{frame}

% -------------------------
\begin{frame}{Rôle du Digital Twin model-based}
    \textbf{Rôle attendu du Digital Twin dans LuxHyVal}

    \vspace{0.2cm}
    \begin{center}
        \begin{tikzpicture}[node distance=1.1cm, every node/.style={draw, rectangle, rounded corners, align=center}]
            \node (a) {Expertise humaine\\et hypothèses};
            \node (b) [below of=a] {Modèles de conception\\(SysML, ontologies)};
            \node (c) [below of=b] {Modèles exécutables\\(simulation)};
            \node (d) [below of=c] {Indicateurs \&\\aide à la décision};

            \draw[->] (a) -- (b);
            \draw[->] (b) -- (c);
            \draw[->] (c) -- (d);
        \end{tikzpicture}
    \end{center}

    \vspace{0.2cm}
    \begin{itemize}
        \item Centralité des \textbf{modèles explicites}
        \item Traçabilité entre hypothèses, simulations et résultats
        \item Support à des décisions discutables et justifiables
    \end{itemize}
\end{frame}

% =====================================================
% AXE 2 — MES TRAVAUX ET MISE AU SERVICE DU POSTDOC
% =====================================================

% -------------------------
\begin{frame}{Rappel synthétique de mes travaux de thèse}
    \textbf{Conception et analyse de Systèmes Multi-Agents complexes (pour Cyberdéfense et au-delà)}

    \vspace{0.2cm}
    \begin{itemize}
        \item Modélisation formelle et simulation de systèmes multi-agents
        \item Développement de Digital Twins incrémentaux :
              \begin{itemize}
                  \item exécutables dès les phases amont,
                  \item raffinés progressivement avec l’acquisition de connaissances (\textit{World Models} \& \textit{Dec-POMDP})
              \end{itemize}
        \item Couplage entre :
              \begin{itemize}
                  \item modèles conceptuels (modèles organisationnels symboliques représentables comme ontologies),
                  \item simulations et optimisation (via approches connexionnistes : RL / MARL),
                  \item analyse des comportements et indicateurs (via techniques \textit{Unsupervised Machine Learning})
              \end{itemize}
    \end{itemize}

    \vfill
    \textit{Voir présentation de thèse}

\end{frame}

% -------------------------
\begin{frame}{Mise au service du postdoc : proposition}
    \textbf{Hypothèse de travail (à discuter et affiner)}

    \vspace{0.2cm}
    \begin{itemize}
        \item Les principes méthodologiques développés durant la thèse
        \item pourraient être adaptés au contexte LuxHyVal
        \item en respectant les contraintes propres au domaine hydrogène
    \end{itemize}

    \vspace{0.3cm}
    \textbf{Objectif}
    \begin{itemize}
        \item Proposer un cadre \textbf{model-based incrémental}
        \item permettant de relier conception, simulation et décision via
              \begin{itemize}
                  \item une approche connexioniste-symbolique mêlant expertise humaine et outils numériques,
                  \item des modèles explicites, exécutables et analysables,
                  \item une démarche itérative et co-construite avec les partenaires du projet.
              \end{itemize}
    \end{itemize}
\end{frame}

% -------------------------
\begin{frame}{Proposition de plan de postdoc}
    \textbf{Plan général (indicatif et ouvert)}

    \vspace{0.2cm}
    \begin{center}
        \begin{tikzpicture}[node distance=1.1cm, every node/.style={draw, ellipse, align=center}]
            \node (p1) {Clarification\\conceptuelle};
            \node (p2) [right of=p1, xshift=4.2cm] {Modélisation\\de conception};
            \node (p3) [below of=p2, yshift=-1.1cm] {Exécutabilité\\et simulation};
            \node (p4) [left of=p3, xshift=-4.2cm] {Analyse \&\\indicateurs};

            \draw[->] (p1) -- (p2);
            \draw[->] (p2) -- (p3);
            \draw[->] (p3) -- (p4);
            \draw[->] (p4) -- (p1);
        \end{tikzpicture}
    \end{center}

    \vspace{0.2cm}
    \begin{itemize}
        \item Démarche itérative
        \item Co-construction avec les partenaires du projet
    \end{itemize}
\end{frame}

% -------------------------
\begin{frame}{Livrables envisagés (indicatifs)}
    \textbf{Livrables scientifiques}

    \begin{itemize}
        \item Publications sur :
              \begin{itemize}
                  \item Approche purement symbolique pour modéliser des systèmes énergétiques complexes et \textbf{réaliste, capable de mise à incrémentale et explicable} ;
                        \begin{itemize}
                            \item Conférences orientées \textit{model-based} : MODELS ?
                        \end{itemize}
                  \item Investiguer approche neuro-symbolique pour \textbf{aider/automatiser la modélisation}, via \textit{World Models} ;
                        \begin{itemize}
                            \item Conférences orientées ML : ICML / AAAI / etc. ?
                        \end{itemize}
                  \item Approche mêlant représentation symboliques et techniques IA / simulation pour analyse des comportements dans un contexte complexe et l'\textbf{aide à la décision / optimisation}.
                        \begin{itemize}
                            \item Conférence orienté SMA : AAMAS / IJCAI / etc. ?
                        \end{itemize}
              \end{itemize}
        \item Contributions méthodologiques transposables
    \end{itemize}

    \vspace{0.3cm}
    \textbf{Livrables techniques}

    \begin{itemize}
        \item Prototypes de modèles exécutables (repo de code source ouvert)
        \item Outils d’analyse et de visualisation (tableaux de bord : Web / desktop)
        \item Documentation favorisant la réplication (guides, Wiki, tutoriels, etc.)
    \end{itemize}
\end{frame}

% -------------------------
\begin{frame}{Conclusion}
    \begin{itemize}
        \item Une compréhension du sujet centrée sur les enjeux \textbf{model-based}
        \item Une proposition de contribution :
              \begin{itemize}
                  \item ouverte à la discussion et ajustable en cours de route,
                  \item voulue comme alignée avec les objectifs de LuxHyVal
                  \item avec une extension symbolique-connexionniste mêlant expertise humaine (explicabilité \& contrôle) et outils numériques (ML et IA plus généralement)
              \end{itemize}
        \item Volonté de co-construire le postdoc avec toutes les parties prenantes
    \end{itemize}
\end{frame}

% -------------------------
\end{document}



% -------------------------
\end{document}
