\documentclass[11pt,a4paper,sans]{moderncv}        

\moderncvstyle{banking}
\moderncvcolor{red}                                
\usepackage{dirtytalk}
\usepackage[utf8]{inputenc}
\usepackage[scale=0.9]{geometry}
\usepackage{graphicx}
\usepackage{ragged2e}

\AfterPreamble{
  \hypersetup{
    colorlinks=false,        % désactive les liens colorés (sinon pas de cadre)
    pdfborder={0 0 1},       % active un cadre autour des liens
    pdfborderstyle={/S/U/W 1}, % facultatif : style souligné
    linkbordercolor={0 0 1}, % cadre bleu pour liens internes
    urlbordercolor={0 0 1},  % cadre bleu pour les URL
    citebordercolor={0 0 1}  % cadre bleu pour citations (si utilisées)
  }
}

\name{Julien}{Soulé}
\title{PhD Student -- Thales/UGA}
\address{35 Rue Mathieu de la Drôme, Valence}{26000}{France}

\phone[mobile]{06 77 63 12 13}
\email{julien.soule@hotmail.fr}
\homepage{julien6.github.io/home/}

%----------------------------------------------------------------------------------
%            content
%----------------------------------------------------------------------------------
\begin{document}

\recipient{\vspace{-0.9cm}Research Committee}{\textit{\textit{Télécom Paris}} -- \textit{Institut Mines-Télécom}\\Palaiseau, France}
\date{\today}
\opening{Dear Members of the Research Committee,}
\closing{{I look forward to the opportunity to further discuss my background and research plans in an interview.}\\[0.6cm]Sincerely,\vspace{-0.6cm}}

\makelettertitle

\justifying

I am writing to express my strong interest in the Post-doctoral Fellow position in Model-Based Reinforcement Learning (MBRL) at \textit{\textit{Télécom Paris}}. With a background in Multi-Agent Reinforcement Learning (MARL), experience with \textit{\textbf{World Models}} \href{https://link.springer.com/chapter/10.1007/978-3-031-63223-5_24}{(Ha \textit{et al.}, 2018)}, and a deep interest in hybrid symbolic–learning architectures, I believe my research profile aligns closely with the position’s objective of developing \textbf{structured and verifiable MBRL} for safety-critical domains such as robotics, transport, and industrial control.

During my doctoral studies at \textit{Université Grenoble Alpes} and with \textit{Thales~LAS}, I developed a \href{https://github.com/julien6/CybMASDE}{\textbf{methodological framework}} for assisting the design of Multi-Agent Systems (MAS) considering its design as an optimization problem consisting in finding a optimal joint-policy satisfying symbolic-based constraints (using \textit{Dec-POMDPs} and implemented via \textit{PettingZoo/Gym}). This framework integrates MARL, simulation through learned environment dynamics using our proposed multi-agent \textit{World Models}, and unsupervised learning techniques for analyzing the behavior of trained agents \href{https://link.springer.com/chapter/10.1007/978-3-031-63223-5_24}{\textbf{(Soulé \textit{et al.}, AIAI~2024)}}, \href{https://sciety-labs.elifesciences.org/articles/by?article_doi=10.21203/rs.3.rs-7166037/v1}{\textbf{(Soulé \textit{et al.}, JAAMAS~2025 -- \textit{under revision})}}. Within this framework, I designed a \href{https://github.com/julien6/MOISE-MARL}{\textbf{organizationally-oriented MARL framework}} capable of integrating sub-symbolic elements such as roles and goals to enhance safety and explainability \href{https://arxiv.org/abs/2503.23615}{\textbf{(Soulé \textit{et al.}, AAMAS~2025)}} in complex, multi-objective environments (using \textit{RLlib} and \textit{JaxMARL/Jax}), including industrial scenarios \href{https://arxiv.org/abs/2505.21559}{\textbf{(Soulé \textit{et al.}, IEEE CLOUD~2025)}}.

I proposed \textit{World Model} architectures that function as joint-observation predictors comparable to digital twins capturing the underlying observational dynamics of the environment (mainly implemented with \textit{PyTorch/scikit-learn}). While initially designed for simulation fidelity, these models allow agents to anticipate dynamic changes, adapt their policies, and maintain robustness under distributional shifts. Through these experiments, I encountered many of the same challenges highlighted in your position, notably the brittleness of learned models and the need for verifiable, interpretable structures. This led me to start formalizing these problems and investigating principled ways to mitigate them.

In this context, I initiated research directions directly related to the focus areas described in your call. Specifically, I began exploring \textbf{neurosymbolic structures} within \textit{World Models}, where learned latent dynamics are complemented by explicit symbolic constraints (e.g., physical laws or rule-based logic), thereby enhancing model fidelity, interpretability, and safety. These ideas strongly resonate with the survey by \textbf{Mohan, Zhang \& Lindauer (2024)}. I have also started exploring \textbf{\textit{World Models}} architectures where distinct sub-modules enable a better capturing of the environment complexity, paralleling the categorical framework proposed by \textbf{Bakirtzis, Savvas \& Topcu (2025)}.

What particularly attracts me to this position is \textit{Télécom Paris}’s emphasis on \textbf{neurosymbolic modeling} and \textbf{compositional reasoning}. I am eager to extend my work on MBRL toward these objectives, developing methods that combine symbolic constraints and modular decomposition to ensure verifiable and adaptive model-based reasoning. My long-term goal is to design systems where explicit symbolic knowledge (e.g., physical laws, logical invariants) is systematically integrated with learned representations, producing agents that adapt reliably while preserving safety guarantees.

Beyond the technical fit, I am deeply drawn to the collaborative and interdisciplinary environment of \textit{\textit{Télécom Paris}} and the \textit{Institut Mines-Télécom}, particularly through the "Architecture of Complex Systems" chair and its industrial partnerships. This postdoctoral opportunity fits naturally within my professional research trajectory and represents an ideal setting to continue developing innovative frameworks in close dialogue with both academic and industrial researchers. I would be enthusiastic to contribute proactively to your team’s ongoing projects and future initiatives.


\makeletterclosing

\end{document}
