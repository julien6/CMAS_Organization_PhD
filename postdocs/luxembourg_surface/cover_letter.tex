\documentclass[11pt,a4paper,sans]{moderncv}        

\moderncvstyle{banking}
\moderncvcolor{red}                                
\usepackage{dirtytalk}
\usepackage[utf8]{inputenc}
\usepackage[scale=0.85]{geometry}
\usepackage{graphicx}
\usepackage{ragged2e}
\usepackage{setspace}
\usepackage{parskip}
% \usepackage[hidelinks]{hyperref}
% \usepackage{biblatex}


% === Load your .bib file ===
% \addbibresource{references.bib}

% \setstretch{1.05}

\AfterPreamble{
  \hypersetup{
    colorlinks=false,        % désactive les liens colorés (sinon pas de cadre)
    pdfborder={0 0 1},       % active un cadre autour des liens
    pdfborderstyle={/S/U/W 1}, % facultatif : style souligné
    linkbordercolor={0 0 1}, % cadre bleu pour liens internes
    urlbordercolor={0 0 1},  % cadre bleu pour les URL
    citebordercolor={0 0 1}  % cadre bleu pour citations (si utilisées)
  }
}

\name{Julien}{Soulé}
\title{Postdoctoral Researcher Application -- UOL07557}
\address{35 Rue Mathieu de la Drôme, Valence}{26000}{France}

\phone[mobile]{06 77 63 12 13}
\email{julien.soule@hotmail.fr}
\homepage{julien6.github.io/home/}

%----------------------------------------------------------------------------------
%            content
%----------------------------------------------------------------------------------
\begin{document}

\recipient{Selection Committee}{SnT -- Interdisciplinary Centre for Security, Reliability and Trust\\University of Luxembourg\\Kirchberg Campus, Luxembourg}
\date{\today}
\opening{Dear Members of the Selection Committee,}
\closing{{I would be delighted to further discuss how my background and ideas could contribute to this project in an interview.}\\[0.6cm]Sincerely,\vspace{-0.6cm}}

\makelettertitle

\justifying

\noindent
I am writing to express my strong interest in the \textit{Research Associate -- AI-Guided Attack Surface Assessment} position at the University of Luxembourg's \textit{SnT}, within the \textit{SEDAN} group.
My research lies at the intersection of \textbf{Multi-Agent Reinforcement Learning (MARL)}, \textbf{cyberdefense}, and \textbf{organizational modeling}, with the overarching goal of designing learning-based cyberdefense agents that are \textbf{effective}, \textbf{interpretable}, and \textbf{operationally controllable}.
My doctoral work is grounded in the broader landscape of {Autonomous Cyberdefense (ACD)}, centered on the \href{https://link.springer.com/book/10.1007/978-3-031-29269-9}{\textbf{\textbf{AICA}}} (Autonomous Intelligent Cyberdefense Agent) initiative.
I have been an active contributor to this research domain through the \textbf{NATO IST-152} working group and its successor, the \href{https://www.aica-iwg.org/}{\textbf{AICA IWG}} (AICA International Working Group), where I serve as treasurer.
This involvement has exposed me to a rich, international community of researchers and practitioners working on AI-driven attack surface assessment and autonomous defensive strategies, providing both inspiration and practical guidance for my research agenda in applying \textbf{RL and MARL techniques} to cyberdefense and attack surface assessment.
Throughout my PhD at \textit{Université Grenoble Alpes} (UGA) in collaboration with \textit{Thales LAS}, I have approached cyberdefense as a sequential decision-making problem under uncertainty, where autonomous agents must explore, monitor, and protect complex, heterogeneous infrastructures.

\section*{Background and main contributions}

During my doctoral studies, I developed a \textbf{methodological framework for assisting the design of cyberdefense multi-agent systems}, implemented in the open-source \href{https://github.com/julien6/CybMASDE}{\textbf{CybMASDE platform}}.
This framework supports the full lifecycle of MAS design by combining formal modeling, MARL, world-model-based simulation, and unsupervised behavior analysis.
It is explicitly tailored to cyberdefense scenarios in which defenders must reason about evolving attack surfaces, sparse signals, and incomplete visibility.

My work can be organized around the following themes:

\begin{itemize}
      \item \textbf{MARL-assisted cyberdefense system design.}
            I proposed a methodological framework that casts the design of cyberdefense agents as a constrained policy-optimization problem over a joint action space.
            Using PettingZoo-style environments and MARL libraries, I implemented iterative \textit{design--train--analyze--transfer} cycles that allow experts to explore defensive strategies against simulated attackers in realistic networked environments (enterprise networks, microservice architectures, cyber-physical settings).

      \item \textbf{Organization-guided MARL for controllable cyber agents.}
            I developed an \href{https://github.com/julien6/MOISE-MARL}{\textbf{organizationally-oriented MARL framework}} that integrates symbolic constraints (roles, missions, hierarchical goals) into MARL.
            This improves convergence and stability in non-stationary multi-agent settings, and provides levers for human operators to \textit{steer} and \textit{constrain} autonomous cyberdefense behaviors (a key requirement for operational deployment in security operations centers.

      \item \textbf{World-model-based simulation and digital twins.}
            I proposed \textbf{world models} for multi-agent cyberdefense scenarios, using neural and probabilistic models (PyTorch, scikit-learn) to approximate the dynamics of complex infrastructures and attacker--defender interactions.
            These models act as \textit{digital twins} of the defended system, enabling safer exploration, what-if analyses, and data-efficient DRL while exposing open challenges related to model drift, uncertainty, and robustness.

      \item \textbf{Behavior analysis and anomaly-oriented explainability.}
            I developed unsupervised analysis methods that cluster and characterize agent behaviors from trajectories and logs, extracting implicit roles, tactics, and anomalous patterns.
            This provides organizational insight into emergent strategies and supports \textbf{explainable MARL} in security contexts, where operators must understand why a system raises an alert or prioritizes specific actions.
\end{itemize}

Collectively, these contributions form a coherent pipeline (from \textbf{modeling} to \textbf{training}, \textbf{analysis}, and \textbf{transfer}) for deploying learning-based cyberdefense agents in realistic environments.

\section*{Fit with the AI-guided attack surface assessment project}

The objectives of your project, namely to \textbf{automate expert-driven attack surface assessment} by combining \textbf{Deep Reinforcement Learning}, \textbf{Large Language Models (LLMs)}, and \textbf{graph-based ML} over heterogeneous security tools, align very closely with my experience and interests.

First, the definition of an \textbf{orchestrator} that coordinates multiple external attack surface management (EASM) tools via DRL echoes my work on multi-agent orchestration of cyberdefense capabilities.
I have practical experience in designing state and action spaces for RL agents interacting with complex software stacks (e.g., Kubernetes-based microservices, networked systems), and in defining reward structures that encode security objectives such as coverage, timeliness, and risk reduction.
I am particularly interested in formulating the orchestration problem as a partially observable, stochastic control problem over a heterogeneous toolchain, where the agent must reason about exploration vs.\ exploitation of the attack surface.

Second, the use of \textbf{LLMs} to iteratively refine tool configurations and interpret text-based outputs (domain names, provider names, certificates, banners, etc.) naturally extends my recent interest in neuro-symbolic and language-guided RL.
In my future work, I aim to investigate how language models can provide high-level priors, abstractions, and hypothesis generation for RL agents.
In your project, I see an exciting opportunity to couple LLM-based semantic interpretation of scan outputs with DRL-based decision-making to drive more informed, adaptive exploration of the attack surface.

Third, the construction of an \textbf{aggregated knowledge graph} of the discovered attack surface, together with \textbf{graph ML} for anomaly detection and prioritization, is closely related to my work on world models and representation learning.
My experience in building structured latent representations and analyzing emergent behaviors would transfer well to building graph-based views of assets, services, and dependencies, and to developing anomaly scores that highlight suspicious or previously unseen exposure points.
More broadly, I am motivated by the perspective of turning raw tool outputs into actionable, ranked recommendations that security practitioners can trust and act upon.

From a technical standpoint, I am comfortable working in \textbf{Python} and \textbf{Linux} environments, using modern ML frameworks (PyTorch, JAX), and interacting with distributed systems and monitoring stacks (e.g., in the context of our work on resilient Kubernetes autoscaling).
My industrial experience at \textit{Thales} and \textit{Atos} has also trained me to deal with real constraints on deployment, interoperability, and security-by-design.

\section*{Collaboration, industrial experience, and long-term motivation}

My PhD has been conducted in close collaboration with industrial partners in cyberdefense, which has given me a strong appreciation of \textbf{practice-driven research}.
I have worked with security experts on modeling realistic attacker--defender scenarios, integrating ML-based components into existing workflows, and discussing the operational implications of autonomous decision-making.
I am confident that this background will help me collaborate effectively with the industrial partner of this project, and to translate research prototypes into tools that are usable by on-the-field practitioners.

In parallel, I have been actively involved in the \textit{Autonomous Intelligent Cyberdefence Agent} international working group, for which I serve as treasurer.
I regularly present and discuss my work with an interdisciplinary community spanning academia, industry, and defense organizations.
These interactions have reinforced my motivation to pursue a research career at the interface of \textbf{ML} and \textbf{cybersecurity}, with a strong emphasis on robustness, explainability, and real-world impact.

In the longer term, I aim to develop a research agenda that unifies \textbf{RL-based orchestration of security tools}, \textbf{language- and graph-based representations} of attack surfaces, and \textbf{human-centered explainability} for SOC operators.
The SnT, and particularly the SEDAN group, provide an ideal environment for such a program, combining strong expertise in applied ML and cybersecurity, close industrial partnerships, and an international, interdisciplinary atmosphere.

\vspace{0.5em}
\noindent
I believe that my background in MARL, cyberdefense, and methodological tool-building would allow me to make a meaningful contribution to the AI-Guided Attack Surface Assessment project and to the broader research activities of SnT.
I would be excited to join your team and collaborate on advancing intelligent, trustworthy, and operationally relevant security solutions.

\makeletterclosing

\end{document}