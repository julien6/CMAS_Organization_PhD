\documentclass[11pt,a4paper,sans]{moderncv}        

\moderncvstyle{banking}
\moderncvcolor{red}                                
\usepackage{dirtytalk}
\usepackage[utf8]{inputenc}
\usepackage[scale=0.85]{geometry}
\usepackage{graphicx}
\usepackage{ragged2e}
\usepackage{setspace}
\usepackage{parskip}
% \usepackage[hidelinks]{hyperref}
% \usepackage{biblatex}


% === Load your .bib file ===
% \addbibresource{references.bib}

% \setstretch{1.05}

\AfterPreamble{
  \hypersetup{
    colorlinks=false,        % désactive les liens colorés (sinon pas de cadre)
    pdfborder={0 0 1},       % active un cadre autour des liens
    pdfborderstyle={/S/U/W 1}, % facultatif : style souligné
    linkbordercolor={0 0 1}, % cadre bleu pour liens internes
    urlbordercolor={0 0 1},  % cadre bleu pour les URL
    citebordercolor={0 0 1}  % cadre bleu pour citations (si utilisées)
  }
}

\name{Julien}{Soulé}
\title{Postdoctoral Researcher Application -- UOL07557}
\address{35 Rue Mathieu de la Drôme, Valence}{26000}{France}

\phone[mobile]{06 77 63 12 13}
\email{julien.soule@hotmail.fr}
\homepage{julien6.github.io/home/}

%----------------------------------------------------------------------------------
%            content
%----------------------------------------------------------------------------------
\begin{document}

\recipient{Selection Committee}{SnT -- Interdisciplinary Centre for Security, Reliability and Trust\\University of Luxembourg\\Kirchberg Campus, Luxembourg}
\date{\today}
\opening{Dear Members of the Selection Committee,}
\closing{{I would be delighted to further discuss my application and ideas in an interview.}\\[0.5cm]Sincerely,\vspace{-0.5cm}}

\makelettertitle
\justifying

\noindent
I am writing to express my strong interest in the \textit{Research Associate -- AI-Guided Attack Surface Assessment} position at the University of Luxembourg's \textit{SnT}, within the \textit{SEDAN} group. My research lies at the intersection of \textbf{Multi-Agent Reinforcement Learning (MARL)}, \textbf{cyberdefense}, and \textbf{organizational modeling}, with the overarching goal of designing learning-based cyberdefense agents that are \textbf{effective}, \textbf{interpretable}, and \textbf{operationally controllable}. My doctoral work is grounded in the broader landscape of {Autonomous Cyberdefense (ACD)}, centered on the \href{https://link.springer.com/book/10.1007/978-3-031-29269-9}{\textbf{\textbf{AICA}}} (Autonomous Intelligent Cyberdefense Agent) initiative. I have been an active contributor to this research domain through the \textbf{NATO IST-152} working group and its successor, the \href{https://www.aica-iwg.org/}{\textbf{AICA IWG}} (AICA International Working Group), where I serve as treasurer. This involvement has exposed me to a rich, international community of researchers and practitioners working on AI-driven attack surface assessment and autonomous defensive strategies, providing both inspiration and practical guidance for my research agenda in applying \textbf{RL and MARL techniques} to cyberdefense and attack surface assessment. Throughout my PhD at \textit{Université Grenoble Alpes} (UGA) in collaboration with \textit{Thales LAS}, I have approached cyberdefense as a sequential decision-making problem under uncertainty, where autonomous agents must explore, monitor, and protect complex, heterogeneous infrastructures.

\section*{Background and main contributions}

During my PhD, I developed a \textbf{methodological framework for cyberdefense multi-agent systems}, implemented in our proposed open-source \href{https://julien6.github.io/CybMASDE/}{\textbf{CybMASDE}} platform.
It supports the full lifecycle of system design by combining formal modeling, MARL, world-model-based simulation, and unsupervised behavior analysis, and is tailored to scenarios with \textbf{evolving attack surfaces}, \textbf{sparse rewards}, and \textbf{partial observability}.

My work can be summarized as follows:

\begin{itemize}
      \item \textbf{MARL-assisted MAS design.}
            I formulated MAS design as a constrained policy-optimization problem with iterative \textit{design--train--analyze--transfer} cycles
            \href{https://link.springer.com/chapter/10.1007/978-3-031-63223-5_24}{\textbf{(Soulé, AIAI~2024)}}; \href{https://sciety-labs.elifesciences.org/articles/by?article_doi=10.21203/rs.3.rs-7166037/v1}{\textbf{(Soulé, JAAMAS~2025, under second-round revision)}}.

      \item \textbf{Organization-guided MARL.}
            I proposed an \href{https://github.com/julien6/MOISE-MARL}{\textbf{organizationally-oriented MARL framework}} integrating symbolic constraints to improve convergence, mitigate non-stationarity, and enhance safety and explainability
            \href{https://dl.acm.org/doi/10.5555/3709347.3743834}{\textbf{(Soulé, AAMAS~2025)}}; \href{https://arxiv.org/abs/2505.21559}{\textbf{(Soulé, IEEE CLOUD~2025)}}.

      \item \textbf{World-model-based multi-agent learning.}
            I adapted \textbf{World Models} \href{https://arxiv.org/abs/1803.10122}{(Ha, 2018)} to multi-agent cyberdefense as digital twins enabling safer and data-efficient MARL, while analyzing robustness and interpretability issues
            \href{https://sciety-labs.elifesciences.org/articles/by?article_doi=10.21203/rs.3.rs-7166037/v1}{\textbf{(Soulé, JAAMAS~2025, under second-round revision)}}; \href{https://arxiv.org/abs/2505.21559}{\textbf{(Soulé, IEEE CLOUD~2025)}}.

      \item \textbf{Behavior analysis and explainability.}
            I developed unsupervised methods to extract implicit roles and objectives from agent trajectories, contributing to \textbf{explainable MARL}
            \href{https://dl.acm.org/doi/10.5555/3709347.3743834}{\textbf{(Soulé, AAMAS~2025)}}.
\end{itemize}


\section*{Fit with the AI-guided attack surface assessment project}

The goals of your project strongly resonate with my research trajectory.
In particular, the definition of an \textbf{intelligent orchestrator} coordinating multiple External Attack Surface Management (EASM) tools naturally echoes my work on the orchestration of distributed cyberdefense capabilities using reinforcement learning.

I have practical experience in designing state and action spaces for RL agents interacting with complex software stacks, as well as in defining reward structures that encode security objectives such as coverage, timeliness, and risk reduction.
I am especially interested in formulating the orchestration of attack surface tools as a \textbf{partially observable stochastic control problem}, where an agent must reason about exploration versus exploitation over a vast and evolving search space.

The use of \textbf{Large Language Models} to iteratively refine tool configurations and interpret text-based outputs (domain names, certificates, service banners, provider information) is also closely aligned with my interests.
I see strong potential in coupling LLM-based semantic interpretation with DRL-based decision-making, allowing agents to leverage high-level abstractions and hypotheses when navigating large attack surfaces.

Finally, the construction of an aggregated \textbf{knowledge graph} of discovered assets and dependencies, together with \textbf{graph-based ML} for anomaly detection and prioritization, strongly relates to my work on structured representations and world models.
My experience in representation learning and behavior analysis would naturally extend to building graph-based views of attack surfaces and to identifying suspicious or previously unseen exposure points, with the ultimate goal of producing actionable and trustworthy recommendations for practitioners.

\section*{Expected contributions}

I envision contributing to the project along the following concrete axes:
\begin{itemize}
      \item Formalizing the EASM orchestration problem as a reinforcement learning task under partial observability, with explicit abstractions and security-driven reward signals.
      \item Prototyping a DRL-based orchestrator interfacing with existing attack surface tools, leveraging my experience in MAS architectures and tool integration.
      \item Investigating the integration of LLM-based semantic analysis with RL to guide adaptive exploration of large and heterogeneous attack surfaces.
      \item Contributing to graph-based representations of discovered assets and to anomaly detection mechanisms supporting prioritized recommendations.
\end{itemize}

\section*{Collaboration and motivation}

My doctoral work was conducted in close collaboration with industrial partners, which has given me a strong appreciation for \textbf{practice-driven research}.
I am accustomed to working with security experts, integrating ML components into existing workflows, and addressing real-world constraints related to deployment, interoperability, and security-by-design.
I am confident that this background will allow me to collaborate effectively with the industrial partner of this project.

More broadly, I aim to pursue a research agenda at the intersection of \textbf{reinforcement learning}, \textbf{cybersecurity}, and \textbf{human-centered explainability}, with a strong emphasis on operational relevance.
The SnT, and in particular the SEDAN group, provide an ideal environment for developing such a program.

\vspace{0.5em}
\noindent
I believe that my background and research vision would allow me to make a meaningful contribution to the AI-Guided Attack Surface Assessment project and to the broader activities of SnT.

\makeletterclosing
\end{document}