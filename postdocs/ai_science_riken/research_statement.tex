\documentclass[11pt,a4paper,sans]{moderncv}        

\moderncvstyle{banking}
\moderncvcolor{red}                                
\usepackage{dirtytalk}
\usepackage[utf8]{inputenc}
\usepackage[scale=0.85]{geometry}
\usepackage{graphicx}
\usepackage{ragged2e}
\usepackage{setspace}
\usepackage{parskip}
\setstretch{1.05}

\AfterPreamble{
  \hypersetup{
    colorlinks=false,
    pdfborder={0 0 1},
    pdfborderstyle={/S/U/W 1},
    linkbordercolor={0 0 1},
    urlbordercolor={0 0 1},
    citebordercolor={0 0 1}
  }
}

\name{Julien}{Soulé}
\title{Research Statement -- Postdoctoral Researcher (Job No.~25-1477, RIKEN iTHEMS)}
\address{35 Rue Mathieu de la Drôme, Valence}{26000}{France}
\phone[mobile]{+33 6 77 63 12 13}
\email{julien.soule@hotmail.fr}
\homepage{julien6.github.io/home/}

\begin{document}

\recipient{\phantom{a}}{\phantom{a}}
\date{}
\opening{}
\closing{{\phantom{s}}}

\makelettertitle

\justifying

\vspace{-2.5cm}
\noindent
My research focuses on the intersection of \textbf{Multi-Agent Systems (MAS)}, \textbf{Reinforcement Learning (RL)}, and \textbf{organizational modeling}, with the long-term objective of developing principled frameworks that make multi-agent learning \textbf{interpretable}, \textbf{controllable}, and \textbf{systematically designable}. During my doctoral studies at \href{https://www.univ-grenoble-alpes.fr/english/}{\textit{Université Grenoble Alpes}} and within the \href{https://www.thalesgroup.com/}{\textit{Thales~LAS}} company (France), I investigated how structured priors describing cooperation patterns, roles, and goals can be integrated into modern reinforcement-learning pipelines to improve stability, safety, and interpretability. My work lies at the interface of symbolic design methods and data-driven learning, combining concepts from agent-oriented software engineering, control theory, and deep reinforcement learning.

\bigskip
\textbf{1. A methodological framework for MAS design}

\noindent
I developed a general framework for assisting the design and analysis of MASs, implemented in the open-source \href{https://github.com/julien6/CybMASDE}{\textbf{CybMASDE platform}}. The framework supports the complete lifecycle of multi-agent design: (i) formal modeling of the environment and organization, (ii) training via multi-agent reinforcement learning, and (iii) post-hoc behavior analysis. It automates iterative design–train–analyze–transfer cycles, allowing symbolic specifications (e.g., roles, missions, and goals) to guide and constrain the learning process.
This methodology was detailed in \href{https://link.springer.com/chapter/10.1007/978-3-031-63223-5_24}{\textbf{AIAI 2024}} and further extended in my paper \href{https://sciety-labs.elifesciences.org/articles/by?article_doi=10.21203/rs.3.rs-7166037/v1}{\textbf{(JAAMAS 2025, under revision)}}, which formalizes the underlying optimization problem and its connection to Dec-POMDP representations.

\bigskip
\textbf{2. Organization-guided MARL}

\noindent
To address non-stationarity and improve convergence in multi-agent learning, I proposed an \href{https://github.com/julien6/MOISE-MARL}{\textbf{organization-guided MARL framework}} that injects symbolic constraints into policy optimization. Each agent’s policy is conditioned by organizational roles, sub-goals, and cooperation protocols derived from the \textit{MOISE+} model. This approach yields faster convergence, better coordination, and improved safety in constrained environments. Theoretical and experimental analyses were published in \href{https://arxiv.org/abs/2503.23615}{\textbf{AAMAS 2025}} and \href{https://arxiv.org/abs/2505.21559}{\textbf{IEEE CLOUD 2025}}. The key insight is that structured priors can act as inductive biases, stabilizing learning while maintaining interpretability — an idea that aligns naturally with iTHEMS’ interest in combining deep learning and formal reasoning.

\bigskip
\textbf{3. World-model-based multi-agent learning}

\noindent
Building upon the concept of \textbf{World Models}~(Ha et al., 2018), I extended model-based RL to multi-agent environments by learning latent representations of collective dynamics. These models approximate environment transitions through recurrent or variational autoencoder components (\textit{PyTorch}, \textit{JAX}), providing predictive and generative capabilities. They also serve as digital twins that enable planning, transfer learning, and explainability. Current work explores \textbf{neuro-symbolic world models} that combine graph-based encodings with organizational abstractions, producing semantically grounded latent spaces that can support reasoning and verification. Such world-model architectures could be leveraged to study emergent coordination in distributed physical or biological systems, echoing the interdisciplinary scope of the AI for Science Team.

\bigskip
\textbf{4. Behavior analysis and explainability}

\noindent
To improve the interpretability of MARL policies, I designed an unsupervised trajectory-analysis method that extracts implicit roles and objectives from execution traces. By clustering latent trajectories and mapping them to organizational descriptors, the method reveals emergent behaviors and their alignment with intended missions. This contributes to explainable MARL and supports model validation in safety-critical or high-stakes domains. These ideas are being further developed within my ongoing collaborations with \textit{Thales Germany} on behavior analytics for learning agents.

\bigskip
\textbf{5. Broader context and relevance to RIKEN iTHEMS}

\noindent
My research contributes to the foundational understanding of multi-agent learning systems and their application to complex, distributed environments. Within RIKEN iTHEMS, I aim to extend these results into a broader program on \textbf{AI for complex systems}, combining theoretical analysis, representation learning, and interdisciplinary modeling. In particular, I plan to:
\begin{itemize}
    \item study the theoretical properties of organization-guided MARL — convergence, equilibrium selection, and constrained optimization under Dec-POMDP settings;
    \item develop \textbf{neuro-symbolic world models} that integrate learned dynamics with interpretable, graph-based causal representations, linking low-level perception to high-level priors; and
    \item validate these methods in multi-agent environments such as \textbf{crowd and traffic dynamics, swarm robotics, ecological collectives, and socio-economic networks} where emergent behaviors can be modeled and controlled.
\end{itemize}
My experience with GPU-based computation, deep-learning libraries (\textit{PyTorch}, \textit{JAX}), and simulation frameworks (\textit{MARLlib}, \textit{PettingZoo}) provides the technical foundation for implementing these research lines in collaboration with iTHEMS researchers from mathematics, physics, and life sciences.

\bigskip
\textbf{Conclusion}

\noindent
Overall, my work strives to bridge symbolic reasoning and learning-based optimization to create interpretable, theoretically grounded, and experimentally validated multi-agent systems. At RIKEN iTHEMS, I wish to contribute to the development of artificial intelligence as a \textbf{scientific framework for understanding and designing complex natural phenomena}, thereby advancing both the theory and the practical impact of multi-agent learning.

\makeletterclosing

\end{document}
