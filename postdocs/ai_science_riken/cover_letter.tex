\documentclass[11pt,a4paper,sans]{moderncv}        

\moderncvstyle{banking}
\moderncvcolor{red}                                
\usepackage{dirtytalk}
\usepackage[utf8]{inputenc}
\usepackage[scale=0.9]{geometry}
\usepackage{graphicx}
\usepackage{ragged2e}

\AfterPreamble{
  \hypersetup{
    colorlinks=false,        % désactive les liens colorés (sinon pas de cadre)
    pdfborder={0 0 1},       % active un cadre autour des liens
    pdfborderstyle={/S/U/W 1}, % facultatif : style souligné
    linkbordercolor={0 0 1}, % cadre bleu pour liens internes
    urlbordercolor={0 0 1},  % cadre bleu pour les URL
    citebordercolor={0 0 1}  % cadre bleu pour citations (si utilisées)
  }
}

\name{Julien}{Soulé}
\title{PhD Student -- Thales/UGA}
\address{35 Rue Mathieu de la Drôme, Valence}{26000}{France}

\phone[mobile]{06 77 63 12 13}
\email{julien.soule@hotmail.fr}
\homepage{julien6.github.io/home/}

%----------------------------------------------------------------------------------
%            content
%----------------------------------------------------------------------------------
\begin{document}

\recipient{\vspace{-0.9cm}Search Committee}{\textit{RIKEN Center for Interdisciplinary Theoretical and Mathematical Sciences (iTHEMS)}\\Wako Campus -- Saitama, Japan}
\date{\today}
\opening{Dear Members of the Search Committee,}
\closing{{I would be honored to join RIKEN iTHEMS and to advance its vision of combining mathematics, theoretical sciences, and AI for fundamental discovery. I look forward to the opportunity to discuss how my research can complement and enhance the AI for Science Team’s ongoing projects.}\\[0.6cm]Sincerely,\\[0.1cm]\vspace{-0.6cm}}

\makelettertitle

\justifying

I am writing to express my strong interest in the Postdoctoral Researcher position (Job No.~25-1477) within the \textit{AI for Science Team} at RIKEN iTHEMS. My doctoral research at Université Grenoble Alpes and Thales LAS focuses on Multi-Agent Reinforcement Learning (MARL), world-model architectures, and organizational modeling, aiming to make multi-agent learning interpretable and systematically designable. This combination of foundational AI and cross-disciplinary modeling strongly resonates with RIKEN’s mission to advance AI for scientific discovery.

During my PhD, I developed a methodological framework for assisting the design and training of multi-agent systems, implemented in our open-source platform \textit{CybMASDE}. It combines formal modeling (Dec-POMDPs), MARL training (JaxMARL, PettingZoo), and unsupervised behavior analysis to bridge symbolic and data-driven approaches. I proposed organization-guided MARL techniques that integrate symbolic constraints (roles, missions) into learning dynamics, and world-model architectures that capture environmental dynamics as multi-agent digital twins. These contributions were published or submitted to AIAI~2024, AAMAS~2025, IEEE~CLOUD~2025, and JAAMAS~2025 (major revision). They lay the foundation for my future research on interpretable and safe multi-agent learning systems.

At RIKEN, I wish to extend these methods toward \textit{AI for Science}, developing neuro-symbolic world models that blend learned representations with scientific constraints and physical laws. My goal is to build AI systems capable of reasoning over structured scientific domains---for example, modeling interacting physical or biological entities as multi-agent processes---and to contribute to the theoretical understanding of learning in structured, multi-scale systems. Such work naturally connects to iTHEMS’s interdisciplinary mission and ongoing efforts in deep learning, simulation, and computational science.

My experience in industrial and academic collaborations has taught me to navigate complex interdisciplinary contexts and translate theoretical advances into applied frameworks. At Thales, I worked on reinforcement-learning methods for autonomous systems and cyber-physical security, and I collaborated with Thales Germany on agent behavior analysis---experiences that sharpened my interest in bridging AI methodology and scientific applications. I am confident that this background will allow me to contribute effectively to RIKEN’s AI for Science initiative.

\makeletterclosing

\end{document}
