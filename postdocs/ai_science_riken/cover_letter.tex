\documentclass[11pt,a4paper,sans]{moderncv}        

\moderncvstyle{banking}
\moderncvcolor{red}                                
\usepackage{dirtytalk}
\usepackage[utf8]{inputenc}
\usepackage[scale=0.85]{geometry}
\usepackage{graphicx}
\usepackage{ragged2e}
\usepackage{setspace}
\usepackage{parskip}
\setstretch{1.05}

\AfterPreamble{
  \hypersetup{
    colorlinks=false,
    pdfborder={0 0 1},
    pdfborderstyle={/S/U/W 1},
    linkbordercolor={0 0 1},
    urlbordercolor={0 0 1},
    citebordercolor={0 0 1}
  }
}

\name{Julien}{Soulé}
\title{PhD Student}
\address{35 Rue Mathieu de la Drôme, Valence}{26000}{France}
\phone[mobile]{06 77 63 12 13}
\email{julien.soule@hotmail.fr}
\homepage{julien6.github.io/home/}

\begin{document}

\recipient{\vspace{-0.7cm}AI for Science Team, iTHEMS}{RIKEN~Center~for Interdisciplinary~Theoretical~and~Mathematical~Sciences\\Wako, Saitama, Japan}
\date{\today}
\opening{Dear Members of the AI for Science Team,}
\closing{{I look forward to the opportunity to discuss my background and research ideas with you.}\\[0.6cm]Sincerely,\vspace{-1cm}}

\makelettertitle
\justifying

\vspace{-0.1cm}
\noindent
I am writing to express my sincere motivation to apply for the \textbf{Postdoctoral Researcher position (Job No.~25-1477)} within the \textit{AI for Science Team} at the \textit{RIKEN Center for Interdisciplinary Theoretical and Mathematical Sciences (iTHEMS)}. I am deeply inspired by iTHEMS' mission to advance the foundations of artificial intelligence and its applications in the natural sciences. This vision resonates strongly with my own research trajectory, which seeks to harness the efficiency of learning-based optimization while grounding it in symbolic reasoning.

During my doctoral studies at \href{https://www.univ-grenoble-alpes.fr/english/}{\textit{Université Grenoble Alpes}} and within the \href{https://www.thalesgroup.com/}{\textit{Thales~LAS}} company (France), I have been developing a \textbf{methodological framework for assisting Multi-Agent System (MAS) design}, implemented in the open-source \href{https://github.com/julien6/CybMASDE}{\textbf{CybMASDE platform}}. This framework integrates formal modeling, \textbf{Multi-Agent Reinforcement Learning (MARL)}, and unsupervised behavior analysis. Its purpose is to make multi-agent learning more interpretable, controllable, and systematically designable, an approach that connects with iTHEMS' vision of AI as a scientific methodology, especially for discovery in fields involving emergent phenomena.
%
Through this work, I contributed to several research directions, including a \textbf{MARL-assisted MAS design methodology} that formulates MAS design as a constrained policy-optimization problem \href{https://sciety-labs.elifesciences.org/articles/by?article_doi=10.21203/rs.3.rs-7166037/v1}{\textbf{(JAAMAS~2025, under revision)}}, \href{https://link.springer.com/chapter/10.1007/978-3-031-63223-5_24}{(AIAI~2024)}; an \href{https://github.com/julien6/MOISE-MARL}{\textbf{organization-guided MARL framework}} that embeds symbolic roles and goals to improve stability and explainability \href{https://arxiv.org/abs/2503.23615}{(AAMAS~2025)}; and an extended \textbf{World Model}~\href{https://link.springer.com/chapter/10.1007/978-3-031-63223-5_24}{(Ha \textit{et al.}, 2018)} to multi-agent settings for learning compact and causal representations \href{https://arxiv.org/abs/2505.21559}{(IEEE CLOUD~2025)}.

I am motivated by using \textbf{AI for complex systems} to understand the dynamics of complex systems that exhibit emergent self-organization and learn how to design them to achieve collective goals. This broader program could span areas such as crowd and traffic dynamics, swarm robotics, ecological collectives, and socio-economic networks. One fundamental goal is to establish a formal understanding of how \textbf{organizational structures} can stabilize and guide MARL dynamics, by analyzing convergence, equilibrium selection, and fairness. I also seek to design semantically grounded \textbf{neuro-symbolic world models} that integrate learned dynamics with interpretable, sub-symbolic reasoning to enhance generalizability and verification. My experience with GPU-based clusters (\textit{Kubernetes}), deep learning libraries (\textit{PyTorch}, \textit{scikit-learn}, \textit{JAX}/\textit{JaxMARL}), and simulation environments (\textit{Gym}/\textit{PettingZoo}) provides a foundation for implementing these approaches with iTHEMS researchers.

Having worked simultaneously in academia and with \textit{Thales~LAS}, I am comfortable navigating between theory and applied objectives. Industrial and academic collaborations (\textit{Grenoble INP--Esisar}) have further reinforced my willingness to work across related multi-agent domains, and to exchange with experts of diverse backgrounds.
%
In the long term, I aspire to pursue an academic career that bridges \textbf{theoretical RL/MARL} and \textbf{research on MASs}. Joining RIKEN iTHEMS represents a unique opportunity to collaborate with scientists in developing AI not merely as a computational tool, but as a genuine framework for scientific reasoning.



\makeletterclosing

\end{document}
