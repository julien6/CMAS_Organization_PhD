
\begin{table*}[t!]

    \caption{Un aperçu de quelques organisations et des environnement hôtes utilisés dans les SMA de cyber-défense étudiés}

    \begin{tabularx}{\linewidth}{
    >{\raggedright\arraybackslash\hsize=.5\hsize}X
    >{\raggedright\arraybackslash\hsize=.6\hsize}X
    >{\raggedright\arraybackslash\hsize=.6\hsize}X
    >{\raggedright\arraybackslash\hsize=.6\hsize}X
    % >{\raggedright\arraybackslash}X
    >{\raggedright\arraybackslash\hsize=.2\hsize}X}
    \toprule

{ \textbf{Organisation}}
& {  \textbf{Avantages principaux}}
& {  \textbf{Inconvénients principaux}}
& {  \textbf{Environnement}}
% & {  \textbf{ Objectifs de cyber-défense suggerés }}
&  {  \textbf{Travaux}}
\\ \midrule

{ Centralisé}
& {  Haute précision pour l'analyse de la situation}
& {  Single-Point-Of-Failure (SPOF), manque de scalabilité}
& {  Petit à moyenne taille, non ouvert, petite entreprise}
% & {  \textbf{Objectifs de type (R1)} : détection d'intrusion, surveillance du réseau}
& {  \cite{vasilomanolakis2015taxonomy, gorodetski2003multi, de2017distributed}}
\\

{ Hiérarchique (distribué)}
& {  Évolutivité, décomposition des tâches}
& {  Perte d'informations, goulots d'étranglement, retards}
& {  Taille moyenne à grande, ouvert, peu de variations}
% & {  \textbf{Objectifs de type (R1) et (R2)} : surveillance du réseau, sauvegardes régulières, contrôles d'accès, correctifs de cyber-défense}
& {  \cite{holloway2009self, lamont2009military}}
\\

{ Décentralisé (Peer-to-Peer)}
& {  Structure non définie a priori, Hautement adaptatif}
& {  Contrôle de l'organisation limitée, intensité de communication}
& {  Ouvert, toute taille, fortes variations}
% & {  \textbf{Objectifs de type (R3)} : reconnaissance de menaces, adaptation aux cyber-attaques}
& {  \cite{holloway2019self, haack2011ant, morteza2015method}}
\\

{ Coalition}
& {  Optimisation de l'organisation autour des tâches}
& {  Peu adapté sur le long terme}
& {  Toute taille, ouvert, peu de variations, peu de ressources}
% & {  \textbf{Objectifs de type (R3)} : contre-mesures de sécurité adaptées, apprentissage des cyber-attaques}
& {  \cite{carvalho2011evolutionary}}
\\

{ Équipes}
& {  Bonne performance pour des tâches régulières}
& {  Haute intensité de communication}
& {  Ouvert, hétérogène, toute taille, peu de variations}
% & {  \textbf{Objectifs de type (R1) et (R2)} : détection de menaces possibles, application de contre-mesures}
& {  \cite{akandwanaho2018generic}}
\\

{ Marché}
& {  Organisation optimisée par concurrence, bonne gestion des agents}
& {  Processus d'allocation complexe et long}
& {  Toute taille, ouvert, peu de variations, peu de ressources}
% & {  \textbf{Objectifs de type (R3)} : investigations forensiques, stratégies de cyber-défense}
& {  \cite{demir2021adaptive}}
\\
        \bottomrule
        
    \end{tabularx}
    \label{tab:general-overview}
\end{table*}
