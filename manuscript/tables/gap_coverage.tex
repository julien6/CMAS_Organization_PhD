\begin{table}[ht]
  \centering
  \caption{Synthèse des verrous identifiés et des besoins associés par hypothèse}
  \label{tab:verrous_hypotheses}
  \renewcommand{\arraystretch}{1.3}
  \begin{tabularx}{\textwidth}{cXXX}
    \hline
    \textbf{Hyp.}  & \textbf{Verrou identifié}                                                                                                      & \textbf{Limites des travaux existants}                                                                                                                 & \textbf{Besoins méthodologiques}                                                                                                                      \\
    \hline
    \textbf{H-MOD} & Absence d'un cadre formel unifié pour représenter la conception comme un problème d'optimisation multi-agent sous contraintes. & Les Dec-POMDP ne modélisent pas directement la conception~; les contraintes organisationnelles ne sont pas intégrées aux formalismes de planification. & Formaliser la conception de SMA comme un Dec-POMDP enrichi, intégrant à la fois la dynamique simulée et les contraintes organisationnelles.           \\

    \textbf{H-TRN} & Manque de méthodes pour modéliser automatiquement un environnement multi-agent à partir de trajectoires partielles.            & Les World Models sont limités au mono-agent ou à des contextes simples~; peu d'approches exploitent des observations distribuées.                      & Apprendre un World Model multi-agent à partir d'expériences collectives, structuré autour de représentations latentes adaptées.                       \\

    \textbf{H-ANL} & Absence de mécanismes pour guider l'apprentissage MARL par des spécifications symboliques (rôles, missions).                   & Les travaux en Constrained RL ne prennent pas en compte les structures organisationnelles~; l'intégration de MOISE+ reste limitée à l'exécution.       & Introduire des contraintes symboliques issues de MOISE+ dans le processus MARL, comme guide à l'apprentissage et au filtrage d'actions.               \\

    \textbf{H-TRF} & Manque de méthodes pour analyser les comportements émergents à l'échelle organisationnelle.                                    & L'explicabilité reste locale (agent, action)~; peu de travaux infèrent des rôles ou objectifs implicites.                                              & Inférer automatiquement des rôles, missions ou objectifs à partir de trajectoires, pour évaluer ou générer une structure organisationnelle émergente. \\
    \hline
  \end{tabularx}
\end{table}
