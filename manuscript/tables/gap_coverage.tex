\begin{table}[ht]
  \centering
  \caption{Synthèse des verrous identifiés et des besoins associés par hypothèse}
  \label{tab:verrous_hypotheses}
  \renewcommand{\arraystretch}{1}
  \resizebox{\textwidth}{!}{%
    \footnotesize
    \begin{tabularx}{\textwidth}{cXXX}
      \hline
      \textbf{Hyp.}  & \textbf{Verrou identifié}                                                                                                                                 & \textbf{Limites des travaux existants}                                                                                                                                                                                               & \textbf{Besoins méthodologiques}                                                                                                                                                                                                                         \\
      \hline
      \textbf{H-TRF} & Absence d'un cadre formel unifié pour représenter la conception comme un problème d'optimisation multi-agent sous contraintes.                            & Les modèles Markoviens actuels ne modélisent pas directement la conception~; les contraintes organisationnelles ne sont pas intégrées et manque d'intégration avec l'environnement réel.                                             & Formaliser la conception de SMA intégrant à la fois la dynamique simulée de l'environnement et les politiques des agents.                                                                                                                                \\

      \textbf{H-MOD} & Manque de méthodes pour modéliser automatiquement un environnement multi-agent à partir de trajectoires partielles ou faciliter la modélisation manuelle. & Les World Models sont limités au mono-agent ou à des contextes simples~; peu d'approches exploitent des observations distribuées~; les modèles Markoviens actuels restent encore trop simples pour aider à la modélisation manuelle. & Apprendre un World Model multi-agent à partir d'expériences collectives, structuré autour de représentations latentes adaptées~; proposer un modèle Markovien utilisable pour modéliser manuellement un environnement de Cyberdéfense en une simulation. \\

      \textbf{H-TRN} & Absence de mécanismes pour guider l'apprentissage MARL par des spécifications symboliques (rôles, missions).                                              & Les travaux en Constrained RL ne prennent pas en compte les structures organisationnelles~; l'intégration de spécifications organisationnelles reste limitée à l'exécution.                                                          & Introduire des contraintes symboliques dans le processus MARL, comme guide à l'apprentissage et au filtrage d'actions.                                                                                                                                   \\

      \textbf{H-ANL} & Manque de méthodes pour analyser les comportements émergents à l'échelle organisationnelle.                                                               & L'explicabilité reste locale (agent, action)~; peu de travaux infèrent des rôles ou objectifs implicites.                                                                                                                            & Inférer automatiquement des rôles, missions ou objectifs à partir de trajectoires, pour évaluer ou générer une structure organisationnelle émergente.                                                                                                    \\
      \hline
    \end{tabularx}
  }
\end{table}
