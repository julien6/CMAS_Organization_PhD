\begin{table}[h!]
    \centering
    \caption{Taxonomie de la méthode MAMAD avec activités, sous-activités et acronymes}
    \label{tab:mamad_taxonomy}
    \renewcommand{\arraystretch}{1.3}
    \resizebox{\textwidth}{!}{%
        \begin{tabular}{|p{2.5cm}|p{3.2cm}|p{5.8cm}|p{3cm}|p{3cm}|}
            \hline
            \textbf{Activité} & \textbf{Sous-activité} & \textbf{Description}                                                                                                       & \textbf{Entrées requises}                                                & \textbf{Sorties produites}                                                  \\
            \hline
            \acn{MOD}         & \acn{MOD-MAN}          & Création manuelle du modèle simulé via un framework markovien générique (Dec-POMDP étendu avec MOISE+MARL).                & Description informelle de l’environnement, des objectifs et contraintes. & Modèle formel exploitable par MARL.                                         \\
            \cline{2-5}
                              & \acn{MOD-AUT}          & Génération automatique du modèle simulé à partir de traces collectées (World Models, VAE+RNN, LSTM, etc.).                 & Traces (actions, observations) collectées dans l’environnement réel.     & Modèle simulé approximatif (fonction de transition et d’observation).       \\
            \hline
            \acn{TRN}         & \acn{TRN-CON}          & Apprentissage multi-agent guidé par spécifications organisationnelles MOISE+MARL (rôles, missions, contraintes).           & Modèle simulé + spécifications MOISE+MARL.                               & Politiques conjointes respectant contraintes et objectifs organisationnels. \\
            \cline{2-5}
                              & \acn{TRN-UNC}          & Apprentissage multi-agent sans contraintes organisationnelles (MOISE+MARL non utilisé ou vide).                            & Modèle simulé.                                                           & Politiques conjointes optimisées uniquement selon la récompense.            \\
            \hline
            \acn{ANL}         & \acn{ANL-MAN}          & Analyse qualitative manuelle des politiques apprises (visualisation, interprétation).                                      & Politiques apprises.                                                     & Interprétation humaine des comportements.                                   \\
            \cline{2-5}
                              & \acn{ANL-SMAN}         & Analyse assistée par TEMM suivie d’un ajustement manuel.                                                                   & Politiques + données de trajectoires.                                    & Rôles et objectifs implicites affinés par l’utilisateur.                    \\
            \cline{2-5}
                              & \acn{ANL-AUT}          & Analyse automatique complète via Auto-TEMM (rôles implicites, objectifs intermédiaires, évaluation SOF/FOF).               & Politiques + données de trajectoires.                                    & Rapport d’analyse automatisé.                                               \\
            \hline
            \acn{TRF}         & \acn{TRF-MAN}          & Transfert manuel des politiques dans l’environnement réel et mise à jour manuelle du modèle simulé.                        & Politiques apprises.                                                     & Politiques déployées et environnement ajusté si nécessaire.                 \\
            \cline{2-5}
                              & \acn{TRF-AUT}          & Transfert et synchronisation automatique (cadre logiciel automatisant le déploiement et l’actualisation du modèle simulé). & Politiques apprises + framework de déploiement.                          & Politiques déployées et modèle simulé mis à jour automatiquement.           \\
            \hline
        \end{tabular}
    }
\end{table}