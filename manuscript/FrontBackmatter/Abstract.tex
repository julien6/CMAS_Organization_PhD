%*******************************************************
% Abstract
%*******************************************************
%\renewcommand{\abstractname}{Abstract}
\pdfbookmark[1]{Abstract}{Abstract}
\begingroup
\let\clearpage\relax
\let\cleardoublepage\relax
\let\cleardoublepage\relax

\begin{otherlanguage}{ngerman}
    \pdfbookmark[1]{Zusammenfassung}{Zusammenfassung}
    \chapter*{Résumé}

    Face à la complexité croissante des menaces en cybersécurité, les approches centralisées montrent leurs limites pour protéger efficacement des systèmes distribués et dynamiques. Cette thèse explore une approche distribuée fondée sur des Systèmes Multi-Agents (SMA), capables de détecter, répondre et s'adapter collectivement à des attaques autonomes et évolutives.
    %
    L'objectif central est de guider la conception organisationnelle d'un SMA pour la cyberdéfense. Les approches actuelles montrent un compromis difficile entre contrôle (avec les modèles symboliques) et performance (avec les approches par apprentissage). Pour dépasser cette tension, la thèse propose une méthode hybride combinant modèles organisationnels symboliques et apprentissage par renforcement multi-agent (MARL).

    La clé de cette méthode consiste à reformuler la conception d'un SMA comme un \textit{problème d'optimisation sous contraintes}, dans lequel la politique conjointe des agents est apprise tout en respectant des contraintes organisationnelles exprimant les exigences du concepteur. Cette approche requiert à la fois une modélisation fidèle de l'environnement et une capacité à analyser et contrôler les comportements obtenus.
    %
    La méthode s'organise en quatre phases : (i) \textbf{modélisation} de l'environnement cible à l'aide de techniques de type \textit{World Models}, pour obtenir une version simulée utilisable à l'entraînement ; (ii) \textbf{entraînement} des agents via MARL, avec intégration de contraintes issues du modèle organisationnel MOISE+ ; (iii) \textbf{analyse} des politiques apprises, en extrayant rôles et objectifs implicites via des méthodes non supervisées sur les trajectoires ; (iv) \textbf{transfert} des résultats dans l'environnement réel, avec mise à jour continue des modèles et politiques à partir des données collectées sur le terrain.

    Un outil logiciel, \textit{CybMASDE}, a été développé pour mettre en œuvre cette méthode. Elle a été évaluée sur trois cas d'usage : un essaim de drones, une infrastructure d'entreprise et une architecture de micro-services. Les résultats montrent une amélioration notable en termes de résilience, d'adaptabilité et d'autonomie par rapport aux approches centralisées.
    %
    Enfin, la thèse ouvre plusieurs pistes de recherche, notamment pour améliorer la modélisation de l'environnement avec des connaissances expertes, renforcer la robustesse de l'apprentissage dans des environnements dynamiques, et automatiser l'analyse organisationnelle à l'aide de représentations latentes.

    \medskip

    \

    \noindent MOTS-CLEFS :
    Système Multi-Agent \raisebox{0.25ex}{\tiny$\bullet$} Cyberdéfense \raisebox{0.25ex}{\tiny$\bullet$} Apprentissage

    \hskip6em\relax par Reinforcement Multi-Agent {\tiny$\bullet$} Conception assisté

\end{otherlanguage}

\vfill

\chapter*{Abstract}

As cyber threats become increasingly complex and frequent, centralized approaches to cybersecurity are proving inadequate for protecting distributed and dynamic systems. This thesis explores a distributed solution based on Multi-Agent Systems (MAS), capable of collectively detecting, responding to, and adapting to autonomous and evolving attacks.

The central objective is to guide the organizational design of a MAS for cybersecurity. Current approaches struggle to balance control (via symbolic models) and performance (via learning-based methods). To overcome this tension, the thesis proposes a hybrid method combining symbolic organizational models with Multi-Agent Reinforcement Learning (MARL).

The key idea is to frame the design of a MAS as a \textit{constrained optimization problem}, where the joint policy of agents is learned while satisfying organizational constraints that reflect the designer’s requirements. This approach necessitates both a faithful simulation of the target environment and mechanisms to analyze and control the resulting behaviors.

The proposed method is structured into four phases: (i) \textbf{modeling} the target environment using \textit{World Models} techniques to create a high-fidelity simulation; (ii) \textbf{training} the agents with MARL, incorporating constraints from the MOISE+ organizational model; (iii) \textbf{analyzing} the learned policies by extracting implicit roles and objectives using unsupervised trajectory analysis; and (iv) \textbf{transferring} the results to the real environment, continuously updating the simulated model and deployed policies with real-world data.

A software tool, \textit{CybMASDE}, was developed to support this method. It was evaluated on three case studies: a drone swarm, an enterprise infrastructure, and a microservices orchestration scenario. Results demonstrate significant improvements in resilience, adaptability, and autonomy compared to centralized systems.

The thesis concludes with several research perspectives, including enhancing environment modeling with expert knowledge, improving learning robustness in dynamic settings, and automating organizational analysis through latent representations.

\medskip

\

\noindent KEYWORDS :
Multi-Agent Systems \raisebox{0.25ex}{\tiny$\bullet$} Cyberdefense \raisebox{0.25ex}{\tiny$\bullet$} Multi-Agent

\hskip5.7em\relax Reinforcement Learning {\tiny$\bullet$} Assisted-Design

\endgroup

\vfill