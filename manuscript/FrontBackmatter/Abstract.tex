%*******************************************************
% Abstract
%*******************************************************
\renewcommand{\abstractname}{Abstract}
\pdfbookmark[1]{Résumé}{Résumé}
\begingroup
\let\clearpage\relax
\let\cleardoublepage\relax
\let\cleardoublepage\relax

\begin{otherlanguage}{ngerman}
    % \pdfbookmark[1]{Résumé}{Résumé}
    \chapter*{Résumé}

    Face à la complexité croissante des menaces en Cybersécurité, les approches centralisées montrent leurs limites pour protéger efficacement des systèmes distribués et dynamiques. Cette thèse explore une approche distribuée fondée sur des \acn{SMA}, capables de détecter, répondre et s'adapter collectivement à des attaques autonomes et évolutives.
    %
    L'objectif central est de guider la conception d'un SMA pour la cyberdéfense en trouvant un mécanisme d'organisation adapté aux contraintes des concepteurs et de l'environnement. Les approches actuelles montrent un compromis difficile entre contrôle (avec les modèles symboliques) et performance (avec les approches par apprentissage). Pour dépasser cette tension, la thèse propose une méthode hybride combinant un modèle organisationnel symbolique et \acn{MARL}.

    La clé de cette méthode consiste à voir la conception d'un SMA au travers d'un \textit{problème d'optimisation sous contraintes}, dans lequel la politique conjointe des agents est apprise tout en respectant des contraintes organisationnelles exprimant les exigences du concepteur. Cette approche requiert à la fois une modélisation fidèle de l'environn\-ement et une capacité à analyser et contrôler les comportements obtenus.
    %
    La méthode intègre les différents travaux selon quatre activités : (i) \textbf{modélisation} de l'environnement cible à l'aide de techniques manuelles ou de type \textit{World Models}, pour obtenir une version simulée~; (ii) \textbf{entraînement} des agents via MARL, avec intégration de contraintes issues du modèle organisationnel $\mathcal{M}OISE^+$~; (iii) \textbf{analyse} des politiques apprises, en extrayant rôles et objectifs implicites via des méthodes non supervisées sur les trajectoires~; (iv) \textbf{transfert} des résultats dans l'environnement réel, avec mise à jour continue des modèles et politiques à partir des données collectées sur le terrain.

    Un outil logiciel, a été développé pour mettre en œuvre cette méthode, et appliqué à trois cas d'usage : un essaim de drones, une infrastructure d'entreprise et une architecture de micro-services. Les résultats montrent une amélioration en termes de résilience, d'adaptabilité et d'autonomie par rapport aux approches centralisées.
    %
    Enfin, la thèse ouvre plusieurs pistes de recherche, notamment pour améliorer la modélisation de l'environnement avec des connaissances expertes, renforcer la robustesse de l'apprentissage dans des environnements dynamiques, et automatiser l'analyse organisationnelle à l'aide de représentations latentes.

    \medskip

    \

    \noindent MOTS-CLEFS :
    Système Multi-Agent \raisebox{0.25ex}{\tiny$\bullet$} Cyberdéfense \raisebox{0.25ex}{\tiny$\bullet$} Apprentissage

    \hskip6em\relax par Reinforcement Multi-Agent {\tiny$\bullet$} Conception assisté

\end{otherlanguage}

\vfill

\chapter*{Abstract}

Faced with the growing complexity of cybersecurity threats, centralized approaches are showing their limits in effectively protecting distributed and dynamic systems. This thesis explores a distributed approach based on \acn{MAS}, capable of collectively detecting, responding to, and adapting to autonomous and evolving attacks.
%
The central objective is to guide the design of a MAS for cyber defense by finding an organizational mechanism suited to the constraints of both the designers and the environment. Current approaches reveal a difficult trade-off between control (with symbolic models) and performance (with learning-based approaches). To overcome this tension, the thesis proposes a hybrid method that combines a symbolic organizational model with \acn{MARL}.

The key to this method is to view MAS design as a \textit{constrained optimization problem}, in which the agents’ joint policy is learned while respecting organizational constraints that express the designer’s requirements. This approach requires both a faithful modeling of the environment and the ability to analyze and control the resulting behaviors.
%
The method integrates the various contributions across four activities: (i) \textbf{modeling} the target environment using manual techniques or \textit{World Models}, to obtain a simulated version; (ii) \textbf{training} the agents via MARL, with integration of constraints derived from the $\mathcal{M}OISE^+$ organizational model; (iii) \textbf{analyzing} the learned policies by extracting implicit roles and objectives through unsupervised methods applied to trajectories; (iv) \textbf{transferring} the results into the real environment, with continuous updates of models and policies based on data collected in the field.

A software tool was developed to implement this method and applied to three use cases: a drone swarm, a corporate infrastructure, and a microservices architecture. The results show improvements in resilience, adaptability, and autonomy compared to centralized approaches.
%
Finally, the thesis opens several research avenues, particularly to improve environment modeling with expert knowledge, strengthen the robustness of learning in dynamic environments, and automate organizational analysis using latent representations.


\medskip

\

\noindent KEYWORDS :
Multi-Agent Systems \raisebox{0.25ex}{\tiny$\bullet$} Cyberdefense \raisebox{0.25ex}{\tiny$\bullet$} Multi-Agent

\hskip5.7em\relax Reinforcement Learning {\tiny$\bullet$} Assisted-Design

\endgroup

\vfill