%*******************************************************
% Abstract
%*******************************************************
%\renewcommand{\abstractname}{Abstract}
\pdfbookmark[1]{Abstract}{Abstract}
\begingroup
\let\clearpage\relax
\let\cleardoublepage\relax
\let\cleardoublepage\relax

\begin{otherlanguage}{ngerman}
    \pdfbookmark[1]{Zusammenfassung}{Zusammenfassung}
    \chapter*{Résumé}

    Alors que la complexité et la fréquence des menaces Cyber continuent d'augmenter, les approches centralisées de Cyberdefense s'avèrent insuffisantes pour protéger les réseaux distribués et dynamiques. Une approche Multi-Agent de la Cyberdefense offre une solution prometteuse, apportant une résilience, une évolutivité, une autonomie et une adaptabilité accrues face à des attaques de plus en plus sophistiquées.

    Cette thèse se concentre sur les mécanismes permettant de déterminer l'organisation des Systèmes Multi-Agents (SMA) pour la Cyberdefense, en tirant parti de la nature autonome et collaborative des agents pour détecter, répondre et atténuer les menaces en temps réel dans des environnements distribués.
    
    Le concept de SMA pour la Cyberdefense est encore peu exploré dans la littérature, et les questions relatives à sa conception, notamment du point de vue de l'organisation, ne sont pas abordées de manière explicite. Notre revue de littérature a montré que les travaux reposant sur une approche d'IA symbolique montre des limites en termes de coût pour la conception là où les travaux reposant sur une approche connexioniste permettent un coût bien moindre mais en imposant des problèmes d'explicabilité et de contrôle.

    En considérant la recherche d'une organisation comme à la base la conception du SMA de Cyberdéfense, notre contribution prend la forme d'une méthode pour sa conception mêlant une approche hybride entre IA symbolique et connexionniste en mêlant modèle organisationel symbolique et apprentissage par renforcement multi-agent. Le clé de voûte de la méthode est de considérer la recherche d'une organisation non pas comme un problème d'ingénierie mais comme un problème d'optimisation sous-contrainte où la politique conjointe des agents doit être optimisée sous les contraintes qui encodent les exigences de conception optionnelles du concepteur. Cette formalisation s'accompagne alors d'un besoin de modéliser fidèlement l'environnement de déploiement dans la modélisation du problème mais aussi de pouvoir expliquer les solutions proposées après résolution tout en assurant une cohérence entre l'environnement réel et le processus de résolution. Notre méthode se construit donc sur un cadre intégrant quatres activités complémentaires: i) La modélisation de l'environnement de déploiement via des techniques \textquote{World Models} en vue de modéliser le problème d'optimisation ; ii) La résolution de ce problème par l’entraînement de la politique conjointe des agents via des techniques MARL dans l'atteinte d'une récompense cumulée suffisante et auquel nous proposons d'ajouter le modèle organisationnel $\mathcal{M}OISE^+$ pour y intégrer les exigences de conception et permettre leurs contrôle au niveau organisationnel; iii) L'analyse des comportements des agents qui consiste à prendre appui sur la méthode empirique proposé qui utilise des techniques d'apprentissage non-supervisé sur les trajectoires des agents entrainés pour analyser les récurrences dans les comportements des agents et en tirer des spécifications organisationnelles plus précises pour décrire les comportements des agents ; iv) Le transfert qui consiste en deux fonctions: la récupération continue des trajectoires réelles des agents déployés dans l'environnement en vu d'améliorer continuellement la fidélité de l'environnement modélisé dans l'étape de Modélisation ; et la récupération de la dernière politique entraînée et son déploiement dans les agents de l'environnement en vu de prendre en compte d'éventuelles changements encore non-connus de l'environment réel. Cette méthode et les contributions associées sont mis en œuvre sous forme d'un outil destiné au développement de SMA pour la Cyberdefense.
    
    Nous avons validé notre méthode à travers trois études de cas : un essaim de drones, une infrastructure d'entreprise et un scénario dans une architecture de micro-service. Les résultats montrent une efficacité en simulation, notamment en termes d'adaptabilité, d'autonomie et de résilience, par rapport aux systèmes centralisés.

    Bien que la méthode soit effective dans des environments simulés simples, elle reposent sur différentes contributions, ouvrant des axes de recherche encore peu explorés. Ainsi, la modélisation de l'environnement par des techniques \textquote{World Models} nécessite une grande puissance de calcul et est coûteuse en temps tandis qu'elle pourrait bénéficier d'un processus tirant partie d'une connaissance ou heuristiques explicites à même de mitiger ces écueils. De même, même si l'entrainement donne des résultats satisfaisant pour des environments simples et assez stable, son utilisation pour des environments complexes lève encore des défis comme la prise en compte de changements en temps réel dans l'entrainement. De façon similaire, l'analyse nécéssite encore de nombreuses amélioration pour atteindre la capacité d'inférer des spécifications organisationnelles précises et pertinentes sans intervention manuelles qui pourraient être adressées par l'utilisation de représentation latentes.

    \medskip

    \

    \noindent MOTS-CLEFS :
    Système Multi-Agent \raisebox{0.25ex}{\tiny$\bullet$} Cyberdéfense \raisebox{0.25ex}{\tiny$\bullet$} Apprentissage 
    
    \hskip6em\relax par Reinforcement Multi-Agent {\tiny$\bullet$} Conception assisté

\end{otherlanguage}

\vfill

\chapter*{Abstract}

As the complexity and frequency of cyber threats continue to increase, centralized approaches to Cyberdefense are proving inadequate to protect distributed and dynamic networks. A Multi-Agent approach to Cyberdefense offers a promising solution, providing enhanced resilience, scalability, autonomy, and adaptability in the face of increasingly sophisticated attacks.

This thesis focuses on the mechanisms for determining the organization of Multi-Agent Systems (SMA) for Cyberdefense, leveraging the autonomous and collaborative nature of agents to detect, respond to, and mitigate threats in real-time within distributed environments.

The concept of SMA for Cyberdefense is still underexplored in the literature, and design-related questions, particularly from the organizational perspective, are not explicitly addressed.

Our contribution, therefore, takes the form of a method for the design and development of SMA dedicated to Cyberdefense. Our method is based on a framework that models the SMA for Cyberdefense, formalizing the design problem and allowing the application of various resolution techniques in simulation or emulation. We propose an assisted-design approach, drawing on research related to guidance and explainability in reinforcement learning during simulation, to automate the generation of optimized SMA organizations under environmental constraints. This framework and design approach are implemented as a tool for the development of SMA for Cyberdefense.

We validated our method through three case studies: a drone swarm, a corporate infrastructure, and a scenario with a micro-service architecture. The results demonstrate effectiveness in simulation, particularly in terms of adaptability, autonomy, and resilience, compared to centralized systems.

However, the MAMAD method remains limited by the accuracy of the simulated model, hindering transfer to reality. Hybrid learning-reconstruction approaches are underway. Since organizational constraints can hamper adaptability, dynamic modulation is being considered. Role inference, which is still heuristic, will be reinforced by methods based on latent representations.



\medskip

\

\noindent KEYWORDS :
Multi-Agent Systems \raisebox{0.25ex}{\tiny$\bullet$} Cyberdefense \raisebox{0.25ex}{\tiny$\bullet$} Multi-Agent 

\hskip5.7em\relax Reinforcement Learning {\tiny$\bullet$} Assisted-Design

\endgroup

\vfill