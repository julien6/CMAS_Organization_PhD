%*******************************************************
% Abstract
%*******************************************************
%\renewcommand{\abstractname}{Abstract}
\pdfbookmark[1]{Abstract}{Abstract}
\begingroup
\let\clearpage\relax
\let\cleardoublepage\relax
\let\cleardoublepage\relax

\begin{otherlanguage}{ngerman}
    \pdfbookmark[1]{Zusammenfassung}{Zusammenfassung}
    \chapter*{Résumé}

    Alors que la complexité et la fréquence des menaces Cyber continuent d'augmenter, les approches centralisées de Cyberdefense s'avèrent insuffisantes pour protéger les réseaux distribués et dynamiques. Une approche Multi-Agent de la Cyberdefense offre une solution prometteuse, apportant une résilience, une évolutivité, une autonomie et une adaptabilité accrues face à des attaques de plus en plus sophistiquées.

    Cette thèse se concentre sur les mécanismes permettant de déterminer l'organisation des Systèmes Multi-Agents (SMA) pour la Cyberdefense, en tirant parti de la nature autonome et collaborative des agents pour détecter, répondre et atténuer les menaces en temps réel dans des environnements distribués.
    
    Le concept de SMA pour la Cyberdefense est encore peu exploré dans la littérature, et les questions relatives à sa conception, notamment du point de vue de l'organisation, ne sont pas abordées de manière explicite.
    
    Notre contribution prend donc la forme d'une méthode pour la conception et le développement de SMA dédiés à la Cyberdefense. Notre méthode repose sur un cadre modélisant le SMA de Cyberdefense, formalisant ainsi le problème de sa conception et permettant l'application de diverses techniques de résolution en simulation ou en émulation. Nous proposons une approche de conception assistée, s'appuyant sur des travaux relatifs au guidage et à l'explicabilité de l'apprentissage par renforcement en simulation, pour automatiser la génération d'organisations optimisées de SMA sous des contraintes environnementales. Ce cadre et cette approche de conception sont mis en œuvre sous forme d'un outil destiné au développement de SMA pour la Cyberdefense.
    
    Nous avons validé notre méthode à travers trois études de cas : un essaim de drones, une infrastructure d'entreprise et un scénario dans une architecture de micro-service. Les résultats montrent une efficacité en simulation, notamment en termes d'adaptabilité, d'autonomie et de résilience, par rapport aux systèmes centralisés.

    Cependant, la méthode MAMAD reste limitée par la précision du modèle simulé, freinant le transfert au réel. Des approches hybrides apprentissage–reconstruction sont en cours. Les contraintes organisationnelles pouvant nuire à l’adaptabilité, une modulation dynamique est envisagée. L’inférence des rôles, encore heuristique, sera renforcée par des méthodes basées sur des représentations latentes.

    \medskip

    \

    \noindent MOTS-CLEFS :
    Système Multi-Agent \raisebox{0.25ex}{\tiny$\bullet$} Cyberdéfense \raisebox{0.25ex}{\tiny$\bullet$} Apprentissage 
    
    \hskip6em\relax par Reinforcement Multi-Agent {\tiny$\bullet$} Conception assisté

\end{otherlanguage}

\vfill

\chapter*{Abstract}

As the complexity and frequency of cyber threats continue to increase, centralized approaches to Cyberdefense are proving inadequate to protect distributed and dynamic networks. A Multi-Agent approach to Cyberdefense offers a promising solution, providing enhanced resilience, scalability, autonomy, and adaptability in the face of increasingly sophisticated attacks.

This thesis focuses on the mechanisms for determining the organization of Multi-Agent Systems (SMA) for Cyberdefense, leveraging the autonomous and collaborative nature of agents to detect, respond to, and mitigate threats in real-time within distributed environments.

The concept of SMA for Cyberdefense is still underexplored in the literature, and design-related questions, particularly from the organizational perspective, are not explicitly addressed.

Our contribution, therefore, takes the form of a method for the design and development of SMA dedicated to Cyberdefense. Our method is based on a framework that models the SMA for Cyberdefense, formalizing the design problem and allowing the application of various resolution techniques in simulation or emulation. We propose an assisted-design approach, drawing on research related to guidance and explainability in reinforcement learning during simulation, to automate the generation of optimized SMA organizations under environmental constraints. This framework and design approach are implemented as a tool for the development of SMA for Cyberdefense.

We validated our method through three case studies: a drone swarm, a corporate infrastructure, and a scenario with a micro-service architecture. The results demonstrate effectiveness in simulation, particularly in terms of adaptability, autonomy, and resilience, compared to centralized systems.

However, the MAMAD method remains limited by the accuracy of the simulated model, hindering transfer to reality. Hybrid learning-reconstruction approaches are underway. Since organizational constraints can hamper adaptability, dynamic modulation is being considered. Role inference, which is still heuristic, will be reinforced by methods based on latent representations.



\medskip

\

\noindent KEYWORDS :
Multi-Agent Systems \raisebox{0.25ex}{\tiny$\bullet$} Cyberdefense \raisebox{0.25ex}{\tiny$\bullet$} Multi-Agent 

\hskip5.7em\relax Reinforcement Learning {\tiny$\bullet$} Assisted-Design

\endgroup

\vfill