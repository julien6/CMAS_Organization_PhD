%*******************************************************
% Acknowledgments
%*******************************************************
\pdfbookmark[1]{Remerciements}{Remerciements}

% \begin{flushright}{\slshape    
%     We have seen that computer programming is an art, \\ 
%     because it applies accumulated knowledge to the world, \\ 
%     because it requires skill and ingenuity, and especially \\
%     because it produces objects of beauty.} \\ \medskip
%     --- \defcitealias{knuth:1974}{Donald E. Knuth}\citetalias{knuth:1974} \citep{knuth:1974}
% \end{flushright}



\bigskip

\begingroup
\let\clearpage\relax
\let\cleardoublepage\relax
\let\cleardoublepage\relax
\chapter*{Remerciements}

Faire un doctorat peut sembler être un chemin solitaire, mais je ne l'ai certainement pas parcouru seul.

Je tiens tout d'abord à remercier mes directeurs de thèse, Jean-Paul, Michel, Louis-Marie et Paul. Ils m'ont guidé tout au long de ce processus, m'ont permis de découvrir ce qu'est la recherche – y compris les erreurs – mais aussi d'intervenir lorsque c'était nécessaire. Ils ont toujours été là pour moi, même pendant leurs vacances, en prenant le temps de relire mon manuscrit de thèse.

Un grand merci également aux membres de mon jury : X1 et X2, pour leurs relectures, X2 pour avoir assumé le rôle de président du jury et X3 pour avoir révisé mon travail tout au long de mon doctorat.

J'ai rencontré de nombreux étudiants incroyables au cours de mon séjour au laboratoire. Je voudrais surtout mentionner Arthur, Vincent, Thinh et Karem pour m'avoir accueilli. Je tiens à remercier Minh Tuan, Sébastien et Maximilian pour leur gentillesse et pour tous les échanges que nous avons eu.

Je remercie également François Suro et Simon Gay, qui m'ont donné régulièrement des retours et m'ont permis de garder le cap quand j'étais motivée mais manquais encore de confiance en mon travail.

Un grand merci à Clément Raïevsky, Romain Liévin et Oum-el-Kheir Aktouf pour m'avoir aidé dans mes premières expériences d'enseignement. Travailler au LCIS n'aurait pas été aussi facile sans Patricia et Carole qui m'ont aidé pour affronter la montagne de formalités administratives, ainsi qu'Arthur et Marie qui m'ont permis d'utiliser les moyens techniques dont j'avais besoin.

Je tiens également à remercier mes amis Dorian et Thomas, qui sont également sur ce chemin du doctorat, ainsi que Joaquin, qui a toujours été là pour m'aider à faire une pause dans mes réflexions sur mes études.

Et enfin, je dois tout à ma famille, qui m'a soutenue sans réserve tout au long de cette aventure. Rien de tout cela n'aurait été possible sans eux.


\endgroup



