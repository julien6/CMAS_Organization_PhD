\begin{tikzpicture}[
    chapter/.style={draw, fill=blue!10, thick, minimum width=9cm, minimum height=1.2cm, text centered, font=\bfseries},
    section/.style={draw, fill=blue!5, thick, minimum width=8cm, minimum height=1cm, text centered, font=\small},
    arrow/.style={-Latex, thick},
    node distance=0.4cm,
    annotated/.style={above,font=\small\itshape, inner sep=1pt, yshift=0.8cm, xshift=-8cm}
]

% Chapitre 11 : Implémentation et outils
\node[chapter] (ch11) {\parbox{10cm}{Chapitre 11 : CybMASDE : Un framework supportant MAMAD}};
\node[section, below=1cm of ch11, xshift=-2cm] (ch11s1) {Mise en place d'une architecture de services};
\node[section, below=1cm of ch11s1] (ch11s2) {Intégration des World Models multi-agents};
\node[section, below=1cm of ch11s2] (ch11s3) {Intégration du framework MOISE+MARL};

\draw[arrow] ($ (ch11.south) + (4.0,0) $) -- ++(0,0) |- (ch11s1.east) node[annotated] {Architecture logicielle modulaire au cœur du framework.};
\draw[arrow] ($ (ch11.south) + (4.0,0) $) -- ++(0,0) |- (ch11s2.east) node[annotated] {Ajout de capacités prédictives via des modèles du monde.};
\draw[arrow] ($ (ch11.south) + (4.0,0) $) -- ++(0,0) |- (ch11s3.east) node[annotated] {Implémentation de la spécification organisationnelle.};

% Chapitre 12 : Protocole expérimental
\node[chapter, below=1cm of ch11s3, xshift=2cm] (ch12) {\parbox{10cm}{Chapitre 12 : Protocole expérimental}};
\node[section, below=1cm of ch12, xshift=-2cm] (ch12s1) {Objectifs d'évaluation};
\node[section, below=1cm of ch12s1] (ch12s2) {Configurations expérimentale};

\draw[arrow] ($ (ch12.south) + (4.0,0) $) -- ++(0,0) |- (ch12s1.east) node[annotated] {Mesurer la pertinence et l'efficacité de la méthode.};
\draw[arrow] ($ (ch12.south) + (4.0,0) $) -- ++(0,0) |- (ch12s2.east) node[annotated] {Définition des scénarios, agents et métriques.};

% Chapitre 13 : Études de cas
\node[chapter, below=1cm of ch12s2, xshift=2cm] (ch13) {\parbox{10cm}{Chapitre 13 : Études de cas}};
\node[section, below=1cm of ch13, xshift=-2cm] (ch13s1) {Les environnements non-orientés Cyberdéfense};
\node[section, below=1cm of ch13s1] (ch13s2) {Infrastructure d'entreprise};
\node[section, below=1cm of ch13s2] (ch13s3) {Essaim de drones};
\node[section, below=1cm of ch13s3] (ch13s4) {Architecture de microservices};

\draw[arrow] ($ (ch13.south) + (4.0,0) $) -- ++(0,0) |- (ch13s1.east) node[annotated] {Exploration d'environnements standards pour valider la généricité.};
\draw[arrow] ($ (ch13.south) + (4.0,0) $) -- ++(0,0) |- (ch13s2.east) node[annotated] {SMA chargé de défendre une entreprise virtuelle.};
\draw[arrow] ($ (ch13.south) + (4.0,0) $) -- ++(0,0) |- (ch13s3.east) node[annotated] {Coordination d'un essaim de drones défensifs.};
\draw[arrow] ($ (ch13.south) + (4.0,0) $) -- ++(0,0) |- (ch13s4.east) node[annotated] {Maintien de performance dans un système distribué instable.};

% Chapitre 14 : Résultats et synthèse
\node[chapter, below=1cm of ch13s4, xshift=2cm] (ch14) {\parbox{10cm}{Chapitre 14 : Résultats expérimentaux et analyse comparative}};
\node[section, below=1cm of ch14, xshift=-2cm] (ch14s1) {Validation expérimentale des hypotheses};
\node[section, below=1cm of ch14s1] (ch14s2) {Comparaison entre scénarios};
\node[section, below=1cm of ch14s2] (ch14s3) {Discussion des résultats et généralité};

\draw[arrow] ($ (ch14.south) + (4.0,0) $) -- ++(0,0) |- (ch14s1.east) node[annotated] {Analyse quantitative de la performance du SMA.};
\draw[arrow] ($ (ch14.south) + (4.0,0) $) -- ++(0,0) |- (ch14s2.east) node[annotated] {Synthèse des résultats à travers les cas étudiés.};
\draw[arrow] ($ (ch14.south) + (4.0,0) $) -- ++(0,0) |- (ch14s3.east) node[annotated] {Limites de l'approche et pistes de généralisation.};

% Transitions entre chapitres
\draw[arrow] ($ (ch11.south) + (4.5,0) $) -- ($ (ch12.north) + (4.5,0) $) node[annotated, yshift=-0.5cm] {L'outil étant en place, nous définissons le protocole pour l'évaluer.};
\draw[arrow] ($ (ch12.south) + (4.5,0) $) -- ($ (ch13.north) + (4.5,0) $) node[annotated, yshift=-0.5cm] {Le protocole est appliqué à différents environnements.};
\draw[arrow] ($ (ch13.south) + (4.5,0) $) -- ($ (ch14.north) + (4.5,0) $) node[annotated, yshift=-0.5cm] {Les résultats sont analysés pour valider la méthode.};

\end{tikzpicture}
