\begin{tikzpicture}[
        chapter/.style={draw, fill=blue!10, thick, minimum width=9cm, minimum height=1.2cm, text centered, font=\bfseries},
        section/.style={draw, fill=blue!5, thick, minimum width=8cm, minimum height=1cm, text centered, font=\small},
        arrow/.style={-Latex, thick},
        node distance=0.4cm,
        annotated/.style={above,font=\small\itshape, inner sep=1pt, yshift=0.8cm, xshift=-8cm}
    ]

    % Chapitre 11 : Implémentation et outils
    \node[chapter] (ch11) {\parbox{10cm}{Chapitre 11 : CybMASDE : Un framework supportant MAMAD}};
    \node[section, below=1cm of ch11, xshift=-2cm] (ch11s1) {Intégration des différentes contributions};

    \draw[arrow] ($ (ch11.south) + (4.0,0) $) -- ++(0,0) |- (ch11s1.east) node[annotated] {Architecture logicielle modulaire au cœur du framework.};

    % Chapitre 12 : Cadre expérimental
    \node[chapter, below=1cm of ch11s1, xshift=2cm] (ch12) {\parbox{10cm}{Chapitre 12 : Cadre expérimental et d'évaluation}};
    \node[section, below=1cm of ch12, xshift=-2cm] (ch12s1) {Description des ensembles d'environnements et d'algorithmes considérés};
    \node[section, below=1cm of ch12s1] (ch12s2) {Conditions de reproductibilité};
    \node[section, below=1cm of ch12s2] (ch12s3) {Baselines expérimentales};
    \node[section, below=1cm of ch12s3] (ch12s4) {Grille d'évaluation};
    \node[section, below=1cm of ch12s4] (ch12s5) {Protocole d'experimentation et d'évaluation};

    \draw[arrow] ($ (ch12.south) + (4.0,0) $) -- ++(0,0) |- (ch12s1.east) node[annotated] {Définir les briques fondatrices pour l'experimentation.};
    \draw[arrow] ($ (ch12.south) + (4.0,0) $) -- ++(0,0) |- (ch12s2.east) node[annotated] {Récapituler les conditions de reproduction expérimentales};
    \draw[arrow] ($ (ch12.south) + (4.0,0) $) -- ++(0,0) |- (ch12s3.east) node[annotated] {Définir les différentes expérimentations};
    \draw[arrow] ($ (ch12.south) + (4.0,0) $) -- ++(0,0) |- (ch12s4.east) node[annotated] {Définir un cadre commun pour l'évaluation};
    \draw[arrow] ($ (ch12.south) + (4.0,0) $) -- ++(0,0) |- (ch12s5.east) node[annotated] {Rassembler les éléments pour définir un protocole};

    % Chapitre 13 : Études de cas
    \node[chapter, below=1cm of ch12s5, xshift=2cm] (ch13) {\parbox{10cm}{Chapitre 13 : Études de cas}};
    \node[section, below=1cm of ch13, xshift=-2cm] (ch13s1) {Expérimentations sur les environnements non-orientés Cyberdéfense};
    \node[section, below=1cm of ch13s1] (ch13s2) {Expérimentations sur l'environnement Company infrastructure};
    \node[section, below=1cm of ch13s2] (ch13s3) {Expérimentations sur l'environnement Microservices Kubernetes};
    \node[section, below=1cm of ch13s3] (ch13s4) {Expérimentations sur l'environnement Drone Swarm};

    \draw[arrow] ($ (ch13.south) + (4.0,0) $) -- ++(0,0) |- (ch13s1.east) node[annotated] {};
    \draw[arrow] ($ (ch13.south) + (4.0,0) $) -- ++(0,0) |- (ch13s2.east) node[annotated] {};
    \draw[arrow] ($ (ch13.south) + (4.0,0) $) -- ++(0,0) |- (ch13s3.east) node[annotated] {};
    \draw[arrow] ($ (ch13.south) + (4.0,0) $) -- ++(0,0) |- (ch13s4.east) node[annotated] {};

    % Chapitre 14 : Résultats et synthèse
    \node[chapter, below=1cm of ch13s4, xshift=2cm] (ch14) {\parbox{10cm}{Chapitre 14 : Résultats expérimentaux et analyse}};
    \node[section, below=1cm of ch14, xshift=-2cm] (ch14s1) {Résultats et discussion des environnements non-orientés Cyberdéfense};
    \node[section, below=1cm of ch14s1] (ch14s2) {Résultats et discussion de l'environnement Company Infrastructure};
    \node[section, below=1cm of ch14s2] (ch14s3) {Résultats et discussion de l'environnement Microservices Kubernetes};
    \node[section, below=1cm of ch14s3] (ch14s4) {Résultats et discussion de l'environnement Drone Swarm};

    \draw[arrow] ($ (ch14.south) + (4.0,0) $) -- ++(0,0) |- (ch14s1.east) node[annotated] {};
    \draw[arrow] ($ (ch14.south) + (4.0,0) $) -- ++(0,0) |- (ch14s2.east) node[annotated] {};
    \draw[arrow] ($ (ch14.south) + (4.0,0) $) -- ++(0,0) |- (ch14s3.east) node[annotated] {};
    \draw[arrow] ($ (ch14.south) + (4.0,0) $) -- ++(0,0) |- (ch14s4.east) node[annotated] {};

    % Transitions entre chapitres
    \draw[arrow] ($ (ch11.south) + (4.5,0) $) -- ($ (ch12.north) + (4.5,0) $) node[annotated, yshift=-0.5cm] {L'outil étant en place, nous définissons le protocole pour l'évaluer.};
    \draw[arrow] ($ (ch12.south) + (4.5,0) $) -- ($ (ch13.north) + (4.5,0) $) node[annotated, yshift=-0.5cm] {Le protocole est appliqué à différents environnements.};
    \draw[arrow] ($ (ch13.south) + (4.5,0) $) -- ($ (ch14.north) + (4.5,0) $) node[annotated, yshift=-0.5cm] {Les résultats sont analysés pour valider la méthode.};

\end{tikzpicture}
