\begin{tikzpicture}[
    chapter/.style={draw, fill=blue!10, thick, minimum width=9cm, minimum height=1.2cm, text centered, font=\bfseries},
    section/.style={draw, fill=blue!5, thick, minimum width=8cm, minimum height=1cm, text centered, font=\small},
    arrow/.style={-Latex, thick},
    node distance=0.4cm,
    annotated/.style={above,font=\small\itshape, inner sep=1pt, yshift=0.8cm, xshift=-7cm}
]

% Chapitre 6 : MAMAD comme réponse
\node[chapter] (ch6) {\parbox{15cm}{\centering Chapitre 6 : La méthode MAMAD}};
\node[section, below=1cm of ch6, xshift=-2cm] (ch6s1) {Principes et boucle itérative fermée};
\node[section, below=1cm of ch6s1] (ch6s2) {L'architecture générale de la méthode MAMAD};

\draw[arrow] ($ (ch6.south) + (4.0,0) $) -- ++(0,0) |- (ch6s1.east) node[annotated] {TODO};
\draw[arrow] ($ (ch6.south) + (4.0,0) $) -- ++(0,0) |- (ch6s2.east) node[annotated] {TODO};

% Chapitre 7 : Phase 1 — Modélisation
\node[chapter, below=1cm of ch6s2, xshift=2cm] (ch7) {Chapitre 7 : Modéliser l'environnement en simulation};
\node[section, below=1cm of ch7, xshift=-2cm] (ch7s1) {Reconstruction partielle de l’environnement};
\node[section, below=1cm of ch7s1] (ch7s2) {Apprentissage de la dynamique observable};
\node[section, below=1cm of ch7s2] (ch7s3) {Modélisation des exigences de conception et de l'objectif};

\draw[arrow] ($ (ch7.south) + (4.0,0) $) -- ++(0,0) |- (ch7s1.east) node[annotated] {TODO};
\draw[arrow] ($ (ch7.south) + (4.0,0) $) -- ++(0,0) |- (ch7s2.east) node[annotated] {TODO};
\draw[arrow] ($ (ch7.south) + (4.0,0) $) -- ++(0,0) |- (ch7s3.east) node[annotated] {TODO};

% Chapitre 8 : Phase 2 — Apprentissage guidé
\node[chapter, below=1cm of ch7s3, xshift=2cm] (ch8) {Chapitre 8 : Entraînement des politiques sous contraintes};
\node[section, below=1cm of ch8, xshift=-2cm] (ch8s1) {Cadres Markoviens utilisés};
\node[section, below=1cm of ch8s1] (ch8s2) {Guider et contraindre l'apprentissage};
\node[section, below=1cm of ch8s2] (ch8s3) {Une politique conjointe composite pour l'incertitude};

\draw[arrow] ($ (ch8.south) + (4.0,0) $) -- ++(0,0) |- (ch8s1.east) node[annotated] {TODO};
\draw[arrow] ($ (ch8.south) + (4.0,0) $) -- ++(0,0) |- (ch8s2.east) node[annotated] {TODO};
\draw[arrow] ($ (ch8.south) + (4.0,0) $) -- ++(0,0) |- (ch8s3.east) node[annotated] {TODO};

% Chapitre 9 : Phase 3 — Analyse
\node[chapter, below=1cm of ch8s3, xshift=2cm] (ch9) {Chapitre 9 : Analyser et interpréter les comportements émergents};
\node[section, below=1cm of ch9, xshift=-2cm] (ch9s1) {Inférer les rôles et objectifs à partir des trajectoires};
\node[section, below=1cm of ch9s1] (ch9s2) {Mesurer l’adéquation organisationnelle};

\draw[arrow] ($ (ch9.south) + (4.0,0) $) -- ++(0,0) |- (ch9s1.east) node[annotated] {TODO};
\draw[arrow] ($ (ch9.south) + (4.0,0) $) -- ++(0,0) |- (ch9s2.east) node[annotated] {TODO};


% Chapitre 10 : Phase 4 — Transfert
\node[chapter, below=1cm of ch9s2, xshift=2cm] (ch10) {Chapitre 10 : Transférer et superviser en environnement réel};
\node[section, below=1cm of ch10, xshift=-2cm] (ch10s1) {Les différents modes de transfert opérationnel};
\node[section, below=1cm of ch10s1] (ch10s2) {Bouclage entre environnement réel et simulation};

\draw[arrow] ($ (ch10.south) + (4.0,0) $) -- ++(0,0) |- (ch10s1.east) node[annotated] {TODO};
\draw[arrow] ($ (ch10.south) + (4.0,0) $) -- ++(0,0) |- (ch10s2.east) node[annotated] {TODO};


\draw[arrow] ($ (ch6.south) + (6.5,0) $) -- ++(0,0) |- (ch7.east);
\draw[arrow] ($ (ch6.south) + (6.5,0) $) -- ++(0,0) |- (ch8.east);
\draw[arrow] ($ (ch6.south) + (6.5,0) $) -- ++(0,0) |- (ch9.east);
\draw[arrow] ($ (ch6.south) + (6.5,0) $) -- ++(0,0) |- (ch10.east);


\end{tikzpicture}
