\begin{tikzpicture}[
    chapter/.style={draw, fill=blue!10, thick, minimum width=9cm, minimum height=1.2cm, text centered, font=\bfseries},
    section/.style={draw, fill=blue!5, thick, minimum width=8cm, minimum height=1cm, text centered, font=\small},
    arrow/.style={-Latex, thick},
    node distance=0.4cm
]

% Chapitre 6 : MAMAD comme réponse
\node[chapter] (ch6) {Chapitre 6 : MAMAD comme réponse};
\node[section, below=of ch6] (ch6s1) {Principe général et boucle itérative};
\node[section, below=of ch6s1] (ch6s2) {Architecture globale};

\draw[arrow] (ch6) -- (ch6s1);
\draw[arrow] (ch6s1) -- (ch6s2);

% Chapitre 7 : Phase 1 — Modélisation
\node[chapter, below=1.6cm of ch6s2] (ch7) {Chapitre 7 : Phase 1 — Modélisation en simulation};
\node[section, below=of ch7] (ch7s1) {Reconstruction partielle de l’environnement};
\node[section, below=of ch7s1] (ch7s2) {Apprentissage de la dynamique observable};
\node[section, below=of ch7s2] (ch7s3) {Intégration des contraintes de déploiement};

\draw[arrow] (ch7) -- (ch7s1);
\draw[arrow] (ch7s1) -- (ch7s2);
\draw[arrow] (ch7s2) -- (ch7s3);

% Chapitre 8 : Phase 2 — Apprentissage guidé
\node[chapter, below=1.6cm of ch7s3] (ch8) {Chapitre 8 : Phase 2 — Apprentissage guidé/contraint};
\node[section, below=of ch8] (ch8s1) {Cadres d’entraînement MARL utilisés};
\node[section, below=of ch8s1] (ch8s2) {Contraintes organisationnelles dans MARL (OAC, TRF)};
\node[section, below=of ch8s2] (ch8s3) {Politique composite et filtrage d’actions};

\draw[arrow] (ch8) -- (ch8s1);
\draw[arrow] (ch8s1) -- (ch8s2);
\draw[arrow] (ch8s2) -- (ch8s3);

% Chapitre 9 : Phase 3 — Analyse
\node[chapter, below=1.6cm of ch8s3] (ch9) {Chapitre 9 : Phase 3 — Analyse post-entraînement};
\node[section, below=of ch9] (ch9s1) {Méthodes d’inférence des rôles et objectifs};
\node[section, below=of ch9s1] (ch9s2) {Indicateurs : SOF, FOF, OF};
\node[section, below=of ch9s2] (ch9s3) {Outil TEMM : fonctionnement, cas d’usage};

\draw[arrow] (ch9) -- (ch9s1);
\draw[arrow] (ch9s1) -- (ch9s2);
\draw[arrow] (ch9s2) -- (ch9s3);

% Chapitre 10 : Phase 4 — Transfert
\node[chapter, below=1.6cm of ch9s3] (ch10) {Chapitre 10 : Phase 4 — Transfert et supervision};
\node[section, below=of ch10] (ch10s1) {Modes de transfert (centralisé/distribué)};
\node[section, below=of ch10s1] (ch10s2) {Bouclage entre environnement réel et simulation};

\draw[arrow] (ch10) -- (ch10s1);
\draw[arrow] (ch10s1) -- (ch10s2);

\end{tikzpicture}
