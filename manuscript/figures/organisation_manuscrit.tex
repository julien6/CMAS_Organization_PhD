\resizebox{\textwidth}{!}{%
\begin{tikzpicture}[
  chapter/.style = {draw, thick, fill=blue!10, minimum width=8cm, align=left, font=\normalsize},
  arrow/.style = {-{Latex[round]}, thick},
  partlabel/.style = {rotate=90, font=\bfseries, align=center},
  dashedline/.style = {red, thick, dashed},
  node distance=0.9cm and 2cm
]

% Chapitres principaux (colonne verticale)
\node[chapter] (c1) {Chapitre 1 : Contexte, concepts et problématique générale};
\node[chapter, below=of c1] (c2) {Chapitre 2 : Revue des travaux liés, \\ émergence et positionnement de l'approche};
\node[chapter, below=of c2] (c3) {Chapitre 3 : Hypothèses pour une réponse générale};

\node[chapter, below=of c3] (c4) {Chapitre 4 : Concepts transversaux liés aux hypothèses};
\node[chapter, below=of c4] (c5) {Chapitre 5 : Revue de littérature, \\ analyse et verrous liés aux hypothèses};

\node[chapter, below=of c5] (c6) {Chapitre 6 : MAMAD comme réponse};

% Branche de droite (phases)
\node[chapter, below= of c6, xshift=-6cm] (c7) {Chapitre 7 : Phase 1 — Modélisation en simulation};
\node[chapter, below=of c7] (c8) {Chapitre 8 : Phase 2 — Apprentissage guidé/contraint};
\node[chapter, below=of c8] (c9) {Chapitre 9 : Phase 3 — Analyse post-entraînement};
\node[chapter, below=of c9] (c10) {Chapitre 10 : Phase 4 — Transfert et supervision};

% Suite verticale
\node[chapter, below=7cm of c6, xshift=-5cm] (c11) {Chapitre 11 : Implémentation et outils};
\node[chapter, below=of c11, xshift=5cm] (c12) {Chapitre 12 : Protocole expérimental};
\node[chapter, below=of c12] (c13) {Chapitre 13 : Études de cas};
\node[chapter, below=of c13] (c14) {Chapitre 14 : Résultats et synthèse};
\node[chapter, below=of c14] (c15) {Chapitre 15 : Bilan de la thèse};

\node[chapter, below=of c15] (c16) {Chapitre 16 : Perspectives et ouvertures};

% Arrows
\foreach \i/\j in {c1/c2, c2/c3, c3/c4, c4/c5, c5/c6, c7/c8, c8/c9, c9/c10, c12/c13, c13/c14, c14/c15, c15/c16}
  \draw[arrow] (\i) -- (\j);

% Flèches horizontales depuis c6
\foreach \dest in {c7, c8, c9, c10}
  \draw[arrow] ($ (c6.south) + (-0.35,0) $) -- ++(0,0) |- (\dest.east);

\draw[arrow] (c10.east) -- ++(1.35,0) -- ($ (c6.south) + (-0.35,0) $);

\draw[arrow] (c6.south) -- ++(0,0) |- (c11.east);

\draw[arrow] (c11.south) -- ++(0,0) |- (c12.west);

\draw[arrow] (c6.south) -- (c12.north);

% Partie labels à gauche
\draw[decorate, decoration={brace, amplitude=15pt}, thick]
  ($(c9.west |- c3.west)+(-0.3,-0.4)$) -- ($(c9.west |- c1.west)+(-0.3,0.4)$)
  node[midway,xshift=-1.4cm,rotate=0]{\textbf{Partie I}};

\draw[decorate, decoration={brace, amplitude=15pt}, thick]
  ($(c9.west |- c5.west)+(-0.3,-0.4)$) -- ($(c9.west |- c4.west)+(-0.3,0.4)$)
  node[midway,xshift=-1.4cm,rotate=0]{\textbf{Partie II}};

\draw[decorate, decoration={brace, amplitude=15pt}, thick]
  ($(c9.west |- c10.west)+(-0.3,-0.4)$) -- ($(c9.west |- c6.west)+(-0.3,0.4)$)
  node[midway,xshift=-1.4cm,rotate=0]{\textbf{Partie III}};

\draw[decorate, decoration={brace, amplitude=15pt}, thick]
  ($(c9.west |- c14.west)+(-0.3,-0.4)$) -- ($(c9.west |- c11.west)+(-0.3,0.4)$)
  node[midway,xshift=-1.4cm,rotate=0]{\textbf{Partie IV}};

\draw[decorate, decoration={brace, amplitude=15pt}, thick]
  ($(c9.west |- c16.west)+(-0.3,-0.4)$) -- ($(c9.west |- c15.west)+(-0.3,0.4)$)
  node[midway,xshift=-1.4cm,rotate=0]{\textbf{Partie V}};

\end{tikzpicture}
}
