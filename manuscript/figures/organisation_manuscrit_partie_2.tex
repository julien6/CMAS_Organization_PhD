\begin{tikzpicture}[
    chapter/.style={draw, fill=blue!10, thick, minimum width=8cm, minimum height=1.2cm, text centered, font=\bfseries},
    section/.style={draw, fill=blue!5, thick, minimum width=7cm, minimum height=1cm, text centered, font=\small},
    arrow/.style={-Latex, thick},
    node distance=0.4cm
]

% Chapitre 4
\node[chapter] (ch4) {Chapitre 4 : Concepts transversaux liés aux hypothèses};

\node[section, below=of ch4] (ch4s1) {Systèmes Multi-Agents : coordination, rôles, interactions};
\node[section, below=of ch4s1] (ch4s2) {Modèles organisationnels : MOISE+};
\node[section, below=of ch4s2] (ch4s3) {Apprentissage par renforcement multi-agent (MARL)};
\node[section, below=of ch4s3] (ch4s4) {Intégration organisation-apprentissage};

\draw[arrow] (ch4) -- (ch4s1);
\draw[arrow] (ch4s1) -- (ch4s2);
\draw[arrow] (ch4s2) -- (ch4s3);
\draw[arrow] (ch4s3) -- (ch4s4);

% Chapitre 5
\node[chapter, below=2.5cm of ch4s4] (ch5) {Chapitre 5 : Revue de littérature, analyse et verrous liés aux hypothèses};

\draw[arrow] (ch4s4) -- (ch5);

\node[section, below=of ch5] (ch5s1) {H1 : Formalisation du problème (Dec-POMDP contraint)};
\node[section, below=of ch5s1] (ch5s2) {H2 : Modélisation par apprentissage de l’environnement (World Models)};
\node[section, below=of ch5s2] (ch5s3) {H3 : Résolution par MARL adaptatif};
\node[section, below=of ch5s3] (ch5s4) {H4 : Guidage organisationnel via contraintes MOISE+ (OAC/TRF)};
\node[section, below=of ch5s4] (ch5s5) {H5 : Analyse organisationnelle (rôles/objectifs émergents)};

\draw[arrow] (ch5) -- (ch5s1);
\draw[arrow] (ch5s1) -- (ch5s2);
\draw[arrow] (ch5s2) -- (ch5s3);
\draw[arrow] (ch5s3) -- (ch5s4);
\draw[arrow] (ch5s4) -- (ch5s5);

\end{tikzpicture}
