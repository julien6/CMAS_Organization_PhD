\begin{tikzpicture}[
    chapter/.style={draw, fill=blue!10, thick, minimum width=8cm, minimum height=1.2cm, text centered, font=\bfseries},
    section/.style={draw, fill=blue!5, thick, minimum width=7cm, minimum height=1cm, text centered, font=\small},
    arrow/.style={-Latex, thick},
    node distance=0.4cm,
    annotated/.style={above,font=\small\itshape, inner sep=1pt, yshift=0.8cm, xshift=-7cm}
  ]

  % Chapitre 4
  \node[chapter] (ch4) {\parbox{10cm}{Chapitre 4 : Les verrous d'une méthode de conception}};

  \node[section, below=1cm of ch4, xshift=-1cm] (ch4s1) {\parbox{8cm}{La modélisation d'un environnement en simulation (H-MOD)}};
  \node[section, below=1cm of ch4s1] (ch4s2) {\parbox{8cm}{L'intégration de contraintes/guidages organisationnelles dans le processus MARL (H-TRN)}};
  \node[section, below=1cm of ch4s2] (ch4s3) {\parbox{8cm}{L'extraction de spécifications organisationnelles émérgentes (H-ANL)}};
  \node[section, below=1cm of ch4s3] (ch4s4) {\parbox{8cm}{Le maintien de cohérence entre l'environnement simulé et l'environnement réel (H-TRF)}};

  \draw[arrow] ($ (ch4.south) + (4.0,0) $) -- ++(0,0) |- (ch4s1.east) node[annotated] {Exploration des travaux concernant la modélisation en simulation.};
  \draw[arrow] ($ (ch4.south) + (4.0,0) $) -- ++(0,0) |- (ch4s2.east) node[annotated] {Exploration des travaux sur l'intégration de contraintes en MARL.};
  \draw[arrow] ($ (ch4.south) + (4.0,0) $) -- ++(0,0) |- (ch4s3.east) node[annotated] {Exploration des moyens de comprendre la politique entraînée de façon organisationnelle.};
  \draw[arrow] ($ (ch4.south) + (4.0,0) $) -- ++(0,0) |- (ch4s4.east) node[annotated] {Exploration des travaux sur la relation entre environnement cible et simulé.};

  % Chapitre 5
  \node[chapter, below=1cm of ch4s4, xshift=1cm] (ch5) {\parbox{10cm}{Chapitre 5 : Les travaux et concepts théoriques mobilisés}};

  \node[section, below=1cm of ch5, xshift=-1cm] (ch5s1) {\parbox{8cm}{Modélisation de l'environnement (H-MOD)}};
  \node[section, below=1cm of ch5s1] (ch5s2) {\parbox{8cm}{Apprentissage par renforcement sous contraintes (H-TRN)}};
  \node[section, below=1cm of ch5s2] (ch5s3) {\parbox{8cm}{Explicabilité et extraction organisationnelle (H-ANL)}};
  \node[section, below=1cm of ch5s3] (ch5s4) {\parbox{8cm}{Transfert simulation vers environnement réel et cohérence (H-TRF)}};

  \draw[arrow] ($ (ch4.south) + (4.5,0) $) -- ($ (ch5.north) + (4.5,0) $) node[annotated, yshift=-0.5cm] {À partir de ces concepts, chaque hypothèse est examinée à travers une revue ciblée.};

  \draw[arrow] ($ (ch5.south) + (4.0,0) $) -- ++(0,0) |- (ch5s1.east) node[annotated] {Formulation formelle du problème de conception.};
  \draw[arrow] ($ (ch5.south) + (4.0,0) $) -- ++(0,0) |- (ch5s2.east) node[annotated] {Génération d'un environnement simulé fidèle aux dynamiques réelles.};
  \draw[arrow] ($ (ch5.south) + (4.0,0) $) -- ++(0,0) |- (ch5s3.east) node[annotated] {Apprentissage guidé par des contraintes/guidages organisationnelles.};
  \draw[arrow] ($ (ch5.south) + (4.0,0) $) -- ++(0,0) |- (ch5s4.east) node[annotated] {Interpreter les comportements appris sous forme de structures émergentes.};

\end{tikzpicture}
