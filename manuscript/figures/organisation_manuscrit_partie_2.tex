\begin{tikzpicture}[
    chapter/.style={draw, fill=blue!10, thick, minimum width=8cm, minimum height=1.2cm, text centered, font=\bfseries},
    section/.style={draw, fill=blue!5, thick, minimum width=7cm, minimum height=1cm, text centered, font=\small},
    arrow/.style={-Latex, thick},
    node distance=0.4cm
]

% Chapitre 4
\node[chapter] (ch4) {\parbox{10cm}{Chapitre 4 : Les SMA et concepts théoriques mobilisés}};

\node[section, below=of ch4, xshift=-1cm] (ch4s1) {\parbox{8cm}{SMA et modèles organisationnels}};
\node[section, below=of ch4s1] (ch4s2) {\parbox{8cm}{Apprentissage par renforcement multi-agent}};
\node[section, below=of ch4s2] (ch4s3) {\parbox{8cm}{Modéliser un environnement simulé : les techniques \textquote{World Models}}};

\draw[arrow] ($ (ch4.south) + (4.0,0) $) -- ++(0,0) |- (ch4s1.east);
\draw[arrow] ($ (ch4.south) + (4.0,0) $) -- ++(0,0) |- (ch4s2.east);
\draw[arrow] ($ (ch4.south) + (4.0,0) $) -- ++(0,0) |- (ch4s3.east);

% Chapitre 5
\node[chapter, below=5cm of ch4] (ch5) {\parbox{10cm}{Chapitre 5 : Revue de littérature, analyse et verrous liés aux hypothèses}};

\node[section, below=of ch5, xshift=-1cm] (ch5s1) {\parbox{8cm}{Un cadre Markovien pour formaliser le problème \\ de conception et sa résolution en MARL (H1)}};
\node[section, below=of ch5s1] (ch5s2) {\parbox{8cm}{Les Worlds Models pour simuler des \\ environnement multi-agents (H2)}};
\node[section, below=of ch5s2] (ch5s3) {\parbox{8cm}{L'intégration de contraintes/guidages \\ organisationnelles dans le processus MARL (H3)}};
\node[section, below=of ch5s3] (ch5s4) {\parbox{8cm}{L'extraction automatisée des spécifications \\ organisationnelles émergentes (H4)}};

\draw[arrow] ($ (ch4.south) + (4.5,0) $) -- ($ (ch5.north) + (4.5,0) $);

\draw[arrow] ($ (ch5.south) + (4.0,0) $) -- ++(0,0) |- (ch5s1.east);
\draw[arrow] ($ (ch5.south) + (4.0,0) $) -- ++(0,0) |- (ch5s2.east);
\draw[arrow] ($ (ch5.south) + (4.0,0) $) -- ++(0,0) |- (ch5s3.east);
\draw[arrow] ($ (ch5.south) + (4.0,0) $) -- ++(0,0) |- (ch5s4.east);

\end{tikzpicture}
