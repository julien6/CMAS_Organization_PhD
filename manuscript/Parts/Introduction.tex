\clearpage
\thispagestyle{empty}
\null
\newpage

\cleardoublepage
\pagenumbering{arabic}
\cleardoublepage
\phantomsection
% \pdfbookmark[1]{INTRODUCTION}{INTRODUCTION}
\addcontentsline{toc}{part}{INTRODUCTION}
\markboth{\spacedlowsmallcaps{INTRODUCTION}}{\spacedlowsmallcaps{INTRODUCTION}}
\noindent {\LARGE\textbf{INTRODUCTION}}

\

\vspace{1em}

\bigskip

\noindent {\Large\textbf{Motivations}}\\

\noindent
Depuis plus d'une décennie, la complexité, l'automatisation et la rapidité d'exécution des cyberattaques remettent en question les approches classiques de cyberdéfense, centralisées et principalement réactives ou basées sur des règles fixes. Dans un contexte où les systèmes à défendre deviennent eux-mêmes distribués, et dynamiques (architectures microservices, essaim de drones, IoT industriels, etc.), les mécanismes de défense doivent eux aussi gagner en autonomie, en adaptabilité et en résilience.

L'idée d'un agent intelligent de cyberdéfense, capable de détecter et contrer de manière autonome les menaces, a été incarnée par le concept d'agent \acn{AICA}~\cite{Kott2023}. La première génération d'\acn{AICA} a pris la forme d'un agent \textit{monolithique}, doté de capacités de perception, décision et action sur un périmètre localisé. Rapidement, la complexité des situations traitées a justifié une évolution vers une architecture plus explicite et modulaire, illustrée par l'architecture \acn{MASCARA}~\cite{Kott2023}, qui propose une architecture modulaire symbolique de type cognitif de l'agent \acn{AICA}.

Dans la continuité de ces travaux, cette approche distribuée a évolué vers une version multi-agent du concept \acn{AICA}, dans laquelle plusieurs agents collaborent pour assurer la défense d'un système complexe, ouvrant la voie au concept de \acn{SMA} de Cyberdéfense. Cette évolution soulève de nombreuses questions~: comment spécifier une organisation adaptée aux contraintes de l'environnement et aux attaques ? Quelles capacités doivent prendre ces agents et pour quel coût ? Comment garantir que leurs comportements respectent des exigences fonctionnelles (performance, adaptation) et non-fonctionnelles (sûreté, explicabilité) ?

\

\bigskip

\noindent {\Large\textbf{Objectifs}}\\

\noindent
L'objectif global de cette thèse est d'obtenir un \acn{SMA} de Cyberdéfense capable de s'adapter aux contraintes dynamiques de l'environnement à défendre, tout en maximisant sa capacité à détecter, prévenir et contrer les menaces.
%
Plutôt que de chercher une solution de \acn{SMA} de Cyberdéfense unique pour répondre à ce problème complexe, nous visons à établir une méthode de conception pour guider ou assister le processus de création d'un tel \acn{SMA}. Cette méthode se veut générique et capable d'intégrer, organiser et orchestrer les différentes contributions manière cohérente.

\noindent
L'idée de mettre en place une telle méthode implique de prendre en compte plusieurs défis~:
\begin{itemize}
    \item \textbf{Modéliser le problème de conception} du \acn{SMA} pour prendre en compte à la fois les objectifs de cyberdéfense et les contraintes imposées par l'environnement ou le concepteur~;
    \item \textbf{Réduire le coût de conception} en s'affranchissant partiellement de la dépendance aux connaissances expertes~;
    \item \textbf{Évaluer quantitativement la performance} du SMA sur des critères tels que la résilience, l'autonomie ou l'adaptabilité~;
    \item \textbf{Assurer l'explicabilité} du comportement global du \acn{SMA}, notamment en identifiant les rôles, objectifs et interactions émergents~;
    \item \textbf{Garantir la sûreté de fonctionnement et le respect des contraintes critiques}, par des mécanismes de contrôle organisationnel~;
    \item \textbf{Permettre l'adaptation dynamique} du \acn{SMA} face à l'évolution des contraintes et des situations dans l'environnement réel.
\end{itemize}

\noindent
Ces défis orientent la recherche vers une méthode qui couvre la conception du \acn{SMA} de Cyberdéfense en cherchant à bénéficier des avantages à la fois des approches connexionniste et symbolique.

\

\bigskip

\noindent {\Large\textbf{Contributions visées}}\\

\noindent
Nous cherchons à établir une méthode de conception de \acn{SMA} pour des \acn{SMA} de Cyberdéfense. En supposant la simulation nécessaire, notre méthode peut être abordée en quatre activités~:
\begin{itemize}
    \item \textbf{Modélisation}~: création d'un environnement simulé soit manuellement en suivant un cadre Markovien générique et des connaissances expertes, soit automatiquement à l'aide de techniques \acn{ML} capturant les dynamiques de l'environnement à partir de ses traces~;
    \item \textbf{Résolution}~: obtention de politiques multi-agent conjointes manuellement ou par des techniques \acn{ML} sous des contraintes exprimées comme spécifications organisationnelles notamment~;
    \item \textbf{Analyse}~: inférence de spécifications organisationnelles émergentes, telles que des rôles ou des objectifs, à partir des trajectoires des agents, en s'appuyant sur des techniques d'apprentissage non supervisée notamment~;
    \item \textbf{Transfert}~: couplage et minimisation de l'écart entre l'environnement réel et l'environnement simulé à la manière de jumeaux digitaux par la mise à jour des politiques des agents réels et de l'environnement simulé.
\end{itemize}

\noindent Ces activités nécessitent plusieurs contributions~:
\begin{itemize}
    \item Une formalisation du problème de conception comme un problème d'optimisation sous les contraintes de l'environnement et des concepteurs. Sur cette base, une formalisation de la méthode pourra être proposée pour modéliser formellement toute la chaîne de traitement orchestrant l'ensemble des contributions~;
    \item Une extension multi-agent de techniques \acn{ML} comme les \textit{World Models} permettra de capturer la dynamique de transition observationnelle, permettant ainsi de générer automatiquement un modèle simulant l'environnement tel que perçu par les agents~;
    \item Un framework qui fournira un cadre générique permettant de guider la modélisation manuelle d'un environnement réel comme une simulation à l'aide de connaissances expertes~;
    \item Un framework qui permettra d'intégrer les exigences des concepteurs dans le processus d'obtention des politiques, telle que dans un processus \acn{MARL} où l'on pourra guider ou contraindre l'apprentissage des politiques~;
    \item Une méthode d'analyse des comportements émergents, capable d'inferrer des spécifications organisationnelles à partir des trajectoires des agents entraînés~;
    \item Un outil implémentant l'ensemble de la méthode comprenant les différentes contributions et son application dans plusieurs environnements sur une partie ou l'ensemble des activités.
\end{itemize}

\noindent La méthode devra être évaluée dans des environnements de Cyberdéfense représentatifs. La méthode pourra également être évaluée dans des environnements non orientés Cyberdéfense pour montrer sa généralisabilité dans d'autres contextes multi-agents.

\

\bigskip

\noindent {\Large\textbf{Plan du manuscrit}}\\

\noindent
La question de l’organisation d’un \acn{SMA} de Cyberdéfense nous a d’abord conduits à effectuer une revue de littérature, dont l’analyse a mis en évidence l’intérêt de spécifier la question de recherche dans le cadre d'un problème d’optimisation. Cette question spécifiée permet de structurer notre démarche autour d’une série d’hypothèses, combinant des approches symboliques et connexionnistes, qui fondent les différentes contributions de la thèse. Ces contributions sont ensuite évaluées expérimentalement, afin de tester la validité des hypothèses et, in fine, de répondre à la question de recherche.

\noindent
Le manuscrit est structuré en cinq parties, composées de trois à cinq chapitres chacune, comme présenté en \autoref{fig:organisation_manuscrit}. Cette organisation suit un raisonnement progressif, détaillé en \autoref{fig:logique_manuscrit}.


\begin{figure}[h!]
    \centering
    \resizebox{\textwidth}{!}{%
        \resizebox{\textwidth}{!}{%
  \begin{tikzpicture}[
    chapter/.style = {draw, thick, fill=blue!10, minimum width=8cm, align=left, font=\normalsize},
    arrow/.style = {-{Latex[round]}, thick},
    partlabel/.style = {rotate=90, font=\bfseries, align=center},
    dashedline/.style = {red, thick, dashed},
    node distance=0.9cm and 2cm,
    annotated/.style={above,font=\small\itshape, inner sep=1pt, yshift=5.5mm, xshift=-4cm}
    ]

    % Chapitres principaux (colonne verticale)
    \node[chapter] (c1) {Chapitre 1 : Repenser la Cyberdéfense pour de nouveaux enjeux};
    \node[chapter, below=of c1] (c2) {Chapitre 2 : Vers des SMA de Cyberdéfense et leur conception};
    \node[chapter, below=of c2] (c3) {Chapitre 3 : Un problème d'optimisation pour structurer une méthode};

    \node[chapter, below=of c3] (c4) {Chapitre 4 : Les concepts théoriques mobilisés};
    \node[chapter, below=of c4] (c5) {Chapitre 5 : Les verrous d'une méthode de conception};

    \node[chapter, below=of c5] (c6) {Chapitre 6 : Présentation globale de la méthode};

    % Flèches verticales principales avec annotations
    \draw[arrow] (c1) -- (c2) node[annotated, xshift=-1.5cm] {Limites des approches de Cyberdéfense et question globale à adresser};
    \draw[arrow] (c2) -- (c3) node[annotated] {Connaissance des verrous et hypothèses};
    \draw[arrow] (c3) -- (c4) node[annotated] {Préciser les fondations théoriques nécessaires};
    \draw[arrow] (c4) -- (c5) node[annotated] {Identifier les verrous pour chaque hypothèse};
    \draw[arrow] (c5) -- (c6) node[annotated] {Assembler une méthode pour répondre aux verrous};

    % Branche de droite (activités)
    \node[chapter, below= of c6, xshift=-7cm] (c7) {Chapitre 7 : Modéliser l'environnement simulé};
    \node[chapter, below=of c7] (c8) {Chapitre 8 : Entraînement des politiques sous contraintes};
    \node[chapter, below=of c8] (c9) {Chapitre 9 : Analyser et interpréter les comportements émergents};
    \node[chapter, below=of c9] (c10) {Chapitre 10 : Transférer et superviser en environnement réel};

    % Suite verticale
    \node[chapter, below=7cm of c6, xshift=-7cm] (c11) {Chapitre 11 : CybMASDE : un framework supportant MAMAD};
    \node[chapter, below=of c11, xshift=7cm] (c12) {Chapitre 12 : Protocole expérimental};
    \node[chapter, below=of c12] (c13) {Chapitre 13 : Études de cas};
    \node[chapter, below=of c13] (c14) {Chapitre 14 : Résultats expérimentaux et analyse comparative};
    \node[chapter, below=of c14] (c15) {Chapitre 15 : Synthèse des apports et évaluation de la méthode};

    \node[chapter, below=of c15] (c16) {Chapitre 16 : Perspectives et ouvertures};

    % Arrows
    \foreach \i/\j in {c1/c2, c2/c3, c3/c4, c4/c5, c5/c6, c12/c13, c13/c14, c14/c15, c15/c16}
    \draw[arrow] (\i) -- (\j);

    % Flèches horizontales depuis c6
    \foreach \dest in {c7, c8, c9, c10}
    \draw[arrow] ($ (c6.south) + (-0.35,0) $) -- ++(0,0) |- (\dest.east);

    \draw[arrow] (c10.east) -- ++(1.325,0) -- ($ (c6.south) + (-0.35,0) $);

    \draw[arrow] (c6.south) -- ++(0,0) |- (c11.east);

    \draw[arrow] ($ (c11.south) + (1.5, 0) $) -- ++(0,0) |- (c12.west) node[annotated] {Support protocole expérimental};

    \draw[arrow] (c6.south) -- (c12.north);


    % Flèches suite expérimentale
    \draw[arrow] (c12) -- (c13) node[annotated] {Application sur des scénarios variés};
    \draw[arrow] (c13) -- (c14) node[annotated] {Consolidation et comparaison des résultats};
    \draw[arrow] (c14) -- (c15) node[annotated] {Synthèse des acquis et limites restantes};
    \draw[arrow] (c15) -- (c16) node[annotated] {Perspectives scientifiques et industrielles};


    % Partie labels à gauche
    \draw[decorate, decoration={brace, amplitude=15pt}, thick]
    ($(c9.west |- c3.west)+(-0.3,-0.4)$) -- ($(c9.west |- c1.west)+(-0.3,0.4)$)
    node[midway,xshift=-1.4cm,rotate=0]{\textbf{Partie I}};

    \draw[decorate, decoration={brace, amplitude=15pt}, thick]
    ($(c9.west |- c5.west)+(-0.3,-0.4)$) -- ($(c9.west |- c4.west)+(-0.3,0.4)$)
    node[midway,xshift=-1.4cm,rotate=0]{\textbf{Partie II}};

    \draw[decorate, decoration={brace, amplitude=15pt}, thick]
    ($(c9.west |- c10.west)+(-0.3,-0.4)$) -- ($(c9.west |- c6.west)+(-0.3,0.4)$)
    node[midway,xshift=-1.4cm,rotate=0]{\textbf{Partie III}};

    \draw[decorate, decoration={brace, amplitude=15pt}, thick]
    ($(c9.west |- c14.west)+(-0.3,-0.4)$) -- ($(c9.west |- c11.west)+(-0.3,0.4)$)
    node[midway,xshift=-1.4cm,rotate=0]{\textbf{Partie IV}};

    \draw[decorate, decoration={brace, amplitude=15pt}, thick]
    ($(c9.west |- c16.west)+(-0.3,-0.4)$) -- ($(c9.west |- c15.west)+(-0.3,0.4)$)
    node[midway,xshift=-1.4cm,rotate=0]{\textbf{Partie V}};

  \end{tikzpicture}
}

    }
    \caption{Schéma de l'organisation du manuscrit}
    \label{fig:organisation_manuscrit}
\end{figure}

\noindent
La \fullpartref{part:contexte}{Contexte de travail} introduit la question de recherche générale en mettant en évidence les limites des approches existantes pour concevoir des SMA dédiés à la cyberdéfense. Elle motive ainsi le recours à une approche hybride, combinant modèles symboliques et techniques connexionnistes, et spécifie la question de recherche dans un cadre d’optimisation sous contraintes. Cette partie se termine par la présentation des hypothèses qui structureront l’ensemble des travaux.
%
S’appuyant sur ces hypothèses, la \fullpartref{part:etat_art}{Etat de l'art} établit les fondements théoriques nécessaires et identifie, pour chacune d’elles, les verrous scientifiques qui devront être levés. Elle constitue ainsi le socle conceptuel sur lequel reposera la méthode proposée.
%
Dans la continuité, la \fullpartref{part:methode}{La méthode MAMAD} décrit la méthode \acn{MAMAD}, conçue pour répondre à la question de recherche. Elle en détaille les quatre activités clés et explicite les contributions spécifiques associées à chacune.
%
Pour évaluer la méthode et valider les hypothèses sous-jacentes, la \fullpartref{part:experimentation}{Validation expérimentale de la méthode} présente le protocole expérimental mis en place. Elle décrit les environnements testés, les objectifs d’évaluation, les métriques utilisées, ainsi que les résultats obtenus et leur analyse comparative.
%
Enfin, la \fullpartref{part:conclusion}{Conclusion} propose une synthèse des contributions de la thèse. Elle discute les limites rencontrées et trace plusieurs perspectives pour prolonger ce travail.
%
\begin{figure}[h!]
    \centering
    \resizebox{\textwidth}{!}{%
        


\tikzset{every picture/.style={line width=0.75pt}} %set default line width to 0.75pt        

\begin{tikzpicture}[x=0.75pt,y=0.75pt,yscale=-1,xscale=1]
    %uncomment if require: \path (0,1743); %set diagram left start at 0, and has height of 1743

    %Shape: Rectangle [id:dp18574584963494045] 
    \draw  [line width=1.5]  (50,150) -- (250,150) -- (250,210) -- (50,210) -- cycle ;
    %Straight Lines [id:da12196343525546238] 
    \draw    (151.5,210) -- (151.5,262)(148.5,210) -- (148.5,262) ;
    \draw [shift={(150,270)}, rotate = 270] [color={rgb, 255:red, 0; green, 0; blue, 0 }  ][line width=0.75]    (10.93,-3.29) .. controls (6.95,-1.4) and (3.31,-0.3) .. (0,0) .. controls (3.31,0.3) and (6.95,1.4) .. (10.93,3.29)   ;
    %Shape: Rectangle [id:dp4847066759253956] 
    \draw   (370,400) -- (570,400) -- (570,440) -- (370,440) -- cycle ;
    %Shape: Rectangle [id:dp3090415206682712] 
    \draw  [line width=1.5]  (50,390) -- (250,390) -- (250,450) -- (50,450) -- cycle ;
    %Straight Lines [id:da08726836991221709] 
    \draw  [dash pattern={on 0.84pt off 2.51pt}]  (450,212) -- (450,400) ;
    \draw [shift={(450,210)}, rotate = 90] [color={rgb, 255:red, 0; green, 0; blue, 0 }  ][line width=0.75]    (10.93,-3.29) .. controls (6.95,-1.4) and (3.31,-0.3) .. (0,0) .. controls (3.31,0.3) and (6.95,1.4) .. (10.93,3.29)   ;
    %Shape: Rectangle [id:dp5698076228521041] 
    \draw  [line width=1.5]  (20,750) -- (280,750) -- (280,810) -- (20,810) -- cycle ;
    %Shape: Rectangle [id:dp4753743401883225] 
    \draw  [line width=1.5]  (0,270) -- (310,270) -- (310,330) -- (0,330) -- cycle ;
    %Straight Lines [id:da04102324085536324] 
    \draw    (151.5,330) -- (151.5,382)(148.5,330) -- (148.5,382) ;
    \draw [shift={(150,390)}, rotate = 270] [color={rgb, 255:red, 0; green, 0; blue, 0 }  ][line width=0.75]    (10.93,-3.29) .. controls (6.95,-1.4) and (3.31,-0.3) .. (0,0) .. controls (3.31,0.3) and (6.95,1.4) .. (10.93,3.29)   ;
    %Shape: Rectangle [id:dp22707656448590674] 
    \draw  [line width=1.5]  (10,630) -- (280,630) -- (280,690) -- (10,690) -- cycle ;
    %Straight Lines [id:da981655106922525] 
    \draw  [dash pattern={on 0.84pt off 2.51pt}]  (250,190) -- (366,190) ;
    \draw [shift={(368,190)}, rotate = 180] [color={rgb, 255:red, 0; green, 0; blue, 0 }  ][line width=0.75]    (10.93,-3.29) .. controls (6.95,-1.4) and (3.31,-0.3) .. (0,0) .. controls (3.31,0.3) and (6.95,1.4) .. (10.93,3.29)   ;
    %Straight Lines [id:da008901650084719437] 
    \draw  [dash pattern={on 0.84pt off 2.51pt}]  (250,420) -- (368,420) ;
    \draw [shift={(370,420)}, rotate = 180] [color={rgb, 255:red, 0; green, 0; blue, 0 }  ][line width=0.75]    (10.93,-3.29) .. controls (6.95,-1.4) and (3.31,-0.3) .. (0,0) .. controls (3.31,0.3) and (6.95,1.4) .. (10.93,3.29)   ;
    %Straight Lines [id:da7631246123173825] 
    \draw    (151.5,690) -- (151.5,742)(148.5,690) -- (148.5,742) ;
    \draw [shift={(150,750)}, rotate = 270] [color={rgb, 255:red, 0; green, 0; blue, 0 }  ][line width=0.75]    (10.93,-3.29) .. controls (6.95,-1.4) and (3.31,-0.3) .. (0,0) .. controls (3.31,0.3) and (6.95,1.4) .. (10.93,3.29)   ;
    %Straight Lines [id:da4329171999280237] 
    \draw    (151.5,570) -- (151.5,622)(148.5,570) -- (148.5,622) ;
    \draw [shift={(150,630)}, rotate = 270] [color={rgb, 255:red, 0; green, 0; blue, 0 }  ][line width=0.75]    (10.93,-3.29) .. controls (6.95,-1.4) and (3.31,-0.3) .. (0,0) .. controls (3.31,0.3) and (6.95,1.4) .. (10.93,3.29)   ;
    %Straight Lines [id:da4945277958586467] 
    \draw    (150.5,449.98) -- (151.37,501.98)(147.5,450.02) -- (148.37,502.03) ;
    \draw [shift={(150,510)}, rotate = 269.05] [color={rgb, 255:red, 0; green, 0; blue, 0 }  ][line width=0.75]    (10.93,-3.29) .. controls (6.95,-1.4) and (3.31,-0.3) .. (0,0) .. controls (3.31,0.3) and (6.95,1.4) .. (10.93,3.29)   ;
    %Shape: Rectangle [id:dp4662158969205311] 
    \draw  [line width=1.5]  (50,510) -- (250,510) -- (250,570) -- (50,570) -- cycle ;
    %Straight Lines [id:da5672568795108123] 
    \draw  [dash pattern={on 0.84pt off 2.51pt}]  (490,442) -- (490,750) ;
    \draw [shift={(490,440)}, rotate = 90] [color={rgb, 255:red, 0; green, 0; blue, 0 }  ][line width=0.75]    (10.93,-3.29) .. controls (6.95,-1.4) and (3.31,-0.3) .. (0,0) .. controls (3.31,0.3) and (6.95,1.4) .. (10.93,3.29)   ;
    %Straight Lines [id:da6449350432746315] 
    \draw  [dash pattern={on 0.84pt off 2.51pt}]  (252,550) -- (450,550) -- (450,750) ;
    \draw [shift={(250,550)}, rotate = 0] [color={rgb, 255:red, 0; green, 0; blue, 0 }  ][line width=0.75]    (10.93,-3.29) .. controls (6.95,-1.4) and (3.31,-0.3) .. (0,0) .. controls (3.31,0.3) and (6.95,1.4) .. (10.93,3.29)   ;
    %Shape: Rectangle [id:dp9145794948784826] 
    \draw   (400,750) -- (630,750) -- (630,810) -- (400,810) -- cycle ;
    %Straight Lines [id:da5051851041653137] 
    \draw    (280,778.5) -- (392,778.5)(280,781.5) -- (392,781.5) ;
    \draw [shift={(400,780)}, rotate = 180] [color={rgb, 255:red, 0; green, 0; blue, 0 }  ][line width=0.75]    (10.93,-3.29) .. controls (6.95,-1.4) and (3.31,-0.3) .. (0,0) .. controls (3.31,0.3) and (6.95,1.4) .. (10.93,3.29)   ;
    %Straight Lines [id:da9009018604767407] 
    \draw    (468.5,810) -- (468.5,862)(465.5,810) -- (465.5,862) ;
    \draw [shift={(467,870)}, rotate = 270] [color={rgb, 255:red, 0; green, 0; blue, 0 }  ][line width=0.75]    (10.93,-3.29) .. controls (6.95,-1.4) and (3.31,-0.3) .. (0,0) .. controls (3.31,0.3) and (6.95,1.4) .. (10.93,3.29)   ;
    %Shape: Rectangle [id:dp9696498661224131] 
    \draw   (400,870) -- (590,870) -- (590,910) -- (400,910) -- cycle ;
    %Straight Lines [id:da7339727457125783] 
    \draw    (151.5,810) -- (151.5,862)(148.5,810) -- (148.5,862) ;
    \draw [shift={(150,870)}, rotate = 270] [color={rgb, 255:red, 0; green, 0; blue, 0 }  ][line width=0.75]    (10.93,-3.29) .. controls (6.95,-1.4) and (3.31,-0.3) .. (0,0) .. controls (3.31,0.3) and (6.95,1.4) .. (10.93,3.29)   ;
    %Shape: Rectangle [id:dp10500637872389273] 
    \draw  [line width=1.5]  (20,870) -- (280,870) -- (280,910) -- (20,910) -- cycle ;
    %Straight Lines [id:da22911770438659773] 
    \draw  [dash pattern={on 0.84pt off 2.51pt}]  (280,890) -- (398,890) ;
    \draw [shift={(400,890)}, rotate = 180] [color={rgb, 255:red, 0; green, 0; blue, 0 }  ][line width=0.75]    (10.93,-3.29) .. controls (6.95,-1.4) and (3.31,-0.3) .. (0,0) .. controls (3.31,0.3) and (6.95,1.4) .. (10.93,3.29)   ;
    %Straight Lines [id:da12093630372596798] 
    \draw    (151.5,910) -- (151.5,962)(148.5,910) -- (148.5,962) ;
    \draw [shift={(150,970)}, rotate = 270] [color={rgb, 255:red, 0; green, 0; blue, 0 }  ][line width=0.75]    (10.93,-3.29) .. controls (6.95,-1.4) and (3.31,-0.3) .. (0,0) .. controls (3.31,0.3) and (6.95,1.4) .. (10.93,3.29)   ;
    %Shape: Rectangle [id:dp13160181319541253] 
    \draw  [line width=1.5]  (26.88,970) -- (286.88,970) -- (286.88,1010) -- (26.88,1010) -- cycle ;
    %Straight Lines [id:da5189191976441709] 
    \draw    (151.5,1010) -- (151.5,1062)(148.5,1010) -- (148.5,1062) ;
    \draw [shift={(150,1070)}, rotate = 270] [color={rgb, 255:red, 0; green, 0; blue, 0 }  ][line width=0.75]    (10.93,-3.29) .. controls (6.95,-1.4) and (3.31,-0.3) .. (0,0) .. controls (3.31,0.3) and (6.95,1.4) .. (10.93,3.29)   ;
    %Shape: Rectangle [id:dp29716526412104993] 
    \draw  [line width=1.5]  (40,1070) -- (260,1070) -- (260,1110) -- (40,1110) -- cycle ;
    %Straight Lines [id:da1456519279417069] 
    \draw  [dash pattern={on 0.84pt off 2.51pt}]  (286.88,990) -- (398,990) ;
    \draw [shift={(400,990)}, rotate = 180] [color={rgb, 255:red, 0; green, 0; blue, 0 }  ][line width=0.75]    (10.93,-3.29) .. controls (6.95,-1.4) and (3.31,-0.3) .. (0,0) .. controls (3.31,0.3) and (6.95,1.4) .. (10.93,3.29)   ;
    %Shape: Rectangle [id:dp04492823912399335] 
    \draw   (400,950) -- (630,950) -- (630,1030) -- (400,1030) -- cycle ;
    %Shape: Rectangle [id:dp07847633718854785] 
    \draw   (370,170) -- (570,170) -- (570,210) -- (370,210) -- cycle ;
    %Shape: Rectangle [id:dp1355817602636692] 
    \draw  [line width=1.5]  (90,1140) -- (110,1140) -- (110,1160) -- (90,1160) -- cycle ;
    %Shape: Rectangle [id:dp07143868735089931] 
    \draw  [line width=0.75]  (350,1139) -- (370,1139) -- (370,1159) -- (350,1159) -- cycle ;
    %Straight Lines [id:da8433905723884118] 
    \draw  [dash pattern={on 0.84pt off 2.51pt}]  (360,1179) -- (360,1197) ;
    \draw [shift={(360,1199)}, rotate = 270] [color={rgb, 255:red, 0; green, 0; blue, 0 }  ][line width=0.75]    (6.56,-1.97) .. controls (4.17,-0.84) and (1.99,-0.18) .. (0,0) .. controls (1.99,0.18) and (4.17,0.84) .. (6.56,1.97)   ;
    %Straight Lines [id:da473045317791972] 
    \draw    (151.5,90) -- (151.5,142)(148.5,90) -- (148.5,142) ;
    \draw [shift={(150,150)}, rotate = 270] [color={rgb, 255:red, 0; green, 0; blue, 0 }  ][line width=0.75]    (10.93,-3.29) .. controls (6.95,-1.4) and (3.31,-0.3) .. (0,0) .. controls (3.31,0.3) and (6.95,1.4) .. (10.93,3.29)   ;
    %Shape: Rectangle [id:dp28718160440236395] 
    \draw  [line width=1.5]  (50,30) -- (250,30) -- (250,90) -- (50,90) -- cycle ;
    %Straight Lines [id:da002924452234624897] 
    \draw    (98.5,1180) -- (98.5,1194) ;
    %Straight Lines [id:da10403872552724802] 
    \draw    (100.5,1180) -- (100.5,1194) ;
    \draw   (101.58,1194) .. controls (100.47,1196) and (99.8,1198) .. (99.58,1200) .. controls (99.36,1198) and (98.69,1196) .. (97.58,1194) ;
    %Shape: Rectangle [id:dp964138966540305] 
    \draw  [fill={rgb, 255:red, 184; green, 233; blue, 134 }  ,fill opacity=1 ] (96.5,1184) -- (102.5,1184) -- (102.5,1190) -- (96.5,1190) -- cycle ;



    % Text Node
    \draw (486,1149.5) node   [align=left] {\begin{minipage}[lt]{174.47pt}\setlength\topsep{0pt}
            \begin{center}
                Données / résultats spécifiques
            \end{center}

        \end{minipage}};
    % Text Node
    \draw (211.5,1149.5) node   [align=left] {\begin{minipage}[lt]{169.17pt}\setlength\topsep{0pt}
            \begin{center}
                Données de jalons globaux
            \end{center}

        \end{minipage}};
    % Text Node
    \draw (217.5,1188.5) node   [align=left] {Développement de chapitres};
    % Text Node
    \draw (481,1189.5) node   [align=left] {Lien entre données / résultats};
    % Text Node
    \draw (150,60) node   [align=left, name=sujet] {\begin{minipage}[lt]{190.21pt}\setlength\topsep{0pt}
            \begin{center}
                Sujet "De l'Organisation\\d'un SMA de Cyberdéfense"
            \end{center}

        \end{minipage}};
    % Text Node
    \draw (238,120.5) node   [align=left] {\textit{amène à une...}};
    % Text Node
    \draw  [fill={rgb, 255:red, 184; green, 233; blue, 134 }  ,fill opacity=1 ]  (137.5,107.5) -- (162.5,107.5) -- (162.5,132.5) -- (137.5,132.5) -- cycle  ;
    \draw (150,120) node   [align=left] {\begin{minipage}[lt]{14.07pt}\setlength\topsep{0pt}
            \begin{center}
                1
            \end{center}

        \end{minipage}};
    % Text Node
    \draw  [fill={rgb, 255:red, 184; green, 233; blue, 134 }  ,fill opacity=1 ]  (297,768) -- (372,768) -- (372,793) -- (297,793) -- cycle  ;
    \draw (334.5,780.5) node   [align=left] {\begin{minipage}[lt]{48.1pt}\setlength\topsep{0pt}
            \begin{center}
                6,7,8,9,10
            \end{center}

        \end{minipage}};
    % Text Node
    \draw (513.67,990.54) node   [align=left] {\begin{minipage}[lt]{153.12pt}\setlength\topsep{0pt}
            \begin{center}
                Validation des hypothèses\\vis-à-vis de la question de\\recherche \& sous-problèmes
            \end{center}

        \end{minipage}};
    % Text Node
    \draw (338,970.5) node   [align=left] {\textit{pour une...}};
    % Text Node
    \draw (150,1090) node   [align=left, name=conclusion] {\begin{minipage}[lt]{53.19pt}\setlength\topsep{0pt}
            \begin{center}
                Conclusion
            \end{center}

        \end{minipage}};
    % Text Node
    \draw (242.5,1040.5) node   [align=left] {\textit{ce qui permet une...}};
    % Text Node
    \draw  [fill={rgb, 255:red, 184; green, 233; blue, 134 }  ,fill opacity=1 ]  (124.5,1028) -- (174.5,1028) -- (174.5,1053) -- (124.5,1053) -- cycle  ;
    \draw (149.5,1040.5) node   [align=left] {\begin{minipage}[lt]{28.25pt}\setlength\topsep{0pt}
            \begin{center}
                15,16
            \end{center}
        \end{minipage}};
    % Text Node
    \draw (156.88,990) node   [align=left, name=resultats] {\begin{minipage}[lt]{142.7pt}\setlength\topsep{0pt}
            \begin{center}
                Discussion des résultats
            \end{center}

        \end{minipage}};
    % Text Node
    \draw (228.5,940.5) node   [align=left] {\textit{permettant une...}};
    % Text Node
    \draw  [fill={rgb, 255:red, 184; green, 233; blue, 134 }  ,fill opacity=1 ]  (137,928) -- (162,928) -- (162,953) -- (137,953) -- cycle  ;
    \draw (149.5,940.5) node   [align=left] {\begin{minipage}[lt]{14.07pt}\setlength\topsep{0pt}
            \begin{center}
                14
            \end{center}

        \end{minipage}};
    % Text Node
    \draw (336,870.5) node   [align=left] {\textit{en utilisant...}};
    % Text Node
    \draw (150,890) node   [align=left] {\begin{minipage}[lt]{180.03pt}\setlength\topsep{0pt}
            \begin{center}
                Cadre experimental et d'évaluation
            \end{center}

        \end{minipage}};
    % Text Node
    \draw (239.5,840.5) node   [align=left] {\textit{évaluées via un...}};
    % Text Node
    \draw  [fill={rgb, 255:red, 184; green, 233; blue, 134 }  ,fill opacity=1 ]  (127,827.5) -- (173,827.5) -- (173,852.5) -- (127,852.5) -- cycle  ;
    \draw (150,840) node   [align=left] {\begin{minipage}[lt]{28.25pt}\setlength\topsep{0pt}
            \begin{center}
                12,13
            \end{center}

        \end{minipage}};
    % Text Node
    \draw (492,890.5) node   [align=left] {\begin{minipage}[lt]{57.14pt}\setlength\topsep{0pt}
            \begin{center}
                CybMASDE
            \end{center}

        \end{minipage}};
    % Text Node
    \draw (563.5,841) node   [align=left] {\textit{implémentées et}\\\textit{intégrées dans l'outil...}};
    % Text Node
    \draw  [fill={rgb, 255:red, 184; green, 233; blue, 134 }  ,fill opacity=1 ]  (455,827.5) -- (479,827.5) -- (479,852.5) -- (455,852.5) -- cycle  ;
    \draw (467,840) node   [align=left] {\begin{minipage}[lt]{13.31pt}\setlength\topsep{0pt}
            \begin{center}
                11
            \end{center}

        \end{minipage}};
    % Text Node
    \draw (155.02,178.08) node   [align=left] {\begin{minipage}[lt]{113.81pt}\setlength\topsep{0pt}
            \begin{center}
                Question de\\recherche globale
            \end{center}

        \end{minipage}};
    % Text Node
    \draw (155,300) node   [align=left, name=revue] {\begin{minipage}[lt]{221.17pt}\setlength\topsep{0pt}
            \begin{center}
                Revue de littérature sur les SMA de\\Cyberdéfense et ses moyens de conception
            \end{center}

        \end{minipage}};
    % Text Node
    \draw (288.5,359.5) node   [align=left] {\textit{qui préconise de pivoter vers une...}};
    % Text Node
    \draw  [fill={rgb, 255:red, 184; green, 233; blue, 134 }  ,fill opacity=1 ]  (137.5,347.5) -- (162.5,347.5) -- (162.5,372.5) -- (137.5,372.5) -- cycle  ;
    \draw (150,360) node   [align=left] {\begin{minipage}[lt]{14.07pt}\setlength\topsep{0pt}
            \begin{center}
                3
            \end{center}

        \end{minipage}};
    % Text Node
    \draw (229.5,240.5) node   [align=left] {\textit{qui initie une...}};
    % Text Node
    \draw (337.5,750.5) node   [align=left] {\textit{menant aux...}};
    % Text Node
    \draw (515.5,779.5) node   [align=left] {\begin{minipage}[lt]{143.35pt}\setlength\topsep{0pt}
            \begin{center}
                Contributions méthodologiques
            \end{center}

        \end{minipage}};
    % Text Node
    \draw (153.5,781) node   [align=left, name=hypothese_ssp] {\begin{minipage}[lt]{174.65pt}\setlength\topsep{0pt}
            \begin{center}
                Hypothèses comblant les lacunes\\pour chaque sous-problème
            \end{center}

        \end{minipage}};
    % Text Node
    \draw (380.5,530.5) node   [align=left] {\textit{orientées vers l'...}};
    % Text Node
    \draw (561,551) node   [align=left] {\textit{qui permettent}\\\textit{d'addresser les...}};
    % Text Node
    \draw (151.73,539.5) node   [align=left, name=hypothese] {\begin{minipage}[lt]{148.77pt}\setlength\topsep{0pt}
            \begin{center}
                Hypothèse d'une\\méthode de conception
            \end{center}

        \end{minipage}};
    % Text Node
    \draw (227,480.5) node   [align=left] {\textit{addressé par l'...}};
    % Text Node
    \draw  [fill={rgb, 255:red, 184; green, 233; blue, 134 }  ,fill opacity=1 ]  (137,467.5) -- (162,467.5) -- (162,492.5) -- (137,492.5) -- cycle  ;
    \draw (149.5,480) node   [align=left] {\begin{minipage}[lt]{14.07pt}\setlength\topsep{0pt}
            \begin{center}
                3
            \end{center}

        \end{minipage}};
    % Text Node
    \draw (254,599) node   [align=left] {\textit{qui structure}\\\textit{les travaux pour une...}};
    % Text Node
    \draw  [fill={rgb, 255:red, 184; green, 233; blue, 134 }  ,fill opacity=1 ]  (135.5,587.5) -- (164.5,587.5) -- (164.5,612.5) -- (135.5,612.5) -- cycle  ;
    \draw (150,600) node   [align=left] {\begin{minipage}[lt]{16.9pt}\setlength\topsep{0pt}
            \begin{center}
                4,5
            \end{center}

        \end{minipage}};
    % Text Node
    \draw (239.5,721) node   [align=left] {\textit{servant à}\\\textit{établir les...}};
    % Text Node
    \draw  [fill={rgb, 255:red, 184; green, 233; blue, 134 }  ,fill opacity=1 ]  (112.5,707.5) -- (187.5,707.5) -- (187.5,732.5) -- (112.5,732.5) -- cycle  ;
    \draw (150,720) node   [align=left] {\begin{minipage}[lt]{48.1pt}\setlength\topsep{0pt}
            \begin{center}
                6,7,8,9,10
            \end{center}

        \end{minipage}};
    % Text Node
    \draw (310.5,400.5) node   [align=left] {\textit{impliquant des}};
    % Text Node
    \draw (307.37,170.5) node   [align=left] {\textit{impliquant des}};
    % Text Node
    \draw (146.5,661) node   [align=left, name=revue_ssp] {\begin{minipage}[lt]{192.26pt}\setlength\topsep{0pt}
            \begin{center}
                Revue de littérature orientée\\méthode pour chaque sous-problème
            \end{center}

        \end{minipage}};
    % Text Node
    \draw  [fill={rgb, 255:red, 184; green, 233; blue, 134 }  ,fill opacity=1 ]  (137.5,227.5) -- (162.5,227.5) -- (162.5,252.5) -- (137.5,252.5) -- cycle  ;
    \draw (150,240) node   [align=left] {\begin{minipage}[lt]{14.07pt}\setlength\topsep{0pt}
            \begin{center}
                2
            \end{center}

        \end{minipage}};
    % Text Node
    \draw (515.5,281) node   [align=left] {\textit{qui permettent}\\\textit{d'englober les...}};
    % Text Node
    \draw (145.8,419) node   [align=left] {Question spécifiée via un\\problème d'optimisation};
    % Text Node
    \draw (471.5,189.5) node   [align=left] {Critères};
    % Text Node
    \draw (466.5,419.5) node   [align=left] {Sous-problèmes};
    % Text Node
    \draw (99.5,1187) node  [font=\tiny] [align=left] {\begin{minipage}[lt]{8.67pt}\setlength\topsep{0pt}
            \begin{center}
                {\tiny x}
            \end{center}

        \end{minipage}};

    \draw[decorate, decoration={brace, amplitude=20pt}, thick]
    ($([yshift=-5pt]revue.west |- hypothese.west)+(-0.3,-0.4)$) -- ($([yshift=5pt]revue.west |- sujet.west)+(-0.3,-0.4)$)
    node[midway,xshift=-2cm,rotate=0]{\textbf{Partie I}};

    \draw[decorate, decoration={brace, amplitude=20pt}, thick]
    ($([yshift=-5pt]revue.west |- revue_ssp.west)+(-0.3,-0.4)$) -- ($([yshift=5pt]revue.west |- hypothese.west)+(-0.3,-0.4)$)
    node[midway,xshift=-2cm,rotate=0]{\textbf{Partie II}};

    \draw[decorate, decoration={brace, amplitude=20pt}, thick]
    ($([yshift=-5pt]revue.west |- hypothese_ssp.west)+(-0.3,-0.4)$) -- ($([yshift=5pt]revue.west |- revue_ssp.west)+(-0.3,-0.4)$)
    node[midway,xshift=-2cm,rotate=0]{\textbf{Partie III}};

    \draw[decorate, decoration={brace, amplitude=20pt}, thick]
    ($([yshift=-5pt]revue.west |- resultats.west)+(-0.3,-0.4)$) -- ($([yshift=5pt]revue.west |- hypothese_ssp.west)+(-0.3,-0.4)$)
    node[midway,xshift=-2cm,rotate=0]{\textbf{Partie IV}};

    \draw[decorate, decoration={brace, amplitude=20pt}, thick]
    ($([yshift=-5pt]revue.west |- conclusion.west)+(-0.3,-0.4)$) -- ($([yshift=5pt]revue.west |- resultats.west)+(-3,-0.4)$)
    node[midway,xshift=-2cm,rotate=0]{\textbf{Partie V}};

\end{tikzpicture}
    }
    \caption{Schéma de la logique sous-jacente de notre raisonnement sous-tendant l'organisation du manuscrit}
    \label{fig:logique_manuscrit}
\end{figure}

\cleardoublepage
