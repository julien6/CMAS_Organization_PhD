\cleardoublepage\pagenumbering{arabic}
\cleardoublepage
\phantomsection
% \pdfbookmark[1]{INTRODUCTION}{INTRODUCTION}
\addcontentsline{toc}{part}{INTRODUCTION}
\markboth{\spacedlowsmallcaps{INTRODUCTION}}{\spacedlowsmallcaps{INTRODUCTION}}
\noindent {\LARGE\textbf{INTRODUCTION}}

\

\vspace{1em}

\bigskip

\noindent {\Large\textbf{Motivations}}\\

Depuis plus d'une décennie, la complexité, l'automatisation et la rapidité d'exécution des cyberattaques remettent en question les approches classiques de cyberdéfense, centralisées et principalement réactives. Dans un contexte où les systèmes à défendre deviennent eux-mêmes distribués, dynamiques et interconnectés (architectures microservices, réseaux de drones, IoT industriels, etc.), les mécanismes de défense doivent eux aussi gagner en autonomie, en adaptabilité et en résilience.

L'idée d'un agent intelligent de cyberdéfense, capable de détecter et contrer de manière autonome les menaces, a été incarnée par le concept d'agent \ac{AICA}~\cite{Kott2023}. La première génération d'\ac{AICA} a pris la forme d'un agent \textit{monolithique}, doté de capacités de perception, décision et action sur un périmètre localisé. Rapidement, la complexité des situations traitées a justifié une évolution vers une architecture plus explicite et modulaire, illustrée par l'architecture \ac{MASCARA}~\cite{Kott2023}, qui propose une approche modulaire symbolique de type cognitif de l'agent \ac{AICA}.

Dans la continuité de ces travaux, l'approche a naturellement évolué vers une version multi-agent du concept \ac{AICA}, dans laquelle plusieurs agents collaborent pour assurer la défense d'un système complexe, ouvrant la voie au concept de \ac{SMA} de Cyberdéfense. Cette évolution soulève de nombreuses questions : comment spécifier une organisation adapté entre agents par rapport aux contraintes de l'environnement et aux attaques ? Quelle forme doit prendre ces agents et quel est le coût associé ? Comment garantir que leurs comportements respectent des exigences fonctionnelles (performance, adaptation) et non-fonctionnelles (sûreté, explicabilité) ?

\

\bigskip

\noindent {\Large\textbf{Objectifs}}\\

L'objectif global de cette thèse est de concevoir un \ac{SMA} de Cyberdéfense capable de s'adapter aux contraintes dynamiques de l'environnement à défendre, tout en maximisant sa capacité à détecter, prévenir et contrer les menaces.

Plutôt que de chercher une solution de \ac{SMA} de Cyberdéfense unique pour répondre à ce problème complexe, nous proposons de développer une méthode de conception permettant de guider le processus de création d'un tel \ac{SMA} tout en permettant de structurer cette recherche. Cette méthode se veut générique, systématique, et capable d'orchestrer des approches symboliques et connexionnistes de manière cohérente.

Concrètement, l'objectif principal implique de résoudre plusieurs sous-problèmes essentiels :
\begin{enumerate}
    \item \textbf{Modéliser le problème de conception} du \ac{SMA} pour prendre en compte à la fois les objectifs de cyberdéfense et les contraintes imposées par l'environnement ou le concepteur ;
    \item \textbf{Réduire le coût de conception} en s'affranchissant partiellement de la dépendance aux connaissances expertes ;
    \item \textbf{Évaluer la performance} du système de manière quantitative, sur plusieurs critères liés à la résilience, l'autonomie et l'adaptabilité ;
    \item \textbf{Assurer l'explicabilité} du comportement global du \ac{SMA}, notamment en identifiant les rôles, objectifs et interactions émergents ;
    \item \textbf{Garantir la sûreté de fonctionnement et le respect des contraintes critiques}, à travers des mécanismes de contrôle organisationnel ;
    \item \textbf{Permettre l'adaptation dynamique} du \ac{SMA} face à l'évolution des contraintes et des situations dans l'environnement réel.
\end{enumerate}

Ces objectifs orientent la définition d'une méthode complète qui couvre toutes les étapes : modélisation de l'environnement, entraînement des politiques, analyse organisationnelle et transfert en conditions réelles.

\

\bigskip

\noindent {\Large\textbf{Contributions visées}}\\

La contribution principale de cette thèse est une \textbf{méthode de conception de \ac{SMA} pour la cyberdéfense}, appelée \ac{MAMAD}. Elle intègre plusieurs apports originaux, qui ensemble, constituent une réponse complète et cohérente à la problématique de recherche.

La méthode repose sur quatre activités :
\begin{itemize}
    \item \textbf{Modélisation}~: création d'un environnement simulé réaliste, à l'aide de techniques de type \textit{World Models}, permettant de capturer les dynamiques de l'environnement cible à partir de données ou de règles de façon automatique ;
    \item \textbf{Entraînement}~: apprentissage de politiques multi-agent conjointes via \ac{MARL}, sous contraintes exprimées dans le modèle organisationnel de $\mathcal{M}OISE^+$ (rôles, missions, permissions, etc.) ;
    \item \textbf{Analyse}~: inférence des structures organisationnelles implicites (rôles et objectifs) à partir des trajectoires des agents, en s'appuyant sur des techniques d'apprentissage non supervisée ;
    \item \textbf{Transfert}~: déploiement et mise à jour continue du système dans l'environnement réel, en maintenant un couplage itératif avec la simulation.
\end{itemize}

La méthode et ces activités sont soutenues par plusieurs contributions :
\begin{itemize}
    \item Une modélisation Markovienne du problème de conception comme un problème d'optimisation de la politique conjointe des agents pour maximiser l'objectif de cyberdéfense sous les contraintes imposées de l'environnement et des concepteurs ;
    \item Une extension des \textit{World Models} pour un contexte multi-agent pour prendre en compte les observations et actions conjointes, permettant de générer automatiquement un modèle simulant la dynamique des observation au fil des interactions des agents ;
    \item Le framework \textit{MOISE+MARL} qui permet d'intégrer les exigences particulières des concepteurs comme des spécifications organisationnelles (comme des rôles ou des objectifs) dans le processus \ac{MARL} en vue de guider ou contraindre les politiques ;
    \item La méthode empirique \ac{TEMM} d'analyse des comportements émergents, capable d'inferrer des spécifications organisationnelles et mesurer la cohérence entre les spécifications organisationnelles prescrites et inferrées ;
    \item Une modélisation de la méthode \ac{MAMAD} décrivant formellement toute la chaine de traitement orchestrant l'ensemble des contributions associées activité et permettant entre autres de prendre en compte la dynamique de l'environnement de déploiement ;
    \item L'outil \ac{CybMASDE}, qui permet d'implémenter l'ensemble de la méthode \ac{MAMAD} et son application dans plusieurs environnements ;
\end{itemize}

La méthode \ac{MAMAD} a été évaluée expérimentalement dans trois environnements contrastés et représentatifs : un essaim de drones autonomes, une infrastructure d'entreprise, et une architecture distribuée à base de microservices.

\

\bigskip

\noindent {\Large\textbf{Plan du manuscrit}}\\

La figure~\autoref{fig:organisation_manuscrit} présente une synthèse schématique de l'organisation globale du manuscrit, illustrant le fil conducteur logique reliant les concepts théoriques, les hypothèses, les contributions et les validations expérimentales.
%
\begin{description}
    \item[Partie I : Contexte de travail :] cette première partie expose les limites des approches actuelles de cyberdéfense, introduit le changement de paradigme vers une approche hybride symbolique-connexioniste, et formalise la question de recherche dans un cadre d'optimisation sous contraintes. Elle présente également les hypothèses qui structurent la suite du travail.

    \item[Partie II : Etat de l'art :] cette partie fournit les fondements théoriques nécessaires, puis identifie les verrous scientifiques liés à chaque hypothèse de la thèse.

    \item[Partie III : La méthode \ac{MAMAD} :] elle décrit en détail la méthode \ac{MAMAD}, ses quatre activités, les principes qui la sous-tendent, et les outils développés pour sa mise en œuvre.

    \item[Partie IV : Validation expérimentale de la méthode :] cette partie présente les protocoles d'évaluation utilisés, les scénarios étudiés, les résultats obtenus, ainsi qu'une analyse comparative mettant en lumière les apports de la méthode.

    \item[Partie V : Conclusion et perspectives :] enfin, cette dernière partie dresse un bilan des contributions, discute les limites du travail, et propose des perspectives scientifiques et industrielles pour prolonger la recherche.
\end{description}
%
\vspace{1em}
%
\begin{figure}[h!]
    \centering
    \resizebox{\textwidth}{!}{%
        \resizebox{\textwidth}{!}{%
  \begin{tikzpicture}[
    chapter/.style = {draw, thick, fill=blue!10, minimum width=8cm, align=left, font=\normalsize},
    arrow/.style = {-{Latex[round]}, thick},
    partlabel/.style = {rotate=90, font=\bfseries, align=center},
    dashedline/.style = {red, thick, dashed},
    node distance=0.9cm and 2cm,
    annotated/.style={above,font=\small\itshape, inner sep=1pt, yshift=5.5mm, xshift=-4cm}
    ]

    % Chapitres principaux (colonne verticale)
    \node[chapter] (c1) {Chapitre 1 : Repenser la Cyberdéfense pour de nouveaux enjeux};
    \node[chapter, below=of c1] (c2) {Chapitre 2 : Vers des SMA de Cyberdéfense et leur conception};
    \node[chapter, below=of c2] (c3) {Chapitre 3 : Un problème d'optimisation pour structurer une méthode};

    \node[chapter, below=of c3] (c4) {Chapitre 4 : Les verrous d'une méthode de conception};
    \node[chapter, below=of c4] (c5) {Chapitre 5 :  Les concepts théoriques mobilisés};

    \node[chapter, below=of c5] (c6) {Chapitre 6 : Présentation globale de la méthode};

    % Flèches verticales principales avec annotations
    \draw[arrow] (c1) -- (c2) node[annotated, xshift=-1.5cm] {Limites des approches de Cyberdéfense et question globale à adresser};
    \draw[arrow] (c2) -- (c3) node[annotated] {Connaissance des verrous et hypothèses};
    \draw[arrow] (c3) -- (c4) node[annotated] {Identifier les verrous pour chaque hypothèse};
    \draw[arrow] (c4) -- (c5) node[annotated] {Préciser les fondations théoriques retenues};
    \draw[arrow] (c5) -- (c6) node[annotated] {Assembler une méthode pour répondre aux verrous};

    % Branche de droite (activités)
    \node[chapter, below= of c6, xshift=-7cm] (c7) {Chapitre 7 : Modéliser l'environnement simulé};
    \node[chapter, below=of c7] (c8) {Chapitre 8 : Entraînement des politiques sous contraintes};
    \node[chapter, below=of c8] (c9) {Chapitre 9 : Analyser les comportements émergents};
    \node[chapter, below=of c9] (c10) {Chapitre 10 : Transférer et superviser en environnement réel};

    % Suite verticale
    \node[chapter, below=7cm of c6, xshift=-7cm] (c11) {Chapitre 11 : CybMASDE : un framework supportant MAMAD};
    \node[chapter, below=of c11, xshift=7cm] (c12) {Chapitre 12 : Cadre expérimental et d'évaluation};
    \node[chapter, below=of c12] (c13) {Chapitre 13 : Études de cas};
    \node[chapter, below=of c13] (c14) {Chapitre 14 : Résultats expérimentaux et analyse comparative};
    \node[chapter, below=of c14] (c15) {Synthèse des apports et évaluation de la méthode};

    \node[chapter, below=of c15] (c16) {Perspectives et ouvertures};

    % Arrows
    \foreach \i/\j in {c1/c2, c2/c3, c3/c4, c4/c5, c5/c6, c12/c13, c13/c14, c14/c15, c15/c16}
    \draw[arrow] (\i) -- (\j);

    % Flèches horizontales depuis c6
    \foreach \dest in {c7, c8, c9, c10}
    \draw[arrow] ($ (c6.south) + (-0.35,0) $) -- ++(0,0) |- (\dest.east);

    \draw[arrow] (c10.east) -- ++(1.325,0) -- ($ (c6.south) + (-0.35,0) $);

    \draw[arrow] (c6.south) -- ++(0,0) |- (c11.east);

    \draw[arrow] ($ (c11.south) + (1.5, 0) $) -- ++(0,0) |- (c12.west) node[annotated] {Support protocole expérimental};

    \draw[arrow] (c6.south) -- (c12.north);


    % Flèches suite expérimentale
    \draw[arrow] (c12) -- (c13) node[annotated] {Application sur des scénarios variés};
    \draw[arrow] (c13) -- (c14) node[annotated] {Consolidation et comparaison des résultats};
    \draw[arrow] (c14) -- (c15) node[annotated] {Synthèse des acquis et limites restantes};
    \draw[arrow] (c15) -- (c16) node[annotated] {Perspectives scientifiques et industrielles};


    % Partie labels à gauche
    \draw[decorate, decoration={brace, amplitude=15pt}, thick]
    ($(c11.west |- c3.west)+(-0.3,-0.4)$) -- ($(c11.west |- c1.west)+(-0.3,0.4)$)
    node[midway,xshift=-1.4cm,rotate=0]{\textbf{Partie I}};

    \draw[decorate, decoration={brace, amplitude=15pt}, thick]
    ($(c11.west |- c5.west)+(-0.3,-0.4)$) -- ($(c11.west |- c4.west)+(-0.3,0.4)$)
    node[midway,xshift=-1.4cm,rotate=0]{\textbf{Partie II}};

    \draw[decorate, decoration={brace, amplitude=15pt}, thick]
    ($(c11.west |- c10.west)+(-0.3,-0.4)$) -- ($(c11.west |- c6.west)+(-0.3,0.4)$)
    node[midway,xshift=-1.4cm,rotate=0]{\textbf{Partie III}};

    \draw[decorate, decoration={brace, amplitude=15pt}, thick]
    ($(c11.west |- c14.west)+(-0.3,-0.4)$) -- ($(c11.west |- c11.west)+(-0.3,0.4)$)
    node[midway,xshift=-1.4cm,rotate=0]{\textbf{Partie IV}};

    \draw[decorate, decoration={brace, amplitude=15pt}, thick]
    ($(c11.west |- c16.west)+(-0.3,-0.4)$) -- ($(c11.west |- c15.west)+(-0.3,0.4)$)
    node[midway,xshift=-1.4cm,rotate=0]{\textbf{Partie V}};

  \end{tikzpicture}
}

    }
    \caption{Schéma de l'organisation du manuscrit}
    \label{fig:organisation_manuscrit}
\end{figure}
%
Ce manuscrit vise ainsi à fournir une méthode complète, rigoureuse et opérationnelle pour concevoir des systèmes multi-agents organisés et autonomes au service de la cyberdéfense.

\cleardoublepage
