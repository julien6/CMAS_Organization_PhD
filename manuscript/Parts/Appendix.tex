\clearpage
\thispagestyle{empty}
\null
\newpage

\cleardoublepage
\phantomsection
\markboth{\spacedlowsmallcaps{Annexes}}{\spacedlowsmallcaps{Annexes}}
\part*{Annexes}
\label{part:annexes}

\clearpage
\thispagestyle{empty}
\null
\newpage

\chapter{Notations de la méthode MAMAD}

\section{Notations générales}

\begin{itemize}
    \item $S$ : ensemble des états possibles.
    \item $A_i$, $A$ : ensemble des actions de l’agent $i$, et ensemble conjoint des actions.
    \item $\Omega_i$, $\Omega$ : espace des observations de l’agent $i$, et espace conjoint.
    \item $H$, $H^{joint}$ : ensemble des historiques individuels et conjoints.
    \item $T$, $T_j$ : fonction de transition de l’environnement ou du jumeau numérique.
    \item $R$, $R^E_t$, $R^G_t$, $R^j_H$ : fonction de récompense (générale, par environnement, par objectif, ou basée sur les historiques).
    \item $\gamma \in [0,1]$ : facteur d’actualisation.
    \item $\pi_i$, $\pi$, $\pi^{joint}$ : politique individuelle, politique globale, politique conjointe.
    \item $\pi^*$ : politique optimale.
    \item $V^\pi$, $V^{\pi^j}_{T_j}$ : fonction de valeur ou fonction observation-valeur adaptée.
    \item $d \in D$ : un Dec-POMDP défini comme $d = (S,\{A_i\},T,R,\{\Omega_i\},O,\gamma)$.
    \item $d^\Omega \in OD_\Omega$ : un Observation-based Dec-POMDP (ODec-POMDP).
    \item $\tilde{h}_t$ : état caché récurrent estimé par le World Model.
\end{itemize}

\section{Notations pour l’activité de modélisation (MOD)}

\begin{itemize}
    \item $G_{\text{inf}}$ : objectif global informel.
    \item $S_{\text{inf}}$ : contraintes organisationnelles informelles.
    \item $E$ : description de l’environnement réel.
    \item $\Omega^j_0$ : ensemble des observations initiales conjointes.
    \item $(Enc, Dec)$ : auto-encodeur utilisé pour compresser et reconstruire les observations.
    \item $z_t$ : représentation latente d’une observation.
    \item $T_z$ : modèle récurrent de dynamique latente (RLDM).
    \item $T_j(h,\omega,a) = \langle \tilde{h}', P(\omega'|h,\omega,a)\rangle$ : JOPM prédisant la prochaine observation et mettant à jour l’état caché.
    \item $S^j_H$ : fonction d’arrêt basée sur les historiques.
    \item $Render^j_H$ : fonction optionnelle de rendu de trajectoires.
    \item $DH_j$ : ensemble d’historiques conjoints utilisés pour l’entraînement.
\end{itemize}

\section{Notations pour l’activité d’entraînement (TRN)}

\begin{itemize}
    \item $MM = \langle OS, ar, rcg, gcg, rag, rrg, grg \rangle$ : spécification MOISE+MARL.
    \item $ar : A \to R$ : relation assignant les agents à des rôles.
    \item $rcg : R \to \{rag, rrg\}$ : relation associant un rôle à un guide de contrainte.
    \item $gcg : G \to grg$ : relation liant un objectif à une contrainte.
    \item $rag(h,\omega)$ : guide d’actions attendu.
    \item $rrg(h,\omega,a)$ : guide de récompenses de rôle.
    \item $grg(h)$ : guide de récompenses d’objectifs.
    \item $\rho_i$ : rôle assigné à l’agent $i$.
    \item $m \in M$, $G_m$ : mission $m$ et ses objectifs associés.
    \item $cht$ : probabilité de conformité stricte à un rôle.
    \item $B$ : buffer d’expérience (transitions encodées).
\end{itemize}

\section{Notations pour l’activité d’analyse (ANL)}

\begin{itemize}
    \item $OF$ : Organizational Fit (adéquation organisationnelle globale).
    \item $SOF$, $FOF$ : Structural et Functional Organizational Fit.
    \item $D_{trans}$ : ensemble de séquences de transitions $(\omega_t, a_t, \omega_{t+1})$.
    \item $D_{obs}$ : ensemble de séquences d’observations $(\omega_t)$.
    \item $p \in P$ : Trajectory-based Pattern.
    \item $sl = \langle h, \{c_{min},c_{max}\}\rangle$ : séquence feuille (historique-cardinalité).
    \item $sn = \langle \langle sl_1, sl_2, ...\rangle, \{c_{min},c_{max}\}\rangle$ : séquence nœud.
    \item $l \in L$ : étiquette assignée à une observation ou action.
    \item $bg : H \to \{0,1\}$ : fonction testant l’appartenance d’un historique à un ensemble $H_g$.
\end{itemize}

\section{Notations pour l’activité de transfert (TRF)}

\begin{itemize}
    \item $\pi^{\text{latest}}_j$ : politique conjointe la plus récente.
    \item $need\_update$ : signal indiquant la nécessité d’une mise à jour.
    \item $launch\_update()$ : procédure déclenchant la mise à jour du modèle/politique.
    \item $batch\_size$ : taille minimale de $B$ pour déclencher une mise à jour.
    \item Modes de déploiement : DIRECT (local), REMOTE (distant).
\end{itemize}


\clearpage
\thispagestyle{empty}
\null
\newpage

% \chapter{Présentation illustrée de CybMASDE}

% TODO