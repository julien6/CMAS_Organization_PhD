
\cleardoublepage
\phantomsection
% \pdfbookmark[1]{Conclusion et perspectives}{Conclusion et perspectives}
\markboth{\spacedlowsmallcaps{Conclusion et perspectives}}{\spacedlowsmallcaps{Conclusion et perspectives}}
\part{Conclusion et perspectives}

\chapter*{Introduction}
\addcontentsline{toc}{chapter}{\textbf{Introduction}}
% TODO

\begin{figure}[h!]
    \centering
    \resizebox{\linewidth}{!}{%
        \begin{tikzpicture}[
    chapter/.style={draw, fill=blue!10, thick, minimum width=9cm, minimum height=1.2cm, text centered, font=\bfseries},
    section/.style={draw, fill=blue!5, thick, minimum width=8cm, minimum height=1cm, text centered, font=\small},
    arrow/.style={-Latex, thick},
    node distance=0.4cm
]

% Chapitre 15 : Bilan de la thèse
\node[chapter] (ch15) {Chapitre 15 : Bilan de la thèse};
\node[section, below=of ch15] (ch15s1) {Synthèse de la démarche et des résultats};
\node[section, below=of ch15s1] (ch15s2) {Limitations techniques, théoriques, expérimentales};

\draw[arrow] (ch15) -- (ch15s1);
\draw[arrow] (ch15s1) -- (ch15s2);

% Chapitre 16 : Perspectives et ouvertures
\node[chapter, below=1.6cm of ch15s2] (ch16) {Chapitre 16 : Perspectives et ouvertures};
\node[section, below=of ch16] (ch16s1) {Perspectives scientifiques à court et long terme};
\node[section, below=of ch16s1] (ch16s2) {Ouvertures industrielles, valorisation};

\draw[arrow] (ch16) -- (ch16s1);
\draw[arrow] (ch16s1) -- (ch16s2);

\end{tikzpicture}

    }
    \caption{Structure de la Partie V — Conclusion et perspectives}
\end{figure}

\chapter{Synthèse des apports et évaluation de la méthode}

\section{Résultats et discussion}
\label{sec:results}

Cette section présente les résultats obtenus par l'application de la méthode MAMAD sur les quatre environnements de test. L'évaluation suit les métriques définies et le protocole de validation, structurés selon les lacunes de recherche visées.

\subsection{G1 – Intégration du MARL dans l'AOSE (Efficacité)}

Nous évaluons tout d'abord l'efficacité de l'apprentissage des trois approches au regard des indicateurs suivants :

\begin{itemize}
    \item \textbf{Récompense cumulée} ($R_{cum}$) ;
    \item \textbf{Stabilité de la politique} ($\sigma_R$) ;
    \item \textbf{Taux de convergence} ($CR$) ;
    \item \textbf{Score de robustesse} ($R_{robust}$).
\end{itemize}

\begin{table}[h!]
    \centering
    \caption{Métriques d'efficacité selon les méthodes et environnements (G1)}
    \begin{tabular}{l|l|cccc}
        \hline
        \textbf{Env.} & \textbf{Méthode} & $R_{cum}$ & $\sigma_R$ & $CR$ & $R_{robust}$ \\
        \hline
        \multirow{3}{*}{Overcooked-AI}
                      & RB              & 85\%      & 8\%        & 220  & 75\%         \\
                      & OB              & 92\%      & 4\%        & 160  & 85\%         \\
                      & MB              & 95\%      & 3\%        & 150  & 90\%         \\
        \hline
        \multirow{3}{*}{Predator-Prey}
                      & RB              & 80\%      & 10\%       & 250  & 70\%         \\
                      & OB              & 88\%      & 5\%        & 180  & 80\%         \\
                      & MB              & 90\%      & 4\%        & 170  & 85\%         \\
        \hline
        \multirow{3}{*}{Warehouse}
                      & RB              & 83\%      & 9\%        & 230  & 72\%         \\
                      & OB              & 91\%      & 5\%        & 170  & 82\%         \\
                      & MB              & 93\%      & 4\%        & 160  & 87\%         \\
        \hline
        \multirow{3}{*}{CyberDefense}
                      & RB              & 78\%      & 12\%       & 280  & 65\%         \\
                      & OB              & 85\%      & 7\%        & 200  & 75\%         \\
                      & MB              & 87\%      & 6\%        & 190  & 80\%         \\
        \hline
    \end{tabular}
    \label{tab:g1_efficiency_full}
\end{table}

La méthode MAMAD (MB) surpasse systématiquement les deux lignes de base en termes de maximisation de la récompense et de stabilité. Le guidage organisationnel accélère la convergence ($CR$) et améliore la robustesse face aux perturbations ($R_{robust}$). Les gains les plus notables apparaissent dans des tâches hautement coopératives comme Overcooked-AI.

\subsection{G2 \& G3 – Conformité et Explicabilité}

Nous évaluons à présent la conformité aux contraintes et l'explicabilité des politiques via les métriques suivantes :

\begin{itemize}
    \item \textbf{Taux de violation des contraintes} ($V_c$) ;
    \item \textbf{Niveau de conformité organisationnelle} ($F_{org}$) ;
    \item \textbf{Score de cohérence} ($S_{cons}$).
\end{itemize}

\begin{table}[h!]
    \centering
    \caption{Métriques de conformité et d'explicabilité selon les méthodes (G2 \& G3)}
    \begin{tabular}{l|l|ccc}
        \hline
        \textbf{Env.} & \textbf{Méthode} & $V_c$ & $F_{org}$ & $S_{cons}$ \\
        \hline
        \multirow{3}{*}{Overcooked-AI}
                      & RB              & 15\%  & 70\%      & 65\%       \\
                      & OB              & 3\%   & 92\%      & 90\%       \\
                      & MB              & 2\%   & 95\%      & 93\%       \\
        \hline
        \multirow{3}{*}{Predator-Prey}
                      & RB              & 18\%  & 65\%      & 60\%       \\
                      & OB              & 5\%   & 88\%      & 85\%       \\
                      & MB              & 4\%   & 90\%      & 88\%       \\
        \hline
        \multirow{3}{*}{Warehouse}
                      & RB              & 12\%  & 68\%      & 63\%       \\
                      & OB              & 4\%   & 91\%      & 89\%       \\
                      & MB              & 3\%   & 94\%      & 91\%       \\
        \hline
        \multirow{3}{*}{CyberDefense}
                      & RB              & 20\%  & 60\%      & 55\%       \\
                      & OB              & 6\%   & 85\%      & 82\%       \\
                      & MB              & 5\%   & 87\%      & 85\%       \\
        \hline
    \end{tabular}
    \label{tab:g2_g3_full}
\end{table}

MAMAD atteint des niveaux élevés de conformité organisationnelle et de cohérence, proches des politiques guidées manuellement (OB). L'analyse fondée sur les trajectoires permet à MB d'inférer les structures organisationnelles avec un taux de violation très faible ($V_c$), surpassant même OB dans certains cas grâce à un meilleur alignement entre apprentissage et organisation cible.

\subsection{G4 – Capacité d'automatisation}

Enfin, nous évaluons la capacité de conception automatisée à l'aide des indicateurs suivants :

\begin{itemize}
    \item \textbf{Performance relative au temps de conception} ($T_{design}$) ;
    \item \textbf{Quantité de connaissances injectées} ($K_{design}$) ;
    \item \textbf{Nombre d'itérations jusqu'à convergence} ($N_{iter}$).
\end{itemize}

\begin{table}[h!]
    \centering
    \caption{Métriques d'automatisation selon les méthodes (G4)}
    \begin{tabular}{l|l|ccc}
        \hline
        \textbf{Env.} & \textbf{Méthode} & $T_{design}$ (jours) & $K_{design}$ (lignes) & $N_{iter}$ \\
        \hline
        \multirow{3}{*}{Overcooked-AI}
                      & RB              & 1.0                 & 20                   & 1          \\
                      & OB              & 2.5                 & 300                  & 2          \\
                      & MB              & 1.5                 & 120                  & 2          \\
        \hline
        \multirow{3}{*}{Predator-Prey}
                      & RB              & 1.2                 & 25                   & 1          \\
                      & OB              & 3.0                 & 320                  & 3          \\
                      & MB              & 1.8                 & 140                  & 3          \\
        \hline
        \multirow{3}{*}{Warehouse}
                      & RB              & 1.5                 & 30                   & 1          \\
                      & OB              & 3.5                 & 340                  & 2          \\
                      & MB              & 2.0                 & 150                  & 2          \\
        \hline
        \multirow{3}{*}{CyberDefense}
                      & RB              & 2.0                 & 40                   & 1          \\
                      & OB              & 4.0                 & 400                  & 3          \\
                      & MB              & 2.5                 & 180                  & 3          \\
        \hline
    \end{tabular}
    \label{tab:g4_full}
\end{table}

Bien que RB nécessite peu d'effort de conception, il ne garantit aucun contrôle organisationnel. OB implique un effort humain important ($K_{design}$ élevé) et des cycles de conception plus longs. MAMAD réduit significativement les spécifications manuelles tout en découvrant automatiquement des structures organisationnelles pertinentes en un nombre d'itérations limité.


\section{Synthèse et réponse à la question de recherche}

\section{Limitations techniques, théoriques, expérimentales}

\chapter{Perspectives et ouvertures}
\section{Perspectives académiques}
\section{Ouvertures et valorisations industrielles}

\chapter*{Conclusion}
\addcontentsline{toc}{chapter}{\textbf{Conclusion}}
% TODO
