\cleardoublepage
\phantomsection
% \pdfbookmark[1]{Etat de l'art}{Etat de l'art}
% \addcontentsline{toc}{part}{Etat de l'art}
\markboth{\spacedlowsmallcaps{Etat de l'art}}{\spacedlowsmallcaps{Etat de l'art}}
\part{Etat de l'art}
\label{part:methode}

\chapter*{Introduction}
\addcontentsline{toc}{chapter}{\textbf{Introduction}}
% TODO : Reformuler
Cette partie a pour but de fournir les bases et concepts nécéssaires à la compréhension des contributions.

\begin{figure}[h!]
    \centering
    \resizebox{\textwidth}{!}{%
        \begin{tikzpicture}[
    chapter/.style={draw, fill=blue!10, thick, minimum width=8cm, minimum height=1.2cm, text centered, font=\bfseries},
    section/.style={draw, fill=blue!5, thick, minimum width=7cm, minimum height=1cm, text centered, font=\small},
    arrow/.style={-Latex, thick},
    node distance=0.4cm
]

% Chapitre 4
\node[chapter] (ch4) {Chapitre 4 : Les SMA et concepts théoriques mobilisés};

\node[section, below=of ch4, xshift=-4cm] (ch4s1) {SMA et modèles organisationnels};
\node[section, below=of ch4s1] (ch4s2) {Apprentissage par renforcement multi-agent};
\node[section, below=of ch4s2] (ch4s3) {Modéliser un environnement simulé : les techniques \textquote{World Models}};

\draw[arrow] ($ (ch4.south) + (3,0) $) -- ++(0.,-0.9) -- (ch4s1.east);
\draw[arrow] ($ (ch4.south) + (3,0) $) -- ++(0.,-2.35) -- (ch4s2.east);
\draw[arrow] ($ (ch4.south) + (3,0) $) -- ++(0.,-3.75) -- (ch4s3.east);

% Chapitre 5
\node[chapter, below=5cm of ch4] (ch5) {Chapitre 5 : Revue de littérature, analyse et verrous liés aux hypothèses};

\node[section, below=of ch5, xshift=-4cm] (ch5s1) {\parbox{8cm}{Un cadre Markovien pour formaliser le problème \\ de conception et sa résolution en MARL (H1)}};
\node[section, below=of ch5s1] (ch5s2) {Les Worlds Models pour simuler des environnement multi-agents (H2)};
\node[section, below=of ch5s2] (ch5s3) {\parbox{8cm}{L'intégration de contraintes/guidages \\ organisationnelles dans le processus MARL (H3)}};
\node[section, below=of ch5s3] (ch5s4) {L'extraction automatisée des spécifications organisationnelles émergentes (H4)};

\draw[arrow] ($ (ch4.south) + (3.5,0) $) -- ($ (ch5.north) + (3.5,0) $);

\draw[arrow] ($ (ch5.south) + (3,0) $) -- ++(0.,-0.9) -- (ch5s1.east);
\draw[arrow] ($ (ch5.south) + (3,0) $) -- ++(0.,-2.35) -- (ch5s2.east);
\draw[arrow] ($ (ch5.south) + (3,0) $) -- ++(0.,-3.75) -- (ch5s3.east);
\draw[arrow] ($ (ch5.south) + (3,0) $) -- ++(0.,-5.35) -- (ch5s4.east);

\end{tikzpicture}

    }
    \caption{Structure de la Partie II — Etat de l'art}
\end{figure}

\chapter{Les SMA et concepts théoriques mobilisés}
% Objectif : définir les notions et modèles fondamentaux communs aux contributions.

\section{SMA et modèles organisationnels}

\section{Apprentissage par renforcement multi-agent}

\section{Modéliser un environnement simulé : les techniques \textquote{World Models}}

\chapter{Les verrous vers une méthode de conception}
% Objectif : pour chaque hypothèse H1-H5, analyser le domaine lié, identifier un manque, et formuler l'hypothèse comme réponse.

\section{Un cadre Markovien pour formaliser le problème de conception et sa résolution en MARL}
\section{Les Worlds Models pour simuler des environnement multi-agents}
\section{L'intégration de contraintes/guidages organisationnelles dans le processus MARL}
\section{L'extraction automatisée des spécifications organisationnelles émergentes}

\chapter*{Conclusion}
\addcontentsline{toc}{chapter}{\textbf{Conclusion}}
% TODO : Faire une conclusion
