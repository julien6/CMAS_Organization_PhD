\clearpage
\thispagestyle{empty}
\null
\newpage

\cleardoublepage
\phantomsection
% \pdfbookmark[1]{CONCLUSION GENERALE}{CONCLUSION GENERALE}
\addcontentsline{toc}{part}{CONCLUSION GENERALE}
\markboth{\spacedlowsmallcaps{CONCLUSION GENERALE}}{\spacedlowsmallcaps{CONCLUSION GENERALE}}
\noindent {\LARGE\textbf{CONCLUSION GENERALE}}

\

\vspace{1em}

\bigskip

\noindent {\Large\textbf{Contexte de travail}}\\

\noindent
Cette thèse s’inscrit dans un contexte de transformation rapide des systèmes intelligents distribués, caractérisés par leur complexité croissante, leur autonomie et leur besoin d’adaptabilité en environnement contraint. Dans des domaines tels que la cybersécurité, la robotique coopérative ou l’orchestration de services cloud, la conception de comportements collectifs efficaces, sûrs et interprétables reste un défi majeur. Parallèlement, l’apprentissage par renforcement multi-agent (MARL) a montré un fort potentiel pour produire automatiquement des politiques coordonnées dans des environnements partiellement observables.

Cependant, les approches classiques du MARL souffrent de plusieurs limites : un manque de contrôle sur les politiques apprises, une faible explicabilité des comportements collectifs, et une difficulté à intégrer des contraintes métier ou organisationnelles. Dans le champ de l'ingénierie dirigée par les modèles (Model-Driven Engineering) et de la conception de systèmes multi-agents (AOSE), les spécifications organisationnelles sont bien établies, mais leur intégration dans les processus d’apprentissage reste largement ouverte.

Face à cette double lacune, cette thèse propose une méthode pour articuler les spécifications organisationnelles avec l’apprentissage multi-agent, dans une logique de co-conception entre ingénierie des systèmes et apprentissage automatique. L’objectif principal est de concevoir une approche itérative, outillée et généralisable, permettant de guider, structurer et analyser le comportement de SMA par apprentissage, tout en assurant leur conformité à des objectifs d’organisation explicites.

\

\vspace{1em}

\bigskip

\noindent {\Large\textbf{Synthèse des contributions}}\\

\noindent
Pour répondre à cette problématique, la thèse introduit la méthode \textbf{MAMAD} (MOISE+MARL-based Automated Design), une approche en quatre étapes visant à supporter le cycle de conception semi-automatisé de SMA intelligents. Les contributions principales se répartissent selon les quatre phases du cycle :

\begin{itemize}
  \item \textbf{(1) Modélisation :} une approche hybride combinant \textit{World Models} et règles symboliques permet de simuler un environnement réaliste à partir de données collectées. Les contraintes organisationnelles sont formalisées sous forme de relations OAC (Observation–Action Constraint) et TRF (Trajectory-based Reward Function), offrant un encodage explicite des objectifs et des interdits.

  \item \textbf{(2) Apprentissage guidé :} le processus MARL est guidé par les contraintes précédemment définies, soit via un filtrage des actions, soit via une politique composite intégrant une pénalisation des violations. Ce cadre permet de restreindre l’espace de recherche tout en conservant la flexibilité d’adaptation propre au MARL.

  \item \textbf{(3) Analyse des politiques :} une méthode post-entraînement, appelée TEMM (Trajectory-based Evaluation in MOISE+MARL), permet d’analyser les comportements appris. Elle repose sur le clustering des trajectoires, l’inférence automatique de rôles et objectifs émergents, et le calcul de métriques comme le SOF (Structural Organizational Fit) et le FOF (Functional Organizational Fit).

  \item \textbf{(4) Transfert et déploiement :} un outil logiciel, CybMASDE, a été développé pour orchestrer l’ensemble du cycle. Il permet d’interfacer la simulation avec un environnement réel, d’automatiser les phases d’entraînement, et de superviser les agents via une API compatible avec des infrastructures comme Kubernetes.
\end{itemize}

La méthode a été évaluée sur quatre environnements aux dynamiques différentes : Overcooked-AI (coopération planifiée), Predator-Prey (coordination tactique), Warehouse (logistique structurée) et CyberDefense (défense distribuée). Les résultats montrent des gains significatifs en efficacité, en conformité aux contraintes, en explicabilité et en capacité d’automatisation par rapport aux approches classiques. MAMAD permet également d’inférer des structures organisationnelles implicites cohérentes avec les spécifications initiales.

L’ensemble de ces apports positionne la méthode à l’interface entre l’ingénierie organisationnelle et l’apprentissage automatique, et en fait un cadre unificateur pour la conception explicable de SMA par apprentissage.


\

\bigskip

\noindent {\Large\textbf{Limitations et perspectives}}\\

\noindent
Malgré ces contributions, plusieurs limites demeurent, tant sur le plan technique que conceptuel. La modélisation initiale de l’environnement nécessite encore des interventions humaines, notamment pour définir les règles de reconstruction ou les contraintes organisationnelles. L’apprentissage guidé peut perturber la convergence dans des environnements fortement stochastiques. L’analyse post-hoc repose sur des métriques internes, sans validation par des utilisateurs finaux. Enfin, les expérimentations restent centrées sur des environnements simulés.

Ces limites ouvrent des perspectives de recherche riches et concrètes :

\begin{itemize}
  \item \textbf{Vers une modélisation automatisée :} l’usage combiné de World Models, d’apprentissage symbolique et de modèles génératifs (LLM, causal discovery) pour construire automatiquement des simulateurs riches à partir de traces et de connaissances métiers.

  \item \textbf{Vers un guidage adaptatif :} la possibilité d’apprendre ou d’ajuster dynamiquement les contraintes organisationnelles selon les performances ou les retours humains, via du méta-apprentissage ou de l’apprentissage actif.

  \item \textbf{Vers une explicabilité organisationnelle formelle :} en couplant les analyses de rôle à des mécanismes XAI, et en développant une théorie de l’explicabilité structurelle fondée sur la décomposition des politiques.

  \item \textbf{Vers le transfert industriel :} plusieurs applications concrètes sont envisageables, notamment dans la supervision de flottes de drones, l’orchestration adaptative de micro-services (cloud), la cybersécurité active ou les systèmes autonomes critiques.

  \item \textbf{Vers une ouverture open-source :} la mise à disposition de l’outil CybMASDE comme framework modulaire permettrait de renforcer les collaborations scientifiques, de faciliter les réutilisations, et de soutenir une stratégie de valorisation hybride (académique/industrielle).
\end{itemize}

\noindent
En définitive, cette thèse propose une méthode complète, opérationnelle et modulable, pour concevoir des systèmes multi-agents intelligents dans des contextes contraints, tout en assurant leur explicabilité et leur conformité organisationnelle. Elle contribue à rapprocher deux mondes historiquement séparés : celui de la conception dirigée par les modèles et celui de l’apprentissage distribué.

En ouvrant la voie à des systèmes plus adaptatifs, explicables et contrôlables, elle constitue une base prometteuse pour une nouvelle génération de systèmes multi-agents capables de concilier autonomie locale et gouvernance globale.

\clearpage
\thispagestyle{empty}
\null
\newpage

\phantomsection
