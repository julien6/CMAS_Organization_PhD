\clearpage
\thispagestyle{empty}
\null
\newpage

\cleardoublepage
\phantomsection
% \pdfbookmark[1]{\acn{CONCLUSION} \acn{ET} PERSPECTIVES}{\acn{CONCLUSION} \acn{ET} PERSPECTIVES}
\markboth{\spacedlowsmallcaps{\acn{CONCLUSION} \acn{ET} PERSPECTIVES}}{\spacedlowsmallcaps{\acn{CONCLUSION} \acn{ET} PERSPECTIVES}}
\part*{\acn{CONCLUSION} \acn{ET} PERSPECTIVES}
\label{part:conclusion}

\clearpage
\thispagestyle{empty}
\null
\newpage


% === \acn{INTRODUCTION} \acn{DE} \acn{LA} \acn{CONCLUSION} ===
% Objectif : rappeler brièvement le fil directeur, introduire la question de recherche et annoncer la structure de la conclusion.
\noindent
Cette conclusion revient sur la question de recherche initiale, propose une synthèse des contributions et un bilan critique de la méthode \acn{MAMAD}, avant d’ouvrir sur des perspectives académiques et industrielles.

% ==============================
\section*{Synthèse des contributions et bilan}
\label{sec:synthese_bilan}

\noindent
Cette section vise à présenter une synthèse structurée des principaux apports de la thèse, en mettant en lumière les contributions spécifiques réalisées pour chaque activité clé du processus de conception et d’évaluation des systèmes multi-agents. Elle propose également une analyse transversale des résultats obtenus, ainsi qu’un bilan critique des limites identifiées et des pistes d’amélioration envisageables.

\subsection*{Rappel de la question de recherche}

\noindent
La problématique qui a guidé l’ensemble de cette thèse peut se formuler de la manière suivante~:

\begin{quote}
  \emph{Comment concevoir un \acn{SMA} de Cyberdéfense capable d'atteindre ses objectifs de défense de manière satisfaisante, tout en s'auto-organisant pour s'adapter dynamiquement aux contraintes de l'environnement et aux exigences de conception ?}
\end{quote}

\noindent
Cette question, posée en introduction, a permit de relier deux dimensions complémentaires~:
(i) la nécessité d’une ingénierie de conception systématique et rigoureuse des systèmes multi-agents, et
(ii) l’opportunité offerte par l’apprentissage par renforcement multi-agent pour faire émerger des comportements collectifs efficaces.

L’ensemble des contributions présentées dans ce manuscrit ont été développées et validées dans l’optique d’apporter une réponse argumentée à cette question de recherche. La section suivante en propose une synthèse structurée, critère par critère et activité par activité, afin de mettre en évidence les apports concrets, les points de couverture partielle et les perspectives d’amélioration.


% === \acn{CONTRIBUTIONS} \acn{PAR} ACTIVITÉ ===
\subsection*{Contributions spécifiques par activité}

Cette sous-section propose une analyse détaillée des contributions spécifiques apportées pour chaque activité clé identifiée au cours de la thèse. Pour chacune d’elles (modélisation, entraînement sous contraintes, analyse organisationnelle et transfert/supervision), le verrou scientifique initial est rappelé, la solution développée est explicitée, et son efficacité est illustrée à travers des validations ou des environnements expérimentaux représentatifs.

\paragraph{\acn{MOD} – Modélisation}

Le premier verrou identifié dans cette thèse concernait la difficulté de disposer d’un environnement de simulation fidèle, modulaire et exploitable par des algorithmes de type \acn{MARL}.
En effet, les environnements multi-agents de la littérature (tels que \textit{Overcooked-AI} ou \textit{Predator-Prey}) présentent deux limitations majeures :
(i) une difficulté à représenter de manière réaliste la complexité d’un système de cyberdéfense distribué, où interagissent simultanément des processus techniques et organisationnels ;
(ii) une absence de lien systématique avec un formalisme théorique explicite, garantissant la reproductibilité et la comparabilité des expériences.

Pour lever ce verrou, deux contributions principales ont été proposées.

\medskip
\noindent
\textbf{(i) Extension des \textit{World Models} au contexte multi-agent.}
Nous avons introduit une méthode permettant d’apprendre automatiquement un modèle de l’environnement (dynamique de transitions et d’observations) à partir de traces collectées, en généralisant les \textit{World Models} classiques au cas multi-agent.
Cette extension prend en compte à la fois les interactions simultanées entre agents et les contraintes organisationnelles pesant sur leurs actions.
Elle offre ainsi une capacité d’auto-génération d’environnements de simulation adaptés à des scénarios variés, tout en réduisant la dépendance à une modélisation experte exhaustive.

\medskip
\noindent
\textbf{(ii) Proposition du modèle \acn{MCAS}.}
En complément, nous avons développé un modèle formel \acn{Dec-POMDP} pré-spécialisé pour la cyberdéfense, appelé \textbf{MCAS}.
Celui-ci constitue une base de modélisation manuelle guidée, dans laquelle le concepteur dispose d’un canevas prêt à l’emploi pour instancier rapidement un environnement respectant les structures organisationnelles attendues (agents, rôles, missions, contraintes).
\acn{MCAS} permet ainsi de combiner modélisation experte et apprentissage automatique, en conservant une traçabilité forte entre les choix de modélisation et les dynamiques simulées.

\medskip
\noindent
\textbf{Intégration et validation.}
Ces deux contributions ont été intégrées dans la plateforme \acn{CybMASDE}, qui fournit un environnement d’exécution modulaire et reproductible pour les expériences.
Elles ont été validées sur plusieurs scénarios expérimentaux~:
\begin{itemize}
  \item \textit{Predator-Prey}, pour tester la robustesse du modèle d’environnement et la capacité à gérer des interactions compétitives simples.
  \item \textit{Company Infrastructure}, pour simuler une infrastructure organisationnelle de cyberdéfense et vérifier que la modélisation rend possible l’intégration de contraintes de sécurité réalistes.
  \item \textit{Drone Swarm}, pour évaluer la capacité du modèle à représenter un graphe dynamique de communication et de coordination multi-agents.
\end{itemize}

\noindent
Les résultats ont montré que la modélisation proposée répond à deux objectifs majeurs :
(i) fournir un cadre formel unifié pour la simulation de \acn{SMA} de cyberdéfense ;
(ii) offrir une flexibilité entre automatisation et expertise humaine, garantissant un compromis entre réalisme, généricité et reproductibilité.
Ainsi, les contributions de la phase \textbf{MOD} posent les fondations du pipeline \acn{MAMAD}, en assurant que l’apprentissage multi-agent repose sur des environnements crédibles et exploitables.


\paragraph{\acn{TRN} – Entraînement sous contraintes}

Le deuxième verrou identifié concernait la difficulté à orienter l’apprentissage multi-agent de manière à garantir, au-delà de la simple performance cumulative, des propriétés essentielles telles que la sûreté, la stabilité et la conformité organisationnelle.
En effet, les approches classiques de \ac{MARL} privilégient l’optimisation de la récompense globale, mais laissent peu de garanties sur le respect de contraintes critiques (non-interférence, cohérence des rôles, coordination selon une mission définie).
Dans le domaine de la cyberdéfense, un apprentissage non guidé peut ainsi mener à des comportements efficaces localement, mais dangereux ou incohérents du point de vue global.

\medskip
\noindent
\textbf{(i) Intégration du modèle organisationnel $\mathcal{M}\acn{OISE}^+$ dans le \acn{MARL}.}
Pour lever ce verrou, la thèse propose une intégration inédite entre le formalisme organisationnel \textbf{MOISE} et les algorithmes d’apprentissage par renforcement multi-agent, donnant naissance au cadre \textbf{\acn{MOISE}+MARL}.
Ce couplage permet de représenter explicitement les rôles, missions et permissions dans un graphe organisationnel, puis de traduire ces spécifications sous forme de \textit{guides de contraintes} appliqués pendant l’apprentissage.
Les actions explorées par les agents sont ainsi filtrées ou pondérées selon leur compatibilité avec les objectifs globaux définis par l’organisation.

\medskip
\noindent
\textbf{(ii) Mécanismes de sûreté et de stabilité.}
Ce guidage organisationnel a deux effets principaux~:
(i) il réduit l’espace des politiques explorées, limitant les comportements incohérents ou dangereux,
(ii) il améliore la stabilité de l’apprentissage en orientant plus rapidement les trajectoires vers des comportements compatibles avec la mission globale.
Les résultats ont mis en évidence une amélioration nette de la \textit{convergence} (récompense cumulée moyenne plus élevée, variance réduite $\sigma_R$) et une meilleure \textit{robustesse aux perturbations} (résilience face à l’injection d’événements inattendus).
De plus, ce filtrage contraint permet de conserver des garanties de sûreté organisationnelle, un point particulièrement critique pour les systèmes de cyberdéfense.

\medskip
\noindent
\textbf{(iii) Validation expérimentale multi-environnements.}
Le cadre MOISE+MARL a été validé dans plusieurs environnements de test~:
\begin{itemize}
  \item \textit{Company Infrastructure}, pour représenter une infrastructure critique de cyberdéfense et montrer que l’intégration de contraintes de sécurité améliore la cohérence des politiques de défense.
  \item \textit{Drone Swarm}, pour démontrer la capacité à stabiliser l’apprentissage dans des scénarios distribués et dynamiques, avec reconfiguration des rôles lors de la perte de nœuds.
  \item \textit{Predator-Prey}, comme environnement de référence simplifié, afin de mesurer l’impact du guidage contraint sur la vitesse de convergence et la robustesse face aux variations aléatoires.
\end{itemize}

\medskip
\noindent
\textbf{(iv) Impact méthodologique.}
La contribution \acn{TRN} illustre l’apport d’une approche \textit{organisation-aware} au sein de l’apprentissage multi-agent.
En intégrant des contraintes formelles, l’apprentissage n’est plus uniquement optimisé pour la performance brute, mais également pour la conformité organisationnelle, la sûreté et la transparence des décisions.
Ce résultat constitue un apport original, en ce qu’il démontre la possibilité de relier un formalisme symbolique ($\mathcal{M}\acn{OISE}^+$) et une technique d’optimisation numérique (\acn{MARL}) dans un cadre unifié.
Il s’agit d’un pas important vers des \acn{SMA} de cyberdéfense capables non seulement d’apprendre, mais aussi de respecter des règles de sûreté et de coordination explicites.

\paragraph{\acn{ANL} – Analyse organisationnelle}

Le troisième verrou identifié portait sur le manque d’\textbf{explicabilité} et de \textbf{capacité d’analyse organisationnelle} dans les systèmes multi-agents entraînés par renforcement.
Dans les approches classiques de \acn{MARL}, l’évaluation repose principalement sur des métriques numériques globales (récompense cumulée, taux de succès, stabilité), qui rendent compte de la performance mais non des mécanismes internes ayant conduit aux comportements observés.
Cette opacité limite la confiance des concepteurs, freine la validation par des experts humains, et rend difficile le transfert vers des environnements réels soumis à des contraintes fortes (comme la cybersécurité).

\medskip
\noindent
\textbf{(i) Proposition de la méthode \acn{TEMM}.}
Pour surmonter cette limite, nous avons introduit \textbf{TEMM}, une approche originale permettant d’analyser les comportements appris à travers leurs trajectoires.
L’idée centrale est de considérer qu’un ensemble de trajectoires reflète une organisation émergente implicite, que l’on peut mettre en évidence en identifiant :
\begin{itemize}
  \item des \textbf{rôles} (groupes d’agents adoptant des comportements similaires ou complémentaires),
  \item des \textbf{objectifs} (séquences d’actions convergeant vers un but commun),
  \item des \textbf{missions} (ensembles coordonnés de rôles et d’objectifs observés dans la dynamique collective).
\end{itemize}
\acn{TEMM} combine des techniques de \textit{clustering temporel}, de détection de séquences et d’analyse de graphes pour reconstruire cette organisation implicite, et la comparer aux spécifications organisationnelles attendues.

\medskip
\noindent
\textbf{(ii) Extension Auto-\acn{TEMM} : automatisation et optimisation des paramètres.}
Afin de renforcer la robustesse et la généricité de l’approche, nous avons développé une extension appelée \textbf{Auto-TEMM}.
Cette variante automatise le choix des hyperparamètres critiques (nombre de clusters, granularité temporelle, seuils de similarité), grâce à des techniques d’optimisation bayésienne et d’apprentissage actif.
Elle permet de réduire la dépendance à l’expertise humaine dans l’analyse, et d’appliquer la méthode de manière reproductible à un large éventail de scénarios.

\medskip
\noindent
\textbf{(iii) Explicabilité organisationnelle versus \acn{XAI} classique.}
Contrairement aux approches de type \textit{Explainable AI} (\acn{XAI}) centrées sur l’interprétation locale des décisions de modèles neuronaux (e.g., importance des features, attribution de gradients), \acn{TEMM} et Auto-\acn{TEMM} proposent une \textbf{explicabilité organisationnelle}.
Celle-ci vise non pas à expliquer une action isolée, mais à reconstruire et interpréter les \textit{schémas d’interaction collectifs} et la structure émergente d’un \acn{SMA}.
Cette perspective est particulièrement pertinente pour des systèmes distribués de cyberdéfense, où l’enjeu n’est pas seulement de justifier une décision individuelle, mais de comprendre la logique organisationnelle globale qui a conduit à la réussite (ou à l’échec) de la mission.

\medskip
\noindent
\textbf{(iv) Validation expérimentale et résultats obtenus.}
\acn{TEMM} et Auto-\acn{TEMM} ont été intégrés dans la plateforme \ac{CybMASDE} et testés sur plusieurs scénarios expérimentaux :
\begin{itemize}
  \item Dans l’environnement \textit{Company Infrastructure}, l’analyse a permis de reconstruire des rôles de type \textit{défenseur proactif} et \textit{superviseur de flux}, mettant en évidence la conformité des politiques apprises avec les spécifications de sécurité initiales.
  \item Dans l’environnement \textit{Drone Swarm}, \acn{TEMM} a révélé des missions émergentes correspondant à des schémas de couverture et de communication distribuée, non explicitement programmés par le concepteur, mais cohérents avec les contraintes organisationnelles.
  \item Dans l’environnement \textit{Predator-Prey}, Auto-\acn{TEMM} a démontré la capacité de l’approche à détecter automatiquement des rôles de \textit{poursuite} et de \textit{blocage}, tout en optimisant les paramètres d’analyse pour obtenir une représentation claire et reproductible.
\end{itemize}

\medskip
\noindent
\textbf{(v) Impact méthodologique et scientifique.}
La contribution \acn{ANL} démontre la faisabilité d’une analyse organisationnelle a posteriori, systématique et partiellement automatisée, des comportements de \acn{SMA} entraînés par renforcement.
Elle introduit une nouvelle forme d’explicabilité, centrée sur la \textbf{structure collective et organisationnelle} plutôt que sur la décision individuelle.
Cela constitue un apport original au domaine, en rapprochant les méthodes de \ac{MARL} des pratiques d’ingénierie dirigée par les modèles, et en ouvrant la voie à une validation interactive avec des experts humains.

\paragraph{\acn{TRF} – Transfert et supervision}

Le quatrième verrou identifié concernait la difficulté à assurer un \textbf{transfert fiable} des politiques apprises dans un environnement simulé vers un système réel, tout en garantissant la cohérence entre les deux mondes.
Dans la littérature \ac{MARL}, cette question est souvent traitée sous l’angle du \textit{sim-to-real transfer}, mais les solutions proposées restent limitées :
elles visent surtout à réduire l’écart de distribution entre simulation et réalité, sans prendre en compte la dimension organisationnelle et les contraintes propres aux systèmes critiques.
Or, dans un contexte de cyberdéfense ou d’orchestration cloud, il est indispensable de disposer de mécanismes permettant non seulement de transférer des comportements, mais aussi de superviser en continu leur adéquation aux contraintes et aux objectifs.

\medskip
\noindent
\textbf{(i) Introduction du jumeau numérique adaptatif.}
Pour répondre à ce verrou, nous avons proposé le concept de \textbf{jumeau numérique adaptatif}.
Il s’agit d’un modèle intermédiaire qui maintient une synchronisation dynamique entre l’environnement simulé (où l’apprentissage est réalisé) et l’environnement réel (où les agents sont déployés).
Le jumeau numérique est mis à jour en continu grâce à des flux de données issus du système réel (logs, métriques de performance, événements de sécurité), et il permet de réinjecter ces informations dans la simulation afin d’adapter les politiques ou de réviser les contraintes organisationnelles.
Ce mécanisme garantit une \textbf{cohérence structurelle et comportementale} entre simulation et réalité, réduisant ainsi le risque de dérive entre les deux contextes.

\medskip
\noindent
\textbf{(ii) Mécanismes de supervision et de reconfiguration.}
Le jumeau numérique adaptatif ne se limite pas au transfert initial.
Il fournit également une capacité de \textbf{supervision organisationnelle en ligne} :
les politiques déployées dans le système réel sont continuellement évaluées au regard des contraintes organisationnelles et des métriques de conformité définies (\acn{SOF}, \acn{FOF}, \acn{OF}).
En cas de non-conformité détectée, le système peut déclencher une \textbf{reconfiguration dynamique} (ajustement des rôles, redéfinition partielle des missions, ou reprise de l’apprentissage en simulation avec contraintes révisées).
Ce processus assure la continuité de la sûreté et de la robustesse, même face à des conditions imprévues ou évolutives.

\medskip
\noindent
\textbf{(iii) Validation expérimentale.}
Le concept de jumeau numérique adaptatif a été expérimenté dans plusieurs contextes représentatifs :
\begin{itemize}
  \item Dans l’environnement \textit{Kubernetes}, le jumeau numérique a permis de modéliser la distribution et la migration de services au sein d’un cluster, et de réinjecter en simulation les événements de charge ou de panne observés. Cela a démontré la faisabilité d’une orchestration \textit{auto-adaptative}, guidée par des contraintes organisationnelles.
  \item Dans l’environnement \textit{Company Infrastructure}, le mécanisme a été utilisé pour tester des politiques de défense dans un simulateur enrichi par des journaux de sécurité réels, montrant la capacité du système à s’ajuster aux menaces émergentes et aux reconfigurations de l’infrastructure.
  \item Dans un scénario simplifié de \textit{Drone Swarm}, le jumeau numérique a assuré la continuité entre un simulateur de communication ad hoc et un émulateur réseau réaliste, démontrant que les rôles appris pouvaient être maintenus et ajustés malgré des pertes de connectivité dynamiques.
\end{itemize}

\medskip
\noindent
\textbf{(iv) Impact méthodologique et applicatif.}
La contribution \acn{TRF} met en évidence une approche intégrée du transfert et de la supervision, qui dépasse la simple adaptation sim-to-real.
Elle propose un \textbf{cadre organisationnel adaptatif}, où la simulation n’est plus un environnement isolé mais un composant couplé en continu au système réel.
Ce résultat constitue un apport original de la thèse : il démontre la possibilité de concevoir des \acn{SMA} capables non seulement d’apprendre et de s’expliquer, mais aussi de \textbf{se transférer et se superviser dynamiquement} dans des environnements critiques et distribués.

% === \acn{CONTRIBUTIONS} \acn{TRANSVERSES} ===
\subsection*{Contributions techniques}

Au-delà des contributions spécifiques à chaque activité (\acn{MOD}, \acn{TRN}, \acn{ANL}, \acn{TRF}), la thèse apporte une contribution transversale majeure sur le plan technique~: le développement de la plateforme \textbf{CybMASDE}.
Celle-ci constitue l’implémentation intégrée du pipeline \ac{MAMAD}, et remplit plusieurs fonctions essentielles~:

\begin{itemize}
  \item \textbf{Environnement modulaire et reproductible} : CybMASDE offre une architecture modulaire permettant d’enchaîner de manière cohérente les étapes de modélisation, d’apprentissage, d’analyse et de transfert. Chaque composant (simulateur, algorithme \acn{MARL}, analyse organisationnelle, supervision) peut être utilisé indépendamment ou en combinaison.
  \item \textbf{Cadre expérimental générique} : la plateforme permet d’instancier une variété d’environnements (jeu coopératif, Predator-Prey, Company Infrastructure, Drone Swarm, Kubernetes), démontrant la généricité de la méthode au-delà du seul contexte de la cyberdéfense.
  \item \textbf{Traçabilité et reproductibilité} : toutes les expériences menées dans CybMASDE sont configurables par fichiers de paramètres, journalisées de manière systématique, et reproductibles. Cette approche garantit la rigueur scientifique et facilite la comparaison entre variantes méthodologiques.
  \item \textbf{Socle pour la valorisation} : en offrant une implémentation concrète du cadre \acn{MAMAD}, CybMASDE constitue un socle technique pour de futurs travaux académiques (open-source, réutilisation de modules) mais aussi pour une éventuelle industrialisation (intégration à des pipelines DevOps ou à des systèmes distribués).
\end{itemize}

Ainsi, CybMASDE n’est pas seulement un outil de validation, mais un \textbf{véritable cadre de conception et d’expérimentation}, traduisant en pratique l’ensemble des apports méthodologiques de la thèse.


\noindent
Bien que ces premiers résultats confirment la faisabilité du principe sous-entendant notre approche structurée de la conception d'un \acn{SMA}, ils ouvrent aussi directement sur les perspectives académiques et industrielles discutées dans la section suivante.

% ==============================
\section*{Perspectives et ouvertures}
\label{sec:perspectives}

% === \acn{BILAN} \acn{CRITIQUE} ===
\subsection*{Bilan et points d’amélioration}

L’évaluation des contributions montre que la méthode \ac{MAMAD} apporte une réponse largement positive à la question de recherche, en combinant modélisation formelle, apprentissage contraint, analyse organisationnelle et supervision adaptative.
Toutefois, plusieurs points perfectibles ont été identifiés. Loin de constituer des limites bloquantes, ils représentent des \textbf{leviers d’amélioration} qui ouvrent naturellement vers des perspectives académiques et industrielles.

\begin{itemize}
  \item \textbf{Automatisation de la modélisation} : la création initiale d’un simulateur reste coûteuse en expertise humaine (définition des règles, patterns organisationnels). L’approche des World Models a constitué une avancée significative dans la modélisation des dynamiques. Cependant, l’intégration de connaissances expertes explicites dans l’entraînement du World Model ou l’ajout de règles interprétables constituerait un axe d’amélioration majeur pour accroître la précision et la généricité de la modélisation.
  \item \textbf{Adaptativité des contraintes} : les contraintes organisationnelles intégrées dans l’apprentissage sont actuellement statiques. Ce constat ouvre la perspective d’un \textbf{méta-apprentissage organisationnel}, où les contraintes pourraient évoluer dynamiquement selon le contexte, le retour utilisateur ou les conditions opérationnelles.
  \item \textbf{Validation en conditions réelles} : les expérimentations menées reposent principalement sur des environnements simulés. Ce point invite à étendre la validation vers des systèmes réels ou hybrides (e.g., infrastructures cloud opérationnelles, prototypes robotiques, systèmes de cybersécurité distribués).
  \item \textbf{Coût computationnel} : certaines phases (apprentissage \acn{MARL}, analyse organisationnelle Auto-\acn{TEMM}) demeurent exigeantes en ressources. Ce point ouvre des perspectives en optimisation parallèle et en passage à l’échelle (\acn{GPU} multi-nœuds, exécution distribuée).
  \item \textbf{Évaluation centrée utilisateur} : l’explicabilité organisationnelle proposée reste évaluée via des métriques internes. Un prolongement naturel consiste à impliquer des experts humains pour juger la clarté, l’utilité et la pertinence des rôles et missions inférés. Ce travail d’évaluation a déjà été amorcé mais reste limité par la complexité de certains scénarios et le manque de lisibilité des éléments produits.
\end{itemize}

% === \acn{PERSPECTIVES} ACADÉMIQUES ===
\subsection*{Perspectives académiques}

Ces points perfectibles dessinent plusieurs pistes de recherche futures, que l’on peut organiser selon des horizons temporels.

\medskip
\noindent
\textbf{(i) Court terme : vers une modélisation davantage automatisée.}
Une amélioration nécessaire consiste à réduire l’effort d’ingénierie dans la création des environnements.
Il s’agit d’explorer la combinaison des \textbf{World Models multi-agents} avec des approches de \textbf{génération de règles interprétables} (LLMs, inférence symbolique), afin d’apprendre automatiquement les dynamiques tout en conservant une traçabilité compréhensible par un expert.

\medskip
\noindent
\textbf{(ii) Moyen terme : vers une adaptation dynamique des contraintes.}
La statique actuelle des contraintes appelle à la mise en place d’un \textbf{méta-apprentissage organisationnel}.
Les guides de contraintes pourraient ainsi s’adapter en fonction :
\begin{itemize}
  \item des erreurs observées pendant l’exécution,
  \item du feedback fourni par un expert humain,
  \item de l’évolution des conditions de l’environnement.
\end{itemize}
Cette perspective ouvre la voie à une intégration plus poussée de l’humain dans la boucle et à une meilleure résilience organisationnelle.

\medskip
\noindent
\textbf{(iii) Long terme : vers une explicabilité organisationnelle formelle.}
Si \acn{TEMM} et Auto-\acn{TEMM} ont permis une première forme d’explicabilité organisationnelle empirique, une formalisation plus rigoureuse est nécessaire.
À long terme, il s’agira de définir un \textbf{cadre logique et théorique}, reliant explicitement comportements observés, structures organisationnelles inférées et justifications causales.
Cela permettrait de renforcer la robustesse scientifique de l’explicabilité dans un contexte multi-agent.

\medskip
\noindent
\textbf{(iv) Transversal : validation centrée utilisateurs.}
Enfin, un axe transversal concerne l’implication systématique d’experts humains dans l’évaluation des rôles et missions inférés.
Une telle validation qualitative permettrait :
\begin{itemize}
  \item d’apprécier la lisibilité et la pertinence des structures explicitées,
  \item de tester l’utilisabilité de la plateforme CybMASDE comme outil de co-conception,
  \item de renforcer l’acceptabilité et le transfert de la méthode vers des environnements opérationnels.
\end{itemize}

\medskip
\noindent
Ainsi, ces perspectives académiques ne sont pas de simples prolongements, mais de véritables \textbf{paliers d’évolution} pour le cadre \acn{MAMAD}.
Elles visent à consolider la méthode sur le plan scientifique (formalisation, automatisation, passage à l’échelle) tout en la rapprochant des pratiques réelles de conception et d’évaluation des \acn{SMA}.

\subsection*{Ouvertures industrielles}

Au-delà de ses apports académiques, la méthode \ac{MAMAD} et son implémentation dans la plateforme \ac{CybMASDE} présentent plusieurs opportunités de valorisation dans des contextes industriels concrets.
Les environnements critiques et distribués, qu’ils soient physiques ou logiciels, posent des défis similaires de sûreté, de coordination et d’explicabilité, auxquels les contributions de cette thèse apportent des réponses adaptées.

\medskip
\noindent
\textbf{(i) Systèmes cyber-physiques et flottes autonomes.}
Les architectures multi-agents entraînées sous contraintes peuvent être déployées pour la supervision et la coordination de \textbf{flottes de drones}, de robots mobiles ou de véhicules connectés.
L’approche \acn{MAMAD} permet :
\begin{itemize}
  \item d’assurer une \textbf{résilience aux défaillances locales}, grâce à la redondance organisationnelle et aux reconfigurations dynamiques,
  \item de faciliter une \textbf{reconfiguration adaptative} de la mission en cas d’événement imprévu (perte de nœud, obstacle, panne),
  \item d’intégrer la supervision organisationnelle directement dans des middlewares robotiques (\acn{ROS}, \acn{DDS}).
\end{itemize}
Ces propriétés sont particulièrement adaptées aux applications de surveillance, de logistique ou de défense.

\medskip
\noindent
\textbf{(ii) Orchestration adaptative dans le cloud (Kubernetes).}
Dans les environnements cloud modernes, la gestion des ressources repose de plus en plus sur des architectures distribuées (micro-services, conteneurs).
L’intégration de \acn{MAMAD} dans des orchestrateurs tels que \textbf{Kubernetes} ouvre plusieurs perspectives~:
\begin{itemize}
  \item guider les stratégies d’élasticité et de migration des services à partir de contraintes organisationnelles explicites,
  \item organiser les politiques de contrôle en rôles spécialisés (planificateur, répartiteur, superviseur),
  \item tendre vers une \textbf{auto-gestion organisationnelle} des infrastructures, où les décisions sont prises localement mais restent cohérentes globalement.
\end{itemize}

\medskip
\noindent
\textbf{(iii) Sécurité informatique distribuée.}
Les environnements de cybersécurité constituent un terrain naturel pour la valorisation de \acn{MAMAD}.
La poursuite des travaux sur les agents \acn{AICA} à même de jouer un rôle encore plus important de \textbf{détection-réaction proactive}, avec supervision organisationnelle.
Les apports majeurs sont :
\begin{itemize}
  \item la capacité à apprendre des stratégies de défense robustes et distribuées,
  \item la possibilité d’orchestrer différents types d’agents (filtrage, surveillance, contre-mesure) selon une logique organisationnelle,
  \item la traçabilité et l’explicabilité des décisions, facilitant la certification et l’intégration dans des systèmes critiques.
\end{itemize}
Cela répond à un besoin industriel croissant : disposer de systèmes de défense intelligents, adaptatifs et auditables.

\medskip
\noindent
\textbf{(iv) Valorisation logicielle et ouverture open-source.}
La plateforme \acn{CybMASDE} pourrait être valorisée en tant que \textbf{cadre de prototypage open-source}, destiné à la recherche et à l’ingénierie des \acn{SMA} contraints.
Plusieurs pistes sont envisageables :
\begin{itemize}
  \item publication d’une version stable, documentée et modulaire,
  \item mise à disposition de composants réutilisables (modules \acn{TEMM}, \acn{HPO}),
  \item extension vers d’autres frameworks de simulation comme Gymnasium~\cite{kwiatkowski2024}, Isaac Gym~\cite{Makoviychuk2021}, \acn{JAX}~\cite{Frostig2019} avec JaxMARL~\cite{Rutherford2024}.
\end{itemize}
Une ouverture open-source sous licence permissive (\acparen{MIT}, \acparen{LGPL}) permettrait à la fois d’accélérer les collaborations académiques et de préparer une double stratégie de valorisation académique et industrielle.

\medskip
\noindent
\textbf{(v) Transfert vers d’autres secteurs industriels.}
Enfin, les principes de \acn{MAMAD} s’appliquent au-delà de la cyberdéfense et du cloud, dans tout domaine nécessitant des systèmes intelligents décentralisés~:
\begin{itemize}
  \item \textbf{Industrie 4.0} : coordination entre lignes de production et logistique distribuée,
  \item \textbf{Smart grids} : régulation organisationnelle des flux énergétiques,
  \item \textbf{Surveillance environnementale} : coopération entre capteurs et drones pour le suivi d’environnements sensibles.
\end{itemize}
Ces secteurs partagent un besoin commun : garantir la robustesse, l’explicabilité et la sûreté des décisions collectives dans des environnements dynamiques et contraints.

\medskip
\noindent
Ainsi, la méthode \acn{MAMAD} offre une \textbf{forte adéquation entre besoins industriels et apports scientifiques} :
elle propose à la fois un cadre théorique formel, un pipeline logiciel opérationnel, et des garanties de sûreté et d’explicabilité indispensables dans des environnements critiques.

% ==============================
% === CLÔTURE \acn{DE} \acn{LA} \acn{CONCLUSION} ===
% Objectif : finir sur une note positive et ouverte.
\bigskip
\noindent
En définitive, cette thèse propose une méthode pour la conception de systèmes multi-agents guidés par des contraintes organisationnelles.
Elle démontre la faisabilité et l’intérêt de l’approche, en montrant qu’il est possible de combiner modélisation formelle, apprentissage par renforcement, analyse organisationnelle et transfert supervisé dans un même cadre cohérent.

\medskip
\noindent
Les contributions majeures tiennent à la fois dans la structuration théorique (cadre \acn{MAMAD}, intégration de MOISE+MARL, explicabilité organisationnelle), dans les innovations méthodologiques (World Models multi-agents, \acn{TEMM} et \acn{Auto-TEMM}, jumeau numérique adaptatif), et dans le développement technique (plateforme \acn{CybMASDE}).
Ensemble, elles apportent une réponse à la question de recherche, tout en établissant un socle robuste pour des développements futurs.

\medskip
\noindent
Cette recherche ne se limite pas à une contribution académique : elle ouvre des pistes concrètes de valorisation industrielle dans des environnements critiques tels que la cybersécurité, le cloud distribué ou les systèmes cyber-physiques.
Elle illustre la pertinence d’une approche organisation-aware dans un contexte où l’autonomie, la sûreté et l’explicabilité sont devenues des exigences incontournables.

\medskip
\noindent
Au-delà des résultats obtenus, cette thèse trace des \textbf{lignes directrices pour les recherches à venir}.
Les perspectives académiques identifiées (automatisation de la modélisation, méta-apprentissage organisationnel, formalisation de l’explicabilité, validation centrée utilisateur) et les ouvertures industrielles (systèmes autonomes, cloud, cybersécurité, industrie 4.0, smart grids) témoignent de la richesse des prolongements possibles.

\medskip
\noindent
En somme, cette thèse contribue à rapprocher deux mondes trop souvent séparés~: celui de l’ingénierie formelle des systèmes et celui de l’apprentissage multi-agent, plus adaptatif.
Elle propose une vision intégrée et évolutive de la conception de \acn{SMA}, et ouvre la voie à une nouvelle génération de systèmes intelligents, \textbf{robustes, explicables et adaptés aux environnements critiques du futur}.



\clearpage
\thispagestyle{empty}
\null
\newpage
