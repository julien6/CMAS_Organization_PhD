\clearpage
\thispagestyle{empty}
\null
\newpage

\cleardoublepage
\phantomsection
% \pdfbookmark[1]{CONCLUSION ET PERSPECTIVES}{CONCLUSION ET PERSPECTIVES}
\markboth{\spacedlowsmallcaps{CONCLUSION ET PERSPECTIVES}}{\spacedlowsmallcaps{CONCLUSION ET PERSPECTIVES}}
\part*{CONCLUSION ET PERSPECTIVES}
\label{part:conclusion}

\clearpage
\thispagestyle{empty}
\null
\newpage

Pour conclure le manuscrit nous présentons d'abord une synthèse critique des apports de la thèse, de leur mise en œuvre et de leur impact au regard des objectifs initiaux. Puis nous ouvrons des perspectives à la fois académiques et industrielles.

La \autoref{sec:synthese_evaluation} revient sur l'ensemble des résultats obtenus, les met en perspective avec la question de recherche, et identifie les principales limites de la méthode. La \autoref{sec:perspectives} discute ensuite des prolongements possibles, qu’il s’agisse d’axes de recherche fondamentaux ou de pistes concrètes de valorisation et d’industrialisation. Il permet ainsi de tracer des lignes directrices pour des travaux futurs, dans la continuité ou en extension des contributions présentées dans ce manuscrit.


\section*{Synthèse des apports et évaluation de la méthode}
\label{sec:synthese_evaluation}

Cette section propose une vue d’ensemble des apports de la thèse et évalue leur pertinence au regard des objectifs initiaux. Il s’agit, d’une part, de dresser une synthèse structurée des contributions introduites dans les différentes parties du manuscrit, et d’autre part, d’en discuter les résultats expérimentaux selon les dimensions clés identifiées~: efficacité de l’apprentissage, conformité aux contraintes organisationnelles, explicabilité des politiques apprises, et capacité d’automatisation du processus de conception.

\subsection{Synthèse et réponse à la question de recherche}
\label{sec:synthese_recherche}

\noindent
Nous proposons ici une synthèse structurée des contributions de la thèse à partir des critères d’évaluation définis en introduction. Pour chacun des critères, nous donnons le niveau de couverture obtenu par la méthode \acn{MAMAD}, sur la base des résultats expérimentaux présentés précédemment. L’analyse distingue les critères fortement satisfaits, ceux partiellement couverts, et ceux encore peu adressés. Elle permet ainsi de formuler une réponse argumentée à la question de recherche.

\subsubsection*{Critères fortement couverts}

Les résultats expérimentaux montrent que la méthode couvre de manière robuste les critères suivants~:

\begin{itemize}
    \item \textbf{C4 – Respect des contraintes / sûreté :} la spécification des contraintes organisationnelles permet un contrôle explicite du processus d’apprentissage. Les résultats montrent une nette amélioration de la convergence ($CR$) et de la stabilité ($\sigma_R$), tout en orientant les politiques vers des comportements compatibles avec les objectifs définis. Par ailleurs, le filtrage des actions autorisées par les contraintes OAC réduit significativement l’espace des politiques explorées, contribuant à une accélération de l’apprentissage sans perte de performance.

    \item \textbf{C5 – Explicabilité et contrôle organisationnel :} la méthode TEMM permet d’inférer les objectifs implicites des agents et d’évaluer leur adéquation aux spécifications, à travers les métriques de cohérence ($S_{cons}$) et de conformité ($F_{org}$). Les outils d’analyse post-hoc (clustering, inférence de rôles, extraction de séquences typiques) structurent les comportements sous forme de rôles et d’objectifs, facilitant leur interprétation. Cette explicabilité organisationnelle va au-delà des approches XAI classiques centrées sur les réseaux de neurones.

    \item \textbf{C7 – Mesurabilité des performances :} les métriques introduites (SOF, FOF, OF) permettent de quantifier le degré de conformité entre les structures organisationnelles spécifiées et celles observées dans les trajectoires. Cette évaluation systématique constitue un apport original du cadre MAMAD et contribue à la comparabilité et l’amélioration incrémentale des solutions.
\end{itemize}

\subsubsection*{Critères partiellement couverts}

Certains critères sont partiellement satisfaits, bien que la méthode offre des leviers intéressants pour les améliorer~:

\begin{itemize}
    \item \textbf{C3 – Capacité d’adaptation :} la méthode prévoit une phase de transfert vers l’environnement cible, mais les expérimentations restent limitées à des environnements simulés. L’usage d’une API pour relier CybMASDE à un système réel ouvre cependant la voie à des validations ultérieures en conditions réelles. L’adaptation dynamique aux contraintes et aux variations de contexte reste encore à approfondir.

    \item \textbf{C8 – Coût de conception minimal :} bien que la méthode permette d’assembler des composants modulaires (\textit{World Models}, règles de reconstruction, contraintes), la création initiale du simulateur demande encore un effort d’ingénierie conséquent. L’automatisation de cette étape reste partielle. De plus, si l’architecture modulaire de CybMASDE facilite l’intégration avec des systèmes existants (Kubernetes, ROS), des adaptations spécifiques sont encore nécessaires, en particulier pour le monitoring en temps réel ou l’intégration dans des pipelines DevOps.

    \item \textbf{C7 – Mesurabilité des performances :} les expériences sont reproductibles dans l’environnement CybMASDE et les paramètres sont loggés à chaque phase. Cependant, la traçabilité complète des décisions d’apprentissage reste à renforcer, notamment en rendant les modèles de contraintes adaptatifs et historisés. Cette amélioration permettrait une meilleure reproductibilité organisationnelle et une évaluation plus fine des performances.
\end{itemize}

\subsubsection*{Critère faiblement couvert}

Un seul critère apparaît à ce stade comme faiblement adressé, ouvrant des perspectives importantes pour des travaux futurs~:

\begin{itemize}
    \item \textbf{C3 – Capacité d’adaptation :} bien que le cadre organisationnel offre un langage pour spécifier les contraintes, celles-ci sont fixées en amont de l’apprentissage. La méthode ne prend pas encore en compte des contraintes évolutives en fonction du contexte ou du retour utilisateur. L’ajout de mécanismes d’ajustement automatique ou d’apprentissage des contraintes (meta-learning) constitue un prolongement naturel pour renforcer l’adaptabilité organisationnelle.
\end{itemize}


\subsubsection*{Réponse à la question de recherche}

\noindent
L’objectif de la thèse était de répondre à la question suivante~:

\begin{quote}
    \emph{Peut-on structurer et guider le processus de conception de systèmes multi-agents à l’aide d’un cadre organisationnel formel intégré dans l’apprentissage par renforcement multi-agent, de manière à produire des comportements robustes, explicables, conformes et transférables ?}
\end{quote}

\noindent
Les résultats expérimentaux, l’analyse comparative des méthodes, et l’évaluation au regard des dix critères définis montrent que la méthode \acn{MAMAD} apporte une réponse largement positive à cette question. Elle permet de structurer la conception par une organisation explicite, de guider l’apprentissage grâce à des contraintes formelles, et d’analyser les comportements a posteriori pour en évaluer la conformité.

En combinant spécification, apprentissage et analyse, MAMAD propose une approche originale, systémique et itérative, capable de produire des systèmes multi-agents plus sûrs, compréhensibles et adaptés à des environnements contraints. Elle ouvre ainsi la voie à de nouvelles méthodes de conception de SMA intelligents, à la frontière entre ingénierie dirigée par les modèles et apprentissage multi-agent.

\subsection{Limitations techniques, théoriques, expérimentales}
\label{sec:limitations}

\noindent
Malgré les résultats positifs obtenus et la couverture étendue des critères d’évaluation, la méthode \acn{MAMAD} présente encore plusieurs limites qui méritent d’être exposées de manière explicite. Ces limites concernent à la fois les aspects techniques de mise en œuvre, les fondements théoriques du cadre proposé, et la portée expérimentale des validations réalisées. En faire l’inventaire permet d’identifier des leviers d’amélioration concrets pour de futurs travaux.

\subsubsection*{Limitations techniques}

\begin{itemize}
    \item \textbf{Complexité de la modélisation initiale~:} la création d’un environnement simulé fidèle, capable de capturer les dynamiques d’un système réel, demande encore une intervention humaine non négligeable. L’écriture de règles de reconstruction, la définition des patterns, ou l’ajustement du \textit{World Model} nécessitent expertise et itérations.

    \item \textbf{Sensibilité à la qualité des trajectoires~:} l’analyse des comportements et l’apprentissage de structures organisationnelles sont fortement dépendants des trajectoires collectées. Si celles-ci sont bruitées, incomplètes ou biaisées, les rôles et objectifs inférés risquent d’être peu représentatifs.

    \item \textbf{Charge computationnelle~:} bien que l’approche soit modulaire, certaines phases (notamment l’apprentissage MARL ou le clustering temporel pour TEMM) restent coûteuses en ressources. Cela peut limiter l’application à des scénarios à très grande échelle sans optimisation spécifique.

    \item \textbf{Intégration dans un cycle de développement logiciel~:} bien que CybMASDE propose une API, l’intégration fluide dans des workflows industriels (CI/CD, supervision temps réel, standards de sécurité) reste à améliorer. Le prototypage est efficace, mais la transition vers des outils de production nécessite encore du travail d’ingénierie.
\end{itemize}

\subsubsection*{Limitations théoriques}

\begin{itemize}
    \item \textbf{Hypothèse d’une organisation formelle définissable~:} la méthode suppose que le concepteur est en mesure d’exprimer des contraintes ou des objectifs organisationnels pertinents, ce qui n’est pas toujours le cas. Dans certains systèmes émergents ou adaptatifs, cette modélisation préalable peut être difficile ou incomplète.

    \item \textbf{Absence de garantie de convergence sous contrainte~:} l’introduction de contraintes dans le processus d’apprentissage peut théoriquement perturber la convergence vers une politique optimale. Dans un cadre Dec-POMDP contraint, peu de résultats garantissent l’optimalité ou la stabilité dans des contextes complexes ou stochastiques.

    \item \textbf{Validité de l’analyse post-hoc~:} les métriques SOF/FOF proposées pour évaluer l’organisation émergente sont informatives mais ne reposent pas encore sur un cadre théorique formellement établi. Leur interprétation reste donc contextuelle et nécessite une validation complémentaire (e.g., par des experts du domaine).
\end{itemize}

\subsubsection*{Limitations expérimentales}

\begin{itemize}
    \item \textbf{Environnements simulés et simplifiés~:} bien que variés (jeu coopératif, prédateurs-proies, entrepôt, cybersécurité), les environnements utilisés sont encore éloignés de la complexité des systèmes réels. Des éléments critiques comme l’imprécision sensorielle, les pannes, les attaques imprévues ou les dynamiques humaines n’ont pas été testés.

    \item \textbf{Validation centrée sur des métriques internes~:} les résultats sont évalués via des scores numériques objectifs, mais peu de validation qualitative ou humaine a été menée. Or, l’explicabilité et la conformité sont également des notions subjectives, qui gagneraient à être évaluées par des utilisateurs experts ou finaux.

    \item \textbf{Limite au niveau du transfert effectif~:} même si la méthode prévoit une phase de transfert, celle-ci n’a pas encore été pleinement expérimentée dans un environnement physique ou connecté (e.g., système Kubernetes réel, flotte de drones en conditions extérieures). Le potentiel de déploiement reste donc à confirmer.
\end{itemize}

\subsubsection*{Bilan}

\noindent
Ces limitations ne remettent pas en cause la validité générale du cadre proposé, mais elles en bornent la portée actuelle. Elles ouvrent également un espace riche pour des travaux futurs, à la fois sur le plan théorique (garanties sous contraintes, méta-apprentissage organisationnel), technique (automatisation complète de la modélisation, passage à l’échelle), et pratique (validation sur systèmes industriels réels avec retour utilisateur). Le chapitre suivant explore ces perspectives en détail.


\section*{Perspectives et ouvertures}
\label{sec:perspectives}

\noindent
Après avoir présenté les contributions de la méthode \acn{MAMAD} et évalué ses résultats ainsi que ses limites, cette section explore les perspectives qu’ouvre ce travail, tant sur le plan académique que dans une optique de valorisation industrielle.

\subsection{Perspectives académiques}
\label{sec:perspectives_academiques}

\noindent
La méthode \acn{MAMAD} ouvre plusieurs pistes de recherche à court, moyen et long terme. Ces perspectives concernent autant l’approfondissement des fondements théoriques du cadre proposé que l’extension de ses capacités techniques ou méthodologiques.

\subsubsection*{Vers une modélisation entièrement automatisée}

Actuellement, la modélisation de l’environnement simulé repose encore partiellement sur des règles définies par l’expert humain (e.g., règles de reconstruction, patterns organisationnels). Un objectif à court terme est d’automatiser cette étape en combinant~:
\begin{itemize}
    \item des techniques de \textbf{World Models} pour apprendre les dynamiques de l’environnement à partir des données~;
    \item des méthodes d’\textbf{inférence symbolique} ou de transformation de langage (LLM) pour générer automatiquement des règles ou des patterns interprétables à partir de traces.
\end{itemize}

\subsubsection*{Vers un apprentissage organisationnel adaptatif}

Une limite importante identifiée est le caractère statique des contraintes. À moyen terme, il est souhaitable de développer un mécanisme d’\textbf{adaptation dynamique des contraintes}, capable de faire évoluer les \textquote{guides de contraintes} pendant l’apprentissage, en fonction :
\begin{itemize}
    \item des erreurs d’exécution observées ;
    \item du feedback du concepteur humain ;
    \item d’une stratégie de méta-apprentissage.
\end{itemize}

Cela nécessite d’explorer les liens entre MARL contraint, apprentissage actif, et optimisation multi-objectifs adaptative.

\subsubsection*{Vers une explicabilité organisationnelle formelle}

L’un des apports clés de MAMAD est la capacité à structurer les comportements appris à travers des rôles et objectifs organisationnels. Cette explicabilité est encore principalement empirique. À plus long terme, il serait pertinent de formaliser ces mécanismes à l’aide :
\begin{itemize}
    \item d’un \textbf{cadre logique} pour relier comportements observés, structures organisationnelles et justifications causales ;
    \item de méthodes XAI (e.g., LIME, SHAP) adaptées aux structures multi-agents, afin d’expliquer non seulement les décisions individuelles, mais les schémas d’interaction collectifs.
\end{itemize}

\subsubsection*{Vers une validation opérationnelle de la méthode}

Enfin, les évaluations ont été conduites selon des métriques internes. Une évolution naturelle consisterait à impliquer des \textbf{experts humains} dans le processus de validation, en particulier pour :
\begin{itemize}
    \item évaluer la clarté et l’utilité des rôles inférés ;
    \item ajuster les spécifications organisationnelles de manière interactive ;
    \item tester l’utilisabilité de CybMASDE dans des cas d’usage concrets.
\end{itemize}

Ces perspectives s’inscrivent dans une démarche d’ingénierie dirigée par les utilisateurs et de co-conception avec l’humain dans la boucle.

\subsection{Ouvertures et valorisations industrielles}
\label{sec:perspectives_industrielles}

\noindent
La méthode \acn{MAMAD} et son implémentation dans l’outil CybMASDE présentent plusieurs opportunités de valorisation dans des contextes industriels concrets, en particulier dans des systèmes critiques et distribués.

\subsubsection*{Systèmes cyber-physiques et flottes autonomes}

Les architectures multi-agents apprises sous contraintes peuvent être appliquées à la \textbf{supervision décentralisée de flottes de drones} ou de robots mobiles. La possibilité de spécifier des rôles et missions, combinée à une analyse organisationnelle en ligne, peut permettre :
\begin{itemize}
    \item une meilleure \textbf{résilience aux défaillances locales} ;
    \item une \textbf{reconfiguration dynamique} de la mission en cas d’imprévu ;
    \item une intégration avec des systèmes robotiques embarqués via ROS ou DDS.
\end{itemize}

\subsubsection*{Orchestration adaptative dans le cloud (Kubernetes)}

Dans les architectures micro-services (e.g., Kubernetes), les agents peuvent apprendre à allouer dynamiquement les ressources selon des objectifs organisationnels. L’approche MAMAD pourrait ici :
\begin{itemize}
    \item servir à guider les stratégies d’élasticité et de migration ;
    \item structurer les politiques de contrôle par rôles (planificateur, répartiteur, superviseur) ;
    \item fournir un socle pour l’\textbf{auto-gestion organisationnelle} des services cloud.
\end{itemize}

\subsubsection*{Sécurité informatique distribuée}

L’application à l’environnement CyberDefense montre que des agents guidés par des contraintes peuvent être déployés comme \textbf{défenseurs proactifs} dans un système de détection-réaction. L’intérêt principal réside dans :
\begin{itemize}
    \item la capacité à apprendre des schémas de défense robustes et distribués ;
    \item la supervision organisationnelle de plusieurs types d’agents (firewall, nettoyage, contre-mesure) ;
    \item la traçabilité des décisions à des fins de vérification ou de certification.
\end{itemize}

\subsubsection*{Valorisation logicielle et ouverture open-source}

CybMASDE pourrait être proposé comme \textbf{cadre de prototypage open-source} pour l’analyse organisationnelle en MARL, notamment en :
\begin{itemize}
    \item publiant une version stable avec documentation et exemples~;
    \item proposant des modules indépendants réutilisables (e.g., TEMM, HPO, OAC)~;
    \item facilitant l’extension vers d’autres frameworks de simulation ou d’entraînement.
\end{itemize}

Une ouverture sous licence open-source (e.g., MIT ou LGPL) permettrait à la fois d’accélérer les collaborations scientifiques et d’envisager une double stratégie de valorisation académique et commerciale.

\subsubsection*{Transfert vers d'autres secteurs}

Enfin, l’approche pourrait être testée dans d'autres domaines à forte dimension organisationnelle~:
\begin{itemize}
    \item \textbf{Industrie 4.0~:} coordination entre unités de production et logistique.
    \item \textbf{Smart grid~:} régulation distribuée des flux énergétiques.
    \item \textbf{Surveillance environnementale~:} coordination de capteurs ou drones autonomes.
\end{itemize}

Ces secteurs partagent un besoin de systèmes intelligents décentralisés, gouvernés par des logiques métier structurantes, ce qui correspond pleinement aux apports de \acn{MAMAD}.


\clearpage
\thispagestyle{empty}
\null
\newpage
