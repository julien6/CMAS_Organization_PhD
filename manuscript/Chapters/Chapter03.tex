%************************************************
\chapter{Problèmes détaillées et hypothèses de travail}\label{ch:problem}
%************************************************


\section{Problèmes à considérer}

Dans le cadre de cette thèse, plusieurs défis et problèmes clés émergent de la littérature sur les systèmes multi-agents (SMA), l’apprentissage par renforcement multi-agent (MARL), l'explicabilité des modèles organisationnels, et la cyberdéfense. Ces problèmes doivent être pris en compte pour répondre efficacement à la question de recherche.

\paragraph{Problème 1 : Complexité et évolutivité des environnements cyberdéfensifs.}
Les environnements de cybersécurité, tels que les réseaux d'entreprise ou les infrastructures critiques, sont souvent dynamiques, évolutifs et très complexes \cite{Calo2017, Kott2019}. Cette complexité pose un problème majeur pour la conception d'un SMA capable de s'adapter en temps réel aux cyberattaques sophistiquées. Il n'existe pas de méthode unifiée dans la littérature pour organiser les agents de manière à ce qu'ils puissent efficacement prendre des décisions autonomes dans ces environnements sous forte contrainte.

\paragraph{Problème 2 : Entraînement sans supervision des agents.}
L'entraînement des agents AICA repose principalement sur des processus d'apprentissage autonome, notamment à travers le MARL, qui permet aux agents de découvrir et d'apprendre des stratégies de défense optimisées \cite{Jamont2015, Theron2020}. Toutefois, les scénarios actuels dans la littérature montrent une lacune dans la manière dont ces agents peuvent s'adapter à des environnements divers et hautement évolutifs, tels que ceux soumis à des malwares intelligents. De même, il y a peu de travaux qui traitent de moyens pour les agents d'apprendre de nouvelles stratégies dans un cadre contraignant arbitrairement défini en plus de l'environnement de déploiement.

\paragraph{Problème 3 : Explicabilité des agents et des systèmes multi-agents.}
Les systèmes multi-agents sont souvent perçus comme des "boîtes noires" en raison de la complexité des décisions qu’ils prennent de manière décentralisée \cite{Theron2018}. Cela soulève des questions cruciales sur la capacité à comprendre et à expliciter les comportements des agents après leur entraînement. Il est essentiel de rendre explicite l'organisation des agents (rôles, missions) pour garantir que leurs décisions ne compromettent pas la sûreté de l'environnement cible.

\paragraph{Problème 4 : Risques liés à l'expérimentation directe en environnement réel.}
Tester directement des agents dans des environnements réels (réseaux d'entreprise, infrastructures critiques) pourrait causer des dommages irréversibles si les agents ne fonctionnent pas correctement \cite{Calo2017}. La nécessité de simuler ces environnements et de reproduire des cyberattaques réalistes devient donc impérative pour éviter les risques liés à l’expérimentation directe.

Ces problèmes identifiés dans la littérature justifient l'exploration de nouvelles solutions pour organiser et entraîner des agents AICA dans des SMA de cyberdéfense adaptatifs.


\section{Hypothèses et contributions proposées}

Cette thèse se situe à l'intersection de la Cyberdéfense, des systèmes multi-agents (SMA), de la prise de décision autonome via l'apprentissage par renforcement multi-agent (MARL) et de l'explicabilité en intelligence artificielle (XAI). Bien que les recherches existantes aient exploré divers aspects des SMA en Cyberdéfense, plusieurs verrous persistent, notamment en termes de coordination, d'adaptabilité et d'évolutivité dans des systèmes décentralisés. Les travaux précédents se sont principalement concentrés sur des modèles centralisés ou des applications spécifiques des SMA, tels que la détection d'intrusions ou la gestion de logiciels malveillants. Cependant, il manque un cadre complet qui aborde systématiquement la conception, la mise en œuvre et l'évaluation des SMA dans des environnements de Cyberdéfense distribués.

Pour répondre aux défis identifiés, nous proposons les hypothèses suivantes qui guideront le développement de notre méthode de conception et d'organisation des agents dans un SMA de Cyberdéfense.

\paragraph{Hypothèse 1 : Reproduction en simulation des environnements et des attaques.}
Nous postulons qu'en reproduisant fidèlement les environnements cibles et les scénarios d'attaques dans une simulation, nous pouvons entraîner les agents sans risque pour l’environnement réel. Cette méthode permet d'explorer en toute sécurité des stratégies de défense autonomes et d'analyser les réponses des agents face à des cybermenaces complexes.

\paragraph{Hypothèse 2 : Entraînement des agents via MARL pour découvrir des stratégies optimisées.}
Nous faisons l'hypothèse que le MARL est une méthode efficace pour permettre aux agents d’apprendre à se défendre de manière autonome et collaborative contre des attaques sophistiquées. Le MARL permet de surmonter les limitations des approches non supervisées en exploitant les interactions entre agents pour générer des stratégies de défense robustes. De plus, il offre la possibilité d’imposer des contraintes supplémentaires lors de l'apprentissage, afin de garantir le respect des exigences de conception.

\paragraph{Hypothèse 3 : Explicabilité des stratégies des agents en termes organisationnels.}
Nous postulons qu'une fois les agents entraînés, leur comportement peut être expliqué en termes organisationnels, tels que des rôles ou des missions, facilitant ainsi la compréhension des stratégies développées par les concepteurs humains. Cela répond à la nécessité de garantir que les actions des agents sont sûres, interprétables et conformes aux attentes avant leur déploiement dans des environnements réels.

\paragraph{Hypothèse 4 : Validation et ajustement des SMA avant déploiement réel.}
Nous postulons que la validation du SMA dans une copie émulée de l’environnement cible permettra de vérifier l'efficacité des agents et d'ajuster leur organisation avant leur déploiement final. Cette étape est cruciale pour minimiser les risques de dysfonctionnement dans un environnement réel.

\

Ces hypothèses seront testées et validées tout au long de cette thèse, à travers une série d’expérimentations basées sur la simulation d’environnements de Cyberdéfense complexes et évolutifs. Les contributions proposées visent à combler les lacunes identifiées dans la littérature et à développer un SMA de Cyberdéfense autonome, efficace et explicable.

\

En cohérence avec les hypothèses énoncées, nous avons structuré nos contributions en trois volets : un \textbf{modèle} qui pose le problème et fourni un cadre de résolution, une \textbf{approche} qui met en oeuvre les moyens de résolutions de façon cohérente et un \textbf{outil} qui met en oeuvre le modèle et l'approche.

\textbf{Modèle - \textit{Cyberdefense MAS Formal Model} (CybSMAFM)} : Ce modèle formel intègre le problème de l’organisation des SMA de Cyberdéfense en modélisant des scénarios dans des environnements donnés. Chaque scénario inclut des agents attaquants (ou rouges) déployés dans un environnement en réseau pour attaquer des services ou utilisateurs (agents verts), protégés par des agents défenseurs. Le modèle repose sur le cadre des \textit{Decentralized Partially Observable Markov Decision Process} (Dec-POMDP), offrant une flexibilité pour modéliser divers environnements et scénarios. Il permet de formaliser ces scénarios à des fins de simulation ou d’émulation, ou encore de représenter des environnements réels pour résoudre des problématiques spécifiques.
\begin{itemize}
    \item Les contributions liées à ce modèle se basent principalement sur les hypothèses 1 et 4.
\end{itemize}

\textbf{Approche - \textit{Cyberdefense MAS Design Approach} (CybSMADA)} : Cette approche, s'appuyant sur le modèle précédent, intègre le couplage MARL et modèle organisationnel pour générer automatiquement des politiques d’agents de Cyberdéfense qui répondent aux exigences et autres contraintes données. Les agents apprennent à maximiser leur performance sous des contraintes environnementales et organisationnelles (rôles, missions). L'approche vise à créer des SMA adaptatifs et résilients, capables de répondre en temps réel à des menaces émergentes en entrainant les agents sur une gamme étendue de scénario en fonction des besoins du concepteur.

Par ailleurs, l’approche inclut des méthodes d’explicitation des politiques d’agents, généralement perçues comme des \textquote{boîtes noires}, pour les transformer en spécifications organisationnelles exploitables. Cela permet aux concepteurs humains de comprendre et d’affiner les stratégies de Cyberdéfense générées automatiquement. Enfin, l'approche propose de transférer la logique des agents de la simulation à l’émulation, minimisant ainsi le coût de conception en ajustant les politiques si nécessaire.
\begin{itemize}
    \item Les contributions liées à cette approche se fondent essentiellement sur les hypothèses 2 et 3.
\end{itemize}

\textbf{Outil - \textit{Cyberdefense MAS Development Environment} (CybMASDE)} : Ce volet regroupe le développement de l’outil qui implémente la méthode, en automatisant le processus ou en assistant le concepteur. L’outil utilise des frameworks et plateformes existants pour simuler et émuler les SMA de Cyberdéfense. De plus, trois études de cas (essaims de drones, infrastructure réseau d’entreprise, architecture de microservices) ont permis d'améliorer l'outil. Ces études de cas ont servi à améliorer itérativement l’efficacité de la méthode vis à vis de l'adaptabilité et la résilience des SMA suggerés par rapport aux systèmes centralisés traditionnels.
\begin{itemize}
    \item Cet outil repose sur l’ensemble des hypothèses, avec une attention particulière à l’hypothèse 4.
\end{itemize}
