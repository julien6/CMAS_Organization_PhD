\begin{table}[h!]
  \centering
  \caption{Synthèse des travaux retenus, verrous, limites et besoins méthodologiques par hypothèse}
  \label{tab:synthese_hypotheses}
  \renewcommand{\arraystretch}{1.2}
  {%
    \footnotesize
    \begin{tabularx}{\textwidth}{cXXXX}
      \hline
      \textbf{Hyp.}
       & \textbf{Travaux retenus}
       & \textbf{Verrous principaux}
       & \textbf{Limites des travaux existants}
       & \textbf{Besoins méthodologiques}                                                                                                                                                                      \\
      \hline

      \textbf{H-MOD}
       & Dec-POMDP comme cadre adaptable ; World Models pour la modélisation automatique
       & Lourdeur de la modélisation manuelle ; absence de bibliothèques spécialisées cyber ; extension multi-agent des World Models non résolue
       & World Models limités au mono-agent ou à des contextes simples ; peu d'approches exploitant des observations distribuées ; modèles markoviens trop simples pour aider à la modélisation manuelle
       & Apprendre un World Model multi-agent structuré autour de représentations latentes ; proposer un modèle markovien utilisable pour modéliser un environnement de cyberdéfense                           \\
      \hdashline

      \textbf{H-TRN}
       & Safe RL (CPO, DCQL) ; Constraint-Guided RL ; intégration potentielle de modèles organisationnels (MOISE+)
       & Faible expressivité organisationnelle des méthodes existantes ; absence de cadre unifié entre organisation symbolique et MARL ; manque de garanties globales
       & Constrained RL ne prend pas en compte les structures organisationnelles ; intégration de spécifications symboliques limitée à l'exécution
       & Introduire des contraintes symboliques dans le processus MARL pour guider l'apprentissage et filtrer les actions                                                                                      \\
      \hdashline

      \textbf{H-ANL}
       & MAVIPER et modèles interprétables ; clustering de trajectoires pour inférence de rôles ; ROMA
       & Absence de lien avec modèles symboliques ; manque d'automatisation ; pas de cadre d'évaluation de l'explicabilité organisationnelle
       & Explicabilité surtout locale (niveau agent/action) ; peu de travaux inférant des rôles, missions ou objectifs collectifs
       & Inférer automatiquement rôles, missions et objectifs à partir de trajectoires ; proposer un cadre théorique pour l'explicabilité organisationnelle                                                    \\
      \hdashline

      \textbf{H-TRF}
       & Domain adaptation / Sim2Real ; Robust RL ; Online calibration (PILCO)
       & Approches partielles (transfert ou recalibrage, mais pas les deux) ; absence de cadre intégré de jumeau numérique adaptatif et de mise à jour conjointe des politiques
       & Domain adaptation et Sim2Real se concentrent sur le transfert initial sans recalibrage dynamique ; robust RL ne met pas à jour le modèle simulé ; calibration en ligne sans adaptation des politiques
       & Développer un framework unificateur de jumeau numérique couplant mise à jour continue du modèle simulé et adaptation conjointe des politiques multi-agents                                            \\
      \hline
    \end{tabularx}
  }
\end{table}
