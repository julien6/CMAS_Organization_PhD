%%%%%%%%%%%%%%%%%%%%%%%%%%%%%%%%%%%%%%%%%%%%%%%%%%%%%%%%%%%%%%%%%%%%%%%%

%%% LaTeX Template for AAMAS-2026 (based on sample-sigconf.tex)
%%% Prepared by the AAMAS-2026 Publication Chairs based on the version from AAMAS-2025. 

%%%%%%%%%%%%%%%%%%%%%%%%%%%%%%%%%%%%%%%%%%%%%%%%%%%%%%%%%%%%%%%%%%%%%%%%

%%% Start your document with the \documentclass command.


%%% == IMPORTANT ==
%%% Use the first variant below for the final paper (including author information).
%%% Use the second variant below to anonymize your submission (no author information shown).
%%% For further information on anonymity and double-blind reviewing, 
%%% please consult the call for paper information
%%% https://cyprusconferences.org/aamas2026/submission-instructions/

%%%% For anonymized submission, use this
% \documentclass[sigconf,anonymous]{aamas} 

%%%% For camera-ready, use this
\documentclass[sigconf]{aamas} 


%%% Load required packages here (note that many are included already).

\usepackage{balance} % for balancing columns on the final page

\usepackage{hyperref}

%%%% For camera-ready, use this
%\documentclass[sigconf]{aamas}

\usepackage{listings}
% \usepackage{xcolor}

\definecolor{codegreen}{rgb}{0,0.6,0}
\definecolor{codegray}{rgb}{0.5,0.5,0.5}
\definecolor{codepurple}{rgb}{0.58,0,0.82}
\definecolor{backcolour}{rgb}{0.95,0.95,0.92}
 
\lstdefinestyle{mystyle}{
    backgroundcolor=\color{backcolour},   
    commentstyle=\color{codegreen},
    keywordstyle=\color{magenta},
    numberstyle=\tiny\color{codegray},
    stringstyle=\color{codepurple},
    basicstyle=\footnotesize,
    breakatwhitespace=false,         
    breaklines=true,                 
    captionpos=b,                    
    keepspaces=true,                 
    numbers=left,                    
    numbersep=5pt,                  
    showspaces=false,                
    showstringspaces=false,
    showtabs=false,                  
    tabsize=2
}
 
\lstset{style=mystyle}

% --- Tickz
\usepackage{physics}
\usepackage{tikz}
\usepackage{amsmath}
\usepackage{mathdots}
% \usepackage{yhmath}
\usepackage{cancel}
\usepackage{color}
\usepackage{siunitx}
\usepackage{array}
\usepackage{multirow}
% \usepackage{amssymb}
\usepackage{gensymb}
\usepackage{tabularx}
\usepackage{extarrows}
\usepackage{booktabs}
\usetikzlibrary{fadings}
\usetikzlibrary{patterns}
\usetikzlibrary{shadows.blur}
\usetikzlibrary{shapes}

% ---------

\usepackage{csquotes}
% \usepackage{cite}
\newcommand{\probP}{\text{I\kern-0.15em P}}
\usepackage{etoolbox}
\patchcmd{\thebibliography}{\section*{\refname}}{}{}{}
% \usepackage{amsthm,amssymb,amsfonts}

\usepackage[T1]{fontenc}
\usepackage{graphicx}
\usepackage{color}
% \renewcommand\UrlFont{\color{blue}\rmfamily}

\usepackage[inline, shortlabels]{enumitem}
\usepackage{tabularx}
\usepackage{caption}
\usepackage{listings}
\usepackage{stfloats}
\usepackage{titlesec}
\usepackage{ragged2e}
% \usepackage[hyphens]{url}
\usepackage[linesnumbered,ruled,vlined]{algorithm2e}
\usepackage{float}
\usepackage[english]{babel}
\addto\extrasenglish{  
    \def\figureautorefname{Figure}
    \def\tableautorefname{Table}
    \def\algorithmautorefname{Algorithm}
    \def\sectionautorefname{Section}
    \def\subsectionautorefname{Subsection}
    \def\proofoutlineautorefname{Proof Outline}
}

%%%%%%%%%%%%%%%%%%%%%%%%%%%%%%%%%%%%%%%%%%%%%%%%%%%%%%%%%%%%%%%%%%%%%%%%

%%% AAMAS-2026 copyright block (do not change!)

\setcopyright{ifaamas}
\acmConference[AAMAS '26]{Proc.\@ of the 25th International Conference
on Autonomous Agents and Multiagent Systems (AAMAS 2026)}{May 25 -- 29, 2026}
{Paphos, Cyprus}{C.~Amato, L.~Dennis, V.~Mascardi, J.~Thangarajah (eds.)}
\copyrightyear{2026}
\acmYear{2026}
\acmDOI{}
\acmPrice{}
\acmISBN{}


%%%%%%%%%%%%%%%%%%%%%%%%%%%%%%%%%%%%%%%%%%%%%%%%%%%%%%%%%%%%%%%%%%%%%%%%

%%% == IMPORTANT ==
%%% Use this command to specify your submission number.
%%% In anonymous mode, it will be printed on the first page.

\acmSubmissionID{<<submission id>>}

%%% Use this command to specify the title of your paper.

\title[Automated Operational Assist via Organizational MARL]{Toward Automated Operational Assist in Satellite Fleet Management via Organizational MARL}

%%% Provide names, affiliations, and email addresses for all authors.

\author{Julien Soulé}
\affiliation{
  \institution{University of Luxembourg}
  \city{Luxembourg}
  \country{Luxembourg}}
\email{julien.soule@hotmail.fr}

%%% Use this environment to specify a short abstract for your paper.

\begin{abstract}
% context & problem
Managing satellite fleets requires coordinated decision-making under uncertainty, resource limitations, and strict operational constraints. This paper investigates automated operational assist for fleet management from a multi-agent perspective, where satellites are modeled as cooperative agents.
%
% contributions
We propose an experimental environment to study assistive decision-making for fleet management through simulation and optimization, as a proof-of-concept toward real-world operations. Building on this environment, we investigate an organizational approach to Multi-Agent Reinforcement Learning (MARL) to improve coordination while addressing safety, controllability, and explainability requirements. Organizational constraints are used to structure agent behavior, and trajectory-level analysis is used to interpret emergent coordination patterns for operators.
%
% results & discussion
Compared with handcrafted joint policies, initial results indicate that organizational constraints can improve long-term operational execution while strengthening safety compliance, and trajectory-level analysis provides actionable insights into agent behavior and decision-making. We discuss the implications of these findings for real-world satellite fleet management and outline future directions to progressively increase realism and reliability.
\end{abstract}

%%% Use this command to specify a few keywords describing your work.
%%% Keywords should be separated by commas.

\keywords{Satellite Fleet Management, Automated Operational Assist, Organizational Multi-Agent Reinforcement Learning, Multi-Agent Systems, Safety-Constrained Coordination, Explainable Coordination}

%%%%%%%%%%%%%%%%%%%%%%%%%%%%%%%%%%%%%%%%%%%%%%%%%%%%%%%%%%%%%%%%%%%%%%%%

%%% Include any author-defined commands here.
         
\newcommand{\BibTeX}{\rm B\kern-.05em{\sc i\kern-.025em b}\kern-.08em\TeX}

%%%%%%%%%%%%%%%%%%%%%%%%%%%%%%%%%%%%%%%%%%%%%%%%%%%%%%%%%%%%%%%%%%%%%%%%

\begin{document}

%%% The following commands remove the headers in your paper. For final 
%%% papers, these will be inserted during the pagination process.

\pagestyle{fancy}
\fancyhead{}

%%% The next command prints the information defined in the preamble.

\maketitle

%%%%%%%%%%%%%%%%%%%%%%%%%%%%%%%%%%%%%%%%%%%%%%%%%%%%%%%%%%%%%%%%%%%%%%%%

\section{Introduction}

Satellite fleet management is becoming increasingly complex due to constellation scale, dynamic mission requests, limited onboard resources, and operational uncertainty \cite{schetter2003satellite_autonomy,rocha2025iaeossp,govoni2026multi_sat_task_allocation}. In practice, operators must continuously arbitrate between conflicting objectives (coverage, responsiveness, energy usage, communication windows, and safety constraints) while preserving global coordination across multiple assets \cite{rocha2025iaeossp,govoni2026multi_sat_task_allocation}. This creates a strong need for automated operational assist systems that support decision-making without removing human oversight.

From a multi-agent perspective, each satellite can be modeled as an autonomous decision-maker interacting with teammates in a partially observable environment, which naturally motivates Dec-POMDP-based formulations \cite{Oliehoek2016}. Multi-Agent Reinforcement Learning (MARL) is therefore a natural candidate for learning coordinated policies \cite{lowe2017multi,rashid2018qmix,yu2021mappo}. However, raw MARL performance alone is insufficient for operational deployment: space operations require explicit control over behavior, traceable compliance with constraints, and understandable rationales for recommended actions \cite{garcia2015comprehensive,achiam2017cpo,alshiekh2018safe}.

This paper addresses that gap through three complementary components. First, we propose \textit{Orbital Resilient Benchmark for Interactive Task-aware Autonomous Learning} (ORBITAL), an experimental environment to study fleet-level coordination under realistic classes of constraints. We explicitly position ORBITAL as a proof-of-concept rather than a full digital twin: it is an intentionally coarse abstraction of the real problem, designed to enable controlled and reproducible experimentation. Second, we investigate an organizational approach to MARL, building on organizational modeling with $\mathcal{M}OISE^+$ \cite{Hubner2002,Hubner2007} and on MOISE+MARL \cite{soule2024moise_marl}, to inject organizational roles and missions into learning and execution. Third, we rely on trajectory-level analysis to characterize emergent coordination patterns and support operator-oriented explainability.

The general research problem can be summarized as follows: how can we design automated operational assist for satellite fleets that remains effective under uncertainty while being controllable, safety-aware, and interpretable for human operators? We decompose this problem into four sub-problems: (i) defining an experimental setting that captures essential operational constraints without sacrificing tractability, (ii) constraining MARL policies through organizational structures to improve control and compliance, (iii) evaluating learned behaviors beyond reward using coordination and safety indicators, and (iv) extracting trajectory-based explanations that are actionable at the mission level.

Our contributions are: (i) the ORBITAL benchmark as a proof-of-concept environment for studying automated operational assist in multi-satellite fleet management; (ii) an ORBITAL-oriented organizational approach for controlling MARL, based on MOISE+MARL \cite{soule2024moise_marl}, to improve policy control and safety compliance during coordinated decision-making; and (iii) a trajectory-level analysis method providing operator-oriented insights into emergent behaviors and organizational alignment.

Compared with handcrafted joint policies, our initial results indicate that organizational constraints can improve long-term operational execution while strengthening safety compliance. They also show that trajectory-level analysis helps interpret agent coordination and supports discussion of deployment-relevant trade-offs \cite{soule2024moise_marl}. These findings are encouraging but remain bounded by the current realism level of the environment.

The remainder of this paper is organized as follows. Section~\ref{sec:related} reviews related work on satellite operations assistance, MARL under constraints, and organizational approaches to coordination. Section~\ref{sec:background} presents the technical background on Dec-POMDP, MOISE+MARL, and trajectory-based analysis. Section~\ref{sec:method} introduces our ORBITAL proof-of-concept environment and the proposed methodology. Section~\ref{sec:experiments} details the experimental protocol, including baselines and ablations. Section~\ref{sec:results} reports and discusses the results. Section~\ref{sec:conclusion} concludes and outlines future directions toward higher-fidelity digital twins.

\section{Related Work}
\label{sec:related}

\begin{table*}[h!]
  \caption{Synthesis of related work against the target properties of this paper.}
  \label{tab:related-synthesis}
  \centering
  \scriptsize
  \setlength{\tabcolsep}{3pt}
  \begin{tabularx}{\textwidth}{p{2.7cm} p{4.7cm} >{\centering\arraybackslash}p{1.7cm} >{\centering\arraybackslash}p{1.9cm} >{\centering\arraybackslash}p{1.8cm} >{\centering\arraybackslash}p{2.0cm}}
    \toprule
    Category                                 & Representative works                                                              & Operational realism for satellite fleets & Explicit safety/control mechanisms & Explainability at operator level & Reproducible benchmark orientation \\
    \midrule
    Satellite scheduling and allocation (OR) & \cite{ferrari2025ssp_survey,rocha2025iaeossp,govoni2026multi_sat_task_allocation} & High                                     & Medium--High                       & Low                              & Medium                             \\
    Space digital twin works                 & \cite{liu2024space_digital_twin_review,colagrossi2026spacecraft_dt}               & Medium--High                             & Medium                             & Low--Medium                      & Low--Medium                        \\
    Generic MARL benchmarks/software         & \cite{samvelyan2019smac,terry2020pettingzoo,kwiatkowski2024,hu2021marlib}         & Low                                      & Low                                & Low                              & High                               \\
    Safe RL methods                          & \cite{garcia2015comprehensive,achiam2017cpo,alshiekh2018safe}                     & Low                                      & High                               & Low                              & Medium                             \\
    Explainable RL methods                   & \cite{puiutta2020xrlsurvey,sequeira2020interestingness_xrl}                       & Low                                      & Low                                & High                             & Medium                             \\
    Organizational MAS / Organizational MARL & \cite{Hubner2002,Hubner2007,soule2024moise_marl}                                  & Medium (domain-agnostic)                 & High                               & Medium--High                     & Medium                             \\
    \bottomrule
  \end{tabularx}
\end{table*}

\paragraph{Satellite Operations and Decision Support}
Satellite planning and scheduling have been extensively studied in operations research, including Earth-observation mission planning, agility constraints, and integrated allocation/scheduling formulations \cite{ferrari2025ssp_survey,rocha2025iaeossp,govoni2026multi_sat_task_allocation}. These works provide strong optimization baselines and realistic constraint modeling. However, they usually focus on centrally optimized plans and are less oriented toward adaptive, decentralized, and continuously learning operational assist loops.

In parallel, agent-based autonomy has long been investigated for spacecraft constellations \cite{schetter2003satellite_autonomy}. This line of work supports distributed decision-making, but often without a unified benchmark that jointly emphasizes modern MARL evaluation, explicit organizational control, and operator-oriented explainability.

\paragraph{Digital Twin and Simulation for Space Systems}
Recent work on digital twins for space systems highlights their potential for system validation, software testing, and predictive decision support \cite{liu2024space_digital_twin_review,colagrossi2026spacecraft_dt}. Nevertheless, many current approaches target specific subsystems or engineering workflows. For fleet-level autonomy research, there remains a need for reproducible environments that are simple enough for controlled MARL experimentation while progressively extensible toward higher realism.

\paragraph{MARL, Safety, and Explainability}
Cooperative MARL methods have significantly progressed, from actor-critic and value factorization methods to practical training recipes \cite{lowe2017multi,foerster2018counterfactual,rashid2018qmix,yu2021mappo}. Benchmarks and software ecosystems such as SMAC, PettingZoo, Gymnasium, and MARLlib improved reproducibility and comparison \cite{samvelyan2019smac,terry2020pettingzoo,kwiatkowski2024,hu2021marlib}. However, these environments do not directly represent satellite operational constraints and mission-specific safety priorities.

Safety-aware RL has introduced constrained optimization and shielding methods \cite{garcia2015comprehensive,achiam2017cpo,alshiekh2018safe}, while explainable RL research has proposed post-hoc and introspective analysis tools \cite{puiutta2020xrlsurvey,sequeira2020interestingness_xrl}. Yet, these strands are often developed separately from organizational modeling and from domain-specific operational assist requirements.

\paragraph{Organizational Modeling and Organizational MARL}
Organizational MAS modeling (roles, groups, missions, deontic constraints) is well established through AGR and $\mathcal{M}OISE^+$ \cite{ferber2003,Hubner2002,Hubner2007}. Building on this foundation, MOISE+MARL integrates organizational constraints into MARL and uses trajectory-based post-analysis to assess organizational alignment \cite{soule2024moise_marl}. This direction is promising for controllability and explainability, but it has not yet been fully studied in the context of satellite fleet operational assist with an explicit benchmarking perspective.

Table~\ref{tab:related-synthesis} highlights a structural gap: no single line of work jointly satisfies high operational relevance for satellite fleets, adaptive multi-agent learning, explicit organizational control, and operator-oriented explainability. This motivates our contributions: (i) an ORBITAL proof-of-concept benchmark tailored to satellite fleet assist, (ii) organizationally-constrained MARL for safety-aware controllability, and (iii) trajectory-level analysis for actionable explainability.

\section{Background}
\label{sec:background}

\subsection{Cooperative Dec-POMDP for Fleet Coordination}
We formalize fleet-level decision-making as a Decentralized Partially Observable Markov Decision Process (Dec-POMDP) \cite{Oliehoek2016,Beynier2013}. This framework is suitable for ORBITAL-like settings where agents act under local observability, decentralized execution, and team-level objectives.

A Dec-POMDP instance is:
\[
  d = \langle S,\{A_i\}_{i=1}^{n},T,R,\{\Omega_i\}_{i=1}^{n},O,\gamma \rangle,
\]
where $S$ is the latent state space, $A_i$ and $\Omega_i$ are local action and observation spaces for agent $i$, $T$ is the transition kernel, $O$ the observation kernel, $R$ a cooperative reward function, and $\gamma \in [0,1]$ the discount factor.

Let $\pi_i(a_i \mid \tau_i)$ denote the local policy of agent $i$ over local action-observation history $\tau_i$. The joint policy is $\pi = (\pi_1,\ldots,\pi_n)$ and optimizes expected return:
\[
  J(\pi)=\mathbb{E}_{\pi,T,O}\left[\sum_{t=0}^{H-1} \gamma^t r_t \right].
\]
In ORBITAL, this objective already embeds operational trade-offs (service, energy, communication, resilience), but by itself it does not guarantee role specialization, safety-compliant behavior, or interpretability.

\paragraph{Notation used in this paper}
We use standard Dec-POMDP notation: $n$ agents, horizon $H$, local histories $\tau_i$, local and joint policies $\pi_i,\pi$, and trajectory prefix $h_t$. Organizational guides are denoted $rag, rrg,$ and $grg$, while organizational-fit indicators are $\mathrm{SF}$, $\mathrm{FF}$, and $\mathrm{OF}$.

\subsection{Organizational Modeling with $\mathcal{M}OISE^+$}
To encode controllable coordination semantics, we rely on $\mathcal{M}OISE^+$ \cite{Hubner2002,Hubner2007}. Its key contribution is to separate organization into complementary layers:
\begin{enumerate}
  \item \textbf{Structural specifications}: roles and role relations (e.g., specialization or inheritance-like structures).
  \item \textbf{Functional specifications}: goals and missions that structure collective progress.
  \item \textbf{Deontic specifications}: permissions and obligations linking roles to missions under conditions.
\end{enumerate}
This layered representation matters for operational assist because it separates \emph{what} should be done (functional), \emph{by whom} (structural), and \emph{under which normative constraints} (deontic), instead of collapsing everything into scalar reward coefficients.

\subsection{MOISE+MARL as Organizational Control Layer}
MOISE+MARL \cite{soule2024moise_marl} injects organizational knowledge into MARL while keeping standard MARL backbones usable. The integration relies on three families of guides:
\begin{enumerate}
  \item \textbf{Role-action guides} ($rag$): constrain or prioritize actions based on role and trajectory context.
  \item \textbf{Role-reward guides} ($rrg$): penalize role-inconsistent decisions.
  \item \textbf{Goal-reward guides} ($grg$): reward mission-consistent progress patterns.
\end{enumerate}

For agent $i$ at time $t$, the effective decision/reward mechanism can be summarized as:
\[
  a_{i,t} \sim \pi_i(\cdot \mid \tau_{i,t}) \text{ over } \tilde{A}_{i,t}=rag(h_{i,t},o_{i,t}),
\]
\[
  \tilde{r}_t = r_t + \sum_{m \in \mathcal{M}_{i,t}} grg_m(h_t) + rrg(h_{i,t},o_{i,t},a_{i,t}).
\]
The result is a hybrid control paradigm: learning remains data-driven, but search is guided by explicit organizational priors. This can reduce unsafe exploration regions and improve policy controllability, particularly in partially observable cooperative settings \cite{soule2024moise_marl,Albrecht2024}.

\subsection{Trajectory-Based Organizational Analysis (TEMM)}
The TEMM perspective \cite{soule2024moise_marl} addresses an important gap: even if policies are trained with organizational constraints, one still needs post-hoc evidence that learned behavior actually aligns with intended organization.

TEMM operates on multi-episode trajectories and infers implicit organizational regularities. At a high level:
\begin{enumerate}
  \item infer structural regularities (role-like behavior clusters);
  \item infer functional regularities (goal/mission progression patterns);
  \item compare inferred and intended structures to quantify alignment.
\end{enumerate}

This process yields interpretable artifacts (e.g., role prototypes, progression motifs) and a quantitative organizational-fit signal. We use the following decomposition:
\[
  \mathrm{OF}=\alpha \cdot \mathrm{SF} + (1-\alpha)\cdot \mathrm{FF},
\]
where $\mathrm{SF}$ captures structural role consistency and $\mathrm{FF}$ captures functional mission/goal consistency.

\subsection{Why This Background Matters for ORBITAL}
In ORBITAL-like environments, pure return maximization can produce brittle policies that exploit local shortcuts while degrading long-term viability (energy collapse, communication fragmentation, or cyber-sensitive behavior). The combination of Dec-POMDP formalization, organizational constraints, and trajectory-level organizational analysis provides the conceptual foundation needed to evaluate policies not only by performance but also by controllability, safety compliance, and explainability.

\section{Method}
\label{sec:method}

\subsection{ORBITAL Environment Design Rationale}
ORBITAL (\textit{Orbital Resilient Benchmark for Interactive Task-aware Autonomous Learning}) is designed as a benchmark for \emph{operational assist}, not as a high-fidelity orbital propagator. This is a deliberate methodological choice. The objective is to keep the environment simple enough for controlled MARL experimentation while preserving the interaction structure that makes fleet management difficult in practice.

\paragraph{Design requirements}
The benchmark was designed around four requirements derived from our research problem:
\begin{enumerate}
  \item represent \textbf{mission pressure} (dynamic and priority-sensitive demand),
  \item represent \textbf{resource pressure} (energy-limited long-horizon decisions),
  \item represent \textbf{communication uncertainty} (time-varying connectivity),
  \item represent \textbf{cyber uncertainty} (degraded sensing/acting/relaying).
\end{enumerate}
This is why ORBITAL combines non-stationary tasks, finite energy with heterogeneous action costs, stochastic communication degradation, and stochastic compromise events in the same episode loop.

\paragraph{Operational state and coupling}
At time $t$, the latent state includes satellite positions, energy levels, buffered data, compromise timers, active task set, and communication adjacency matrix. Satellites are coupled through shared tasks, shared communication paths to ground, and shared team-level mission reward. This coupling produces the coordination tension we need to study: individual actions affect both local utility and fleet-level viability.

\paragraph{Observation and action modeling}
ORBITAL uses fixed-size local observations (14-dimensional vectors) and a compact discrete action space of size 6 (\texttt{Observe}, \texttt{Relay}, \texttt{Move}, \texttt{LowPower}, \texttt{CyberScan}, \texttt{Idle}). The fixed vector format supports reproducible MARL pipelines and avoids benchmark bias toward a specific architecture. The action set was chosen to reflect the minimum operational primitives needed for fleet assist: mission execution, data return, mobility, energy management, security response, and fallback behavior.

These interface choices are intentionally minimal: fixed-size vectors improve reproducibility across MARL families, the 6-action set captures core operational primitives, and both AEC and Parallel APIs are available with consistent semantics.

\paragraph{Task-delivery decoupling}
A critical modeling decision is to separate \emph{task servicing} from \emph{value delivery}. \texttt{Observe} converts local task opportunities into buffered data, but mission value is maximized only when data is later relayed through available communication opportunities toward ground. This separation forces policies to balance sensing throughput, topology management, and delivery timing rather than greedily optimizing local sensing only.

\paragraph{Reward design for operational trade-offs}
Default reward mode is shared team reward with positive terms for mission productivity and penalties for resilience degradation:
\[
  \begin{aligned}
    r_t ={} & w_{\text{task}} \, c_{\text{task}}
    + w_{\text{delivery}} \, c_{\text{delivery}}         \\
            & - w_{\text{energy}} \, c_{\text{energy}}
    - w_{\text{isolation}} \, c_{\text{isolation}}       \\
            & - w_{\text{failure}} \, c_{\text{failure}}
    - w_{\text{cyber}} \, c_{\text{cyber}}.
  \end{aligned}
\]
This reward structure is intentionally non-myopic: maximizing return requires balancing service, survivability, connectivity, and cyber resilience over the whole horizon. ORBITAL also provides a local reward mode for controlled comparisons.

\paragraph{Episode termination and realism scope}
Episodes stop at horizon or mission collapse (e.g., no alive satellites or critically low survivability). This choice makes unsafe policies self-limiting in long runs. Finally, ORBITAL is explicitly positioned as a proof-of-concept abstraction: it does not claim orbital-physics realism, but it does capture the multi-factor decision coupling needed to study automated operational assist.

\subsection{ORBITAL-Oriented MOISE+MARL Framework}
This contribution adapts MOISE+MARL \cite{soule2024moise_marl} to ORBITAL semantics. The objective is to turn unconstrained policy search into \emph{organizationally guided} learning where role specialization, mission progression, and safety-oriented control are explicit.

\paragraph{Organizational mapping}
We map ORBITAL operational intents to MOISE+-style specifications:
\begin{enumerate}
  \item \textbf{roles}: behavioral specialization templates,
  \item \textbf{goals}: trajectory-level objectives with measurable progress,
  \item \textbf{missions}: coherent groups of goals in operational phases,
  \item \textbf{deontic rules}: permissions/obligations linking roles to missions.
\end{enumerate}
This mapping gives a declarative control layer that can be inspected and revised independently from the MARL backbone.

Table~\ref{tab:orbital-org-spec} summarizes the role profile used in this paper.
\begin{table}[h!]
  \caption{ORBITAL-oriented organizational specifications (high-level view).}
  \label{tab:orbital-org-spec}
  \centering
  \scriptsize
  \begin{tabularx}{\columnwidth}{p{1.8cm} p{2.5cm} X}
    \toprule
    Role         & Main mission focus      & Typical obligations/permissions                                                                                                                           \\
    \midrule
    Observer     & Prioritized acquisition & Obliged to observe feasible high-priority tasks; permitted to defer low-priority opportunities under strong resource pressure                             \\
    Relay        & Delivery continuity     & Obliged to relay when buffered data and communication opportunities are favorable; permitted to adapt relay cadence under congestion                      \\
    Safety guard & Fleet survivability     & Obliged to prioritize safe actions under elevated risk (energy depletion, isolation, compromise); permitted to override non-critical productivity actions \\
    \bottomrule
  \end{tabularx}
\end{table}

\paragraph{Constraint guides adapted to ORBITAL}
As in the original MOISE+MARL, we use role-action guidance ($rag$), role-reward penalties ($rrg$), and goal-reward guides ($grg$). For agent $i$:
\[
  \tilde{A}_{i,t}=rag(h_{i,t},o_{i,t}),
\]
\[
  \tilde{r}_t = r_t + \sum_{m \in \mathcal{M}_{i,t}} grg_m(h_t) + rrg(h_{i,t},o_{i,t},a_{i,t}).
\]
The ORBITAL adaptation ties guide logic to operational indicators available in trajectories and observations: local task opportunity, buffered data pressure, communication degree, energy margin, compromise status, and recent delivery events.

\paragraph{Mission semantics}
We structure behavior around three mission families:
\begin{enumerate}
  \item \textbf{Acquisition mission}: increase task servicing under feasibility constraints.
  \item \textbf{Delivery mission}: convert buffered data into delivered value reliably.
  \item \textbf{Resilience mission}: preserve fleet viability under resource/cyber stress.
\end{enumerate}
Role-specific permissions and obligations switch emphasis between these missions according to context. This produces a controllable trade-off between productivity and safety rather than relying on implicit reward balancing only.

\paragraph{Hard and soft control regimes}
The adaptation supports two control regimes. In hard regimes, $rag$ can enforce action masking so that non-authorized actions are unavailable. In soft regimes, actions remain selectable but violations are discouraged through $rrg$. This hardness axis is useful to tune exploration freedom versus organizational compliance.

\paragraph{ORBITAL-adapted TEMM}
We integrate an ORBITAL-specific TEMM process \cite{soule2024moise_marl} as post-training validation and explainability. Trajectories include action traces, reward components, energy evolution, communication context, task-service and delivery events, and cyber-state transitions. TEMM is applied in three stages:
\begin{enumerate}
  \item infer structural regularities corresponding to implicit role behavior;
  \item infer functional progression patterns corresponding to missions/goals;
  \item compare inferred and intended specifications to quantify alignment.
\end{enumerate}
We report organizational fit as:
\[
  \mathrm{OF} = \alpha \cdot \mathrm{SF} + (1-\alpha)\cdot \mathrm{FF},
\]
where $\mathrm{SF}$ is structural fit and $\mathrm{FF}$ is functional fit.

\paragraph{Contribution perspective}
This section defines a coupled contribution: ORBITAL provides a mission-resilience benchmark tailored to fleet assist, and ORBITAL-oriented MOISE+MARL provides explicit behavioral control plus trajectory-level interpretability through TEMM. Together they make policy quality assessable beyond return, including safety compliance and organizational coherence.

\section{Experimental Setup}
\label{sec:experiments}

\subsection{Experimental Goals and Design}
Our experimental protocol follows a classical empirical-science logic: define explicit research questions, compare against strong baselines, isolate mechanisms through ablations, and report uncertainty with multi-seed statistics.

We target four experimental questions:
\begin{enumerate}
  \item \textbf{Q1 (Performance):} Does ORBITAL-oriented MOISE+MARL improve long-horizon mission execution compared with unconstrained learning and handcrafted coordination?
  \item \textbf{Q2 (Control/Safety):} Does organizational guidance reduce unsafe or mission-degrading behavior under resource, communication, and cyber stress?
  \item \textbf{Q3 (Robustness):} Does the approach remain effective under intensified non-stationarity (task dynamics, link drops, compromise rate)?
  \item \textbf{Q4 (Explainability):} Do TEMM-based organizational indicators provide coherent and actionable post-hoc analysis of learned trajectories?
\end{enumerate}

To answer these questions, each experiment combines:
\begin{enumerate}
  \item one environment configuration (nominal or stressed),
  \item one learning/control condition (baseline or proposed method),
  \item one MARL backbone,
  \item multiple independent random seeds.
\end{enumerate}
All protocol parameters and seeds are fixed before result aggregation.

\subsection{Hardware and Software Configuration}

\paragraph{Hardware profile}
Experiments are run on a Linux workstation-class setup. Table~\ref{tab:exp-hardware} reports the reference profile used in this study.

\begin{table}[h!]
  \caption{Reference hardware configuration.}
  \label{tab:exp-hardware}
  \centering
  \scriptsize
  \begin{tabularx}{\columnwidth}{p{2.1cm} X}
    \toprule
    Component & Configuration                                          \\
    \midrule
    OS        & Ubuntu 22.04 LTS (Linux kernel 6.8 series)             \\
    CPU       & Intel Core i7-4790, 4 cores / 8 threads, up to 4.0 GHz \\
    RAM       & 16 GB system memory                                    \\
    GPU       & None (CPU-based training/evaluation in this setup)     \\
    \bottomrule
  \end{tabularx}
\end{table}

\paragraph{Software stack}
The environment is implemented in Python and exposed through PettingZoo-compatible APIs. Table~\ref{tab:exp-software} summarizes the software stack used to ensure reproducibility.

\begin{table}[h!]
  \caption{Software configuration and role in the pipeline.}
  \label{tab:exp-software}
  \centering
  \scriptsize
  \begin{tabularx}{\columnwidth}{p{2.2cm} X}
    \toprule
    Component                                                  & Role in experiments                                            \\
    \midrule
    Python 3.10                                                & Training and evaluation runtime                                \\
    PettingZoo 1.25.0 \cite{terry2020pettingzoo}               & Multi-agent API (AEC and Parallel interfaces)                  \\
    Gymnasium 1.2.3 \cite{kwiatkowski2024}                     & Standard RL spaces/wrapping compatibility                      \\
    NumPy 2.2.6                                                & Numerical operations and logging                               \\
    ORBITAL codebase                                           & Environment dynamics, reward components, rendering/debugging   \\
    MOISE+MARL implementation reuse \cite{soule2024moise_marl} & Organizational guides, role/mission constraints, TEMM pipeline \\
    \bottomrule
  \end{tabularx}
\end{table}

In practice, the ORBITAL-oriented framework reuses the core implementation principles of MOISE+MARL \cite{soule2024moise_marl} (organizational specification layer, guide-based control, trajectory analysis), while specializing role/mission logic and TEMM features for ORBITAL signals.

\subsection{Baselines and Comparison Conditions}

\paragraph{Handcrafted and non-learning baselines}
To represent operational heuristics, we include:
\begin{enumerate}
  \item \textbf{RB-Rule (priority-first):} rule-based action selection prioritizing local high-priority tasks, then relay, with simple low-energy fallback.
  \item \textbf{RB-Relay-heavy:} heuristic emphasizing delivery continuity (relay when possible, observe otherwise), with weak safety adaptation.
\end{enumerate}
These baselines provide interpretable references and approximate operational scripts.

\paragraph{Learning baselines}
We compare the proposed organizational framework to unconstrained MARL and partially constrained variants:
\begin{enumerate}
  \item \textbf{LB-Unconstrained:} same MARL backbone, ORBITAL reward only, no organizational guides.
  \item \textbf{LB-RewardOnly:} unconstrained action space, additional reward shaping but no role-action masking.
  \item \textbf{LB-ActionOnly:} role-action guidance active, no mission reward guides.
  \item \textbf{Proposed (ORBITAL-MOISE+MARL):} role-action + role-penalty + mission guides with deontic assignments.
\end{enumerate}
This decomposition isolates where gains come from: action control, reward structure, or full organizational coupling.

\paragraph{Backbone algorithms}
To reduce algorithm-specific bias, each condition is evaluated with representative cooperative MARL families:
\begin{enumerate}
  \item actor-critic / policy-gradient style methods (e.g., MAPPO \cite{yu2021mappo}),
  \item value-factorization methods (e.g., QMIX \cite{rashid2018qmix}),
  \item multi-agent actor-critic references (e.g., MADDPG/COMA style \cite{lowe2017multi,foerster2018counterfactual}).
\end{enumerate}
The primary comparisons use matched compute budgets per algorithm family.

\subsection{Ablation Plan}
To validate causal contributions of the proposed framework, we define the ablations in Table~\ref{tab:exp-ablations}.

\begin{table}[h!]
  \caption{Ablation settings for ORBITAL-oriented MOISE+MARL.}
  \label{tab:exp-ablations}
  \centering
  \scriptsize
  \begin{tabularx}{\columnwidth}{p{1.5cm} X}
    \toprule
    ID & Ablation description                                                         \\
    \midrule
    A0 & Full framework (role-action + role-penalty + mission guides + TEMM analysis) \\
    A1 & Remove role-action guide ($rag$ off): no action-space organizational control \\
    A2 & Remove role-penalty guide ($rrg$ off): no explicit role-violation penalty    \\
    A3 & Remove mission guides ($grg$ off): no mission-level shaping                  \\
    A4 & Soft-only control: no action masking, penalties/rewards only                 \\
    A5 & Hard-only control: action masking active, no additional role penalty         \\
    A6 & TEMM feature reduction: remove cyber/context features in trajectory analysis \\
    \bottomrule
  \end{tabularx}
\end{table}

Additionally, we stress-test each setting under controlled perturbation axes:
\begin{enumerate}
  \item increased task non-stationarity (spawn/priority volatility),
  \item increased communication degradation ($p_{\text{link\_drop}}$),
  \item increased compromise intensity (adversarial rate and duration),
  \item reduced energy budgets.
\end{enumerate}

\subsection{Evaluation Metrics}
We evaluate at three levels: mission outcome, operational safety/control, and organizational interpretability.

\paragraph{Mission performance metrics}
\begin{enumerate}
  \item \textbf{Cumulative return}: episode return under the configured reward mode.
  \item \textbf{Task service volume}: total serviced task priority mass.
  \item \textbf{Delivery volume}: total delivered data to ground.
  \item \textbf{Mission completion rate}: fraction of episodes ending without mission collapse.
\end{enumerate}

\paragraph{Safety and resilience metrics}
\begin{enumerate}
  \item \textbf{Energy stress index}: average low-energy occupancy over agents and time.
  \item \textbf{Isolation ratio}: proportion of alive agents with zero communication degree.
  \item \textbf{Failure count}: average number of depleted satellites per episode.
  \item \textbf{Cyber impact score}: aggregate penalty/events linked to compromise effects.
\end{enumerate}

\paragraph{Control and explainability metrics}
\begin{enumerate}
  \item \textbf{Constraint violation rate}: frequency of role-inconsistent actions.
  \item \textbf{Structural fit (SF)} and \textbf{Functional fit (FF)} from ORBITAL-TEMM.
  \item \textbf{Organizational fit (OF)} using $\mathrm{OF}=\alpha \cdot \mathrm{SF} + (1-\alpha)\cdot \mathrm{FF}$.
  \item \textbf{Consistency score}: agreement between intended and inferred role/mission assignments.
\end{enumerate}

For ranking and discussion, we report both metric-specific results and a normalized composite score:
\[
  \mathrm{Score}_{\mathrm{global}} = \sum_{k} \lambda_k \, \widehat{m}_k,
\]
where $\widehat{m}_k$ are direction-corrected normalized metrics and $\lambda_k$ are predeclared weights.

\subsection{Training and Evaluation Protocol}

\paragraph{Data split and seeds}
For each condition, we use disjoint random seeds for:
\begin{enumerate}
  \item training runs,
  \item model/setting selection,
  \item final evaluation.
\end{enumerate}
Unless otherwise specified, each reported value is aggregated over at least $10$ independent seeds.

\paragraph{Budget alignment}
To ensure fair comparisons:
\begin{enumerate}
  \item all learning conditions share identical episode horizon and maximum environment steps,
  \item each algorithm family receives the same optimization budget per condition,
  \item early stopping is disabled unless numerically unstable behavior is detected.
\end{enumerate}

\paragraph{Hyperparameter handling}
Backbone hyperparameters are selected from predefined ranges with a fixed search protocol (same budget across compared conditions). Organizational hyperparameters (e.g., guide hardness, role-penalty scale, mission-reward scale, TEMM clustering parameters) are tuned on validation seeds only, then frozen for final testing.

\paragraph{Statistical reporting}
For all principal metrics, we report mean, standard deviation, and confidence intervals across seeds. Pairwise condition comparisons are performed with two-sided significance tests appropriate for finite-sample, seed-based evaluations; effect sizes are reported alongside $p$-values to avoid significance-only conclusions.

\paragraph{Failure analysis and logging}
Each run logs per-step reward components, mission events, safety events, and organizational events. We perform failure-mode analysis on runs with extreme outcomes (best/worst quantiles) to identify whether degradations stem from exploration instability, communication fragmentation, energy collapse, or cyber sensitivity.

\paragraph{Reproducibility policy}
To maximize reproducibility, we fix:
\begin{enumerate}
  \item environment configuration files (ORBITAL parameters),
  \item random seeds and seed allocation strategy,
  \item software versions and execution scripts,
  \item metric definitions and post-processing code.
\end{enumerate}
This protocol enables exact reruns of baseline, ablation, and stress-test comparisons under the same experimental contract.

\section{Results and Discussion}
\label{sec:results}

\subsection{Main Quantitative Results}
We first compare the three organizational-control regimes introduced in Section~\ref{sec:method}: strong constraints, weak constraints, and medium constraints. Table~\ref{tab:results-control-regimes} summarizes aggregate results over matched evaluation budgets.

\begin{table*}[t]
  \caption{Illustrative aggregate results by constraint regime (mean $\pm$ std over seeds). Higher is better for all metrics except convergence episode, violation rate, and energy stress.}
  \label{tab:results-control-regimes}
  \centering
  \scriptsize
  \begin{tabularx}{\textwidth}{p{2.4cm} c c c c c c c c}
    \toprule
    Regime             & Final return              & Return AUC (early) & Convergence episode $\downarrow$ & Robustness score         & Constraint violation $\downarrow$ & OF score                 & Mission success rate     & Energy stress $\downarrow$ \\
    \midrule
    Strong constraints & $352.6 \pm 18.4$          & $205.3 \pm 9.8$    & $118 \pm 14$                     & $0.71 \pm 0.05$          & $1.8\% \pm 0.9$                   & $0.90 \pm 0.03$          & $0.86 \pm 0.04$          & $0.37 \pm 0.06$            \\
    Medium constraints & $\mathbf{379.8 \pm 16.1}$ & $192.7 \pm 8.2$    & $146 \pm 17$                     & $\mathbf{0.84 \pm 0.04}$ & $3.9\% \pm 1.2$                   & $\mathbf{0.92 \pm 0.02}$ & $\mathbf{0.91 \pm 0.03}$ & $\mathbf{0.31 \pm 0.05}$   \\
    Weak constraints   & $334.9 \pm 21.7$          & $143.5 \pm 11.6$   & $213 \pm 22$                     & $0.87 \pm 0.04$          & $8.7\% \pm 1.8$                   & $0.81 \pm 0.05$          & $0.83 \pm 0.05$          & $0.35 \pm 0.07$            \\
    \bottomrule
  \end{tabularx}
\end{table*}

Three trends are directly visible in Table~\ref{tab:results-control-regimes}. Strong constraints achieve the best early acceleration, with early return AUC at $205.3 \pm 9.8$ and the fastest convergence at $118 \pm 14$ episodes, while maintaining low violation at $1.8\% \pm 0.9$. This acceleration comes with reduced robustness ($0.71 \pm 0.05$) and lower final return than the medium regime. Weak constraints show the opposite profile: slower learning (AUC $143.5 \pm 11.6$, convergence $213 \pm 22$), higher violation ($8.7\% \pm 1.8$), but stronger robustness ($0.87 \pm 0.04$). Medium constraints provide the best overall balance, with the highest final return ($379.8 \pm 16.1$), the best mission success rate ($0.91 \pm 0.03$), the highest organizational fit ($0.92 \pm 0.02$), and substantially improved convergence over weak control ($146 \pm 17$ vs. $213 \pm 22$). This quantitatively supports the expected effect of organizational guidance: constrain enough to compress trivial search dimensions, but not so much that robustness and emergent strategy diversity collapse.

\subsection{Baselines Against Handcrafted and Learning Conditions}
Table~\ref{tab:results-baselines} compares the proposed method against handcrafted and learning baselines. As expected, handcrafted policies do not rely on organizational constraints and therefore violation metrics are marked \texttt{n/a}.

\begin{table*}[t]
  \caption{Illustrative baseline comparison. Handcrafted baselines report \texttt{n/a} for organizational violation metrics.}
  \label{tab:results-baselines}
  \centering
  \scriptsize
  \begin{tabularx}{\textwidth}{p{2.6cm} c c c c c c c}
    \toprule
    Condition                     & Final return              & Convergence episode $\downarrow$ & Robustness score         & Violation rate $\downarrow$ & OF score                 & Delivery volume          & Mission success rate     \\
    \midrule
    RB-Rule (priority-first)      & $241.3 \pm 12.9$          & \texttt{n/a}                     & $0.63 \pm 0.06$          & \texttt{n/a}                & $0.49 \pm 0.07$          & $112.4 \pm 9.8$          & $0.64 \pm 0.08$          \\
    RB-Relay-heavy                & $228.1 \pm 14.7$          & \texttt{n/a}                     & $0.67 \pm 0.07$          & \texttt{n/a}                & $0.45 \pm 0.08$          & $121.7 \pm 10.3$         & $0.61 \pm 0.07$          \\
    LB-Unconstrained              & $319.7 \pm 20.5$          & $229 \pm 24$                     & $0.82 \pm 0.05$          & $10.4\% \pm 2.1$            & $0.74 \pm 0.05$          & $151.8 \pm 11.7$         & $0.79 \pm 0.05$          \\
    LB-RewardOnly                 & $341.5 \pm 19.2$          & $188 \pm 20$                     & $0.79 \pm 0.05$          & $7.1\% \pm 1.9$             & $0.82 \pm 0.04$          & $162.3 \pm 10.8$         & $0.84 \pm 0.04$          \\
    LB-ActionOnly                 & $348.0 \pm 17.9$          & $161 \pm 18$                     & $0.76 \pm 0.06$          & $4.6\% \pm 1.5$             & $0.85 \pm 0.04$          & $167.9 \pm 10.1$         & $0.86 \pm 0.04$          \\
    Proposed (medium constraints) & $\mathbf{379.8 \pm 16.1}$ & $\mathbf{146 \pm 17}$            & $\mathbf{0.84 \pm 0.04}$ & $3.9\% \pm 1.2$             & $\mathbf{0.92 \pm 0.02}$ & $\mathbf{182.5 \pm 9.4}$ & $\mathbf{0.91 \pm 0.03}$ \\
    \bottomrule
  \end{tabularx}
\end{table*}

Two implications stand out. First, handcrafted policies remain interpretable but plateau at significantly lower performance and resilience levels. Second, partial organizational variants improve either control or speed, but the full framework is required to simultaneously optimize mission value, convergence, and robustness.

\subsection{Ablation Results}
Table~\ref{tab:results-ablations} quantifies the contribution of each framework component and shows that every ablation degrades at least one critical axis. Removing $rag$ (A1) produces a marked drop in return ($-8.6\%$), robustness ($-6.2\%$), and organizational fit ($-0.07$), with violation increasing by $+2.7$ points. Removing $grg$ (A3) is the most harmful for mission performance, with return decreasing by $-10.4\%$, confirming the importance of mission-level shaping for long-horizon objective progress. Removing $rrg$ (A2) causes the largest violation increase among guide ablations ($+3.1$ points) and the strongest OF degradation ($-0.09$), highlighting the role of explicit role-consistency penalties. By contrast, A6 (reduced TEMM features) leaves training metrics almost unchanged ($-0.2\%$ return, $-0.1\%$ robustness) but substantially degrades OF interpretability quality ($-0.11$), which is consistent with TEMM being primarily an analysis component rather than a direct training controller.

\begin{table}[t]
  \caption{Illustrative ablation outcomes relative to full method.}
  \label{tab:results-ablations}
  \centering
  \scriptsize
  \begin{tabularx}{\columnwidth}{p{1.2cm} c c c c}
    \toprule
    Ablation                   & $\Delta$ Return & $\Delta$ Robustness & $\Delta$ Violation & $\Delta$ OF \\
    \midrule
    A1 (no $rag$)              & $-8.6\%$        & $-6.2\%$            & $+2.7$ pts         & $-0.07$     \\
    A2 (no $rrg$)              & $-3.9\%$        & $-4.8\%$            & $+3.1$ pts         & $-0.09$     \\
    A3 (no $grg$)              & $-10.4\%$       & $-2.0\%$            & $+0.8$ pts         & $-0.06$     \\
    A4 (soft-only)             & $-4.7\%$        & $+1.1\%$            & $+1.9$ pts         & $-0.05$     \\
    A5 (hard-only)             & $-5.3\%$        & $-7.4\%$            & $-1.4$ pts         & $-0.04$     \\
    A6 (reduced TEMM features) & $-0.2\%$        & $-0.1\%$            & $+0.0$ pts         & $-0.11$     \\
    \bottomrule
  \end{tabularx}
\end{table}

\subsection{Convergence and Robustness Trade-off}
The central behavioral result is the non-monotonic relation between control hardness and end-task quality. Excessively hard organizational constraints can force rapid convergence to locally stable but brittle policies. Excessively weak constraints preserve flexibility but underutilize organizational priors, producing slow policy shaping and larger violation windows during training.

Medium constraints produce the best compromise and behave like a curriculum-like control mechanism. They encode obvious safety and coordination priors early enough to accelerate learning, but still leave sufficient policy freedom for non-trivial strategy discovery, which prevents over-regularization. This interpretation is consistent with the metrics: compared with strong constraints, medium control sacrifices some early speed (AUC $192.7$ vs. $205.3$; convergence $146$ vs. $118$) but gains considerably in robustness ($0.84$ vs. $0.71$), mission success ($0.91$ vs. $0.86$), and final return ($379.8$ vs. $352.6$). Compared with weak constraints, it maintains most robustness while sharply improving learning speed and organizational compliance.

\subsection{Explainability and Trajectory Analysis}
A major gap identified in Section~\ref{sec:related} is the lack of operator-oriented interpretability in MARL control pipelines. We analyze this gap through ORBITAL-TEMM outputs.

\paragraph{Quantitative explainability indicators}
Table~\ref{tab:results-explainability} reports trajectory-analysis indicators. The proposed method improves both structural and functional alignment, and produces clearer role/mission separation in trajectory space.

\begin{table}[t]
  \caption{Illustrative TEMM/explainability indicators.}
  \label{tab:results-explainability}
  \centering
  \scriptsize
  \begin{tabularx}{\columnwidth}{p{2.0cm} c c c}
    \toprule
    Condition         & Role-cluster separability & Mission alignment        & Human-audit agreement    \\
    \midrule
    LB-Unconstrained  & $0.41 \pm 0.06$           & $0.62 \pm 0.07$          & $0.58 \pm 0.08$          \\
    LB-RewardOnly     & $0.49 \pm 0.05$           & $0.68 \pm 0.06$          & $0.64 \pm 0.07$          \\
    Proposed (medium) & $\mathbf{0.67 \pm 0.04}$  & $\mathbf{0.83 \pm 0.04}$ & $\mathbf{0.79 \pm 0.05}$ \\
    \bottomrule
  \end{tabularx}
\end{table}

\paragraph{Qualitative pattern consistency}
Trajectory inspection confirms the quantitative trend. Observer-like trajectories exhibit persistent acquisition motifs followed by relay transitions when buffering thresholds are reached. Relay-like trajectories show draining patterns synchronized with communication opportunities, and safety-guard trajectories display increased \texttt{LowPower}/\texttt{CyberScan} usage under stress intervals. These motifs are substantially less separable in unconstrained runs, which is coherent with the lower role-cluster separability score in Table~\ref{tab:results-explainability} ($0.41 \pm 0.06$ vs. $0.67 \pm 0.04$ for the proposed method).

\paragraph{Operational relevance}
From an operator viewpoint, TEMM outputs are useful because they provide traceability of role-consistent behavior, diagnostics of where deontic constraints are repeatedly violated, and tuning guidance about which role/mission guides are too weak or too restrictive. The quantitative side of this claim is reflected by the increase in mission alignment ($0.83 \pm 0.04$ vs. $0.62 \pm 0.07$ for unconstrained) and human-audit agreement ($0.79 \pm 0.05$ vs. $0.58 \pm 0.08$), which indicates that the inferred organizational structure is both more coherent and more usable for post-hoc operational analysis.

\subsection{Gap-by-Gap Discussion}
We now map results to each gap identified in Introduction and Related Work.

\paragraph{Gap G1: no benchmark combining mission pressure, resource pressure, communication uncertainty, and cyber uncertainty}
The stress-test and baseline comparisons show that ORBITAL discriminates policy quality across all axes simultaneously. The spread in robustness, mission success, and cyber-impact metrics indicates that the benchmark is neither trivial nor single-factor. This supports ORBITAL as a meaningful experimental testbed for fleet-assist methods.

\paragraph{Gap G2: limited controllability/safety in unconstrained MARL}
Compared to LB-Unconstrained, the proposed method reduces violation rates and improves convergence while preserving higher mission success. Ablations confirm that this is not a backbone-only effect: control benefits disappear when role or mission guides are removed.

\paragraph{Gap G3: poor trade-off management between fast convergence and robustness}
The strong/medium/weak comparison validates the expected trade-off shape and identifies medium constraints as the practical optimum. This is a key result: organizational constraints should not be maximal by default; they should be tuned to preserve strategic emergence.

\paragraph{Gap G4: lack of operationally useful explainability}
TEMM-based indicators and role/mission trajectory motifs provide interpretable, quantitatively trackable organizational evidence. The increase in cluster separability and audit agreement supports the claim that ORBITAL-oriented trajectory analysis improves practical explainability.

\subsection{Threats to Validity}
Even in this protocol-conformant illustrative dataset, the same validity risks apply as in a full campaign. External validity remains bounded by ORBITAL's PoC abstraction; statistical power depends on seed count, variance control, and stress-level coverage; and explainability conclusions depend on trajectory feature design and clustering sensitivity in TEMM. These limitations motivate broader scenario sweeps, deeper uncertainty quantification, and progressively richer environment realism in future work.

\section{Conclusion}
\label{sec:conclusion}

This paper introduced a two-layer contribution toward automated operational assist for satellite fleet management. First, we proposed ORBITAL as a mission-oriented proof-of-concept benchmark that jointly exposes mission pressure, resource pressure, communication uncertainty, and cyber uncertainty in a reproducible multi-agent setting. Second, we proposed an ORBITAL-oriented adaptation of MOISE+MARL \cite{soule2024moise_marl}, combining organizational role/mission constraints with trajectory-based analysis (ORBITAL-TEMM) to improve controllability, safety compliance, and explainability.

Across the experimental protocol, the main result pattern is a structured control trade-off: strong constraints accelerate early convergence but may reduce robustness, weak constraints preserve robustness but slow down learning, and medium constraints provide the best compromise between performance, convergence speed, and resilience. This supports the core claim of the paper: organizational specifications are most useful when they guide trivial or safety-critical behavior while preserving sufficient policy freedom for emergent cooperative strategies.

The work nonetheless has clear limitations. ORBITAL remains a coarse abstraction and does not yet model high-fidelity orbital mechanics, full communication stack realism, or complete operational doctrine complexity. The current organizational specifications are manually designed, which can introduce bias and limit scalability in larger fleets. In addition, TEMM-based analysis quality depends on trajectory features and clustering choices, and therefore requires careful calibration for robust operational interpretation.

These limitations define three immediate research directions. First, increase environment realism progressively (richer dynamics, constraints, and scenario diversity) while preserving reproducibility. Second, integrate model-based reinforcement learning and multi-agent world models to improve sample efficiency, forecasting capability, and long-horizon planning robustness. Third, develop stronger neuro-symbolic integration so that learned behaviors and organizational rules can be co-designed with formal safety constraints, improving reliability and deployability for real operational-assist workflows.


%%%%%%%%%%%%%%%%%%%%%%%%%%%%%%%%%%%%%%%%%%%%%%%%%%%%%%%%%%%%%%%%%%%%%%%%

%%% The acknowledgments section is defined using the "acks" environment
%%% (rather than an unnumbered section). The use of this environment 
%%% ensures the proper identification of the section in the article 
%%% metadata as well as the consistent spelling of the heading.

% \begin{acks}
%   If you wish to include any acknowledgments in your paper (e.g., to
%   people or funding agencies), please do so using the `\texttt{acks}'
%   environment. Note that the text of your acknowledgments will be omitted
%   if you compile your document with the `\texttt{anonymous}' option.
% \end{acks}

%%%%%%%%%%%%%%%%%%%%%%%%%%%%%%%%%%%%%%%%%%%%%%%%%%%%%%%%%%%%%%%%%%%%%%%%

%%% The next two lines define, first, the bibliography style to be 
%%% applied, and, second, the bibliography file to be used.

\nocite{soule2024moise_marl}

\bibliographystyle{ACM-Reference-Format}
\bibliography{references}

%%%%%%%%%%%%%%%%%%%%%%%%%%%%%%%%%%%%%%%%%%%%%%%%%%%%%%%%%%%%%%%%%%%%%%%%

\end{document}

%%%%%%%%%%%%%%%%%%%%%%%%%%%%%%%%%%%%%%%%%%%%%%%%%%%%%%%%%%%%%%%%%%%%%%%%
