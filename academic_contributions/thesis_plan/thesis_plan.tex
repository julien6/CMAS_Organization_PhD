\documentclass{article}

\usepackage[T1]{fontenc}
\usepackage{graphicx}
%\usepackage{color}
%\renewcommand\UrlFont{\color{blue}\rmfamily}

\usepackage{amsmath,amssymb,amsfonts}
\usepackage{algorithmic}
\usepackage[inline, shortlabels]{enumitem}
\usepackage{tabularx}
\usepackage{caption}
% \usepackage{titlesec}
\usepackage[english]{babel}
\captionsetup{font=it}
\usepackage{ragged2e}
\usepackage{hyperref}
\usepackage{pifont}
\usepackage{footmisc}
\usepackage{multirow}

% --- Tickz
\usepackage{physics}
\usepackage{amsmath}
\usepackage{tikz}
\usepackage{mathdots}
\usepackage{yhmath}
\usepackage{cancel}
\usepackage{color}
\usepackage{siunitx}
\usepackage{array}
\usepackage{multirow}
\usepackage{amssymb}
\usepackage{gensymb}
\usepackage{tabularx}
\usepackage{extarrows}
\usepackage{booktabs}
\usetikzlibrary{fadings}
\usetikzlibrary{patterns}
\usetikzlibrary{shadows.blur}
\usetikzlibrary{shapes}

% ---------

\usepackage{pdfpages}
\usepackage{booktabs}
\usepackage{csquotes}
\usepackage{lipsum}  
\usepackage{arydshln}
\usepackage{smartdiagram}
\usepackage[inkscapeformat=png]{svg}
\usepackage{textcomp}
\usepackage{tabularray}\UseTblrLibrary{varwidth}
\usepackage{xcolor}
\def\BibTeX{{\rm B\kern-.05em{\sc i\kern-.025em b}\kern-.08em
    T\kern-.1667em\lower.7ex\hbox{E}\kern-.125emX}}
\usepackage{cite}
\usepackage{amsmath}
\newcommand{\probP}{\text{I\kern-0.15em P}}
\usepackage{etoolbox}
\patchcmd{\thebibliography}{\section*{\refname}}{}{}{}

\setlength{\extrarowheight}{2.5pt}

\renewcommand{\arraystretch}{1.7}

\setlength{\extrarowheight}{2.5pt}
\renewcommand{\arraystretch}{0.2}
\renewcommand{\arraystretch}{1.7}

\newcommand{\before}[1]{\textcolor{red}{#1}}
\newcommand{\after}[1]{\textcolor{green}{#1}}

\newcommand{\old}[1]{\textcolor{orange}{#1}}
\newcommand{\rem}[1]{\textcolor{red}{#1}}
\newcommand{\todo}[1]{\textcolor{orange}{\newline \textit{\textbf{TODO:} #1}} \newline \newline }


% --------------------------------
%             DOCUMENT
% --------------------------------

\begin{document}
%
\title{An Approach for Multi-Agent System Design with Reinforcement Learning}
%
%\titlerunning{Abbreviated paper title}
% If the paper title is too long for the running head, you can set
% an abbreviated paper title here
%
\author{Julien Soulé\inst{1}\orcidID{0000-1111-2222-3333} \and
    Jean-Paul Jamont\inst{1}\orcidID{1111-2222-3333-4444} \and
    Michel Occelo\inst{1}\orcidID{2222--3333-4444-5555} \and
    Louis-Marie Traonouez\inst{2}\orcidID{2222--3333-4444-5555} \and
    Paul Théron\inst{3}\orcidID{2222--3333-4444-5555}}
%
\authorrunning{J. Soulé et al.}
% First names are abbreviated in the running head.
% If there are more than two authors, 'et al.' is used.
%
\institute{Univ. Grenoble Alpes, Grenoble INP, LCIS, 26000, Valence, France
    \email{\{julien.soule, jean-paul.jamont, michel.occello\}@lcis.grenoble-inp.fr}
    \and
    Thales Land and Air Systems, BU IAS, Rennes, France
    \email{louis-marie.traonouez@thalesgroup.com}
    \and
    AICA IWG, La Guillermie, France \\
    \email{paul.theron@orange.fr}
}
%
\maketitle              % typeset the header of the contribution
%
\section{Introduction}

\subsection{Genèse / contexte du sujet}

\begin{itemize}
    \item AICA, besoins nouveaux, IoT/IoBT, etc.
    \item Approche centralisée peu adaptée, etc. pour des raisons d’interruptions de communication, hétérogénéité des SI, etc.
    \item Une approche MA pourrait être appliquée -> Système Multi-Agents de Cyberdéfense (SMAC)
    \item Mais sujet nouveau : pas de modélisation, travaux formels…
\end{itemize}

\subsection{Problématique générale}
\begin{itemize}

    \item Quelle méthode pour concevoir un SMAC qui atteint ses objectifs de cyberdéfense tout en satisfaisant les contraintes de déploiement et opérationelles du système hôte qu'il doit défendre ?
\end{itemize}

\subsection{Positionnement et thèse défendue}
\begin{itemize}

    \item Un ensemble d’agents collaboratifs répond effectivement aux nouveaux besoins, etc. mieux que des solutions centralisées
    \item Une méthode modélisant le "domaine" (environnement réseau + actions/observations possibles des red/blue/green teams), du "problème" (objectif de cyberdéfense / blue team + contraintes opérationelles/déploiement) sous forme d'un **problème d'optimisation sous-contraintes*\item ; permet de fournir des moyens objectifs d’évaluer de façon consistante si le SMA tient ses promesses dans plusieurs scénarios d’attaque…
\end{itemize}




\section{Un aperçu du domaine et du problème }

\subsection{Définitions et propriétés}
\begin{itemize}

    \item Définitions \& propriétés fondammentales pour la suite
    \item cyberdéfense, RL et SMA (+IA hybride éventuellement)
    \item ex : ouverture, dynamique, auto/réorganisation, explicabilité, etc.
\end{itemize}
\subsection{Travaux liés}
\begin{itemize}

    \item Travaux liés à l’AICA et autres SMA de Cyberdéfense
          \begin{itemize}
              \item SMA : organisation, modèle organisationel...
              \item Travaux de l'Autonomous Cyber Operation
              \item ...
          \end{itemize}
\end{itemize}
\subsection{Limitations et discussion}
\begin{itemize}

    \item Manque de généricité, consistance, pas/peu objectif, peu formel, etc.
    \item Besoin d’un cadre théorique consistant et générique si possible
\end{itemize}
\subsection{Choix du cadre théorique}
\begin{itemize}
    \item Motivation pour : Green, blue, red teams + réseau de noeud avec système d’ataque/contre-attaques d’après les standards de cyberdéfense, reste générique, etc.
\end{itemize}
\subsection{Vers une méthode générale}
\begin{itemize}

    \item Motiver les choix préliminaires pour la méthode en mettant en relation les travaux liés de façon cohérente
    \item Modèle organisationel et MARL, Cyberdéfense avec MITRE...
    \item Cela doit permettre au lecteur de comprendre où je veux en venir par la suite...
\end{itemize}


\section{CybMASFM: Cyberdefense Multi-Agent Systems Formal Model}
\begin{itemize}

    \item \textbf{Ici, on cherche à poser les bases pour un modèle qui exprime de façon cohérente et sans-ambiguités/formellement les éléments dont nous avons besoin pour le domaine et le problème}
    \item \textbf{Cela est fondammental pour la suite car l'approche et l'outil se repose sur CybMASFM}
\end{itemize}

\subsection{Approches de modélisation pour un SMA de Cyberdéfense}
\begin{itemize}

    \item Modèles partiels (Attack-defense tree, petri nets, etc)
    \item Modèles de la théorie des jeux (POSG, Dec-POMDP, etc.)
\end{itemize}

\subsection{Comparaison et choix du modèle}
\begin{itemize}

    \item Tableau comparatifs, discussion…
    \item Motivations pour le Dec-PODMP d’après le rapprochement entre incertitude des observations, conditions pour les actions, dynamique de l’env, récompense collective et non individuelle, etc.
    \item Le Dec-POMDP reste assez générique : plusieurs approche d’organisation / mécanismes / algos d’IA sont envisageable dans ce cadre formel
\end{itemize}

\subsection{Modèle formel du domaine}
\begin{itemize}

    \item Montrer comment le cadre théorique non-formel (environnement + teams) peut être formalisé dans le Dec-POMDP
          \begin{itemize}
              \item Le domaine doit aussi inclure la question de l'organisation dans l'équipe bleu (+ rouge éventuellement)
          \end{itemize}
    \item Le modèle formel du domaine (i.e le modèle formel de l'environnement et le modèle formel de l'organisation) doit permettre d'exprimer formellement la question de l'organisation comme un problème d'optimisation sous contraintes (i.e de l'environnement et de l'organisation voulue par le concepteur càd ses spécifications)
\end{itemize}

\subsection{Modèle de l'environnement}
\begin{itemize}

    \item Montrer comment lier le Dec-POMDP avec les connaissances liées aux attaques (MITRE ATTACK) et où on peut mettre en place des contre-mesures (MITRE DEFEND)
    \item TODO...
\end{itemize}

\subsection{Modèle d'une organisation dans les SMA}
\begin{itemize}

    \item Expliquer comment nous envisageons les organisations possibles au travers de : Conscience/inconscience de l’organisation + Centré agent / organisation.
          \begin{itemize}
              \item Positionnement par rapport à la littérature
          \end{itemize}
    \item Expliquer notre vision où une organisation dans un MAS peut être résumée comme : le résultat d'une recherche dans un espace des organisations contraint par les contraintes liées à l'environnement et celles du concepteur (spécifications initiales du concepteur)
    \item On peut illustrer cela dans les 3 cas ci-dessous
    \item **Les organisations "totalement prédéfinis"*\item :
          \begin{itemize}
              \item Les spécifications initiales du concepteur contraignent totalement le processus de conception à une seule organisation possible
              \item 1 seul épisode : chaque agent suit une liste de règles (associant un ensemble d’observation à une action) qui ne changent jamais (sans apprentissage mais basé sur la connaissance/expertise du concepteur) : hiérarchie, mécanisme d’enchère, coalition, etc.
              \item Coalition based Multi Agent System AICA (CMASA), Market based Multi Agent System AICA (MMASA), etc.
          \end{itemize}
    \item \textbf{Les organisations "totalement indéfinies"}
          \begin{itemize}
              \item Les spécifications initiales du concepteur n'ont aucun impact sur la façon dont on peut concevoir les agents (le concepteur humain ou MARL peut choisir librement comment concevoir les règles des agents)
              \item Plusieurs épisodes : chaque agent voit ses propres règles changer quand cela maximise la récompense (QLearning, DQN en full automatique ou bien le concepteur humain)
              \item Aboutit à une solution local (ensemble de politique) mais pas facilement explicable en MARL (d'où le besoin d'avoir des spécifications en même temps)
              \item QLearning based Multi Agent System AICA (QMAS), etc.
          \end{itemize}
    \item \textbf{Les organisation "semi-définies"}
    \item Les spécifications initiales du concepteur couvrent partiellement l'espace des organisations possibles
          \begin{itemize}
              \item Par exemple : trouver des SMA satisfaisant l'architecture hierarchique mais sans savoir précisement quel agent doit adopter quel rôle, etc.
          \end{itemize}
    \item Mécanisme d’organisation déjà en place mais les hyper-paramètres doivent être ajustés avec un apprentissage
    \item Mix entre agents aux politiques définis et indéfinis
    \item D’abord indéfini avec recherche du mécanisme d’organisation défini optimal pour le scénario donné
    \item Adaptive Multi Agent System AICA (AMASA) : Une architecture polyvalente pour un SMA de Cyberdéfense, etc.
\end{itemize}

\subsection{Expression du problème}
\textbf{Ici, j'utilise les modèles formels précédents pour poser le problème de façon formelle}

\textbf{ici, on ne doit pas encore comprendre que le MARL sera choisi pour la suite en combinaison des modèles d'organisation de SMA comme Moise+}

\begin{itemize}

    \item L'environnement peut être traduit en contraintes qui réduit l'espace des organisations (i.e joint-policy) à celles qui sont réellement possibles (cf. Moise+)
    \item Les spécifications appliquées à l'OE contraignent également l'espace des organisations (i.e joint-policy) à celles qui sont spécifiées par le concepteur
    \item Les objectifs de cyberdéfense peuvent être traduits en une fonction à maximiser
    \item \textbf{Problème} : Trouver l’ensemble des politiques (joint-policy) respectant les contraintes telles que sur un épisode, la récompense cumulée des agents bleus soit maximale / supérieur à un seuil
    \item \begin{itemize}
              \item Pour une fonction de récompense donnée Rew, les variables inconnues à maximiser sont les politiques PIj (1<j<nb agents):
                    $max (Sum{1, .., i, .., nb\_it} Rew(PI1, PI2, …, PIn))$
              \item De plus, la joint-policy obtenue doit être associé à des specifications compréhensibles pour un être humain
                    \begin{itemize}
                        \item $Design(Env., Specs\_init, JointPolicy\_init) = (Specs\_opt, JointPolicy\_opt)$
                    \end{itemize}
          \end{itemize}
    \item \textbf{Maintenant que l'on a posé le problème il faut le résoudre...}
\end{itemize}



\section{CybMASDA: Cyberdefense Multi-Agent Systems Developpment Approach}

\begin{itemize}

    \item \textbf{Ici, on se positione au niveau du concepteur et du développeur qui veut un résultat tangible à la fin}
    \item \textbf{On propose une approche qui utilise le problème formel décrit précédement pour concevoir l'organisation du MAS en simulation puis on l'implémente réellement en émulation}
\end{itemize}

\subsection{Approche de conception théorique combinant MARL et modèle d'organisation de MAS}

\begin{itemize}

    \item Présenter les travaux qui visent à passer des joint-policy du MARL aux spécifications du modèle organisationel Moise+
          \begin{itemize}

              \item Travaux sur les résultats obtenus en MARL après entrainement pour en extraire les spécifications en Moise+
              \item Travaux pour contraindre l'entrainement MARL à respecter des spécifications décrites avec Moise+
          \end{itemize}
    \item La définition du problème autorise au moins 2 cas différents qui seront présentés sous forme d'exemples :
          \begin{itemize}
              \item \textbf{Spec\_Init vide, joint-policy -> Moise+}: Moise+ intervient après l'entrainement en MARL
              \item \begin{itemize}
                        \item La joint-policy est obtenue après entrainement en MARL (qui inclut implicitement les contraintes de l'environnement) puis il faut faire un travail d'analyse (au moins partiellement automatisable) pour en extraire les spécifications de l'organisation de Moise+
                    \end{itemize}
              \item \textbf{Spec\_Init non vide, joint-policy -> Moise+}: Moise+ intervient après et pendant l'entrainement en MARL
              \item \begin{itemize}
                        \item La joint-policy est obtenue après entrainement en MARL qui doit prendre en compte les contraintes des Spec\_Init (en plus des contraintes de l'environnement) puis il faut faire un travail d'analyse (au moins partiellement automatisable) pour en extraire les spécifications de l'organisation de Moise+
                    \end{itemize}
          \end{itemize}
    \item Montrer qu'on peut aussi imaginer d'autres exemples...
          \begin{itemize}
              \item On a définit "à la main" une joint-policy et on veut savoir les spécifications de l'organisation en Moise+
              \item On a des spécifications initiales qui contraignent le MARL à converger vers une seule organisation possible et on veut determiner la joint-policy correspondante
          \end{itemize}
    \item \textbf{L'attendu de cette approche est que le concepteur doit être en mesure de posseder une joint-policy / des joint-policies avec les spécifications de l'organisation associées qui sont suffisament performantes et respectent les contraintes}
\end{itemize}

\subsection{Approche de développement basée sur des cycles de simulation et émulation}

\subsection{La simulation pour la conception d'organisation de MAS candidates}
\begin{itemize}

    \item Présentations des travaux correspondant au mieux aux besoins des SMA et de la cyberdéfense
    \item Comparaison et aboutissement sur l’idée d’étendre l’environnement CybORG du framework PettingZoo (code libre, issu d’un travail de recherche précédent publié à IJCAI, contexte d’application très proche et pertinent, compatibilité avec le modèle Dec-POMDP précédent, etc.)
    \item L'approche de conception de l'organisation peut être de façon sûre appliquée en simulation car il n'y a pas de risque d'endommager le système cible
    \item Possibilité de faire du "system identification" pour créer le modèle de simulation et ainsi réduire le gap entre émulation et simulation
          \begin{itemize}
              \item Cela permettrait de ne pas utiliser CybORG
          \end{itemize}
    \item + autres avantages de la simulation
\end{itemize}

\subsection{L'émulation pour valider un SMAC candidat}
\begin{itemize}

    \item Reproduction du système cible sous une forme émulée (avec container)
    \item Mise en place d'un dispositif experimental pour transferer les agents de la simulation en émulation
    \item Validation des SMAC candidats et implémentation dans le système cible

\end{itemize}


\section{CybMASDE: Cyberdefense Multi-Agent Systems Developpment Environment}
\begin{itemize}

    \item Montrer comment CybMASDE peut être utilisé de façon systématique/consistante pour définir un scénario (env, red team, green team) + une blue team (i.e un SMA de cyberdéfense) en définissant soit même les politiques des agents
    \item Montrer que les modèles de la simulation peuvent être mappés à des modèles émulés afin de verifier la veritable performance des SMA de cyberdéfense proposé AVEC l'intérêt du transfer learning pour l’apprentissage dans la simulation (car rapide et leger) et vérification dans l’émulation (machines virtuelles)
\end{itemize}



\section{Mise en place d'un modèle AICA}
\begin{itemize}

    \item \textbf{Ici l'idée est de créer un modèle AICA integré dans notre méthode et qui pourra être utilisé comme un SMA polyvalent capable d'être ajusté pour des scénarios différents}
\end{itemize}
\subsection{Traduction de MASCARA dans CybMASFM}
\begin{itemize}

    \item Traduire au moins dans l'idée les composants de MASCARA comme des spécifications de l'organisations qui seront pris en compte pour generer une organisation dans CybMASDA
\end{itemize}
\subsection{Intégration de l'AICA dans CybMASDA et CybMASDE}
\begin{itemize}

    \item Expliquer comment nous avons pris en compte l'AICA dans l'outil afin qu'il soit utilisable directement
    \item Problématiques d'implémentation dues à la complexité de l'architecture
\end{itemize}


\section{Expérimentation et comparaisons}
\begin{itemize}

    \item Présentation de 3 cas d’études : Drone swarm, company network et Kubernetes
    \item Sur le modèle du tutoriel en utilisant l'outil CybMASDE
\end{itemize}
\subsection{Présentation des études de cas dans CybMASFM}
\begin{itemize}

    \item Montrer comment nous utilisons notre modèle concrètement avec CybMASDE pour comprendre et définir le problème dans chaque étude de cas
\end{itemize}

\subsection{Evaluation pour les 3 cas d’étude dans CybMASDA}
\begin{itemize}

    \item Montrer comment nous utilisons CybMASDE pour résoudre le problème
          \begin{itemize}
              \item En utilisant également le modèle AICA
          \end{itemize}
\end{itemize}

\subsection{Synthèse et discussions}
\begin{itemize}

    \item Discuter le niveau de l'impact de la méthode
          \begin{itemize}
              \item Sans spécifications initiales
              \item Avec spécifications initiales (incluant l'AICA)
          \end{itemize}
    \item Discuter des organisations qui semblent pertinentes avec une analyse quantitative sur les 3 cas d'étude
    \item Discuter des résultats et la cohérence de ces-derniers par rapport à d'autres résultats
\end{itemize}



\section{Conclusion}
\begin{itemize}

    \item Conclure sur la partie "académique" de la contribution
    \item Expliquer qu'au delà de répondre à la question de la thèse, la contribution permet aussi de répondre à des problèmes de l'industrie sur la protection de système sur certains aspects
\end{itemize}

\subsection{Vers une application industrielle comme aide à la décision}
\begin{itemize}

    \item Simuler des événements qui pourraient arriver et essayer de gagner de l'expérience en prévision du moment où l’attaque sera déjà en cours
    \item Parler de l'experience de l'approche / outil en industrie avec Thales
\end{itemize}

\subsection{Limitations et perspectives}

\begin{itemize}

    \item Evoquer la difficulté du passage à l'échelle
    \item Le manque de maturité (TRL ne dépasse pas 3/4)
    \item Travaux connexes sur l'explicabilité au niveau collectif via d'autres approches pas nécéssairement dans la sécurité des réseaux...
\end{itemize}

\section{Bibliographie}


\end{document}
