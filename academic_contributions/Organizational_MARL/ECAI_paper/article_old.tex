%%%%%%%%%%%%%%%%%%%%%%%%%%%%%%%%%%%%%%%%%%%%%%%%%%%%%%%%%%%%%%%%%%%%%%%%

%%% LaTeX Template for ECAI Papers 
%%% Prepared by Ulle Endriss (version 1.0 of 2023-12-10)

%%% To be used with the ECAI class file ecai.cls.
%%% You also will need a bibliography file (such as mybibfile.bib).

%%%%%%%%%%%%%%%%%%%%%%%%%%%%%%%%%%%%%%%%%%%%%%%%%%%%%%%%%%%%%%%%%%%%%%%%

%%% Start your document with the \documentclass{} command.
%%% Use the first variant for the camera-ready paper.
%%% Use the second variant for submission (for double-blind reviewing).

\documentclass{ecai} 

%\documentclass[doubleblind]{ecai} 

%%%%%%%%%%%%%%%%%%%%%%%%%%%%%%%%%%%%%%%%%%%%%%%%%%%%%%%%%%%%%%%%%%%%%%%%

%%% Load any packages you require here. 

\usepackage{latexsym}
\usepackage{amssymb}
\usepackage{amsmath}
\usepackage{amsthm}
\usepackage{booktabs}
\usepackage{enumitem}
\usepackage{graphicx}
\usepackage{color}

\usepackage[T1]{fontenc}
\usepackage{graphicx}
%\usepackage{color}
%\renewcommand\UrlFont{\color{blue}\rmfamily}

\usepackage{amsmath,amssymb,amsfonts}
% \usepackage[inline, shortlabels]{enumitem}
\usepackage{tabularx}
\usepackage{caption}
% \usepackage{titlesec}
\usepackage[english]{babel}
\captionsetup{font=it}
\usepackage{ragged2e}
\usepackage{hyperref}
\usepackage{pifont}
\usepackage{footmisc}
\usepackage{multirow}
\usepackage{algorithm2e}

% --- Tickz
\usepackage{physics}
\usepackage{amsmath}
\usepackage{tikz}
\usepackage{mathdots}
\usepackage{yhmath}
\usepackage{cancel}
\usepackage{color}
\usepackage{siunitx}
\usepackage{array}
\usepackage{multirow}
\usepackage{amssymb}
\usepackage{gensymb}
\usepackage{tabularx}
\usepackage{extarrows}
\usepackage{booktabs}
\usetikzlibrary{fadings}
\usetikzlibrary{patterns}
\usetikzlibrary{shadows.blur}
\usetikzlibrary{shapes}

% ---------

\usepackage{pdfpages}
\usepackage{booktabs}
\usepackage{csquotes}
\usepackage{lipsum}  
\usepackage{arydshln}
\usepackage{smartdiagram}
\usepackage[inkscapeformat=png]{svg}
\usepackage{textcomp}
\usepackage{tabularray}\UseTblrLibrary{varwidth}
\usepackage{xcolor}
\def\BibTeX{{\rm B\kern-.05em{\sc i\kern-.025em b}\kern-.08em
    T\kern-.1667em\lower.7ex\hbox{E}\kern-.125emX}}
% \usepackage{cite}
\usepackage{amsmath}
\newcommand{\probP}{\text{I\kern-0.15em P}}
\usepackage{etoolbox}

%%%%%%%%%%%%%%%%%%%%%%%%%%%%%%%%%%%%%%%%%%%%%%%%%%%%%%%%%%%%%%%%%%%%%%%%

%%% Define any theorem-like environments you require here.

\newtheorem{theorem}{Theorem}
\newtheorem{lemma}[theorem]{Lemma}
\newtheorem{corollary}[theorem]{Corollary}
\newtheorem{proposition}[theorem]{Proposition}
\newtheorem{fact}[theorem]{Fact}
\newtheorem{definition}{Definition}

%%%%%%%%%%%%%%%%%%%%%%%%%%%%%%%%%%%%%%%%%%%%%%%%%%%%%%%%%%%%%%%%%%%%%%%%

%%% Define any new commands you require here.

% \newcommand{\BibTeX}{B\kern-.05em{\sc i\kern-.025em b}\kern-.08em\TeX}

\renewcommand{\arraystretch}{1.7}

\setlength{\extrarowheight}{2.5pt}
\renewcommand{\arraystretch}{0.2}
\renewcommand{\arraystretch}{1.7}

\newcommand{\before}[1]{\textcolor{red}{#1}}
\newcommand{\after}[1]{\textcolor{green}{#1}}

\newcommand{\old}[1]{\textcolor{orange}{#1}}
\newcommand{\rem}[1]{\textcolor{red}{#1}}
\newcommand{\todo}[1]{\textcolor{orange}{\newline \textit{\textbf{TODO:} #1}} \newline \newline }



\newcounter{relation}
\setcounter{relation}{0}
\renewcommand{\therelation}{\arabic{relation}}
\newcommand{\relationautorefname}{Relation}

\newenvironment{relation}[1][]{%
    \refstepcounter{relation}%
    \noindent \raggedright \textit{\textbf{Relation. \therelation}} \hfill$}
{%
$ \hfill \phantom{x}

}

\newcounter{proof}
\setcounter{proof}{0}
\renewcommand{\theproof}{\arabic{proof}}
\newcommand{\proofautorefname}{Proof}

\renewenvironment{proof}[1][]{
    \refstepcounter{proof}
    \noindent \raggedright \textit{\textbf{Proof. \theproof}}

    \setlength{\leftskip}{1em}

}
{

\
\setlength{\leftskip}{0pt}
}

%%%%%%%%%%%%%%%%%%%%%%%%%%%%%%%%%%%%%%%%%%%%%%%%%%%%%%%%%%%%%%%%%%%%%%%%

\begin{document}

%%%%%%%%%%%%%%%%%%%%%%%%%%%%%%%%%%%%%%%%%%%%%%%%%%%%%%%%%%%%%%%%%%%%%%%%

\begin{frontmatter}

    %%% Use this command to specify your submission number.
    %%% In doubleblind mode, it will be printed on the first page.

    \paperid{123}

    %%% Use this command to specify the title of your paper.

    \title{Unveiling Cooperative Intelligence: Towards Explainable Multi-Agent Reinforcement Learning with Organizational Models}

    % JS: keywords:
    %     Multi-Agent Reinforcement Learning
    %     Explainability
    %     Organizational Models
    %     Cooperative Intelligence

    %%% Use this combinations of commands to specify all authors of your 
    %%% paper. Use \fnms{} and \snm{} to indicate everyone's first names 
    %%% and surname. This will help the publisher with indexing the 
    %%% proceedings. Please use a reasonable approximation in case your 
    %%% name does not neatly split into "first names" and "surname".
    %%% Specifying your ORCID digital identifier is optional. 
    %%% Use the \thanks{} command to indicate one or more corresponding 
    %%% authors and their email address(es). If so desired, you can specify
    %%% author contributions using the \footnote{} command.

    \author[A,B]{\fnms{Julien}~\snm{Soulé}\thanks{Corresponding Author. Email: julien.soule@lcis.grenoble-inp.fr}}
    \author[A]{\fnms{Jean-Paul}~\snm{Jamont}}
    \author[A]{\fnms{Michel}~\snm{Occello}}
    \author[B]{\fnms{Louis-Marie}~\snm{Traonouez}}
    \author[C]{\fnms{Paul}~\snm{Théron}}

    \address[A]{Univ. Grenoble Alpes, Grenoble INP, LCIS, 26000, Valence, France}
    \address[B]{Thales Land and Air Systems, BL IAS, Rennes, France}
    \address[C]{AICA IWG, La Guillermie, France}

    %%% Use this environment to include an abstract of your paper.

    \begin{abstract}
        This paper addresses the challenge of explainability in Multi-Agent Reinforcement Learning (MARL) by proposing a novel algorithm leveraging the $\mathcal{M}OISE^+$ Organizational Model. While previous studies have focused on understanding individual agent behaviors in traditional RL, our work emphasizes the need to elucidate the implicit cooperation among multiple agents in MARL systems. We introduce the \emph{Partial Relation between Agents' History and Organizational Model} (PRAHOM) algorithm, which links agents' histories to organizational specifications, facilitating the inference of cooperative structures from trained agents' behaviors. PRAHOM serves dual purposes: constraining the learning process based on organizational constraints and inferring organizational specifications from trained agents' histories. PRAHOM is designed to constrain agent policy spaces and generate valuable organizational specifications. Empirical evaluations conducted in cooperative Atari-like game environments validate the effectiveness of our algorithm, showing alignment with hand-crafted expectations and superior performance in certain scenarios. This work contributes to advancing AI explainability in MARL systems, offering a principled framework for understanding emergent cooperative behaviors. By bridging the gap between individual agent decision-making and cooperation, our algorithm enhances transparency and interpretability in complex Multi-Agent Systems.
    \end{abstract}

\end{frontmatter}

%%%%%%%%%%%%%%%%%%%%%%%%%%%%%%%%%%%%%%%%%%%%%%%%%%%%%%%%%%%%%%%%%%%%%%%%

\section{Introduction}

% Context
Explainable Artificial Intelligence (XAI) has emerged as a critical requirement for the widespread adoption of AI systems, particularly in multi-agent settings where multiple agents interact and cooperate to achieve complex goals~\citep{doshivelez2017rigorous,gunning2019xai}. While significant progress has been made in explaining the behavior of single agents~\citep{ribeiro2016classifier,lundberg2017unified}, the challenge of explicating the cooperative strategies and emergent organizational structures in Multi-Agent Reinforcement Learning~\citep{busoniu2008survey} (MARL) systems remains largely unaddressed~\citep{kok2006collaborative,omidshafiei2019learning}.

% Problem
In MARL systems, a set of trained agents can achieve goals through implicit cooperation and coordination. However, very few works have attempted to analyze the trained agents' policies to make this cooperation explicit~\citep{albrecht2018survey,perolat2017pool}. This lack of explainability hinders the understanding of the complex social, cooperation, and coordination schemes that emerge between the trained agents, limiting the trust and adoption of MARL systems in real-world applications~\citep{kok2006collaborative,omidshafiei2019learning}.
% The idea of benefitting of the particularly adaptive and general MARL mechanism to get an approximated suited MAS organization and producing associated specifications, requires to link the MARL training of a set of policies in a bidirectional way with a MAS organizational model. For instance, a hierarchy described in a MAS model would constrain the possible policies to get ultimately trained in MARL. Reversely, a set of trained policies could be described in an organizational model, thus indicating resemblance with known MAS organization architectures.

% Contribution
To address this problem, we propose a novel approach called Partial Relation between Agents' History and Organizational Model (PRAHOM). PRAHOM leverages the MOISE+ organizational model~\citep{hubner2007moise} to provide explanations for the cooperative behaviors observed in MARL systems. The key idea is to establish a relationship between the agents' action histories, obtained from their learned policies, and the organizational specifications, such as roles, goals, and social links.
% This paper first informally introduce the aforementioned broad idea as a new specific research study we refer to as \textquote{Organizational Multi-Agent Reinforcement Learning} (OMARL). We present DMO (Dec-POMDP $\mathcal{M}OISE^+$ OMARL) a first attempt to formally describe OMARL processes using the Decentralized Partially Observable Markov Decision Process (Dec-POMDP) as model for MARL and $\mathcal{M}OISE^+$ as an organizational model. Then, we propose PRAHOM (Partial Relations between Agents' Histories and Organizational Model), a DMO process that allows both getting some organizational specifications from \textquote{trained} policies and constraining policies training with respect to given specifications. As PRAHOM only changes the action set and relies \textquote{observed} actions, it is agnostic of the MARL algorithm used jointly with function approximators as policies. We used PRAHOM within a proposed design approach to generate an efficient organization and associated OCPV specifications based on the environment, the initial design constraints, and goals.

% Results
We formally prove that PRAHOM effectively constrains the resulting agents' policy space according to the specified organizational constraints. Additionally, we demonstrate that PRAHOM can produce valuable organizational specifications from numerous trained agents' histories. The approach is evaluated in cooperative Atari-like game environments~\citep{perolat2017pool}, where the inferred organizational models closely match or outperform the expected hand-crafted results.

% Outline
The remainder of the paper is organized as follows: Section 2 provides background information on MARL, XAI, and organizational modeling. Section 3 introduces the PRAHOM algorithm and its formal foundations. Section 4 describes the experimental setup and evaluation environments. Section 5 presents and discusses the results of our empirical evaluation. Section 6 reviews related work, and Section 7 concludes the paper and outlines future research directions.

%%%%%%%%%%%%%%%%%%%%%%%%%%%%%%%%%%%%%%%%%%%%%%%%%%%%%%%%%%%%%%%%%%%%%%%%

% JS : A GARDER SI ON EN A BESOIN

% \begin{eqnarray}\label{eq:vcg}
% p_i(\boldsymbol{\hat{v}}) & = &
% \sum_{j \neq i} \hat{v}_j(f(\boldsymbol{\hat{v}}_{-i})) - 
% \sum_{j \neq i} \hat{v}_j(f(\boldsymbol{\hat{v}})) 
% \end{eqnarray}

% \begin{theorem}[Fermat, 1637]\label{thm:fermat}
% No triple $(a,b,c)$ of natural numbers satisfies the equation 
% $a^n + b^n = c^n$ for any natural number $n > 2$.
% \end{theorem}

% \begin{proof}
% A full proof can be found in the supplementary material.
% \end{proof}

% \begin{table}[h]
% \caption{Locations of selected conference editions.}
% \centering
% \begin{tabular}{ll@{\hspace{8mm}}ll} 
% \toprule
% AISB-1980 & Amsterdam & ECAI-1990 & Stockholm \\
% ECAI-2000 & Berlin & ECAI-2010 & Lisbon \\
% ECAI-2020 & \multicolumn{3}{l}{Santiago de Compostela (online)} \\
% \bottomrule
% \end{tabular}
% \end{table}

%%%%%%%%%%%%%%%%%%%%%%%%%%%%%%%%%%%%%%%%%%%%%%%%%%%%%%%%%%%%%%%%%%%%%%%%

\section{Key concepts and Related works}

% présenter une revue "complète" de la littérature connexe dans les domaines de l'apprentissage par renforcement multi-agents (MARL) et de l'IA explicable.

% discuter des algos précédents pour améliorer l'explicabilité dans les systèmes MARL et identifier les lacunes dans la recherche existante.

\subsection{Key Concepts}

In Multi-Agent Systems (MASs), organization is a fundamental concept that impact how agents are coordinating their activities to collaboratively achieve a common goal~\citep{Hubner2002}. In essence, we assume the entity of the organization (we simply call \textbf{organization}) always exists through the running agents interactions even though it may be implicit.
An \textbf{organizational model} specifies (at least partially) the organization whether it is used as medium to describe an explicit known organization in a top-down way, or describing an implicit organization in a bottom-up way. Examples of organizational models are the Agent/Group/Role (AGR) model~\citep{Ferber2004} or more complex ones such as $\mathcal{M}OISE^{+}$~\citep{Hubner2002}. Organizational models can take into account aspects such as structural coordination, dynamic interactions, and the achievement of common objectives~\citep{Ferber2004, Abbas2015}. We call the \textbf{specifications} of an organization, the set of components used in an instance of an organizational model to specify the organization.

We assume an organization in a MAS can be understood regarding the Agent Centered Point of View (ACPV) vs. Organization Centered Point of View (OCPV) and agent's organization awareness vs. unawareness~\citep{Picard2009}.
Typical examples are emergent MAS (ACPV and organization unawareness), coalition based MAS (ACPV and organization awareness), organization based MAS (OCPV and organization awareness), and Agent oriented engineering (OCPV and organization unawareness)~\citep{Picard2009}.
We assume an \textbf{architecture} (also called organizational paradigm) is an abstract organization gathering a range of organizations sharing common characteristics~\citep{Horling2004}. Finally, MAS designing/development methods, have been proposed jointly with organizational models to help designers finding suited specifications of an organization so a MAS can reach a goal efficiently in a environment such as GAIA~\citep{Wooldridge2000}, ADELFE~\citep{Bernon2003} or DIAMOND~\citep{Jamont2005}.

In most \textbf{self/re-organization} mechanisms agents' policies are defined and fixed by the designer from ACPV/OCPV so that an optional emerging/chosen organization allows reaching a global goal~\citep{Picard2009}. We can envision Multi-Agent Reinforcement Learning (MARL) as a particular ACPV mechanism that aims to replace the designer by simultaneously making emerge agents' policies (micro-level) and consequently the emerging organization (macro-level) relying on quantitative feedbacks. In literature, that mechanism is mostly considered to satisfy the need that agents reach efficiently a specific goal with few other considerations. Typical examples include agents' policies modeled as neural networks that are updated using various algorithms such as Deep Q-Network or REINFORCE. In such examples, an emerging implicit organization among agents can converge.

In some environments, such as computers network with highly complex and non-visual interactions, the lack of intuitive comprehension of the environment can make MAS methods difficult to apply to develop a MAS whose organization optimally reach a goal. In such cases, the use of MARL could allow to have sufficiently and non over-fitted trained agents optionally respecting additional arbitrary designer's constraints (coming from a architecture for instance). We think an observer/designer could understand, interpret, and produce the specifications of valuable organizations by translating them into organizational models. For instance determining the individual, social, collective levels described in $\mathcal{M}OISE^{+}$~\citep{Hubner2002}. At least it may give relevant insights for guiding the design process.

\subsection{Related works}

Whether adopting OCPV or ACPV, when designing a MAS, a human designer has to define the logic in agents themselves intending the MAS to reach some specific goals via some kind of emergent or expected organization pattern. It often takes place as an iterative process where designer are proceeding by trial and error. Despite the designer skills, that process may be hard and costly to converge towards a sufficiently estimated successful MAS. That process gets more difficult when the target deployment environment is not easily readable or handleable due to the complexity and internal safety policies such as for company infrastructure networks. Additionally, unexpected emergent phenomena may appear without giving guarantees for safety.
%We refer to that problem as the \textquote{human-based design difficulty}.

Instead of adopting a risky direct empiric approach, a common approach is to rely on a intermediary step by simulating the target deployment system, analogous to \textquote{digital twins}. Indeed, simulation can provide a monitoring framework that leaves room for a safe design process, while having an assessment of the resulting MAS designs. If the simulation is close enough to the target system in terms of fidelity, then we can expect the designed MAS to be transferred to the target system for indeed reaching the goals.
In that respect, creating a MAS that aims to reach a goal in any given environment, firstly focuses on finding a suited MAS design only in the associated simulation. We also, hypothesized the simulation to allow RL techniques and to give, at least, raw description of the current environment and agents states.

One can think of benefitting of a MARL process that would automatically converge to an optimal or sub-optimally sufficient solution as for establishing the rules in agents (called \textbf{policies}) that drive the MAS to the goal. Yet, unlike human-based design where the agent's logic is explicitly specified, trained policies with MARL may be approximated by black box generalizing functions such as Neural Networks. That stresses out the need to address the explainability issue not only for individual agents' policies but for the joint-policy in its entirety through OCPV. We think, this problem can be partially addressed by considering that a joint-policy can be (at least partially) described in terms of the organizational specifications. We refer to research in the processes falling into that broad approach under the term of \textquote{Organization oriented MARL} (OMARL).

We position OMARL at the crossroads of MAS and MARL and can be envisioned as a mixed ACPV/OCPV organization mechanism with no organization awareness. It falls into the broad topic of explainability in AI at an organizational level within MARL. In order to appreciate OMARL positioning in literature, we chose the following boolean keywords formula we think better describe OMARL:
\begin{gather*}
    ("multi" \land "agent") \land ("reinforcement \ learning")\\ \land ("explainability" \lor ("organization" \lor "collective"))
\end{gather*}

We identified few article dealing with a multi-agent vision of explainability. Most notable ones are:

% Collective explainable AI: Explaining cooperative strategies and agent contribution in multiagent reinforcement learning with shapley values
\citep{Heuillet2022} proposes a novel approach to explain cooperative strategies in multiagent reinforcement learning (RL) using Shapley values, a game theory concept used in eXplainable AI (XAI). The study aims to make deep RL more comprehensible and address the need for methods that provide better understanding and interpretability. The experimental results on Multiagent Particle and Sequential Social Dilemmas demonstrate the effectiveness of Shapley values in explaining the rationale behind decisions taken by agents. However, the article also highlights that Shapley values can only provide general explanations about a model and cannot explain specific actions taken by agents. The authors suggest that future work should focus on addressing these limitations. The study's implications extend to areas such as non-discriminatory decision making, ethical and responsible AI-derived decisions, and policy making under fairness constraints.

% Magent: A many-agent reinforcement learning platform for artificial collective intelligence
\citep{Zheng2018} presents MAgent, a platform for many-agent reinforcement learning (MARL) that aims to facilitate research on artificial collective intelligence. MAgent provides a flexible and efficient environment for training and evaluating MARL algorithms, enabling the study of complex multi-agent behaviors. The platform supports various scenarios, including cooperative, competitive, and mixed environments, and offers a high degree of scalability to accommodate a large number of agents. MAgent also includes a comprehensive set of baselines and evaluation metrics to benchmark the performance of MARL algorithms. The research contributes to the advancement of collective intelligence and the development of robust multi-agent systems.

% Self-Organized Group for Cooperative Multi-agent Reinforcement Learning
\citep{Shao2022} introduces a method called Self-Organized Group (SOG) for cooperative multi-agent reinforcement learning. In this approach, a certain number of agents are randomly elected to be conductors, and the corresponding groups are constructed with conductor-follower consensus, allowing the groups to be re-organized at regular intervals. The organized group under the unified command of a conductor is found to embed the multi-agent system with stronger zero-shot generalization ability compared to traditional methods. The SOG method provides strong adaptability to scenarios with varying numbers of agents and varying agent sight. The paper presents this approach as a mechanism to enhance cooperative multi-agent tasks with dynamic characteristics, aiming to improve the adaptability and generalization of multi-agent reinforcement learning systems

% A multi-agent reinforcement learning model of common-pool resource appropriation
\citep{Perolat2017} introduces a model that focuses on common-pool resource appropriation, a multi-agent social dilemma that includes issues such as sustainable use of fresh water, common fisheries, grazing pastures, and irrigation systems. The model emphasizes the importance of trial-and-error learning in addressing the challenges of common-pool resource sustainability and inequality. It explores the emergent behavior of groups of independently learning agents in a partially observed Markov game, shedding light on the relationship between exclusion, cooperation, and sustainability in the context of resource appropriation. The research highlights the potential of deep reinforcement learning in understanding and addressing complex societal and environmental challenges related to common-pool resource management. The paper provides valuable insights into the application of multi-agent reinforcement learning in the context of real-world social dilemmas and resource management

% MARLeME: A Multi-Agent Reinforcement Learning Model Extraction Library
\citep{Kazhdan2020} introduces MARLeME, a library designed to enhance the explainability of Multi-Agent Reinforcement Learning (MARL) systems by approximating them with symbolic models. The library aims to improve the interpretability of MARL systems, which is crucial for understanding the behavior of multiple agents interacting in a shared environment. By providing a means to extract symbolic models from MARL systems, MARLeME contributes to the advancement of explainable AI in the context of multi-agent systems.

% ROMA: Multi-Agent Reinforcement Learning with Emergent Roles
\citep{Wang2020} introduces a role-oriented MARL (Multi-Agent Reinforcement Learning) approach where roles are emergent, and agents with similar roles tend to share their learning and specialize in certain sub-tasks. The framework aims to combine the flexibility and adaptability of MARL with the concept of roles, allowing agents with similar roles to exhibit similar behaviors. The approach is designed to learn specialized, dynamic, and identifiable roles without relying on predefined role structures and behaviors. The research provides a novel perspective on the application of role-oriented MARL in complex multi-agent systems, offering potential advancements in the field of reinforcement learning.

% Promoting Coordination through Policy Regularization in Multi-Agent Deep Reinforcement Learning
\citep{Roy2020} addresses the challenge of inducing coordination between agents in multi-agent reinforcement learning. The research investigates the use of policy regularization to promote inter-agent coordination and discusses two approaches based on inter-agent modeling and synchronized sub-policy selection. The proposed methods are designed to improve cooperative behaviors without relying on explicit communication channels, allowing agents to exhibit coordinated behaviors during testing when acting in a decentralized fashion. The paper presents two policy regularization methods, TeamReg and CoachReg, and evaluates their performance on challenging cooperative multi-agent problems, showing improved results. The research contributes to the advancement of coordination-driven multi-agent approaches in reinforcement learning and provides valuable insights into promoting inter-agent coordination through policy regularization.

% Social Influence as Intrinsic Motivation for Multi-Agent Deep Reinforcement Learning
\citep{Jaques2019} proposes a mechanism for achieving coordination and communication in Multi-Agent Reinforcement Learning (MARL) by rewarding agents for having causal influence over other agents' actions. This causal influence is assessed using counterfactual reasoning, where agents simulate alternate actions to compute their effect on the behavior of other agents. The paper demonstrates that this approach leads to enhanced coordination and communication, as well as more meaningful learned communication protocols. The proposed method is shown to significantly increase the learning curves of the deep reinforcement learning agents, leading to more diversified team behavior and more successful performance of the population as a whole. The paper also highlights that the influence rewards for all agents can be computed in a decentralized way, opening up new opportunities for research in this area.

% Efficient multi-agent reinforcement learning through automated supervision
\citep{Chongjie2008} proposes a unified mechanism for achieving coordination and communication in Multi-Agent Reinforcement Learning (MARL). The approach involves training multiple agents to independently maximize their own individual reward without sharing weights. The paper introduces a method for automated supervision, which enables the agents to learn to coordinate and communicate effectively. This automated supervision mechanism leads to enhanced coordination, communication, and more meaningful learned communication protocols, ultimately improving the learning curves of the deep reinforcement learning agents and the overall performance of the agent population

% A unified framework for reinforcement learning, co-learning and meta-learning how to coordinate in collaborative multi-agent systems
\citep{Tosic2010} presents a comprehensive framework for addressing coordination in collaborative multi-agent systems. The framework integrates reinforcement learning, co-learning, and meta-learning to enable agents to learn how to coordinate effectively. By leveraging this unified approach, the paper aims to enhance the coordination and communication capabilities of multi-agent systems, ultimately improving their overall performance.

Beside, these works explainability in RL is also widely considered in individual agents but are thought to be useful to understand the global organization. Among those works, most notable ones are:

Rule extraction from trained neural networks: involves obtaining human-interpretable rules that approximate the policy of the neural network. Various algorithms have been developed for this task, including decompositional, pedagogical, and eclectics approaches. These algorithms aim to provide comprehensible descriptions of the network's hypothesis that closely approximate its policy. For example, NN2Rules is a decompositional approach that extracts a set of decision rules from the parameters of the trained neural network model, making the decision rules more interpretable. Rule extraction algorithms enable neural networks to justify their classification responses using explainable classification rules, enhancing the transparency and interpretability of the models~\citep{Hailesilassie2016}~\citep{Sato2001}~\citep{Lal2022}.

Specification-Guided Reinforcement Learning (SGRL): addresses the problem of generating an optimal policy in reinforcement learning (RL) with respect to a given task in an unknown environment. Traditionally, the task is encoded in the form of a reward function, which can be cumbersome for long-horizon goals. An alternative approach is to use logical specifications, such as Linear Temporal Logic (LTL) and SpectRL, to define the task, opening the direction of RL from logical specifications. SGRL aims to synthesize control policies for robotic systems and other autonomous agents by leveraging formal logical constructs to express the task or objective. This approach has led to the development of highly performant algorithms that enable RL from logical specifications, enhancing the transparency and trustworthiness of RL systems~\citep{Bansal2022}~\citep{Jothimurugan2023}.

Learning from Logical Specifications: covers the broader area of learning from logical specifications, including the development of reinforcement learning algorithms that leverage the compositional structure of the specification to learn control policies for complex tasks~\citep{Jothimurugan2021}.

\section{Theoretical foundations}

% introduire les fondements théoriques de notre algo:
%   - apprentissage par renforcement multi-agents (MARL)
%   - cadre du modèle organisationnel des systèmes multi-agents (MOISE+) et montrer sa pertinence pour améliorer l'explicabilité de l'IA au sein du MARL.

% détailler l'algorithme PRAHOM en décrivant ses composants et fonctionnalités...

% fournir des preuves formelles de son efficacité dans la restriction des espaces de politiques des agents durant et après l'apprentissage par rapport à des spécifications organisationnelles considérées comme des contraintes.

% fournir des preuves (formelles dans l'idéal) de son efficacité dans la génération de spécifications organisationnelles à partir des historiques des agents entrainés.

We propose the DMO process to integrate MAS organization designing process by translating the $\mathcal{M}OISE^{+}$ organizational model within the formalism used for MARL. Then, we formally describe how agents' policies and training process can be linked to the $\mathcal{M}OISE^{+}$ organizational model.

\subsection{MARL model}

The chosen MARL model is based on the Decentralized Partially Observable Markov Decision Process (Dec-POMDP)~\citep{Oliehoek2016} because it considers multiple agents in a similar MAS fashion. It relies on stochastic processes to model uncertainty of the environment for the changes induced by actions, in received observations, in communication\dots Additionally, unlike Partially Observable Stochastic Games (POSG), the reward function can be common to agents which fosters training for collaborative oriented actions~\citep{Beynier2013}.
A Dec-POMDP $d \in D$ (with $D$ the set of Dec-POMDP) is a 7-tuple $d = (S,\{A_i\},T,R,\{\Omega_i\},O,\gamma)$ , where:
\begin{itemize}
    \item $S = \{s_1, ..s_{|S|}\}$: The set of the possible states.
    \item $A_{i} = \{a_{1}^{i},..,a_{|A_{i}|}^{i}\}$: The set of the possible actions for agent $i$.
    \item $T$ so that $T(s,a,s') = \probP{(s'|s,a)}$ : The set of conditional transition probabilities between states
    \item $R: S \times A \times S \rightarrow \mathbb{R}$: The reward function
    \item $\Omega_{i} = \{o_{1}^{i},..,o_{|\Omega_{i}|}^{i}\}$: The set of observations for agent $ag_i$
    \item $O$ so that $O(s',a,o) = \probP{(o|s',a)}$ : The set of conditional observation probabilities.
    \item $\gamma \in [0,1]$, the discount factor
\end{itemize}

Considering $m$ \textbf{teams} (also referred as \textbf{groups}) each containing several agents among $\mathcal{A}$, we also detail the minimal formalism notation we re-used for solving the Dec-POMDP for a given team $i, 0 \leq i \leq m$ containing $n$ agents~\citep{Beynier2013,Albrecht2024}:

\begin{itemize}

    \item $\Pi$: the set of policies. A \textbf{policy} $\pi \in \Pi, \pi: \Omega \rightarrow A$ deterministically maps an observation to an action. It represents the agent's internal logic;
    \item $\Pi_{joint}$: the set of joint-policies. A \textbf{joint-policy} $\pi_{joint} \in \Pi_{joint}, \pi_{joint}: \Omega^n \rightarrow A^n = \Pi^n$ chooses an action for each agent regarding their respective observation. It can be viewed as a set of the policies used in agents;
    \item $H$: the set of histories. A \textbf{history} over $z \in \mathbb{N}$ steps (where $z$ is generally the maximum number of steps for an episode) is the $z$-tuple $h = ((\omega_{k}, a_{k}) | k \leq z, \omega \in \Omega, a \in A)$
    \item $H_{joint}$: the set of joint-histories. A \textbf{joint-history} over $z \in \mathbb{N}$ steps $h_{joint} \in H_{joint}, h_{joint} = \{h_1,h_2..h_n\}$ is the set of the agents' histories.
    \item $U_{joint,i}(<\pi_{joint,i}, \pi_{joint,-i}>): \Pi_{joint} \rightarrow \mathbb{R}$: gives the \textbf{expected cumulative reward} over a finite horizon (if $\gamma < 1$ or the number of steps in an episode is finite), with $\pi_{joint,i}$ the joint policy for team $i$ and $\pi_{joint,-i}$ all of the other concatenated joint-policies (considered as fixed);
    \item $BR_{joint,i}(\pi_{joint,i}) = argmax_{\pi_{joint,i}}(U(<\pi_{joint,i},\pi_{joint,-i}>))$: gives the \textbf{best response} $\pi_{joint,i}^*$ in the sense that the team cannot change any of the policies in the joint-policy $\pi_{joint,i}^*$ to get a better expected cumulative reward than $U_i^* = U_{joint,i}(<\pi_{joint,i}^*, \pi_{joint,-i}>)$;
    \item $SR_{joint,i}(\pi_{joint,i}, s) = \{\pi_{joint,i} | U(<\pi_{joint,i},\pi_{joint,-i}>) \geq s\}$: gives the \textbf{sufficient response} as the set of joint-policies getting at least $s \in \mathbb{R}, s \leq U_i^*$ as expected cumulative reward.
\end{itemize}

We refer to \textbf{solving} the Dec-POMDP for the team $i$ as finding a joint policy $\pi_{joint,i} \in \Pi_{joint}, \pi_{joint,i} = BR_{joint,i}(\pi_{joint,i})$ that maximize the expected cumulative reward over a finite horizon.
We refer to \textbf{sub-optimally solving} the Dec-POMDP at $s$ expectancy as finding a the joint policies $\pi_{joint,i} \in \Pi_{joint}, \pi_{joint,i} = SR_{joint,i}(\pi_{joint,i}, s)$ that gets the expected cumulative reward over a finite horizon at least at $s \in \mathbb{R}, s \leq U_i^*$.


\subsection{Organizational model}

Among, the existing organizational models \textquote{Agent/Group/Role}~\citep{Ferber2004} and $\mathcal{M}OISE^+$~\citep{Hubner2002} provide a relevant high-level description of the structures and interactions within the MAS. However, we favor $\mathcal{M}OISE^+$ because it provides an advanced formal description for an organization without incompatibilities with MARL, especially for formal description of agents' policies. It takes into account explicitly the social aspects between agents where \textquote{AGR} focuses on the integration of standards oriented towards design. Additionally, it provides a sufficiently detailed vision of organization to be understood at different point of views.
Based on $\mathcal{M}OISE^+$~\citep{Hubner2007} formalism, we only give the minimal elements of the formalism we used for our approach.

\paragraph{\textbf{Organization specifications (OS)}}: $\mathcal{OS} = \langle \mathcal{SS}, \mathcal{FS}, \mathcal{DS} \rangle$, the set of all organization specifications, where $\mathcal{SS}$ are the \textbf{Structural Specifications}, $\mathcal{FS}$ are the \textbf{Functional Specifications}, and $\mathcal{DS}$ are the \textbf{Deontic Specifications}

\paragraph{\textbf{Structural Specifications (SS)}}: $\mathcal{SS} = \langle \mathcal{R}, \mathcal{IR}, \mathcal{G} \rangle$, where:

\begin{itemize}

    \item $\mathcal{R}_{ss}$: the set of all roles (denoted $\rho \in \mathcal{R}$);

    \item $\mathcal{IR}: \mathcal{R} \rightarrow \mathcal{R}$: the inheritance relation between roles ($\mathcal{IR}(\rho_1) = \rho_2$ means $\rho_1$ inherits from $\rho_2$ also denoted $\rho_1 \sqsubset \rho_2$);

    \item $RG \subseteq GR$ the set of root groups, $GR = \langle \mathcal{R}, \mathcal{SG}, \mathcal{L}^{intra}, \mathcal{L}^{inter}, \mathcal{C}^{intra}, \mathcal{C}^{inter}, np, ng \rangle$, the set of all groups, where

          \begin{itemize}

              \item $\mathcal{R} \subseteq \mathcal{R}_{ss}$: the set of non-abstract roles;

              \item $\mathcal{SG} \subseteq \mathcal{GR}$: the set of sub-groups;

              \item $\mathcal{L} = \mathcal{R} \cross \mathcal{R} \cross \mathcal{TL}$: the set of links. A link is a 3-tuple $(\rho_s,\rho_d,t) \in \mathcal{L}$ (also denoted as a predicate $link(\rho_s,\rho_d,t))$, where $\rho_{s}$ is the source role, $\rho_{d}$ is the destination role, and $t \in \mathcal{TL}, \mathcal{TL} = \{acq, com, aut\}$ is the link type;
                    \begin{itemize}
                        \item If $t = acq$ (acquaintance), the agents playing the source role $\rho_{\mathrm{s}}$ are allowed to have a representation of the agents playing the destination role $\rho_{d}$;
                        \item If $t = com$ (communication), the $\rho_{\mathrm{s}}$ agents are allowed to communicate with $\rho_{d}$ agents;
                        \item If $t = aut$ (authority), the $\rho_{\mathrm{s}}$ agents are allowed to have authority on $\rho_{d}$ agents. It requires an acquaintance and communication link.
                    \end{itemize}
              \item $\mathcal{L}^{intra} \subseteq \mathcal{L}$: the set of intra-group links;
              \item $\mathcal{L}^{inter} \subseteq \mathcal{L}$: the set of inter-group links;

              \item $\mathcal{C} = \mathcal{R} \cross \mathcal{R}$: the set of compatibilities. A compatibility is a couple $(a,b) \in \mathcal{C}$ (also denoted $\rho_a \bowtie \rho_b$), means agents playing role $\rho_a \in \mathcal{R}$ can also play role $\rho_b \in \mathcal{R}$;
              \item $\mathcal{C}^{intra} \subseteq \mathcal{C}$: the set of intra-group compatibilities;
              \item $\mathcal{C}^{inter} \subseteq \mathcal{C}$: the set of inter-group compatibilities;

              \item $np: \mathcal{R} \rightarrow \mathbb{N} \times \mathbb{N}$: the relation giving the cardinality of agents adopting a role;
              \item $ng: \mathcal{SG} \rightarrow \mathbb{N} \times \mathbb{N}$: the relation giving the cardinality of each sub-group.

          \end{itemize}

\end{itemize}

\paragraph{\textbf{Functional Specifications (FS)}}: $\mathcal{FS} = \langle \mathcal{SCH}, \mathcal{PO} \rangle$, where:

\begin{itemize}
    \item $\mathcal{SCH} = \langle\mathcal{G}, \mathcal{M}, \mathcal{P}, mo, nm \rangle$: the set of \textbf{social scheme}, where:
          \begin{itemize}
              \item $\mathcal{G}$ is the set of global goal;

              \item $\mathcal{M}$ is the set of mission labels;
              \item $\mathcal{P} = \langle \mathcal{G}, \{\mathcal{G}\}^s, OP, [0,1] \rangle, s \in \mathbb{N}^*$ is the set of plans that builds the tree structure of the goals.
              %
              A plan $p \in \mathcal{P}$ is 4-tuple $p=(g_f,\{g_i\}_{0 \leq i \leq s}, op, p), g_f \in \mathcal{G}, g_i \in \mathcal{G}, op \in OP, OP = \{sequence, choice, parallel\}, p \in [0,1]$, meaning that the goal $g_f$ is achieved if some of the sub-goals $g_i$ are achieved with a success probability $p$ and according to the operator $op$:
              %
              \begin{itemize}
                \item if $op = sequence$, the $g_i$ can only be achieved in the same order sequentially;
                \item if $op = choice$, only one of the $g_i$ has to be achieved;
                \item if $op = parallel$, the $g_i$ can only be achieved sequentially or simultaneously.
              \end{itemize}

              \item $mo: \mathcal{M} \rightarrow \mathbb{P}(\mathcal{G})$: specifies the set of goals a mission is associated to;
              \item $nm: \mathcal{M} \rightarrow \mathbb{N} \times \mathbb{N}$ the cardinality of agents committed for each mission.
          \end{itemize}
    \item $\mathcal{PO}: \mathcal{M} \cross \mathcal{M}$: the set of \textbf{preference order}. A preference order is couple $(m_1, m_2), m_1 \in \mathcal{M}, m_2 \in \mathcal{M}$ (also denoted $m_{1} \prec m_{2}$) meaning that if there is a moment when an agent is permitted to commit to $m_{1}$ and also $m_{2}$, it has a social preference for committing to $m_{1}$.
\end{itemize}

\paragraph{\textbf{Deontic Specifications (DS)}}: $\mathcal{DS} = \langle \mathcal{OBL},\mathcal{PER} \rangle$, the set of deontic specifications, where:

\begin{itemize}
    \item $\mathcal{TC}$: the set of \textbf{time constraints}. A time constraint $tc \in \mathcal{TC}$ specifies a set of periods during which a permission or obligation is valid ($Any \in \mathcal{TC}$ means everytime);
    \item $\mathcal{OBL}: \mathcal{R} \cross \mathcal{M} \cross \mathcal{TC}$: the set of \textbf{obligations}. An obligation is a 3-tuple $(\rho_a,m,tc)$ (aslo denoted $obl(\rho_a,m,tc)$) meaning an agent playing role $\rho_a \in \mathcal{R}$ is obliged to commit on mission $m \in \mathcal{M}$ for a given time constraint $tc \in \mathcal{TC}$;
    \item $\mathcal{PER}$: the set of \textbf{permissions}. A permission is a 3-tuple $(\rho_a,m,tc)$ (aslo denoted $per(\rho_a,m,tc)$) meaning an agent playing role $\rho_a \in \mathcal{R}$ is permitted to commit on mission $m \in \mathcal{M}$ for a given time constraint $tc \in \mathcal{TC}$;
\end{itemize}

\subsection{Towards a DMO process linking MARL and Organizational model}

OMARL deals with all the processes linking MARL and Organization model as for being able to get information from joint-policies to build an organizational model, and constraining possible output joint-policies subset regarding organizational specifications. DMO is a first attempt to propose formally described OMARL processes using the Dec-POMDP for MARL and $\mathcal{M}OISE^+$ for the organizational model.

In a DMO process, we consider solving sub-optimally the Dec-POMDP $d \in D$ for a single team $i$ (comprising $n$ agents in $\mathcal{A}$) at a $s = U_i^* - \delta$ (with $\delta \in \mathbb{R}$) expectancy, we may obtain a set $S\Pi_{joint,s,i} = SR_{d,joint,i}(\pi_{joint,i},s) = \{\pi_{joint,i,s,1}, \pi_{joint,i,s,2} .. \pi_{joint,i,s,d}\}$ with the $S\Pi_{joint,s,i,k} \in \Pi_{joint}$ ($k \in \mathbb{N}, k \leq d$) and $SR_{d,joint,i}(\pi_{joint,i},s)$ gives the sub-optimal joint-policies at $s$ expectancy for the Dec-POMDP $d$ also denoted $SR_{joint,i,s}(d)$. As an example, we may have $d$ different convergent joint-policies reaching a given expected cumulative reward after several training due to non-deterministic parameters in training.

Yet as an extra constraint, we only want joint-policies allowed by the constraining organizational specifications $OS_{cons}$ we denote $S\Pi_{joint,OS_{cons}} = \{\pi_{joint,i,1}, \pi_{joint,i,2} .. \pi_{joint,i,b}\}$ with $b \in \mathbb{N}$ and the $\pi_{joint,k} \in \Pi_{joint}$ ($k \in \mathbb{N}, k \leq b$). Then, one can envision to use the sub-optimal joint-policies that satisfy the specifications $S\Pi_{joint,s,i} \cap S\Pi_{joint,OS_{cons}}$ to infer some associated specifications $OS_{s,i}$ of the implicit resulting organization.

As a summary, we define an OMARL compliant DMO process if it can be described as a relation between the sub-optimal joint-policies set and associated organizational specifications set the following way:

\

\begin{relation}\label{rel:def_dmo}
    dmo: D \times OS_{cons} \rightarrow (S\Pi_{joint,s,i} \cap S\Pi_{joint,OS_{cons}}) \times OS_{s,i}
\end{relation}
Can be understood as for a given problem comprising a environment and a goal to be achieved by agents, and some design constraints; we can minimally get sufficiently successful trained agents that are compliant with given specifications and (at least partial) associated organizational specifications.

\

Defining a DMO process is challenging due to the lack of common ground to describe the links between organizational specifications and policies.

In \autoref{proof:to_extended_dmo}, we extended the \autoref{rel:def_dmo} DMO definition into another convenient form in \autoref{rel:def_extended_dmo}.

\

\begin{relation}\label{rel:def_extended_dmo}
    $\phantom{X}$ dmo: D \times OS_{cons} \rightarrow (SR_{joint,i,s}[D] \cap sop[OS_{cons}]) \times pos[SR[D] \cap sop[OS_{cons}]]
\end{relation}

\begin{proof}\label{proof:to_extended_dmo}

    Let $dmo_{2}: D \times OS_{cons} \rightarrow (SR_{joint,i,s}[D] \cap sop[OS_{cons}]) \times \allowbreak pos[SR[D] \cap sop[OS_{cons}]]$ \\

    By definition, $SR_{joint,i,s}[D] \cap sop[OS_{cons}] = (S\Pi_{joint,s,i} \cap S\Pi_{joint,OS_{cons}})$ \\

    By definition, $pos[\Pi_{joint, OS_{cons}}] = OS_{s,i}$

    Hence, $dmo_{2}: D \times OS_{cons} \rightarrow (S\Pi_{joint,s,i} \cap S\Pi_{joint,OS_{cons}}) \times OS_{s,i}$

    Hence, $dmo_{2} \subseteq dmo$
\end{proof}

\noindent Where $pos$ and $sop$ are provided respectively in \autoref{rel:pos} and \autoref{rel:sop}.

\

\begin{relation}\label{rel:pos}
    pos: S\Pi_{joint,s,i} \rightarrow OS_{s,i}
\end{relation}
Represents how we can infer organizational specifications out of \textquote{trained} joint-policies.

\

\begin{relation}\label{rel:sop}
    sop: OS \rightarrow (S\Pi_{joint, OS_{cons}})
\end{relation}
Represents how we can determine the joint-policies that satisfy some organizational specifications

\subsection{Details of PRAHOM algorithm}

Continuing the previous decomposition in two activities, we detailed our process to get the specifications out of the agents' policies; and the process to get the joint-policies satisfying given specifications.

As a synthesis of these two processes, we propose the PRAHOM as a DMO compliant process presented in \autoref{alg:PRAHOM}.

\RestyleAlgo{ruled}
\SetKwComment{Comment}{// }{}

\begin{algorithm}[hbt!]
    \caption{Partial Action-based $\mathcal{M}OISE^+$ Identification DMO (PRAHOM)}\label{alg:PRAHOM}

    \KwData{$d \in D$, the Dec-POMDP to solve}
    \KwData{$ep_{max} \in \mathbb{N}$, the maximum number of episodes}
    \KwData{$step_{max} \in \mathbb{N}$, the maximum number of steps per episodes}
    \KwData{$s \in \mathbb{R}$, the cumulative reward expectancy}
    \KwData{$\pi_{joint}$, the joint-policies to be trained}
    \KwData{$OS_{cons} \in OS_{cons}$, the design specifications to respect}
    \KwData{$Solv: \Pi \times A \times O \times \mathbb{R} \rightarrow \Pi$, the MARL algorithm for updating a single policy in order to solve the Dec-POMDP}

    \KwResult{$(s\pi_{joint,s,i} \in S\Pi_{joint,s,i}, os_{s,i} \in OS_{s,i})$, the sub-optimal policies associated organization specifications}

    $s\pi_{joint,s,i} = PRAHOM-osh-training(s,i)$

    $os_{s,i} = PRAHOM-hos(s\pi_{joint,s,i})$

\end{algorithm}

\paragraph{\textbf{Infering OS from joint-policies}}

Seeking to implement a process represented by $pos$, we faced two main problems

First, for the agent $i$'s policy $\pi_{i} \in \pi_{joint}$, a minimal step to go further in the analyze of the policy is to know the $\{(\omega_k, a_k) \in \pi_{i}\}$ couples (the rules the agent follow). But it is not obvious to know them because of the explainability issues in black box policy approximation models such as NN-based ones.

To address that problem, rather than using directly agents' policy, we use the history as it may be built with observed resulting actions when observations are received during a series of test episodes. Indeed, for a given policy $\pi \in \Pi$ the associated history is by definition $h \in h_{joint} = ((\omega_k,a_k) | k \in \mathbb{N})$ and the $(\omega_k,a_k) \in \pi$. Taking into account several agents, joint-history can be seen as an approximation of the joint-policy that we refer to as reconstructed joint-policy $R\Pi_{joint} \subset \Pi_{joint}$. A reconstructed policy with a history is $r\pi_i = {(\omega, a) \in h | h \in H}$;

The second problem is linked to the difficulty to to infer information related to organizational specifications $OS$ such as roles, groups, goals\dots out of joint-histories (or reconstructed joint-policy).
To solve that difficulty we associate each action in MARL with some organization specifications as a \textquote{many to many} relation. It sets up a first frame for identifying organizational specifications in played action series. We address that problem in the remainder of this section.

We propose some relations between $\mathcal{M}OISE^+$ specifications and joint-histories. Their premises comes from noticing some specifications in the $\mathcal{M}OISE^+$ organizational model can be obviously mapped to subsets of actions from a single sub-optimal joint-policy, we propose the following relations:

\begin{itemize}
    \item $rh: \mathcal{R} \rightarrow \mathcal{P}(H)$: gives the histories associated with a role
    \item $sgh: \mathcal{SG} \rightarrow \mathcal{P}(H)$: gives the histories associated with a subgroup
    \item $lh: \mathcal{L} \rightarrow \mathcal{P}(H)$: gives the histories associated with an intra link
    \item $gh: \mathcal{G} \rightarrow \mathcal{P}(H)$: gives the histories associated with a global goal
    \item $mh: \mathcal{M} \rightarrow \mathcal{P}(H)$: gives the histories associated with a mission label
\end{itemize}

From these relation, we propose to rely on the previous relations through \emph{PRAHOM-hos} presented in \autoref{alg:PRAHOM-hos} that is consistent with $pos$.

\RestyleAlgo{ruled}
\SetKwComment{Comment}{// }{}

\begin{algorithm}[hbt!]
    \caption{PRAHOM-hos}\label{alg:PRAHOM-hos}
    \KwData{$\pi_{joint}$, the joint-policies to be analyzed}
    \KwResult{$os \in OS$, the associated organization specifications}

    $h_{joint} = testEpisode(\pi_{joint})$

    \ForEach{$h_{joint,i} \in h_{joint}$}{
        \ForEach{$h_{j} \in h_{joint,i}$}{

            $r_j = \langle r, determineRoles(h_{j}) \rangle$

            $sg_j = determineSubgroups(h_{j})$

            $l_j = determineIntraLinks(h_{j})$

            $g_{ind,j} = determineIndGoals(h_{j})$

            $m_{ind,j} = determineIndMissions(h_{j})$

            $p_{ind,j} = determineIndPlans(h_{j}, m_{ind,j}, g_{ind,j})$

        }

        $g = determineGoals(g, h_{joint,i}, {g_{ind,j}})$

        $m = determineMissions(m, h_{joint,i}, {m_{ind,j}})$

        $p = determinePlans(p, h_{joint,i}, {p_{ind,j}})$

        $mo = determineMissionToGoals(mo, g_i, m_i)$
    }

    $ng = determineSgCard(\{r_j\}_i)$

    $np = determineRoleCard(\{r_j\}_i)$

    $c = determineIntraCompatibilities(\{r_j\}_i)$

    $per = determinePermissions(\{r_j\}_i, m)$

    $obl = determineObligations(\{r_j\}_i, m)$

    $nm = determineAgentCardPerMission(mo)$

    $ss = \langle \{r_j\}_i, \{sg_j\}_i, \{l_j\}_i, \emptyset, c, \emptyset, np, ng \rangle$

    $fs = \langle {g, m, p, mo, nm}, \emptyset \rangle$

    $ds = \langle per, obl \rangle$

    $os = \langle ss, fs, ds \rangle$

\end{algorithm}

As we have only one group, we do not consider the inter-links and inter-compatibilities. Additionally as a simplification, we consider only one social scheme. As a general remark, the \textquote{determine*} relations are not all fully described and rely on custom implementation (cf. \autoref{gym-wrapper} for further information).

PRAHOM-hos first look at the individual level.
First, it tries to figure out the roles played by agents ($determineRoles$) by sampling history sub-sequences $h \in H$ and comparing with known history sub-sequences whose we know the associated role via $ra$.
Then, it tries to get the intra links the agent has with other agents. ($determineIntraLinks$ and $determineInterLinks$) via $lh$.
Then, it tries to known the subgroup where the agent is ($determineSubgroup$) via $sgh$.
Then, it tries to know the individual goals the agent aim to ($determineIndGoals$) via $gh$.
Then, it tries to know the mission the agent is committed to ($determineIndMissions$) via $mh$.
Finally, it tries to build a part of the plan the agent is playing at its individual level ($determineIndPlans$) via $ph$ and the inferred individual goal and the inferred individual mission.

Then PRAHOM-hos look at the social and collective level.
At the end of each joint-policy we try to reinforce a global view of the goals $g$, missions $m$, plans $p$, and the knowledge of the mission to goal $mo$; with the partially inferred information at the individual level.

At the end of the algorithm, it tries to synthesize the knowledge infered until know to have better view of the number of agent per subgroup $ng$, the agent cardinality per mission $nm$, the role cardinality $np$, the compatibilities between roles $c$, the permissions and obligations $per$ and $obl$.


\paragraph{\textbf{Constraining joint-policies satisfying OS}}

\

We consider a given MARL algorithm that iteratively converges towards a joint-policy so that each agent's policy is updated at each step until a finite horizon. In \autoref{proof:jpc_to_ac} we proved that constraining the available action set for each agent to the action sets authorized by the organization specifications at each step of the MARL training; implies constraining the converged joint-policies to the ones that satisfy the given organization specifications. We used that result, to setup our $sop$ used jointly with MARL learning through \emph{PRAHOM-osh-training} presented in \autoref{alg:PRAHOM-osh-training}.

\RestyleAlgo{ruled}
\SetKwComment{Comment}{// }{}

\begin{algorithm}[hbt!]
    \caption{PRAHOM-osh-training}\label{alg:PRAHOM-osh-training}
    \KwData{$d \in D$, the Dec-POMDP to solve}
    \KwData{$ep_{max} \in \mathbb{N}$, the maximum number of episodes}
    \KwData{$step_{max} \in \mathbb{N}$, the maximum number of steps per episodes}
    \KwData{$s \in \mathbb{R}$, the cumulative reward expectancy}
    \KwData{$\pi_{joint}$, the joint-policies to be trained}
    \KwData{$OS_{cons} \in OS_{cons}$, the design specifications to respect}
    \KwData{$solv: \Pi \times A \times O \times \mathbb{R} \rightarrow \Pi$, the MARL algorithm for updating a single policy in order to solve the Dec-POMDP}
    \KwResult{$s\pi_{joint} \in S\Pi_{joint,s,i}$, the sub-optimal joint-policies that satisfy the design specifications}

    $forbidden = \Pi_{joint} \setminus (DMO-pos^{-1}(OS_{cons}))$

    $s\pi_{joint} = \{\}$

    \ForEach{$1 \leq ep \leq ep_{max}$}{

        $h_{joint} = \{\emptyset^n\}$

        $\omega_{0} = \emptyset$

        $a_{0} = \emptyset$

        \ForEach{$1 \leq step \leq step_{max}$}{

            \ForEach{$\pi_i \in \pi_{joint}$}{

                $\omega_{step} = getObservation()$

                $r_{step} = getReward()$

                $\pi_{i,step} = Solv(\pi_{i,step-1}, \omega_{step-1}, a_{step-1}, r_{step})$  \Comment*[r]{Update the individual policy}

                $A_{step} = \{a | (\omega_{step}, a) \in \pi, \pi \in \pi_{joint},\pi_{joint} \notin forbidden, \{(\omega', a') \in h\} \in \pi\}$ \Comment*[r]{Restricted available actions}

                $a_{step} = \pi_{i,step}(\omega_{step})$ \Comment*[r]{Choose next action}

                $h_{joint,i} = \langle h_{joint,i}, (\omega_{step}, a_{step}) \rangle$ \Comment*[r]{Update the history}

            }

        }
    }

    $s\pi_{joint} = \{\pi_{joint}\}$


\end{algorithm}

\emph{PRAHOM-osh-training} first converts the $OS_{cons} \in OS_{cons}$ design specifications into a $forbidden$ set of the non-authorized joint-policies. It uses a non-described reverse \emph{PRAHOM-hos} algorithm to achieve it.

Then, \emph{PRAHOM-osh-training} uses a given MARL algorithm to update the agents' individual policies with previous action, observation and resulting reward. Then, it first computes the authorized actions set $A_{step}$ according to the current history $h_{joint,i}$. Then, an action is chosen among authorized actions. That action $a_{step}$ is added in history to be used for updating the agent's policy in the next step.
Finally, considering only one iteration of episode, it returns a sub-optimal joint-policy $\pi_{joint}$ satisfying $OS_{cons}$
We can note the $OS_{cons}$ could forbid the MARL algorithm to provide a joint-policy that reaches a expected cumulative reward.


\section{Methodology}

% décrire la méthodologie pour évaluer PRAHOM
%   - configuration expérimentale: les environnements de jeu coopératifs PettingZoo de type Atari avec les paramètres spécifiques utilisés, présentation et utilisation de "PRAHOM Wrapper" pour intégrer PRAHOM dans les environnements de jeu.
%   - mesures utilisées pour évaluer l'efficacité de notre algo et fournir un aperçu du processus de collecte et d'analyse des données.

We selected 4 environments to assess our approach: \textquote{Predator-prey with communication}~\citep{Lowe2017},
%\textquote{DeepMind MuJoCo Multi-Agent Soccer Environment}~\citep{Liu2019},
\textquote{Google Research Football Environment}~\citep{Kurach2020},\textquote{Pistonball}~\citep{Terry2021}, and \textquote{Knights Archers Zombies}~\citep{Terry2021}.

% We selected 9 environments to assess our approach: \textquote{Predator-prey with comm}, \textquote{simple reference}, \textquote{simple speaker listener}, \textquote{DeepMind MuJoCo Multi-Agent Soccer Environment}, \textquote{Pistonball}, \textquote{Waterworld}, \textquote{Emtombed: Cooperative}, \textquote{Cooperative Pong}, \textquote{Knights Archers Zombies}, \textquote{Moving Company}, \textquote{Captain-sailor}.

We applied our approach in three cases:
\begin{itemize}
    \item No training specifications (NTS)
    \item Partially constraining training specifications (PTS)
    \item Fully constraining training specifications (FTS)
\end{itemize}

To evaluate the impact of our approach in training we payed attention to the following criteria: convergence time, stability, global performance, change sensibility. Results are presented in Table~\ref{tab:training_AOMEA_results}.

\begin{table}[t!]

    \centering

    \begin{tblr}{colspec={llll},rows={m},measure=vbox,stretch=-1}

        \textbf{Environment} & \textbf{PTS/NTS} & \textbf{PTS/FTS} & \textbf{Perf. stability \\ (avg. / max)} \\

        \hline

        { PPL }
        & { 4.7 }
        & { 1.3 }
        & { 0.9 } \\

        \hline[dashed]

        { PPY }
        & { 6.3 }
        & { 2.2 }
        & { 0.78 } \\

        \hline[dashed]

        { KAZ }
        & { 4.0 }
        & { 1.1 }
        & { 0.42 } \\

        \hline[dashed]

        { CYB }
        & { 12 }
        & { 3.3 }
        & { 0.36 } \\


    \end{tblr}

    \caption{View of the AOMEA approach impact during training}

    \label{tab:training_AOMEA_results}

\end{table}


We also took into account criteria after training: roles, links, compatibilities, social schemes. Results are presented in Table~\ref{tab:trained_AOMEA_results}

\begin{table}[t!]

    \centering

    \begin{tblr}{colspec={llll},rows={m},measure=vbox,stretch=-1}

        \textbf{Environment} & \textbf{Roles} & \textbf{Links} & \textbf{Global performance} \\

        \hline

        { 1 }
        & {  }
        & {  } \\
        & {  } \\

        \hline[dashed]

        { 2 }
        & {  }
        & {  } \\
        & {  } \\

        \hline[dashed]

        { 3 }
        & {  }
        & {  } \\
        & {  } \\

        \hline[dashed]

        { 4 }
        & {  }
        & {  } \\
        & {  } \\

        \hline[dashed]

        { 5 }
        & {  }
        & {  } \\
        & {  } \\

    \end{tblr}

    \caption{View of the OOMARL approach impact after training}

    \label{tab:trained_OOMARL_results}

\end{table}


\paragraph{\textbf{OOMARL Gym Wrapper}\label{gym-wrapper}} We developed a \textquote{Gym Wrapper} to help in automate the setting up of OOMARL for a given Gym Environment. It enables linking actions with $\mathcal{M}OISE^+$ specifications, define the training specifications and provide functions to extract the resulting sub-optimal raw organizational specifications.


\section{Results}

% présenter les résultats de nos évaluations empiriques
%   - performances de l'algorithme PRAHOM vis-à-vis des contraintes dans l'espaces des politiques des agents et la génération de spécifications organisationnelles à partir des historiques des agents entrainés

% discuter des implications de nos résultats et mettre en évidence toutes les observations ou tendances notables observées au cours des expériences.

\section{Discussion}

% discussion complète des implications de nos résultats et de leur signification plus large dans le contexte de MARL et du XAI

% explorer les explications potentielles des résultats observés

% discuter des limites de notre algo et proposons des pistes de recherche future pour remédier à ces limites.

\section{Conclusion}

% résumer nos contributions et nos principales conclusions.

% importance d’améliorer l’explicabilité de l’IA dans les systèmes MARL et souligner l’impact potentiel de PRAHOM sur l’avancement du domaine.

% proposer des remarques finales et décrire les orientations des recherches futures afin d’explorer et d’affiner davantage notre algo.



%%%%%%%%%%%%%%%%%%%%%%%%%%%%%%%%%%%%%%%%%%%%%%%%%%%%%%%%%%%%%%%%%%%%%%%%

%%% Use this environment to include acknowledgements (optional).
%%% This will be omitted in doubleblind mode.

\begin{ack}
    This work was supported by \emph{Thales Land Air Systems} within the framework of the \emph{Cyb'Air} chair and the \emph{AICA IWG}.
\end{ack}

%%%%%%%%%%%%%%%%%%%%%%%%%%%%%%%%%%%%%%%%%%%%%%%%%%%%%%%%%%%%%%%%%%%%%%%%

%%% Use this command to include your bibliography file.

\bibliography{references}

\newpage

\section*{Annexes}

\subsection*{Action constraining during training implies result joint-policy constraining}
\begin{proofoutline}\label{proof:jpc_to_ac}

    We provide an overview of our approach to constrain the possible policies of trained agents through a simple abstract example. While this example is somewhat artificial, it serves to illustrate the general principle of our approach and gives insights into why it is indeed effective in constraining policies.

    \noindent Let's consider an example with this initial configuration:

    \begin{itemize}
        \item $d=\langle S,A,T,R,\Omega, O, \gamma \rangle \in D$, the Dec-POMDP to solve (i.e maximizing $R$);
        \item $\mathcal{A}, |\mathcal{A}| = n \in \mathbb{N}$, the  $n$ agents involved in the Dec-POMDP;
        \item $s \in \mathbb{R}$, the cumulative reward expectancy to reach;
        \item $\pi_{joint} \in \Pi_{joint}, \allowbreak \pi_{joint} = \{\pi_1..\pi_n\}, \pi_k \in \Pi (k \leq n)$, the joint-policy to update;
        \item $ep_{max}$, the maximum number of episodes;
        \item $step_{max}$, the maximum number of steps per episode;
        \item $u_{marl}: \Pi_{joint} \times H_{joint} \times R_{joint} \rightarrow \Pi_{joint}$, the MARL algorithm that uses the joint-reward and joint-history to update a joint-policy;
    \end{itemize}
    %
    \noindent We assume some organizational specifications are defined, applied to agents, and associated with matching history subsets (at least from a theoretical point view):
    \item $os \in \mathcal{OS}$, the organizational specifications containing: $\mathcal{R}$, the roles that agents may be constrained to; $\mathcal{M}$, the missions that agents may be committed to; $\mathcal{OBL}$, the obligations indicating whether an agent playing a role $\rho \in \mathcal{R}$ is obligated to commit on mission $m \in \mathcal{M}$. In this example, we do not consider permissions; $rh: \mathcal{R} \rightarrow \mathcal{P}(H)$: gives the expected history subset for a role; $mh: \mathcal{M} \rightarrow \mathcal{P}(H)$: gives the expected history subset for a mission; $da: \mathcal{OBL} \rightarrow \mathcal{P}(A)$: gives the agents constrained to a role and obligated to commit on a mission.

    \

    \noindent We suppose there exists a set of joint-policies $S\Pi_{joint} = \{s\pi_{joint,1}.. s\pi_{joint,d}\} \allowbreak (d \in \mathbb{N})$, that enables reaching at least the $s$ cumulative reward expectancy.

    \noindent We suppose there exists a set of joint-policies $O\Pi_{joint} = \{o\pi_{joint,1}.. o\pi_{joint,d'}\} (d' \in \mathbb{N})$ that satisfy the applied organizational specifications, so that an agent playing role $\rho \in \mathcal{R}$ and obligated to commit on mission $m \in \mathcal{M}$ should have its policy $o\pi_{joint,i} \ (i \leq d')$ to generate any matching history $h \in (rh(\rho) \cap mh(m))$.

    \noindent We assume there exists a non-empty set of joint-policies $P\Pi = S\Pi \cap O\Pi \allowbreak = \{p\pi_{joint,1}..p\pi_{joint,q}\}, q \in \mathbb{N}$ that both reach at least the $s$ cumulative reward expectancy and satisfy the organizational specifications $os$.

    \

    Based on these assumptions and initial data, we apply PRAHOM on the first iterations and generalize it to indefinite number of iteration, in order to determine whether it enables building a policy that does belong to $P\Pi$. Although all constraints integration modes are effective in constraining policies, in this example, we chose the $correct\_policy$ mode to apply our algorithm for it offers a clear way to understand the proof outline.
    We consider the first episode. Initially, a constrained policy $\pi_{joint} = \pi_{joint,c}$ built from the initial policy $\pi_{joint,init,0}$ and the observable policy constraint $c\pi_{joint}$.

    At first step, agents have an empty history $h_{joint} = \langle \rangle$, null rewards $rh_{joint} = \langle (0)^n \rangle $. Thus, the initial policies $\pi_{joint,0} \in \Pi_{joint}$ are not updated for now. Receiving the initial observations for each agents $\omega_{joint,0} \in \Omega_{joint}$, agents choose their respective next actions $a_{joint,0}$ using their policies $\allowbreak \pi_{joint,0}$. The observations and actions are stored in history $h_{joint} \allowbreak = \allowbreak \langle \allowbreak (\omega_{joint,0}, \allowbreak a_{joint,0}) \rangle$. Then, the action are applied, hence generating new observations $\omega_{joint,1}$ and rewards $r_{joint,1}$ stored in $rh_{joint}$ for the next step.

    % \

    % At second step, agents have the current history $h_{joint} = \langle (\omega_{joint,0}, a_{joint,0}) \rangle$, and rewards $rh_{joint} = \langle (0)^n, r_{joint,1} \rangle$. Thus, the policies are updated accordingly $\pi_{joint,1} = u_{marl}(\pi_{joint,0},h_{joint},rh_{joint})$. From received observation $\omega_{joint,1}$, agents choose their next actions $a_{joint,1}$ using their policies $\pi_{joint,1}$. The observations and actions are stored in the joint-histories $h_{joint} = \langle (\omega_{joint,0},a_{joint,0}), (\omega_{joint,1},a_{joint,1}) \rangle$. Then, the actions are applied, hence generating new observations $\omega_{joint,2}$ and rewards $r_{joint,2}$ stored in $rh_{joint}$ for the next step.

    Generalizing until the $p < step_{max}$ step, agents have the current history $h_{joint} = \langle (\omega_{joint,0}, \allowbreak a_{joint,0}), (\omega_{joint,1}, a_{joint,1})..(\omega_{joint,p-1}, a_{joint,p-1}) \rangle$, and rewards $rh_{joint} = \langle (0)^n, r_{joint,1}, r_{joint,2}..r_{joint,p} \rangle$. Thus, the policies are updated accordingly $\pi_{joint,p} = u_{marl}(\pi_{joint,p-1},h_{joint},rh_{joint})$. From received observation $\omega_{joint,p}$, agents choose their next actions $a_{joint,p}$ using their policies $\pi_{joint,p}$. The observations and actions are stored in history $h_{joint} \allowbreak = \allowbreak \langle \allowbreak (\omega_{joint,0},a_{joint,0}), \allowbreak (\omega_{joint,1},a_{joint,1}), (\omega_{joint,2},a_{joint,2})..(\omega_{joint,p},a_{joint,p}) \rangle$. Then, the actions are applied, hence generating new observations $\omega_{joint,p+1}$ and rewards $r_{joint,p+1}$ stored in $rh_{joint}$ for the next step.

    \

    When $p = step_{max}$, the episode is finished, we assume the cumulative reward reaches at least $s$. The generated histories are $h_{joint} = \langle (\omega_{joint,0}, \allowbreak a_{joint,0}) .. \allowbreak (\omega_{joint,step_{max}},a_{joint,step_{max}}) \rangle$. Throughout all steps, it is built using the $\pi_{joint,k}, \allowbreak k < step_{max}$ and $\pi_{joint,k} \allowbreak = \allowbreak \{sample(c\pi_{joint}(\omega_{joint})) \allowbreak \ \allowbreak if \allowbreak \ \allowbreak \omega_{joint} \in Dom(c\pi_{joint}) \allowbreak \ \allowbreak else \allowbreak \ \allowbreak \pi_{joint,k}(\omega_{joint,init,k})\}$.
    %
    By definition, $\langle (\omega_{joint,j}, \allowbreak sample(c\pi_{joint}(\omega_{joint,j}))), \allowbreak j < step_{max} \rangle, \allowbreak \omega_{joint,j} \allowbreak \in \Omega_{joint}$, the joint-history generated using the observable constrained policy satisfy the organizational specifications. Thus, the policy represented by $sample(c\pi_{joint}(\omega_{joint,j}))$ belongs to $O\Pi_{joint}$ and possibly $S\Pi_{joint}$.
    %
    By construction, $\langle (\pi_{joint,init,k}(\omega_{joint,k}))_{k < step_{max}} \rangle$, the joint-history generated using the initial policy trained over $k$ steps so that the cumulative reward reach at least $s$. Thus, the policy $\pi_{joint,init,step_{max}}$ belong to $S\Pi_{joint}$.

    Considering several episodes, $s$ is reach for a policy in $\allowbreak S\Pi_{joint,ep_{max},step_{max}}$. Moreover, since a history $h_{joint}$ belongs, at least, to histories generated by a policy in $O\Pi_{joint}$. Thus, $\pi_{joint,ep_{max},step_{max}} \in S\Pi \cap O\Pi, \pi_{joint,ep_{max},step_{max}} \in P\Pi$. So, built policies indeed satisfy organizational specifications while reaching sufficient cumulative reward expectancy.

    \

    As for the other constraint integration modes, we briefly outline the main ideas supporting why it is also effective as well as the $correct\_policy$ mode:
    $\mathbf{correct}$, corrects the action according to an observable policy constraint after the initial policy has chosen it. Without other consideration, it can be modeled by building a constrained policy $\pi_c$ encompassing both the observable policy constraint and the initial policy. Therefore, this goes back to the $correct\_policy$ case;
    $\mathbf{penalize}$, adjust the reward comparing chosen action by the policy and the expected ones according to an observable policy constraint. We assume that the policy can be updated according to rewards so that it asymptotically tends be equal to any constrained policy formed from the current policy and the observable policy constraint. Therefore, it also goes back to the $correct\_policy$ case.

\end{proofoutline}

\end{document}
%%%%%%%%%%%%%%%%%%%%%%%%%%%%%%%%%%%%%%%%%%%%%%%%%%%%%%%%%%%%%%%%%%%%%%
