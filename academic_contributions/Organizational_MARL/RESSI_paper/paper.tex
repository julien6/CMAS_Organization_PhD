\documentclass[conference]{IEEEtran}
\IEEEoverridecommandlockouts
% The preceding line is only needed to identify funding in the first footnote. If that is unneeded, please comment it out.
\usepackage{amsmath,amssymb,amsfonts}
\usepackage{algorithmic}
\usepackage{csquotes}
\usepackage{graphicx}
\usepackage[inline]{enumitem}
\usepackage{tabularx}
\usepackage{caption}
\usepackage[T2A,T1]{fontenc}
\usepackage[french]{babel}
\captionsetup{font=it}
\usepackage{ragged2e}
\usepackage{hyperref}
\usepackage{footmisc}
\usepackage{booktabs}
\usepackage{smartdiagram}
\usepackage{textcomp}
\usepackage{xcolor}
\def\BibTeX{{\rm B\kern-.05em{\sc i\kern-.025em b}\kern-.08em
    T\kern-.1667em\lower.7ex\hbox{E}\kern-.125emX}}
\usepackage{cite}

\usepackage{etoolbox}
\patchcmd{\thebibliography}{\section*{\refname}}{}{}{}

\setlength{\extrarowheight}{2.5pt}

% \renewcommand{\arraystretch}{1.7}

\newcommand{\old}[1]{\textcolor{orange}{#1}}
\newcommand{\rem}[1]{\textcolor{red}{#1}}
\newcommand{\todo}[1]{\textcolor{orange}{\newline \textit{\textbf{TODO:} #1}} \newline \newline }

\makeatletter
\newcommand{\linebreakand}{%
  \end{@IEEEauthorhalign}
  \hfill\mbox{}\par
  \mbox{}\hfill\begin{@IEEEauthorhalign}
}
\makeatother


\begin{document}

\title{Vers un Processus de Conception de Systèmes Multi-Agents de Cyber-défense\\
    % {\footnotesize \textsuperscript{Note}}
    % \thanks{Identify applicable funding agency here. If none, delete this.}
}

% \IEEEaftertitletext{\vspace{-1\baselineskip}}

\author{

    \IEEEauthorblockN{Julien Soulé}
    \IEEEauthorblockA{\textit{Thales Land and Air Systems, BU IAS}}
    %Rennes, France \\
    \IEEEauthorblockA{\textit{Univ. Grenoble Alpes,} \\
        \textit{Grenoble INP, LCIS, 26000,}\\
        Valence, France \\
        julien.soule@lcis.grenoble-inp.fr}

    \and

    \IEEEauthorblockN{Jean-Paul Jamont\IEEEauthorrefmark{1}, Michel Occello\IEEEauthorrefmark{2}}
    \IEEEauthorblockA{\textit{Univ. Grenoble Alpes,} \\
        \textit{Grenoble INP, LCIS, 26000,}\\
        Valence, France \\
        \{\IEEEauthorrefmark{1}jean-paul.jamont,\IEEEauthorrefmark{2}michel.occello\}@lcis.grenoble-inp.fr
    }

    % \and

    % \IEEEauthorblockN{Michel Occello}
    % \IEEEauthorblockA{\textit{Univ. Grenoble Alpes,} \\
    % \textit{Grenoble INP, LCIS, 26000,}\\
    % Valence, France \\
    % michel.occello@lcis.grenoble-inp.fr}

    % \and

    \linebreakand
    \hspace{-0.5cm}
    \IEEEauthorblockN{Paul Théron}
    \IEEEauthorblockA{
        % \textit{Co-leader du RTG 152 OTAN, 1er Président de l'AICA IWG} \\
        \hspace{-0.5cm}
        \textit{AICA IWG} \\
        \hspace{-0.5cm}
        La Guillermie, France \\
        %lieu-dit Le Bourg, France \\
        \hspace{-0.5cm}
        paul.theron@orange.fr}

    \and
    \hspace{0.5cm}
    \IEEEauthorblockN{Louis-Marie Traonouez}
    \IEEEauthorblockA{
        \hspace{0.5cm}
        \textit{Thales Land and Air Systems, BU IAS} \\
        \hspace{0.5cm}
        Rennes, France \\
        \hspace{0.5cm}
        louis-marie.traonouez@thalesgroup.com}}


\maketitle

\begin{abstract}

    % Contexte
    Un ensemble d'agents cyber-défenseurs autonomes déployés au plus près des points d'entrée sensibles d'un système hôte constituent un Système Multi-Agent de Cyberdéfense. Ces agents peuvent founir une réponse adaptée face à la complexité et l'évolutivité des Cyber-attaques tout en satisfaisant les contraintes de déploiement du système hôte.
    % Problème
    Cependant, la conception empirique d'un tel système déployable et opérationel sur le système cible requiert un cout important.
    % Contribution
    Notre approche vise à combiner un processus d'apprentissage par renforcement avec les spécifications de l'organisation afin de faciliter le processus de conception vers un système aux performances optimales.

\end{abstract}

% \begin{IEEEkeywords}
% sécurité décentralisée, sécurité autonome, système multi-agent, réorganisation, auto-organisation
% \end{IEEEkeywords}

\section{Introduction}

% Le contexte
% Alors que de plus en plus de réseaux et d'appareils connectés sont utilisés, en particulier dans l'<<~Internet of Things~>> (IoT) et l'<<~Internet of Battle Things~>> (IoBT), le besoin de leur propre sécurité est devenu un défi très important. Les systèmes tels que les capteurs sans fil ou les véhicules autonomes ont une surface d'attaque accrue, car ils offrent de nouveaux vecteurs d'attaque pour corrompre, détruire et propager de nouvelles cyber-attaques sur les réseaux connectés.
%\cite{ccdc_army_research_laboratory_internet_2017}.

L'agent AICA (Autonomous Intelligent Cyber-defence Agent) théorisé par l'\textquote{AICA IWG}\footnote{Ce groupe de travail (voir \url{https://www.aica-iwg.org/}) s'appuie sur les résultats du \textit{Research Task Group IST-152} de l'OTAN qui a travaillé sur le concept des \textquote{Intelligent, Autonomous and Trusted Agents for Cyber Defense and Resilience}.} doit être déployé sur des systèmes hôtes pour détecter, identifier et caractériser des anomalies/attaques, élaborer et piloter l’exécution de contre-mesures et dialoguer avec l'extérieur.
% À cette fin, il est conçu comme proactif, discret et capable d’apprendre.
Une vision de l'AICA comme un Système Multi-Agent de Cyber-défense (SMAC) déléguant les différents aspects de Cyber-défense à des agents autonomes, vise notamment à répondre à l'augmentation de la surface d'attaque des systèmes \textquote{IoT} présentant des failles dans certains nœuds\cite{kott2018autonomous}. De plus, elle prend en compte les problèmes liés à l'ouverture, le passage à l'échelle et l'autonomie du système hôte.

% Problème
Etant au coeur d'un SMAC, l'organisation impacte la façon dont les agents interagissent entre eux et avec leur environnement pour atteindre un objectif de Cyber-défense. Ainsi, la conception d'un AICA de type SMAC peut être vu comme un problème d'optimisation consistant à trouver l'organisation satisfaisant les contraintes de déploiement de l'environnement et favorisant la meilleure performance dans l'atteinte d'un objectif de Cyber-défense.
Néanmoins, la complexité et le manque de lisbilité de certains systèmes en réseaux rend difficile et non-sûre une recherche empirique de l'organisation optimale directement sur le système à proteger.

% Contribution
Dans ce papier, la section II présente une travail en cours utilisant le \textquote{Multi-Agent Reinforcement Learning} (MARL) avec le modèle organisationel $\mathcal{M}OISE^{+}$\cite{Hubner2002} afin de spécifier l'organisation du SMAC entrainé et guider son entrainement.
La section III discute de son application sur un scénario d'un essaim de drones issu du CAGE Challenge 3\cite{cage_challenge_3_announcement2022}.

\section{Approche théorique de conception}


\section{Application sur un essaim de drones}


\section{Conclusion}
Un SMA de cyber-défense déployé sur un système hôte en réseau permettrait de relever les défis liés à la complexité et la rapidité de cyber-attaques. Notre étude donne un aperçu d'organisations possibles respectant des objectifs de cyber-défense et des contraintes de l'environnement de déploiement d'un SMA de cyber-défense.
Elle souligne aussi le besoin de définir un cadre théorique et technique spécifique à l'organisation d'un SMA de cyber-défense dans un environnement réseau. Un tel cadre permettra d'explorer, d'évaluer et de tirer des recommandations sur l'organisation d'un SMA de cyber-défense que nous valoriserons pour le développement d'un agent AICA.

\section*{Références}

\bibliographystyle{abbrv}
%\bibliographystyle{IEEEtran}

\bibliography{local_references}

\end{document}
