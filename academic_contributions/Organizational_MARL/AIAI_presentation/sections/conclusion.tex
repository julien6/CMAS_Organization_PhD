\section{Conclusion and perspectives}
\begin{frame}{Conclusion and perspectives}
    {}

    \begin{block}{Contributions}
        MAS methodological works rely on the designer's knowledge to design a suited MAS organization but do not provide automatic or assisted ways to determine relevant organizational mechanisms.
        %solely from the design requirements and the global goal.
        MARL techniques have been successfully applied to train agents automatically to reach the given goal without explicit characterization of emergent collective strategies.
        AOMEA's originality is to augment a MARL process with an explicit organizational model towards a methodological purpose to address these issues. We first exposed how AOMEA is intended to be used in MAS engineering as an additional tool to assist in the design process.
        Then, we explained the AOMEA's theoretical core with links between Dec-POMDP and the $\mathcal{M}OISE^+$ through the \emph{PRAHOM} process.
        Furthermore, we implemented the \emph{PRAHOM PettingZoo wrapper} as a Proof of Concept for practically applying AOMEA and we showed it enables getting some organizational specifications that satisfy the design constraints and allow achieving the given goal.
        Finally, we applied our approach in four PettingZoo environments to assess the impact on and after training. The obtained performance results show to be comparable to known ones showing our approach to be viable.

        Even though \emph{PRAHOM} is agnostic of the MARL algorithm because it uses agents' histories to infer organizational specifications, reconstructing agents' collective behaviors a posteriori may be difficult. Indeed, a major perspective for improving \emph{PRAHOM} is to go further with supervised and non-supervised learning techniques in addition to empirical statistical approaches for identifying valuable organizational specifications from joint-histories. Moreover, it is worth investigating recent works in MARL techniques such as hierarchical learning because they already seek to characterize emergent strategies throughout learning.
    \end{block}

    \begin{alertblock}{Perspectives}

    \end{alertblock}

\end{frame}
