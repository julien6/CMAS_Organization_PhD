\AtBeginSection[]{
    \begin{frame}
        \frametitle{}
        \tableofcontents[currentsection]
    \end{frame}
}

%%%%%%%%%%%%%%%%%%%%%%%%%%%%%%%%%%%%

\section{Introduction}
\begin{frame}[allowframebreaks]{Introduction}

    \begin{block}{MAS paradigm for complex \& distributed problems}
        \begin{itemize}
            \item \textbf{paradigm}: missions delegated to agents achieved through cooperative mechanisms~\cite{Raileanu2023};
            \item \textbf{benefits}: handle conflicting goals, parallel computation, system robustness, scalability\dots
        \end{itemize}
    \end{block}

    \begin{block}{\textbf{Organization}: key for MAS designing}
        \begin{itemize}
            \item \textbf{coordination}: how to collaboratively achieve a common goal~\cite{Hubner2007};
            \item \textbf{dynamic \& uncertain environments}: flexible runtime behavior to adapt~\cite{Kathleen2020};
        \end{itemize}
    \end{block}

    \begin{block}{Design methods}
        \begin{itemize}
            \item \textbf{approach + organizational model}: methods provide protocols relying on designers' experience to hand-craft agents' \textbf{policies} so resulting MAS achieve goals;
            \begin{itemize}
                \item Examples: \emph{GAIA}~\cite{Wooldridge2000,Cernuzzi2014}, \emph{ADELFE}~\cite{Mefteh2015}, or \emph{DIAMOND}~\cite{Jamont2015}, \emph{KB-ORG}~\cite{Sims2008}
            \end{itemize}
          \item \textbf{simulation to reality}: 1) safe \& efficient MAS design in high fidelity simulated environment; \quad 2) transfer to real environment to perform adequately~\cite{Schon2021}.
        \end{itemize}

        \quad $\rightarrow$ \textbf{Iterative process proceeding by trial and error}

    \end{block}

    \begin{alertblock}{Current design limitations}
        \begin{itemize}
            \item prerequisite for designers' experience;
            \item costly due to environment limitations or designers limitations;
            \item \dots
        \end{itemize}
    \end{alertblock}

    \begin{exampleblock}{Autonomous Intelligent Cyberdefense Agents~\cite{Kott2023} (AICA)}
        \begin{itemize}
            \item develop cooperative Cyberdefense agents deployed in highly complex computer networks
            \item lack of visual and intuitive comprehension of the networked environments such as company networks
        \end{itemize}
    \end{exampleblock}

    \begin{alertblock}{Main gaps}
        \begin{enumerate}
            \item Finding automatically suited agents' policies satisfying design constraints
            \item Making explicit the organizational mechanisms that emerge from trained agents for the design process.
        \end{enumerate}
    \end{alertblock}

    \begin{alertblock}{Intended contributions}
        To address these issues, we introduce AMOEA, a MAS design approach whose underlying idea is to link a given MARL process with an organizational model that links the on-training agents' policies with explicit organizational specifications. It can be viewed as a tool for engineering to automatically generate relevant exploitable organizational specifications only regarding the performance in achieving the given goal and the design constraints. For the designer, the obtained organizational specifications are insights into the organizational mechanisms to set up for developing a MAS that meets performance requirements.
    \end{alertblock}


\end{frame}