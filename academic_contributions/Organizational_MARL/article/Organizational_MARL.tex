\documentclass[conference]{IEEEtran}
\IEEEoverridecommandlockouts
% The preceding line is only needed to identify funding in the first footnote. If that is unneeded, please comment it out.
\usepackage{amsmath,amssymb,amsfonts}
\usepackage{algorithmic}
\usepackage{graphicx}
\usepackage[inline, shortlabels]{enumitem}
\usepackage{tabularx}
\usepackage{caption}
\usepackage{titlesec}
\usepackage[T2A,T1]{fontenc}
\usepackage[english]{babel}
\captionsetup{font=it}
\usepackage{ragged2e}
\usepackage{hyperref}
\usepackage{pifont}
\usepackage{footmisc}
\usepackage{multirow}

% --- Tickz
\usepackage{physics}
\usepackage{amsmath}
\usepackage{tikz}
\usepackage{mathdots}
\usepackage{yhmath}
\usepackage{cancel}
\usepackage{color}
\usepackage{siunitx}
\usepackage{array}
\usepackage{multirow}
\usepackage{amssymb}
\usepackage{gensymb}
\usepackage{tabularx}
\usepackage{extarrows}
\usepackage{booktabs}
\usetikzlibrary{fadings}
\usetikzlibrary{patterns}
\usetikzlibrary{shadows.blur}
\usetikzlibrary{shapes}

% ---------

\usepackage{pdfpages}
\usepackage{booktabs}
\usepackage{csquotes}
\usepackage{lipsum}  
\usepackage{arydshln}
\usepackage{smartdiagram}
\usepackage[inkscapeformat=png]{svg}
\usepackage{textcomp}
\usepackage{tabularray}\UseTblrLibrary{varwidth}
\usepackage{xcolor}
\def\BibTeX{{\rm B\kern-.05em{\sc i\kern-.025em b}\kern-.08em
    T\kern-.1667em\lower.7ex\hbox{E}\kern-.125emX}}
\usepackage{cite}
\usepackage{amsmath}
\newcommand{\probP}{\text{I\kern-0.15em P}}
\usepackage{etoolbox}
\patchcmd{\thebibliography}{\section*{\refname}}{}{}{}

\setlength{\extrarowheight}{2.5pt}

% \renewcommand{\arraystretch}{1.7}

% \setlength{\extrarowheight}{2.5pt}
% \renewcommand{\arraystretch}{0.2}
% \renewcommand{\arraystretch}{1.7}

% --------------
\titleclass{\subsubsubsection}{straight}[\subsection]

\newcounter{subsubsubsection}[subsubsection]
\renewcommand\thesubsubsubsection{\thesubsubsection.\arabic{subsubsubsection}}
\renewcommand\theparagraph{\thesubsubsubsection.\arabic{paragraph}} % optional; useful if paragraphs are to be numbered

\titleformat{\subsubsubsection}
  {\normalfont\normalsize\bfseries}{\thesubsubsubsection}{1em}{}
\titlespacing*{\subsubsubsection}
{0pt}{3.25ex plus 1ex minus .2ex}{1.5ex plus .2ex}

\makeatletter
\renewcommand\paragraph{\@startsection{paragraph}{5}{\z@}%
  {3.25ex \@plus1ex \@minus.2ex}%
  {-1em}%
  {\normalfont\normalsize\bfseries}}
\renewcommand\subparagraph{\@startsection{subparagraph}{6}{\parindent}%
  {3.25ex \@plus1ex \@minus .2ex}%
  {-1em}%
  {\normalfont\normalsize\bfseries}}
\def\toclevel@subsubsubsection{4}
\def\toclevel@paragraph{5}
\def\toclevel@paragraph{6}
\def\l@subsubsubsection{\@dottedtocline{4}{7em}{4em}}
\def\l@paragraph{\@dottedtocline{5}{10em}{5em}}
\def\l@subparagraph{\@dottedtocline{6}{14em}{6em}}
\makeatother

\setcounter{secnumdepth}{4}
\setcounter{tocdepth}{4}
% --------------


\newcommand{\before}[1]{\textcolor{red}{#1}}
\newcommand{\after}[1]{\textcolor{green}{#1}}

\newcommand{\old}[1]{\textcolor{orange}{#1}}
\newcommand{\rem}[1]{\textcolor{red}{#1}}
\newcommand{\todo}[1]{\textcolor{orange}{\newline \textit{\textbf{TODO:} #1}} \newline \newline }

\makeatletter
\newcommand{\linebreakand}{%
  \end{@IEEEauthorhalign}
  \hfill\mbox{}\par
  \mbox{}\hfill\begin{@IEEEauthorhalign}
}
\makeatother




% ---------------------------


\begin{document}

\title{A Step Towards Automated Design of Multi-Agent System Organization using Reinfocement Learning\\
    % {\footnotesize \textsuperscript{Note}}
    % \thanks{Identify applicable funding agency here. If none, delete this.}
}

% \IEEEaftertitletext{\vspace{-1\baselineskip}}

\author{

    \IEEEauthorblockN{Julien Soulé}
    \IEEEauthorblockA{\textit{Thales Land and Air Systems, BU IAS}}
    %Rennes, France \\
    \IEEEauthorblockA{\textit{Univ. Grenoble Alpes,} \\
        \textit{Grenoble INP, LCIS, 26000,}\\
        Valence, France \\
        julien.soule@lcis.grenoble-inp.fr}

    \and

    \IEEEauthorblockN{Jean-Paul Jamont\IEEEauthorrefmark{1}, Michel Occello\IEEEauthorrefmark{2}}
    \IEEEauthorblockA{\textit{Univ. Grenoble Alpes,} \\
        \textit{Grenoble INP, LCIS, 26000,}\\
        Valence, France \\
        \{\IEEEauthorrefmark{1}jean-paul.jamont,\IEEEauthorrefmark{2}michel.occello\}@lcis.grenoble-inp.fr
    }

    % \and

    % \IEEEauthorblockN{Michel Occello}
    % \IEEEauthorblockA{\textit{Univ. Grenoble Alpes,} \\
    % \textit{Grenoble INP, LCIS, 26000,}\\
    % Valence, France \\
    % michel.occello@lcis.grenoble-inp.fr}

    % \and

    \linebreakand

    \hspace{-0.5cm}
    \IEEEauthorblockN{Paul Théron}
    \IEEEauthorblockA{
        \hspace{-0.5cm}
        \textit{AICA IWG} \\
        \hspace{-0.5cm}
        La Guillermie, France \\
        \hspace{-0.5cm}
        %lieu-dit Le Bourg, France \\
        paul.theron@orange.fr}

    \and

    \hspace{0.5cm}
    \IEEEauthorblockN{Louis-Marie Traonouez}
    \IEEEauthorblockA{
        \hspace{0.5cm}
        \textit{Thales Land and Air Systems, BU IAS} \\
        \hspace{0.5cm}
        Rennes, France \\
        \hspace{0.5cm}
        louis-marie.traonouez@thalesgroup.com}}


\maketitle


\begin{abstract}

    % context
    Multi-Agent Reinforcement Learning has been successfully applied in various contexts to generate behaviors allowing agents to collaboratively achieve a goal in a non-fully observable environment. These behaviors are approximated functions that map observation to action allowing agents are to maximize the cumulative reward over an episode.
    % problem
    Yet, from a designer point of view, those trained behaviors raise explainability and safety issues because they do not give human-understandable or exploitable specifications of their organizational aspects such as individual, social or collective levels.
    % This particularly concerns black box models in deep-learning or random forest.
    % hypothesis / contribution
    This paper aims to define the organizational explainability problem formally. Then, we expose our approach to extract the organization specifications out of trained agents behaviors.
    % results
    We applied our approach in three manageable cooperative Atari games likely to have emergent organization among agents. Resulting organization specifications are indeed consistent with human expectations.

\end{abstract}

\begin{IEEEkeywords}
    multi-agent systems, reinforcement learning, organization, design
\end{IEEEkeywords}

\section{Introduction}

% Context
In Multi-Agent Systems (MASs), organization is a fundamental concept that impact how agents are coordinating their activities to collaboratively achieve a common goal\cite{Hubner2002}. In essence, we assume the entity of the organization (we simply call \textbf{organization}) always exists through the running agents interactions even though it may be implicit.
An \textbf{organizational model} specifies (at least partially) the organization whether it is used as medium to describe an explicit known organization in a top-down way, or describing an implicit organization in a bottom-up way. Examples of organizational models are the Agent/Group/Role (AGR) model\cite{Ferber2004} or more complex ones such as Moise$^+$\cite{Hubner2002}. Organizational models can take into account aspects such as structural coordination, dynamic interactions, and the achievement of common objectives\cite{Ferber2004, Abbas2015}. We call the \textbf{specifications} of an organization, the set of components used in an instance of an organizational model to specify the organization.

We assume an organization in a MAS can be understood regarding the Agent Centered Point of View (ACPV) vs. Organization Centered Point of View (OCPV) and agent's organization awareness vs. unawareness\cite{Picard2009}. Typical examples are emergent MAS (ACPV and organization unawareness), coalition based MAS (ACPV and organization awareness), organization based MAS (OCPV and organization awareness), and Agent oriented engineering (OCPV and organization unawareness)\cite{Picard2009}.
We assume an \textbf{architecture} (also called organizational paradigm) is an abstract organization gathering a range of organizations sharing common characteristics\cite{Horling2004}. Finally, MAS designing/development methods, have been proposed jointly with organizational models to help designers finding suited specifications of an organization so a MAS can reach a goal efficiently in a environment such as GAIA\cite{Wooldridge2000}, ADELFE\cite{Bernon2003} or DIAMOND\cite{Jamont2005}.

In most \textbf{self/re-organization} mechanisms agents' behaviors are defined and fixed by the designer from ACPV/OCPV so that an optional emerging/chosen organization allows reaching a global goal\cite{Picard2009}. We can envision Multi-Agent Reinforcement Learning (MARL) as a particular ACPV mechanism that aims to replace the designer by simultaneously making emerge agents' behaviors (micro-level) and consequently the emerging organization (macro-level) relying on quantitative feedbacks. In literature, that mechanism is mostly considered to satisfy the need that agents reach efficiently a specific goal with few other considerations. Typical examples include agents' behaviors modeled as neural networks that are updated using various algorithms such as Deep Q-Network or REINFORCE. In such examples, an emerging implicit organization among agents can converge.

In some environments, such as computers network with highly complex and non-visual interactions, the lack of intuitive comprehension of the environment can make MAS methods difficult to apply to develop a MAS whose organization optimally reach a goal. In such cases, the use of MARL could allow to have sufficiently and non over-fitted trained agents optionally respecting additional arbitrary designer's constraints (coming from a architecture for instance). We think an observer/designer could understand, interpret, and produce the specifications of valuable organizations by translating them into organizational models. For instance determining the individual, social, collective levels described in Moise$^+$\cite{Hubner2002}. At least it may give relevant insights for guiding the design process.

% Problem
The idea of benefitting of the particularly adaptive and general MARL mechanism to approximate a suited MAS organization and producing associated specifications, requires to link the MARL training of a set of behaviors in a bidirectional way with a MAS organizational model. For instance, a hierarchy described in a MAS model would constrain the possible behaviors to get ultimately trained in MARL. Reversely, a set of trained behaviors could be described in an organizational model, thus indicating resemblance with known MAS organization architecture.

% Contribution/Hypothesis
This paper first aims to formalize the previously described idea through a formal model. That model aims to unify the concepts and links between agents' behaviors, their training with MARL, architecture, and the Moise$^+$ organizational model. Relying on that model, we exposes our approach to generate an efficient organization and associated specifications based on the environment, the initial agents' behaviors models, the design constraints, and the global goal.

% Results
We applied our approach to three simple Atari games involving several agents that must converge to a specific organization to achieve a goal efficiently. Obtained organization specifications are exploitable and coherent with human expectations.

The remainder of the article is built as following.
In section II, we give an overview of related works to our intended contribution as for MARL/RL, specifications, MAS organizational models and mechanisms. It shows the originality of our idea.
In section III, we introduce a formal model to properly formalize our idea by unifying MAS and MARL related concepts.
In section IV, we present our approach to help designers in designing a suited organization regarding environment, goal, agents' behavior model and design constraints.
In section V, we discuss results obtained after applications of our approach in three simple cooperative Atari games.
In section VI, we conclude on the viability and relevance of our model and approach and we highlight limitations to overcome and future works as well.

\section{Related works}

\begin{itemize}
    
    \item Rule extraction from trained neural networks
    \item specification-Guided Reinforcement Learning
    
    \begin{itemize}
        \item **Compositional Reinforcement Learning**: This approach involves the development of a compositional learning approach, called DiRL, that leverages the specification to decompose the task into high-level planning and reinforcement learning[1][8].
        \item **Formal Specifications in Reinforcement Learning**: The use of formal specifications in reinforcement learning is a related topic, emphasizing the importance of applying existing techniques for reinforcement learning from logical specifications and encouraging researchers to utilize specification languages like LTL and SpectRL[2][5].
        \item **Learning from Logical Specifications**: This topic covers the broader area of learning from logical specifications, including the development of reinforcement learning algorithms that leverage the compositional structure of the specification to learn control policies for complex tasks[6][9].
        \item **Reward Generation and Reinforcement Learning**: Work on reward generation and reinforcement learning from logical specifications is also related, focusing on the generation of reward machines from temporal specifications and the application of reinforcement learning algorithms for language-based specifications[7].
        
        % Citations:
        % [1] https://www.researchgate.net/publication/365928557_Specification-Guided_Reinforcement_Learning
        % [2] https://keyshor.github.io/teaching/aaai_tutorial/proposal.pdf
        % [3] https://www.youtube.com/watch?v=EiKLpW31Mls
        % [4] https://link.springer.com/chapter/10.1007/978-3-031-22337-2_29
        % [5] https://keyshor.github.io/teaching/aaai_tutorial/
        % [6] https://www.cis.upenn.edu/~alur/NeurIPS21.pdf
        % [7] https://www.cis.upenn.edu/~alur/Kishor-dissertation.pdf
        % [8] https://arxiv.org/abs/2106.13906v3
        % [9] https://openreview.net/pdf?id=ion6Lo5tKtJ
    \end{itemize}

\end{itemize}


\section{A general formal approach of organization}

In previous section, we proposed to view some MAS organization related concepts through the Dec-POMDP formalism to integrate MAS domain within CybMASFM. Yet, that static description of organization and mechanisms is not sufficient for designing organization.

We based our vision of organization considering four typical cases of organization between OCPV/ACPV and awareness/unawareness of organization\cite{Picard2009}:

\begin{itemize}
    \item Emergent MAS: Organization only exists as an emergent phenomena and is the result of interpretation between agents interactions over time. This is a bottom-up process.
    \item Coalition based MAS: Interactions between agents in thought within agents themselves but global picture of the organization is shared among all agents. Indeed, organization results from cooperation schemes. This is a bottom-up process.
    \item Agent-Oriented Engineering: Organization is thought during designing but agents are not aware for they are hard coded within agents themselves. This is a top-down process.
    \item Organization oriented MAS: Organization is thought commonly between all agents as they share the same organization representation.
\end{itemize}

Here we aim to unify the designing processes of organization in MAS through the CybMASFM model.

\section{Two levels of MAS engineering to compute from MARL trained MAS}

The engineering of a Multi-Agent System must take into account two levels:

\begin{itemize}
    \item Questions at the Multi-Agent System level (system-centric approach)
    \begin{itemize}
        \item Number of agents, what heterogeneity?
        \item What is the common medium (Environment) shared by the agents?
        \item What communication mechanisms are available to agents?
        \item What are the communication languages, ontologies, interaction protocols used by the agents?
        \item What is the organization within which the agents operate? How is it established?
        \item How do the agents coordinate their actions? How to ensure coherent operation?
    \end{itemize}

    \item Agent level questions (agent-centered approach)
    \begin{itemize}
        \item What does an agent represent? What actions should be encapsulated in an agent?
        \item How do agents represent the environment and organization in which they operate?
        \item How do agents handle interactions with other agents?
        \item What is the internal structure of the agents?
    \end{itemize}
\end{itemize}

\section{Case studies}


\section{Conclusion}


\section*{References}

% \bibliographystyle{abbrv}
\bibliographystyle{IEEEtran}

\bibliography{references}

\end{document}
