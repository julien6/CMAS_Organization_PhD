\documentclass[runningheads]{llncs}

\usepackage[T1]{fontenc}
\usepackage{graphicx}
%\usepackage{color}
%\renewcommand\UrlFont{\color{blue}\rmfamily}

\usepackage{amsmath,amssymb,amsfonts}
\usepackage[inline, shortlabels]{enumitem}
\usepackage{tabularx}
\usepackage{caption}
\usepackage{listings}
% \usepackage{titlesec}
\usepackage[english]{babel}
\captionsetup{font=it}
\usepackage{ragged2e}
\usepackage[hyphens]{url}
\usepackage{hyperref}
\usepackage{xurl}
\usepackage{pifont}
\usepackage{footmisc}
\usepackage{multirow}
\usepackage{enumitem}
\usepackage{algorithm2e}
\usepackage{float}
\usepackage{listings}
\usepackage{xcolor}

\definecolor{codegreen}{rgb}{0,0.6,0}
\definecolor{codegray}{rgb}{0.5,0.5,0.5}
\definecolor{codepurple}{rgb}{0.58,0,0.82}
\definecolor{backcolour}{rgb}{0.95,0.95,0.92}
 
\lstdefinestyle{mystyle}{
    backgroundcolor=\color{backcolour},   
    commentstyle=\color{codegreen},
    keywordstyle=\color{magenta},
    numberstyle=\tiny\color{codegray},
    stringstyle=\color{codepurple},
    basicstyle=\footnotesize,
    breakatwhitespace=false,         
    breaklines=true,                 
    captionpos=b,                    
    keepspaces=true,                 
    numbers=left,                    
    numbersep=5pt,                  
    showspaces=false,                
    showstringspaces=false,
    showtabs=false,                  
    tabsize=2
}
 
\lstset{style=mystyle}

% --- Tickz
\usepackage{physics}
\usepackage{amsmath}
\usepackage{tikz}
\usepackage{mathdots}
\usepackage{yhmath}
\usepackage{cancel}
\usepackage{color}
\usepackage{siunitx}
\usepackage{array}
\usepackage{multirow}
\usepackage{amssymb}
\usepackage{gensymb}
\usepackage{tabularx}
\usepackage{extarrows}
\usepackage{booktabs}
\usetikzlibrary{fadings}
\usetikzlibrary{patterns}
\usetikzlibrary{shadows.blur}
\usetikzlibrary{shapes}

% ---------
% \usepackage{titlesec}
\usepackage{pdfpages}
\usepackage{booktabs}
\usepackage{csquotes}
\usepackage{lipsum}  
\usepackage{arydshln}
\usepackage{smartdiagram}
\usepackage[inkscapeformat=png]{svg}
\usepackage{textcomp}
\usepackage{tabularray}\UseTblrLibrary{varwidth}
\usepackage{xcolor}
\def\BibTeX{{\rm B\kern-.05em{\sc i\kern-.025em b}\kern-.08em
    T\kern-.1667em\lower.7ex\hbox{E}\kern-.125emX}}
\usepackage{cite}
\usepackage{amsmath}
\newcommand{\probP}{\text{I\kern-0.15em P}}
\usepackage{etoolbox}
\patchcmd{\thebibliography}{\section*{\refname}}{}{}{}

\setlength\tabcolsep{0.5pt}

% \renewcommand{\arraystretch}{1.7}

% \setlength{\extrarowheight}{2.5pt}
% \renewcommand{\arraystretch}{0.2}
% \renewcommand{\arraystretch}{1.7}

% --------------
% \titleclass{\subsubsubsection}{straight}[\subsection]

% \newcounter{subsubsubsection}[subsubsection]
% \renewcommand\thesubsubsubsection{\thesubsubsection.\arabic{subsubsubsection}}
% \renewcommand\theparagraph{\thesubsubsubsection.\arabic{paragraph}} % optional; useful if paragraphs are to be numbered

% \titleformat{\subsubsubsection}
%   {\normalfont\normalsize\bfseries}{\thesubsubsubsection}{1em}{}
% \titlespacing*{\subsubsubsection}
% {0pt}{3.25ex plus 1ex minus .2ex}{1.5ex plus .2ex}

% \makeatletter
% \renewcommand\paragraph{\@startsection{paragraph}{5}{\z@}%
%   {3.25ex \@plus1ex \@minus.2ex}%
%   {-1em}%
%   {\normalfont\normalsize\bfseries}}
% \renewcommand\subparagraph{\@startsection{subparagraph}{6}{\parindent}%
%   {3.25ex \@plus1ex \@minus .2ex}%
%   {-1em}%
%   {\normalfont\normalsize\bfseries}}
% \def\toclevel@subsubsubsection{4}
% \def\toclevel@paragraph{5}
% \def\toclevel@paragraph{6}
% \def\l@subsubsubsection{\@dottedtocline{4}{7em}{4em}}
% \def\l@paragraph{\@dottedtocline{5}{10em}{5em}}
% \def\l@subparagraph{\@dottedtocline{6}{14em}{6em}}
% \makeatother

% \setcounter{secnumdepth}{4}
% \setcounter{tocdepth}{4}
% --------------

\newcommand{\before}[1]{\textcolor{red}{#1}}
\newcommand{\after}[1]{\textcolor{green}{#1}}

\newcommand{\old}[1]{\textcolor{orange}{#1}}
\newcommand{\rem}[1]{\textcolor{red}{#1}}
\newcommand{\todo}[1]{\textcolor{orange}{\newline \textit{\textbf{TODO:} #1}} \newline \newline }



\newcounter{relation}
\setcounter{relation}{0}
\renewcommand{\therelation}{\arabic{relation}}
\newcommand{\relationautorefname}{Relation}

\newenvironment{relation}[1][]{%
    \refstepcounter{relation}%
    \noindent \raggedright \textit{\textbf{Relation. \therelation}} \hfill$}
{%
$ \hfill \phantom{x}

}

\newcounter{proof}
\setcounter{proof}{0}
\renewcommand{\theproof}{\arabic{proof}}
\newcommand{\proofautorefname}{Proof}

\renewenvironment{proof}[1][]{
    \refstepcounter{proof}
    \noindent \raggedright \textit{\textbf{Proof. \theproof}}

    \setlength{\leftskip}{1em}

}
{

\
\setlength{\leftskip}{0pt}
}

% --------------------------------
%             DOCUMENT
% --------------------------------

\begin{document}
\title{An Organization-oriented MARL Algorithm for Explainability and Specifications-guided Learning}
%
%\titlerunning{Abbreviated paper title}
% If the paper title is too long for the running head, you can set
% an abbreviated paper title here
%
% \author{Julien Soulé\inst{1}\orcidID{0000-1111-2222-3333} \and
% Jean-Paul Jamont\inst{1}\orcidID{1111-2222-3333-4444} \and
% Michel Occello\inst{1}\orcidID{2222--3333-4444-5555} \and
% Louis-Marie Traonouez\inst{2}\orcidID{2222--3333-4444-5555} \and
% Paul Théron\inst{3}\orcidID{2222--3333-4444-5555}}

%%% Double blind review %%%
\author{}
\authorrunning{}
\institute{}
% \author{Julien Soulé\inst{1} \and
% Jean-Paul Jamont\inst{1} \and
% Michel Occello\inst{1} \and
% Louis-Marie Traonouez\inst{2} \and
% Paul Théron\inst{3}}
% %
% \authorrunning{J. Soulé et al.}
% % First names are abbreviated in the running head.
% % If there are more than two authors, 'et al.' is used.
% %
% \institute{Univ. Grenoble Alpes, Grenoble INP, LCIS, 26000, Valence, France
%     \email{\{julien.soule, jean-paul.jamont, michel.occello\}@lcis.grenoble-inp.fr}
%     \and
%     Thales Land and Air Systems, BL IAS, Rennes, France
%     \email{louis-marie.traonouez@thalesgroup.com}
%     \and
%     AICA IWG, La Guillermie, France \\
%     \email{paul.theron@orange.fr}
% }


\maketitle              % typeset the header of the contribution

% MAS have been succefully
% For many MAS, the organization has become a critical success factor. 
% Several related methods exist to design MAS. 
% However, these methods are ...
% To enhance the quality and effectiveness of ... this paper presents an assisted approach for MAS Organization Engineering (AMOEA). 
% AMOEA guides the designer of...
% 1 phrase par contrib/key point

\begin{abstract}

    This paper addresses the challenge of making Multi-Agent Reinforcement Learning (MARL) explicitly usable for both constraining the agents' training regarding some specifications and explaining trained agents in \textquote{human-readable} manner. While previous studies have focused on a single agent for guiding its training or explicating its behavior, a multi-agent context shows the need to elucidate the implicit cooperation among multiple agents in MARL systems during and after training. We propose a novel algorithmic approach leveraging the $\mathcal{M}OISE^+$ Organizational Model, called \emph{Partial Relation between Agents' History and Organizational Model} (PRAHOM). Our algorithm's underlying principle consists in linking organizational specifications, such as roles or missions, to their respective agents' histories characterizing their behaviors. PRAHOM serves dual purposes: constraining the learning process based on organizational constraints and inferring organizational specifications from trained agents' histories. Empirical evaluations conducted in cooperative Atari-like game environments validate the effectiveness of our algorithm, showing alignment with hand-crafted expectations and superior performance in some scenarios.
    %This work seeks to contribute to AI explainability in MARL systems by offering a principled framework for understanding emergent cooperative behaviors.
    By bridging the gap between individual agent decision-making and cooperation, PRAHOM aims to enhance transparency and interpretability in complex Multi-Agent Systems.
    \keywords{Multi-Agent Reinforcement Learning \and Explainability \and Organizational Models \and Cooperative Intelligence \and Assisted engineering}

    % % context
    % Multi-Agent Systems (MAS) have been successfully applied in industry for their ability to address complex, distributed problems, especially in IoT-based systems.
    % Their efficiency in achieving given objectives and meeting design requirements is strongly dependent on the MAS organization during the engineering process of an application-specific MAS. To design a MAS that can achieve given goals, available methods rely on the designer's knowledge of the deployment environment.
    % % Yet, in some cases, the deployment environment is not easily readable or handleable due to the complexity and may lead to unexpected emergent phenomena raising safety concerns.
    % However, high complexity and low readability in some deployment environments make the application of these methods to be costly or raise safety concerns.
    % % That stresses out the need for methodological works for assisted MAS design that could be addressed with collective AI techniques.
    % % hypothesis / contribution
    % In order to ease the MAS organization design regarding those concerns, we introduce an original Assisted MAS Organization Engineering Approach (AOMEA). AOMEA relies on combining a Multi-Agent Reinforcement Learning (MARL) process with an organizational model to suggest relevant organizational specifications to help in MAS engineering.
    % % We introduce , an novel design approach to assist the MAS design whose underlying idea is to use Multi-Agent Reinforcement Learning with organizational specifications for both understanding and constraining the training process regarding design constraints.
    % % results
    % % We applied our approach in cooperative Atari games and a Cyberdefense drone swarm scenario of the 3rd CAGE Challenge. Obtained specifications are indeed consistent with design constraints and provide insights of relevant collective strategies that led to develop an explainable MAS with scores close to finalists' ones.

    % \keywords{Multi-Agent Systems \and Design \and Assisted engineering}
\end{abstract}

\section{Introduction}

% Context:

% General problem: Need for automating/assisting MAS design -> through XAI... -> short general explanation -> split into 2 gaps:

% Gaps:
%   - (G1): Automating the search for joint-policies taking into account organizational constraints;
%   - (G2): Giving means to explain and understand the trained joint-policies for MAS design purposes.

% Proposition: An algorithm relying on MARL and OM -> short description

% Plan:
    % II) Background and motivations:
            % - What are the related works?
            % - Why choosing to use MARL and OM?
            % - Theoretical basics of MARL and OM
    % III) Proposed algorithm
            % - How we combined OMARL and OM formally to get PRAHOM?
            % - What are the sub-gaps and how we addressed these
            %       - Addressing (G1)
            %           - Sub-gaps: (G1.A1), (G1.A2)
            %       - Addressing (G2)
            %           - Sub-gaps: (G2.A1), (G2.A1)
    % IV) Algorithm implementation
            % - How we implemented PRAHOM technically into the PRAHOM Wrapper?
            % - What are the sub-gaps and how we addressed these
            %       - Addressing (G1.A1)
            %           - Sub-gaps: (G1.A2.T1), (G1.A2.T2)
            %       ...

    % V) Case study: Prey-predator
            % - Show a full use of the PRAHOM Wrapper / Tutorial-like of PRAHOM Wrapper step by step
                % 1) Configuration of obs/act-label
                % 2) Setting up the OSH model,
                % 3) Configuring and launching training
                % 4) Configuring and running OS determination

            % - Discuss raw results (with generated figures)

    % VI) Conclusion
% ====================================================================================================

\section{Related works and background}

\section{Proposed algorithm: PRAHOM}

\section{Algorithm implementation: PRAHOM Wrapper}

\section{Case study: Predator-prey environment}

\section{Conclusion}

\section*{References}

% \bibliographystyle{abbrv}
\bibliographystyle{splncs04}

\bibliography{references}

\end{document}