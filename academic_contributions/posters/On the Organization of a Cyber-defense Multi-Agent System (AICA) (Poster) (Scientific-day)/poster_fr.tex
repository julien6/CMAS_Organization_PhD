\documentclass[final]{beamer}

% ====================
% Packages
% ====================

\usepackage[T1]{fontenc}
\usepackage{lmodern}
\usepackage[french=quotes]{csquotes} \MakeOuterQuote{"}
\usepackage[orientation=portrait,size=a0,scale=1.0]{beamerposter}
\usetheme{gemini}
\usecolortheme{nott}
\usepackage{graphicx}
\usepackage{booktabs}
\usepackage{tikz}
\usepackage{multicol}
\usepackage{pgfplots}
\usepackage[inkscapeformat=png]{svg}
\pgfplotsset{compat=1.14}
\usepackage{anyfontsize}
\usepackage{xcolor}
\usepackage[skip=2pt,font=normalsize]{subcaption}
\usepackage{adjustbox}
\usepackage{tabularx}
\usepackage{caption}
\usepackage{tabularx}
\usepackage{changepage}
\usepackage[english]{babel}
\captionsetup{font=it}

\usepackage{makecell}
\renewcommand\theadalign{bc}
\renewcommand\theadfont{\bfseries}
\renewcommand\theadgape{\Gape[4pt]}
\renewcommand\cellgape{\Gape[4pt]}

%%%

\usepackage{listings}
\usepackage{xcolor}

\definecolor{codegreen}{rgb}{0,0.6,0}
\definecolor{codegray}{rgb}{0.5,0.5,0.5}
\definecolor{codepurple}{rgb}{0.58,0,0.82}
\definecolor{backcolour}{rgb}{0.95,0.95,0.92}

\lstdefinestyle{mystyle}{
    backgroundcolor=\color{backcolour},
    commentstyle=\color{codegreen},
    keywordstyle=\color{magenta},
    numberstyle=\tiny\color{codegray},
    stringstyle=\color{codepurple},
    basicstyle=\footnotesize,
    breakatwhitespace=false,
    breaklines=true,
    captionpos=b,
    keepspaces=true,
    numbers=left,
    numbersep=5pt,
    showspaces=false,
    showstringspaces=false,
    showtabs=false,
    tabsize=2
}

\lstset{style=mystyle}


%%%

% \setbeamersize{text margin left=30mm,text margin right=30mm} 

% \addtobeamertemplate{block begin}{%
%   \setlength{\textwidth}{1.2\textwidth}%
%   \leftskip=10pt\rightskip=10pt\vspace{10pt}
%   \par\vspace{10pt}
% }{}


% \setbeamertemplate{itemize/enumerate body begin}{\normalsize}
% \setbeamertemplate{itemize/enumerate subbody begin}{\normalsize}
% \setbeamertemplate{itemize/enumerate subsubbody begin}{\normalsize}

\usepackage{tikz}
\usetikzlibrary{shapes.geometric, arrows}

% Defining Tickz Style
\tikzstyle{startstop} = [rectangle, rounded corners, minimum width=3cm, minimum height=1cm, text centered, text width=10cm, draw=white, fill=white]

% \tikzstyle{io} = [trapezium, trapezium left angle=70, trapezium right angle=110, minimum width=3cm, minimum height=1cm, text centered, text width=4.5cm, draw=black, fill=blue!30 ]

\tikzstyle{process} = [rectangle, minimum width=3cm, minimum height=1cm, text centered, text width=6cm, draw=black, fill=white, text width=10cm]

% \tikzstyle{decision} = [diamond, minimum width=3cm, minimum height=1cm, text centered, draw=black, fill=green!30]

\tikzstyle{arrow} = [ultra thick,->,>=stealth]


% ====================
% Lengths
% ====================

% If you have N columns, choose \sepwidth and \colwidth such that
% (N+1)*\sepwidth + N*\colwidth = \paperwidth
\newlength{\sepwidth}
\newlength{\colwidth}
\setlength{\sepwidth}{-2ex}
\setlength{\colwidth}{0.45\paperwidth}

\newcommand{\separatorcolumn}{\begin{column}{\sepwidth}\end{column}}


\newenvironment{variableblock}[3]{%
    \setbeamercolor{block body}{#2}
    \setbeamercolor{block title}{#3}
    \begin{block}{#1}}{\end{block}}

\addtobeamertemplate{block begin}
{
    \setlength{\textwidth}{1.07\textwidth}%
}
{%\vspace{1ex plus 0.5ex minus 0.5ex} % Pads top of block
    % separates paragraphs in a block
    %\setlength{\parskip}{24pt plus 1pt minus 1pt}%
    \begin{adjustwidth}{0.5cm}{0.5cm}
        }
        \addtobeamertemplate{block end}
        {\end{adjustwidth}%
    \vspace{1ex plus 0.5ex minus 0.5ex}
}% Pads bottom of block
{}
%{\vspace{10ex plus 1ex minus 1ex}} % Seperates blocks from each other

% ====================
%Title
% ====================

\title{De l'Organisation d'un Système Multi-Agent de Cyberdéfense}

\author{\textbf{Julien Soulé} \inst{1,2} \and Jean-Paul Jamont \inst{1} \and Michel Occello \inst{1} Paul Théron \inst{3} Louis-Marie Traonouez \inst{2}}

\institute[shortinst]{\inst{1} Univ. Grenoble Alpe, Grenoble INP, LCIS, Valence, France \samelineand \inst{2} Thales Land and Air Systems, BL IAS, Rennes, France \samelineand \inst{3} AICA IWG, La Guillermie, France}


% ====================
% Footer (optional)
% ====================

\footercontent{
    % \href{https://www.lipsum.com}{\textbf{https://www.lipsum.com}} \hfill

    % \raggedgauche

    % \textbf{Conclave de doctorat Connect 2023} \hfill

    \textbf{\underline{Contact :}} \ \href{mailto:julien.soule@lcis.grenoble-inp.fr}{\textit{julien.soule@lcis.grenoble-inp.fr}}}
% (peut être laissé de côté pour supprimer le pied de page)

\logoright{\includegraphics[height=3cm]{logos/grenoble-inp_logo.png}}
{\includegraphics[height=2cm]{logos/lcis_logo.png}}
{\includegraphics[height=3.5cm]{logos/uga_logo.jpg}}
{\includegraphics[height=3.5cm]{logos/la-ruche_logo.png}}

\begin{document}

\begin{frame}[t]

\vspace{-2ex}

\begin{columns}[t]
    \separatorcolumn

\hspace{-1ex}

\begin{column}{\colwidth}

    \begin{variableblock}{Contexte}{bg=lightgray,fg=black}{bg=lightgray,fg=black}

        \begin{itemize}

            \item \headingNoLine{Systèmes avec une surface d'attaque importante}
                  \begin{itemize}
                      \item Objets connectés (IoT/IoBT), drones, domotique, véhicules tout-terrain, etc.
                      \item Infrastructures cybernétiques, hétérogènes et distribuées
                  \end{itemize}

            \item \headingNoLine{Réagir aux cyberattaques en cours : cyberdéfense}
                  \begin{itemize}
                      \item Charges de travail importantes à traiter en peu de temps, etc.
                      \item Interruption de communication, brouillage, etc.
                  \end{itemize}

                  \quad $\Longrightarrow$ Besoin de : \headingNoLine{réactivité, flexibilité, autonomie, choix de stratégies...}

                  \

            \item \headingNoLine{Système multi-agents de cyberdéfense (MAS)} :

                  \begin{itemize}

                      \item Agents dotés de compétences/connaissances différentes mais atteignant un objectif commun grâce à la coopération, l'interaction et l'organisation

                      \item Fournir des moyens de gérer l'ouverture, l'évolutivité et l'autonomie du système hôte en déléguant différents aspects de la cyberdéfense aux agents

                  \end{itemize}

                  \

                  \begin{figure}
                      \centering
                      % \includegraphics[width=0.9\columnwidth]{images/MASCARA_Organization.pdf}
                      \includesvg[width=0.8\columnwidth]{images/MAS_definition_illustration.svg}
                      \caption{Vue schématique d'un système multi-agent de cyberdéfense en action}
                      \label{fig:mon_étiquette}
                  \end{figure}

                  % \vspace{-1ex}

            \item \headingNoLine{Agent de cyberdéfense intelligent autonome} (AICA)

                  \begin{itemize}
                      \item "\textbf{AICA IWG}" (cf \url{https://www.aica-iwg.org/}) ayant succédé au \textit{Groupe de travail de recherche IST-152} de l'OTAN qui se concentrait sur les "Agents intelligents, autonomes et de confiance pour la cyberdéfense et la résilience".
                  \end{itemize}

                  % \vspace{-2ex}

        \end{itemize}

        % \begin{figure}
        % \centering
        % % \includegraphics[width=0.9\columnwidth]{images/MASCARA_Organization.pdf}
        % \includesvg[width=\columnwidth]{images/MASCARA Organisation.svg}
        % \caption{Architecture de référence AICA}
        % \label{fig:mon_étiquette}
        % \end{figure}

    \end{variableblock}

    \begin{variableblock}{Revue MAS de cyberdéfense}{bg=lightorange,fg=black}{bg=lightorange,fg=black}

        \begin{itemize}

            % \item \textbf{Objectif} : Analyse des MAS de cyberdéfense disponibles pour trouver les relations entre l'organisation, les objectifs de cyberdéfense et l'environnement de déploiement.

            \item \textbf{Dans plus de 60\% des ouvrages connexes :}
                  \begin{itemize}
                      \item Les objectifs de cyberdéfense se concentrent sur \textbf{la détection des anomalies et des intrusions}
                      \item Quels que soient les objectifs, \textbf{organisation centralisée} est la plus courante
                  \end{itemize}

            \item \textbf{Deux principales approches d'organisation dans le contexte de la cyberdéfense}
                  \begin{itemize}
                      Organisations centralisées : bonnes performances pour analyser la situation / contrôler le système ; courant sur les systèmes de taille moyenne mais sur les réseaux dynamiques
                      Organisations décentralisées : meilleure autonomie car plus auto-organisées pour faire face aux cybermenaces ; mais non établies comme solutions génériques de cyberdéfense
                  \end{itemize}

                  % \item \textbf{Limites de l'étude}
                  % \begin{itemize}
                  % \item Subjectivité de la classification
                  % \item Comparaison des MAS disponibles difficile en raison de la diversité des objectifs, des environnements, des architectures d'agents, des protocoles d'interaction\dots
                  % \end{itemize}

        \end{itemize}

    \end{variableblock}

    \begin{variableblock}{Objectifs}{bg=lightgray,fg=black}{bg=lightgray,fg=black}

        \headingNoLine{Quels sont les facteurs organisationnels qui permettent à un MAS de cyberdéfense de fonctionner de manière optimale au regard des contraintes de l'environnement et de ses propres objectifs ?}

        \begin{itemize}
            % \item Comment gérer la concurrence des systèmes de cyberdéfense déjà déployés ?
            \item Quels mécanismes dynamiques d’organisation et de déploiement ?
        \end{itemize}

        % \begin{figure}
        % \centering
        % \includesvg[width=0.9\columnwidth]{images/General_Approach.svg}
        % \end{figure}

    \end{variableblock}

    \begin{variableblock}{Objectif de recherche en modélisation}{bg=lightorange,fg=black}{bg=lightorange,fg=black}

        \headingNoLine{Un modèle de l'ensemble du système cybernétique :}
        \begin{itemize}
            \item Environnement réseau de nœuds \& Cyber-défenseurs et cyber-attaquants
        \end{itemize}

        $\Longrightarrow$ \textbf{Processus de décision de Markov décentralisé et partiellement observable (Dec-POMDP)}

        \begin{figure}
            \centering
            % \includegraphics[width=0.9\columnwidth]{images/objectifs.png}
            \includesvg[width=0.8\columnwidth]{images/MARL_cyberdefense.svg}
            \caption{Une vue du modèle Dec-POMDP pour l'apprentissage par renforcement multi-agents (MARL)}
            \label{fig:mon_étiquette}
        \end{figure}

    \end{variableblock}

\end{column}

\column de séparation

\hspace{-1ex}

\begin{column}{\colwidth}

\begin{variableblock}{Résoudre les axes de recherche}{bg=lightgray,fg=black}{bg=lightgray,fg=black}

    \headingNoLine{Préoccupation} : un processus de conception itératif de MAS est coûteux
    \begin{itemize}
        \item \textbf{Nécessite de l'expérience} : nécessite une connaissance minimale de l'environnement de déploiement mais peut-être très complexe ;
        \item \textbf{Prend beaucoup de temps} : de nombreuses possibilités de conception mais des contraintes de temps ;
        \item \textbf{Apprentissage limité} accès/expérimentation restreints, peu d'experts\dots
    \end{itemize}

    \headingNoLine{Relation partielle entre les histoires des agents et les modèles organisationnels (PRAHOM)} :
    \begin{itemize}
        \item S'appuie sur MARL (Proximal Policy Optimization) pour trouver les politiques optimales des agents ;
        \item Analyse les comportements via des spécifications organisationnelles pour XAI multi-agent.
              % \item Fournit automatiquement des spécifications organisationnelles pertinentes et compréhensibles
              % \item Les concepteurs peuvent intégrer des contraintes de conception
    \end{itemize}

    \quad $\Longrightarrow$ \headingNoLine{Approche d'ingénierie de modèle organisationnel assistée (AOMEA)}

    \begin{figure}
        \centering
        \includegraphics[width=0.55\linewidth]{figures/AOMEA_illustrative_view.png}
        \caption{Vue illustrative de l'AOMEA}
        \label{fig:mon_étiquette}
    \end{figure}

    % \vspace{-0.8ex}
    \vspace{-2ex}

\end{variableblock}

\begin{variableblock}{Objectif de recherche sur la mise en œuvre}{bg=lightorange,fg=black}{bg=lightorange,fg=black}

\headingNoLine{Implémentation du modèle en tant que simulateur :} nous avons proposé l'environnement de développement de systèmes multi-agents de cyberdéfense (CybMASDE) :

\begin{itemize}
    \item création/chargement/enregistrement d'un environnement spécifié avec des agents
    \item lancer l'exécution des agents de cet environnement en simulation/émulation
    \item visualisation de l'environnement réseau sous forme de graphique
\end{itemize}

\begin{figure}
    \centering
    \includegraphics[width=0.8\linewidth]{images/interface_CybMASDE.png}
    \caption{Présentation de CybMASDE}
    \label{fig:interface_simulateur}
\end{figure}

\headingNoLine{Implémentation de la résolution en tant que bibliothèque :} nous avons proposé le \textit{PRAHOM PettingZoo Wrapper} :

\begin{tabularx}{\linewidth}{p{32cm}p{7cm}}
{\begin{lstlisting}[language=Python, caption={View of \emph{PRAHOM Wrapper} use for \emph{Moving Company}}, label={lst:wrapper_mc}]
env=moving_company_v0.raw_env(render_mode="human")
roles=organizational_model(structural_specifications(roles=["role_0","role_1""role2"],...)
jt_histories=joint_histories(env.possible_agents).add_joint_history(jth)
osj_rel=osj_relation(env.possible_agents).link_os(roles,jt_histories,env.possible_agents)
roles_agents_pc=joint_policy_constraint([osj_rel.get_joint_histories
(roles,env.possible_agents)])
train_env=moving_company_v0.parallel_env()
eval_env=moving_company_v0.parallel_env()
env=prahom_wrapper(env,osj_rel,roles_agents_pc,label_to_obj)
env.train_under_constraints(train_env,eval_env)
raw_specs=env.generate_specs()
\end{lstlisting}} &
{
        \vspace{1.8ex}
        \includegraphics[width=1.0\linewidth]{figures/entreprise_en_mouvement_v0.png}} \\
\end{tabulairex}

\vspace{-2ex}

\end{variableblock}

\begin{variableblock}{Perspectives}{bg=lightgray,fg=black}{bg=lightgray,fg=black}
    \vspace{-1ex}
    \begin{itemize}
        \item \textbf{Meilleure explicabilité} : Grands modèles de langage (LLM)
        \item \textbf{Réduire l'écart entre la simulation et la réalité} : amélioration de l'émulation \& identification du système
        \item \textbf{Évaluation \& Généricité} : Études de cas industrielles \& académiques
    \end{itemize}
\end{variableblock}

\end{column}
\column de séparation


\end{columns}
\end{frame}

\end{document}
