\documentclass[final]{beamer}

% ====================
% Packages
% ====================

\usepackage[T1]{fontenc}
\usepackage{lmodern}
\usepackage[french=quotes]{csquotes} \MakeOuterQuote{"}
\usepackage[orientation=portrait,size=a0,scale=1.0]{beamerposter}
\usetheme{gemini}
\usecolortheme{nott}
\usepackage{graphicx}
\usepackage{booktabs}
\usepackage{tikz}
\usepackage{multicol}
\usepackage{pgfplots}
\usepackage[inkscapeformat=png]{svg}
\pgfplotsset{compat=1.14}
\usepackage{anyfontsize}
\usepackage{xcolor}
\usepackage[skip=2pt,font=normalsize]{subcaption}
\usepackage{adjustbox}
\usepackage{tabularx}
\usepackage{caption}
\usepackage{tabularx}
\usepackage{changepage}
\usepackage[english]{babel}
\captionsetup{font=it}

\usepackage{makecell}
\renewcommand\theadalign{bc}
\renewcommand\theadfont{\bfseries}
\renewcommand\theadgape{\Gape[4pt]}
\renewcommand\cellgape{\Gape[4pt]}

%%%

\usepackage{listings}
\usepackage{xcolor}

\definecolor{codegreen}{rgb}{0,0.6,0}
\definecolor{codegray}{rgb}{0.5,0.5,0.5}
\definecolor{codepurple}{rgb}{0.58,0,0.82}
\definecolor{backcolour}{rgb}{0.95,0.95,0.92}
 
\lstdefinestyle{mystyle}{
    backgroundcolor=\color{backcolour},   
    commentstyle=\color{codegreen},
    keywordstyle=\color{magenta},
    numberstyle=\tiny\color{codegray},
    stringstyle=\color{codepurple},
    basicstyle=\footnotesize,
    breakatwhitespace=false,         
    breaklines=true,                 
    captionpos=b,                    
    keepspaces=true,                 
    numbers=left,                    
    numbersep=5pt,                  
    showspaces=false,                
    showstringspaces=false,
    showtabs=false,                  
    tabsize=2
}
 
\lstset{style=mystyle}


%%%

% \setbeamersize{text margin left=30mm,text margin right=30mm} 

% \addtobeamertemplate{block begin}{%
%   \setlength{\textwidth}{1.2\textwidth}%
%   \leftskip=10pt\rightskip=10pt\vspace{10pt}
%   \par\vspace{10pt}
% }{}


% \setbeamertemplate{itemize/enumerate body begin}{\normalsize}
% \setbeamertemplate{itemize/enumerate subbody begin}{\normalsize}
% \setbeamertemplate{itemize/enumerate subsubbody begin}{\normalsize}

\usepackage{tikz}
\usetikzlibrary{shapes.geometric, arrows}

% Defining Tickz Style
\tikzstyle{startstop} = [rectangle, rounded corners, minimum width=3cm, minimum height=1cm, text centered, text width=10cm, draw=white, fill=white]

% \tikzstyle{io} = [trapezium, trapezium left angle=70, trapezium right angle=110, minimum width=3cm, minimum height=1cm, text centered, text width=4.5cm, draw=black, fill=blue!30 ]

\tikzstyle{process} = [rectangle, minimum width=3cm, minimum height=1cm, text centered, text width=6cm, draw=black, fill=white, text width=10cm]

% \tikzstyle{decision} = [diamond, minimum width=3cm, minimum height=1cm, text centered, draw=black, fill=green!30]

\tikzstyle{arrow} = [ultra thick,->,>=stealth]


% ====================
% Lengths
% ====================

% If you have N columns, choose \sepwidth and \colwidth such that
% (N+1)*\sepwidth + N*\colwidth = \paperwidth
\newlength{\sepwidth}
\newlength{\colwidth}
\setlength{\sepwidth}{-2ex}
\setlength{\colwidth}{0.45\paperwidth}

\newcommand{\separatorcolumn}{\begin{column}{\sepwidth}\end{column}}


\newenvironment{variableblock}[3]{%
  \setbeamercolor{block body}{#2}
  \setbeamercolor{block title}{#3}
  \begin{block}{#1}}{\end{block}}

\addtobeamertemplate{block begin}
  {
  \setlength{\textwidth}{1.07\textwidth}%
  }
  {%\vspace{1ex plus 0.5ex minus 0.5ex} % Pads top of block
     % separates paragraphs in a block
   %\setlength{\parskip}{24pt plus 1pt minus 1pt}%
   \begin{adjustwidth}{0.5cm}{0.5cm}
}
\addtobeamertemplate{block end}
  {\end{adjustwidth}%
   \vspace{1ex plus 0.5ex minus 0.5ex}
   }% Pads bottom of block
  {}
  %{\vspace{10ex plus 1ex minus 1ex}} % Seperates blocks from each other

% ====================
%Title
% ====================

\title{On the Organization of a Multi-Agent System for Cyber-defense}

\author{\textbf{Julien Soulé} \inst{1,2} \and Jean-Paul Jamont \inst{1} \and Michel Occello \inst{1} Paul Théron \inst{3} Louis-Marie Traonouez \inst{2}}

\institute[shortinst]{\inst{1} Univ. Grenoble Alpe, Grenoble INP, LCIS, Valence, France \samelineand \inst{2} Thales Land and Air Systems, BL IAS, Rennes, France \samelineand \inst{3} AICA IWG, La Guillermie, France}


% ====================
% Footer (optional)
% ====================

\footercontent{
% \href{https://www.lipsum.com}{\textbf{https://www.lipsum.com}} \hfill

% \raggedleft

% \textbf{Ph.D. Connect Conclave 2023} \hfill

\textbf{\underline{Contact:}} \ \href{mailto:julien.soule@lcis.grenoble-inp.fr}{\textit{julien.soule@lcis.grenoble-inp.fr}}}
% (can be left out to remove footer)


\logoright{\includegraphics[height=3cm]{logos/grenoble-inp_logo.png}}
{\includegraphics[height=2cm]{logos/lcis_logo.png}}
{\includegraphics[height=3.5cm]{logos/uga_logo.jpg}}
{\includegraphics[height=3.5cm]{logos/la-ruche_logo.png}}


\begin{document}

\begin{frame}[t]

    \vspace{-2ex}


    \begin{columns}[t]
        \separatorcolumn

        \hspace{-1ex}

        \begin{column}{\colwidth}

            \begin{variableblock}{Context}{bg=lightgray,fg=black}{bg=lightgray,fg=black}

                \begin{itemize}

                    \item \headingNoLine{Systems with important attack surface}
                          \begin{itemize}
                              \item Connected objects (IoT/IoBT), drones, home automation, all-terrain vehicles, etc.
                              \item Cyber, heterogeneous and distributed infrastructures
                          \end{itemize}

                    \item \headingNoLine{Reacting to occurring cyber-attacks: cyber-defense}
                          \begin{itemize}
                              \item Significant workloads to be processed in a short time, etc.
                              \item Communication interruption, jamming, etc.
                          \end{itemize}

                          \quad $\Longrightarrow$ Need for: \headingNoLine{responsiveness, flexibility, autonomy, choice of strategies...}

                          \

                    \item \headingNoLine{Cyber-defense Multi-Agent System (MAS)}:

                          \begin{itemize}

                              \item Agents with different skills/knowledge but achieving a common objective through cooperation, interaction and organization

                              \item Providing ways to manage the \headingNoLine{openness, scalability and autonomy} of the host system by delegating different aspects of cyber-defense to agents

                          \end{itemize}

                          \

                          \begin{figure}
                              \centering
                              % \includegraphics[width=0.9\columnwidth]{images/MASCARA_Organization.pdf}
                              \includesvg[width=0.8\columnwidth]{images/MAS_definition_illustration.svg}
                              \caption{Schematic view of a cyber defense multi-agent system in action}
                              \label{fig:my_label}
                          \end{figure}

                          % \vspace{-1ex}

                    \item \headingNoLine{Autonomous Intelligent Cyber-defense Agent} (AICA)

                          \begin{itemize}
                              \item "\textbf{AICA IWG}" (c.f \url{https://www.aica-iwg.org/}) having succeeded to the NATO \textit{Research Task Group IST-152} which focused on "Intelligent, Autonomous and Trusted Agents for Cyber Defense and Resilience".
                          \end{itemize}

                          % \vspace{-2ex}

                \end{itemize}

                % \begin{figure}
                %     \centering
                %     % \includegraphics[width=0.9\columnwidth]{images/MASCARA_Organization.pdf}
                %     \includesvg[width=\columnwidth]{images/MASCARA Organization.svg}
                %     \caption{AICA Reference Architecture}
                %     \label{fig:my_label}
                % \end{figure}

            \end{variableblock}

            \begin{variableblock}{Cyber-defense MAS Review}{bg=lightorange,fg=black}{bg=lightorange,fg=black}

                \begin{itemize}

                    % \item \textbf{Objective}: Analysis of available cyber-defense MAS to find the relationships between organization, cyber-defense objectives and deployment environment.

                    \item \textbf{In more than 60\% of the related works:}
                          \begin{itemize}
                              \item Cyber-defense objectives focus on \textbf{anomaly and intrusion detection}
                              \item Regardless of objectives, \textbf{centralized organization} the most common
                          \end{itemize}

                    \item \textbf{Two main approaches of organization in cyber-defense context}
                          \begin{itemize}
                              \item \textit{\textbf{Centralized}} organizations: good \textbf{performance} to analyze situation / control system; common on medium-sized systems but dynamic networks
                              \item \textit{\textbf{Decentralized}} organizations: better autonomy because more \textbf{self-organized} to deal with cyber-threats; but not established as generic cyber-defense solutions
                          \end{itemize}

                          % \item \textbf{Study limitations}
                          %       \begin{itemize}
                          %           \item Subjectivity of classification
                          %           \item Comparison of available MAS difficult because of diversity of objectives, environments, agent architectures, interaction protocols\dots
                          %       \end{itemize}

                \end{itemize}

            \end{variableblock}

            \begin{variableblock}{Objectives}{bg=lightgray,fg=black}{bg=lightgray,fg=black}

                \headingNoLine{What are the organizational factors that enable a cyber defense MAS to function optimally regarding the constraints of the environment and its own objectives?}

                \begin{itemize}
                    % \item How to deal with already deployed cyber-defense systems concurrency? 
                    \item What dynamic organizational and deployment mechanisms?
                \end{itemize}

                % \begin{figure}
                %     \centering
                %     \includesvg[width=0.9\columnwidth]{images/General_Approach.svg}
                % \end{figure}

            \end{variableblock}

            \begin{variableblock}{Modeling research focus}{bg=lightorange,fg=black}{bg=lightorange,fg=black}

                \headingNoLine{A model of the whole cyber system:}
                \begin{itemize}
                    \item Node networked environment \& Cyber-defender and cyber-attackers
                \end{itemize}

                $\Longrightarrow$ \textbf{Decentralized-Partially Observable Markov Decision Process (Dec-POMDP)}

                \begin{figure}
                    \centering
                    % \includegraphics[width=0.9\columnwidth]{images/objectifs.png}
                    \includesvg[width=0.8\columnwidth]{images/MARL_cyberdefense.svg}
                    \caption{An MARL model view}
                    \label{fig:my_label}
                \end{figure}

            \end{variableblock}

        \end{column}

        \separatorcolumn

        \hspace{-1ex}

        \begin{column}{\colwidth}

            \begin{variableblock}{Solving research focus}{bg=lightgray,fg=black}{bg=lightgray,fg=black}

                \headingNoLine{Concern}: An iterative Design Process of MAS is costly
                \begin{itemize}
                    \item \textbf{Needs experience}: requires minimal knowledge of the deployment environment but possibly highly complex;
                    \item \textbf{Time consuming}: many design possibilities but time constraints;
                    \item \textbf{Limited learning} restricted allowed access/experimentation, few experts\dots
                \end{itemize}

                \headingNoLine{Partial Relation Between Agents' Histories and Organizational Models (PRAHOM)}:
                \begin{itemize}
                    \item Leverages on MARL (Proximal Policy Optimization) to find optimal agents' policies;
                    \item Analyzes behaviors through Organizational Specifications for Multi-Agent XAI.
                          % \item Provide understandable relevant organizational specifications automatically
                          % \item Designers can integrate design constraints
                \end{itemize}

                \quad $\Longrightarrow$ \headingNoLine{Assisted Organizational Model Engineering Approach (AOMEA)}

                \begin{figure}
                    \centering
                    \includegraphics[width=0.55\linewidth]{figures/AOMEA_illustrative_view.png}
                    \caption{Illustrative view of AOMEA}
                    \label{fig:my_label}
                \end{figure}

                % \vspace{-0.8ex}
                \vspace{-2ex}

            \end{variableblock}

            \begin{variableblock}{Implementation research focus}{bg=lightorange,fg=black}{bg=lightorange,fg=black}


                \headingNoLine{Implementing model as a simulator:} we proposed the \textit{Cyber-defense Multi Agent System Development Environment} (CybMASDE):

                \begin{itemize}
                    \item creating/loading/saving a specified environment with agents
                    \item launching the execution of the agents of this environment in simulation/emulation
                    \item visualizing the networked environment as a graph
                \end{itemize}

                \begin{figure}
                    \centering
                    \includegraphics[width=0.8\linewidth]{images/interface_CybMASDE.png}
                    \caption{Overview of CybMASDE}
                    \label{fig:interface_simulateur}
                \end{figure}

                \headingNoLine{Implementing solving as a library:} we proposed the \textit{PRAHOM PettingZoo Wrapper}:

                \begin{tabularx}{\linewidth}{p{32cm}p{7cm}}
                    {\begin{lstlisting}[language=Python, caption={View of \emph{PRAHOM Wrapper} use for \emph{Moving Company}}, label={lst:wrapper_mc}]
env=moving_company_v0.raw_env(render_mode="human")
roles=organizational_model(structural_specifications(roles=["role_0","role_1""role2"],...)
jt_histories=joint_histories(env.possible_agents).add_joint_history(jth)
osj_rel=osj_relation(env.possible_agents).link_os(roles,jt_histories,env.possible_agents)
roles_agents_pc=joint_policy_constraint([osj_rel.get_joint_histories
(roles,env.possible_agents)])
train_env=moving_company_v0.parallel_env()
eval_env=moving_company_v0.parallel_env()
env=prahom_wrapper(env,osj_rel,roles_agents_pc,label_to_obj)
env.train_under_constraints(train_env,eval_env)
raw_specs=env.generate_specs()
\end{lstlisting}} &
                    {
                    \vspace{1.8ex}
                    \includegraphics[width=1.0\linewidth]{figures/moving_company_v0.png}} \\
                \end{tabularx}

                \vspace{-2ex}

            \end{variableblock}

            \begin{variableblock}{Perspectives}{bg=lightgray,fg=black}{bg=lightgray,fg=black}
                \vspace{-1ex}
                \begin{itemize}
                    \item \textbf{Better explainability}: Large Language Models (LLM)
                    \item \textbf{Reduce Simulation-to-reality Gap}: Emulation improvement \& System identification
                    \item \textbf{Assessment \& Genericty}: Industrial \& Academical case studies
                \end{itemize}
            \end{variableblock}

        \end{column}
        \separatorcolumn



    \end{columns}
\end{frame}

\end{document}

