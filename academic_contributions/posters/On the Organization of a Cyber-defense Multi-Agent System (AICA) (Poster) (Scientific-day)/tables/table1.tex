\begin{table}[h!]

     \begin{tabularx}{\linewidth}{
     >{\raggedright\arraybackslash\hsize=.9\hsize}X
     >{\raggedright\arraybackslash\hsize=.2\hsize}X}
     \toprule

{ {  \scriptsize   \textbf{Main Objectives}}}
& {  \scriptsize   \textbf{Works}}
\\ \midrule

{  \scriptsize   \textbf{\textbf{R1}}: intrusion detection, network monitoring, detection of possible threats}
& {  \scriptsize   \cite{vasilomanolakis2015taxonomy}, \cite{ gorodetski2003multi}, \cite{de2017distributed}, \cite{holloway2009self, lamont2009military}, \cite{akandwanaho2018generic}, ...}
\\

{  \scriptsize   \textbf{\textbf{R2}}: applying countermeasures, access controls, cyber defense patches, cyber defense strategies}
& {  \scriptsize   \cite{holloway2009self}, \cite{lamont2009military}, \cite{akandwanaho2018generic}, ...}
\\

{  \scriptsize   \textbf{\textbf{R3}}: forensic investigations, developing suitable countermeasures, learning about cyber attacks, adapting to cyber attacks}
& {  \scriptsize   \cite{holloway2019self}, \cite{haack2011ant}, \cite{morteza2015method}, \cite{demir2021adaptive}, ...}
\\
         \bottomrule
        
     \end{tabularx}

     \caption{An overview of the cyber defense functions supported by the studied cyber-defense MAS}

     \label{tab:reference-cyberdefense}
    
\end{table}