%Version 3 October 2023
% See section 11 of the User Manual for version history
%
%%%%%%%%%%%%%%%%%%%%%%%%%%%%%%%%%%%%%%%%%%%%%%%%%%%%%%%%%%%%%%%%%%%%%%
%%                                                                 %%
%% Please do not use \input{...} to include other tex files.       %%
%% Submit your LaTeX manuscript as one .tex document.              %%
%%                                                                 %%
%% All additional figures and files should be attached             %%
%% separately and not embedded in the \TeX\ document itself.       %%
%%                                                                 %%
%%%%%%%%%%%%%%%%%%%%%%%%%%%%%%%%%%%%%%%%%%%%%%%%%%%%%%%%%%%%%%%%%%%%%

%%\documentclass[referee,sn-basic]{sn-jnl}% referee option is meant for double line spacing

%%=======================================================%%
%% to print line numbers in the margin use lineno option %%
%%=======================================================%%

%%\documentclass[lineno,sn-basic]{sn-jnl}% Basic Springer Nature Reference Style/Chemistry Reference Style

%%======================================================%%
%% to compile with pdflatex/xelatex use pdflatex option %%
%%======================================================%%

%%\documentclass[pdflatex,sn-basic]{sn-jnl}% Basic Springer Nature Reference Style/Chemistry Reference Style


%%Note: the following reference styles support Namedate and Numbered referencing. By default the style follows the most common style. To switch between the options you can add or remove �Numbered� in the optional parenthesis. 
%%The option is available for: sn-basic.bst, sn-vancouver.bst, sn-chicago.bst%  
 
%%\documentclass[sn-nature]{sn-jnl}% Style for submissions to Nature Portfolio journals
%%\documentclass[sn-basic]{sn-jnl}% Basic Springer Nature Reference Style/Chemistry Reference Style
\documentclass[sn-mathphys-num]{sn-jnl}% Math and Physical Sciences Numbered Reference Style 
%%\documentclass[sn-mathphys-ay]{sn-jnl}% Math and Physical Sciences Author Year Reference Style
%%\documentclass[sn-aps]{sn-jnl}% American Physical Society (APS) Reference Style
%%\documentclass[sn-vancouver,Numbered]{sn-jnl}% Vancouver Reference Style
%%\documentclass[sn-apa]{sn-jnl}% APA Reference Style 
%%\documentclass[sn-chicago]{sn-jnl}% Chicago-based Humanities Reference Style

%%%% Standard Packages
%%<additional latex packages if required can be included here>

\usepackage{graphicx}%
\usepackage{multirow}%
\usepackage{amsmath,amssymb,amsfonts}%
\usepackage{amsthm}%
\usepackage{mathrsfs}%
\usepackage[title]{appendix}%
\usepackage{xcolor}%
\usepackage{textcomp}%
\usepackage{manyfoot}%
\usepackage{booktabs}%
\usepackage{algorithm}%
\usepackage{algorithmicx}%
\usepackage{algpseudocode}%
\usepackage{listings}%
%%%%

%%%%%=============================================================================%%%%
%%%%  Remarks: This template is provided to aid authors with the preparation
%%%%  of original research articles intended for submission to journals published 
%%%%  by Springer Nature. The guidance has been prepared in partnership with 
%%%%  production teams to conform to Springer Nature technical requirements. 
%%%%  Editorial and presentation requirements differ among journal portfolios and 
%%%%  research disciplines. You may find sections in this template are irrelevant 
%%%%  to your work and are empowered to omit any such section if allowed by the 
%%%%  journal you intend to submit to. The submission guidelines and policies 
%%%%  of the journal take precedence. A detailed User Manual is available in the 
%%%%  template package for technical guidance.
%%%%%=============================================================================%%%%

%% as per the requirement new theorem styles can be included as shown below
\theoremstyle{thmstyleone}%
\newtheorem{theorem}{Theorem}%  meant for continuous numbers
%%\newtheorem{theorem}{Theorem}[section]% meant for sectionwise numbers
%% optional argument [theorem] produces theorem numbering sequence instead of independent numbers for Proposition
\newtheorem{proposition}[theorem]{Proposition}% 
%%\newtheorem{proposition}{Proposition}% to get separate numbers for theorem and proposition etc.

\theoremstyle{thmstyletwo}%
\newtheorem{example}{Example}%
\newtheorem{remark}{Remark}%

\theoremstyle{thmstylethree}%
\newtheorem{definition}{Definition}%

\raggedbottom
%%\unnumbered% uncomment this for unnumbered level heads



\title[Explainability in MARL via KOSIA and GOSIA]{Explainability in Multi-Agent Reinforcement Learning: Investigating KOSIA and GOSIA Approaches}

%%=============================================================%%
%% GivenName	-> \fnm{Joergen W.}
%% Particle	-> \spfx{van der} -> surname prefix
%% FamilyName	-> \sur{Ploeg}
%% Suffix	-> \sfx{IV}
%% \author*[1,2]{\fnm{Joergen W.} \spfx{van der} \sur{Ploeg} 
%%  \sfx{IV}}\email{iauthor@gmail.com}
%%=============================================================%%

\author*[1]{\fnm{Julien} \sur{Soulé}}\email{julien.soule@lcis.grenoble-inp.fr}

\author[1]{\fnm{Jean-Paul} \sur{Jamont}}\email{jean-paul.jamont@lcis.grenoble-inp.fr}

\author[1]{\fnm{Michel} \sur{Occello}}\email{michel.occello@lcis.grenoble-inp.fr}
% \equalcont{These authors contributed equally to this work.}

\author[2]{\fnm{Louis-Marie} \sur{Traonouez}}\email{louis-marie.traonouez@thalesgroup.com}
% \equalcont{These authors contributed equally to this work.}

\author[3]{\fnm{Paul} \sur{Theron}}\email{paul.theron@orange.fr}

\affil*[1]{\orgdiv{Univ. Grenoble Alpes, Grenoble INP}, \orgname{LCIS}, \orgaddress{\street{50 Rue Barthélémy de Laffemas}, \city{Valence}, \postcode{26000}, \state{Auvergne-Rhône-Alpes}, \country{France}}}

\affil[2]{\orgdiv{Thales LAS / IAS / La Ruche}, \orgaddress{\city{Rennes}, \country{France}}}

\affil[3]{\orgdiv{AICA IWG}, \orgaddress{\city{La Guillermie}, \country{France}}}

% \begin{document}

%% Abstract
\abstract{
Explainability in Multi-Agent Reinforcement Learning (MARL) is crucial for understanding and validating the behavior of agents in complex environments. This paper introduces two complementary approaches, KOSIA (Knowledge Organizational Specification Identification Approach) and GOSIA (General Organization Specification Inference Approach), to enhance the explicability of MARL systems. We investigate the strengths and limitations of various techniques for GOSIA, providing a comprehensive evaluation framework. Our contributions address existing gaps in the literature and propose new methodologies for improving transparency in MARL.
}

%% Keywords
\keywords{Explainable AI, Multi-Agent Reinforcement Learning, Organizational Models, KOSIA, GOSIA, MOISE+}

\begin{document}

\maketitle

\section{Introduction}
\label{sec:intro}
\subsection{Background and Motivation}
    Explainability in AI and MARL: Importance and challenges. Overview of organizational models in MARL, focusing on MOISE+.
    
\subsection{Theoretical and Technical Gaps}
    \begin{itemize}
        \item Lack of systematic approaches to link organizational specifications with agent behavior in MARL.
        \item Need for robust methods to infer organizational structures from agent trajectories.
        \item Challenges in generalizing learned behaviors to new organizational contexts.
    \end{itemize}
    
\subsection{Contributions}
    \begin{itemize}
        \item Introduction of the KOSIA approach to match agent histories with known organizational specifications.
        \item Development of the GOSIA approach for inferring organizational specifications from agent behaviors using advanced clustering and representation techniques.
        \item Comprehensive analysis of various similarity measures and clustering techniques applied within GOSIA.
        \item Experimental validation and comparison of the proposed methods in standard MARL environments.
    \end{itemize}

\section{Related Works}
\label{sec:related}
    \subsection{Explainability in Reinforcement Learning}
        Overview of single-agent explainability techniques. Transition to multi-agent systems: new challenges and existing solutions.
        
    \subsection{Organizational Models in MAS}
        Introduction to MOISE+ and its application in MAS. Related works on integrating organizational models with MARL.
        
    \subsection{Explicability in MARL}
        Current state of explainability in MARL. Gaps in existing approaches and the need for KOSIA and GOSIA.

\section{The KOSIA Approach}
\label{sec:kosia}
    \subsection{Conceptual Framework}
        Definition and scope of KOSIA. Theoretical basis: linking agent histories to known organizational specifications.
    
    \subsection{Methodology}
        Pattern-matching techniques for identifying organizational specifications. Implementation details of KOSIA in a MARL setting.
        
    \subsection{Strengths and Limitations}
        Analysis of KOSIA's effectiveness in predefined organizational contexts. Limitations when organizational specifications are unknown or incomplete.

\section{The GOSIA Approach}
\label{sec:gosia}
    \subsection{Conceptual Framework}
        Definition and scope of GOSIA. Theoretical basis: inferring new organizational specifications from agent trajectories.
    
    \subsection{Methodology}
        Overview of clustering techniques (e.g., Hierarchical Clustering, K-Means, Gaussian Mixture Models). Similarity measures for trajectories (e.g., Dynamic Time Warping, Longest Common Subsequence, Edit Distance). Representation of agent trajectories (e.g., PCA, t-SNE, Autoencoders). Implementation details of GOSIA in a MARL setting.
        
    \subsection{Analysis of Techniques for GOSIA}
        Comparison of clustering techniques: pros and cons. Effectiveness of different similarity measures for trajectory comparison. Evaluation of representation techniques for capturing essential behaviors.

\section{Experimental Setup}
\label{sec:experiment}
    \subsection{Environment and Framework}
        Description of the MARL environments used (e.g., PettingZoo environments). Integration of KOSIA and GOSIA within the experimental framework.
    
    \subsection{Evaluation Metrics}
        Metrics for assessing explainability and performance. Criteria for comparing KOSIA and GOSIA outcomes.
        
    \subsection{Results}
        Quantitative results: Performance improvements with KOSIA and GOSIA. Qualitative analysis: Insights from inferred organizational specifications.
    
    \subsection{Discussion}
        Interpretation of experimental results. Discussion on the generalizability of KOSIA and GOSIA. Limitations and potential improvements.

\section{Conclusion and Future Work}
\label{sec:conclusion}
    Summary of contributions: How KOSIA and GOSIA enhance explainability in MARL. Discussion on the impact of the proposed approaches on MARL research. Future research directions: Exploring advanced techniques for trajectory analysis, potential application in real-world scenarios.

%% References
\bibliography{references} % Assuming you have a references.bib file


\end{document}
