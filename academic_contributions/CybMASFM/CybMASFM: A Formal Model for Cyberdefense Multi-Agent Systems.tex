\documentclass[conference]{IEEEtran}
\IEEEoverridecommandlockouts
% The preceding line is only needed to identify funding in the first footnote. If that is unneeded, please comment it out.
\usepackage{amsmath,amssymb,amsfonts}
\usepackage{algorithmic}
\usepackage{graphicx}
\usepackage[inline]{enumitem}
\usepackage{tabularx}
\usepackage{caption}
\usepackage[T2A,T1]{fontenc}
\usepackage[english]{babel}
\captionsetup{font=it}
\usepackage{ragged2e}
\usepackage{hyperref}
\usepackage{footmisc}
\usepackage{booktabs}
\usepackage{csquotes}
\usepackage{smartdiagram}
\usepackage{textcomp}
\usepackage{xcolor}
\def\BibTeX{{\rm B\kern-.05em{\sc i\kern-.025em b}\kern-.08em
    T\kern-.1667em\lower.7ex\hbox{E}\kern-.125emX}}
\usepackage{cite}

\usepackage{etoolbox}
\patchcmd{\thebibliography}{\section*{\refname}}{}{}{}

\setlength{\extrarowheight}{2.5pt}

% \renewcommand{\arraystretch}{1.7}

\newcommand{\old}[1]{\textcolor{orange}{#1}}
\newcommand{\rem}[1]{\textcolor{red}{#1}}
\newcommand{\todo}[1]{\textcolor{orange}{\newline \textit{\textbf{TODO:} #1}} \newline \newline }

\makeatletter
\newcommand{\linebreakand}{%
  \end{@IEEEauthorhalign}
  \hfill\mbox{}\par
  \mbox{}\hfill\begin{@IEEEauthorhalign}
}
\makeatother


\begin{document}

\title{CybMASFM: A Formal Model for cyber-defense Multi-Agent Systems\\
% {\footnotesize \textsuperscript{Note}}
% \thanks{Identify applicable funding agency here. If none, delete this.}
}

% \IEEEaftertitletext{\vspace{-1\baselineskip}}

\author{

\IEEEauthorblockN{Julien Soulé}
\IEEEauthorblockA{\textit{Thales Land and Air Systems, BU IAS}}
%Rennes, France \\
\IEEEauthorblockA{\textit{Univ. Grenoble Alpes,} \\
\textit{Grenoble INP, LCIS, 26000,}\\
Valence, France \\
julien.soule@lcis.grenoble-inp.fr}

\and

\IEEEauthorblockN{Jean-Paul Jamont\IEEEauthorrefmark{1}, Michel Occello\IEEEauthorrefmark{2}}
\IEEEauthorblockA{\textit{Univ. Grenoble Alpes,} \\
\textit{Grenoble INP, LCIS, 26000,}\\
Valence, France \\
\{\IEEEauthorrefmark{1}jean-paul.jamont,\IEEEauthorrefmark{2}michel.occello\}@lcis.grenoble-inp.fr
}

% \and

% \IEEEauthorblockN{Michel Occello}
% \IEEEauthorblockA{\textit{Univ. Grenoble Alpes,} \\
% \textit{Grenoble INP, LCIS, 26000,}\\
% Valence, France \\
% michel.occello@lcis.grenoble-inp.fr}

\and

\IEEEauthorblockN{Paul Théron}
\IEEEauthorblockA{
\textit{Co-leader of the IST-152 OTAN, 1st president of the AICA IWG} \\
La Guillermie, France \\
%lieu-dit Le Bourg, France \\
paul.theron@orange.fr}

% \linebreakand

\and

\IEEEauthorblockN{Louis-Marie Traonouez}
\IEEEauthorblockA{\textit{Thales Land and Air Systems, BU IAS} \\
Rennes, France \\
louis-marie.traonouez@thalesgroup.com}}


\maketitle

\begin{abstract}

Collaboration among cyber-defender agents in a networked host system is a promising approach to tackle cyber-attacks as close as entry points. Indeed, cyber-defense agents that are making up a Cyber-defense Multi-Agent System with a flexible organization could handle scalability and adaptivity issues relying on self/re-organization mechanisms. Yet, before empirically trying to implement it, we aim to frame the problem of organization as the design of cyber-defense agents that have to collaborate to reach a cyber-defense goal under the deployment environment constraints; and the means to solve that problem as organizational mechanisms such as multi-agent paradigms or multi-agent deep learning algorithms.
The paper deals with a general formal model that aims to help framing the design of a cyber-defense multi-agent system by positioning it in related works of cyber-defense, multi-agent systems and reinforcement learning domains.

\end{abstract}

\begin{IEEEkeywords}
Dec-POMDP, cyberdefense, multi-agent systems, reinforcement learning
\end{IEEEkeywords}

\section{Introduction}

As stated within \textquote{Autonomous Intelligent cyber-defense Agents} works\cite{kott2023autonomous}, the increasing complexity of cyber-threats and the limitations of centralized cyber-defense systems have led to fostering a multi-agent approach of cyber-defense. In this approach, the cyber-defender agents are making up a cyber-defense Multi-Agent System (CMAS). This Multi-Agent System (MAS) can not be considered alone. Indeed, we refer to a scenario as comprising the cyber-defender agents team (also called Blue Team), the cyber-attacker agents team (also called Red Team), the regular users agents (aslo called Green Team) and the host nodes networked environment where agents are deployed on.

Willing to implement such a scenario for real, we first faced theoretical considerations as for the problems to be expressed in scenarios and can they be solved .

We refer to organization among agents as the set of their respective individual behavior whose the collective action impacts the distance expressing how far agents are from their common goal for anytime. We assume research in cyber-defense Multi-Agent System is mainly focusing on finding an organization that allows agents being the closest as possible to their common cyber-defense goal at given points.

% A theoretical scenario refers to a set of abstract descriptions or models defining at least partially each component of a scenario. Moreover, we assume a theoretical scenario can always be mapped to real scenarios where the components are real software or hardware assets relying on available hardware and software technology. We refer to development as the whole process allowing real scenarios to be obtained from a theoretical scenario. Therefore, research in cyber-defense MAS requires developing theoretical scenarios to get some feedback of various organizations among cyber-defender agents.
Currently, we did not identify any directly usable work as for comprehensively creating such a scenario as it requires considering both a Multi-Agent Systems (MAS) designing point-of-view for agents teams and a cyber-defense point-of-view for the networked environment, attacks and countermeasures.

From a MAS designing point-of-view, approaches such as in GAIA\cite{wooldridge2000gaia}, VOWELS\cite{demazeau2002} or DIAMOND\cite{jamont2007} show to be too far from the cyber-defense context. Indeed, cyber-defense in multi-agent paradigm brings additional comprehension difficulties related to the high complexity, important data and the low readability of the states a node networked system environment can be in depending on how agents can observe and act.

From a cyber-defense point-of-view, apart from theoretical AICA works, a multi-agent approach of cyber-defense has not been directly addressed. Yet, works related to Autonomous Cyber Operation (ACO) may have explored aspects of the multi-agent paradigm in cyber-defense using simulators and emulators\cite{Veksler2018-mc,vyas2023automated}. However, concerned simulators or emulators do not fully fit into the multi-agent paradigm of cyber-defense. Moreover, no ACO works can be considered to be extensively covering the methodological aspects in research and development of cyber-defense MAS.

Gaps between these two point-of-views showed development of a cyber-defense MAS lacks of consistent, practical methodological ways to define and design several scenarios, implement these scenarios in simulation or emulation, assess and monitoring them in simulation or emulation, and analyze results according to structured metrics for an objective comparison between several cyber-defense MAS.

Our main contribution is the cyber-defense Multi-Agent Systems Formal Model (CybMASFM) comprising both a formal view of cyber-defense Multi-Agent Systems and its interactions.

In section II, we first proposed expectations a methodology for developing cyber-defense MAS could comply with. Then we analyzed the expectation cover in available multi-agent cyber-defense development methodological related works in order to identify gaps through the non covered expectations.
In section III, we introduced the CybMASDA development approach as a theoretical method to design a cyber-defense MAS relying on a formal model.
In section IV, we presented the CybMASDE development environment as a practical mean to use CybMASDA.
In section V, we detailed practical application of CybMASDA through CybMASDE for three case studies involving different attack scenarios.
In section VI, we conclude on the relevancy of our practical method.


\section{Related works}

\section{Overview}

An agent is a software entity deployed on any host system as a daemon.
For example, its source code can be Python, Java, C/C++, etc and running on a Linux or Windows.
As any software program, it requires computing and memory resources such as CPU and RAM usage.
It is able to apply action such as reading, writing, modifying, executing files, sending commands, etc.
After applying action, an agent get observations such as file content, modification acknowledgement, command's output, etc.
Based upon these observations, a decision making process is played to choose the next action to play. That decision making process can range from very simple ones such as random or reactive to complex ones such as cognitive process, reinforcement learning, etc.

In game theory, that decision making process is called a "strategy" and is represented as a mathematical relation associating any situation to an action.
Furthermore, in Dec-POMDP formalism, that relation is also called "behavior" or "policy" and it associates observations to an action.

\section{Conclusion}


\section*{Références}

% \bibliographystyle{abbrv}
\bibliographystyle{IEEEtran}

\bibliography{references}

\end{document}
