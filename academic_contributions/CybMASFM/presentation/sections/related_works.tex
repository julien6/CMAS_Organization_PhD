\AtBeginSection[]{
	\begin{frame}
		\frametitle{}
		\tableofcontents[currentsection]
	\end{frame}
}

%%%%%%%%%%%%%%%%%%%%%%%%%%%%%%%%%%%%
 
 \section{Related works and gaps}
 
	\subsection{Related works}
 
	\begin{frame}{Related works}
		{Attack graphs}

            \begin{block}{Attack graphs~\cite{CPhilips1998}}
                Graphical representation of the system as a set of nodes and the possible attacks as edges between those nodes.
            \end{block}

            \begin{prosblock}{Main pros}
                \begin{itemize}
                    \item Formalization: vulnerabilities imply expressing concretely the consequences on the network.
                \end{itemize}
            \end{prosblock}

            \begin{consblock}{Main cons}
                \begin{itemize}
                    \item No defense: cyber-defense is not properly taken into account.
                \end{itemize}
            \end{consblock}
 
	\end{frame}

	\begin{frame}{Related works}
		{Attack-Defense trees}

            \begin{block}{Attack-Defense trees~\cite{BKordy2010}}
                Graphical models representing the attacker's goals and the defender's countermeasures as a tree structure.
            \end{block}

            \begin{prosblock}{Main pros}
                \begin{itemize}
                    \item Attack and defense: cyber-attackers' actions can be decorated with defender's countermeasures;
                    \item High genericity: suited for various scenarios.
                \end{itemize}
            \end{prosblock}

            \begin{consblock}{Main cons}
                \begin{itemize}
                    \item Low detail level: too abstract for a comprehensive understanding of the impacts of action on the environment.
                \end{itemize}
            \end{consblock}

	\end{frame}

	\begin{frame}{Related works}
		{Petri nets model}

            \begin{block}{Petri nets model}
                As Petri nets can be used to describe concurrent processes, it is possible to model attackers and defenders in a networked system such as in \textquote{Mirai vs white worm}~\cite{SYamaguchi2020}
            \end{block}

            \begin{prosblock}{Main pros}
                \begin{itemize}
                    \item High formalization: possible to simulate precisely a battle between cyber-defenders and cyber-attackers. 
                \end{itemize}
            \end{prosblock}

            \begin{consblock}{Main cons}
                \begin{itemize}
                    \item No ready to use framework: requires spending time on modeling for each context;
                    \item High complexity: difficult to get an overall picture for large systems.
                \end{itemize}
            \end{consblock}

	\end{frame}

	\begin{frame}{Related works}
		{Game models}

            \begin{block}{Game models}
                Envisioned \textquote{Game theoretical} models include: \textquote{Partially Observable Stochastic Game} (POSG) and \textquote{Decentralized Partially Observable Markov Decision Process} (Dec-POMDP)~\cite{beynier2010}.

            \end{block}

            \begin{prosblock}{Main pros}
                \begin{itemize}
                    \item Formalization: both POSGs and Dec-POMDPs are mathematical modeling of decision-making problems;
                    \item Collaborative oriented: in a Dec-POMDP, agents receive a common reward to achieve a common goal~\cite{bernstein2013}.
                \end{itemize}
            \end{prosblock}

            \begin{consblock}{Main cons}
                \begin{itemize}
                    \item No ready to use framework: requires spending time on modeling for each context;
                    % \item Individual oriented: in a POSG, agents may have different goals as each agent has its own reward function~\cite{jk2020}.
                \end{itemize}
            \end{consblock}

	\end{frame}

        \subsection{Theoretical and technical gap}
        \begin{frame}{Theoretical and technical gap}
            {}

            \setstretch{0.1}

            \begin{table}
        
                \begin{tabularx}{\linewidth}{
                >{\centering\arraybackslash\hsize=0.7\hsize}X
                >{\centering\arraybackslash\hsize=0.5\hsize}X
                >{\centering\arraybackslash\hsize=0.5\hsize}X
                >{\centering\arraybackslash\hsize=0.5\hsize}X
                >{\centering\arraybackslash\hsize=0.5\hsize}X
                >{\centering\arraybackslash\hsize=0.5\hsize}X
                }
                \toprule
            
                { {\textbf{}}}
                & {\textbf{\scriptsize Attack \& Defense}}
                & {\textbf{\scriptsize Genericity}}
                & {\textbf{\scriptsize Formalization}}
                & {\textbf{\scriptsize Collaborative oriented}}
                & {\textbf{\scriptsize Practicality}}
    
                \\ \midrule
                
                {  \textbf{\scriptsize Attack graphs} }
                & { \scriptsize  only attack  }
                & { \scriptsize  medium }
                & { \scriptsize  medium }
                & { \scriptsize  no }
                & { \scriptsize  medium }
    
                \\
                \\
                \\

                {  \textbf{\scriptsize AD trees} }
                & { \scriptsize  both }
                & { \scriptsize  good }
                & { \scriptsize  medium }
                & { \scriptsize  possible }
                & { \scriptsize  good }
    
                \\
                \\
                \\

                {  \textbf{\scriptsize Petri nets} }
                & { \scriptsize  both possible }
                & { \scriptsize  good }
                & { \scriptsize  good }
                & { \scriptsize  possible }
                & { \scriptsize  low }
    
                \\
                \\
                \\

                {  \textbf{\scriptsize Game theory (Dec-POMDP)} }
                & { \scriptsize  both possible }
                & { \scriptsize  good }
                & { \scriptsize  good }
                & { \scriptsize  possible }
                & { \scriptsize  good }
    
                \\
                
                \bottomrule
                    
                \end{tabularx}
            
            \end{table}

            \setstretch{1}

            \begin{block}{Theoretical and technical gaps}
                \begin{itemize}
                    \item Taking into account both cyber-attackers and collaborative cyber-defenders;
                    \item Having a adaptative formalization allowing to implement it in highly evolving real context while keeping being practical.
                \end{itemize}
            \end{block}

            Yet Dec-POMDP is close and can be extended to meet our needs\dots
  
	\end{frame}