\begin{table*}[t!]

    \caption{Cyberdefense Multi-Agent System Waterfall lifecycle with expectations}

    \begin{tabularx}{\linewidth}{
    >{\raggedright\arraybackslash\hsize=.1\hsize}X
    >{\raggedright\arraybackslash\hsize=.9\hsize}X}
    \toprule

    { \textbf{Step}}
    & {  \textbf{Expectations}}
    
    \\ \midrule

        { Requirements Analysis } & {
        \begin{tabularx}{\linewidth}{
        >{\raggedright\arraybackslash\hsize=.9\hsize}X}
           { (E1.1) Identifiy the networked host nodes system properties with envisioned deployed cyber-attacker and cyber-defender agents } \\
           { (E1.2) Identifiy cyber-attacker and cyber-defender agents metrics from environment properties to indicate distance towards their respective common goals } \\
           { (E1.3) Identifiy the cyber-attacker and cyber-defender agents respective required observations and actions to reach their goals}
        \end{tabularx}} \\ \\

        { Design } & {
        \begin{tabularx}{\linewidth}{
        >{\raggedright\arraybackslash\hsize=.9\hsize}X}
           { (E2.1) Provide a detailed network plan relying on available softwares and systems from the identified networked host nodes system requirements } \\
           { (E2.2) Provide theoretical behavioral models for cyber-attacker agents using distributed artificial intelligence and cyber-attack domain knowledge } \\
           { (E2.3) Provide theoretical behavioral models for cyber-defender agents using distributed artificial intelligence and cyberdefense domain knowledge }
        \end{tabularx}} \\ \\
        
        { Implementation } & {
        \begin{tabularx}{\linewidth}{
        >{\raggedright\arraybackslash\hsize=.9\hsize}X}
           { (E3.1) Develop the designed networked host node system into a mock experimental environment allowing it to be dynamic and agents to be executed in a timely manner while providing means to monitor them and their impacts on the environment } \\
           { (E3.2) Develop the designed behavioral models into several agents in an experimental environment }
        \end{tabularx}} \\ \\
        
        { Testing } & {
        \begin{tabularx}{\linewidth}{
        >{\raggedright\arraybackslash\hsize=.9\hsize}X}
           { (E4.1) Validate implemented agents behaviors through associated respective goals' metrics} \\
           { (E4.2) Monitor agents and networked environment at any time in the experimental environment thanks to provided means }
        \end{tabularx}} \\ \\
        
        { Deployment } & {
        \begin{tabularx}{\linewidth}{
        >{\raggedright\arraybackslash\hsize=.9\hsize}X}
           { (E5.1) Adapt and package cyber-defender agents for target deployment host system augmenting their capabilities to be monitored and communicate from outside } \\
           { (E5.2) Unpack, install cyber-defender agents in nodes and tune them for global configuration allow complying with host system requirements and policies}
        \end{tabularx}} \\ \\
        
        { Maintenance } & {
        \begin{tabularx}{\linewidth}{
        >{\raggedright\arraybackslash\hsize=.9\hsize}X}
           { (E6.1) Monitor deployed running agent so raised alert can be processed by external entities when unexpected events or failure are encountered } \\
           { (E6.2) Manage agents as for updating, adding or removing agents in nodes by external entities when changes are needed to maintain cyber-defender agents compliance with their goals and additional external requirements }
        \end{tabularx}} \\
    
    \bottomrule
        
    \end{tabularx}
    \label{tab:cmas-waterfall-lifecycle}
\end{table*}