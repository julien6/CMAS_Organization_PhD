\AtBeginSection[]{
	\begin{frame}
		\frametitle{}
		\tableofcontents[currentsection]
	\end{frame}
}

%%%%%%%%%%%%%%%%%%%%%%%%%%%%%%%%%%%%

	\section{Introduction}
	\begin{frame}[allowframebreaks]{Introduction}

	   \begin{block}{AICA: Autonomous Intelligent Cyberdefense Agent \cite{theron_autonomous_2021}}

            An agent theorized by « IST-152 NATO » between 2016-2019 that is to be deployed on networked nodes to:
 
	       \begin{itemize}
                \item Detect, identify and characterize anomalies/attacks
                \item Plan and execute countermeasures
                \item Communicate with C2, operators\dots
                \item Being autonomous, stealthy, inter-operable, able to learn
		\end{itemize}

        \end{block}

        \begin{block}{MASCARA: Multi Agent System Centric AICA Reference Architecture \cite{theron_autonomous_2021}}
            \begin{itemize}
                \item A multi-agent vision of a decentralized and distributed AICA;
                \item A set of collaborative cyber-defenders fighting back against cyber-attacker(s) deployed over a networked system.
            \end{itemize}
        \end{block}

        \begin{alertblock}{Main concerns}
            No available consistent, clear and general framework to deal with cyber-defenders fighting against cyber-attackers in a networked system:
            \begin{itemize}
                \item Need to clarify how agents, environment and their interactions should be envisioned consistently;
                \item Need to assess collective cyber-defense experimentally through various criteria.
            \end{itemize}

        \end{alertblock}

        \begin{block}{Intended contributions}
            \begin{itemize}
                \item A formal model for cyber-defenders fighting against cyber-attackers in a networked system;
                \item A simulation tool to assess the efficiency of cyber-defenders / cyber-attackers collective actions for various attack scenarios.
            \end{itemize}
        \end{block}
 
	\end{frame}