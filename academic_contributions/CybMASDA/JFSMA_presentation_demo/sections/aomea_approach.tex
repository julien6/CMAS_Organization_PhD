% \AtBeginSection[]{
%     \begin{frame}
%         \frametitle{}
%         \tableofcontents[currentsection]
%     \end{frame}
% }

%%%%%%%%%%%%%%%%%%%%%%%%%%%%%%%%%%%%

\section{Preuve de Concept sur exemple}

\begin{frame}[fragile]{Preuve de Concept sur exemple}{Initialisation}

    \begin{columns}

        \begin{column}{0.6\textwidth}

            \begin{enumerate}
                \item Initialiser environment PettingZoo
                \item Mapping observation labels
            \end{enumerate}
    
\begin{lstlisting}[language=Python,basicstyle=\scriptsize]
from custom_envs.movingcompany import moving_company_v0
from prahom_wrapper.prahom_wrapper import prahom_wrapper

env = moving_company_v0.parallel_env(render_mode="human")

label_to_obs = {
    "empty_corridor_0": [0,1,0,0,1,0,0,1,0],
    "empty_corridor_1": [0,1,0,0,2,0,0,1,0],
    "empty_corridor_2": [0,1,0,0,3,0,0,1,0],
    "obs_goal_0": [0,1,0,0,2,0,0,5,1]
    ...
    "go_down": 1,
    "go_up": 2,
}
\end{lstlisting}
    
        \end{column}
    
        \begin{column}{0.4\textwidth}
            \centering
            \begin{figure}
                \includegraphics[width=\linewidth]{figures/moving_company_v0.png}
                \caption{Exemple de rendu graphique de l'environment \textquote{Moving Company}}
            \end{figure}
        \end{column}
    
    \end{columns}


\end{frame}


\begin{frame}

    % Boucle pour afficher toutes les images de frame000.png à frame099.png
    \foreach \i in {01, 02, 03, 04, 05, 06, 07, 08, 09, 10, 11, 12, 13, 14, 15, 16, 17, 18, 19, 20, 21, 22} {
    
        \begin{onlyenv}<\i>
        % \begin{frame}[plain]
            \frametitle{Exemple : Moving Company après entrainement}
            \centering
            \includegraphics[width=0.5\textwidth]{figures/mcy_frames/frame_\i_delay-0.01s.png}
        % \end{frame}
    
        \end{onlyenv}
    
    }
    
\end{frame}


\begin{frame}[fragile]{Preuve de Concept sur exemple}{}

    \begin{columns}

        \begin{column}{0.6\textwidth}
    
            \textbf{Phase 1 : Modélisation}
    
            \begin{itemize}
                \item Modèle simulé de l'environnement ($1.1$) incluant les objectifs des agents ($1.2$);
                \item Définir les comportements attendus des rôles, objectifs comme ensemble d'historiques;
                \item Contraindre les agents à des rôles (et missions) ($1.3$).
            \end{itemize}

            \begin{lstlisting}[language=Python,basicstyle=\scriptsize]
org_model = organizational_model(
    structural_specifications=ss(
        roles={ "role_0": history_subset(pattern="[empty_corridor_1,go_up](1,7),[empty_corridor_3,go_down](1,8)")},
        role_inheritance_relations=None, root_groups=None),
    
    functional_specifications=fs(
        social_scheme=sch(
            goals={"goal_0": history_subset(pattern="[#Any](0,*),[obs_goal_0]")},missions=["mission_0"], goals_structure=None,mission_to_goals={"mission_0": ["goal_0"]},mission_to_agent_cardinality=None),
        social_preferences=None),

    deontic_specifications={
        "role_0": {("mission_0", "Any"): ["agent_0"]}})\end{lstlisting}

        \end{column}
    
        \begin{column}{0.4\textwidth}
            \centering
            \adjustbox{trim={0.\width} {0.82\height} {0.\width} {0.\height}, clip}{%
                \includegraphics[width=1.2\linewidth]{figures/AOMEA_illustrative_view}
            }
        \end{column}
    
    \end{columns}
    
    
    \end{frame}
    
    
    \begin{frame}[fragile]{Preuve de Concept sur exemple}{}
    
    \begin{columns}
    
        \begin{column}{0.6\textwidth}
    
            \textbf{Phase 2 : Résolution}
    
            \begin{itemize}
                \item Algorithme MARL orienté organisation (OMARL) : processus MARL + $\mathcal{M}OISE^+$;
                \item Résoudre en respectant les historiques contraints des rôles ($2.1$);
                \item Obtient la politique conjointe entrainée ($2.1$)
            \end{itemize}

            \begin{lstlisting}[language=Python,basicstyle=\scriptsize]
pz_env.train_under_constraints(
    label_to_obs=label_to_obs,
    constraint_integration_mode="CORRECT",algorithm_configuration="default_MAPPO"
    org_model_constraint=org_model)
            \end{lstlisting}

        \end{column}
    
        \begin{column}{0.4\textwidth}
            \centering
            \adjustbox{trim={0.\width} {0.56\height} {0.\width} {0.\height}, clip}{%
                \includegraphics[width=1.2\linewidth]{figures/AOMEA_illustrative_view}
            }
        \end{column}
    
    \end{columns}
    
    \end{frame}
    
    \begin{frame}[fragile]{Preuve de Concept sur exemple}
    
    \begin{columns}
    
        \begin{column}{0.6\textwidth}
    
            \textbf{Phase 3 : Analyse}
    
            \begin{itemize}
                \item Pattern-matching (KOSIA) ou Apprentissage Non-Supervisé (GOSIA) pour determiner SO ($3.1$);
                \item Informations pour comprendre rôles/ objectifs permettant d'atteindre l'objectif final ($3.1 \ \& \ 3.2$);
                \item Correction et ajustement des SO corrigés ($3.3$).
            \end{itemize}

            \begin{lstlisting}[language=Python,basicstyle=\scriptsize]
pz_env.generate_organizational_specifications(use_kosia=False, use_gosia=True,gosia_configuration={"generate_figures": True})
            \end{lstlisting}

        \end{column}
    
        \begin{column}{0.4\textwidth}
            \centering
            \adjustbox{trim={0.\width} {0.35\height} {0.\width} {0.110\height}, clip}{%
                \includegraphics[width=1.2\linewidth]{figures/AOMEA_illustrative_view}
            }
        \end{column}
    
    \end{columns}
    
    \end{frame}
    
    \begin{frame}[fragile]{Preuve de Concept sur exemple}

        \begin{figure}
            \includegraphics[width=0.7\linewidth]{figures/role_clustering.png}
            \caption*{Exemple de Dendrogramme obtenu après \textit{Hierarchical Clustering}}
        \end{figure}

    \end{frame}
        

    \begin{frame}[fragile]{Preuve de Concept sur exemple}
    
    \begin{columns}
    
        \begin{column}{0.6\textwidth}
    
            \textbf{Phase 4 : Développement}

            \begin{itemize}
                \item Peu automatisé car difficultés : \textit{Simulation-to-reality gap}, sûreté de fonctionnement, \textit{transfer learning}\dots

                \item Preconisation $\rightarrow$ Approche manuelle:
                \begin{enumerate}
                    \item Implémenter les politiques selon les SO corrigées :
                    \item Explicitement (arbre de décision\dots) ;
                    \item En simulation $\rightarrow$ émulation $\rightarrow$ réel\dots ;
                \end{enumerate}
            \end{itemize}

            \

            \begin{itemize}
                \item Travail en cours\dots
            \end{itemize}

        \end{column}
    
        \begin{column}{0.4\textwidth}
            \centering
            \adjustbox{trim={0.\width} {0.15\height} {0.\width} {0.57\height}, clip}{%
                \includegraphics[width=1.2\linewidth]{figures/AOMEA_illustrative_view}
            }
        \end{column}
    
    \end{columns}
    
    \end{frame}    
