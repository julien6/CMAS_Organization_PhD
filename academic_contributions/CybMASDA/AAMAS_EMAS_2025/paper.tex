%%%%%%%%%%%%%%%%%%%%%%%%%%%%%%%%%%%%%%%%%%%%%%%%%%%%%%%%%%%%%%%%%%%%%%%%

%%% LaTeX Template for AAMAS-2025 (based on sample-sigconf.tex)
%%% Prepared by the AAMAS-2025 Program Chairs based on the version from AAMAS-2025. 

%%%%%%%%%%%%%%%%%%%%%%%%%%%%%%%%%%%%%%%%%%%%%%%%%%%%%%%%%%%%%%%%%%%%%%%%

%%% Start your document with the \documentclass command.


%%% == IMPORTANT ==
%%% Use the first variant below for the final paper (including auithor information).
%%% Use the second variant below to anonymize your submission (no authoir information shown).
%%% For further information on anonymity and double-blind reviewing, 
%%% please consult the call for paper information
%%% https://aamas2025.org/index.php/conference/calls/submission-instructions-main-technical-track/

%%%% For anonymized submission, use this
\documentclass[sigconf,anonymous]{aamas} 

%%%% For camera-ready, use this
%\documentclass[sigconf]{aamas}

\usepackage{listings}
% \usepackage{xcolor}

\definecolor{codegreen}{rgb}{0,0.6,0}
\definecolor{codegray}{rgb}{0.5,0.5,0.5}
\definecolor{codepurple}{rgb}{0.58,0,0.82}
\definecolor{backcolour}{rgb}{0.95,0.95,0.92}
 
\lstdefinestyle{mystyle}{
    backgroundcolor=\color{backcolour},   
    commentstyle=\color{codegreen},
    keywordstyle=\color{magenta},
    numberstyle=\tiny\color{codegray},
    stringstyle=\color{codepurple},
    basicstyle=\footnotesize,
    breakatwhitespace=false,         
    breaklines=true,                 
    captionpos=b,                    
    keepspaces=true,                 
    numbers=left,                    
    numbersep=5pt,                  
    showspaces=false,                
    showstringspaces=false,
    showtabs=false,                  
    tabsize=2
}
 
\lstset{style=mystyle}

% --- Tickz
\usepackage{physics}
\usepackage{tikz}
\usepackage{amsmath}
\usepackage{mathdots}
% \usepackage{yhmath}
\usepackage{cancel}
\usepackage{color}
\usepackage{siunitx}
\usepackage{array}
\usepackage{multirow}
% \usepackage{amssymb}
\usepackage{gensymb}
\usepackage{tabularx}
\usepackage{extarrows}
\usepackage{booktabs}
\usetikzlibrary{fadings}
\usetikzlibrary{patterns}
\usetikzlibrary{shadows.blur}
\usetikzlibrary{shapes}

% ---------

\usepackage{balance} % for balancing columns on the final page
\usepackage{csquotes}
% \usepackage{cite}
\newcommand{\probP}{\text{I\kern-0.15em P}}
\usepackage{etoolbox}
\patchcmd{\thebibliography}{\section*{\refname}}{}{}{}
% \usepackage{amsthm,amssymb,amsfonts}

\usepackage[T1]{fontenc}
\usepackage{graphicx}
\usepackage{hyperref}
\usepackage{color}
% \renewcommand\UrlFont{\color{blue}\rmfamily}

\usepackage[inline, shortlabels]{enumitem}
\usepackage{tabularx}
\usepackage{caption}
\usepackage{listings}
\usepackage{titlesec}
\usepackage{ragged2e}
% \usepackage[hyphens]{url}
\usepackage[linesnumbered,ruled,vlined]{algorithm2e}
\usepackage{float}

\usepackage[english]{babel}
\addto\extrasenglish{  
    \def\figureautorefname{Figure}
    \def\tableautorefname{Table}
    \def\algorithmautorefname{Algorithm}
    \def\sectionautorefname{Section}
    \def\subsectionautorefname{Subsection}
    \def\proofoutlineautorefname{Proof Outline}
}

\newcommand{\supertiny}{\fontsize{1}{2}\selectfont}


%%%%%%%%%%%%%%%%%%%%%%%%%%%%%%%%%%%%%%%%%%%%%%%%%%%%%%%%%%%%%%%%%%%%%%%%

%%% AAMAS-2025 copyright block (do not change!)

\setcopyright{ifaamas}
\acmConference[AAMAS '25]{Proc.\@ of the 24th International Conference
on Autonomous Agents and Multiagent Systems (AAMAS 2025)}{May 19 -- 23, 2025}
{Detroit, Michigan, USA}{A.~El~Fallah~Seghrouchni, Y.~Vorobeychik, S.~Das, A.~Nowe (eds.)}
\copyrightyear{2025}
\acmYear{2025}
\acmDOI{}
\acmPrice{}
\acmISBN{}


%%%%%%%%%%%%%%%%%%%%%%%%%%%%%%%%%%%%%%%%%%%%%%%%%%%%%%%%%%%%%%%%%%%%%%%%

%%% == IMPORTANT ==
%%% Use this command to specify your EasyChair submission number.
%%% In anonymous mode, it will be printed on the first page.

\acmSubmissionID{<<EasyChair submission id>>}

%%% Use this command to specify the title of your paper.

\title[AAMAS-2025 CybMASDE]{Improving Multi-Agent Systems Organization Design through Assisted Engineering with Reinforcement Learning}

%%% Provide names, affiliations, and email addresses for all authors.

\author{Julien Soulé}
\affiliation{
  \institution{Univ. Grenoble Alpes}
  \city{Valence}
  \country{France}}
\email{julien.soule@lcis.grenoble-inp.fr}

\author{Jean-Paul Jamont}
\affiliation{
  \institution{Univ. Grenoble Alpes}
  \city{Valence}
  \country{France}}
\email{jean-paul.jamont@lcis.grenoble-inp.fr}

\author{Michel Occello}
\affiliation{
  \institution{Univ. Grenoble Alpes}
  \city{Valence}
  \country{France}}
\email{michel.occello@lcis.grenoble-inp.fr}

\author{Louis-Marie Traonouez}
\affiliation{
  \institution{Thales Land and Air Systems, BU IAS}
  \city{Rennes}
  \country{France}}
\email{louis-marie.traonouez@thalesgroup.com}

\author{Paul Théron}
\affiliation{
  \institution{AICA IWG}
  \city{La Guillermie}
  \country{France}}
\email{paul.theron@orange.fr}

\begin{abstract}
  La conception et la validation des systèmes multi-agents (SMA) posent des défis importants en raison de leur complexité et de leur nature décentralisée. Dans cet article, nous proposons une méthodologie intégrée pour l'ingénierie des SMA, basée sur une spécification formelle des comportements agents, une architecture flexible orientée rôles, et un cadre de vérification utilisant des techniques de validation formelle et de test. Nous démontrons l'efficacité de notre approche à travers des études de cas dans des contextes d'ingénierie éthique et de tolérance aux pannes.
\end{abstract}

%%% The code below was generated by the tool at http://dl.acm.org/ccs.cfm.
%%% Please replace this example with code appropriate for your own paper.


%%% Use this command to specify a few keywords describing your work.
%%% Keywords should be separated by commas.

\keywords{Agent-Oriented Software Engineering \and Multi-Agent Reinforcement Learning \and Assisted-Design \and Organizational Models}

%%%%%%%%%%%%%%%%%%%%%%%%%%%%%%%%%%%%%%%%%%%%%%%%%%%%%%%%%%%%%%%%%%%%%%%%

%%% Include any author-defined commands here.
         
% \newcommand{\BibTeX}{\rm B\kern-.05em{\sc i\kern-.025em b}\kern-.08em\TeX}

%%%%%%%%%%%%%%%%%%%%%%%%%%%%%%%%%%%%%%%%%%%%%%%%%%%%%%%%%%%%%%%%%%%%%%%%

\begin{document}

%%% The following commands remove the headers in your paper. For final 
%%% papers, these will be inserted during the pagination process.

\pagestyle{fancy}
\fancyhead{}

%%% The next command prints the information defined in the preamble.

\maketitle

%%%%%%%%%%%%%%%%%%%%%%%%%%%%%%%%%%%%%%%%%%%%%%%%%%%%%%%%%%%%%%%%%%%%%%%%

\section{Introduction}

Autonomous agents operating within complex environments need to coordinate efficiently to achieve collective objectives, often under strict constraints. Existing Multi-Agent Reinforcement Learning (MARL) methods are effective in policy learning but lack mechanisms for guiding agents towards organizational roles and ensuring safety requirements. This paper introduces a novel approach combining the **MOISE+MARL** framework, which integrates organizational models with MARL, and **HEMM**, a history-based evaluation algorithm that infers organizational structures and roles from agent behaviors. This method enables the design and deployment of organizationally constrained multi-agent systems that are efficient, safe, and interpretable.

\section{Related Works}

Various works have explored the use of roles and organizations in MAS, such as the works in \cite{gleize2008moise,winikoff2021agile}. The MOISE+ model \cite{gleize2008moise} provides a formal specification for defining roles and missions, but lacks integration with MARL learning processes. Additionally, recent advancements in agile methodologies for AOSE \cite{winikoff2021agile,winikoff2023bdd} have shown that system-level stories can help bridge the gap between design and implementation in MAS. Our work builds upon these by introducing a structured approach that combines organizational constraints with MARL, offering an automated pipeline for both policy learning and deployment.

\section{Proposed Organizational Design Approach for MAS}

\subsection{Overview}
We propose a four-phase method for designing and deploying Multi-Agent Systems, which ensures compliance with organizational and safety constraints while optimizing for the given objectives. The approach combines organizational specifications from MOISE+ with reinforcement learning techniques, followed by automatic policy explanation using HEMM to generate interpretable agent behaviors.

\subsection{Phase 1: Modeling}
In this phase, we simulate the target environment and define the objectives.
\begin{itemize}
    \item \textbf{Inputs:} Real-world environment specifications, objectives, constraints.
    \item \textbf{Outputs:} Formalized problem representation in a simulated environment.
    \item \textbf{Process:} The environment is modeled using a simulation tool that captures both the dynamics of the agents and the constraints imposed by the target objectives. MOISE+ roles and missions are incorporated as part of the formalized environment.
\end{itemize}

\subsection{Phase 2: Solving}
In this phase, the formalized problem is solved using **MOISE+MARL**.
\begin{itemize}
    \item \textbf{Inputs:} Problem representation, organizational constraints.
    \item \textbf{Outputs:} Stable policies that satisfy both objectives and safety guarantees.
    \item \textbf{Process:} The MOISE+MARL framework integrates organizational constraints into the learning process. By defining roles and missions as constraints, the agents’ learning process is guided towards these predefined organizational structures, ensuring safety and robustness.
\end{itemize}

\subsection{Phase 3: Analyzing}
The trained policies are analyzed using **HEMM**.
\begin{itemize}
    \item \textbf{Inputs:} Joint policies learned in Phase 2.
    \item \textbf{Outputs:} Inferred roles and missions, "blueprint" of agent behaviors.
    \item \textbf{Process:} HEMM infers roles and missions from the agent's behavior by clustering their action sequences and observation paths. The variance between inferred roles and the predefined roles gives a metric for organizational adequacy. The results are interpreted as a "blueprint" for human designers to refine and deploy in the next phase.
\end{itemize}

\subsection{Phase 4: Developing}
In this phase, the "blueprints" are used to develop the real-world MAS.
\begin{itemize}
    \item \textbf{Inputs:} Inferred roles and missions, simulation environment.
    \item \textbf{Outputs:} A deployed MAS in both emulated and real environments.
    \item \textbf{Process:} The "blueprints" are used to guide the development of an MAS in an emulated environment. Following verification in the emulation phase, the system is deployed in the real environment. Automated processes inject learned policies into the system’s agents, ensuring compliance with safety and organizational constraints.
\end{itemize}

\section{Evaluation}

\subsection{CybMASDE: a development environment for the approach}
We propose **CybMASDE**, a tool that supports the phases of our approach by providing interfaces for environment modeling, policy learning, and policy deployment. CybMASDE also facilitates user interaction during the analysis and development phases through visualization of inferred roles and automated deployment tools.

\subsection{Result}
The results from initial experiments demonstrate the effectiveness of the proposed method in both simulated environments (predator-prey, resource gathering) and real-world scenarios (e.g., cyber-defense simulations). Our tool significantly reduces development time while improving policy stability and adherence to organizational constraints.

\section{Conclusion}
This paper presented a method for designing and deploying organizationally constrained MAS by combining MOISE+MARL for role-based policy learning and HEMM for automated policy analysis. The proposed method offers a scalable and interpretable solution for MAS deployment in safety-critical environments, providing both learning efficiency and explainability of agent behaviors.



\section*{References}

\bibliographystyle{ACM-Reference-Format}
\renewcommand
\refname{}
\bibliography{references}

\end{document}

%%%%%%%%%%%%%%%%%%%%%%%%%%%%%%%%%%%%%%%%%%%%%%%%%%%%%%%%%%%%%%%%%%%%%%%%