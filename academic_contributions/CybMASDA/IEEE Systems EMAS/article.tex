\documentclass[conference]{IEEEtran}
\IEEEoverridecommandlockouts
% The preceding line is only needed to identify funding in the first footnote. If that is unneeded, please comment it out.

\usepackage{hyperref}

% --- Tickz
\usepackage{physics}
\usepackage{tikz}
\usepackage{amsmath}
\usepackage{mathdots}
% \usepackage{yhmath}
\usepackage{cancel}
\usepackage{color}
\usepackage{siunitx}
\usepackage{array}
\usepackage{multirow}
% \usepackage{amssymb}
\usepackage{gensymb}
\usepackage{tabularx}
\usepackage{extarrows}
\usepackage{booktabs}
\usetikzlibrary{fadings}
\usetikzlibrary{patterns}
\usetikzlibrary{shadows.blur}
\usetikzlibrary{shapes}

% ---------

\usepackage{balance} % for balancing columns on the final page
\usepackage{csquotes}
% \usepackage{cite}
\newcommand{\probP}{\text{I\kern-0.15em P}}
\usepackage{etoolbox}
\patchcmd{\thebibliography}{\section*{\refname}}{}{}{}
% \usepackage{amsthm,amssymb,amsfonts}

\usepackage[T1]{fontenc}
\usepackage{graphicx}
\usepackage{color}
% \renewcommand\UrlFont{\color{blue}\rmfamily}

\usepackage[inline, shortlabels]{enumitem}
\usepackage{tabularx}
\usepackage{caption}
\usepackage{listings}
\usepackage{stfloats}
\usepackage{titlesec}
\usepackage{amssymb}
\usepackage{amsmath}
\usepackage{ragged2e}
% \usepackage[hyphens]{url}
\usepackage[linesnumbered,ruled,vlined]{algorithm2e}
\usepackage{float}
\usepackage[english]{babel}
\addto\extrasenglish{  
    \def\figureautorefname{Figure}
    \def\tableautorefname{Table}
    \def\algorithmautorefname{Algorithm}
    \def\sectionautorefname{Section}
    \def\subsectionautorefname{Subsection}
    \def\proofoutlineautorefname{Proof Outline}
}

% --------------
\titleclass{\subsubsubsection}{straight}[\subsection]

\newcounter{subsubsubsection}[subsubsection]
\renewcommand\thesubsubsubsection{\thesubsubsection.\arabic{subsubsubsection}}
\renewcommand\theparagraph{\thesubsubsubsection.\arabic{paragraph}} % optional; useful if paragraphs are to be numbered

\titleformat{\subsubsubsection}
  {\normalfont\normalsize\bfseries}{\thesubsubsubsection}{1em}{}
\titlespacing*{\subsubsubsection}
{0pt}{3.25ex plus 1ex minus .2ex}{1.5ex plus .2ex}

\makeatletter
\renewcommand\paragraph{\@startsection{paragraph}{5}{\z@}%
  {3.25ex \@plus1ex \@minus.2ex}%
  {-1em}%
  {\normalfont\normalsize\bfseries}}
\renewcommand\subparagraph{\@startsection{subparagraph}{6}{\parindent}%
  {3.25ex \@plus1ex \@minus .2ex}%
  {-1em}%
  {\normalfont\normalsize\bfseries}}
\def\toclevel@subsubsubsection{4}
\def\toclevel@paragraph{5}
\def\toclevel@paragraph{6}
\def\l@subsubsubsection{\@dottedtocline{4}{7em}{4em}}
\def\l@paragraph{\@dottedtocline{5}{10em}{5em}}
\def\l@subparagraph{\@dottedtocline{6}{14em}{6em}}
\makeatother

\setcounter{secnumdepth}{4}
\setcounter{tocdepth}{4}


\setboolean{@twoside}{false}

% --------------


\newcommand{\before}[1]{\textcolor{red}{#1}}
\newcommand{\after}[1]{\textcolor{green}{#1}}

\newcommand{\old}[1]{\textcolor{orange}{#1}}
\newcommand{\rem}[1]{\textcolor{red}{#1}}
\newcommand{\todo}[1]{\textcolor{orange}{\newline \textit{\textbf{TODO:} #1}} \newline \newline }

\makeatletter
\newcommand{\linebreakand}{%
  \end{@IEEEauthorhalign}
  \hfill\mbox{}\par
  \mbox{}\hfill\begin{@IEEEauthorhalign}
}
\makeatother




% ---------------------------


\begin{document}

\title{A Method for Assisting Multi-agent System Design Using Reinforcement Learning}

% \IEEEaftertitletext{\vspace{-1\baselineskip}}

\author{

    \IEEEauthorblockN{Julien Soulé}
    \IEEEauthorblockA{\textit{Thales Land and Air Systems, BU IAS}}
    %Rennes, France \\
    \IEEEauthorblockA{\textit{Univ. Grenoble Alpes,} \\
        \textit{Grenoble INP, LCIS, 26000,}\\
        Valence, France \\
        julien.soule@lcis.grenoble-inp.fr}

    \and

    \IEEEauthorblockN{Jean-Paul Jamont\IEEEauthorrefmark{1}, Michel Occello\IEEEauthorrefmark{2}}
    \IEEEauthorblockA{\textit{Univ. Grenoble Alpes,} \\
        \textit{Grenoble INP, LCIS, 26000,}\\
        Valence, France \\
        \{\IEEEauthorrefmark{1}jean-paul.jamont,\IEEEauthorrefmark{2}michel.occello\}@lcis.grenoble-inp.fr
    }

    % \and

    % \IEEEauthorblockN{Michel Occello}
    % \IEEEauthorblockA{\textit{Univ. Grenoble Alpes,} \\
    % \textit{Grenoble INP, LCIS, 26000,}\\
    % Valence, France \\
    % michel.occello@lcis.grenoble-inp.fr}

    % \and

    \linebreakand

    \hspace{-0.5cm}
    \IEEEauthorblockN{Paul Théron}
    \IEEEauthorblockA{
        \hspace{-0.5cm}
        \textit{AICA IWG} \\
        \hspace{-0.5cm}
        La Guillermie, France \\
        \hspace{-0.5cm}
        %lieu-dit Le Bourg, France \\
        paul.theron@orange.fr}

    \and

    \hspace{0.5cm}
    \IEEEauthorblockN{Louis-Marie Traonouez}
    \IEEEauthorblockA{
        \hspace{0.5cm}
        \textit{Thales Land and Air Systems, BU IAS} \\
        \hspace{0.5cm}
        Rennes, France \\
        \hspace{0.5cm}
        louis-marie.traonouez@thalesgroup.com}}


\maketitle

\begin{abstract}
    Designing Multi-Agent Systems that can autonomously collaborate in dynamic environments is a significant challenge. Traditional Agent-Oriented Software Engineering (AOSE) approaches rely heavily on expert knowledge to define organizational structures, limiting scalability and adaptability. In response, we introduce SAMMASD, a semi-automated methodology that integrates Multi-Agent Reinforcement Learning with organizational constraints using the $\mathcal{M}OISE^+$ framework. SAMMASD offers a novel approach by leveraging MARL to automate the exploration and optimization of MAS design, which reduces the need for manual configuration and enhances flexibility. Our method includes phases for modeling, solving, analysis, and deployment, each utilizing MARL to streamline the development of role-based agent behaviors while ensuring safety and interpretability. We demonstrate the efficacy of SAMMASD in a warehouse management scenario, where it significantly reduces dependency on expert input and allows for scalable, adaptive MAS deployment. By blending organizational principles with MARL, SAMMASD bridges a critical gap in AOSE methodologies, offering a powerful tool for automating the design of complex MAS environments.
\end{abstract}

\begin{IEEEkeywords}
    Agent-oriented Software Engineering, Multi-Agent Reinforcement Learning, Assisted-Design, Organizational Models
\end{IEEEkeywords}


\section{Introduction}

Multi-Agent Systems (MAS) are increasingly applied in complex environments where autonomous agents must collaborate to achieve shared goals. The design and deployment of such systems, however, require carefully constructed organizational structures that guide agent behaviors effectively. Traditionally, Agent-Oriented Software Engineering (AOSE) methodologies rely heavily on expert input for defining roles, behaviors, and interactions among agents. While these methods provide robustness, they often struggle with scalability and adaptability when applied to dynamic and evolving environments.

Multi-Agent Reinforcement Learning (MARL) seems a promising approach to automate parts of MAS design, enabling agents to learn optimized behaviors for specific goals. However, existing MARL techniques often lack the mechanisms needed to incorporate explicit organizational constraints. Such constraints are essential for ensuring that agent behaviors align with higher-level goals related to safety, interpretability, and role-based accountability.

To address these challenges, we introduce the \textbf{SAMMASD} (Semi-Automated MARL-based MAS Design) methodology. SAMMASD bridges the gap between traditional AOSE and the potential of MARL by integrating the $\mathcal{M}OISE^+$~\cite{Hubner2007} organizational framework into the MARL learning process. This integration offers a way to automate the exploration and optimization of MAS designs while respecting organizational rules that enhance scalability, adaptability, and interpretability. SAMMASD provides a structured, semi-automated process that includes four phases: modeling, solving, analysis, and deployment.

The \textbf{modeling phase} involves defining the environment, goals, and organizational constraints in a way that accurately captures the operational context. In the \textbf{solving phase}, we leverage MARL algorithms to develop agent behaviors that maximize performance under these constraints. During the \textbf{analysis phase}, we infer roles and missions from learned behaviors, assessing their alignment with the specified organizational rules. Finally, the \textbf{deployment phase} focuses on transferring these optimized policies into real-world applications, ensuring the MAS meets both operational and safety requirements.

In this paper, we validate SAMMASD through a warehouse management scenario, assessing its capacity to:
%
\begin{enumerate}[label={\roman*)}]
  \item \textbf{Automate the MAS design process};
  \item \textbf{Reduce reliance on expert knowledge};
  \item \textbf{Facilitate scalable deployment}.
\end{enumerate}

\noindent The results indicate that SAMMASD can effectively bridge the gap between automated learning and traditional AOSE by providing a robust framework that enables both adaptability and adherence to organizational rules.


This work contributes to the ongoing development of AOSE methodologies by presenting a novel approach that combines the learning capabilities of MARL with structured organizational design principles. Through this integration, SAMMASD offers a tool for the automated and interpretable design of MAS.

The remained is organized as follows. \autoref{sec:related_works} shows that despite being away from MARL, AOSE methods could benefit from it. \autoref{sec:sammasd_presentation} introduces the SAMMASD method to address this gap by relying on a proposed MARL framework. \autoref{sec:evaluation} presents a SAMMASD implementation as tool for the experimental and evaluation framework to assess our method through a proposed environment. \autoref{sec:results_discussion} presents and discusses the results. \autoref{sec:conclusion_perspectives} concludes with the SAMMASD method and its perspectives.

\section{Related works}
\label{sec:related_works}

The field of AOSE has developed numerous methodologies for structuring, developing, and managing MAS. Traditional AOSE approaches, such as GAIA \cite{gaia1998}, ADELFE \cite{adelfe2002}, and INGENIAS \cite{ingenias2004}, have provided foundational frameworks that incorporate roles, interactions, and organizational models. While these methodologies offer robust structures, they largely depend on manual specification by experts, which can limit scalability and adaptability, particularly in dynamic environments.

In recent years, advancements in MARL have opened new possibilities for automating aspects of MAS design. MARL enables agents to autonomously learn behaviors that maximize their collective performance. However, despite its potential, MARL has been underutilized within AOSE methodologies for automating organizational aspects such as role allocation, mission inference, and organizational compliance.

Some research efforts have sought to bridge this gap. For example, Hammar \cite{hammar2019} has applied MARL to adapt policies dynamically in response to changing environments, but this work does not explicitly address how MARL could support AOSE's organizational needs. Thomas \cite{thomas2023} and Iocchi \cite{iocchi2023} have explored MARL frameworks in industrial MAS contexts, demonstrating their capacity for automating behavior adaptation and achieving specific goals. However, these studies do not fully integrate MARL within AOSE methodologies to facilitate role-based organizational design.

SAMMASD builds on these advances by embedding the $\mathcal{M}OISE^+$ organizational framework into MARL, enabling an approach that can automate the design of MAS while adhering to organizational constraints. Previous AOSE methodologies such as KB-ORG \cite{kborg2001} have used templates to simplify organizational design, but they still require manual configuration. By contrast, SAMMASD leverages MARL to infer roles and missions directly from agent behaviors, reducing the dependency on expert input and improving scalability.

Further, the use of $\mathcal{M}OISE^+$ within SAMMASD adds another dimension to existing MARL applications by incorporating safety and interpretability constraints. Techniques such as Constrained Policy Optimization \cite{zhao2024} and Safe Exploration \cite{melcer2024} have addressed safety in MARL, but they do not focus on enforcing organizational rules, which are essential for interpretable and compliant agent behavior. Shielding techniques, like those introduced by ElSayed-Aly et al. \cite{elsayed2021}, have also focused on preventing unsafe actions, yet they do not provide a complete organizational structure as $\mathcal{M}OISE^+$ does.

Unlike \textit{AGR}~\cite{ferber2003} (Agent Group Role), which does not explicitly handle goals as an organizational specification, the $\mathcal{M}OISE^+$ model addresses both roles and goals. Roles can be seen as sets of behavioral rules that agents must follow when interacting with each other or their environment to achieve their global goals. In contrast, goals represent states that agents aim to reach without a predefined path, thus requiring autonomous learning mechanisms. We believe that goals complement roles by enhancing stability and adaptability.

In summary, SAMMASD contributes to the AOSE and MARL fields by integrating these two areas in a way that enhances MAS design with automated, organization-aware learning capabilities. This integration not only addresses the need for scalable and adaptive MAS but also ensures that organizational rules and safety considerations are embedded in the learning process, making SAMMASD a novel methodology that aligns with modern AOSE goals.



\section{Proposed organizational MAS design method}
\label{sec:sammasd_presentation}

The core of the SAMMASD method is to consider the design of a MAS as a constrained optimization problem. The variable to be optimized is the \textbf{joint policy}, which represents the internal logic of the agents in determining their next actions. The goal is to maximize the cumulative reward under the given constraints. SAMMASD includes organizational specifications in MARL as additional constraints, allowing the user to control not only individual agent behaviors but also to manage the entire MAS through these specifications. We formalize this approach by proposing the MOISE+MARL framework and then using it to construct the four phases of our method.

\subsection{\textbf{MOISE+MARL framework}}

The MOISE+MARL framework guides agents in their learning process, ensuring they adhere to defined organizational roles and goals while achieving effective policies within their environment. This framework integrates a Decentralized Partially Observable Markov Decision Process (Dec-POMDP) with the $\mathcal{M}OISE^+$ organizational model, which specifies the roles, missions, and constraints that agents must respect.

\textbf{Dec-POMDP} \quad is defined by the tuple $D = \langle N, S, A, P, R, \Omega, O, \gamma \rangle$, where:
%
\begin{itemize*}[label={},itemjoin={; }]
    \item $\mathcal{A}$ is the set of $N$ agents in the system
    \item $S$ represents all possible states of the environment, including the agents
    \item $A = \times_{i=1}^N A_i$ is the set of actions available to each agent $i$
    \item $T: S \times A^N \to S$ is the state transition function that defines the next state $s'$ from a current state $s$ and joint action $a$
    \item $R: S \times A^N \times S \to \mathbb{R}$ is the reward function that assigns rewards based on transitions $(s, a, s')$
    \item $\Omega = \times_{i=1}^N \Omega_i$ is the set of observations available to each agent $i$
    \item $O: S \times A \to \Omega$ is the observation function that defines the observation $\omega$ received by an agent following action $a$ in state $s$
    \item $\gamma \in [0, 1]$ is the discount factor.
\end{itemize*}

\textbf{$\mathcal{M}OISE^+$ Organizational Model} \quad is represented by the tuple $\mathcal{OS} = \langle \mathcal{SS}, \mathcal{FS}, \mathcal{DS} \rangle$, where:
\begin{itemize*}[label={},itemjoin={; }]
    \item $\mathcal{SS} = \langle \mathcal{R} \rangle$ represents the structural specifications, defined by a set of roles $\mathcal{R}$
    \item $\mathcal{FS} = \langle \mathcal{G}, \mathcal{M}, mo \rangle$ represents functional specifications, including goals $\mathcal{G}$, missions $\mathcal{M}$, and a mapping function $mo: \mathcal{M} \to \mathcal{P}(\mathcal{G} \times [0,1])$ that associates missions with goals weighted by priority values
    \item $\mathcal{DS} = \mathcal{R} \times \mathcal{M} \times T_c \times \{0,1\}$ specifies deontic constraints as tuples $(\rho_a, m, t_c, p)$, where agents in role $\rho_a$ are permitted ($p=0$) or required ($p=1$) to undertake mission $m$ during specified time constraints $t_c$.
\end{itemize*}

The \textbf{Constraint Guides} \quad serve as key mechanisms that translate $\mathcal{M}OISE^+$ organizational elements into Dec-POMDP rules. These guides include:

\begin{itemize}
    \item \textbf{Role Action Guide} (RAG) \quad $rag: H \times \Omega \rightarrow \mathcal{P}(A \times \mathbb{R})$. This guide restricts an agent's actions based on its role, using a rule-based set of actions $A$ associated with each observation $\omega$ and a constraint hardness $ch \in [0,1]$. The constraint hardness controls how strictly the agent must follow role-specific actions.
    \item \textbf{Role Reward Guide} (RRG) \quad $rrg: H \times \Omega \times A \to \mathbb{R}$. This guide penalizes an agent if it chooses an action not allowed by its role, encouraging adherence to role-specific behaviors by reducing rewards when unauthorized actions are taken.
    \item \textbf{Goal Reward Guide} (GRG) \quad $grg: H \rightarrow \mathbb{R}$. This guide provides a bonus when an agent's actions align with predefined goal sequences, motivating agents to pursue goals while adhering to role and mission constraints.
\end{itemize}

Finally, \textbf{Linkers} are used to connect the $\mathcal{M}OISE^+$ organizational structure with the constraint guides:
\begin{itemize}
    \item \textbf{Agent to Role} (AR) \quad $ar: \mathcal{A} \to \mathcal{R}$ links each agent to a specific role.
    \item \textbf{Role to Constraint Guide} (RCG) \quad $rcg: \mathcal{R} \rightarrow rag \cup rrg$ maps roles to their respective action and reward guides.
    \item \textbf{Goal to Constraint Guide} (GCG) \quad $gcg: \mathcal{G} \rightarrow grg$ links each goal to its associated reward guide.
\end{itemize}

The value function for the MOISE+MARL framework, adapted to these constraint guides, is defined in \hyperref[eq:single_value_function]{Definition 1}. This value function accounts for role-specific (in red) and mission-specific adjustments (in blue), influencing both the action space and reward distribution based on the agent's assigned roles and missions.

\begin{figure*}[t]
  \label{eq:single_value_function}
  \raggedright
  \textbf{\quad Definition 1: Value function adapted to constraint guides in AEC.}
  \begin{gather*}
    \text{\quad \quad} V^{\pi^j}(s_t) = \hspace{-0.75cm} \sum_{\textcolor{red}{ \substack{a_{t} \in A \text{ if } rn() < ch_{t}, \\ 
    a_{t} \in A_{t} \text{ else}}
    }}{\hspace{-0.7cm} \pi_i(a_{t} | \omega_t)} \sum_{s_{t+1} \in S}{\hspace{-0.1cm} T(s_{t+1} | s_t, a_{t})[R(s_t,a_{t},s_{t+1}) + \hspace{-0.1cm} \textcolor{blue}{ \sum_{m \in \mathcal{M}_i}{ \hspace{-0.1cm} v_m(t) \frac{grg_m(h_{t+1})}{1 - p + \epsilon} } } + \textcolor{red}{(1-ch_t) \times rrg(\omega_t,a_{t+1})} + V^{\pi^j_{i+1 \ mod \ N}}(s_{t+1})]}
  \end{gather*}  
  %
  \textcolor{red}{\[\text{With } rag(h_t, \omega_t) = A_{t} \times \mathbb{R} \text{, } (a_t, ch_{t}) \in A_{t} \times \mathbb{R} \text{ ; } \text{ and } rn: \emptyset \to [0,1[ \text{, a uniform random function}\]}
  %
  \vspace{-0.5cm}
  \textcolor{blue}{
  \begin{gather*}
  \text{With } \omega_t = O(\omega_t | s_t, a_t) \text{ ; } h_t = \{h_0 = \langle \rangle, h_{t+1} = \langle h_t, (\omega_{t+1}, a_{t+1}) \rangle \} \text{ ; } grg_m(h) = \hspace{-0.8cm} \sum_{(grg_i,w_i) \in mo(m)}{\hspace{-0.8cm} w_i \times grg_i(h)} \text{ ; } \epsilon \in \mathbb{R}_{>0} \text{ ; }
  \end{gather*}
  }
  \vspace{-0.75cm}
  \textcolor{blue}{
  \begin{gather*}
  v_m(t) = \{ 1 \text{ if } t \in t_c \text{ ; else } 0 \} \text{ ; } \text{ and } \mathcal{M}_i = \{m_j | (ar(i),m_j,t_c,p) \in \mathcal{M}\}
  \end{gather*}
  }
  \vspace{-0.6cm}
  \end{figure*}

The MOISE+MARL framework, denoted as $MM = \langle D, \mathcal{OS}, ar, rcg, \allowbreak gcg, rag, rrg, grg \rangle$, enables agents to maximize the value function $V^{\pi^{j}}$ while following role-based and goal-oriented constraints.

\noindent Agents thus follow trajectories $h \in H$ that respect the rules of their assigned roles and missions, ensuring both individual and collective alignment with the system's overall goals.


\subsection{General overview of the method}

Our SAMMASD method is built around four main phases: (1) modeling the environment, goal, and organizational constraints according to the proposed framework, (2) learning policies using various MARL algorithms, (3) analyzing behaviors and inferring organizational specifications with a proposed method, and (4) developing and deploying the MAS. This approach guides the agent learning process while enforcing strict organizational constraints, ensuring the efficiency of the learned policies. The lifecycle of a SAMMASD-designed MAS is illustrated in \autoref{fig:cycle}.

% Formal description of the phases


\begin{figure}[h!]
  \centering
  


\tikzset{every picture/.style={line width=0.75pt}} %set default line width to 0.75pt        

\begin{tikzpicture}[x=0.75pt,y=0.75pt,yscale=-1,xscale=1]
%uncomment if require: \path (0,3307); %set diagram left start at 0, and has height of 3307

%Shape: Smiley Face [id:dp29065495216725257] 
\draw  [line width=1.5]  (85.38,2800.11) .. controls (85.38,2797.7) and (87.16,2795.75) .. (89.36,2795.75) .. controls (91.55,2795.75) and (93.34,2797.7) .. (93.34,2800.11) .. controls (93.34,2802.52) and (91.55,2804.48) .. (89.36,2804.48) .. controls (87.16,2804.48) and (85.38,2802.52) .. (85.38,2800.11) -- cycle ; \draw  [line width=1.5]  (87.61,2798.63) .. controls (87.61,2798.39) and (87.78,2798.19) .. (88,2798.19) .. controls (88.22,2798.19) and (88.4,2798.39) .. (88.4,2798.63) .. controls (88.4,2798.87) and (88.22,2799.07) .. (88,2799.07) .. controls (87.78,2799.07) and (87.61,2798.87) .. (87.61,2798.63) -- cycle ; \draw  [line width=1.5]  (90.31,2798.63) .. controls (90.31,2798.39) and (90.49,2798.19) .. (90.71,2798.19) .. controls (90.93,2798.19) and (91.11,2798.39) .. (91.11,2798.63) .. controls (91.11,2798.87) and (90.93,2799.07) .. (90.71,2799.07) .. controls (90.49,2799.07) and (90.31,2798.87) .. (90.31,2798.63) -- cycle ; \draw  [line width=1.5]  (87.37,2801.86) .. controls (88.69,2803.02) and (90.02,2803.02) .. (91.35,2801.86) ;
%Shape: Rectangle [id:dp42672371521059915] 
\draw  [dash pattern={on 5.63pt off 4.5pt}][line width=1.5]  (74.03,2763.75) -- (192,2763.75) -- (192,2813.93) -- (74.03,2813.93) -- cycle ;
%Shape: Smiley Face [id:dp9817389082285293] 
\draw  [line width=1.5]  (144.45,2803.6) .. controls (144.45,2801.19) and (146.24,2799.24) .. (148.43,2799.24) .. controls (150.63,2799.24) and (152.41,2801.19) .. (152.41,2803.6) .. controls (152.41,2806.01) and (150.63,2807.97) .. (148.43,2807.97) .. controls (146.24,2807.97) and (144.45,2806.01) .. (144.45,2803.6) -- cycle ; \draw  [line width=1.5]  (146.68,2802.12) .. controls (146.68,2801.88) and (146.86,2801.68) .. (147.08,2801.68) .. controls (147.3,2801.68) and (147.48,2801.88) .. (147.48,2802.12) .. controls (147.48,2802.36) and (147.3,2802.56) .. (147.08,2802.56) .. controls (146.86,2802.56) and (146.68,2802.36) .. (146.68,2802.12) -- cycle ; \draw  [line width=1.5]  (149.39,2802.12) .. controls (149.39,2801.88) and (149.57,2801.68) .. (149.79,2801.68) .. controls (150.01,2801.68) and (150.18,2801.88) .. (150.18,2802.12) .. controls (150.18,2802.36) and (150.01,2802.56) .. (149.79,2802.56) .. controls (149.57,2802.56) and (149.39,2802.36) .. (149.39,2802.12) -- cycle ; \draw  [line width=1.5]  (146.44,2805.35) .. controls (147.77,2806.51) and (149.1,2806.51) .. (150.42,2805.35) ;
%Shape: Smiley Face [id:dp49419175504212776] 
\draw  [line width=1.5]  (179.09,2781.5) .. controls (179.09,2779.09) and (180.87,2777.13) .. (183.06,2777.13) .. controls (185.26,2777.13) and (187.04,2779.09) .. (187.04,2781.5) .. controls (187.04,2783.91) and (185.26,2785.86) .. (183.06,2785.86) .. controls (180.87,2785.86) and (179.09,2783.91) .. (179.09,2781.5) -- cycle ; \draw  [line width=1.5]  (181.31,2780.01) .. controls (181.31,2779.77) and (181.49,2779.58) .. (181.71,2779.58) .. controls (181.93,2779.58) and (182.11,2779.77) .. (182.11,2780.01) .. controls (182.11,2780.25) and (181.93,2780.45) .. (181.71,2780.45) .. controls (181.49,2780.45) and (181.31,2780.25) .. (181.31,2780.01) -- cycle ; \draw  [line width=1.5]  (184.02,2780.01) .. controls (184.02,2779.77) and (184.2,2779.58) .. (184.42,2779.58) .. controls (184.64,2779.58) and (184.81,2779.77) .. (184.81,2780.01) .. controls (184.81,2780.25) and (184.64,2780.45) .. (184.42,2780.45) .. controls (184.2,2780.45) and (184.02,2780.25) .. (184.02,2780.01) -- cycle ; \draw  [line width=1.5]  (181.07,2783.24) .. controls (182.4,2784.4) and (183.73,2784.4) .. (185.05,2783.24) ;
%Flowchart: Punched Tape [id:dp3565745198144521] 
\draw  [fill={rgb, 255:red, 255; green, 255; blue, 255 }  ,fill opacity=1 ] (291.67,2877.34) .. controls (291.67,2880.23) and (301.36,2882.58) .. (313.31,2882.58) .. controls (325.26,2882.58) and (334.95,2880.23) .. (334.95,2877.34) .. controls (334.95,2874.45) and (344.64,2872.11) .. (356.6,2872.11) .. controls (368.55,2872.11) and (378.24,2874.45) .. (378.24,2877.34) -- (378.24,2919.23) .. controls (378.24,2916.34) and (368.55,2913.99) .. (356.6,2913.99) .. controls (344.64,2913.99) and (334.95,2916.34) .. (334.95,2919.23) .. controls (334.95,2922.12) and (325.26,2924.46) .. (313.31,2924.46) .. controls (301.36,2924.46) and (291.67,2922.12) .. (291.67,2919.23) -- cycle ;
%Straight Lines [id:da23451091058783402] 
\draw [line width=1.5]    (320.63,2891.89) -- (349.47,2889.91) ;
\draw [shift={(352.46,2889.7)}, rotate = 176.08] [color={rgb, 255:red, 0; green, 0; blue, 0 }  ][line width=1.5]    (8.53,-2.57) .. controls (5.42,-1.09) and (2.58,-0.23) .. (0,0) .. controls (2.58,0.23) and (5.42,1.09) .. (8.53,2.57)   ;
%Straight Lines [id:da05993633349010663] 
\draw [line width=1.5]    (320.63,2894.07) -- (335.84,2901.48) ;
\draw [shift={(338.53,2902.79)}, rotate = 205.98] [color={rgb, 255:red, 0; green, 0; blue, 0 }  ][line width=1.5]    (8.53,-2.57) .. controls (5.42,-1.09) and (2.58,-0.23) .. (0,0) .. controls (2.58,0.23) and (5.42,1.09) .. (8.53,2.57)   ;
%Shape: Smiley Face [id:dp5316832937595011] 
\draw  [line width=1.5]  (312.91,2893.34) .. controls (312.91,2890.93) and (314.69,2888.98) .. (316.89,2888.98) .. controls (319.09,2888.98) and (320.87,2890.93) .. (320.87,2893.34) .. controls (320.87,2895.75) and (319.09,2897.7) .. (316.89,2897.7) .. controls (314.69,2897.7) and (312.91,2895.75) .. (312.91,2893.34) -- cycle ; \draw  [line width=1.5]  (315.14,2891.86) .. controls (315.14,2891.61) and (315.32,2891.42) .. (315.54,2891.42) .. controls (315.76,2891.42) and (315.94,2891.61) .. (315.94,2891.86) .. controls (315.94,2892.1) and (315.76,2892.29) .. (315.54,2892.29) .. controls (315.32,2892.29) and (315.14,2892.1) .. (315.14,2891.86) -- cycle ; \draw  [line width=1.5]  (317.85,2891.86) .. controls (317.85,2891.61) and (318.02,2891.42) .. (318.24,2891.42) .. controls (318.46,2891.42) and (318.64,2891.61) .. (318.64,2891.86) .. controls (318.64,2892.1) and (318.46,2892.29) .. (318.24,2892.29) .. controls (318.02,2892.29) and (317.85,2892.1) .. (317.85,2891.86) -- cycle ; \draw  [line width=1.5]  (314.9,2895.08) .. controls (316.23,2896.25) and (317.55,2896.25) .. (318.88,2895.08) ;
%Shape: Smiley Face [id:dp5491508300746957] 
\draw  [line width=1.5]  (338.38,2904.97) .. controls (338.38,2902.56) and (340.16,2900.61) .. (342.35,2900.61) .. controls (344.55,2900.61) and (346.33,2902.56) .. (346.33,2904.97) .. controls (346.33,2907.38) and (344.55,2909.34) .. (342.35,2909.34) .. controls (340.16,2909.34) and (338.38,2907.38) .. (338.38,2904.97) -- cycle ; \draw  [line width=1.5]  (340.6,2903.49) .. controls (340.6,2903.25) and (340.78,2903.05) .. (341,2903.05) .. controls (341.22,2903.05) and (341.4,2903.25) .. (341.4,2903.49) .. controls (341.4,2903.73) and (341.22,2903.93) .. (341,2903.93) .. controls (340.78,2903.93) and (340.6,2903.73) .. (340.6,2903.49) -- cycle ; \draw  [line width=1.5]  (343.31,2903.49) .. controls (343.31,2903.25) and (343.49,2903.05) .. (343.71,2903.05) .. controls (343.93,2903.05) and (344.1,2903.25) .. (344.1,2903.49) .. controls (344.1,2903.73) and (343.93,2903.93) .. (343.71,2903.93) .. controls (343.49,2903.93) and (343.31,2903.73) .. (343.31,2903.49) -- cycle ; \draw  [line width=1.5]  (340.36,2906.72) .. controls (341.69,2907.88) and (343.02,2907.88) .. (344.34,2906.72) ;
%Shape: Smiley Face [id:dp21362593128550156] 
\draw  [line width=1.5]  (352.64,2888.69) .. controls (352.64,2886.28) and (354.42,2884.32) .. (356.61,2884.32) .. controls (358.81,2884.32) and (360.59,2886.28) .. (360.59,2888.69) .. controls (360.59,2891.1) and (358.81,2893.05) .. (356.61,2893.05) .. controls (354.42,2893.05) and (352.64,2891.1) .. (352.64,2888.69) -- cycle ; \draw  [line width=1.5]  (354.86,2887.2) .. controls (354.86,2886.96) and (355.04,2886.77) .. (355.26,2886.77) .. controls (355.48,2886.77) and (355.66,2886.96) .. (355.66,2887.2) .. controls (355.66,2887.44) and (355.48,2887.64) .. (355.26,2887.64) .. controls (355.04,2887.64) and (354.86,2887.44) .. (354.86,2887.2) -- cycle ; \draw  [line width=1.5]  (357.57,2887.2) .. controls (357.57,2886.96) and (357.75,2886.77) .. (357.97,2886.77) .. controls (358.19,2886.77) and (358.36,2886.96) .. (358.36,2887.2) .. controls (358.36,2887.44) and (358.19,2887.64) .. (357.97,2887.64) .. controls (357.75,2887.64) and (357.57,2887.44) .. (357.57,2887.2) -- cycle ; \draw  [line width=1.5]  (354.62,2890.43) .. controls (355.95,2891.59) and (357.28,2891.59) .. (358.6,2890.43) ;
%Left Arrow [id:dp22187584774212898] 
\draw   (215,2804.55) -- (220.28,2802) -- (220.28,2803.27) -- (263.54,2803.27) -- (263.54,2805.82) -- (220.28,2805.82) -- (220.28,2807.09) -- cycle ;
%Left Arrow [id:dp1861077704673879] 
\draw   (315.35,2834) -- (317.89,2837.8) -- (316.62,2837.8) -- (316.62,2868.91) -- (314.07,2868.91) -- (314.07,2837.8) -- (312.8,2837.8) -- cycle ;
%Left Arrow [id:dp2590948740182193] 
\draw   (130.55,2868.91) -- (128,2865.11) -- (129.27,2865.11) -- (129.27,2834) -- (131.82,2834) -- (131.82,2865.11) -- (133.09,2865.11) -- cycle ;
%Left Arrow [id:dp7631269314674067] 
\draw   (262.54,2900.55) -- (257.26,2903.09) -- (257.26,2901.82) -- (214,2901.82) -- (214,2899.27) -- (257.26,2899.27) -- (257.26,2898) -- cycle ;
%Shape: Arc [id:dp8010751146858193] 
\draw  [draw opacity=0] (78.55,2898.86) .. controls (77.97,2897.43) and (79.7,2895.07) .. (82.41,2893.59) .. controls (85.13,2892.11) and (87.81,2892.08) .. (88.39,2893.51) -- (83.47,2896.19) -- cycle ; \draw   (78.55,2898.86) .. controls (77.97,2897.43) and (79.7,2895.07) .. (82.41,2893.59) .. controls (85.13,2892.11) and (87.81,2892.08) .. (88.39,2893.51) ;  
%Shape: Arc [id:dp2168479262754166] 
\draw  [draw opacity=0] (79.96,2900.21) .. controls (79.37,2898.78) and (80.79,2896.59) .. (83.12,2895.32) .. controls (85.45,2894.06) and (87.81,2894.19) .. (88.39,2895.63) -- (84.17,2897.92) -- cycle ; \draw   (79.96,2900.21) .. controls (79.37,2898.78) and (80.79,2896.59) .. (83.12,2895.32) .. controls (85.45,2894.06) and (87.81,2894.19) .. (88.39,2895.63) ;  
%Shape: Arc [id:dp1657064934185728] 
\draw  [draw opacity=0] (81.36,2901.56) .. controls (81.36,2901.56) and (81.36,2901.56) .. (81.36,2901.56) .. controls (80.78,2900.13) and (81.88,2898.11) .. (83.82,2897.06) .. controls (85.76,2896) and (87.81,2896.31) .. (88.39,2897.74) -- (84.88,2899.65) -- cycle ; \draw   (81.36,2901.56) .. controls (81.36,2901.56) and (81.36,2901.56) .. (81.36,2901.56) .. controls (80.78,2900.13) and (81.88,2898.11) .. (83.82,2897.06) .. controls (85.76,2896) and (87.81,2896.31) .. (88.39,2897.74) ;  
%Shape: Arc [id:dp6696163073703636] 
\draw  [draw opacity=0] (82.77,2902.92) .. controls (82.77,2902.92) and (82.77,2902.92) .. (82.77,2902.92) .. controls (82.77,2902.92) and (82.77,2902.92) .. (82.77,2902.92) .. controls (82.19,2901.48) and (82.97,2899.63) .. (84.53,2898.79) .. controls (86.08,2897.94) and (87.81,2898.42) .. (88.39,2899.86) -- (85.58,2901.39) -- cycle ; \draw   (82.77,2902.92) .. controls (82.77,2902.92) and (82.77,2902.92) .. (82.77,2902.92) .. controls (82.77,2902.92) and (82.77,2902.92) .. (82.77,2902.92) .. controls (82.19,2901.48) and (82.97,2899.63) .. (84.53,2898.79) .. controls (86.08,2897.94) and (87.81,2898.42) .. (88.39,2899.86) ;  
%Shape: Arc [id:dp5914598807756752] 
\draw  [draw opacity=0] (84.18,2904.27) .. controls (83.6,2902.83) and (84.07,2901.15) .. (85.23,2900.52) .. controls (86.4,2899.89) and (87.81,2900.54) .. (88.4,2901.97) -- (86.29,2903.12) -- cycle ; \draw   (84.18,2904.27) .. controls (83.6,2902.83) and (84.07,2901.15) .. (85.23,2900.52) .. controls (86.4,2899.89) and (87.81,2900.54) .. (88.4,2901.97) ;  

%Image [id:dp3722282424817167] 
\draw (291.67,2795.75) node  {\includegraphics[width=7.64pt,height=13.09pt]{figures/robot.png}};
%Shape: Rectangle [id:dp9197785817800539] 
\draw  [line width=1.5]  (275.37,2763.75) -- (390.8,2763.75) -- (390.8,2813.93) -- (275.37,2813.93) -- cycle ;
%Image [id:dp9715658782589778] 
\draw (382.32,2779.46) node  {\includegraphics[width=7.64pt,height=13.09pt]{figures/robot.png}};
%Image [id:dp635616861971029] 
\draw (352.78,2801.57) node  {\includegraphics[width=7.64pt,height=13.09pt]{figures/robot.png}};
%Shape: Rectangle [id:dp647928357040308] 
\draw  [fill={rgb, 255:red, 0; green, 0; blue, 0 }  ,fill opacity=1 ] (291.67,2769.57) -- (301.85,2769.57) -- (301.85,2781.21) -- (291.67,2781.21) -- cycle ;
%Shape: Rectangle [id:dp9626828362725837] 
\draw  [fill={rgb, 255:red, 0; green, 0; blue, 0 }  ,fill opacity=1 ] (373.15,2792.84) -- (383.33,2792.84) -- (383.33,2804.48) -- (373.15,2804.48) -- cycle ;
%Shape: Ellipse [id:dp6171740062199291] 
\draw  [fill={rgb, 255:red, 0; green, 0; blue, 0 }  ,fill opacity=1 ] (347.69,2775.39) .. controls (347.69,2772.17) and (349.97,2769.57) .. (352.78,2769.57) .. controls (355.59,2769.57) and (357.87,2772.17) .. (357.87,2775.39) .. controls (357.87,2778.6) and (355.59,2781.21) .. (352.78,2781.21) .. controls (349.97,2781.21) and (347.69,2778.6) .. (347.69,2775.39) -- cycle ;
%Shape: Triangle [id:dp8145134127966778] 
\draw  [fill={rgb, 255:red, 0; green, 0; blue, 0 }  ,fill opacity=1 ] (322.22,2792.84) -- (327.31,2804.48) -- (317.13,2804.48) -- cycle ;
%Shape: Rectangle [id:dp07981685971419106] 
\draw  [fill={rgb, 255:red, 0; green, 0; blue, 0 }  ,fill opacity=1 ] (89.45,2769.57) -- (99.64,2769.57) -- (99.64,2781.21) -- (89.45,2781.21) -- cycle ;
%Shape: Rectangle [id:dp9786998324005067] 
\draw  [fill={rgb, 255:red, 0; green, 0; blue, 0 }  ,fill opacity=1 ] (170.94,2792.84) -- (181.12,2792.84) -- (181.12,2804.48) -- (170.94,2804.48) -- cycle ;
%Shape: Ellipse [id:dp6465785854464419] 
\draw  [fill={rgb, 255:red, 0; green, 0; blue, 0 }  ,fill opacity=1 ] (145.47,2775.39) .. controls (145.47,2772.17) and (147.75,2769.57) .. (150.57,2769.57) .. controls (153.38,2769.57) and (155.66,2772.17) .. (155.66,2775.39) .. controls (155.66,2778.6) and (153.38,2781.21) .. (150.57,2781.21) .. controls (147.75,2781.21) and (145.47,2778.6) .. (145.47,2775.39) -- cycle ;
%Shape: Triangle [id:dp5909890868954251] 
\draw  [fill={rgb, 255:red, 0; green, 0; blue, 0 }  ,fill opacity=1 ] (120.01,2792.84) -- (125.1,2804.48) -- (114.92,2804.48) -- cycle ;
%Shape: Smiley Face [id:dp661163164093121] 
\draw  [line width=1.5]  (85.52,2909.38) .. controls (85.52,2906.98) and (87.3,2905.03) .. (89.5,2905.03) .. controls (91.7,2905.03) and (93.48,2906.98) .. (93.48,2909.38) .. controls (93.48,2911.78) and (91.7,2913.73) .. (89.5,2913.73) .. controls (87.3,2913.73) and (85.52,2911.78) .. (85.52,2909.38) -- cycle ; \draw  [line width=1.5]  (87.75,2907.9) .. controls (87.75,2907.66) and (87.93,2907.46) .. (88.15,2907.46) .. controls (88.37,2907.46) and (88.55,2907.66) .. (88.55,2907.9) .. controls (88.55,2908.14) and (88.37,2908.33) .. (88.15,2908.33) .. controls (87.93,2908.33) and (87.75,2908.14) .. (87.75,2907.9) -- cycle ; \draw  [line width=1.5]  (90.46,2907.9) .. controls (90.46,2907.66) and (90.63,2907.46) .. (90.85,2907.46) .. controls (91.07,2907.46) and (91.25,2907.66) .. (91.25,2907.9) .. controls (91.25,2908.14) and (91.07,2908.33) .. (90.85,2908.33) .. controls (90.63,2908.33) and (90.46,2908.14) .. (90.46,2907.9) -- cycle ; \draw  [line width=1.5]  (87.51,2911.12) .. controls (88.84,2912.28) and (90.16,2912.28) .. (91.49,2911.12) ;
%Shape: Rectangle [id:dp9256921796782376] 
\draw  [dash pattern={on 5.63pt off 4.5pt}][line width=1.5]  (74.17,2873.12) -- (192,2873.12) -- (192,2923.15) -- (74.17,2923.15) -- cycle ;
%Shape: Smiley Face [id:dp12230401154700177] 
\draw  [line width=1.5]  (144.6,2912.86) .. controls (144.6,2910.46) and (146.38,2908.51) .. (148.58,2908.51) .. controls (150.77,2908.51) and (152.56,2910.46) .. (152.56,2912.86) .. controls (152.56,2915.26) and (150.77,2917.21) .. (148.58,2917.21) .. controls (146.38,2917.21) and (144.6,2915.26) .. (144.6,2912.86) -- cycle ; \draw  [line width=1.5]  (146.83,2911.38) .. controls (146.83,2911.14) and (147,2910.94) .. (147.22,2910.94) .. controls (147.44,2910.94) and (147.62,2911.14) .. (147.62,2911.38) .. controls (147.62,2911.62) and (147.44,2911.81) .. (147.22,2911.81) .. controls (147,2911.81) and (146.83,2911.62) .. (146.83,2911.38) -- cycle ; \draw  [line width=1.5]  (149.53,2911.38) .. controls (149.53,2911.14) and (149.71,2910.94) .. (149.93,2910.94) .. controls (150.15,2910.94) and (150.33,2911.14) .. (150.33,2911.38) .. controls (150.33,2911.62) and (150.15,2911.81) .. (149.93,2911.81) .. controls (149.71,2911.81) and (149.53,2911.62) .. (149.53,2911.38) -- cycle ; \draw  [line width=1.5]  (146.59,2914.6) .. controls (147.91,2915.76) and (149.24,2915.76) .. (150.57,2914.6) ;
%Shape: Smiley Face [id:dp8847243900502049] 
\draw  [line width=1.5]  (179.23,2890.23) .. controls (179.23,2887.83) and (181.01,2885.88) .. (183.21,2885.88) .. controls (185.4,2885.88) and (187.19,2887.83) .. (187.19,2890.23) .. controls (187.19,2892.63) and (185.4,2894.58) .. (183.21,2894.58) .. controls (181.01,2894.58) and (179.23,2892.63) .. (179.23,2890.23) -- cycle ; \draw  [line width=1.5]  (181.46,2888.75) .. controls (181.46,2888.51) and (181.63,2888.32) .. (181.85,2888.32) .. controls (182.07,2888.32) and (182.25,2888.51) .. (182.25,2888.75) .. controls (182.25,2888.99) and (182.07,2889.19) .. (181.85,2889.19) .. controls (181.63,2889.19) and (181.46,2888.99) .. (181.46,2888.75) -- cycle ; \draw  [line width=1.5]  (184.16,2888.75) .. controls (184.16,2888.51) and (184.34,2888.32) .. (184.56,2888.32) .. controls (184.78,2888.32) and (184.96,2888.51) .. (184.96,2888.75) .. controls (184.96,2888.99) and (184.78,2889.19) .. (184.56,2889.19) .. controls (184.34,2889.19) and (184.16,2888.99) .. (184.16,2888.75) -- cycle ; \draw  [line width=1.5]  (181.22,2891.97) .. controls (182.54,2893.13) and (183.87,2893.13) .. (185.2,2891.97) ;
%Shape: Rectangle [id:dp5525291488755686] 
\draw  [fill={rgb, 255:red, 0; green, 0; blue, 0 }  ,fill opacity=1 ] (89.6,2878.92) -- (99.78,2878.92) -- (99.78,2890.53) -- (89.6,2890.53) -- cycle ;
%Shape: Rectangle [id:dp35042622253694655] 
\draw  [fill={rgb, 255:red, 0; green, 0; blue, 0 }  ,fill opacity=1 ] (171.08,2902.13) -- (181.27,2902.13) -- (181.27,2913.73) -- (171.08,2913.73) -- cycle ;
%Shape: Ellipse [id:dp9658079314838142] 
\draw  [fill={rgb, 255:red, 0; green, 0; blue, 0 }  ,fill opacity=1 ] (145.62,2884.72) .. controls (145.62,2881.52) and (147.9,2878.92) .. (150.71,2878.92) .. controls (153.52,2878.92) and (155.8,2881.52) .. (155.8,2884.72) .. controls (155.8,2887.93) and (153.52,2890.53) .. (150.71,2890.53) .. controls (147.9,2890.53) and (145.62,2887.93) .. (145.62,2884.72) -- cycle ;
%Shape: Triangle [id:dp5926435260290868] 
\draw  [fill={rgb, 255:red, 0; green, 0; blue, 0 }  ,fill opacity=1 ] (120.15,2902.13) -- (125.25,2913.73) -- (115.06,2913.73) -- cycle ;
%Shape: Arc [id:dp2058241396036773] 
\draw  [draw opacity=0] (133.56,2911.66) .. controls (132.26,2911.1) and (132.06,2908.03) .. (133.1,2904.81) .. controls (134.15,2901.58) and (136.04,2899.43) .. (137.34,2899.99) -- (135.45,2905.83) -- cycle ; \draw   (133.56,2911.66) .. controls (132.26,2911.1) and (132.06,2908.03) .. (133.1,2904.81) .. controls (134.15,2901.58) and (136.04,2899.43) .. (137.34,2899.99) ;  
%Shape: Arc [id:dp9303770446429336] 
\draw  [draw opacity=0] (135.39,2911.51) .. controls (135.39,2911.51) and (135.39,2911.51) .. (135.39,2911.51) .. controls (135.39,2911.51) and (135.39,2911.51) .. (135.39,2911.51) .. controls (134.1,2910.95) and (133.77,2908.25) .. (134.67,2905.49) .. controls (135.56,2902.72) and (137.34,2900.94) .. (138.63,2901.5) -- (137.01,2906.51) -- cycle ; \draw   (135.39,2911.51) .. controls (135.39,2911.51) and (135.39,2911.51) .. (135.39,2911.51) .. controls (135.39,2911.51) and (135.39,2911.51) .. (135.39,2911.51) .. controls (134.1,2910.95) and (133.77,2908.25) .. (134.67,2905.49) .. controls (135.56,2902.72) and (137.34,2900.94) .. (138.63,2901.5) ;  
%Shape: Arc [id:dp23450230368676772] 
\draw  [draw opacity=0] (137.22,2911.35) .. controls (137.22,2911.35) and (137.22,2911.35) .. (137.22,2911.35) .. controls (137.22,2911.35) and (137.22,2911.35) .. (137.22,2911.35) .. controls (135.93,2910.79) and (135.48,2908.47) .. (136.23,2906.17) .. controls (136.98,2903.86) and (138.63,2902.45) .. (139.93,2903.02) -- (138.58,2907.19) -- cycle ; \draw   (137.22,2911.35) .. controls (137.22,2911.35) and (137.22,2911.35) .. (137.22,2911.35) .. controls (137.22,2911.35) and (137.22,2911.35) .. (137.22,2911.35) .. controls (135.93,2910.79) and (135.48,2908.47) .. (136.23,2906.17) .. controls (136.98,2903.86) and (138.63,2902.45) .. (139.93,2903.02) ;  
%Shape: Arc [id:dp32480365085094887] 
\draw  [draw opacity=0] (139.06,2911.2) .. controls (139.06,2911.2) and (139.06,2911.2) .. (139.06,2911.2) .. controls (137.76,2910.64) and (137.2,2908.69) .. (137.79,2906.85) .. controls (138.39,2905) and (139.92,2903.97) .. (141.22,2904.53) -- (140.14,2907.87) -- cycle ; \draw   (139.06,2911.2) .. controls (139.06,2911.2) and (139.06,2911.2) .. (139.06,2911.2) .. controls (137.76,2910.64) and (137.2,2908.69) .. (137.79,2906.85) .. controls (138.39,2905) and (139.92,2903.97) .. (141.22,2904.53) ;  
%Shape: Arc [id:dp7108436649867611] 
\draw  [draw opacity=0] (140.89,2911.05) .. controls (139.6,2910.48) and (138.91,2908.91) .. (139.36,2907.53) .. controls (139.8,2906.14) and (141.22,2905.48) .. (142.51,2906.05) -- (141.7,2908.55) -- cycle ; \draw   (140.89,2911.05) .. controls (139.6,2910.48) and (138.91,2908.91) .. (139.36,2907.53) .. controls (139.8,2906.14) and (141.22,2905.48) .. (142.51,2906.05) ;  

%Shape: Arc [id:dp7234198948762418] 
\draw  [draw opacity=0] (171.31,2898) .. controls (171.31,2898) and (171.31,2898) .. (171.31,2898) .. controls (169.93,2898.17) and (168.53,2895.53) .. (168.19,2892.12) .. controls (167.85,2888.7) and (168.7,2885.8) .. (170.08,2885.64) .. controls (170.08,2885.64) and (170.08,2885.64) .. (170.08,2885.64) -- (170.69,2891.82) -- cycle ; \draw   (171.31,2898) .. controls (171.31,2898) and (171.31,2898) .. (171.31,2898) .. controls (169.93,2898.17) and (168.53,2895.53) .. (168.19,2892.12) .. controls (167.85,2888.7) and (168.7,2885.8) .. (170.08,2885.64) .. controls (170.08,2885.64) and (170.08,2885.64) .. (170.08,2885.64) ;  
%Shape: Arc [id:dp05399242918401237] 
\draw  [draw opacity=0] (172.88,2896.92) .. controls (172.88,2896.92) and (172.88,2896.92) .. (172.88,2896.92) .. controls (171.5,2897.09) and (170.15,2894.85) .. (169.86,2891.92) .. controls (169.57,2888.99) and (170.45,2886.49) .. (171.83,2886.32) -- (172.35,2891.62) -- cycle ; \draw   (172.88,2896.92) .. controls (172.88,2896.92) and (172.88,2896.92) .. (172.88,2896.92) .. controls (171.5,2897.09) and (170.15,2894.85) .. (169.86,2891.92) .. controls (169.57,2888.99) and (170.45,2886.49) .. (171.83,2886.32) ;  
%Shape: Arc [id:dp7827225311205266] 
\draw  [draw opacity=0] (174.46,2895.84) .. controls (174.46,2895.84) and (174.46,2895.84) .. (174.46,2895.84) .. controls (173.08,2896) and (171.76,2894.16) .. (171.52,2891.72) .. controls (171.28,2889.28) and (172.2,2887.17) .. (173.58,2887.01) -- (174.02,2891.42) -- cycle ; \draw   (174.46,2895.84) .. controls (174.46,2895.84) and (174.46,2895.84) .. (174.46,2895.84) .. controls (173.08,2896) and (171.76,2894.16) .. (171.52,2891.72) .. controls (171.28,2889.28) and (172.2,2887.17) .. (173.58,2887.01) ;  
%Shape: Arc [id:dp9906438850599013] 
\draw  [draw opacity=0] (176.03,2894.76) .. controls (174.65,2894.92) and (173.38,2893.47) .. (173.19,2891.52) .. controls (172.99,2889.57) and (173.95,2887.86) .. (175.33,2887.69) -- (175.68,2891.23) -- cycle ; \draw   (176.03,2894.76) .. controls (174.65,2894.92) and (173.38,2893.47) .. (173.19,2891.52) .. controls (172.99,2889.57) and (173.95,2887.86) .. (175.33,2887.69) ;  
%Shape: Arc [id:dp545106976508444] 
\draw  [draw opacity=0] (177.61,2893.68) .. controls (177.61,2893.68) and (177.61,2893.68) .. (177.61,2893.68) .. controls (176.23,2893.84) and (174.99,2892.79) .. (174.85,2891.33) .. controls (174.7,2889.86) and (175.7,2888.54) .. (177.08,2888.38) -- (177.34,2891.03) -- cycle ; \draw   (177.61,2893.68) .. controls (177.61,2893.68) and (177.61,2893.68) .. (177.61,2893.68) .. controls (176.23,2893.84) and (174.99,2892.79) .. (174.85,2891.33) .. controls (174.7,2889.86) and (175.7,2888.54) .. (177.08,2888.38) ;  

%Down Arrow [id:dp8971518008111754] 
\draw   (230,2776) -- (232.5,2776) -- (232.5,2764) -- (237.5,2764) -- (237.5,2776) -- (240,2776) -- (235,2784) -- cycle ;


% Text Node
\draw (187.23,2773.35) node  [font=\scriptsize] [align=left] {\begin{minipage}[lt]{8.67pt}\setlength\topsep{0pt}
\begin{center}
{\footnotesize \textbf{\textcolor[rgb]{0.82,0.01,0.11}{?}}}
\end{center}

\end{minipage}};
% Text Node
\draw (152.6,2795.46) node  [font=\scriptsize] [align=left] {\begin{minipage}[lt]{8.67pt}\setlength\topsep{0pt}
\begin{center}
{\footnotesize \textbf{\textcolor[rgb]{0.82,0.01,0.11}{?}}}
\end{center}

\end{minipage}};
% Text Node
\draw (93.53,2791.97) node  [font=\scriptsize] [align=left] {\begin{minipage}[lt]{8.67pt}\setlength\topsep{0pt}
\begin{center}
{\footnotesize \textbf{\textcolor[rgb]{0.82,0.01,0.11}{?}}}
\end{center}

\end{minipage}};
% Text Node
\draw (182.5,2877.5) node  [font=\scriptsize] [align=left] {\begin{minipage}[lt]{8.67pt}\setlength\topsep{0pt}
\begin{center}
{\footnotesize $\displaystyle \mathbf{\textcolor[rgb]{0.82,0.01,0.11}{\pi }\textcolor[rgb]{0.82,0.01,0.11}{_{3}}}$}
\end{center}

\end{minipage}};
% Text Node
\draw (97.6,2901.34) node  [font=\scriptsize] [align=left] {\begin{minipage}[lt]{8.67pt}\setlength\topsep{0pt}
\begin{center}
{\footnotesize $\displaystyle \mathbf{\textcolor[rgb]{0.82,0.01,0.11}{\pi }\textcolor[rgb]{0.82,0.01,0.11}{_{1}}}$}
\end{center}

\end{minipage}};
% Text Node
\draw (358.3,2787.5) node  [font=\scriptsize] [align=left] {\begin{minipage}[lt]{8.67pt}\setlength\topsep{0pt}
\begin{center}
{\footnotesize $\displaystyle \textcolor[rgb]{0.82,0.01,0.11}{(}\mathbf{\textcolor[rgb]{0.82,0.01,0.11}{\pi }\textcolor[rgb]{0.82,0.01,0.11}{_{2}}}\textcolor[rgb]{0.82,0.01,0.11}{)}$}
\end{center}

\end{minipage}};
% Text Node
\draw (368.3,2770.5) node  [font=\scriptsize] [align=left] {\begin{minipage}[lt]{8.67pt}\setlength\topsep{0pt}
\begin{center}
{\footnotesize $\displaystyle \textcolor[rgb]{0.82,0.01,0.11}{(}\mathbf{\textcolor[rgb]{0.82,0.01,0.11}{\pi }\textcolor[rgb]{0.82,0.01,0.11}{_{3}}}\textcolor[rgb]{0.82,0.01,0.11}{)}$}
\end{center}

\end{minipage}};
% Text Node
\draw (299.81,2787.31) node  [font=\scriptsize] [align=left] {\begin{minipage}[lt]{8.67pt}\setlength\topsep{0pt}
\begin{center}
{\footnotesize $\displaystyle \textcolor[rgb]{0.82,0.01,0.11}{(}\mathbf{\textcolor[rgb]{0.82,0.01,0.11}{\pi }\textcolor[rgb]{0.82,0.01,0.11}{_{1}}}\textcolor[rgb]{0.82,0.01,0.11}{)}$}
\end{center}

\end{minipage}};
% Text Node
\draw (154.64,2902.5) node  [font=\scriptsize] [align=left] {\begin{minipage}[lt]{8.67pt}\setlength\topsep{0pt}
\begin{center}
{\footnotesize $\displaystyle \mathbf{\textcolor[rgb]{0.82,0.01,0.11}{\pi }\textcolor[rgb]{0.82,0.01,0.11}{_{2}}}$}
\end{center}

\end{minipage}};
% Text Node
\draw (336.27,2821.93) node  [font=\footnotesize] [align=left] {\begin{minipage}[lt]{83.6pt}\setlength\topsep{0pt}
\begin{center}
\textit{Target environment}
\end{center}

\end{minipage}};
% Text Node
\draw (336,2948) node   [align=left] {\begin{minipage}[lt]{62.53pt}\setlength\topsep{0pt}
\begin{center}
{\footnotesize "blueprints" of}\\{\footnotesize suggested MAS}
\end{center}

\end{minipage}};
% Text Node
\draw (139.36,2933.19) node   [align=left] {\begin{minipage}[lt]{96.97pt}\setlength\topsep{0pt}
\begin{center}
\textit{{\footnotesize Simulated environment + Trained agents}}
\end{center}

\end{minipage}};
% Text Node
\draw (138.5,2824.5) node  [font=\footnotesize] [align=left] {\begin{minipage}[lt]{93.02pt}\setlength\topsep{0pt}
\begin{center}
\textit{Simulated environment}
\end{center}

\end{minipage}};
% Text Node
\draw (171.5,2848.5) node  [font=\footnotesize] [align=left] {\textbf{2) Training}};
% Text Node
\draw (357.3,2848.5) node  [font=\footnotesize] [align=left] {\textbf{4) Transfer}};
% Text Node
\draw (348.3,2877.5) node  [font=\scriptsize] [align=left] {\begin{minipage}[lt]{8.67pt}\setlength\topsep{0pt}
\begin{center}
{\footnotesize $\displaystyle \mathbf{\textcolor[rgb]{0.82,0.01,0.11}{\pi }\textcolor[rgb]{0.82,0.01,0.11}{_{3}}}$}
\end{center}

\end{minipage}};
% Text Node
\draw (354.3,2900.5) node  [font=\scriptsize] [align=left] {\begin{minipage}[lt]{8.67pt}\setlength\topsep{0pt}
\begin{center}
{\footnotesize $\displaystyle \mathbf{\textcolor[rgb]{0.82,0.01,0.11}{\pi }\textcolor[rgb]{0.82,0.01,0.11}{_{2}}}$}
\end{center}

\end{minipage}};
% Text Node
\draw (305.3,2886.5) node  [font=\scriptsize] [align=left] {\begin{minipage}[lt]{8.67pt}\setlength\topsep{0pt}
\begin{center}
{\footnotesize $\displaystyle \mathbf{\textcolor[rgb]{0.82,0.01,0.11}{\pi }\textcolor[rgb]{0.82,0.01,0.11}{_{1}}}$}
\end{center}

\end{minipage}};
% Text Node
\draw (236.43,2889.27) node  [font=\footnotesize] [align=left] {\textbf{3) Analyze}};
% Text Node
\draw (233.5,2792.5) node  [font=\footnotesize] [align=left] {\textbf{1) Modeling}};


\end{tikzpicture}
  \caption{Lifecycle of a MAS designed with SAMMASD: i) Users start by modeling the environment, global goal and extra requirements; \quad ii) Then, they launch the training of agents with MARL techniques; \quad iii) A post-training analysis is performed to get "blueprint" of the trained MAS; \quad v) These "blueprint" can be used to semi-automatically deploy effective agents in the environment's effectors}
  \label{fig:cycle}
\end{figure}



\subsection{Phase 1: Modeling}

The modeling stage aims to create a simulated model that accurately captures the dynamics and constraints of the target environment, defines organizational specifications and goals. The real target environment must necessarily include effectors where agents capable of observing and acting in the environment will be deployed. This model will serve as a basis for training agents in a controlled environment. This step is critical to ensure that agents learn in a simulation that is faithful to the reality of the target system in a safe and solution-seeking setting.

The required inputs are:
\begin{enumerate*}[label={\roman*)}]
    \item \textbf{Environment}: An "emulated" copy of the real environment or the target environment itself if possible.
    \item \textbf{Problem description}: A detailed description of the goals to be achieved by the agents, i.e., the desired states.
    \item \textbf{Additional constraints}: Specific requirements to be met, which may include standards, organizational rules, or safety constraints.
\end{enumerate*}

Once the elements are collected, modeling follows these steps: 


\paragraph{\textbf{1) Modeling the simulated environment}} \quad

\noindent We define environment modeling as the development of an approximated observation function $\hat{O}: S \times A \to \Omega, \hat{O}(s_t,a_t) = \omega_{t+1}$ such that $|\hat{O} \cap O| \geq f$, where $f \in \mathbb{R}$ is fidelity to the real environment described by $O$.
The designer must faithfully reproduce the logic of the environment that leads the agent to receive observations at each state transition. 
For that purpose, several methods may be envisioned.

\textbf{System Identification} uses mathematical modeling, relying on parameter adjustments from observed data, which typically requires significant human input. This method works well when system dynamics are well-understood and equation-based. In contrast, \textbf{Imitation Learning} reproduces observed behaviors without a mathematical model, capturing complex patterns but often needing extensive examples to ensure robustness.

\textbf{Surrogate Modeling} and \textbf{Digital Twins} are more suitable for high-complexity or frequently updated systems. Surrogate Modeling uses statistical or machine learning to create simplified system representations, allowing efficient optimization with lower computational demands but potentially sacrificing precision. A Digital Twin, however, provides a synchronized digital replica that updates continuously, offering high fidelity. This approach, while precise, is more resource-intensive and usually requires an emulated system. Overall, the choice of method should balance precision, flexibility, and automation based on application needs.

In our method, we favor \textbf{Imitation Learning} techniques as many of these techniques do not require human intervention, are applicable to a majority of environments, and can capture environmental complexity after sufficient training. Although these techniques may lack readability, particularly those based on neural network architectures, we only consider fidelity as the main criterion of our method.

In this context, the suggested method proceeds as follows: observer agents (possibly human) are deployed in effectors' locations to collect traces (also called trajectories) by exploring the real or a secure copy of the environment. These traces are collected and used to train a model based on a neural network architecture. The Recurrent Neural Network architecture is particularly suitable for predicting the next observation, as it is optimized for learning from sequences.

\paragraph{\textbf{2) Reward Function Formulation}} \quad

\noindent We define reward function formulation as follows: \quad i) finding a clear description of the different desired states and their descriptions as (sub-)trajectories, in general, to best characterize the overall goal; \quad ii) finding a way to measure only the distance between the current state and these desired trajectories. Therefore, the user must establish a reward function logic that they believe is best suited to achieving the set goals.

Finding a clear description of the desired states is not always straightforward depending on the environment. Although assumed in our method, this search may be supported by Inverse reinforcement learning (IRL) techniques, which consist of learning from observed behaviors to deduce the ultimate goal. We recommend viewing an goal as independent of the agents. Therefore, the goal should not seek to directly influence agents to adopt expected behavior (this is more the role's view). Once established, these desired states can be presented as a set of (sub-)trajectories.

Regarding finding a way to measure the distance between the current state and the set of desired trajectories, we leave it to the user to define the best way to measure this distance. However, we suggest a general approach using similarity measures based on the longest common sequence between the (sub-)trajectory characterizing the goal and the current agents' history. This measure is simple and relevant when the sequence associated with a trajectory is continuous (as in our experiments). Nevertheless, it is less applicable when the sequence associated with the goal is discontinuous. Other measures from time series analysis are also conceivable.

\paragraph{\textbf{3) Formulation of $\mathcal{M}OISE^+$MARL Specifications}} \quad

\noindent We define the formulation of organizational specifications in

\noindent $\mathcal{M}OISE^+$MARL as follows: \quad i) defining roles ($\mathcal{R}$), goals ($\mathcal{G}$), and missions ($\mathcal{M}, mo$); ii) associating each agent with a respective role ($ar$); \quad iii) associating each role with constraint guides RAG and RGG defining their specific logic; iv) associating each sub-goal with their GCG constraint guide defining their logic ($gcg$).

Considering roles, goals, and missions as simple labels, their definition is assumed. The challenge lies in defining the corresponding constraint guides.

Defining a RAG, RRG, or GRG relation requires defining a potentially large number of histories, possibly partially redundant. Therefore, an extensive definition of a set of histories can be tedious. Further, the underlying idea of constraint guides is that when an agent follows a trajectory, it can be analyzed as part of a predefined set. For example, a RAG relation can be seen as determining the next action depending on whether the trajectory belongs to a given set and the new observation received. We suggest defining these relations comprehensively, allowing designers to define the logic to determine if a history belongs to a predefined set $H_g$ formalized as the relation $b_g: H \to \{0,1\}$.

We also propose a pattern of history inspired by Natural Language Processing, denoted $p \in P$, as a way to define a set of histories comprehensively. A history pattern $p \in P$ is defined as follows: $p$ is: either a "leaf sequence" denoted as a couple of history-cardinality $s_l = (h, \{c_min,c_max\})$ (where $h \in H, c_{min} \in \mathbb{N}, c_{max} \in \mathbb{N} \cup "*")$; or a "node sequence" denoted as a couple of a tuple of concrete sequences and cardinality $s_n = (\langle s_{l_1}, s_{l_1}\dots \rangle, \{c_min,c_max\})$. For example, the pattern $p = \allowbreak "[o_1,a_1,[o_2,a_2](0,2)](1,*)"$ can be formalized as the node sequence $\allowbreak \langle ((o_1,a_1),(1,1)), ((o_2,a_2),(0,2))\rangle(1,"*")$, indicating the history set $H_p$ containing at least once the sub-sequence consisting of a first pair $(o_1,a_1)$ and then at most two repetitions of the pair $(o_2,a_2)$.
The relation $b_g$ then becomes $b_g(h) = m(p_g,h), \text{ with } m: P \times H \to \{0,1\}$ indicating if a history $h \in H$ matches a history pattern $p \in P$ describing a history set $H_g$.

\

At the end of this phase, the output is the $\mathcal{M}OISE^+$MARL model comprising:
\begin{enumerate*}[label={\roman*)}]
    \item \textbf{A model of the environment} as a function approximating the observation function of the real environment;
    \item \textbf{A reward function} indicating goally how close or far the agents are from the global goal;
    \item \textbf{A defined $\mathcal{M}OISE^+$MARL tuple} including roles, goals, missions, and their associated "Constraint Guides."
\end{enumerate*}

\subsection{Phase 2: Solving}

We consider solving the previously formulated $\mathcal{M}OISE^+$MARL problem as finding a policy $\pi^s \in \Pi$: \quad i) achieving a cumulative reward at least above a given threshold $s \in \mathbb{R}$ such that $V^{\pi^s} \geq s$; \quad ii) for the collected rewards set $R = \{r_1, r_2\dots\}$ the variance of the reward must be below a given stability threshold $\sigma_{max}^2$ such that $\sigma_{max}^2 \leq \sigma_{max}^2$. The choice of $s$ and $\sigma_max$ is generally determined empirically depending on the environment, goals, and additional constraints. Although roles help partially define each agent's policies, MARL learning is guided by goals to optimally complete these policies.

Although no procedure exists to find at least one solution entirely automatically, the method recommends trying different MARL algorithms to benefit from their various properties suited to the environment, global goal, and given additional constraints.

\textbf{Value-based} algorithms such as Deep Q-Network (DQN) or Q-mix are generally suitable for \textbf{Independent Learning (IL)} scenarios, where each agent learns individually without explicit coordination. Although this approach is easy to implement, it may result in oscillations in estimated values, limiting overall stability.

\textbf{Policy-based} algorithms like MAPPO (Multi-Agent Proximal Policy Optimization) leverage \textbf{Centralized Learning Decentralized Execution (CLDE)} strategies. This approach allows agents to learn in a coordinated manner using global information while ensuring decentralized execution for better adaptability. They are well-suited for reducing variance, though convergence may be slower due to stability constraints.

\textbf{Actor-critic} algorithms such as MADDPG (Multi-Agent Deep Deterministic Policy Gradient) are flexible and compatible with \textbf{centralized or decentralized} approaches, depending on the required coordination. Using CLDE, MADDPG can benefit from centralization during learning while maintaining independence during execution, combining high performance with increased stability.

\textbf{Model-based} algorithms such as DynaQ or DynaQ+ can be implemented in a \textbf{centralized} manner or within \textbf{CLDE} learning, allowing planning based on a shared model. This improves performance and stability, but efficiency depends on the model's accuracy.

Based on our experiments, the MAPPO and MADDPG algorithms or other "actor-critic" algorithms provide satisfactory results for most environments, goals, and additional constraints with minimal intervention for hyper-parameter selection. When the real environment is simple enough and has low dynamics to be accurately captured by an environment model, solving with the DynaQ+ algorithm is also quite efficient and stable.

\

At the end of this phase, the output is a \textbf{joint policy satisfying the resolution requirements} in terms of performance and stability.

\subsection{Phase 3: Analysis}

We consider analyzing the previously obtained joint policy as determining MOISE+MARL organizational specifications from the observed behaviors of the agents with this policy. These organizational specifications are intended to be used as "blueprints" for later development. This phase is formalized as the relation $e: \mathcal{P}(H^{j}) \to MM$, which associates a set of joint histories with a set of MOISE+MARL specifications.

We found no existing methods that fully address our needs for abstract role, goal, or organizational fit determination. To fill this gap, we introduce the \textbf{History-based Evaluation in MOISE+MARL} (HEMM) method, which automatically infers and evaluates roles and missions by analyzing agent behaviors across multiple episodes.

\

\noindent HEMM applies unsupervised learning techniques to derive abstract roles and missions from observed behaviors, quantifying how close are agents from a structural and functional organization by comparing inferred specifications with actual behaviors. By analyzing joint histories and employing techniques like clustering, HEMM progressively infers abstract $\mathcal{M}OISE^+$ organizational specifications. A summary of the method follows~\hyperref[fn:github]{\footnotemark[1]}.
%
\footnotetext[1]{\label{fn:github} Additional details, code, datasets, and specifications are available at \url{https://github.com/REDACTED_FOR_BLIND_REVIEW}; videos and examples can be found at \url{https://drive.google.com/drive/folders/1ihso6E5qTio3ZBXf16fxLlBJUUmOx3uw?usp=sharing} \textbf{(anonymous link)}}

\paragraph{\textbf{1) Inferring Roles and Their Inheritance}}

In HEMM, a role $\rho$ is defined by a Common Longest Sequence (CLS) in the agents' action histories. We say role $\rho_2$ inherits from $\rho_1$ if the CLS for $\rho_2$ is part of the CLS for $\rho_1$. HEMM uses hierarchical clustering to identify CLS patterns across agent histories, presenting these as dendrograms to infer roles, inheritance, and structural fit between current and abstract role sequences.

\paragraph{\textbf{2) Inferring Goals, Plans, and Missions}}

HEMM defines goals as common joint-observations derived from successful agent histories. It calculates a joint-observation transition graph, which is clustered using K-means to reveal trajectory clusters. By sampling these trajectories, HEMM identifies abstract goals and plans based on sequence patterns. Missions are then formed by grouping the goals achieved by agents. By comparing inferred abstract goals with current agent observations, HEMM computes the "functional organizational fit."

\paragraph{\textbf{3) Inferring Obligations and Permissions}}

In HEMM, obligations are defined for agents fulfilling specific mission goals exclusively within certain time frames, while permissions allow for goal variability. HEMM assigns agents to missions as obligations or permissions, and calculates organizational fit as the sum of structural and functional fits.


\

The K-mean and hierarchical clustering techniques require manual configuration to obtain roles and goals, avoiding introducing perturbations that could lead to determining false organizational specifications. Despite this, the method recommends thoroughly understanding the obtained roles and goals to manually identify and remove any remaining perturbations. The refined $\mathcal{M}OISE^+$MARL specifications can then be used as "blueprints."

For each role in the "blueprint," we seek to represent the common policy of agents associated with that role as a set of decision trees. To do this, we adapt the corresponding $rag$ relation by determining a history pattern tree, where a node is a pattern-observation pair and the edges are the expected actions. These trees describe agents' policies comprehensively, making it easier to refine and adjust behavior rules.

\

At the end of this phase, the output is the set of refined $\mathcal{M}OISE^+$MARL specifications.

\subsection{Phase 4: Transfer}

The final phase aims to use the generated "blueprints" to develop and deploy a MAS on the real target environment. Semi-manual development is advised to ensure sufficient understanding to control and guarantee safety assurances.

The method suggests a procedure for partially automating the development and deployment of the MAS from the "blueprints" in a secure manner by following these steps:

\paragraph{\textbf{1) Transfer to Emulated Environment}}

Before real deployment, tests are conducted in an emulated environment to ensure that the MAS complies with safety constraints and performance requirements. To prepare for automatic deployment, we suggest that the effectors have planned deployment locations (bootstrap). In these deployment locations, we deploy daemon processes capable of receiving different policy types, including history pattern trees. Copying these policies into daemon processes can be automated or manual.

Furthermore, an automated copy process could be advantageous if all SAMMASD method steps can be pipelined, making agents' resulting policies adaptable to environmental changes, requirements, or goals. SAMMASD then becomes an "online" creation process.

\paragraph{\textbf{2) Evaluation in Emulated Environment}}

After deployment, this step aims to ensure that agents function as specified in the Analysis stage in the emulated environment and that agents can reach their goals while meeting additional requirements. If not, the method requires reviewing agent policies by changing history pattern decision trees in particular.

\paragraph{\textbf{3) Transfer to Real Target Environment}}

Once validated in the emulated environment, the verified policies are copied into the real environment's effectors. The method then recommends that designers ensure agents function correctly to achieve the goal and meet additional requirements, especially if the environment has changed since the Modeling stage, potentially rendering agents unsuitable. Otherwise, the method should be restarted from the Modeling or directly changing policies manually.

\section{Evaluation setup on a working example}
\label{sec:evaluation}

We developed a tool that we propose to facilitate the implementation of the SAMMASD method through a warehouse flow management scenario. Then, we present and discuss the results for this scenario.

\subsection{A development environment for the method}

To support the SAMMASD method, we have developed a tool named \textbf{Cyber Multi-agent System Development Environment} (CybMASDE), which provides a comprehensive environment for modeling, training, and deploying multi-agent systems. CybMASDE integrates several components, including the PettingZoo~\cite{Terry2021} which is a library that offers a standard API simplifying the development of multi-agent environments and facilitates the use of MARL algorithms. CybMASDE uses the MARLlib~\cite{hu2022marllib} library which offers a wide range of state-of-the-art MARL algorithms and fine-tuned policy models for various environments. It also enables Hyper-Parameter Optimization (HPO) of MARL algorithms to adapt to new environments. CybMASDE also uses the Tensorflow library to model the real environment into a simulated model. CybMASDE includes both a full-featured API for advanced usage and a basic graphical interface for quick access to essential functions.


\subsection{Proposed environment}
To evaluate our method via CybMASDE, we propose an environment called Warehouse Flow Management (WFM), a grid environment that represents robots that must cooperate in a manufacturing warehouse. We choose to consider a simulated WFM environment as if it were the real environment. This environment is represented in \autoref{fig:warehouse}.
Using a simulation simplifies and verifies the operating principle by reducing environmental complexity.

\begin{figure}[h!]
  \centering
  


\tikzset{every picture/.style={line width=0.75pt}} %set default line width to 0.75pt        

\begin{tikzpicture}[x=0.75pt,y=0.75pt,yscale=-1,xscale=1]
%uncomment if require: \path (0,2955); %set diagram left start at 0, and has height of 2955

%Image [id:dp5875918891930108] 
\draw (75.57,2734.2) node  {\includegraphics[width=5.35pt,height=7.88pt]{figures/robot.png}};
%Shape: Rectangle [id:dp6606511370911707] 
\draw  [fill={rgb, 255:red, 155; green, 155; blue, 155 }  ,fill opacity=1 ] (337.44,2484) -- (386,2484) -- (386,2582) -- (337.44,2582) -- cycle ;
%Shape: Rectangle [id:dp8063675095945988] 
\draw  [fill={rgb, 255:red, 155; green, 155; blue, 155 }  ,fill opacity=1 ] (337.44,2582) -- (386,2582) -- (386,2680) -- (337.44,2680) -- cycle ;
%Shape: Rectangle [id:dp22238139791392153] 
\draw  [fill={rgb, 255:red, 155; green, 155; blue, 155 }  ,fill opacity=1 ] (182.06,2598.33) -- (275.94,2598.33) -- (275.94,2650.6) -- (182.06,2650.6) -- cycle ;
%Shape: Rectangle [id:dp2365241991633944] 
\draw  [fill={rgb, 255:red, 155; green, 155; blue, 155 }  ,fill opacity=1 ] (182.06,2513.4) -- (275.94,2513.4) -- (275.94,2565.67) -- (182.06,2565.67) -- cycle ;
%Rounded Rect [id:dp26651039736158255] 
\draw  [fill={rgb, 255:red, 65; green, 117; blue, 5 }  ,fill opacity=1 ] (363.34,2621.82) .. controls (363.34,2619.67) and (365.08,2617.93) .. (367.22,2617.93) -- (378.88,2617.93) .. controls (381.02,2617.93) and (382.76,2619.67) .. (382.76,2621.82) -- (382.76,2640.18) .. controls (382.76,2642.33) and (381.02,2644.07) .. (378.88,2644.07) -- (367.22,2644.07) .. controls (365.08,2644.07) and (363.34,2642.33) .. (363.34,2640.18) -- cycle ;
%Rounded Rect [id:dp4236041404213995] 
\draw  [fill={rgb, 255:red, 65; green, 117; blue, 5 }  ,fill opacity=1 ] (363.34,2523.82) .. controls (363.34,2521.67) and (365.08,2519.93) .. (367.22,2519.93) -- (378.88,2519.93) .. controls (381.02,2519.93) and (382.76,2521.67) .. (382.76,2523.82) -- (382.76,2542.18) .. controls (382.76,2544.33) and (381.02,2546.07) .. (378.88,2546.07) -- (367.22,2546.07) .. controls (365.08,2546.07) and (363.34,2544.33) .. (363.34,2542.18) -- cycle ;
%Shape: Rectangle [id:dp7508106122229439] 
\draw  [color={rgb, 255:red, 155; green, 155; blue, 155 }  ,draw opacity=0 ][fill={rgb, 255:red, 139; green, 87; blue, 42 }  ,fill opacity=1 ] (217.67,2627.73) -- (224.14,2627.73) -- (224.14,2634.27) -- (217.67,2634.27) -- cycle ;
%Shape: Rectangle [id:dp8224971884832759] 
\draw  [color={rgb, 255:red, 155; green, 155; blue, 155 }  ,draw opacity=0 ][fill={rgb, 255:red, 139; green, 87; blue, 42 }  ,fill opacity=1 ] (217.67,2614.67) -- (224.14,2614.67) -- (224.14,2621.2) -- (217.67,2621.2) -- cycle ;
%Shape: Rectangle [id:dp30176012216927695] 
\draw  [color={rgb, 255:red, 155; green, 155; blue, 155 }  ,draw opacity=0 ][fill={rgb, 255:red, 139; green, 87; blue, 42 }  ,fill opacity=1 ] (217.67,2542.8) -- (224.14,2542.8) -- (224.14,2549.33) -- (217.67,2549.33) -- cycle ;
%Shape: Rectangle [id:dp0450787042591585] 
\draw  [color={rgb, 255:red, 155; green, 155; blue, 155 }  ,draw opacity=0 ][fill={rgb, 255:red, 139; green, 87; blue, 42 }  ,fill opacity=1 ] (217.67,2529.73) -- (224.14,2529.73) -- (224.14,2536.27) -- (217.67,2536.27) -- cycle ;
%Shape: Rectangle [id:dp4110703691579449] 
\draw  [color={rgb, 255:red, 0; green, 0; blue, 0 }  ,draw opacity=0 ][fill={rgb, 255:red, 139; green, 87; blue, 42 }  ,fill opacity=1 ] (217.67,2622.93) .. controls (217.67,2620.17) and (219.91,2617.93) .. (222.67,2617.93) -- (245.04,2617.93) .. controls (247.8,2617.93) and (250.04,2620.17) .. (250.04,2622.93) -- (250.04,2626) .. controls (250.04,2628.76) and (247.8,2631) .. (245.04,2631) -- (222.67,2631) .. controls (219.91,2631) and (217.67,2628.76) .. (217.67,2626) -- cycle ;
%Shape: Rectangle [id:dp34399071091019273] 
\draw  [color={rgb, 255:red, 0; green, 0; blue, 0 }  ,draw opacity=0 ][fill={rgb, 255:red, 139; green, 87; blue, 42 }  ,fill opacity=1 ] (217.67,2538) .. controls (217.67,2535.24) and (219.91,2533) .. (222.67,2533) -- (245.04,2533) .. controls (247.8,2533) and (250.04,2535.24) .. (250.04,2538) -- (250.04,2541.07) .. controls (250.04,2543.83) and (247.8,2546.07) .. (245.04,2546.07) -- (222.67,2546.07) .. controls (219.91,2546.07) and (217.67,2543.83) .. (217.67,2541.07) -- cycle ;
%Shape: Rectangle [id:dp6335118501918053] 
\draw  [color={rgb, 255:red, 0; green, 0; blue, 0 }  ,draw opacity=0 ][fill={rgb, 255:red, 139; green, 87; blue, 42 }  ,fill opacity=1 ] (207.96,2524.93) .. controls (207.96,2522.17) and (210.2,2519.93) .. (212.96,2519.93) -- (215.91,2519.93) .. controls (218.67,2519.93) and (220.91,2522.17) .. (220.91,2524.93) -- (220.91,2554.13) .. controls (220.91,2556.89) and (218.67,2559.13) .. (215.91,2559.13) -- (212.96,2559.13) .. controls (210.2,2559.13) and (207.96,2556.89) .. (207.96,2554.13) -- cycle ;
%Shape: Rectangle [id:dp137915950930817] 
\draw  [color={rgb, 255:red, 0; green, 0; blue, 0 }  ,draw opacity=0 ][fill={rgb, 255:red, 139; green, 87; blue, 42 }  ,fill opacity=1 ] (207.96,2609.87) .. controls (207.96,2607.11) and (210.2,2604.87) .. (212.96,2604.87) -- (215.91,2604.87) .. controls (218.67,2604.87) and (220.91,2607.11) .. (220.91,2609.87) -- (220.91,2639.07) .. controls (220.91,2641.83) and (218.67,2644.07) .. (215.91,2644.07) -- (212.96,2644.07) .. controls (210.2,2644.07) and (207.96,2641.83) .. (207.96,2639.07) -- cycle ;
%Shape: Rectangle [id:dp35124002049495284] 
\draw  [line width=1.5]  (72,2484) -- (386,2484) -- (386,2680) -- (72,2680) -- cycle ;
%Image [id:dp8729739961557643] 
\draw (147.1,2582) node  {\includegraphics[width=9.71pt,height=14.7pt]{figures/robot.png}};
%Shape: Rectangle [id:dp6745634275168286] 
\draw  [fill={rgb, 255:red, 155; green, 155; blue, 155 }  ,fill opacity=1 ] (72,2549.33) -- (120.56,2549.33) -- (120.56,2614.67) -- (72,2614.67) -- cycle ;
%Shape: Rectangle [id:dp8514781790750696] 
\draw  [fill={rgb, 255:red, 155; green, 155; blue, 155 }  ,fill opacity=1 ] (72,2614.67) -- (120.56,2614.67) -- (120.56,2680) -- (72,2680) -- cycle ;
%Shape: Rectangle [id:dp43369330200395684] 
\draw  [fill={rgb, 255:red, 155; green, 155; blue, 155 }  ,fill opacity=1 ] (72,2484) -- (120.56,2484) -- (120.56,2549.33) -- (72,2549.33) -- cycle ;
%Shape: Triangle [id:dp4196897584705377] 
\draw  [fill={rgb, 255:red, 0; green, 0; blue, 0 }  ,fill opacity=1 ] (214.43,2546.07) -- (217.67,2552.6) -- (211.2,2552.6) -- cycle ;
%Shape: Rectangle [id:dp5440544556793916] 
\draw  [fill={rgb, 255:red, 0; green, 0; blue, 0 }  ,fill opacity=1 ] (211.2,2526.47) -- (217.67,2526.47) -- (217.67,2533) -- (211.2,2533) -- cycle ;
%Shape: Cross [id:dp8291985040737369] 
\draw  [fill={rgb, 255:red, 0; green, 0; blue, 0 }  ,fill opacity=1 ] (214.39,2536.27) -- (214.96,2536.27) -- (214.96,2538.97) -- (217.67,2538.97) -- (217.67,2539.63) -- (214.96,2539.63) -- (214.96,2542.34) -- (214.39,2542.34) -- (214.39,2539.63) -- (211.68,2539.63) -- (211.68,2538.97) -- (214.39,2538.97) -- cycle ;
%Right Arrow [id:dp8648952852974343] 
\draw  [fill={rgb, 255:red, 0; green, 0; blue, 0 }  ,fill opacity=1 ] (224.92,2538.71) -- (232.29,2538.71) -- (232.29,2537.88) -- (233.86,2539.53) -- (232.29,2541.19) -- (232.29,2540.36) -- (224.92,2540.36) -- cycle ;
%Shape: Triangle [id:dp5703393943519683] 
\draw  [fill={rgb, 255:red, 0; green, 0; blue, 0 }  ,fill opacity=1 ] (243.57,2539.68) -- (246.8,2542.8) -- (240.33,2542.8) -- cycle ;
%Shape: Rectangle [id:dp4376867112662228] 
\draw  [fill={rgb, 255:red, 0; green, 0; blue, 0 }  ,fill opacity=1 ] (240.33,2536.27) -- (246.8,2536.27) -- (246.8,2539.68) -- (240.33,2539.68) -- cycle ;
%Shape: Triangle [id:dp8170881689353535] 
\draw  [fill={rgb, 255:red, 0; green, 0; blue, 0 }  ,fill opacity=1 ] (243.57,2624.9) -- (246.8,2627.73) -- (240.33,2627.73) -- cycle ;
%Shape: Ellipse [id:dp6470872454641863] 
\draw  [fill={rgb, 255:red, 0; green, 0; blue, 0 }  ,fill opacity=1 ] (240.33,2623.76) .. controls (240.33,2622.35) and (241.78,2621.2) .. (243.57,2621.2) .. controls (245.35,2621.2) and (246.8,2622.35) .. (246.8,2623.76) .. controls (246.8,2625.17) and (245.35,2626.32) .. (243.57,2626.32) .. controls (241.78,2626.32) and (240.33,2625.17) .. (240.33,2623.76) -- cycle ;

%Shape: Arc [id:dp351171372517763] 
\draw  [draw opacity=0] (211.2,2559.13) .. controls (211.2,2559.13) and (211.2,2559.13) .. (211.2,2559.13) .. controls (211.2,2559.13) and (211.2,2559.13) .. (211.2,2559.13) .. controls (209.41,2559.13) and (207.96,2557.67) .. (207.96,2555.87) -- (211.2,2555.87) -- cycle ; \draw   (211.2,2559.13) .. controls (211.2,2559.13) and (211.2,2559.13) .. (211.2,2559.13) .. controls (211.2,2559.13) and (211.2,2559.13) .. (211.2,2559.13) .. controls (209.41,2559.13) and (207.96,2557.67) .. (207.96,2555.87) ;  
%Shape: Arc [id:dp9057589079954145] 
\draw  [draw opacity=0] (207.96,2523.2) .. controls (207.96,2523.2) and (207.96,2523.2) .. (207.96,2523.2) .. controls (207.96,2521.4) and (209.41,2519.93) .. (211.2,2519.93) -- (211.2,2523.2) -- cycle ; \draw   (207.96,2523.2) .. controls (207.96,2523.2) and (207.96,2523.2) .. (207.96,2523.2) .. controls (207.96,2521.4) and (209.41,2519.93) .. (211.2,2519.93) ;  
%Shape: Arc [id:dp8699428292698086] 
\draw  [draw opacity=0] (246.8,2533) .. controls (246.8,2533) and (246.8,2533) .. (246.8,2533) .. controls (248.59,2533) and (250.04,2534.46) .. (250.04,2536.27) -- (246.8,2536.27) -- cycle ; \draw   (246.8,2533) .. controls (246.8,2533) and (246.8,2533) .. (246.8,2533) .. controls (248.59,2533) and (250.04,2534.46) .. (250.04,2536.27) ;  
%Shape: Arc [id:dp01986661390154798] 
\draw  [draw opacity=0] (217.67,2519.93) .. controls (217.67,2519.93) and (217.67,2519.93) .. (217.67,2519.93) .. controls (219.46,2519.93) and (220.91,2521.4) .. (220.91,2523.2) -- (217.67,2523.2) -- cycle ; \draw   (217.67,2519.93) .. controls (217.67,2519.93) and (217.67,2519.93) .. (217.67,2519.93) .. controls (219.46,2519.93) and (220.91,2521.4) .. (220.91,2523.2) ;  
%Shape: Arc [id:dp8747370263149703] 
\draw  [draw opacity=0] (250.04,2542.8) .. controls (250.04,2542.8) and (250.04,2542.8) .. (250.04,2542.8) .. controls (250.04,2544.6) and (248.59,2546.07) .. (246.8,2546.07) -- (246.8,2542.8) -- cycle ; \draw   (250.04,2542.8) .. controls (250.04,2542.8) and (250.04,2542.8) .. (250.04,2542.8) .. controls (250.04,2544.6) and (248.59,2546.07) .. (246.8,2546.07) ;  
%Shape: Arc [id:dp5279598672990709] 
\draw  [draw opacity=0] (220.91,2555.87) .. controls (220.91,2555.87) and (220.91,2555.87) .. (220.91,2555.87) .. controls (220.91,2557.67) and (219.46,2559.13) .. (217.67,2559.13) -- (217.67,2555.87) -- cycle ; \draw   (220.91,2555.87) .. controls (220.91,2555.87) and (220.91,2555.87) .. (220.91,2555.87) .. controls (220.91,2557.67) and (219.46,2559.13) .. (217.67,2559.13) ;  
%Straight Lines [id:da42875733460370635] 
\draw    (207.96,2523.2) -- (207.96,2555.87) ;
%Straight Lines [id:da9559383533551191] 
\draw    (220.91,2549.33) -- (220.91,2555.87) ;
%Straight Lines [id:da7482320251711243] 
\draw    (220.91,2523.2) -- (220.91,2529.73) ;
%Straight Lines [id:da5727563917540581] 
\draw    (250.04,2536.27) -- (250.04,2542.8) ;
%Straight Lines [id:da9142838451709618] 
\draw    (224.14,2546.07) -- (246.8,2546.07) ;
%Straight Lines [id:da024772822378669845] 
\draw    (224.14,2533) -- (246.8,2533) ;
%Straight Lines [id:da5330901612213492] 
\draw    (211.2,2559.13) -- (217.67,2559.13) ;
%Straight Lines [id:da12218029117576723] 
\draw    (211.2,2519.93) -- (217.67,2519.93) ;
%Shape: Triangle [id:dp43922196669018176] 
\draw  [fill={rgb, 255:red, 0; green, 0; blue, 0 }  ,fill opacity=1 ] (214.43,2631) -- (217.67,2637.53) -- (211.2,2637.53) -- cycle ;
%Shape: Cross [id:dp9945533953957002] 
\draw  [fill={rgb, 255:red, 0; green, 0; blue, 0 }  ,fill opacity=1 ] (214.39,2621.2) -- (214.96,2621.2) -- (214.96,2623.91) -- (217.67,2623.91) -- (217.67,2624.57) -- (214.96,2624.57) -- (214.96,2627.27) -- (214.39,2627.27) -- (214.39,2624.57) -- (211.68,2624.57) -- (211.68,2623.91) -- (214.39,2623.91) -- cycle ;
%Right Arrow [id:dp501811153541224] 
\draw  [fill={rgb, 255:red, 0; green, 0; blue, 0 }  ,fill opacity=1 ] (224.92,2623.64) -- (232.29,2623.64) -- (232.29,2622.81) -- (233.86,2624.47) -- (232.29,2626.12) -- (232.29,2625.29) -- (224.92,2625.29) -- cycle ;
%Shape: Rectangle [id:dp5218198278316464] 
\draw  [fill={rgb, 255:red, 255; green, 255; blue, 255 }  ,fill opacity=1 ] (260.4,2620.93) .. controls (260.4,2619.28) and (261.74,2617.93) .. (263.4,2617.93) -- (270.35,2617.93) .. controls (272.01,2617.93) and (273.35,2619.28) .. (273.35,2620.93) -- (273.35,2628) .. controls (273.35,2629.66) and (272.01,2631) .. (270.35,2631) -- (263.4,2631) .. controls (261.74,2631) and (260.4,2629.66) .. (260.4,2628) -- cycle ;
%Shape: Arc [id:dp32882122034592665] 
\draw  [draw opacity=0] (211.2,2644.07) .. controls (211.2,2644.07) and (211.2,2644.07) .. (211.2,2644.07) .. controls (211.2,2644.07) and (211.2,2644.07) .. (211.2,2644.07) .. controls (209.41,2644.07) and (207.96,2642.6) .. (207.96,2640.8) -- (211.2,2640.8) -- cycle ; \draw   (211.2,2644.07) .. controls (211.2,2644.07) and (211.2,2644.07) .. (211.2,2644.07) .. controls (211.2,2644.07) and (211.2,2644.07) .. (211.2,2644.07) .. controls (209.41,2644.07) and (207.96,2642.6) .. (207.96,2640.8) ;  
%Shape: Arc [id:dp5981817583565532] 
\draw  [draw opacity=0] (207.96,2608.13) .. controls (207.96,2608.13) and (207.96,2608.13) .. (207.96,2608.13) .. controls (207.96,2606.33) and (209.41,2604.87) .. (211.2,2604.87) -- (211.2,2608.13) -- cycle ; \draw   (207.96,2608.13) .. controls (207.96,2608.13) and (207.96,2608.13) .. (207.96,2608.13) .. controls (207.96,2606.33) and (209.41,2604.87) .. (211.2,2604.87) ;  
%Shape: Arc [id:dp6259682686955037] 
\draw  [draw opacity=0] (246.8,2617.93) .. controls (246.8,2617.93) and (246.8,2617.93) .. (246.8,2617.93) .. controls (248.59,2617.93) and (250.04,2619.4) .. (250.04,2621.2) -- (246.8,2621.2) -- cycle ; \draw   (246.8,2617.93) .. controls (246.8,2617.93) and (246.8,2617.93) .. (246.8,2617.93) .. controls (248.59,2617.93) and (250.04,2619.4) .. (250.04,2621.2) ;  
%Shape: Arc [id:dp10371853577149648] 
\draw  [draw opacity=0] (217.67,2604.87) .. controls (217.67,2604.87) and (217.67,2604.87) .. (217.67,2604.87) .. controls (219.46,2604.87) and (220.91,2606.33) .. (220.91,2608.13) -- (217.67,2608.13) -- cycle ; \draw   (217.67,2604.87) .. controls (217.67,2604.87) and (217.67,2604.87) .. (217.67,2604.87) .. controls (219.46,2604.87) and (220.91,2606.33) .. (220.91,2608.13) ;  
%Shape: Arc [id:dp7047945330251639] 
\draw  [draw opacity=0] (250.04,2627.73) .. controls (250.04,2627.73) and (250.04,2627.73) .. (250.04,2627.73) .. controls (250.04,2629.54) and (248.59,2631) .. (246.8,2631) -- (246.8,2627.73) -- cycle ; \draw   (250.04,2627.73) .. controls (250.04,2627.73) and (250.04,2627.73) .. (250.04,2627.73) .. controls (250.04,2629.54) and (248.59,2631) .. (246.8,2631) ;  
%Shape: Arc [id:dp7559284490507128] 
\draw  [draw opacity=0] (220.91,2640.8) .. controls (220.91,2640.8) and (220.91,2640.8) .. (220.91,2640.8) .. controls (220.91,2642.6) and (219.46,2644.07) .. (217.67,2644.07) -- (217.67,2640.8) -- cycle ; \draw   (220.91,2640.8) .. controls (220.91,2640.8) and (220.91,2640.8) .. (220.91,2640.8) .. controls (220.91,2642.6) and (219.46,2644.07) .. (217.67,2644.07) ;  
%Straight Lines [id:da9035186258354964] 
\draw    (207.96,2608.13) -- (207.96,2640.8) ;
%Straight Lines [id:da21468365948732426] 
\draw    (220.91,2634.27) -- (220.91,2640.8) ;
%Straight Lines [id:da2935839419101631] 
\draw    (220.91,2608.13) -- (220.91,2614.67) ;
%Straight Lines [id:da5198889937211966] 
\draw    (250.04,2621.2) -- (250.04,2627.73) ;
%Straight Lines [id:da6365355765950893] 
\draw    (224.14,2631) -- (246.8,2631) ;
%Straight Lines [id:da1839455350742425] 
\draw    (224.14,2617.93) -- (246.8,2617.93) ;
%Straight Lines [id:da43918025379144954] 
\draw    (211.2,2644.07) -- (217.67,2644.07) ;
%Straight Lines [id:da9838400653785506] 
\draw    (211.2,2604.87) -- (217.67,2604.87) ;
%Image [id:dp9228518493231206] 
\draw (169.11,2526.47) node  {\includegraphics[width=9.71pt,height=14.7pt]{figures/robot.png}};
%Image [id:dp6344757927326945] 
\draw (173,2614.67) node  {\includegraphics[width=9.71pt,height=14.7pt]{figures/robot.png}};
%Image [id:dp0824808306939826] 
\draw (312.19,2604.87) node  {\includegraphics[width=9.71pt,height=14.7pt]{figures/robot.png}};
%Shape: Triangle [id:dp42235480036354867] 
\draw  [fill={rgb, 255:red, 0; green, 0; blue, 0 }  ,fill opacity=1 ] (373.05,2533.14) -- (376.29,2536.27) -- (369.81,2536.27) -- cycle ;
%Shape: Rectangle [id:dp025603949819479332] 
\draw  [fill={rgb, 255:red, 0; green, 0; blue, 0 }  ,fill opacity=1 ] (369.81,2529.73) -- (376.29,2529.73) -- (376.29,2533.14) -- (369.81,2533.14) -- cycle ;
%Straight Lines [id:da7326448937458538] 
\draw    (366.58,2542.8) -- (376.53,2542.8) ;
\draw [shift={(379.53,2542.8)}, rotate = 180] [fill={rgb, 255:red, 0; green, 0; blue, 0 }  ][line width=0.08]  [draw opacity=0] (3.57,-1.72) -- (0,0) -- (3.57,1.72) -- cycle    ;
%Straight Lines [id:da7119286112789345] 
\draw    (366.58,2523.2) -- (376.53,2523.2) ;
\draw [shift={(379.53,2523.2)}, rotate = 180] [fill={rgb, 255:red, 0; green, 0; blue, 0 }  ][line width=0.08]  [draw opacity=0] (3.57,-1.72) -- (0,0) -- (3.57,1.72) -- cycle    ;
%Shape: Rectangle [id:dp07538987626809934] 
\draw  [fill={rgb, 255:red, 255; green, 255; blue, 255 }  ,fill opacity=1 ] (341.33,2529.47) .. controls (341.33,2527.81) and (342.67,2526.47) .. (344.33,2526.47) -- (351.28,2526.47) .. controls (352.93,2526.47) and (354.28,2527.81) .. (354.28,2529.47) -- (354.28,2536.53) .. controls (354.28,2538.19) and (352.93,2539.53) .. (351.28,2539.53) -- (344.33,2539.53) .. controls (342.67,2539.53) and (341.33,2538.19) .. (341.33,2536.53) -- cycle ;
%Shape: Rectangle [id:dp5540842007977813] 
\draw  [fill={rgb, 255:red, 255; green, 255; blue, 255 }  ,fill opacity=1 ] (341.33,2549.07) .. controls (341.33,2547.41) and (342.67,2546.07) .. (344.33,2546.07) -- (351.28,2546.07) .. controls (352.93,2546.07) and (354.28,2547.41) .. (354.28,2549.07) -- (354.28,2556.13) .. controls (354.28,2557.79) and (352.93,2559.13) .. (351.28,2559.13) -- (344.33,2559.13) .. controls (342.67,2559.13) and (341.33,2557.79) .. (341.33,2556.13) -- cycle ;
%Straight Lines [id:da0636335084540185] 
\draw    (366.58,2640.8) -- (376.53,2640.8) ;
\draw [shift={(379.53,2640.8)}, rotate = 180] [fill={rgb, 255:red, 0; green, 0; blue, 0 }  ][line width=0.08]  [draw opacity=0] (3.57,-1.72) -- (0,0) -- (3.57,1.72) -- cycle    ;
%Straight Lines [id:da6068797143198703] 
\draw    (366.58,2621.2) -- (376.53,2621.2) ;
\draw [shift={(379.53,2621.2)}, rotate = 180] [fill={rgb, 255:red, 0; green, 0; blue, 0 }  ][line width=0.08]  [draw opacity=0] (3.57,-1.72) -- (0,0) -- (3.57,1.72) -- cycle    ;
%Shape: Rectangle [id:dp1521480893231475] 
\draw  [fill={rgb, 255:red, 255; green, 255; blue, 255 }  ,fill opacity=1 ] (341.33,2607.87) .. controls (341.33,2606.21) and (342.67,2604.87) .. (344.33,2604.87) -- (351.28,2604.87) .. controls (352.93,2604.87) and (354.28,2606.21) .. (354.28,2607.87) -- (354.28,2614.93) .. controls (354.28,2616.59) and (352.93,2617.93) .. (351.28,2617.93) -- (344.33,2617.93) .. controls (342.67,2617.93) and (341.33,2616.59) .. (341.33,2614.93) -- cycle ;
%Shape: Rectangle [id:dp5801294041672793] 
\draw  [fill={rgb, 255:red, 255; green, 255; blue, 255 }  ,fill opacity=1 ] (341.33,2627.47) .. controls (341.33,2625.81) and (342.67,2624.47) .. (344.33,2624.47) -- (351.28,2624.47) .. controls (352.93,2624.47) and (354.28,2625.81) .. (354.28,2627.47) -- (354.28,2634.53) .. controls (354.28,2636.19) and (352.93,2637.53) .. (351.28,2637.53) -- (344.33,2637.53) .. controls (342.67,2637.53) and (341.33,2636.19) .. (341.33,2634.53) -- cycle ;
%Shape: Rectangle [id:dp9678654995334497] 
\draw  [fill={rgb, 255:red, 255; green, 255; blue, 255 }  ,fill opacity=1 ] (341.33,2647.07) .. controls (341.33,2645.41) and (342.67,2644.07) .. (344.33,2644.07) -- (351.28,2644.07) .. controls (352.93,2644.07) and (354.28,2645.41) .. (354.28,2647.07) -- (354.28,2654.13) .. controls (354.28,2655.79) and (352.93,2657.13) .. (351.28,2657.13) -- (344.33,2657.13) .. controls (342.67,2657.13) and (341.33,2655.79) .. (341.33,2654.13) -- cycle ;
%Shape: Triangle [id:dp12780211023556087] 
\draw  [fill={rgb, 255:red, 0; green, 0; blue, 0 }  ,fill opacity=1 ] (373.05,2631.43) -- (376.29,2634.27) -- (369.81,2634.27) -- cycle ;
%Shape: Ellipse [id:dp15008593545960025] 
\draw  [fill={rgb, 255:red, 0; green, 0; blue, 0 }  ,fill opacity=1 ] (369.81,2630.29) .. controls (369.81,2628.88) and (371.26,2627.73) .. (373.05,2627.73) .. controls (374.84,2627.73) and (376.29,2628.88) .. (376.29,2630.29) .. controls (376.29,2631.71) and (374.84,2632.85) .. (373.05,2632.85) .. controls (371.26,2632.85) and (369.81,2631.71) .. (369.81,2630.29) -- cycle ;

%Shape: Arc [id:dp6836346965493294] 
\draw  [draw opacity=0][fill={rgb, 255:red, 155; green, 155; blue, 155 }  ,fill opacity=1 ] (224.14,2533) .. controls (222.36,2533) and (220.91,2531.54) .. (220.91,2529.73) -- (224.14,2529.73) -- cycle ; \draw   (224.14,2533) .. controls (222.36,2533) and (220.91,2531.54) .. (220.91,2529.73) ;  
%Shape: Arc [id:dp6560740676828933] 
\draw  [draw opacity=0][fill={rgb, 255:red, 155; green, 155; blue, 155 }  ,fill opacity=1 ] (220.91,2549.33) .. controls (220.91,2549.33) and (220.91,2549.33) .. (220.91,2549.33) .. controls (220.91,2547.53) and (222.36,2546.07) .. (224.14,2546.07) -- (224.14,2549.33) -- cycle ; \draw   (220.91,2549.33) .. controls (220.91,2549.33) and (220.91,2549.33) .. (220.91,2549.33) .. controls (220.91,2547.53) and (222.36,2546.07) .. (224.14,2546.07) ;  
%Shape: Ellipse [id:dp8897513054260964] 
\draw  [fill={rgb, 255:red, 0; green, 0; blue, 0 }  ,fill opacity=1 ] (211.2,2614.67) .. controls (211.2,2612.86) and (212.65,2611.4) .. (214.43,2611.4) .. controls (216.22,2611.4) and (217.67,2612.86) .. (217.67,2614.67) .. controls (217.67,2616.47) and (216.22,2617.93) .. (214.43,2617.93) .. controls (212.65,2617.93) and (211.2,2616.47) .. (211.2,2614.67) -- cycle ;
%Shape: Arc [id:dp8741248380179243] 
\draw  [draw opacity=0][fill={rgb, 255:red, 155; green, 155; blue, 155 }  ,fill opacity=1 ] (220.91,2634.27) .. controls (220.91,2634.27) and (220.91,2634.27) .. (220.91,2634.27) .. controls (220.91,2632.46) and (222.36,2631) .. (224.14,2631) -- (224.14,2634.27) -- cycle ; \draw   (220.91,2634.27) .. controls (220.91,2634.27) and (220.91,2634.27) .. (220.91,2634.27) .. controls (220.91,2632.46) and (222.36,2631) .. (224.14,2631) ;  
%Shape: Arc [id:dp08355774422679829] 
\draw  [draw opacity=0][fill={rgb, 255:red, 155; green, 155; blue, 155 }  ,fill opacity=1 ] (224.14,2617.93) .. controls (222.36,2617.93) and (220.91,2616.47) .. (220.91,2614.67) -- (224.14,2614.67) -- cycle ; \draw   (224.14,2617.93) .. controls (222.36,2617.93) and (220.91,2616.47) .. (220.91,2614.67) ;  
%Shape: Ellipse [id:dp8088402735209159] 
\draw  [fill={rgb, 255:red, 0; green, 0; blue, 0 }  ,fill opacity=1 ] (288.46,2710.7) .. controls (288.46,2709.73) and (289.26,2708.95) .. (290.25,2708.95) .. controls (291.23,2708.95) and (292.03,2709.73) .. (292.03,2710.7) .. controls (292.03,2711.67) and (291.23,2712.45) .. (290.25,2712.45) .. controls (289.26,2712.45) and (288.46,2711.67) .. (288.46,2710.7) -- cycle ;
%Shape: Triangle [id:dp37809883705319325] 
\draw  [fill={rgb, 255:red, 0; green, 0; blue, 0 }  ,fill opacity=1 ] (293.81,2715.95) -- (295.6,2719.45) -- (292.03,2719.45) -- cycle ;
%Shape: Rectangle [id:dp9359145463740597] 
\draw  [fill={rgb, 255:red, 0; green, 0; blue, 0 }  ,fill opacity=1 ] (295.6,2708.95) -- (299.17,2708.95) -- (299.17,2712.45) -- (295.6,2712.45) -- cycle ;

%Shape: Triangle [id:dp6504561453454727] 
\draw  [fill={rgb, 255:red, 0; green, 0; blue, 0 }  ,fill opacity=1 ] (183.1,2738.93) -- (184.89,2740.45) -- (181.32,2740.45) -- cycle ;
%Shape: Ellipse [id:dp7734607304087251] 
\draw  [fill={rgb, 255:red, 0; green, 0; blue, 0 }  ,fill opacity=1 ] (181.32,2738.32) .. controls (181.32,2737.56) and (182.12,2736.95) .. (183.1,2736.95) .. controls (184.09,2736.95) and (184.89,2737.56) .. (184.89,2738.32) .. controls (184.89,2739.08) and (184.09,2739.69) .. (183.1,2739.69) .. controls (182.12,2739.69) and (181.32,2739.08) .. (181.32,2738.32) -- cycle ;

%Shape: Triangle [id:dp05619216743707289] 
\draw  [fill={rgb, 255:red, 0; green, 0; blue, 0 }  ,fill opacity=1 ] (183.1,2731.78) -- (184.89,2733.45) -- (181.32,2733.45) -- cycle ;
%Shape: Rectangle [id:dp17125558731272816] 
\draw  [fill={rgb, 255:red, 0; green, 0; blue, 0 }  ,fill opacity=1 ] (181.32,2729.95) -- (184.89,2729.95) -- (184.89,2731.78) -- (181.32,2731.78) -- cycle ;


%Rounded Rect [id:dp9169420583163905] 
\draw  [fill={rgb, 255:red, 74; green, 144; blue, 226 }  ,fill opacity=1 ] (75.24,2638.15) .. controls (75.24,2636.01) and (76.98,2634.27) .. (79.12,2634.27) -- (90.78,2634.27) .. controls (92.92,2634.27) and (94.66,2636.01) .. (94.66,2638.15) -- (94.66,2656.52) .. controls (94.66,2658.66) and (92.92,2660.4) .. (90.78,2660.4) -- (79.12,2660.4) .. controls (76.98,2660.4) and (75.24,2658.66) .. (75.24,2656.52) -- cycle ;
%Straight Lines [id:da7085023615230004] 
\draw    (78.47,2657.13) -- (88.42,2657.13) ;
\draw [shift={(91.42,2657.13)}, rotate = 180] [fill={rgb, 255:red, 0; green, 0; blue, 0 }  ][line width=0.08]  [draw opacity=0] (3.57,-1.72) -- (0,0) -- (3.57,1.72) -- cycle    ;
%Shape: Rectangle [id:dp3433800974074901] 
\draw  [fill={rgb, 255:red, 255; green, 255; blue, 255 }  ,fill opacity=1 ] (105.02,2624.2) .. controls (105.02,2622.54) and (106.36,2621.2) .. (108.02,2621.2) -- (114.97,2621.2) .. controls (116.62,2621.2) and (117.97,2622.54) .. (117.97,2624.2) -- (117.97,2631.27) .. controls (117.97,2632.92) and (116.62,2634.27) .. (114.97,2634.27) -- (108.02,2634.27) .. controls (106.36,2634.27) and (105.02,2632.92) .. (105.02,2631.27) -- cycle ;
%Shape: Ellipse [id:dp06369292120865211] 
\draw  [fill={rgb, 255:red, 0; green, 0; blue, 0 }  ,fill opacity=1 ] (81.71,2647.33) .. controls (81.71,2645.53) and (83.16,2644.07) .. (84.95,2644.07) .. controls (86.74,2644.07) and (88.19,2645.53) .. (88.19,2647.33) .. controls (88.19,2649.14) and (86.74,2650.6) .. (84.95,2650.6) .. controls (83.16,2650.6) and (81.71,2649.14) .. (81.71,2647.33) -- cycle ;
%Straight Lines [id:da9269526800581604] 
\draw    (78.47,2637.53) -- (88.42,2637.53) ;
\draw [shift={(91.42,2637.53)}, rotate = 180] [fill={rgb, 255:red, 0; green, 0; blue, 0 }  ][line width=0.08]  [draw opacity=0] (3.57,-1.72) -- (0,0) -- (3.57,1.72) -- cycle    ;
%Shape: Rectangle [id:dp6216270188458906] 
\draw  [fill={rgb, 255:red, 255; green, 255; blue, 255 }  ,fill opacity=1 ] (105.02,2643.8) .. controls (105.02,2642.14) and (106.36,2640.8) .. (108.02,2640.8) -- (114.97,2640.8) .. controls (116.62,2640.8) and (117.97,2642.14) .. (117.97,2643.8) -- (117.97,2650.87) .. controls (117.97,2652.52) and (116.62,2653.87) .. (114.97,2653.87) -- (108.02,2653.87) .. controls (106.36,2653.87) and (105.02,2652.52) .. (105.02,2650.87) -- cycle ;
%Shape: Rectangle [id:dp9111612565754093] 
\draw  [fill={rgb, 255:red, 255; green, 255; blue, 255 }  ,fill opacity=1 ] (105.02,2663.4) .. controls (105.02,2661.74) and (106.36,2660.4) .. (108.02,2660.4) -- (114.97,2660.4) .. controls (116.62,2660.4) and (117.97,2661.74) .. (117.97,2663.4) -- (117.97,2670.47) .. controls (117.97,2672.12) and (116.62,2673.47) .. (114.97,2673.47) -- (108.02,2673.47) .. controls (106.36,2673.47) and (105.02,2672.12) .. (105.02,2670.47) -- cycle ;
%Rounded Rect [id:dp3662895167475615] 
\draw  [fill={rgb, 255:red, 74; green, 144; blue, 226 }  ,fill opacity=1 ] (75.24,2572.82) .. controls (75.24,2570.67) and (76.98,2568.93) .. (79.12,2568.93) -- (90.78,2568.93) .. controls (92.92,2568.93) and (94.66,2570.67) .. (94.66,2572.82) -- (94.66,2591.18) .. controls (94.66,2593.33) and (92.92,2595.07) .. (90.78,2595.07) -- (79.12,2595.07) .. controls (76.98,2595.07) and (75.24,2593.33) .. (75.24,2591.18) -- cycle ;
%Straight Lines [id:da25541173941389306] 
\draw    (78.47,2591.8) -- (88.42,2591.8) ;
\draw [shift={(91.42,2591.8)}, rotate = 180] [fill={rgb, 255:red, 0; green, 0; blue, 0 }  ][line width=0.08]  [draw opacity=0] (3.57,-1.72) -- (0,0) -- (3.57,1.72) -- cycle    ;
%Straight Lines [id:da6490603362023004] 
\draw    (78.47,2572.2) -- (88.42,2572.2) ;
\draw [shift={(91.42,2572.2)}, rotate = 180] [fill={rgb, 255:red, 0; green, 0; blue, 0 }  ][line width=0.08]  [draw opacity=0] (3.57,-1.72) -- (0,0) -- (3.57,1.72) -- cycle    ;
%Shape: Rectangle [id:dp09534537751920102] 
\draw  [fill={rgb, 255:red, 255; green, 255; blue, 255 }  ,fill opacity=1 ] (105.02,2558.87) .. controls (105.02,2557.21) and (106.36,2555.87) .. (108.02,2555.87) -- (114.97,2555.87) .. controls (116.62,2555.87) and (117.97,2557.21) .. (117.97,2558.87) -- (117.97,2565.93) .. controls (117.97,2567.59) and (116.62,2568.93) .. (114.97,2568.93) -- (108.02,2568.93) .. controls (106.36,2568.93) and (105.02,2567.59) .. (105.02,2565.93) -- cycle ;
%Shape: Triangle [id:dp17187780991693935] 
\draw  [fill={rgb, 255:red, 0; green, 0; blue, 0 }  ,fill opacity=1 ] (84.95,2578.73) -- (88.19,2585.27) -- (81.71,2585.27) -- cycle ;
%Shape: Rectangle [id:dp20760619700938943] 
\draw  [fill={rgb, 255:red, 255; green, 255; blue, 255 }  ,fill opacity=1 ] (105.02,2578.47) .. controls (105.02,2576.81) and (106.36,2575.47) .. (108.02,2575.47) -- (114.97,2575.47) .. controls (116.62,2575.47) and (117.97,2576.81) .. (117.97,2578.47) -- (117.97,2585.53) .. controls (117.97,2587.19) and (116.62,2588.53) .. (114.97,2588.53) -- (108.02,2588.53) .. controls (106.36,2588.53) and (105.02,2587.19) .. (105.02,2585.53) -- cycle ;
%Shape: Rectangle [id:dp1254756717146519] 
\draw  [fill={rgb, 255:red, 255; green, 255; blue, 255 }  ,fill opacity=1 ] (105.02,2598.07) .. controls (105.02,2596.41) and (106.36,2595.07) .. (108.02,2595.07) -- (114.97,2595.07) .. controls (116.62,2595.07) and (117.97,2596.41) .. (117.97,2598.07) -- (117.97,2605.13) .. controls (117.97,2606.79) and (116.62,2608.13) .. (114.97,2608.13) -- (108.02,2608.13) .. controls (106.36,2608.13) and (105.02,2606.79) .. (105.02,2605.13) -- cycle ;
%Rounded Rect [id:dp5122167570992515] 
\draw  [fill={rgb, 255:red, 74; green, 144; blue, 226 }  ,fill opacity=1 ] (75.24,2507.48) .. controls (75.24,2505.34) and (76.98,2503.6) .. (79.12,2503.6) -- (90.78,2503.6) .. controls (92.92,2503.6) and (94.66,2505.34) .. (94.66,2507.48) -- (94.66,2525.85) .. controls (94.66,2527.99) and (92.92,2529.73) .. (90.78,2529.73) -- (79.12,2529.73) .. controls (76.98,2529.73) and (75.24,2527.99) .. (75.24,2525.85) -- cycle ;
%Shape: Rectangle [id:dp7997165245576967] 
\draw  [fill={rgb, 255:red, 0; green, 0; blue, 0 }  ,fill opacity=1 ] (81.71,2513.4) -- (88.19,2513.4) -- (88.19,2519.93) -- (81.71,2519.93) -- cycle ;
%Straight Lines [id:da5891946374027393] 
\draw    (78.47,2526.47) -- (88.42,2526.47) ;
\draw [shift={(91.42,2526.47)}, rotate = 180] [fill={rgb, 255:red, 0; green, 0; blue, 0 }  ][line width=0.08]  [draw opacity=0] (3.57,-1.72) -- (0,0) -- (3.57,1.72) -- cycle    ;
%Straight Lines [id:da04869453147761371] 
\draw    (78.47,2506.87) -- (88.42,2506.87) ;
\draw [shift={(91.42,2506.87)}, rotate = 180] [fill={rgb, 255:red, 0; green, 0; blue, 0 }  ][line width=0.08]  [draw opacity=0] (3.57,-1.72) -- (0,0) -- (3.57,1.72) -- cycle    ;
%Shape: Rectangle [id:dp461817136317475] 
\draw  [fill={rgb, 255:red, 255; green, 255; blue, 255 }  ,fill opacity=1 ] (105.02,2493.53) .. controls (105.02,2491.88) and (106.36,2490.53) .. (108.02,2490.53) -- (114.97,2490.53) .. controls (116.62,2490.53) and (117.97,2491.88) .. (117.97,2493.53) -- (117.97,2500.6) .. controls (117.97,2502.26) and (116.62,2503.6) .. (114.97,2503.6) -- (108.02,2503.6) .. controls (106.36,2503.6) and (105.02,2502.26) .. (105.02,2500.6) -- cycle ;
%Shape: Rectangle [id:dp3455464999335194] 
\draw  [fill={rgb, 255:red, 255; green, 255; blue, 255 }  ,fill opacity=1 ] (105.02,2513.13) .. controls (105.02,2511.48) and (106.36,2510.13) .. (108.02,2510.13) -- (114.97,2510.13) .. controls (116.62,2510.13) and (117.97,2511.48) .. (117.97,2513.13) -- (117.97,2520.2) .. controls (117.97,2521.86) and (116.62,2523.2) .. (114.97,2523.2) -- (108.02,2523.2) .. controls (106.36,2523.2) and (105.02,2521.86) .. (105.02,2520.2) -- cycle ;
%Shape: Rectangle [id:dp8470866636319361] 
\draw  [fill={rgb, 255:red, 255; green, 255; blue, 255 }  ,fill opacity=1 ] (105.02,2532.73) .. controls (105.02,2531.08) and (106.36,2529.73) .. (108.02,2529.73) -- (114.97,2529.73) .. controls (116.62,2529.73) and (117.97,2531.08) .. (117.97,2532.73) -- (117.97,2539.8) .. controls (117.97,2541.46) and (116.62,2542.8) .. (114.97,2542.8) -- (108.02,2542.8) .. controls (106.36,2542.8) and (105.02,2541.46) .. (105.02,2539.8) -- cycle ;
%Shape: Rectangle [id:dp403164852226098] 
\draw  [fill={rgb, 255:red, 0; green, 0; blue, 0 }  ,fill opacity=1 ] (106.64,2492.17) -- (116.35,2492.17) -- (116.35,2501.97) -- (106.64,2501.97) -- cycle ;
%Shape: Rectangle [id:dp5621230825935843] 
\draw  [fill={rgb, 255:red, 0; green, 0; blue, 0 }  ,fill opacity=1 ] (106.64,2511.77) -- (116.35,2511.77) -- (116.35,2521.57) -- (106.64,2521.57) -- cycle ;
%Shape: Rectangle [id:dp7389264507087268] 
\draw  [fill={rgb, 255:red, 0; green, 0; blue, 0 }  ,fill opacity=1 ] (106.64,2531.37) -- (116.35,2531.37) -- (116.35,2541.17) -- (106.64,2541.17) -- cycle ;
%Shape: Triangle [id:dp30029278903908696] 
\draw  [fill={rgb, 255:red, 0; green, 0; blue, 0 }  ,fill opacity=1 ] (111.49,2557.5) -- (116.35,2567.3) -- (106.64,2567.3) -- cycle ;
%Shape: Triangle [id:dp7255517580925794] 
\draw  [fill={rgb, 255:red, 0; green, 0; blue, 0 }  ,fill opacity=1 ] (111.49,2596.7) -- (116.35,2606.5) -- (106.64,2606.5) -- cycle ;
%Shape: Ellipse [id:dp44835561869908225] 
\draw  [fill={rgb, 255:red, 0; green, 0; blue, 0 }  ,fill opacity=1 ] (106.64,2647.33) .. controls (106.64,2644.63) and (108.81,2642.43) .. (111.49,2642.43) .. controls (114.17,2642.43) and (116.35,2644.63) .. (116.35,2647.33) .. controls (116.35,2650.04) and (114.17,2652.23) .. (111.49,2652.23) .. controls (108.81,2652.23) and (106.64,2650.04) .. (106.64,2647.33) -- cycle ;
%Shape: Ellipse [id:dp04851624698727397] 
\draw  [fill={rgb, 255:red, 0; green, 0; blue, 0 }  ,fill opacity=1 ] (106.64,2666.93) .. controls (106.64,2664.23) and (108.81,2662.03) .. (111.49,2662.03) .. controls (114.17,2662.03) and (116.35,2664.23) .. (116.35,2666.93) .. controls (116.35,2669.64) and (114.17,2671.83) .. (111.49,2671.83) .. controls (108.81,2671.83) and (106.64,2669.64) .. (106.64,2666.93) -- cycle ;
%Rounded Rect [id:dp7879328388247069] 
\draw  [fill={rgb, 255:red, 139; green, 87; blue, 42 }  ,fill opacity=1 ] (70,2696.05) .. controls (70,2694.89) and (70.94,2693.95) .. (72.1,2693.95) -- (78.61,2693.95) .. controls (79.77,2693.95) and (80.71,2694.89) .. (80.71,2696.05) -- (80.71,2702.35) .. controls (80.71,2703.51) and (79.77,2704.45) .. (78.61,2704.45) -- (72.1,2704.45) .. controls (70.94,2704.45) and (70,2703.51) .. (70,2702.35) -- cycle ;
%Rounded Rect [id:dp525688441145794] 
\draw  [fill={rgb, 255:red, 65; green, 117; blue, 5 }  ,fill opacity=1 ] (177.32,2696.05) .. controls (177.32,2694.89) and (178.26,2693.95) .. (179.42,2693.95) -- (185.93,2693.95) .. controls (187.09,2693.95) and (188.03,2694.89) .. (188.03,2696.05) -- (188.03,2702.35) .. controls (188.03,2703.51) and (187.09,2704.45) .. (185.93,2704.45) -- (179.42,2704.45) .. controls (178.26,2704.45) and (177.32,2703.51) .. (177.32,2702.35) -- cycle ;
%Rounded Rect [id:dp43081690437304654] 
\draw  [fill={rgb, 255:red, 74; green, 144; blue, 226 }  ,fill opacity=1 ] (288,2696.05) .. controls (288,2694.89) and (288.94,2693.95) .. (290.1,2693.95) -- (296.61,2693.95) .. controls (297.77,2693.95) and (298.71,2694.89) .. (298.71,2696.05) -- (298.71,2702.35) .. controls (298.71,2703.51) and (297.77,2704.45) .. (296.61,2704.45) -- (290.1,2704.45) .. controls (288.94,2704.45) and (288,2703.51) .. (288,2702.35) -- cycle ;
%Rounded Rect [id:dp7559064819268064] 
\draw  [fill={rgb, 255:red, 155; green, 155; blue, 155 }  ,fill opacity=1 ] (70,2714.05) .. controls (70,2712.89) and (70.94,2711.95) .. (72.1,2711.95) -- (78.61,2711.95) .. controls (79.77,2711.95) and (80.71,2712.89) .. (80.71,2714.05) -- (80.71,2720.35) .. controls (80.71,2721.51) and (79.77,2722.45) .. (78.61,2722.45) -- (72.1,2722.45) .. controls (70.94,2722.45) and (70,2721.51) .. (70,2720.35) -- cycle ;
%Rounded Rect [id:dp7575247800639395] 
\draw  [fill={rgb, 255:red, 255; green, 255; blue, 255 }  ,fill opacity=1 ] (176.6,2714.05) .. controls (176.6,2712.89) and (177.54,2711.95) .. (178.7,2711.95) -- (185.21,2711.95) .. controls (186.37,2711.95) and (187.31,2712.89) .. (187.31,2714.05) -- (187.31,2720.35) .. controls (187.31,2721.51) and (186.37,2722.45) .. (185.21,2722.45) -- (178.7,2722.45) .. controls (177.54,2722.45) and (176.6,2721.51) .. (176.6,2720.35) -- cycle ;
%Shape: Rectangle [id:dp7743067472888825] 
\draw  [fill={rgb, 255:red, 0; green, 0; blue, 0 }  ,fill opacity=1 ] (169.76,2526.47) -- (176.24,2526.47) -- (176.24,2533) -- (169.76,2533) -- cycle ;
%Shape: Triangle [id:dp266876728986688] 
\draw  [fill={rgb, 255:red, 0; green, 0; blue, 0 }  ,fill opacity=1 ] (153.58,2582) -- (156.81,2588.53) -- (150.34,2588.53) -- cycle ;
%Shape: Rectangle [id:dp39069234134302544] 
\draw  [fill={rgb, 255:red, 255; green, 255; blue, 255 }  ,fill opacity=1 ] (260.4,2536) .. controls (260.4,2534.34) and (261.74,2533) .. (263.4,2533) -- (270.35,2533) .. controls (272.01,2533) and (273.35,2534.34) .. (273.35,2536) -- (273.35,2543.07) .. controls (273.35,2544.72) and (272.01,2546.07) .. (270.35,2546.07) -- (263.4,2546.07) .. controls (261.74,2546.07) and (260.4,2544.72) .. (260.4,2543.07) -- cycle ;
%Shape: Triangle [id:dp913274679122853] 
\draw  [fill={rgb, 255:red, 0; green, 0; blue, 0 }  ,fill opacity=1 ] (266.87,2539.64) -- (271.73,2544.32) -- (262.02,2544.32) -- cycle ;
%Shape: Rectangle [id:dp13937683912646381] 
\draw  [fill={rgb, 255:red, 0; green, 0; blue, 0 }  ,fill opacity=1 ] (262.02,2534.52) -- (271.73,2534.52) -- (271.73,2539.64) -- (262.02,2539.64) -- cycle ;

%Shape: Rectangle [id:dp038836437830894965] 
\draw  [fill={rgb, 255:red, 255; green, 255; blue, 255 }  ,fill opacity=1 ] (184,2526.2) .. controls (184,2524.54) and (185.35,2523.2) .. (187,2523.2) -- (193.95,2523.2) .. controls (195.61,2523.2) and (196.95,2524.54) .. (196.95,2526.2) -- (196.95,2533.27) .. controls (196.95,2534.92) and (195.61,2536.27) .. (193.95,2536.27) -- (187,2536.27) .. controls (185.35,2536.27) and (184,2534.92) .. (184,2533.27) -- cycle ;
%Shape: Rectangle [id:dp19254423580543967] 
\draw  [fill={rgb, 255:red, 255; green, 255; blue, 255 }  ,fill opacity=1 ] (184,2545.8) .. controls (184,2544.14) and (185.35,2542.8) .. (187,2542.8) -- (193.95,2542.8) .. controls (195.61,2542.8) and (196.95,2544.14) .. (196.95,2545.8) -- (196.95,2552.87) .. controls (196.95,2554.52) and (195.61,2555.87) .. (193.95,2555.87) -- (187,2555.87) .. controls (185.35,2555.87) and (184,2554.52) .. (184,2552.87) -- cycle ;
%Shape: Rectangle [id:dp2520491425384046] 
\draw  [fill={rgb, 255:red, 255; green, 255; blue, 255 }  ,fill opacity=1 ] (184,2611.13) .. controls (184,2609.48) and (185.35,2608.13) .. (187,2608.13) -- (193.95,2608.13) .. controls (195.61,2608.13) and (196.95,2609.48) .. (196.95,2611.13) -- (196.95,2618.2) .. controls (196.95,2619.86) and (195.61,2621.2) .. (193.95,2621.2) -- (187,2621.2) .. controls (185.35,2621.2) and (184,2619.86) .. (184,2618.2) -- cycle ;
%Shape: Ellipse [id:dp15553609566598348] 
\draw  [fill={rgb, 255:red, 0; green, 0; blue, 0 }  ,fill opacity=1 ] (185.62,2614.67) .. controls (185.62,2611.96) and (187.8,2609.77) .. (190.48,2609.77) .. controls (193.16,2609.77) and (195.33,2611.96) .. (195.33,2614.67) .. controls (195.33,2617.37) and (193.16,2619.57) .. (190.48,2619.57) .. controls (187.8,2619.57) and (185.62,2617.37) .. (185.62,2614.67) -- cycle ;
%Shape: Rectangle [id:dp876514020937001] 
\draw  [fill={rgb, 255:red, 255; green, 255; blue, 255 }  ,fill opacity=1 ] (184,2630.73) .. controls (184,2629.08) and (185.35,2627.73) .. (187,2627.73) -- (193.95,2627.73) .. controls (195.61,2627.73) and (196.95,2629.08) .. (196.95,2630.73) -- (196.95,2637.8) .. controls (196.95,2639.46) and (195.61,2640.8) .. (193.95,2640.8) -- (187,2640.8) .. controls (185.35,2640.8) and (184,2639.46) .. (184,2637.8) -- cycle ;
%Shape: Triangle [id:dp7709663344043356] 
\draw  [fill={rgb, 255:red, 0; green, 0; blue, 0 }  ,fill opacity=1 ] (318.02,2607.91) -- (321.26,2610.75) -- (314.78,2610.75) -- cycle ;
%Shape: Ellipse [id:dp7056365106577567] 
\draw  [fill={rgb, 255:red, 0; green, 0; blue, 0 }  ,fill opacity=1 ] (314.78,2606.77) .. controls (314.78,2605.36) and (316.23,2604.21) .. (318.02,2604.21) .. controls (319.81,2604.21) and (321.26,2605.36) .. (321.26,2606.77) .. controls (321.26,2608.19) and (319.81,2609.33) .. (318.02,2609.33) .. controls (316.23,2609.33) and (314.78,2608.19) .. (314.78,2606.77) -- cycle ;

%Shape: Rectangle [id:dp6280600322241623] 
\draw  [fill={rgb, 255:red, 255; green, 255; blue, 255 }  ,fill opacity=1 ] (341.33,2509.87) .. controls (341.33,2508.21) and (342.67,2506.87) .. (344.33,2506.87) -- (351.28,2506.87) .. controls (352.93,2506.87) and (354.28,2508.21) .. (354.28,2509.87) -- (354.28,2516.93) .. controls (354.28,2518.59) and (352.93,2519.93) .. (351.28,2519.93) -- (344.33,2519.93) .. controls (342.67,2519.93) and (341.33,2518.59) .. (341.33,2516.93) -- cycle ;
%Shape: Triangle [id:dp6117778913102669] 
\draw  [fill={rgb, 255:red, 0; green, 0; blue, 0 }  ,fill opacity=1 ] (347.8,2513.5) -- (352.66,2518.19) -- (342.95,2518.19) -- cycle ;
%Shape: Rectangle [id:dp5299063055137347] 
\draw  [fill={rgb, 255:red, 0; green, 0; blue, 0 }  ,fill opacity=1 ] (342.95,2508.39) -- (352.66,2508.39) -- (352.66,2513.5) -- (342.95,2513.5) -- cycle ;



% Text Node
\draw (118.08,2734.5) node  [font=\scriptsize] [align=left] {robot carrier};
% Text Node
\draw (226.5,2716.5) node  [font=\scriptsize] [align=left] {pick/drop zone};
% Text Node
\draw (121.07,2717.5) node  [font=\scriptsize] [align=left] {obstacle zone};
% Text Node
\draw (344.5,2697.5) node  [font=\scriptsize] [align=left] {input product sign};
% Text Node
\draw (236.5,2698.5) node  [font=\scriptsize] [align=left] {output product sign};
% Text Node
\draw (125,2699.5) node  [font=\scriptsize] [align=left] {crafting scheme};
% Text Node
\draw (342.82,2712.5) node  [font=\scriptsize] [align=left] {primary products};
% Text Node
\draw (238,2734.5) node  [font=\scriptsize] [align=left] {secondary products};


\end{tikzpicture}
  \caption{An illustrative view of the "Warehouse Flow Management" environment: agents can move up, down, left, and right, pick up and drop a product in a pick/drop area if they are close enough. Agents must coordinate to: i) pick up primary products from the input conveyor pick/drop areas (blue zones); ii) drop them in the crafting machine pick/drop areas (brown zones), which transform primary products into a single secondary product according to the crafting schema; iii) retrieve the created secondary products to drop them in the output conveyor pick/drop areas (green zones)}
  \label{fig:warehouse}
\end{figure}

\subsection{Phase 1: Modeling}

In the initial phase, users define the environment dynamics, goals, and constraints. The CybMASDE tool facilitates this process through the PettingZoo and Gymnasium APIs, which offer a wide range of pre-built environments for rapid prototyping. Users can choose from existing environments or create custom ones that accurately reflect the desired operational scenarios. The CybMASDE tool's graphical interface simplifies the process of defining environment-specific parameters, such as state and action spaces, reward functions, and observational models. For more complex environments, users can leverage the API to directly code or integrate specialized environments into the CybMASDE tool.

Furthermore, the CybMASDE tool provides an automated modeling of the environment using traces and a RNN from the Tensorflow~\cite{tensorflow2015-whitepaper} library.
These traces are then structured into fixed-length formats using sliding windows, preparing them for training. The RNN architecture comprises two SimpleRNN layers, each with 64 units and ReLU activation, followed by a dense output layer. The network is compiled with the Adam optimizer and mean squared error loss to support the regression task, enabling it to predict subsequent observations based on the current sequence of actions and observations. Training is conducted over 50 epochs with a batch size of 32, which is optimized based on the complexity of the environment, ensuring adequate exposure to diverse trajectories. Post-training, the model undergoes evaluation using test data, with fine-tuning adjustments made according to metrics such as mean squared error, allowing for enhanced predictive accuracy. The trained RNN is then incorporated into the CybMASDE tool as a new PettingZoo environment.

the CybMASDE tool also provides tools for specifying organizational constraints in line with the $\mathcal{M}OISE^+$MARL framework. Users can define agent roles, missions, and goals through structured input forms, while an API endpoint allows for the direct upload of JSON files that contain these specifications. This feature enables the modeling of complex organizational structures necessary for scenarios with intricate role hierarchies or mission dependencies.

\textbf{WFM case study}: In this phase, the inputs include the WFM PettingZoo environment to get traces by running agents randomly, and an informal description of the goal and additional constraints such as safety requirements or roles.
% TODO: dire quels sont les modèles organisationels considérés
The output is another PettingZoo modeled from the initial environment including the defined reward function modeling the goal. It is automatically modeled using the proposed "Imitation Learning" with RNN based on collected agent traces. The process also yields a JSON file representing the $\mathcal{M}OISE^+$MARL model with linkers and constraint guides whose example is given here:

{
\footnotesize
\begin{verbatim}
{ 
  "linkers": {
    "ar": {"agent_0": "r_0", "agent_1": "r_1", "agent_2": "r_2"},
    "rcg": {  "r_0": ["rag_0","rrg_0"],
              "r_1": ["rag_1","rrg_1"],
              "r_2": ["rag_2","rrg_2"]},
    "gcg": {"g_0": "grg_0", "g_1": "grg_1"}
  },
  "constraint_guides": {
    "rag": {"rag_0": {
      "patterns": {"([o1,a1](1,1),o2)": [["a2",1], ["a0",0]]},
      "scripts": ["rag_0.py"]}},
    "rrg": {"rrg_0": {
      "patterns": {"([o1,a1](1,1),(o2, a2))": 30},
      "scripts": ["rrg_0.py"]}},
    "grg": {"grg_0": {
      "patterns": {"([#Any](0,*),o4](1,1))": 100},
      "scripts": ["grg_0.py"]}}
  },
  "moise_specifications": {
    "mo": {"m_0": ["g_0"], "m_1": ["g_1"], "m_2": ["g_1"]},
    "deontic_specifications": { "r_0": ["m_0", 0, "Any"],
                                "r_1": ["m_2", 0, "Any"],
                                "r_2": ["m_2", 0, "Any"]}
  }
}
\end{verbatim}
}

\subsection{Phase 2: Solving}

In the training phase, the CybMASDE tool utilizes MARLlib, which supports various MARL algorithms, such as MAPPO, MADDPG, DynaQ+, QMIX (Q-value Mixing), and COMA (Counterfactual Multi-Agent (COMA)). Through the interface, users can select appropriate algorithms based on the environment characteristics and learning requirements. If not specified, all algorithms are assessed automatically. Once selected, the CybMASDE tool manages the training process, taking advantage of parallel computing capabilities to accelerate learning, which is especially beneficial for large-scale environments with numerous agents.

A standout feature of the CybMASDE tool is its HPO module, which automatically tunes hyperparameters like learning rates, discount factors, and exploration rates. Users can configure optimization parameters through the graphical interface or use API calls for batch processing and fine-tuning over multiple training runs. The CybMASDE tool provides real-time monitoring of training metrics, enabling users to track the progress of agents as they refine their policies. The API also supports remote access, allowing users to start, stop, and analyze training runs from external scripts.

\textbf{WFM case study}: The input for this phase is the JSON file representing the $\mathcal{M}OISE^+$MARL model and the PettingZoo environment. The output is a trained MARLlib model of the agent's policies. This model can be saved as a checkpoint folder, which contains files for the weights (e.g., .pth or .pt) and the parameters (e.g., .yaml or .json) required to execute the learned policy.

\subsection{Phase 3: Analysis}

Once training is complete, the CybMASDE tool assists in analyzing agent behaviors and inferring organizational roles using the HEMM method. The tool applies unsupervised learning techniques, such as hierarchical clustering and K-means, to identify common patterns in agent histories, which are then mapped to roles, missions, and goals as defined in the $\mathcal{M}OISE^+$MARL framework.

The analysis results are presented through visualizations, including dendrograms and state-transition graphs, accessible via the graphical interface. These visualizations help users understand the emergent behaviors and the distribution of roles across agents. Additionally, the API provides detailed data output, which can be used for further analysis in external tools. The CybMASDE tool's capability to infer organizational specifications ensures that agent policies align with predefined constraints, making it easier to enforce compliance and safety standards across the MAS.

\textbf{WFM case study}: The input in this phase is the trained MARLlib model checkpoint. The output is an updated JSON representation of the $\mathcal{M}OISE^+$MARL model, which adds patterns for rag, rrg, and grg, potentially introducing new roles, missions, or goals that were not initially assigned to any role.

\subsection{Phase 4: Transfer}

In the final phase, the CybMASDE tool supports the deployment of agents in real-world environments or continued testing in emulated settings. The tool offers functionalities to deploy trained policies directly into Gymnasium-compatible environments, allowing for rapid testing and validation. Users can verify agent behavior in real time, using the interface to adjust parameters and observe the effects on the system's performance.

For real-world applications, the CybMASDE tool enables the export of agent policies into formats suitable for integration with other platforms, ensuring compatibility and ease of deployment. The system supports incremental deployment, where policies can be updated or replaced independently, ensuring modularity and resilience. Furthermore, the CybMASDE tool's API provides endpoints for managing deployed agents, monitoring their adherence to organizational constraints, and evaluating their performance against operational benchmarks.

\textbf{WFM case study}: The input for this phase is the curated JSON representation of the $\mathcal{M}OISE^+$MARL model. The output is the initial PettingZoo environment, where agents are deployed with roles linked to refined Constraint Guides, as well as deployment into the real environment. The deployment process ensures that agents adhere to safety and performance requirements, reflecting the trained policies accurately in the operational context.

\section{Results and discussion}
\label{sec:results_discussion}

In this section, we present the results of applying SAMMASD to the WFM scenario. The primary goals assessed include SAMMASD's capabilities to: \quad i) automate the MAS design process; \quad ii) reduce reliance on expert knowledge; \quad iii) facilitate scalable deployment. We evaluate SAMMASD across these goals, using metrics and qualitative observations to substantiate the methodology's effectiveness.

\subsection{Automation of the MAS design process}

SAMMASD offers notable automation across the MAS design phases, from modeling to deployment. By integrating the $\mathcal{M}OISE^+$ organizational model within a MARL framework, SAMMASD supports agents in learning roles and goals with reduced manual configuration.

In the WFM environment, the automated environment modeling phase captured key real-world conditions with a fidelity level exceeding 90\%, based on the accuracy of predicted state transitions. This suggests that the RNN-based imitation learning model effectively approximated the operational dynamics of the warehouse.

During training, SAMMASD utilized MARL algorithms that worked toward optimizing agent behaviors. The training process achieved an average cumulative reward that met the predefined threshold within 50 training epochs, suggesting that SAMMASD can identify and refine policies in alignment with organizational constraints. Additionally, the HEMM analysis phase inferred roles and missions with limited manual intervention, clustering agent behaviors into coherent roles with a 95\% alignment to predefined organizational goals.

These findings suggest that SAMMASD supports automation of key elements within the MAS design process, potentially reducing the need for manual design adjustments.

\subsection{Reduction of reliance on expert knowledge}

SAMMASD’s approach to minimizing dependency on expert knowledge is demonstrated through its ability to infer organizational specifications based on agent behaviors. Using automated tools within CybMASDE, SAMMASD reduced the requirement for expert input across various design phases.

During the modeling phase, there was no need for experts to manually define role-based behaviors or organizational constraints, as the CybMASDE tool generated these from the collected data. The role inference accuracy achieved by HEMM was validated against manually defined roles, with an alignment rate of over 90\%, indicating SAMMASD’s capability to derive organizational structures that are consistent with expert-defined parameters. 

Moreover, the HPO module optimized agent behaviors with minimal expert tuning, autonomously identifying favorable parameter configurations. This approach may help reduce the extent of expert involvement, offering a potential pathway for non-experts to engage in MAS development.

\subsection{Scalability of deployment}

SAMMASD’s deployment process was evaluated for scalability in both emulated and real-world settings. Its modular framework facilitated incremental deployment, with agents exhibiting consistent behaviors across various WFM configurations.

Performance metrics for different agent numbers (ranging from 3 to 10 agents) in the WFM environment showed stability, with less than a 5\% variance in reward metrics across configurations, suggesting SAMMASD’s adaptability to different scales without significant performance fluctuations.

Regarding deployment time, SAMMASD’s transfer mechanisms appeared to reduce setup duration by approximately 60\% compared to traditional AOSE methodologies. CybMASDE’s incremental deployment functionality further contributed to scalability with limited manual intervention, potentially streamlining the deployment process for larger MAS configurations.


\section{Conclusion and perspectives}
\label{sec:conclusion_perspectives}

This paper introduces SAMMASD, a semi-automated approach for MAS design using MARL, evaluated through a WFM scenario via the CybMASDE tool. SAMMASD aims to address certain challenges in AOSE methodologies by embedding organizational specifications into MARL training, which could enhance explainability and compliance.

Preliminary findings suggest that SAMMASD may reduce reliance on expert input, potentially making the approach accessible to a broader user base. The observed automation in training and role inference might contribute to scalability, although areas such as full deployment automation and adaptability to dynamic environments require further exploration.

Future work may involve enhancing CybMASDE with automated deployment and online learning capabilities, and extending SAMMASD to domains with high safety requirements. These improvements could increase SAMMASD's utility for complex MAS design scenarios, while ongoing development in adaptability and deployment automation might strengthen its practical relevance across diverse environments.

\section*{References}

% \bibliographystyle{abbrv}
\bibliographystyle{IEEEtran}
\renewcommand
\refname{}
\bibliography{references}

\end{document}
