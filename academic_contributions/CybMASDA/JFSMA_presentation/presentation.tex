\documentclass{beamer}

\usetheme{metropolis}

\usepackage{fontspec}

\title{Une Approche basée sur l’Apprentissage par Renforcement pour l’Ingénierie Organisationnelle d’un SMA}
\author{Julien Soulé, Jean-Paul Jamont, Michel Occello, Louis-Marie Traonouez, Paul Théron}
\date{\today}

\begin{document}

\begin{frame}
  \titlepage
\end{frame}

\begin{frame}{Introduction}
    \begin{itemize}
        \item Les Systèmes Multi-Agents (SMA) de Cyberdéfense doivent assurer la protection de systèmes en réseau hétérogènes et distribués.
        \item L'organisation des agents est cruciale pour atteindre les objectifs de cyberdéfense tout en respectant les contraintes environnementales.
        \item Problème : La recherche empirique d'une organisation efficace est complexe et risquée.
        \item Contribution : Proposition d'une approche générique combinant l'apprentissage par renforcement et un modèle organisationnel pour assister la conception de l'organisation d'un SMA.
    \end{itemize}
\end{frame}

\begin{frame}{Cadre théorique}
    \begin{itemize}
        \item Modèle organisationnel MOISE+ : Permet de lier les politiques des agents à des spécifications organisationnelles formelles (structurelles, fonctionnelles, déontiques).
        \item Apprentissage par renforcement multi-agent (MARL) : Processus où les agents apprennent à collaborer pour maximiser une récompense commune.
        \item Défi : Combiner ces deux cadres pour fournir des organisations expliquant les comportements émergents des agents.
    \end{itemize}
\end{frame}

\begin{frame}{Approche AOMEA}
    \begin{itemize}
        \item AOMEA : Assisted Multi-Agent System Organization Engineering Approach.
        \item Automatisation de la conception SMA en combinant le MARL avec des spécifications organisationnelles.
        \item Phases de l'approche : 
        \begin{enumerate}
            \item Modélisation de l'environnement et des spécifications.
            \item Résolution avec algorithme MARL pour trouver des politiques optimales.
            \item Analyse des politiques apprises et extraction des rôles.
            \item Développement du SMA final en utilisant les spécifications organisationnelles raffinées.
        \end{enumerate}
    \end{itemize}
\end{frame}

\begin{frame}{Outil d'ingénierie}
    \begin{itemize}
        \item PRAHOM (Partial Relations with Agent History and Organization Model) : Outil pour lier les politiques des agents à des modèles organisationnels.
        \item Intégré à PettingZoo : API permettant d'appliquer des algorithmes MARL avec des contraintes organisationnelles.
        \item Utilisation : Entraînement des agents avec des rôles contraints, inférence des spécifications organisationnelles après apprentissage.
    \end{itemize}
\end{frame}

\begin{frame}{Évaluation dans des environnements de jeu coopératif}
    \begin{itemize}
        \item Environnements simulés : Pistonball, Predator-Prey, Knights Archers Zombies (KAZ), Cyberdéfense.
        \item Cas étudiés :
        \begin{enumerate}
            \item Sans spécifications organisationnelles (NTS).
            \item Avec spécifications partielles (PTS).
            \item Avec spécifications complètes (FTS).
        \end{enumerate}
        \item Résultats : L'approche AOMEA accélère la convergence et améliore la stabilité des performances.
    \end{itemize}
\end{frame}

\begin{frame}{Conclusion}
    \begin{itemize}
        \item AOMEA propose un cadre méthodologique pour intégrer des modèles organisationnels au MARL.
        \item Les résultats montrent une convergence accélérée dans les environnements étudiés.
        \item Limites : Difficulté à reconstruire les comportements émergents de manière systématique.
        \item Travaux futurs : Approfondir les techniques non supervisées pour mieux caractériser les organisations émergentes.
    \end{itemize}
\end{frame}

\end{document}
