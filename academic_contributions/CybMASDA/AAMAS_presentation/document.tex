\documentclass[9pt, aspectratio=169]{beamer}
% \documentclass[10pt]{beamer}
\usepackage[utf8]{inputenc}
\usepackage[T1]{fontenc}
\usepackage[english]{babel}
\usetheme{Frankfurt}

\usepackage[backend=biber, style=authoryear]{biblatex}
\addbibresource{references.bib}

%\usepackage{lmodern}
\usepackage{amsfonts,amssymb,amsmath}
\usepackage[english]{babel}
\usetheme{Frankfurt}

\usepackage{csquotes}
\usepackage{setspace}

\usepackage{colortbl}
\usepackage{tabularx}
\renewcommand\tabularxcolumn[1]{m{#1}}

\setbeamertemplate{navigation symbols}{}
\setbeamertemplate{footline}[frame number]{}

% --- Tickz
\usepackage{physics}
\usepackage{amsmath}
\usepackage{tikz}
\usepackage{mathdots}
\usepackage{yhmath}
\usepackage{cancel}
\usepackage{color}
\usepackage{siunitx}
\usepackage{array}
\usepackage{multirow}
\usepackage{amssymb}
\usepackage{gensymb}
\usepackage{tabularx}
\usepackage{extarrows}
\usepackage{booktabs}
\usetikzlibrary{fadings}
\usetikzlibrary{patterns}
\usetikzlibrary{shadows.blur}
\usetikzlibrary{shapes}

% ---------

\usepackage{booktabs}
\usepackage{setspace}
\usepackage{amssymb}
\usepackage{adjustbox}
\usepackage{pifont}
\usepackage[inkscapeformat=png]{svg}
\usepackage{graphicx}
\usepackage{times}
\setbeamertemplate{caption}[numbered]
% % \setbeamertemplate{bibliography item}{[\theenumiv]}

\setbeamerfont{bibliography item}{size=\tiny}
\setbeamerfont{bibliography entry author}{size=\tiny}
\setbeamerfont{bibliography entry title}{size=\tiny}
\setbeamerfont{bibliography entry location}{size=\tiny}
\setbeamerfont{bibliography entry note}{size=\tiny}

\setbeamerfont{frametitle}{size=\large}

\usepackage{caption}
\usepackage{float}
\usepackage{xcolor}
\usepackage{listings}
\usepackage{animate}

\definecolor{codegreen}{rgb}{0,0.6,0}
\definecolor{codegray}{rgb}{0.5,0.5,0.5}
\definecolor{codepurple}{rgb}{0.58,0,0.82}
\definecolor{backcolour}{rgb}{0.95,0.95,0.92}
 
\lstdefinestyle{mystyle}{
    backgroundcolor=\color{backcolour},   
    commentstyle=\color{codegreen},
    keywordstyle=\color{magenta},
    numberstyle=\tiny\color{codegray},
    stringstyle=\color{codepurple},
    basicstyle=\footnotesize,
    breakatwhitespace=false,         
    breaklines=true,                 
    captionpos=b,                    
    keepspaces=true,                 
    numbers=left,                    
    numbersep=5pt,                  
    showspaces=false,                
    showstringspaces=false,
    showtabs=false,                  
    tabsize=2
}
 
\lstset{style=mystyle}

\usepackage{ragged2e}
\setbeamercolor{section in foot}{fg=white,bg=darkorange}
\setbeamercolor{subsection in foot}{fg=white,bg=darkorange}
\setbeamercolor{frametitle}{fg=white, bg=darkorange}
\setbeamercolor{title}{fg=white, bg=darkorange}
\setbeamercolor{frame}{bg=darkorange}
\setbeamercolor{block title}{bg=darkorange,fg=white}

\setbeamercolor{item}{fg=darkorange}

% \definecolor{darkorange}{rgb}{0.81, 0.52, 0.05}
\definecolor{darkorange}{rgb}{1,0.5,0}
\definecolor{darkorange2}{rgb}{1, 0.64, 0.2}
\definecolor{honeydew}{rgb}{1, 0.85, 0.45}


\newenvironment{variableblock}[3]{%
  \setbeamercolor{block body}{#2}
  \setbeamercolor{block title}{#3}
  \begin{block}{#1}}{\end{block}}

\newenvironment{prosblock}[1]{%
  % \setbeamercolor{block body}{bg=blue,fg=white}
  \setbeamercolor{block title}{bg=blue,fg=white}
  \begin{block}{#1}}{\end{block}}

\newenvironment{consblock}[1]{%
  % \setbeamercolor{block body}{bg=red,fg=white}
  \setbeamercolor{block title}{bg=red,fg=white}
  \begin{block}{#1}}{\end{block}}

\newcommand{\cmark}{\ding{51}}%
\newcommand{\xmark}{\ding{55}}%

\renewcommand{\arraystretch}{1.5}

% Please add the following required packages to your document preamble:
\usepackage{booktabs}
\usepackage{multirow}
\usepackage{colortbl}
% Beamer presentation requires \usepackage{colortbl} instead of \usepackage[table,xcdraw]{xcolor}

\usepackage{tabularray}\UseTblrLibrary{varwidth}
\usepackage{xcolor}
\def\BibTeX{{\rm B\kern-.05em{\sc i\kern-.025em b}\kern-.08em
    T\kern-.1667em\lower.7ex\hbox{E}\kern-.125emX}}
% \usepackage{cite}
\usepackage{amsmath}
\newcommand{\probP}{\text{I\kern-0.15em P}}
\usepackage{etoolbox}
\patchcmd{\thebibliography}{\section*{\refname}}{}{}{}

\setlength\tabcolsep{0.5pt}

\renewcommand{\arraystretch}{0.9}
\setlength{\tabcolsep}{2pt}

\usepackage{pgffor}

\begin{document}

\author{\textbf{Julien Soulé$^{1,2}$}, Jean-Paul Jamont$^1$, Michel Occello$^1$, Louis-Marie Traonouez$^2$, Paul Théron$^3$}

\title{\textbf{Towards Assisted MAS Design: A Library for
Explainable MARL with Organizational Model}}

\subtitle{ECAI 2024 Demo Presentation}

% \logo{\includegraphics[scale=0.01]{figures/grenoble-inp_logo.png}}

\institute{\footnotesize \textit{University Grenoble Alpes, Grenoble INP, LCIS, 26000, Valence, France \\
$^1$\{julien.soule, jean-paul.jamont, michel.occello\}@lcis.grenoble-inp.fr \\ \phantom{U} \\
Thales Land and Air Systems, BL IAS, 35000, Rennes, France \\
$^2$\{julien.soule, louis-marie.traonouez\}@thalesgroup.com \\ \phantom{U} \\
AICA IWG, La Guillermie, France \\
$^3$paul.theron@orange.fr}}


\date{\textit{\footnotesize May 9, 2024}}

%\subject{}
\setbeamercovered{transparent}
%\setbeamertemplate{navigation symbols}{}
\begin{frame}[plain]
	\maketitle\vspace{-0.8cm}
	\begin{figure}[ht!]
		\centering
            \includegraphics[height=0.8cm]{figures/la-ruche_logo.png}
            \hspace{0.8cm}
            \includegraphics[height=0.8cm]{figures/lcis_logo.png}
            \hspace{0.8cm}
		\includegraphics[height=0.8cm]{figures/grenoble-inp_logo.png}
            \hspace{0.8cm}
            \includegraphics[height=0.8cm]{figures/uga_logo.jpg}
	\end{figure}
\end{frame}

\begin{frame}{Content}
  \tableofcontents
\end{frame}

% \AtBeginSection[]{
%     \begin{frame}
%         \frametitle{}
%         \tableofcontents[currentsection]
%     \end{frame}
% }

%%%%%%%%%%%%%%%%%%%%%%%%%%%%%%%%%%%%

\addtocounter{framenumber}{-2}

\section{Introduction}

\begin{frame}{Introduction}{General context}

    \begin{columns}

        \begin{column}{0.7\textwidth}

            \begin{itemize}
                \item \textbf{Increasing attack surface}
                      \begin{itemize}
                          \item IoT/IoBT: drones, autonomous vehicles
                      \end{itemize}
                \item \textbf{Challenges during attack}
                      \begin{itemize}
                          \item Operators' limitations: time constraints, workload, complexity\dots
                          \item Environment's limitations: jamming, communication interruption\dots
                      \end{itemize}
            \end{itemize}

            \ \\

            $\Longrightarrow$ Need for: \textbf{reactivity, flexibiity, autonomy}\dots

            \begin{itemize}
                \item A Multi-Agent approach for Cyberdefense
                      \begin{itemize}
                          \item An agent\dots
                          \item A Multi-Agent System (MAS)\dots
                      \end{itemize}
            \end{itemize}

            \ \\

            $\Longrightarrow$ Promising for: \textbf{adaptation, scalability, sub-task delegation}\dots

        \end{column}

        \begin{column}{0.4\textwidth}
            \begin{figure}
                \includegraphics[width=\linewidth]{figures/casino.jpg}
                \caption*{\tiny\url{https://hackread.com/hackers-casinos-fish-tank-smart-thermometer-hack/}}
            \end{figure}

            \vspace{0.cm}

            \includegraphics[width=\linewidth]{figures/company_network.png}
        \end{column}

    \end{columns}

\end{frame}

\begin{frame}{Introduction}{General context}

    \begin{columns}

        \begin{column}{0.5\textwidth}

            \begin{itemize}
                \item MASCARA (Multi Agent Centric AICA Reference Architecture)
                      \begin{itemize}
                          \item AICA theorized by “IST-152 NATO” (2016-2019)
                      \end{itemize}

                \item Detect, identify and characterize anomalies/attacks

                      \begin{itemize}
                          \item Plan and execute countermeasures
                      \end{itemize}

                \item Communicate with C2 / operators…

                      \begin{itemize}
                          \item Be autonomous, stealthy, interoperable, capable of learning
                      \end{itemize}

                \item MASCARA: A Multi-Agent vision of AICA
                      \begin{itemize}
                          \item An implicit organization
                      \end{itemize}
            \end{itemize}

        \end{column}

        \hspace{-2ex}
        \begin{column}{0.6\textwidth}
            \includegraphics[width=\linewidth]{figures/mascara.png}
        \end{column}

    \end{columns}

\end{frame}

\begin{frame}{Introduction}{Problem}

    \begin{alertblock}{General problem}
        \textbf{What organizational mechanisms of the Cyberdefense MAS (AICA) to optimize its operation taking into account its constraints?}
    \end{alertblock}

    \begin{columns}

        \begin{column}{0.6\textwidth}
            \begin{figure}
                \centering
                \includegraphics[width=0.95\linewidth]{figures/general_problem_illustration.png}
            \end{figure}
        \end{column}

        \begin{column}{0.5\textwidth}
            \textbf{Literature study of available Cyberdefense MAS (CMAS) organizations}

            \begin{itemize}
                \item Few works dealing with a Multi-Agent approach to Cyberdefense
                \item Difficult to have a general vision of the effects of the organizations involved depending on the deployment environment
            \end{itemize}

            \ \\

            $\Longrightarrow$ \textbf{Need a study framework to address the problem…}
        \end{column}
    \end{columns}

\end{frame}

\begin{frame}{Introduction}{Approach for addressing the problem}

    A methodological contribution:
    \begin{itemize}

        % \item Review of work for the development of Cyberdefense MAS
        %       \begin{itemize}
        %           \item Expectations of a methodology for the development of Cyberdefense MAS
        %           \item Overview of available work versus expectations
        %           \item Discussion on methodological obstacles
        %       \end{itemize}

        \item \textbf{Need for modeling problem \& design foundation}
              \begin{itemize}
                  \item[$\rightarrow$] \textbf{CybMASFM}: Markovian framework + Digital Twins (simulation/emulation coupling)
              \end{itemize}

        \item \textbf{Need for automated safe design}
              \begin{itemize}
                  \item[$\rightarrow$] \textbf{CybMASDA}: Comprehensive design process
                      \begin{itemize}
                          \item[$\rightarrow$] OMARL: MARL + Organizational model
                      \end{itemize}
              \end{itemize}

        \item \textbf{Need for practical design means}
              \begin{itemize}
                  \item[$\rightarrow$] \textbf{CybMASDE}: implemented CybMASDA as an API + GUI
              \end{itemize}

    \end{itemize}

    \ \\
    \begin{itemize}

        \item Academic \& industrial \textbf{case studies} for AICA\dots
              \begin{itemize}
                  \item Drone swarm
                  \item Company Infrastructure
                  \item Kubernetes/Drones environment
              \end{itemize}

    \end{itemize}

\end{frame}

\section{CybMASFM}

\begin{frame}{Cyberdefense Multi-Agent Systems Formal Model}{Overview}

    \begin{columns}

        \hspace{-2ex}

        \begin{column}{0.45\textwidth}

            \begin{figure}
                \includegraphics[width=\linewidth]{figures/marl_illustration.png}
            \end{figure}

            {\tiny \begin{spacing}{0.5}
                Soulé, J., Jamont, J.-P., Occello, M., Théron, P., \& Traonouez, L.-M. Towards a Multi-Agent Simulation of Cyber-attackers and Cyber-defenders Battles. IEEE SMC 2023.
            \end{spacing}}

        \end{column}

        \begin{column}{0.65\textwidth}
            \vspace{-2ex}

            \begin{center}
                \begin{minipage}{0.95\linewidth}
                    \centering
                    \begin{block}{Markovian models for MARL: Dec-POMDP}
                        {\small
                            Decentralized Partially Observable Markov Decision Process (Dec-POMDP)~\cite{Oliehoek2016}
                            \begin{itemize}
                                \item considers multiple agents in a similar MAS fashion
                                \item stochastic processes for uncertainty in environmental changes including observations;
                                \item reward function is common to agents which fosters training for collaborative oriented actions~\cite{Beynier2013}
                            \end{itemize}
                        }
                        \

                        { \scriptsize

                        $(S,\{A_i\},T,R,\{\Omega_i\},O,\gamma)$ , where
                        \begin{itemize}
                            \item $S = \{s_1, ..s_{|S|}\}$: The set of the possible states;
                            \item $A_{i} = \{a_{1}^{i},..,a_{|A_{i}|}^{i}\}$: The set of the possible actions for agent $i$;
                            \item $T$ so that $T(s,a,s') = \probP{(s'|s,a)}$ : The set of conditional transition probabilities;
                            \item $R: S \times A \times S \rightarrow \mathbb{R}$: The reward function
                            \item $\Omega_{i} = \{o_{1}^{i},..,o_{|\Omega_{i}|}^{i}\}$: The set of observations for agent $ag_i$;
                            \item $O$ so that $O(s',a,o) = \probP{(o|s',a)}$ : The set of conditional observation probabilities;
                            \item $\gamma \in [0,1]$, the discount factor.
                        \end{itemize}

                        }

                    \end{block}

                \end{minipage}
            \end{center}

        \end{column}

    \end{columns}


\end{frame}

\section{CybMASDA}

\begin{frame}{Cyberdefense Multi-Agent Systems Development Approach}{MAS Design context}

    \begin{figure}
        \includegraphics[width=0.7\linewidth]{figures/problem_illustration.png}
    \end{figure}

\end{frame}


\begin{frame}{Cyberdefense Multi-Agent Systems Development Approach}{Problem}

    \begin{alertblock}{Current design limitations}

        Methods require designers' experience but\dots
        \begin{itemize}
            \item \textbf{environment limitations}: complexity, limited access, non \dots
            \item \textbf{designers limitations}: availability, time consuming\dots
        \end{itemize}

        \vspace{-2ex}

        \begin{center}
            \begin{minipage}{11cm}
                \begin{block}{}
                    $\Longrightarrow$ \textbf{Problem}: Increasing design environment knowledge for design $\rightarrow$ \textbf{costly}
                \end{block}
            \end{minipage}
        \end{center}

        \vspace{-2ex}
        \begin{center}
            \begin{minipage}{0.95\linewidth}
                \centering
                \begin{exampleblock}{Autonomous Intelligent Cyberdefense Agents~\cite{Kott2023} (AICA)}

                    \begin{columns}
                        \hspace{5ex}
                        \begin{column}{0.85\textwidth}
                            \textbf{Cyberdefense Multi-Agent System}: malware identification, countermeasures\dots \\
                            $\Longrightarrow$ No visual/intuitive comprehension of complex networked environments
                        \end{column}
                        \begin{column}{0.22\textwidth}
                            \hspace{-2.5ex}
                            \includegraphics[width=0.8\linewidth]{figures/AICA_IWG.jpg}
                        \end{column}
                    \end{columns}

                \end{exampleblock}
            \end{minipage}
        \end{center}

    \end{alertblock}

    \begin{alertblock}{Targeted gaps}
        \begin{enumerate}
            \item[\phantom{X} (G1)] \textbf{Automating the search for suitable agents' policies satisfying design constraints};
            \item[\phantom{X} (G2)] \textbf{Explicating the emerging organizational mechanisms to assist the hand-craft design}.
        \end{enumerate}
    \end{alertblock}

\end{frame}

\begin{frame}{Cyberdefense Multi-Agent Systems Development Approach}{Addressing the gaps}

    \begin{alertblock}{Targeted gaps}
        \begin{enumerate}
            \item[\phantom{X} (G1)] \textbf{Automating the search for suitable agents' policies satisfying design constraints};
                \\ $\Longrightarrow$ Multi-Agent Reinforcement Learning (MARL)?
            \item[\phantom{X} (G2)] \textbf{Explicating the emerging organizational mechanisms to assist the hand-craft design}.
                \\ $\Longrightarrow$ Organizational Model (OM)?
        \end{enumerate}
    \end{alertblock}


    \begin{table}[]

        \centering
        \begin{tabular}{@{}ccc
                >{\columncolor[HTML]{FFFFFF}}c clc@{}}
            \toprule
            \cellcolor[HTML]{FFFFFF}{\color[HTML]{FFFFFF} }                                                                                                                                  &
            \textbf{MARL}                                                                                                                                                                    &
            \textbf{OM}                                                                                                                                                                      &
            \cellcolor[HTML]{FFFFFF}{\color[HTML]{000000} }                                                                                                                                  &
            \textbf{OM + MARL = OMARL}                                                                                                                                                       &
                                                                                                                                                                                             &
            \\ \cmidrule(r){1-3} \cmidrule(lr){5-5}
            \textbf{(G1)}                                                                                                                                                                    &
            \cellcolor[HTML]{FFFFFF}{\color[HTML]{34FF34} \begin{tabular}[c]{@{}c@{}}\small Find suitable\\ \small policies automatically\end{tabular}}                                      &
            \cellcolor[HTML]{FFFFFF}{\color[HTML]{FE0000} \small No automated way}                                                                                                           &
            \cellcolor[HTML]{FFFFFF}{\color[HTML]{000000} }                                                                                                                                  &
            \cellcolor[HTML]{FFFFFF}{\color[HTML]{34FF34} \begin{tabular}[c]{@{}c@{}c@{}}\small Find suitable\\ \small policies automatically\\ \small \phantom{XXX}\end{tabular}}           &
                                                                                                                                                                                             &
            \\

            \textbf{(G2)}                                                                                                                                                                    &
            \cellcolor[HTML]{FFFFFF}{\color[HTML]{FE0000} \begin{tabular}[c]{@{}c@{}}\small No explicit cooperation\\ \small /organization scheme\end{tabular}}                              &
            \cellcolor[HTML]{FFFFFF}{\color[HTML]{34FF34} \begin{tabular}[c]{@{}c@{}}\small Formalize implicit\\ \small organization as\\ \small Organizational Specifications\end{tabular}} &
            \multirow{-3}{*}{\cellcolor[HTML]{FFFFFF}{\color[HTML]{000000} \vspace{4ex}$\Longrightarrow$}}                                                                                   &
            \cellcolor[HTML]{FFFFFF}{\color[HTML]{34FF34} \begin{tabular}[c]{@{}c@{}}\small Formalize implicit\\ \small organization as\\ \small Organizational Specifications\end{tabular}} &
            \multirow{-3}{*}{\vspace{4ex}$\Longrightarrow$}                                                                                                                                  &
            \multirow{-5}{*}{\textbf{ \begin{tabular}[c]{@{}c@{}c@{}} \small Assisted \\ \small Design\\ \small Approach...\end{tabular}}}                                                     \\ \bottomrule
        \end{tabular}
    \end{table}

    % \begin{prosblock}{Contribution: AOMEA}

    %     \textbf{Assisted MAS Organization Engineering Approach (AOMEA)} based upon:
    %     \begin{itemize}
    %         \item \textbf{Multi-Agent Reinforcement Learning (MARL)}: automatically find suitable joint-policies;
    %         \item \textbf{Organizational model (OM)}: formalize an implicit organization as \textbf{Organizational Specifications (OS)};
    %         \item \textbf{Link MARL \& OM}: link explicit OS with \textbf{histories}/\textbf{trajectories} of on-training policies.
    %     \end{itemize}

    %     \

    %     In order to:
    %     \begin{enumerate}
    %         \item \textbf{Constrain MARL}: design constraints to satisfy during training to achieve the goals;
    %         \item \textbf{Generate Organizational specifications}: automatically compute OS from agents' behaviors.

    %               $\rightarrow$ exploitable insights into relevant mechanisms for MAS design.
    %     \end{enumerate}

    % \end{prosblock}


\end{frame}
\begin{frame}{(G1) Multi-Agent Reinforcement Learning}{MARL basics}

    \begin{columns}

        \hspace{-2ex}

        \begin{column}{0.4\textwidth}

            \begin{figure}
                \includegraphics[width=\linewidth]{figures/marl_basics.png}
            \end{figure}

        \end{column}

        \begin{column}{0.7\textwidth}
            \vspace{-2ex}

            \begin{center}
                \begin{minipage}{0.95\linewidth}
                    \centering
                    \begin{block}{Markovian models for MARL: Dec-POMDP}
                        {\small
                            Decentralized Partially Observable Markov Decision Process (Dec-POMDP)~\cite{Oliehoek2016}
                            \begin{itemize}
                                \item considers multiple agents in a similar MAS fashion
                                \item stochastic processes for uncertainty in environmental changes including observations;
                                \item reward function is common to agents which fosters training for collaborative oriented actions~\cite{Beynier2013}
                            \end{itemize}
                        }
                        \

                        { \scriptsize

                        $(S,\{A_i\},T,R,\{\Omega_i\},O,\gamma)$ , where
                        \begin{itemize}
                            \item $S = \{s_1, ..s_{|S|}\}$: The set of the possible states;
                            \item $A_{i} = \{a_{1}^{i},..,a_{|A_{i}|}^{i}\}$: The set of the possible actions for agent $i$;
                            \item $T$ so that $T(s,a,s') = \probP{(s'|s,a)}$ : The set of conditional transition probabilities;
                            \item $R: S \times A \times S \rightarrow \mathbb{R}$: The reward function
                            \item $\Omega_{i} = \{o_{1}^{i},..,o_{|\Omega_{i}|}^{i}\}$: The set of observations for agent $ag_i$;
                            \item $O$ so that $O(s',a,o) = \probP{(o|s',a)}$ : The set of conditional observation probabilities;
                            \item $\gamma \in [0,1]$, the discount factor.
                        \end{itemize}

                        }

                    \end{block}

                \end{minipage}
            \end{center}

        \end{column}

    \end{columns}

\end{frame}

\begin{frame}{(G1) Multi-Agent Reinforcement Learning}{MARL for solving/designing}

    \begin{block}{MARL for methodological purpose in literature?}

        Effective joint-policies but \textbf{not explicitly} specified/understandable
        $\Longrightarrow$ Few related works
            {\small
                \begin{itemize}
                    \item Kazhdan et. al.~\cite{Kazhdan2020} proposed means to extract symbolic models $\rightarrow$ \textbf{not scalable};
                    \item Wang et. al.~\cite{Wang2020}: introduced a role-oriented MARL approach $\rightarrow$ \textbf{roles only};
                    \item Zheng et. al.~\cite{Zheng2018} presented a platform for MARL $\rightarrow$ \textbf{empirical tools}.
                \end{itemize}
            }
    \end{block}

    \begin{block}{Solving a Dec-POMDP}
        \begin{itemize}
            \item \textbf{solving}: finding a joint policy $\pi_{joint,i} \in \Pi_{joint}$ maximizing cumulative reward over time;
            \item \textbf{sub-optimally solving}: finding a joint policy $\pi_{joint,i} \in \Pi_{joint}$ so that expected cumulative reward over time at least at $s \in \mathbb{R}$.
        \end{itemize}
    \end{block}

    \begin{exampleblock}{Examples of MARL Algorithms}
        {\footnotesize

            \centering
            \begin{minipage}{0.5\textwidth}
                \centering
                \begin{itemize}
                    \item \textbf{Independent Learning}: IQL, IDQN
                    \item \textbf{Centralized Training, Decentralized Execution}: MADDPG, COMA, VDN
                \end{itemize}
            \end{minipage}\hfill
            \begin{minipage}{0.5\textwidth}
                \centering
                \begin{itemize}
                    \item \textbf{Cooperative MARL}: QMIX, MAPPO
                    \item \textbf{Hierarchical MARL}: Feudal Networks, Hierarchical Actor-Critic
                \end{itemize}
            \end{minipage}\hfill
        }
    \end{exampleblock}

\end{frame}

\begin{frame}{(G2) Organizational Model}{$\mathcal{M}OISE^+$}

    \begin{figure}
        \centering
        \includegraphics[width=0.75\linewidth]{figures/moise_model.png}
    \end{figure}

    \begin{spacing}{0.25}
        {\tiny Hübner, J. F., Sichman, J. S., and Boissier, O. (2002).
            A model for the structural, functional, and deontic specification of
            organizations in multiagent systems.
            In Bittencourt, G. and Ramalho, G. L., editors, Proceedings of the 16th Brazilian Symposium on Artificial Intelligence (SBIA’02), volume 2507 of LNAI, pages 118–128, Berlin. Springer.}
    \end{spacing}

\end{frame}

\begin{frame}{(G2) Organizational Model}{\textit{Soccer team example}}

    \vspace{-2.5ex}

    \begin{columns}
        \hspace{-16ex}
        \begin{column}{0.5\textwidth}
            \centering
            \begin{figure}[H]
                \includegraphics[width=0.7\textwidth]{figures/soccer_ss.png}
                \caption*{Structural Specifications}
            \end{figure}
        \end{column}
        \hspace{-20ex}
        \begin{column}{0.5\textwidth}
            \centering
            \begin{figure}[H]
                \centering
                \includegraphics[width=1.2\textwidth]{figures/soccer_fs.png}
                \caption*{Functional Specifications}
            \end{figure}
        \end{column}
    \end{columns}

    % \begin{minipage}{0.5\textwidth}
    %     \centering

    % \end{minipage}\hfill
    % %
    % \begin{minipage}{0.5\textwidth}
    %     \centering

    % \end{minipage}\hfill

    \ \\

    \begin{minipage}{\textwidth}
        \centering
        \begin{figure}[H]
            \centering
            \includegraphics[width=0.4\linewidth]{figures/soccer_ds.png}
            \caption*{Deontic Specifications}
        \end{figure}
    \end{minipage}

    % \begin{figure}
    %     \centering
    %     \includegraphics[width=\linewidth]{figures/soccer_os.png}
    % \end{figure}

\end{frame}

\AtBeginSection[]{
    \begin{frame}
        \frametitle{}
        \tableofcontents[currentsection]
    \end{frame}
}

%%%%%%%%%%%%%%%%%%%%%%%%%%%%%%%%%%%%

\section{AOMEA approach}

\subsection{Overview}

\begin{frame}{AOMEA approach}{Overview}

    \begin{figure}[h!]
        \centering
        \includegraphics[width=0.3\linewidth]{figures/AOMEA_illustrative_view}
        \caption{A summary view of our approach to MAS design}
        \label{fig:design_approach}
    \end{figure}

\end{frame}


\begin{frame}{AOMEA approach}{Overview}

    \begin{columns}

        \begin{column}{0.5\textwidth}

            \textbf{Phase 1: Modeling}

            \begin{itemize}
                \item Designers have to manually develop a simulation of the target environment ($1.1$) where agents must cooperate to achieve the designer's goal efficiently ($1.2$);
                \item Designers can link parts of an agent's history with known OS;
                \item Optionally, designers may also want to integrate constraints as well ($1.3$).
            \end{itemize}

        \end{column}

        \begin{column}{0.5\textwidth}
            \centering
            \adjustbox{trim={0.\width} {0.82\height} {0.\width} {0.\height}, clip}{%
                \includegraphics[width=\linewidth]{figures/AOMEA_illustrative_view}
            }
        \end{column}

    \end{columns}

\end{frame}

\begin{frame}{AOMEA approach}{Overview}

    \begin{columns}

        \begin{column}{0.5\textwidth}

            \textbf{Phase 2: Solving}

            \begin{itemize}
                \item a MARL algorithm is used and satisfy constraints on policies regarding known relations between histories and OS;
                \item finds optimal policies satisfying the given design organizational specifications ($2.1$);
                \item gets the associated OS ($2.2$)
            \end{itemize}

        \end{column}

        \begin{column}{0.5\textwidth}
            \centering
            \adjustbox{trim={0.\width} {0.56\height} {0.\width} {0.\height}, clip}{%
                \includegraphics[width=\linewidth]{figures/AOMEA_illustrative_view}
            }
        \end{column}

    \end{columns}


\end{frame}

\begin{frame}{AOMEA approach}{Overview}

    \begin{columns}

        \begin{column}{0.5\textwidth}

            \textbf{Phase 3: Analyzing}

            \begin{itemize}
                \item Designers observe the trained agents' policies ($3.2$);
                \item Designers takes into account the inferred associated OS ($3.1$) to understand they reach the goal;
                \item Designers get some indications of the OS capable of achieving the goal: the curated OS ($3.3$).
            \end{itemize}


        \end{column}

        \begin{column}{0.5\textwidth}
            \centering
            \adjustbox{trim={0.\width} {0.35\height} {0.\width} {0.188\height}, clip}{%
                \includegraphics[width=\linewidth]{figures/AOMEA_illustrative_view}
            }
        \end{column}

    \end{columns}

\end{frame}

\begin{frame}{AOMEA approach}{Overview}

    \begin{columns}

        \begin{column}{0.5\textwidth}

            \textbf{Phase 4: Developing}

            \begin{itemize}
                \item Designers takes into account the curated organizational specifications as a blueprint for implementing a MAS;
                \item Regular MAS development with one of the available methods can be applied hence addressing safety issues;
                \item Implemented agents are launched in simulations for final assessing.
            \end{itemize}

        \end{column}

        \begin{column}{0.5\textwidth}
            \centering
            \adjustbox{trim={0.\width} {0.15\height} {0.\width} {0.57\height}, clip}{%
                \includegraphics[width=\linewidth]{figures/AOMEA_illustrative_view}
            }
        \end{column}

    \end{columns}

\end{frame}


\subsection{Theoretical core}

\begin{frame}[allowframebreaks]{AOMEA approach}{Theoretical core}

    \begin{block}{\emph{Partial Relations with Agent History and Organization Model} algorithm (PRAHOM)}
        Synthesis of two processes that fall into the OMARL purposes
        \begin{enumerate}
            \item \textbf{Inferring Organizational Specifications}: gets the specifications from the agents' policies;
            \item \textbf{Constraining Policies Space}: gets the joint-policies satisfying the given design specifications
        \end{enumerate}
    \end{block}
\end{frame}

\begin{frame}{AOMEA approach}{Theoretical core}

    \textbf{Inferring Organizational Specifications}

    \begin{columns}

        \begin{column}{0.3\textwidth}

            \begin{itemize}
                \item \textbf{Knowledge-based Organizational Specifications Identification (KOSIA)}
                \item \textbf{General Organizational Specifications Infererence (GOSIA)}
            \end{itemize}

        \end{column}

        \begin{column}{0.8\textwidth}
            \begin{figure}
                \centering
                \includegraphics[width=0.95\linewidth]{figures/GOSIA_view.png}
                \caption{A summary view of the GOSIA process}
                \label{fig:gosia_process}
            \end{figure}
        \end{column}

    \end{columns}





\end{frame}

\begin{frame}{AOMEA approach}{Theoretical core}

    \textbf{Constraining Policies Space} during training

    \begin{columns}

        \begin{column}{0.3\textwidth}

            \begin{itemize}
                \item At each step, available actions set is changed to match policy constraints defined by users;
                \item Constraints integrated through: external correction, learning, internal policy change.
            \end{itemize}

        \end{column}

        \begin{column}{0.8\textwidth}
            \begin{figure}
                \centering
                \includegraphics[width=0.7\linewidth]{figures/prahom_view.png}
                \caption{A summary view of the PRAHOM process}
                \label{fig:prahom_process}
            \end{figure}
        \end{column}

    \end{columns}

\end{frame}

\subsection{Engineering tool}

\begin{frame}[fragile]{AOMEA approach}{Engineering tool}

    \begin{block}{\emph{PRAHOM PettingZoo Wrapper}\label{PettingZoo-wrapper}}
        \begin{itemize}
            \item Uses \textbf{PettingZoo}: a library with a standardized API that facilitates the application of MARL algorithms;
            \item Proposed as a tool to help automate the setting up of \emph{PRAHOM} for a given PettingZoo environment
        \end{itemize}
    \end{block}

    \

    \begin{lstlisting}[language=Python, caption=PRAHOM PettingZoo Wrapper basic use, label={lst:wrapper_basic_use}]
    from omarl_experiments import prahom_wrapper
    env=PettingZoo_env.parallel_env(render_mode="human")
    specs_to_hist={"structural_specifications":{"roles":{"follower":{"23":41,"14":[74,0]}}...},"functional_specifications":{"links":{"(leader,follower,aut)":".*14.*?89"}...}...}
    policy_specs_constr={"agent_0":{"structural_specifications":"roles":["follower"]}}
    env=prahom_wrapper(env,action_to_specs,training_specs)
    env.train("default_PPO")
    trained_specs,agent_to_specs=env.prahom_specs()
    \end{lstlisting}

\end{frame}

% \AtBeginSection[]{
    \begin{frame}
        \frametitle{}
        \tableofcontents[currentsection]
    \end{frame}
}

%%%%%%%%%%%%%%%%%%%%%%%%%%%%%%%%%%%%

\section{Evaluation in cooperative game environments}

\begin{frame}{Evaluation in cooperative game environments}


    % Evaluation
    %     In order to verify and demonstrate the approach, is applied on the following case study.
    % 	Case study


    In order to assess AOMEA, we considered using \emph{PRAHOM} in available simulated environments made up of agents that have to achieve a goal with the best performance through various collective strategies whose some can be easily understood (presented in \autoref{fig:simulated_environments}).
    We selected three Atari-like environments for their visual rendering is a convenient way to assess the results with manual observations\footnotemark[1].
    We also considered a Cyberdefense environment as a first attempt to apply \emph{PRAHOM} in a non-visual Cyberdefense environment:

    \footnotetext[1]{Additional explanation and the examples discussed using \emph{PRAHOM PettingZoo wrapper} are available at \url{https://github.com/julien6/omarl_experiments?tab=readme-ov-file\#tutorial-predator-prey-with-communication}}


    \begin{itemize}
        \item \textquote{Drone swarm - 3rd CAGE Challenge}~\cite{cage_challenge_3_announcement} (CYB) consists of cyberdender agents deployed on networked drones fighting against maliciously deployed malware programs. We may expect agents to \allowbreak isolate compromised drones;
        \item \textquote{Pistonball} (PBL)~\cite{Terry2021} consists of a series of pistons to bring a ball from right to left side hence requiring neighbors' representation;
        \item \textquote{Predator-prey with communication}~\cite{Lowe2017} (PPY) consists of predators monitored by a leader to catch faster prey hence requiring hunting strategies;
        \item \textquote{Knights Archers Zombies}~\cite{Terry2021} (KAZ) consists in knights and archers learning how to kill zombies hence requiring efficient agent spatial positioning.
    \end{itemize}
    %
    \begin{figure}[H]
        \centering
        \includegraphics[width=0.8\linewidth]{figures/envs_4x1.png}
        \caption{Overview of the selected environments: CYB, PBL, PPY, and KAZ}
        \label{fig:simulated_environments}
    \end{figure}
    %
    \noindent We applied AOMEA in three cases:
    \begin{itemize}
        \item No organizational specifications (NTS): agents have to learn the most efficient collective strategies without any constraints or indications.
        \item Partially constraining organizational specifications (PTS): some constraints or indications are given to help converge faster or meet requirements.
        \item Fully constraining organizational specifications (FTS): manually crafted joint-policies are given for they are a reference regarding learned joint-policies.
    \end{itemize}

    \noindent Here, we do not present the details of the constraints that were given in NTS and FTS (available in Git repository\footnotemark[1]).
    %
    \begin{figure}[h!]
        \centering
        \includegraphics[width=0.8\textwidth]{figures/prahom_learning_curve.png}
        \caption{Average reward for each iteration in the PBL environment for the NTS, PTS, and FTS cases}
        \label{fig:prahom_learning_curve}
    \end{figure}
    %
    \begin{figure}[h!]
        \centering
        \includegraphics[width=0.8\textwidth]{figures/prahom_pca_analysis.png}
        \caption{PCA of the trained agents' histories in the PBL environment}
        \label{fig:prahom_pca_analysis}
    \end{figure}
    %
    We evaluate the impact of \emph{PRAHOM} on the following criteria: convergence time ratios between PTS, NTS, and FTS for reaching a threshold cumulative reward. Performance stability shows how the trained agents can achieve the goal generally by assessing several environments generated with different parameters. Results are presented in Table~\ref{tab:training_AOMEA_results}.
    %
    \begin{table}[t!]

    \centering

    \begin{tblr}{colspec={llll},rows={m},measure=vbox,stretch=-1}

        \textbf{Environment} & \textbf{Convergence time} & \textbf{} & \textbf{Stability} \\

        \hline

        { 1 }
        & {  }
        & {  } \\
        & {  } \\

        \hline[dashed]

        { 2 }
        & {  }
        & {  } \\
        & {  } \\

        \hline[dashed]

        { 3 }
        & {  }
        & {  } \\
        & {  } \\

        \hline[dashed]

        { 4 }
        & {  }
        & {  } \\
        & {  } \\

        \hline[dashed]

        { 5 }
        & {  }
        & {  } \\
        & {  } \\

    \end{tblr}

    \caption{View of the OOMARL approach impact during training}

    \label{tab:training_OOMARL_results}

\end{table}

    %
    As a general observation, we can notice convergence time is longer for NTS than for PTS which is also longer than for FTS. As expected, the search space is decreasing, hence a shorter convergence time. For instance, we noticed a faster convergence to a sub-optimal solution in the PBL environment by providing organizational specifications as presented in \autoref{fig:prahom_learning_curve}. Although PTS converges faster than NTS to a comparable cumulative reward, NTS may outperform PTS because trained agents' policies are hand-tailored to solve the problem much more finely than the designer's organizational specifications can do. Low-performance stability in the more complex CYB environment indicates that the trained agents have difficulty finding general strategies compared to the agents in the other environments.

    We also took into account the following criteria after training: roles, links, and global performance. A qualitative analysis is presented in Table~\ref{tab:trained_AOMEA_results}
    %
    \begin{table}[t!]

    \centering

    \begin{tblr}{colspec={llll},row{1}={rowsep=1mm,m},row{2-Z}={rowsep=0.5mm,m},measure=vbox,stretch=-10}

        \textbf{ \small Env.} & \textbf{ \small Rôles émerg.} & \textbf{ \small Links émerg.} & \textbf{\small Global \\ perf.} \\

        \hline

        { \small PBL }
        & { \small Claire émerg. \\ des rôles}
        & { \small Représentat° \\ locale }
        & { \small Proche de \\ optimal } \\

        \hline[dashed]

        { \small PPY }
        & { \small Différentat° \\ inhérente }
        & { \small Rares stratégies}
        & { \small Hautem\textsuperscript{t} \\ variable } \\

        \hline[dashed]

        { \small KAZ }
        & { \small Différentat° \\ inhérente }
        & { \small Représentat° \\ locale }
        & { \small Hautem\textsuperscript{t} \\ variable } \\

        \hline[dashed]

        { \small CYB }
        & { \small Emerg. roles \\ peu claire }
        & { \small Qques stratég-\\ies apparentes }
        & { \small Assez \\ bon } \\

    \end{tblr}

    \caption{Analyse qualitative des spécifications organisationnelles déduites après entrainement dans le cas NTS}

    \label{tab:trained_AOMEA_results}

\end{table}

    %
    % //TODO: Moise+ schemes and comparison with expected ones
    %
    For the PBL environment, we can notice roles being equivalent for agents are expected to act the same. Indeed, trained agents' histories are close hence showing a common emerging role. We generate the PCA presented in \autoref{fig:prahom_pca_analysis} by expressing agents' histories as vectors containing the observation-action couples. We can notice most agents’ histories are in the left bottom zone (circled in red). It shows most pistons seem to act similarly as expected. We observe no organizational specifications except roles have been generated because agents cannot communicate. For the KAZ environment, we can notice two distinct roles: archers tend to move away from zombies, while knights tend to approach them. For the PPY environment, we can observe the output specifications indicate authority links between the leader predator and the simple predators to enable collective strategies for circling prey. Finally, the CYB environment shows communications between blue agents are indeed understood as communication links that enable isolating infiltrated drones or trying to fix and alert recently suspected drones.

    For the CYB environment, we developed our custom MAS via a simple hand-crafted decision tree as preconized in AOMEA in light of the organizational specifications we curated by removing noisy results. Our approach did not suggest general roles but relevant strategy patterns have been identified. For instance, regarding links between agents' roles, we noticed that the agents sending messages frequently seem to be spotted as suspected by their neighbors. In addition, a cyber-defender agent in the communication radius of a suspected drone tends to switch off its communication and reactivate afterward. Even though these insights are few, the mean score we got with our curated MAS is about -2000 which is indeed close to the top 5 scores. This shows AOMEA to be indeed applicable to the Cyberdefense context additionally bringing safety guarantees.


\end{frame}

\section{Conclusion and perspectives}
	\begin{frame}{Conclusion and perspectives}
		{}

            \begin{block}{Contributions}
                \begin{itemize}
                    \item A \textbf{Dec-POMDP} modeling of networked node likely to be attacked and defended by agents.
                    \item A first \textbf{work in progress simulator} whose some capabilities have been assessed through a \textbf{MITRE ATT\&CK} scenario.
                \end{itemize}
            \end{block}

            \begin{alertblock}{Perspectives}
                To meet the \textbf{AICA} needs, we identified some limitations to overcome:

                \begin{itemize}
                    \item More \textbf{realistic} scenarios so agents can explore and act as in seemingly similar to reality information systems;
                    \item Ease the \textbf{transfer learning} of trained agent models from simulations to emulated or real systems;
                    \item More \textbf{coordination} between agents, especially leveraging on communication to reach a goal\dots
                \end{itemize}
            
            \end{alertblock}
	
	\end{frame}









\section{MOISE+MARL Framework}

\begin{frame}{The $\mathcal{M}OISE^+$ Framework (1/2)}

  \begin{figure}
    \centering
    \includegraphics[width=0.75\linewidth]{figures/moise_model.png}
  \end{figure}

  \begin{spacing}{0.25}
    {\tiny Hübner, J. F., Sichman, J. S., and Boissier, O. (2002).
      A model for the structural, functional, and deontic specification of
      organizations in multiagent systems.
      In Bittencourt, G. and Ramalho, G. L., editors, Proceedings of the 16th Brazilian Symposium on Artificial Intelligence (SBIA’02), volume 2507 of LNAI, pages 118–128, Berlin. Springer.}
  \end{spacing}

\end{frame}

\begin{frame}{The $\mathcal{M}OISE^+$ Framework (2/2)}


  \vspace{-2.5ex}

  \begin{columns}
    \hspace{-16ex}
    \begin{column}{0.5\textwidth}
      \centering
      \begin{figure}[H]
        \includegraphics[width=0.7\textwidth]{figures/soccer_ss.png}
        \caption*{Structural Specifications}
      \end{figure}
    \end{column}
    \hspace{-20ex}
    \begin{column}{0.5\textwidth}
      \centering
      \begin{figure}[H]
        \centering
        \includegraphics[width=1.2\textwidth]{figures/soccer_fs.png}
        \caption*{Functional Specifications}
      \end{figure}
    \end{column}
  \end{columns}

  \ \\

  \begin{minipage}{\textwidth}
    \centering
    \begin{figure}[H]
      \centering
      \includegraphics[width=0.4\linewidth]{figures/soccer_ds.png}
      \caption*{Deontic Specifications}
    \end{figure}
  \end{minipage}

\end{frame}




% ===============================

\begin{frame}{Approach: The MOISE+MARL Framework}

  \begin{columns}[c] % c = vertically center content
    \begin{column}{0.4\textwidth}
      \begin{itemize}
        \item Combine Dec-POMDP with MOISE+ organizational model.
        \item Agents are assigned roles and missions as constraints.
        \item Use constraint guides to adjust:
              \begin{itemize}
                \item \textbf{Actions} via RoleActionGuides (RAG)
                \item \textbf{Rewards} via RoleRewardGuides and GoalRewardGuides
              \end{itemize}
      \end{itemize}
    \end{column}
    \begin{column}{0.6\textwidth}
      \begin{figure}
        \centering
        \includegraphics[width=1.\linewidth]{figures/mm_simple_representation.png}
      \end{figure}
    \end{column}
  \end{columns}
\end{frame}

\begin{frame}{Approach}{The MOISE+MARL Framework}
  \begin{figure}
    \hspace{-0.76cm}
    \includegraphics[width=1.05\linewidth]{figures/modified_state_value_function.png}
  \end{figure}
\end{frame}

\begin{frame}{Evaluation: The TEMM Method}
  \begin{itemize}
    \item Unsupervised analysis of learned behaviors.
    \item Infers implicit roles and missions from trajectories.
    \item Quantifies \textbf{Organizational Fit} (alignment between behaviors and organizational structure).
    \item Highlights structural and functional consistencies.
  \end{itemize}
\end{frame}

\section{Experimental Setup}

\begin{frame}{Experimental setup}{Environments}

  \vspace{-0cm}

  \begin{columns}[c]

    \hspace{-1cm}

    \begin{column}{0.5\textwidth}
      \begin{itemize}
        \item Tested in 4 environments
        \item Compared Baseline vs. MOISE+MARL:
              \begin{itemize}
                \item Higher convergence and robustness.
                \item Improved explainability through consistent role adherence.
                \item Higher Organizational Fit (up to +89\%).
              \end{itemize}
      \end{itemize}
    \end{column}

    \hspace{-1cm}

    \begin{column}{0.5\textwidth}
      \begin{tabular}{@{}c@{\hspace{1cm}}c@{}}
        \makebox[.48\textwidth][c]{\animategraphics[loop,autoplay,scale=0.15]{8}{figures/wm/frame}{0}{33}}  &
        \makebox[.48\textwidth][c]{\animategraphics[loop,autoplay,scale=0.18]{8}{figures/overcooked/frame}{0}{66}}                  \\
        \small{Warehouse Management}                                                                       & \small{Overcooked-AI} \\
        \makebox[.48\textwidth][c]{\animategraphics[loop,autoplay,scale=0.135]{8}{figures/mpe/frame}{0}{25}} &
        \makebox[.48\textwidth][c]{\animategraphics[loop,autoplay,scale=0.135]{8}{figures/cyborg/frame}{0}{33}}                      \\
        \small{Predator-Prey}                                                                              & \small{CybORG}        \\
      \end{tabular}
    \end{column}
  \end{columns}
\end{frame}

\begin{frame}{Experimental setup}{Environments}
  \begin{tabular}{cc}
    \makebox[.48\textwidth][c]{\animategraphics[loop,autoplay,scale=0.1]{8}{figures/wm/frame}{0}{33}}  &
    \makebox[.48\textwidth][c]{\animategraphics[loop,autoplay,scale=0.1]{8}{figures/overcooked/frame}{0}{66}}                  \\
    \small{Warehouse Management}                                                                       & \small{Overcooked-AI} \\
    \makebox[.48\textwidth][c]{\animategraphics[loop,autoplay,scale=0.1]{8}{figures/mpe/frame}{0}{25}} &
    \makebox[.48\textwidth][c]{\animategraphics[loop,autoplay,scale=0.1]{8}{figures/cyborg/frame}{0}{33}}                      \\
    \small{Predator-Prey}                                                                              & \small{CybORG}        \\
  \end{tabular}
\end{frame}

\begin{frame}{Experimental setup}{Environments}

  \begin{figure}
    \centering
    \makebox[\textwidth] {
      \animategraphics[loop,autoplay,scale=0.25]{8}{figures/cyborg/frame}{0}{33}
    }
  \end{figure}

\end{frame}

\begin{frame}{Experimental setup}{Environments}

  \begin{figure}
    \centering
    \makebox[\textwidth] {
      \animategraphics[loop,autoplay,scale=0.25]{8}{figures/mpe/frame}{0}{25}
    }
  \end{figure}

\end{frame}

\begin{frame}{Experimental setup}{Environments}

  \begin{figure}
    \centering
    \makebox[\textwidth] {
      \animategraphics[loop,autoplay,scale=0.25]{8}{figures/overcooked/frame}{0}{66}
    }
  \end{figure}

\end{frame}

\begin{frame}{Experimental setup}{Environments}

  \begin{figure}
    \centering
    \makebox[\textwidth] {
      \animategraphics[loop,autoplay,scale=0.25]{8}{figures/wm/frame}{0}{33}
    }
  \end{figure}

\end{frame}

\begin{frame}{Experimental setup}{Organizational constraints}

  TODO

\end{frame}


\begin{frame}{Experimental setup}{Evaluation protocol}

  TODO

\end{frame}



\section{Results}

\begin{frame}{Key Results}
  \begin{itemize}
    \item Organizational constraints improve policy stability and interpretability.
    \item Policy-based and actor-critic algorithms benefit most.
    \item TEMM successfully recovers organizational structure from learned behavior.
  \end{itemize}
  \vspace{1em}
  \textbf{MOISE+MARL = better control, more explainable agents.}
\end{frame}

\section{Conclusion}


\begin{frame}{Conclusion and Perspectives}
  \begin{itemize}
    \item MOISE+MARL bridges MARL and organizational modeling.
    \item Future directions:
          \begin{itemize}
            \item Dynamic organizational constraints.
            \item Automatic specification generation (e.g., LLMs).
            \item TEMM optimization for real-time evaluation.
          \end{itemize}
  \end{itemize}
  \vspace{0.5em}
  \centering
\end{frame}


\appendix
%\setbeamertemplate{headline}{}
\setbeamertemplate{mini frames}{}

% \AtBeginSection[]{
% 	\begin{frame}
% 		\frametitle{}
% 		\tableofcontents[currentsection]
% 	\end{frame}
% }

% %%%%%%%%%%%%%%%%%%%%%%%%%%%%%%%%%%%%

\section*{\phantom{Thanks}}

\begin{frame}{}

  \vspace{6ex}

  \centering
  {
    \Huge
    \emph{Thank You}
  }

  \vspace{6ex}

  \begin{columns}

    \hspace{-27ex}

    \begin{column}{0.5\textwidth}
      \raggedleft
      {\Large Demo video $\Longrightarrow$}
    \end{column}

    \hspace{-12ex}

    \begin{column}{0.5\textwidth}
      \includegraphics[width=0.5\linewidth]{figures/demo_qr_code.png}
    \end{column}

  \end{columns}

  \vspace{3ex}

  \centering
  {\Large
    \url{https://t.ly/4JBxr}
  }

\end{frame}

% \AtBeginSection[]{
% 	\begin{frame}
% 		\frametitle{}
% 		\tableofcontents[currentsection]
% 	\end{frame}
% }

% %%%%%%%%%%%%%%%%%%%%%%%%%%%%%%%%%%%%

\section*{\phantom{References}}

\begin{frame}[allowframebreaks]{References}{}

    % \bibliographystyle{plain}
    % \bibliography{local_references}
    \printbibliography

\end{frame}

\begin{frame}{Annexes}
    {Context}

    \begin{block}{Multi-Agent Systems (MAS) paradigm for complex \& distributed problems}
        \begin{itemize}
            \item \textbf{task decomposition}: missions delegated to agents achieved through cooperation~\cite{Raileanu2023};
            \item \textbf{benefits}: handle conflicting goals, parallel computation, system robustness, scalability\dots
        \end{itemize}
    \end{block}

    \begin{block}{\textbf{Organization}: key for MAS designing}
        \begin{itemize}
            \item \textbf{coordination}: how to collaboratively achieve a common goal~\cite{Hubner2007};
            \item \textbf{dynamic \& uncertain environments}: flexible runtime behavior to adapt~\cite{Kathleen2020};
        \end{itemize}
    \end{block}

    \begin{block}{Methods and practice for MAS design}
        \begin{itemize}
            \item \textbf{approach + organizational model}: methods rely on designers' experience to hand-craft agents' \textbf{policies} so resulting MAS achieve goals;
                  %   \begin{itemize}
                  %       \item Examples: \emph{GAIA}~\cite{Wooldridge2000,Cernuzzi2014}, \emph{ADELFE}~\cite{Mefteh2015}, or \emph{DIAMOND}~\cite{Jamont2015}, \emph{KB-ORG}~\cite{Sims2008}
                  %   \end{itemize}
            \item \textbf{simulation to reality}: 1) safe \& efficient MAS design in high fidelity simulated environment; \quad 2) transfer to real environment to perform adequately~\cite{Schon2021}.
        \end{itemize}
        \vspace{1ex}
        \quad $\Longrightarrow$ \textbf{Iterative process proceeding by trial and error}

    \end{block}

\end{frame}

\begin{frame}{Annexes}
    {MAS basics}

    \begin{block}{Keywords}
        \begin{itemize}
            \item \textbf{Agent}: entity immersed in an environment perceiving observation and making decision autonomously to achieve some goals;
            \item \textbf{MAS}: a set of agents collaborating with self/re-organizing mechanisms to achieve their goal;
            \item \textbf{Organization}: the agents' interactions even though it may be implicit;
            \item \textbf{Organizational Model (OM)}: medium to formally describe an explicit/implicit organization;
            \item \textbf{Organizational Specifications (OS)}: components of an OM to characterize an organization
        \end{itemize}
    \end{block}

    \begin{block}{Organizational model: $\mathcal{M}OISE^+$}
        \begin{itemize}
            \item more complex than \emph{Agent Group Roles} (integration of standards);
            \item takes into account the social aspects between agents explicitly;
            \item possible to link agents' policies to organizational specifications.
        \end{itemize}
    \end{block}

\end{frame}

\begin{frame}{Annexes}
    {MARL basics}

    \begin{block}{Keywords}
        \begin{itemize}
            \item \textbf{Policy}: the \textquote{logic} to choose next action according to observation for an agent;
            \item \textbf{History/trajectory}: the tuple of (observation, action) couples over an episode;
            \item \textbf{Joint-policy / Joint-history}: all of the agents' policies / histories as tuples;
            \item \textbf{Reinforcement learning}: an agent updates its policy to maximize a cumulative reward;
            \item \textbf{Multi-Agent Reinforcement Learning (MARL)}: extends to multiple agents that learn while considering the actions of other agents;
        \end{itemize}
    \end{block}

\end{frame}


\end{document}
