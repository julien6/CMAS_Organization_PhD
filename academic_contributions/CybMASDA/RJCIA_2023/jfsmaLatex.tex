% Patron de document pour un article a JFSMA pour LaTeX 2e
% Copyright (c) 2007 Bruno BEAUFILS
% Les quolibets et autres insultes sont à envoyer à bruno.beaufils@univ-lille.fr
% Modifications par Gauthier Picard (gauthier.picard@onera.fr) - 31-mai-07
% Modifications par Pierre Chevaillier (chevaillier@enib.fr) - 20-mar-2007
% Modifications par Yann Krupa (krupa@emse.fr) et Gauthier Picard (gauthier.picard@onera.fr) - 31-mar-2011
% Modifications par Emmanuel Adam (emmanuel.adam@univ-valenciennes.fr) - 31-mar-2011
% Modifications par Gauthier Picard (gauthier.picard@onera.fr) - 16-nov-2014
% Modifications par Gauthier Picard (gauthier.picard@onera.fr) - 29-nov-2017
% Modifications mineures par Maxime Morge (maxime.morge@univ-lille.fr) - 3-nov-2023
% L'option doit être :
% - contribution, pour une contribution scientifique originale
% - dissemination, pour une contribution scientifique déjà publiée mais
% inédite en français et traduite ici
% - sota, pour un état de l'art
% - demonstration, pour un article support à démonstration de logiciels
% - final, pour la version finale
\documentclass[demonstration]{jfsma}

\titre{Recommandations aux auteurs d'articles pour les conférences
  JFSMA éditées chez Cépaduès :\\ démonstration (version \LaTeX)}

\auteur{Bruno Beaufils\up{a}}{bruno.beaufils@univ-lille.fr}
\auteur{Gauthier Picard\up{b}}{gauthier.picard@onera.fr}
\auteur{Pierre Chevaillier\up{c}}{pierre.chevaillier@enib.fr}
\auteur{Yann Le Floc'h\up{c,d}}{ylf@leon.bzh}
%%%Si besoin d'ajouter des auteurs à la ligne :
\auteurSuite{René Mandiau\up{e}}{rene.mandiau@uphf.fr}
\auteurSuite{Emmanuel Adam\up{e}}{emmanuel.adam@uphf.fr}

\institution{\up{a}%
  Univ. Lille, CNRS, Centrale Lille, UMR 9189, CRIStAL, F-59000 Lille, France}
\institution{\up{b}%
  ONERA/DTIS, Université de Toulouse, France}
\institution{\up{c}%
  Lab-STICC, UMR CNRS 6285, ENIB, France}
\institution{\up{d}%
  Pount ar Creac'h, Plonévez du Faou, Penn ar bed}
\institution{\up{e}%
  LAMIH, Université Polytechnique Hauts-De-France, France}

\begin{document}

\maketitle

\begin{resume}
  Ce document est à la fois une recommandation et un modèle d'utilisation. Il
  présente les recommandations de compositions typographiques adressées aux
  auteurs d'articles soumis à JFSMA. Ces recommandations ont comme but de
  proposer une unité de présentation des actes qui seront publiés par l'éditeur
  Cépaduès. Ces recommandations sont suivies de conseils pour les utilisateurs
  de l'outil de composition typographique \LaTeX. Le source de ce document doit
  servir de base de travail aux auteurs utilisant cet outil.

  Le résumé ne doit pas dépasser les 150 mots (un peu moins de 10 cm de haut).
  Il est typographié avec une police Times 12 points italique, ainsi que son titre qui
  est en gras. Il n'y a pas de saut de paragraphe dans le résumé.
\end{resume}

\motscles{Exemple type, Format, Modèle}

\bigskip

\begin{abstract}
  The English version of the abstract has the same goal as the French one. It
  must be short and typeset the same way.


  The content must be similar. For simplicity, it is not the case here.
\end{abstract}
\keywords{Example, Model, Template}

\section{Introduction}

Le contenu de l'article peut être rédigé avec n'importe quel formateur ou
traitement de texte, pourvu qu'il réponde aux critères de présentation donnés
ici. L'objectif visé est de proposer une unité de présentation des actes, et
nous vous invitons à tenter de respecter ce modèle autant que le permet votre
logiciel favori.

Pour tous renseignements complémentaires n'hésitez pas à contacter Maxime Morge,
président du comité de programme, dont l'adresse électronique
est~\texttt{\href{mailto:Maxime.Morge@univ-lille.fr}{Maxime.Morge@univ-lille.fr}}.

\subsection{Style général}

Utilisez un format de papier A4 (21x29.7 cm). Les marges gauche et droite sont
de 2~cm, la marge haute de 2.8~cm et la marge basse de 2.9~cm. Centrez la zone
de texte sur la page. L'ensemble du texte doit tenir dans un rectangle de
17x24~cm.

Votre texte doit être disposé en deux colonnes séparées de 1~cm. Évidemment, le
texte dans les colonnes doit être justifié des deux côtés. Les deux colonnes de
la dernière page doivent, si possible, être de longueur égale. Ne faites figurer
aucun en-tête, pied de page et numérotation. Ces informations seront ajoutées
lors de l'assemblage des articles.

La base du texte est en Times 12~points. Essayez de mettre, si possible, un
espacement (12~points) entre 2 paragraphes et après une section ou
sous-section. Les paragraphes ne sont pas décalés.

Le titre de l'article est en Times 20 points gras en français. Indiquez le nom
des auteurs (prénom et nom complet) et leurs affiliations. Les auteurs sont
présentés sur une seule ligne, en 12 points, et régulièrement espacés. L'adresse
électronique de chaque auteur est placée sous le nom de l'auteur, en 10 points.
Les affiliations sont en taille 10~points, centrées sur la largeur de la page.
Elles sont présentées les unes sous les autres, dans la limite de 2 lignes par
affiliation.

L'article doit débuter par un résumé en français et une suite de mots clés
suivi de leurs traductions rigoureuses en anglais. Le résumé doit occuper au
maximum 10 cm (150 mots) y compris son titre qui est en Times 12~points gras.

Dans les sections, le titre est en 14~points gras. Les titres des
sous-sections sont en 12~points gras. Il est fortement recommandé de ne pas
descendre en dessous de deux niveaux de titre. Les sous-sections sont
numérotées comme suit : 1.1, 1.1.1, etc.

\subsection{Types de contributions}

Le style fourni permet de gérer plusieurs types de soumissions : contribution
scientifique originale, contribution scientifique déjà publiée mais
inédite en français et traduite ici, état de l'art, démonstration.

\textbf{Le type de l'article doit être explicitement inscrit en entête de la
  première page} en gris, taille 20~points avec un police sans empâtement
(Arial, Helvetica ou équivalent). \textbf{Cette marque comme les numéros de
  pages devront être supprimés pour l'édition des actes}. Le présent document
montre un exemple de démonstration.

L'ensemble d'un article de contribution scientifique ou d'état de l'art ne doit
pas dépasser 10 pages, références incluses, 4 pages pour une démonstration.


\subsection{Figures, tables et formules}

Les figures et tables peuvent occuper la largeur d'une colonne ou toute la
largeur du texte (17~cm), selon les besoins. Elles doivent être centrées
horizontalement. Veillez à la bonne visibilité de vos graphiques : taille des
textes, épaisseur et style des traits, etc. L'impression des actes est
partiellement en couleur. Merci de les utiliser avec parcimonie uniquement
dans les figures, quand c'est absolument nécessaire.

La légende d'une figure est en Times 12~points, centrée, au dessous de la
figure. Elle est précédée par le mot «\textsc{Figure}», composé en petites
capitales, suivie du numéro de la figure.

Le titre d'une table est en Times 12~points, centré, au dessus de la table. Il
est précédé par le mot «\textsc{Table}» composé en petites capitales, suivi du
numéro de la table.

Les figures, tables et formules devront être numérotées de manière
indépendante.

\subsection{Travaux antérieurs}

Les références bibliographiques peuvent être citées sous la forme
alpha-historique, comme dans (Nexpert, 1929), ou sous une forme numérique, comme
dans~\cite{foo}, si elles sont numérotées dans la bibliographie. Il est
cependant demandé que tout au long de l'article la même forme de référence soit
utilisée de manière consistante.

La bibliographie doit être placée à la fin de l'article. Dans la
bibliographie, les références sont données dans l'ordre alphabétique.

\section{Le coin \LaTeXe}

\subsection{Source du document}

Pour les auteurs utilisant \LaTeXe\ une classe de document particulière
(\texttt{jfsma.cls}) doit être utilisée. Le fichier source de ce texte
(\texttt{jfsma.tex}) est lui-même une base pour obtenir une sortie conforme avec
\LaTeXe. Ce patron est minimaliste et vous aurez besoin de votre manuel \LaTeXe\
pour insérer équations, images et autres tableaux. Certaines indications sont
cependant données ci après.

La classe de document \texttt{jfsma} est une modification de la classe article.
Elle hérite du comportement des articles \LaTeXe. Les commandes et
environnements suivants ont été ajoutés pour la composition du titre (via la
commande \verb|\maketitle|) ainsi que des résumés :
\begin{itemize}
\item \verb|\titre| est une commande permettant de spécifier le titre de la
  soumission ;
\item \verb|\auteur| est une commande permettant de spécifier le nom (premier
  paramètre) et l'adresse électronique (second paramètre) d'un auteur. Il doit y
  avoir autant d'appel à cette commande que d'auteur, tant que le nombre
  d'auteurs tient sur la ligne. Pour présenter les auteurs sur deux lignes,
  utiliser la commande \verb|\auteurSuite| ;
\item \verb|\auteurSuite| est également une commande permettant de spécifier le
  nom (premier paramètre) et l'adresse électronique (second paramètre) d'un
  auteur. Ces auteurs s'affichent sur la seconde ligne ;
\item \verb|\institution| est une commande permettant de spécifier une
  institution attachée à un ou plusieurs des auteurs ;
\item \verb|resume| est un environnement permettant de typographier le résumé en
  francais de l'article ;
\item \verb|\motscles| est une commande permettant de spécifier la liste des
  mots-clés français de l'article ;
\item \verb|abstract| et \verb|\keywords| sont les environnements et commandes
  version anglaise correspondant à \verb|resume| et \verb|\motscles|.
\end{itemize}
Pour spécifier le type de soumission, les options de classe suivantes sont
disponibles : \texttt{contribution}, \texttt{dissemination}, \texttt{sota},
\texttt{demonstration} et \texttt{final}. Par exemple, pour le présent document,
nous avons utilisé :

\begin{verbatim}
\documentclass[demonstration]{jfsma}
\end{verbatim}

Les \textit{packages} \texttt{inputenc} et \texttt{fontenc} sont préchargés,
vous pouvez donc directement utiliser des caractères accentués dans le source de
votre document.

Le \textit{package} \texttt{hyperref} est préchargé vous pouvez donc directement
utiliser les commandes qu'il offre, notamment \verb|\href{}{}| pour l'édition
correcte d'URL ou d'adresse électronique.

Pour insérer des images le \textit{package} \verb|graphicx| est préchargé, ce
qui vous permet d'insérer des images de la manière classique (sortie en pdf)~:
{\scriptsize%
\begin{verbatim}
\includegraphics[width=\columnwidth]{figures/jfsma.png}
\end{verbatim}
}

Ce qui donne quelque chose ressemblant à la figure~\ref{fig:maFigure}. Les images
vectorielles doivent être privilégiées aux images matricielles (bitmap) car,
contrairement à ces dernières, elles peuvent être agrandies ou rétrécies à
volonté sans perdre de leur qualité.

\newpage

\begin{figure}[hbtp]
  \centering
   \includegraphics[width=\columnwidth]{figures/jfsma.png}
  \caption{Un exemple d'inclusion de figure}
  \label{fig:maFigure}
\end{figure}

Pour les références bibliographiques :
\begin{itemize}
\item si vous utilisez Bib\TeX, il vous est conseillé d'utiliser la commande
  \verb|\bibliographystyle{plain}| ;
\item si vous utilisez BIB\LaTeX, il vous est conseillé d'utiliser l'option
  de classe \texttt{sorting=nyt}.
\end{itemize}

\subsection{Génération d'une version imprimable}

Pour l'édition des actes, il est nécessaire de supprimer l'entête et le pied de
page de type de soumission, en utilisez l'option de classe \texttt{final}.
Vous devez transmettre une archive qui contient :

\begin{enumerate}
\item une version électronique, «~\emph{prête à imprimer}~», de votre document.
  Le seul format accepté est le format \texttt{PDF} ;
\item le source de cette version finale incluant les illustrations.
\end{enumerate}


\textbf{Remerciements.} Merci pour vos œuvres.

\begin{thebibliography}{9}
{\small
\bibitem{bar} U. Nexpert, \emph{Le livre,} Son Editeur, 1929.

\bibitem{foo} I. Troiseu-Pami, Un article intéressant, \emph{Journal de
    Spirou}, Vol. 17, pp. 1-100, 1987
}
\end{thebibliography}

\end{document}

%%% Local Variables: 
%%% mode: latex
%%% TeX-master: "jfsmaLatex"
%%% ispell-local-dictionary: "francais"
%%% TeX-command-extra-options: "-shell-escape"
%%% End: 