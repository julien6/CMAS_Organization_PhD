% \AtBeginSection[]{
%     \begin{frame}
%         \frametitle{}
%         \tableofcontents[currentsection]
%     \end{frame}
% }

%%%%%%%%%%%%%%%%%%%%%%%%%%%%%%%%%%%%

\addtocounter{framenumber}{-1}

\section{Introduction}
\begin{frame}{Introduction}{Contexte de la Conception des SMA}

    \begin{center}
        \begin{minipage}{0.95\linewidth}
            \centering
            \begin{exampleblock}{Agents de Cyberdéfense Autonomes et Intelligents~\parencite{Kott2023} (AICA)}

                \begin{columns}
                    \hspace{5ex}
                    \begin{column}{0.85\textwidth}
                        \textbf{Système Multi-Agent de Cyberdéfense} : identification de logiciels malveillants, contre-mesures\dots \\
                        $\Longrightarrow$ Pas de compréhension visuelle/intuitive des environnements en réseau complexes
                    \end{column}
                    \begin{column}{0.22\textwidth}
                        \hspace{-2.5ex}
                        \includegraphics[width=0.8\linewidth]{figures/AICA_IWG.jpg}
                    \end{column}
                \end{columns}

            \end{exampleblock}
        \end{minipage}
    \end{center}

    \begin{figure}
        \centering
        % \includegraphics[width=0.9\columnwidth]{images/MASCARA_Organization.pdf}
        \includesvg[width=0.7\columnwidth]{figures/MAS_definition_illustration.svg}
        \caption*{Vue schématique d'un SMA de Cyberdéfense en action}
        \label{fig:my_label}
    \end{figure}

\end{frame}

\begin{frame}{Introduction}{Problème}


    \begin{alertblock}{Limites actuelles de conception}

        Les méthodes nécessitent l'expérience des concepteurs mais\dots
        \begin{itemize}
            \item \textbf{limites de l'environnement} : complexité, accès limité, non \dots
            \item \textbf{limites des concepteurs} : disponibilité, coûteux en temps\dots
        \end{itemize}

        \vspace{-2ex}

        \begin{center}
            \begin{minipage}{13.5cm}
                \begin{block}{}
                    $\Longrightarrow$ \textbf{Problème} : Accroître la connaissance de l'environnement pour la conception $\rightarrow$ \textbf{coûteux}
                \end{block}
            \end{minipage}
        \end{center}

    \end{alertblock}

    \begin{exampleblock}{Verrous ciblés}
        \begin{enumerate}
            \item[\phantom{X} (G1)] \textbf{Automatiser la recherche de politiques adaptées sous contraintes de conception};
            \item[\phantom{X} (G2)] \textbf{Expliciter les mécanismes organisationnels émergents pour assister la conception}.
        \end{enumerate}
        
        \

        \centering
        \textquote{Voir la conception d'un SMA comme un problème d'optimisation sous-contraintes}
    \end{exampleblock}

\end{frame}
\begin{frame}{Introduction}{Répondre aux verrous}

    \begin{exampleblock}{Verrous ciblés}
        \begin{enumerate}
            \item[\phantom{X} (G1)] \textbf{Automatiser la recherche de politiques adaptées sous contraintes de conception};
                \\ $\Longrightarrow$ Apprentissage par Renforcement Multi-Agent (MARL)?
            \item[\phantom{X} (G2)] \textbf{Expliciter les mécanismes organisationnels émergents pour assister la conception}.
                \\ $\Longrightarrow$ Modèle Organisationnel (OM)?
        \end{enumerate}
    \end{exampleblock}


    \begin{table}[]

        \centering
        \begin{tabular}{@{}ccc
                >{\columncolor[HTML]{FFFFFF}}c clc@{}}
            \toprule
            \cellcolor[HTML]{FFFFFF}{\color[HTML]{FFFFFF} }                                                                                                                                  &
            \textbf{MARL}                                                                                                                                                                    &
            \textbf{OM}                                                                                                                                                                      &
            \cellcolor[HTML]{FFFFFF}{\color[HTML]{000000} }                                                                                                                                  &
            \textbf{OM + MARL = OMARL}                                                                                                                                                       &
                                                                                                                                                                                             &
            \\ \cmidrule(r){1-3} \cmidrule(lr){5-5}
            \textbf{(G1)}                                                                                                                                                                    &
            \cellcolor[HTML]{FFFFFF}{\color[HTML]{046E11} \begin{tabular}[c]{@{}c@{}}\small Trouver des politiques\\ \small adaptées automatiquement\end{tabular}}                            &
            \cellcolor[HTML]{FFFFFF}{\color[HTML]{FE0000} \begin{tabular}[c]{@{}c@{}}\small Aucune méthode \\ \small automatisée\end{tabular}}                                                                                                 &
            \cellcolor[HTML]{FFFFFF}{\color[HTML]{000000} }                                                                                                                                  &
            \cellcolor[HTML]{FFFFFF}{\color[HTML]{046E11} \begin{tabular}[c]{@{}c@{}c@{}}\small Trouver des politiques\\ \small adaptées automatiquement\\ \small \phantom{XXX}\end{tabular}}&
                                                                                                                                                                                             &
            \\

            \textbf{(G2)}                                                                                                                                                                    &
            \cellcolor[HTML]{FFFFFF}{\color[HTML]{FE0000} \begin{tabular}[c]{@{}c@{}}\small Aucun schéma \\ \small organisation explicite\end{tabular}}                      &
            \cellcolor[HTML]{FFFFFF}{\color[HTML]{046E11} \begin{tabular}[c]{@{}c@{}}\small Formaliser organisation\\ \small implicite en \\ \small Spécs. Org.\end{tabular}} &
            \multirow{-3}{*}{\cellcolor[HTML]{FFFFFF}{\color[HTML]{000000} \vspace{4ex}$\Longrightarrow$}}                                                                                   &
            \cellcolor[HTML]{FFFFFF}{\color[HTML]{046E11} \begin{tabular}[c]{@{}c@{}}\small Formaliser organisation\\ \small implicite en\\ \small Spécs. Org. \end{tabular}} &
            \multirow{-3}{*}{\vspace{4ex}$\Longrightarrow$}                                                                                                                                  &
            \multirow{-5}{*}{\textbf{ \begin{tabular}[c]{@{}c@{}c@{}} \small Approche de \\ \small Conception \\ \small Assistée...\end{tabular}}}                                           \\ \bottomrule
        \end{tabular}
    \end{table}

    % \begin{prosblock}{Contribution : AOMEA}

    %     \textbf{Approche de Conception Assistée pour les SMA (AOMEA)} basée sur :
    %     \begin{itemize}
    %         \item \textbf{Apprentissage par Renforcement Multi-Agent (MARL)} : trouver automatiquement des politiques communes adaptées ;
    %         \item \textbf{Modèle Organisationnel (OM)} : formaliser une organisation implicite en tant que \textbf{Spécifications Organisationnelles (OS)};
    %         \item \textbf{Lien entre MARL \& OM} : lier les OS explicites avec les \textbf{historiques}/\textbf{trajectoires} des politiques en cours d'entraînement.
    %     \end{itemize}

    %     \

    %     Dans le but de :
    %     \begin{enumerate}
    %         \item \textbf{Contraindre le MARL} : concevoir des contraintes à respecter durant l'entraînement pour atteindre les objectifs ;
    %         \item \textbf{Générer des spécifications organisationnelles} : calculer automatiquement les OS à partir des comportements des agents.

    %               $\rightarrow$ fournir des informations exploitables sur les mécanismes pertinents pour la conception de SMA.
    %     \end{enumerate}

    % \end{prosblock}

\end{frame}
