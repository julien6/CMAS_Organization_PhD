\documentclass{article}
\usepackage{amsmath}
\usepackage{amssymb}
\usepackage{graphicx}
\usepackage{hyperref}
\usepackage{algorithm}
\usepackage{algorithmic}
\usepackage{cite}

\title{Incorporating Organizational Structures in Multi-Agent Reinforcement Learning: A Framework for Explainable and Safe Multi-Agent Systems}

\author{Your Name}

\begin{document}

\maketitle

\begin{abstract}
Multi-Agent Reinforcement Learning (MARL) has shown promise in complex decision-making scenarios involving multiple agents. However, ensuring coordination, safety, and explainability of learned policies remains challenging. We propose a novel framework that integrates the MOISE+ organizational model into the MARL process to address these challenges. Our approach, PRAHOMT (Partial Relation between Agents’ History and Organizational Model for Training), constrains agents' learning processes by linking their behaviors to predefined organizational roles and missions. This method not only enhances policy convergence and stability but also provides a structured way to interpret agent behaviors. We validate our approach in a predator-prey scenario, demonstrating improvements in learning efficiency and adherence to safety constraints. Furthermore, we compare PRAHOMT with alternative organizational models and reinforcement learning techniques, providing a comprehensive evaluation of its effectiveness.

\end{abstract}

\section{Introduction}
In the field of Multi-Agent Systems (MAS), the integration of organizational structures with Multi-Agent Reinforcement Learning (MARL) is crucial for achieving coordinated and explainable agent behaviors. Traditional MARL methods often struggle with issues such as policy convergence, safety, and interpretability. To address these challenges, we introduce PRAHOMT, an extension of the PRAHOM framework, that incorporates the MOISE+ organizational model into the MARL training process. Our contributions are threefold:
\begin{itemize}
    \item We develop PRAHOMT, a novel algorithm that integrates organizational roles and missions into the MARL framework, guiding the learning process and constraining policy space.
    \item We demonstrate the effectiveness of PRAHOMT in a simulated predator-prey environment, showing improved convergence and stability compared to standard MARL techniques.
    \item We provide a detailed comparison with other organizational models and alternative reinforcement learning approaches, highlighting the unique advantages of using MOISE+ in MARL.
\end{itemize}

\section{Related Work}
\subsection{Organizational Models in MAS}
Organizational models such as MOISE+ and AGR have been proposed to structure and guide agent interactions in MAS. While MOISE+ provides a robust framework for defining roles, missions, and goals, AGR focuses on agent grouping and relationships. Our choice of MOISE+ is motivated by its expressiveness in defining organizational constraints that can directly influence the MARL process. We compare PRAHOMT with both MOISE+ and AGR, analyzing their respective strengths and limitations in the context of reinforcement learning.

\subsection{Reinforcement Learning Techniques}
Reinforcement learning methods, particularly policy-based approaches, have been widely used in MAS. However, they often face challenges such as sample inefficiency and difficulty in ensuring safe and interpretable behaviors. Alternative techniques, such as Neuro-Evolutionary Direct Policy Search (DPS), Linear Quadratic Regulator (LQR), and Model Predictive Control (MPC), offer potential solutions but lack the direct integration with organizational structures. We argue that our use of MOISE+ with MARL provides a unique advantage in ensuring safety and explainability.

\section{Proposed Framework: PRAHOMT}
\subsection{Algorithm Overview}
PRAHOMT builds on the PRAHOM approach by incorporating organizational constraints defined in the MOISE+ model. Each agent's policy is influenced not only by its interactions with the environment but also by its role and mission within the organization. The learning process is constrained by these organizational specifications, ensuring that the resulting policies adhere to predefined safety and coordination requirements.

\subsection{Agent Roles and Missions}
In PRAHOMT, each agent is assigned a role and a mission based on the MOISE+ model. Roles define the expected behaviors and interactions of the agent, while missions specify the goals it should achieve. These organizational elements are used to guide the agent's policy learning, reducing the exploration space and improving convergence speed.

\subsection{Learning Process}
The learning process in PRAHOMT is structured as follows:
\begin{itemize}
    \item \textbf{Initialization:} Agents are initialized with policies that are compliant with their assigned roles and missions.
    \item \textbf{Policy Update:} During each training episode, agents update their policies based on both environmental feedback and organizational constraints.
    \item \textbf{Convergence Check:} The learning process is monitored for adherence to organizational specifications, with policies adjusted as necessary to ensure compliance.
\end{itemize}

\section{Experimental Evaluation}
\subsection{Predator-Prey Scenario}
We validate PRAHOMT using a predator-prey environment, where predator agents must coordinate to capture prey while adhering to organizational constraints. Our results show that PRAHOMT agents achieve faster convergence and more stable policies compared to baseline MARL agents.

\subsection{Comparison with Other Techniques}
We compare PRAHOMT with standard MARL approaches and other organizational models, such as AGR. Our analysis demonstrates that PRAHOMT provides significant improvements in terms of learning efficiency, policy stability, and adherence to safety constraints.

\section{Discussion}
\subsection{Benefits of Organizational Constraints}
The use of organizational constraints in PRAHOMT enhances the explainability and safety of agent behaviors. By linking policies to predefined roles and missions, we provide a structured way to interpret and verify agent actions.

\subsection{Limitations and Future Work}
While PRAHOMT shows promise in structured environments, its performance in more complex, dynamic scenarios remains to be explored. Future work will focus on extending PRAHOMT to handle more intricate organizational structures and dynamic role allocations.

\section{Conclusion}
We have presented PRAHOMT, a novel framework that integrates the MOISE+ organizational model into the MARL process. Our approach improves policy convergence, stability, and explainability, making it suitable for real-world applications where safety and interpretability are paramount. Through experimental validation and comparative analysis, we have demonstrated the effectiveness of PRAHOMT in guiding agent learning and behavior in a structured and safe manner.

\bibliographystyle{plain}
\bibliography{references}

\end{document}
