\documentclass[11pt]{letter}
\usepackage{geometry}
\geometry{margin=1in}
\usepackage{hyperref}

\signature{Julien Soulé\\Thales LAS / LCIS Lab}
\address{Thales Land and Air Systems\\Valence, France}
\date{\today}

\begin{document}

\begin{letter}{Editorial Board\\Journal of Autonomous Agents and Multi-Agent Systems (JAAMAS)}

\opening{Dear Editors,}

I am pleased to submit our manuscript entitled \textit{“Assisting Multi-Agent System Design with $\mathcal{M}OISE^+$ and MARL: The MAMAD Method”} for your consideration for publication in the \textit{Journal of Autonomous Agents and Multi-Agent Systems (JAAMAS)}.

This article builds upon our previous work presented at AAMAS 2025, where we introduced the MOISE+MARL framework as a novel organizationally-guided approach for enhancing control and explainability in Multi-Agent Reinforcement Learning (MARL). In this extended journal version, we situate this framework within a more comprehensive methodology for designing multi-agent systems called MAMAD—an iterative, semi-automated pipeline structured around four core activities: Modeling, Training, Analyzing, and Transferring.

Our goal is to contribute to the intersection of Agent-Oriented Software Engineering (AOSE) and MARL by proposing a unified lifecycle that connects symbolic organizational specifications with learning-based agent coordination. This method not only increases the automation and structure of MAS development, but also provides explainable, constraint-compliant behaviors in complex environments.

We believe this submission is aligned with JAAMAS, as it addresses both theoretical and practical aspects of multi-agent systems, combining formal organizational modeling with modern learning-based techniques. We hope this work will be of interest to researchers working on MAS design automation, MARL, and explainable AI.

We thank you for your time and consideration.

\closing{Sincerely,}

\end{letter}
\end{document}