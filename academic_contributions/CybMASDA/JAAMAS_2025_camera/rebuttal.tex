\documentclass[11pt]{article}

\usepackage[margin=1in]{geometry}
\usepackage{amsmath,amssymb,amsfonts}%
\usepackage{amsthm}%
\usepackage{mathrsfs}%
% \usepackage[title]{appendix}%
% \usepackage{xcolor}%
% \usepackage{textcomp}%
% \usepackage{manyfoot}%
% \usepackage{booktabs}%
% \usepackage{algorithm}%
% \usepackage{algorithmicx}%
% \usepackage{algpseudocode}%
% \usepackage{listings}%

\usepackage{hyperref}

%%%% For camera-ready, use this
%\documentclass[sigconf]{aamas}

\usepackage{listings}
\usepackage{xcolor}

\usepackage{pgfplots}
\pgfplotsset{compat=1.18}
\usepackage{amsmath}

\definecolor{codegreen}{rgb}{0,0.6,0}
\definecolor{codegray}{rgb}{0.5,0.5,0.5}
\definecolor{codepurple}{rgb}{0.58,0,0.82}
\definecolor{backcolour}{rgb}{0.95,0.95,0.92}
 
\lstdefinestyle{mystyle}{
    backgroundcolor=\color{backcolour},   
    commentstyle=\color{codegreen},
    keywordstyle=\color{magenta},
    numberstyle=\tiny\color{codegray},
    stringstyle=\color{codepurple},
    basicstyle=\footnotesize,
    breakatwhitespace=false,         
    breaklines=true,                 
    captionpos=b,                    
    keepspaces=true,                 
    numbers=left,                    
    numbersep=5pt,                  
    showspaces=false,                
    showstringspaces=false,
    showtabs=false,                  
    tabsize=2
}
 
\lstset{style=mystyle}

% --- Tickz
\usepackage{physics}
\usepackage{tikz}
\usepackage{amsmath}
\usepackage{mathdots}
% \usepackage{yhmath}
\usepackage{cancel}
\usepackage{color}
\usepackage{siunitx}
\usepackage{array}
\usepackage{multirow}
% \usepackage{amssymb}
\usepackage{tabularx}
\usepackage{extarrows}
\usepackage{booktabs}
\usepackage{bookmark}
\usetikzlibrary{fadings}
\usetikzlibrary{patterns}
\usetikzlibrary{shadows.blur}
\usetikzlibrary{shapes}

% ---------

\usepackage{balance} % for balancing columns on the final page
\usepackage{csquotes}
% \usepackage{cite}
\newcommand{\probP}{\text{I\kern-0.15em P}}
\usepackage{etoolbox}
\patchcmd{\thebibliography}{\section*{\refname}}{}{}{}
% \usepackage{amsthm,amssymb,amsfonts}

\usepackage[T1]{fontenc}
\usepackage{graphicx}
\usepackage{color}
% \renewcommand\UrlFont{\color{blue}\rmfamily}

\usepackage[inline, shortlabels]{enumitem}
\usepackage{tabularx}
\usepackage{caption}
\usepackage{listings}
\usepackage{stfloats}
\usepackage{titlesec}
\usepackage{ragged2e}
% \usepackage[hyphens]{url}
\usepackage{float}
\usepackage[english]{babel}
\addto\extrasenglish{  
    \def\figureautorefname{Figure}
    \def\tableautorefname{Table}
    \def\algorithmautorefname{Algorithm}
    \def\sectionautorefname{Section}
    \def\subsectionautorefname{Subsection}
    \def\proofoutlineautorefname{Proof Outline}
}

\usepackage[linesnumbered, ruled, vlined]{algorithm2e}
\SetKwComment{Comment}{$\triangleright$\ }{}
\SetAlgoNlRelativeSize{0}
\SetAlgoNlRelativeSize{-1}

\usepackage{amssymb}
\usepackage{pifont}
\newcommand{\cmark}{\ding{51}}%
\newcommand{\xmark}{\ding{55}}%

% Optional: color macros for highlighting (blue = review comment)
\newcommand{\rev}[1]{%
  \par\noindent%
  {\color{blue}\textbf{#1}}%
}

\newcommand{\res}[1]{%
  \par\noindent%
  {#1}%
}

\title{Response to Reviewers\\
\large Manuscript ID: 49dc0175-aead-455f-be2d-109ccfe512f9\\
\large \textit{"Assisting Multi-Agent System Design with $\mathcal{M}OISE^+$ and MARL: The MAMAD Method"}}

\author{}

\date{}

\begin{document}

\maketitle

We thank the Editor and the Reviewers for their careful assessment of our manuscript. Below we briefly address the editorial points and list the corresponding revisions.
\begin{enumerate}[label=\textbf{\arabic*.}]
    \item \textbf{Accurate results.} Revised Section~6 to correct and clarify experiments, metrics, tables, and figures.
    \item \textbf{Tone and claims.} Softened statements in the Abstract, Introduction, and Conclusion to avoid overgeneralization about MAMAD.
    \item \textbf{Limitations.} Expanded the Conclusion to state bounds of applicability (dependence on organizational abstractions, simulator fidelity, limits of plan extraction, and that MAMAD is not a universal AOSE replacement).
    \item \textbf{Submission.} The updated manuscript and this point-by-point response are included with the revision.
\end{enumerate}

We have revised the manuscript accordingly and provide detailed responses below. Reviewer comments are shown in \textcolor{blue}{blue}, followed by our responses.

\bigskip

\medskip\hrule\medskip
\medskip\hrule\medskip


\section{Response to Reviewer 1}

\vspace{1em}

\rev{1. End of subsection 3.2, the organisational specification is defined. I suggest that the elements of "OS" can be defined.} \\

\res{Thank you for pointing out this lack of clarity. We have revised the end of subsection 3.2 to explicitly define each component of the organizational specification (OS). In the updated version, roles, missions, goals, and norms are now clearly introduced with their semantics in the $\mathcal{M}OISE^+$ framework and how they are used in our method.}

\vspace{1em}

\rev{2. Also in subsection 3.2 it is explained which elements from Moise are "extracted" like goals, missions, etc. Next, it is said that other elements are implicit or captured in other ways. I wonder what happened with the agent's plans. It does seem in the scope of this paper but when consider an agent architecture like BDI (Belief Desire Intention) it would be a great achievement to extract the agent's plan. Something like that could be considered to any extent in this work?} \\

\res{We appreciate this insightful comment. You are correct that extracting or leveraging agent plans (particularly in BDI-like architectures) is a promising extension of this line of work.

    In this paper, our primary focus was to establish the minimal and most general bridge between organizational structures and MARL, and therefore we concentrated on roles, missions (i.e., structured sets of goals), and normative constraints. As a result, explicit plans were not integrated at this stage.

    That said, we fully agree that extending MOISE+MARL to handle plan-level structures is both meaningful and technically ambitious. Our current ongoing research explores two directions:

    \begin{enumerate}
        \item \textbf{Plan-guided learning}: using symbolic plan templates to constrain goal decomposition during MARL training so that sub-goals follow an intended hierarchical structure rather than relying on flat or manually weighted reward specifications.
        \item \textbf{Plan extraction from learned behavior}: after training, we analyze recurrent patterns in joint-observation trajectories to extract candidate "latent plans." Our approach leverages trajectory clustering and temporal alignment tools (e.g., PCA for visualization, Dynamic Time Warping and Smith-Waterman alignment to filter noise and detect ordering constraints). The goal is to infer sub-goal sequences, alternatives, and parallelizable actions based on statistically consistent execution traces.
    \end{enumerate}

    Although this is still exploratory work, it offers a path toward bridging data-driven policies and symbolic plans, with the long-term ambition of enabling automatic synthesis of BDI-style plans from RL behaviors. We now briefly articulate this research perspective in the conclusion of the revised manuscript to show how agent plans could eventually be integrated.}

\vspace{1em}

\rev{3. Fig. 1 shows MOISE+MARL. The text explains some elements that appear in Fig. 1, like "rag", "rrg", etc. But, I recommend a more complete legend can be inserted so it is clear to the reader all elements illustrated in Fig. 1} \\

\res{Thank you for the suggestion. We now include a concise legend directly beneath Fig. 1, explaining elements such as rag, rrg, and gag. This improves readability and avoids forcing readers to infer notation exclusively from the caption text.}

\vspace{1em}

\rev{4. In the caption of Fig. 2, some items are enumerated; it seems we jump from "iii" to "v", without using item "iv".} \\

\res{Thank you for catching this. We corrected the enumeration in the figure caption and ensured consistency with the four activities of the MAMAD method.}

\vspace{1em}

\rev{5. Algorithm 2, lines 1 and 2 mention the "manual formalisation of symbolic requirements". Could you give more details about how this is done? Is the idea to use Moise to formally determine the requirements?} \\

Thank you for raising this point. We expanded our explanation to clarify that "manual formalization" refers to converting informal or textual system requirements into formal symbolic elements and reward structures. Specifically:

\begin{itemize}
    \item \textbf{$\text{manual\_formalize}(\mathcal{G}_{\text{inf}})$} refers to transforming informal end-goals into a reward function that quantifies progress toward system objectives using minimal, interpretable indicators;
          \begin{itemize}
              \item this process is intentionally lightweight and remains empirical, as reward design is a known challenge in RL;
              \item we encourage the use of simple, transparent reward signals to avoid unintentionally biasing agent learning or embedding subjective intermediate objectives.
          \end{itemize}
\end{itemize}
\begin{itemize}
    \item \textbf{$\text{manual\_formalize}(\mathcal{C}_{\text{inf}})$} handles informal design constraints and organizational expectations. It maps loosely expressed requirements into MOISE+MARL symbolic abstractions. Concretely, this means specifying:
          \begin{itemize}
              \item \textbf{role constraints} as rules of the form \emph{(observations -> authorised action set)}, and
              \item \textbf{goal constraints} as rules of the form \emph{(observations -> reward bonus/malus)},
          \end{itemize}
\end{itemize}
which together establish how agents are expected to act in specific local contexts and how role- or goal-compliant behavior is incentivized. We clarified these steps in Algorithm 2 and in the associated text.

\vspace{1em}

\rev{6. Section 5. The authors mention the CybMASDE tool developed for the experiments. Could the authors provide more details in subsection 5.1? Which language was used to implement? How does this tool integrate with other frameworks used, like TensorFlow, PyTorch? Maybe a diagram illustrating how this tool works could be added to the paper.} \\

\res{We appreciate the reviewer’s interest in the implementation. We added additional implementation details in Section 5.1, including language choice, architecture, and integration points with learning libraries. CybMASDE is fully implemented in Python and interfaces directly with Ray/RLLib. It supports both TensorFlow and PyTorch through RLLib backends.

    A sequence diagram has been added to illustrate how users define an organizational specification, configure MARL settings, and run the modeling-training-analysis-transfer (MTA) loop. The tool and full documentation are publicly available:

    \begin{itemize}
        \item Code: \href{https://github.com/julien6/CybMASDE}{https://github.com/julien6/CybMASDE}
        \item Docs: \href{https://julien6.github.io/CybMASDE/}{https://julien6.github.io/CybMASDE/}
    \end{itemize}}

\vspace{1em}

\rev{7. Yet in the same topic. How general is this tool, CybMASDE? Could it be easily employed in other scenarios? And how about with another organisational model?} \\

\res{Thank you for this question. CybMASDE is designed to support new MAS problems with minimal overhead. For a new scenario, users mainly need to:

    \begin{itemize}
        \item provide the observation and action space,
        \item manually formalize informal goals into simple reward functions,
        \item optionally define symbolic constraints (roles/goals) as $\mathcal{M}OISE^+$ elements, and
        \item specify (or let the system tune) hyper-parameters for each activity.
    \end{itemize}
    Once these structured inputs are provided, the pipeline executes without modification.

    Regarding other organization models: although $\mathcal{M}OISE^+$ is the most expressive model we targeted, we intentionally restricted the initial integration to roles and goals, which positions MOISE+MARL as a superset from which subsets (e.g., Agent-Group-Roles) can be instantiated. Indeed, we have already instantiated an AGR+MARL version to study ablation effects in our AAMAS article~\cite{soule2025moisemarl}. Models based on at least roles and goals can therefore be incorporated with limited adaptation. Models built on different abstractions would require a new mapping layer to interface symbolic elements with the MARL framework.

    We clarify these aspects in the revised text.}

\vspace{1em}

\rev{8. Besides, since the authors are using Moise as an organization model. Have you considered the use of the JaCaMo framework? Do you think this could be possible? Notice that considering the AOSE and MAOP (Multi-Agent Oriented Programming) communities a sort of complete unified framework with automation could be a quite useful for MAS developers.} \\

\res{Thank you for raising this point. JaCaMo, integrating Jason, CArtAgO, and $\mathcal{M}OISE^+$, is indeed a mature and influential AOSE/MAOP platform. Our contribution is complementary: rather than supporting manual programming of BDI agents, MAMAD and CybMASDE focus on automating MAS design using RL while preserving symbolic organizational abstractions as constraints and explainability anchors.

    Our pipeline therefore connects directly to MARL environments and learning engines (Ray/RLLib) rather than a BDI interpreter. Nevertheless, we fully agree that interoperability between symbolic agent programming (e.g., Jason plans) and learned policies is a highly relevant direction. We now explicitly mention in the conclusion that exporting learned organizational constraints and behavioral structures into JaCaMo environments is part of our planned future work toward unified MAS engineering workflows.}

\vspace{1em}

\rev{9. Subsection 5.3 briefly mentions some features of the environment scenarios selected, like coordination, decision-making and role-based. But later, when the scenarios are presented, we identify more features like task allocation, for instance. I suggest that in the Introduction, all features presented in the scenarios can be listed for the reader. The list is quite complete, I think it is only missing a negotiation problem to cover all basic MAS problems.} \\

\res{Thank you for the suggestion. We revised the Introduction to list all MAS challenges present in the evaluation environments, including coordination, distributed decision-making, role assignment, task allocation, and cooperative resource management. While negotiation is not explicitly addressed in our scenarios, we acknowledge its importance and now mention negotiation-based settings as a promising direction for extending our evaluation suite.}

\vspace{1em}

\rev{10. Subsection 5.3 presents the scenarios and the organisation specifications. Could the normative aspects of each scenario also be included? Or will the normative elements be extracted from the execution of the system?} \\

\res{Thank you for pointing this out. We now clarify that:

    \begin{itemize}
        \item when organizational constraints are predefined, deontic rules are explicitly encoded in the $\mathcal{M}OISE^+$ specification and enforced during training;
        \item in the emergent-analysis setting, the Auto-TEMM procedure infers roles and associated sets of goals from agent behaviour traces. This allows us to reconstruct \emph{de facto} deontic structures (i.e., couples of the form \emph{(role, mission/goal-set)}) based on the empirical frequency and consistency with which agents exhibit behaviours aligned with particular roles and goals.
    \end{itemize}
    This update clarifies when norms are specified vs inferred.}

\vspace{1em}

\rev{11. Figures 4, 6 and 7 have poor quality; if possible, please improve them.} \\

\res{Thank you. We replaced these screenshots with higher-resolution screenshots as much as possible.}

\vspace{1em}

\rev{12. In case another organisational model (instead of Moise) is used. How difficult would it be to use a different model?} \\

\res{We expanded the discussion to clarify that organizational models whose abstractions (set of organizational specifications) are subsets of $\mathcal{M}OISE^+$ (such as \emph{Agent-Group-Roles} that could be seen as having only roles) are already supported, as demonstrated in our prior work in our AAMAS articles (by proposing \emph{AGR+MARL}). Models that do not rely on roles and/or goals would require revisiting the symbolic-to-MARL mapping layer. This requirement is now clearly stated.}

\vspace{1em}

\rev{13. Please review all references. There are several references that are missing the year of publication.} \\

\res{Thank you for noticing. We reviewed and corrected the bibliography accordingly.}

\vspace{1em}

\rev{14. I recommend that more details be given towards future work. Is there an idea to improve the CybMASDE tool? The automation is a great contribution, but are the authors considering a complete unified framework? One that gives complete support to tool integration through different layers?} \\

\res{Thank you for this valuable suggestion. In the revised manuscript, we expanded the future-work section to highlight research directions aligned with the MAMAD pipeline:

    \begin{itemize}
        \item \textbf{Modeling:} integrating neuro-symbolic representations into multi-agent world models to better encode symbolic constraints in latent dynamics and improve coherence between symbolic and learned models.
        \item \textbf{Training:} exploring formal verification of emergent behaviors using role-based hard constraints, and investigating hierarchical role structures where roles and sub-goals possess weighted priority levels instead of uniform importance.
        \item \textbf{Analysis:} while Auto-TEMM improves automation, generalizing symbolic behavior extraction remains challenging. We plan to investigate using large language models to semantically interpret clusters of transitions and produce interpretable role descriptions and goal structures.
        \item \textbf{Transfer:} maintaining alignment between simulation and real-world environments remains an open challenge. We plan to incorporate feedback-driven triggers to detect simulation-to-reality divergence and re-train both the world model and policies incrementally, minimizing retraining costs and ensuring long-term consistency.
    \end{itemize}
    We believe these directions contribute toward a more unified, end-to-end MAS development pipeline that bridges symbolic AOSE principles and automated data-driven learning.}

\clearpage

\section{Response to Reviewer 2}

\vspace{1em}

\rev{1. The claimed contribution is related to MAS design. The very paper's title states it. However, the paper does not provide a clear definition of what is meant by "designing MAS". Such design may involve different activities, depending on the chosen method. From the description, it seems that the proposed method improves agent behaviour under organizational constraints, but the agents themselves appear to be previously designed (at least partially). This should be clarified.} \\

\res{We thank the reviewer for highlighting this conceptual ambiguity. We now clearly define MAS design in the Introduction as the set of activities required to move from informal requirements to executable multi-agent behaviour, including:

    \begin{itemize}
        \item formalizing organizational abstractions (roles, missions, norms),
        \item generating agent behaviour consistent with these abstractions,
        \item validating emergent behaviour against expectations,
        \item and deploying agents in the target environment.
    \end{itemize}
    We clarify that in our method, \textbf{agents are not manually programmed}; instead, their policies are learned under organizational constraints. We also explicitly state that while the agent architecture (policy function class) is predefined, \textbf{behaviour is synthesized through learning, not hand-coded}, which is the contribution of MAMAD.

    We added an explanation in Section 1.1 and at the start of Section 3.3 introducing the MOISE+MARL framework.}

\vspace{1em}

\rev{2. Section 1.1 (fourth paragraph): the idea of "bridging real-world environments with ML-based approaches" is not clear. Is this about using real-world inputs for MAS design? Or about transferring ML-learned knowledge into real-world settings? Or is it both? After a complete reading, one can answer these questions. However, this should be clear in the introduction to better motivate and ground the assumptions of the work. Furthermore, can such bridging also apply to simulated/virtual environments? If not, why not?} \\

\res{

    We agree this statement needed clarification. We revised Section 1.1 to specify that "bridging" refers to two complementary directions:

    \begin{itemize}
        \item \textbf{Simulation-to-reality (Sim2Real) transfer:} training and validating policies in high-fidelity simulated environments (digital twins) and applying established transfer techniques (such as domain randomization and offline/online fine-tuning on limited real data) to deploy agents' policies safely to the target system.
        \item \textbf{Reality-to-simulation feedback:} continuously updating the digital twin using data from the real system (most likely through agents' observations) so that the digital twin remains representative of current operating conditions.
    \end{itemize}

    This bidirectional connection enables a practical engineering workflow where (i) simulated environments serve as first-class design and experimentation platforms, (ii) transfer is gated by validation metrics and safety checks before deploying policies to agents, and (iii) real-world data is used to refine simulators and policies over time.
}

\vspace{1em}

\rev{3. The role of "specialized knowledge" (Section 1.1, first paragraph) should be explicitly contextualized. Is it domain knowledge, knowledge of AOSE methodologies, or ML expertise?} \\

\res{Since, the concerned paragraph is actually dealing with the AOSE limitaions, we now explicitly state that "specialized knowledge" primarly refers to:

    \begin{itemize}
        \item \textbf{domain expertise} (to express operational constraints),
        \item \textbf{AOSE methodology expertise} (to model organization-level abstractions)
    \end{itemize}

    To a lesser extent, the \textbf{ML engineering expertise} (but not necessarly MARL directly) may also appear as relevant with MARL since it can help in the practical implementation or analysis of agents or at least provides tools to better picture agents' behavior through unsupervised Machine Learning techniques for instance.
    The introduction has been updated to reflect this.}

\vspace{1em}

\rev{4. The authors should clarify the meaning of agent behaviour "controllable and interpretable through AOSE symbolic principles" (Sec 1.2, 1st paragraph). What are these AOSE symbolic principles? How can they be used to control and interpret agents' behaviour? If these symbols are organizational abstractions (roles, missions, etc.), their relation with the agent behaviour should be described earlier to better motivate the work.} \\

\res{We thank the reviewer for this observation. We clarified that AOSE symbolic principles refer to \textbf{$\mathcal{M}OISE^+$ organizational abstractions} (roles, missions, goals, norms) and that these elements:

    \begin{itemize}
        \item constrain which actions an agent is encouraged to perform (hard constraints captured within roles), or reshape their reward structure (soft constraints captured through goal). Norms are to describe which roles are associated with which goals for agents,
        \item serve as interpretable anchors for analyzing emergent behaviour.
    \end{itemize}
    We added a short explanation earlier in the Introduction and referenced Section 3.3 where this is formalized.}

\vspace{1em}

\rev{5. Section 3.2: Shouldn't goals be part of the formal MOISE+MARL organizational specification?} \\

\res{This is a correct observation, goals are indeed part of the organizational specification. We corrected the formulation in Section 3.2 to explicitly include goals in the formal tuple.}

\vspace{1em}

\rev{6. The paper (p. 20) claims that the transferring activity deploys a policy in the environment. However, a policy is u provided to agents (which select actions from observations), not to the environment itself. By definition, the environment contains non-autonomous elements; it does not "hold" or "follow" policies.} \\

\res{We agree. We corrected the wording in Section 4 to state: "deploy the most recent joint policy $\pi^j_{\text{latest}}$ into the agents that are evolving within the real environment $\mathcal{E}$" instead of "\dots to the environment".}

\vspace{1em}

\rev{7. Definition 1: $\pi_i$ is defined as a function mapping observations to actions. Thus, in the first term of the equation, the argument of $\pi_i$ should be an observation from $\Omega$ instead of an action conditioned to an observation. If this is not the case, an explanation is required.} \\

\res{We thank the reviewer for pointing out this misunderstanding. The notation $\pi_i(a_t,\omega_t)$ was intended to denote the probability, under policy $\pi_i$, of selecting action $a_t$ given observation $\omega_t$. We acknowledge that Definition 1 did not make this clear. We have revised the explanatory paragraphs to state explicitly that $\pi_i(a_t\mid\omega_t)$ denotes the conditional probability of action $a_t$ given observation $\omega_t$ under policy $\pi_i$. This follows standard MARL notation, where policies are represented as conditional probability distributions over actions given observations; we now clarify it in the text to avoid confusion.}

\vspace{1em}

\rev{8. Definition 1: the transition function T maps states x actions x states to a probability (cf. Section 3.1). Thus, in the second term of the equation, the argument of T should be such a triplet instead of a conditional probability} \\

\res{As previously analyzed this lack of clarity in notation comes from the habit in standard MARL formalism to switch from relational view to a probabilistic reading. We now explicitly state that the transition function $T(s,a,s')$ can equivalently be read as a probability distribution over next states. In standard MARL notation~\cite{marl-book}, the relational form $T(s,a,s')$ is often used interchangeably with its probabilistic reading, written as the conditional distribution $T(s' \mid a, s) \;=\; \mathbb{P}(s' \mid \langle s, a \rangle)$, which is only a notational shortcut for the same underlying quantity. We have added completed the definition of $T$ to include this notation explicitly in 3.1.}

\vspace{1em}

\rev{9. Definition 1: the meaning of $v_m(t)$ is unclear} \\

\res{We thank the reviewer for pointing this out. The quantity $v_m(t)$ indicates whether the time constraint associated with mission $m$ is still active at time $t$. Concretely, $v_m(t)=1$ while the mission's time-to-live has not expired (so the mission reward contributes to the cumulative reward), and $v_m(t)=0$ once the time constraint is violated (so the mission reward is no longer counted). We added a short explanatory sentence after Definition 1.}

\vspace{1em}

\rev{10. Definition 1: the function rrg declared to map history $\times$ observation $\times$ action to a numeric value, but its arguments do not follow this format.} \\

\res{Thank you for pointing this out. Initially, we approximated $rrg$ using only the current observation and action, although the formulation can take into account the previous history. We have updated Definition 1 and Definition 2 for consistent notation.}

\vspace{1em}

\rev{11. Definition 1: The summation in the first argument presented involve outputs of $\pi$ which are not numeric and therefore cannot be summed.} \\

\res{As stated previously, the notation $\pi_i(a_t\mid\omega_t)$ denotes the conditional probability of action $a_t$ given observation $\omega_t$ under policy $\pi_i$. Therefore, the summation over $\pi_i(a_t\mid\omega_t)$ is valid as it sums over probabilities. Similarly, the summation over $T(s_t,a_t,s_{t+1})$ sums over probabilities of transitioning to next states, which are also numeric values.
The key point is that the state-value function takes all possible actions and next states into account, weighted by their probabilities under the policy and transition function, respectively.
We hope previous changes have clarified that all sums refer to numeric reward terms or probability values. We also rewrote the definition for clarity and correctness.}

\vspace{1em}

\rev{12. Definition 2 presents similar issues: summations applied to outputs of $\pi$ and $T^j$, which are not numeric values.} \\

As stated previously, those notations denote conditional probabilities, making the summations valid. However, we acknowledge that Definition 2 also contained other inaccuracies. We also hope the previous changes have clarified that all sums refer to numeric reward terms or probability values.

\vspace{1em}

\rev{13. Algorithm 1: "need\_update" should be explained.} \\
\rev{14. Algorithm 1: the "transfer function" apparently deploys the joint policy to the agents, but only one of its parameters is explained; both should be.} \\
\rev{15. Algorithm 1: the operation in line 6 of Algorithm 1 is not clear and requires explanation.} \\

\res{We reformatted the description of Algorithm 1 to clarify:

    \begin{itemize}
        \item the purpose of the \textbf{need\_update} flag, which indicates whether the world model requires retraining in 4.5,
        \item the parameters of the \textbf{transfer} process for deploying the joint policy to all agents in the target environment and maintaining the consistency of the world model with the real environment,
        \item and the operation in line 6 (i.e. \textbf{the transfer process}), which we describe as updating the world model using collected experience data to ensure it remains representative of the current environment dynamics. This also enables agents to adapt their policies based on the latest environment conditions.
    \end{itemize}
    These clarifications have been added directly into the algorithm comments and the surrounding text.}

\vspace{1em}

\rev{16. Algorithms 2, 3, 4, and 5 are included but never mentioned in the text. They must be either removed or integrated into the explanations.} \\

\res{We have integrated Algorithms 2, 3, 4, and 5 into the main text where relevant, providing clear references to their functionality and purpose.}

\vspace{1em}

\rev{17. When referring to specific instructions of algorithms, the text should cite algorithm line numbers to improve clarity.} \\

\res{We have revised the whole "MAMAD Method" section of the manuscript to include more detailed instructions from the algorithms.}

\vspace{1em}

\rev{18. Several functions (e.g., $T^j$, $R_{H^j}$) are only informally described and would benefit from precise formalization. This lack of rigour undermines clarity and reproducibility.} \\

\res{We reformatted Section 3. to add more precise definitions and domains for these functions, and included short explanatory notes on their operational interpretation.}

\vspace{1em}

\rev{19. The proposed extension to Multi-Agent World Models needs details. The paper assumes the existence of encoder/decoder functions for high-dimensional joint observations but does not provide details. If this is merely an assumption, it should be stated explicitly. Otherwise, the encoding/decoding process should be detailed (data representation, consistency guarantees, etc.).} \\

\res{We clarified that:

    \begin{itemize}
        \item high-dimensional joint observations are encoded via learned VAE encoders,
        \item latent decoding is used to reconstruct next-state predictions,
        \item and if the simulator already has a ground-truth state function, the encoder/decoder reduces to the identity.
    \end{itemize}
    We reformatted completely the specific Subsection 4.2 as well as synthetic figure describing its architecture.}

\vspace{1em}

\rev{20. The explanation of the connection between MARL and MOISE is presented entirely in the caption of Figure 1, which is inappropriate. A shorter caption and proper textual description are recommended. The same applies to other figures in the paper.} \\

\res{We have shortened captions for Figure 1 and others, and moved explanatory text into the main body.}

\vspace{1em}

\rev{21. Figure 2 requires a legend to make its elements understandable.} \\

\res{We have added a legend to Figure 2 to clarify its elements.}

\vspace{1em}

\rev{22. The acronym JOPT appears on page 12 without definition.} \\
\rev{23. Page 14: $\omega^j_t\in H^j$ should possibly be $\omega^j_t \in \Omega^j$} \\
\rev{24. "an joint-observation" -> "a joint-observation".} \\
\rev{25. "End-to-end automation. improved significantly" -> "End-to-end automation improved significantly".} \\

\res{We applied the following fixes:

    \begin{itemize}
        \item corrected JOPT by JOPM,
        \item corrected $\omega^j_t\in\Omega^j$ in the "Multi-Agent \textit{World Models} for Automatic Generation of the Simulated Model" Subsection,
        \item corrected "a joint-observation",
        \item fixed sentence "End-to-end automation improved significantly".
    \end{itemize}}

\clearpage

\section{Response to Reviewer 3}

\vspace{1em}

\rev{1. I disagree that the proposed approach is an extension of the MOISE+MARL framework as stated in the Introduction ("Extending the MOISE+MARL framework, we adopt an" and "We propose the MAMAD method which extends the MOISE+MARL framework [15] towards MAS design."). The approach includes the MOISE+MARL framework in its activities as mentioned at the end of page 2 "While this represents a significant advancement toward integrating organizational reasoning into MARL, this framework is not yet conceived as part of a comprehensive MAS design methodology." Therefore, the remaining of the section needs to be adjusted to be consistent.} \\

\res{We thank the reviewer for this important correction. We revised the Introduction to clarify that \textbf{MAMAD does not extend MOISE+MARL}, but rather \textbf{integrates and leverages the MOISE+MARL framework} as one component of its training and analysis phases, within a broader MAS design methodology. We updated wording in Section 1 accordingly for consistency.}

\vspace{1em}

\rev{2. The relationship with Digital Twin should be suppressed or better explained. Probably a sentence to satisfy some previous review, but Digital Twin is introduced abruptly without any justification (Section 4.1). In Section 4.2, the concept of digital twin is used in a very broad sense as replicating the real environment ("effectively serving as high-fidelity digital twins of the target environment."). The question here is : What is the difference made between a digital twin and a simulation model?} \\

\res{We agree that our earlier mention of Digital Twins lacked context. We revised Section 4 to clearly state that we refer to \textbf{high-fidelity simulation models} and \textbf{high-fidelity digital twins} interchangeably in this article.}

\vspace{1em}

\rev{3. There is a strong assumption, if understood correctly, that the real-world system should be based on $\mathcal{M}OISE^+$. If this holds, this should be clearly stated in the manuscript.} \\

\res{We appreciate the opportunity to clarify our assumptions. First, we do not consider the environment as including agents. Therefore, we do not require real-world systems to be implemented according to $\mathcal{M}OISE^+$. Rather, we treat a $\mathcal{M}OISE^+$ organisational specification as a compact, symbolic description that can help characterise and constrain agents' expected behaviour. Agents remain autonomous entities interacting with the environment. $\mathcal{M}OISE^+$ is used only as a modelling layer to guide training and to interpret emergent policies. We do not force real systems to follow $\mathcal{M}OISE^+$ (that would be impractical) and we now state this explicitly in Section 3 when introducing MOISE+MARL.}

\vspace{1em}

\rev{4. The Results could be better explored and discussed to strength the manuscript. But the proposed method seems robust and it is well described that hinder this light weakness.} \\

\res{We agree that the discussion can be deepened. We enriched Section 6.2 with:
    \begin{itemize}
        \item interpretation of trends,
        \item explanation of performance differences across tasks,
    \end{itemize}}

\vspace{1em}

\rev{5. [title] (Minor) "Assisting Multi-Agent System Design with $\mathcal{M}OISE^+$ and MARL: The MAMAD Method" -> "Assisting Multi-Agent Systems Design with $\mathcal{M}OISE^+$ and MARL: The MAMAD Method"} \\

\res{Thank you for this suggestion. We have updated the title to "Multi-Agent Systems Design" for grammatical correctness.}

\vspace{1em}

\rev{6. [sect.1] What is an "autonomous logistics"?} \\

\res{We refers to warehouse or factory logistics operations where agents (robots, drones) autonomously perform tasks such as item retrieval, transportation, and inventory management. We revised the sentence to clarify that we refer to "autonomous multi-agent systems for logistics operations".}

\vspace{1em}

\rev{7. [sect.1] "requires designing agents that are both autonomous and coordinated, adaptable and structured." What does it mean a "structured" agent? Is it the agent or the MAS that should be structured?} \\

\res{We removed this term which is not suitable for this context.}

\vspace{1em}

\rev{8. [sect.1] What are "principled methodologies"? Cannot it simply be "methodologies"?} \\

\res{We replaced "principled methodologies" with "methodologies" for simplicity and clarity.}

\vspace{1em}

\rev{9. [sect.1.2] Heading "Problem statement and research gaps" -> "Objectives". The problem statement has already been introduced in the Context section. Maybe the subsections 1.1, 1.2, and 1.3 could be consolidated simply as Introduction.} \\

\res{We acknowledge the problem is already introduced in the Context section and changed the 1.2 title from "Problem statement and research gaps" to "Research gaps" to better reflect its content.}

\vspace{1em}

\rev{10. [sect.1.2] How about replace the term "missions" by "goals". The model is based on $\mathcal{M}OISE^+$ which introduces this concept, but it is not broadly known and the manuscript sometimes refer to goals, sometimes to missions. Be consistent in the use of terms to avoid confusion, even as examples as it is this case. See also sect.1.3 item 1.} \\

\res{We agree that consistent terminology is crucial. We now acknowledge \textbf{goals} could be used by themselves. Yet, we also aim to keep \textbf{missions} specifically to denote sets of goals as defined in $\mathcal{M}OISE^+$. We hope the revised manuscript ensure this distinction clearly and consistently enough.}

\vspace{1em}

\rev{11. [sect.1.3] (If maintained) Heading "Contributions and paper organization" -> "Contributions"} \\

\res{We have updated the heading to "Contributions" as suggested.}

\vspace{1em}

\rev{12. [sect.1.3] "(ii) the global goal, " Of whom? Couldn't it be "global system goals"?} \\

\res{We have updated the text to refer to "global MAS goal" for clarity.}

\vspace{1em}

\rev{13. [sect.1.3, Analysis activity] Where do the "predefined specifications" come from to be compared with the inferred one?} \\

\res{The predefined specifications are derived from the user-defined design requirements established during the Requirements Engineering activity. These specifications may serve as a benchmark for evaluating the inferred organizational specifications. It is now clarified in the revised text.}

\vspace{1em}

\rev{14. [sect.1.3] "validating both (G3) Organizational-level explainability and (G2) Compliance with design requirements." Why G3 before G2?} \\

res{Despite keeping an ordinal numbering is natural, we think the current order better reflects the logical flow of the design process: first ensuring that emergent behaviors can be explained in terms of organizational structures (G3), and then verifying that these structures align with the initial design requirements (G2). We clarified this reasoning in the revised text.}

\vspace{1em}

\rev{15. [sect.2.1] What is a "fully autonomous pipeline"?} \\

\res{We clarified "fully autonomous pipeline" to "fully automated design workflow" to better reflect the intended meaning.}

\vspace{1em}

\rev{16. [sect.2.1] ", offering an almost fully or partly automated pipeline" What is the difference between an almost fully automated pipeline and a partly automated pipeline?} \\

\res{We removed the ambiguous distinction and now refer simply to "automated design workflow" to avoid confusion.}

\vspace{1em}

\rev{17. [sect.2.1] "though it does not incorporate organizational modeling." Is it organizational modeling which is required or some way of constraining the agents to behave as expected? It is clear that the proposed approach uses organizational modeling for it, but couldn't other approaches be proposed to guarantee design requirements?} \\

\res{We corrected it with "though it does not incorporate means to control or guide agent via an organizational model" which better reflects the intent.}

\vspace{1em}

\rev{18. [sect.2.2] Replace "missions" by "goals"} \\

\res{As you suggested we favoured the use of "goal" but also keeping the term "mission": "goals (regrouped through missions)"}

\vspace{1em}

\rev{19. [sect.3.1] Not clear what means "While both formalisms typically rely on access to the true state, limiting their realism, Dec-POMDPs remain suitable for embedding organizational constraints." Maybe "While both formalisms typically rely on access to the true state, Dec-POMDPs remain suitable even limiting the realism by embedding organizational constraints." ???} \\

\res{We revised the sentence to: "While both formalisms typically rely on access to the true state, which may limit their realism in partially observable settings, Dec-POMDPs remain suitable for embedding organizational constraints." This clarifies that Dec-POMDPs can still effectively incorporate organizational constraints despite the challenges posed by partial observability.}

\vspace{1em}

\rev{20. [sect.3.1] Probably missing \{ \} around A\_i and Omega\_i since the Dec-POMDP is defined for the system and not for a specific agent i. Then A used in the next paragraph can be defined as $A = \{Ai\}$} \\

\res{We corrected the notation to use sets $\{A_i\}$ and $\{\Omega_i\}$ for action and observation spaces, respectively. We also defined $A = \langle A_1, A_2, \ldots, A_{|\mathcal{A}|} \rangle$ for the joint action space in the subsequent paragraph for clarity.}

\vspace{1em}

\rev{21. [sect.3.1] $h = ((\omega_k , a_k ))_{k \leq z}$ Why double parentheses.} \\

\res{We corrected the notation to use single parentheses: $h = ((\omega_k, a_k))_{k \leq z}$ for clarity and consistency to express the history of observations and actions.}

\vspace{1em}

\rev{22. [sect.3.2] "missions (goals)" -> "missions (i.e., set of goals)" Otherwise one may interpret that missions are goals and this is misleading.} \\

\res{We revised the phrasing to "missions (i.e., structured sets of goals)" to clarify that missions are collections of goals, avoiding any potential misinterpretation.}

\vspace{1em}

\rev{23. [sect.3.2] What does "mo" refer to in the OS definition?} \\

\res{$mo: \mathcal{M} \rightarrow \mathbb{P}(\mathcal{G})$ is the mapping function that associates each mission $m \in \mathcal{M}$ to a set of goals $\mathcal{G}$. We have added this explanation directly after the OS definition for clarity.}

\vspace{1em}

\rev{24. [sect.3.4] "It allows computing the organizational fit between emergent behaviors and expected roles, goals, and missions." Not clear where does the expected components come from. Are they the organization specification defined by the user?} \\

\res{We clarified that expected roles, goals, and missions are derived from \textbf{(a)} user-defined organizational specifications when provided, or \textbf{(b)} inferred organizational specifications when using Auto-TEMM/TEMM. This distinction is now explicitly stated in Section 3.4.}

\vspace{1em}

\rev{25. [sect4.1] "If the learned policy shows low organizational fit, " with respect to which reference organizational specification?} \\

\res{We thank you for your comments about clarifying organizational fit \& reference specs. We clarified that expected roles/goals come from \textbf{(a)} user OS when specified or \textbf{(b)} emergent OS when using TEMM.}

\vspace{1em}

\rev{26. [sect.4.1, Algorithm 1] Where is launch\_MTA(). I assume it launches the "Process (MTA)" but it is a guess, not clear in the algorithm.} \\

\res{We have reformatted quite extensively the description of Algorithm 1 to explicitly reference \texttt{launch\_MTA()} and clarify its purpose as initiating the Multi-Agent Training Activity (MTA).}

\vspace{1em}

\rev{27. [sect.4.2] "We assume to leave the work of formalizing informal design requirements into MOISE+MARL organizational specifications and informal goal description into an History-based Reward Function." Not clear. What does it mean to leave into an History-based Reward Function?} \\

\res{We have completely reformatted the 4 section also clarifying the informal-to-formal mapping process into MOISE+MARL organizational specifications and history-based reward functions. We now explicitly state that this mapping is assumed to be performed by the user or an external process prior to applying MAMAD.}

\vspace{1em}

\rev{28. [sect.4.3.1] Why to define $\Omega^{(T^j)}_0$ (item iii) since it is not part of the 5 tuple?} \\

\res{We thank you for pointing out this oversight. This we have added this element in the definition of the ODec-POMDP since it is required to start an episode.}

\vspace{1em}

\rev{29. [sect.4.3.1] The text and the Definition 2 does not seem to be aligned.} \\

\res{We have revised Definition 2 and the surrounding text to ensure consistency and alignment between the formal definition and its explanation.}

\vspace{1em}

\rev{"30. Upon observing $\tilde{\omega}^j_t$, each agent $i \in \mathcal{A}$ select an action from $A_{i,t}$ with $A^j_t = \langle A_{0,t}, A_{1,t}, A_{|\mathcal{A}|,t}\rangle$ (from Role Reward Guides) with probability $ch_t$, or from $A^t$ otherwise.". Shouldn't it be "$A^j$" instead of "$A^t$"? Besides shouldn't it be "$A^t$ ... with probability $ch_t$, or from $A^j_t$ otherwise"?}

\res{We corrected the notation to "$A^j_t$" instead of "$A^t$" and clarified the action selection process to ensure it accurately reflects the intended meaning.}

\vspace{1em}

\rev{31. [sect.6] Please compare to the Definition 2:
    (i) Table 1 highlight the best results in each metric;
    (ii) "The largest gain is observed in highly cooperative tasks like Overcooked-AI." Why? A justification should be provided if you include such statement.} \\

\res{We have revised the results discussion in Section 6 to include justification for the observed gains in cooperative tasks, attributing it to the effectiveness of organizational constraints in guiding agent coordination.}

\vspace{1em}

\rev{33. [abstract] (Minor) Since the extended acronym MAS (Multi-Agent Systems) is provided in the main text, it should be provided in the Abstract too.} \\

\res{We have updated the abstract to include the full form of MAS (Multi-Agent Systems) upon its first mention for clarity.}

\vspace{1em}

\rev{34. [abstract] Not clear how could you know before hand what are the "expected performance".} \\

\res{We actually refer to "expected performance" as the performance targets or benchmarks defined by the user based on design requirements or prior knowledge of the task.}

\vspace{1em}

\rev{35. [sec.1] "these AOSE works still faces major" -> "these AOSE works still face major"} \\

\res{We have corrected the grammatical error to "these AOSE works still face major".}

\vspace{1em}

\rev{36. [sect.1] "explicit modeling with possibly explicit roles, " -> "modeling representation, such as, roles, "} \\

\res{We have revised the sentence to "modeling representation, such as, roles," for clarity and conciseness.}

\vspace{1em}

\rev{37. [sect.1] "However, ML-based approaches" -> "ML-based approaches"} \\

\res{We have removed "However," to streamline the sentence to "ML-based approaches".}

\vspace{1em}

\rev{38. [sect.1.2] "Then, the core" -> "The core"} \\

\res{We have revised the sentence to "The core" for conciseness.}

\vspace{1em}

\rev{39. [sect.1.2] Why double semi-colons ; ; } \\

\res{We have removed the double semicolons and ensured that all list items end with a single semicolon for consistency.}

\vspace{1em}

\rev{40. [sect.1.2] Is it really required to use the acronym Gx to each research gap since they are used only in the Introduction and headings in Section 2?} \\

\res{We have decided to keep the Gx acronyms for clarity and consistency throughout the manuscript, even if they are only used in specific sections.}

\vspace{1em}

\rev{41. [sect.1.2] "scenarios.; " -> "scenarios; "} \\

\res{We have corrected the punctuation to "scenarios;".}

\vspace{1em}

\rev{42. [sect.1.2] "process.; " -> "process; "} \\

\res{We have corrected the punctuation to "process;".}

\vspace{1em}

\rev{43. [sect.1.3] "automation. improved significantly, " -> "automation improved significantly, "} \\

\res{We have corrected the sentence to "automation improved significantly," for grammatical accuracy.}

\vspace{1em}

\rev{44. [sect.1.3] "formulism" -> "formalism"} \\

\res{We have corrected the typographical error to "formalism".}

\vspace{1em}

\rev{45. [sect.1.3] Uses interchangeably "pipeline" and "workflow". Pick a term and uses it through the complete manuscript.} \\

\res{We have standardized the terminology throughout the manuscript, consistently using "workflow" to refer to the design process.}

\vspace{1em}

\rev{46. [sect.2.1] "For example, INGENIAS" -> "INGENIAS"} \\

\res{We have removed "For example," to streamline the sentence to "INGENIAS".}

\vspace{1em}

\rev{47. [sect.2.1] Provide the extended acronym for RL the first time it appears in the text.} \\

\res{We have added the full form of RL (Reinforcement Learning) upon its first mention in the text for clarity.}

\vspace{1em}

\rev{48. [sect.2.3] Provide the extended acronym for RNNs the first time it appears in the text.} \\

\res{We have added the full form of RNNs (Recurrent Neural Networks) upon its first mention in the text for clarity.}

\vspace{1em}

\rev{49. [sect.2.3] "A few works" -> "Few works"} \\

\res{We have revised the sentence to "Few works" for conciseness.}

\vspace{1em}

\rev{50. [sect.3.1] Double semi-colons (; ; ) at the end of each item. Why? Found in several list of items.} \\

\res{We have removed the double semicolons and ensured that all list items end with a single semicolon for consistency. They are due to Latex formatting issues.}

\vspace{1em}

\rev{51. [sect.3.2] "permitted or required" -> "permitted or obliged"} \\

\res{We have replaced "required" with "obliged" for clarity and precision.}

\vspace{1em}

\rev{52. [sect.3.3, Figure 1] "contribution ar the" -> "contribution are the"} \\

\res{We have corrected the typographical error to "contribution are the".}

\vspace{1em}

\rev{53. [sect.3.3, Figure 1] ", which are three new" -> "the three new"} \\

\res{We have revised the sentence to "the three new" for clarity.}

\vspace{1em}

\rev{54. [sect.3.3, Figure 1] "an observation received by the agent $\omega \in \Omega$, " -> "an observation $\omega \in \Omega$ received by the agent, "} \\

\res{We have revised the sentence to "an observation $\omega \in \Omega$ received by the agent," for improved readability.}

\vspace{1em}

\rev{55. [sect.3.3, Figure 1] "associates expected actions $A \in \mathcal{P}(A)$ each associated with" -> "associates each expected actions in $A \in \mathcal{P}(A)$ with"} \\

\res{We have revised the sentence to "associates each expected action in $A \in \mathcal{P}(A)$ with" for clarity.}

\vspace{1em}

\rev{56. [sect.3.3] Why is it required "Resolving the Dec-POMDP with MOISE+MARL" heading? Besides, the figure should be better located to make sense since it has a long caption. It maybe better to place the caption text in the section rather than in the figure caption.} \\

\res{We have integrated the figure caption text into the main body of the section for better flow and readability. The figure has also been repositioned to enhance its relevance to the surrounding text.}

\vspace{1em}

\rev{57. [sect.3.5] Why is Learning World Models the last section and not the first since it is the first step in the MAMAD methodology?} \\

\res{We have chosen this order to present the MARL-symbolic logic (the Dec-POMDP formalism, the $\mathcal{M}OISE^+$, MOISE+MARL) that introduce notation we conveniently use in the world models subsection.}

\vspace{1em}

\rev{58. [sect.3.5] "implmented" -> "implemented"} \\

\res{We have corrected the typographical error to "implemented".}

\vspace{1em}

\rev{59. [sect.3.5] Provide the extended acronym for MLP the first time it appears in the text.} \\

\res{We have added the full form of MLP (Multi-Layer Perceptron) upon its first mention in the text for clarity.}

\vspace{1em}

\rev{60. [sect.4.1] "even though it often determined" -> "even though it is often determined"} \\

\res{This sentence has been revised.}

\vspace{1em}

\rev{61. [sect.4.1] "empirically As described" -> "empirically as described"} \\

\res{This sentence has been revised.}

\vspace{1em}

\rev{62. [sect.4.1, Figure 2] Caption missing full-stop (.)} \\

\res{This caption now has a full-stop.}

\vspace{1em}

\rev{63. [sect.4.1] Provide the extended acronym for JOPM the first time it appears in the text.} \\

\res{We have added the full form of JOPM (Joint Organizational Policy Model) upon its first mention in the text for clarity.}

\vspace{1em}

\rev{64. [sect.4.2] "explicitely" -> "explicitly"} \\

\res{We have corrected the typographical error to "explicitly".}

\vspace{1em}

\rev{65. [sect.4.2] "using an joint-observation" -> "using a joint-observation"} \\

\res{We have corrected the article to "a joint-observation".}

\vspace{1em}

\rev{66. [sect.4.3.1] All items end with ".; " and they should end simply with "; "} \\

\res{We have removed the period before the semicolon in all list items to ensure they end with a single semicolon for consistency.}

\vspace{1em}

\rev{67. [sect.4.5] "an Update Trigger" -> "a Update Trigger"} \\

\res{We have corrected the article to "a Update Trigger".}

\vspace{1em}

\rev{68. [sec.5.1] "VAE encoders" Extended version VAE} \\

\res{We have added the full form of VAE (Variational Autoencoder) upon its first mention in the text for clarity.}

\vspace{1em}

\rev{69. [sec.5.1] "MMA API" Extended version MMA API. What is it?} \\

\res{We have added the full form of MMA API (MOISE+MARL API) upon its first mention in the text for clarity. It is the implementation of the MOISE+MARL framework used in this work.}

\vspace{1em}

\rev{70. [sec.5.1] "PPO clip" Extended version PPO} \\

\res{We have added the full form of PPO (Proximal Policy Optimization) upon its first mention in the text for clarity.}

\vspace{1em}

\rev{71. [sec.5.3] "drom warm network" -> Maybe "drone swarm network"???} \\

\res{We have corrected the typographical error to "drone swarm network".}

\clearpage

\section{Response to Reviewer 4}

\vspace{1em}

\rev{1. However, MAMAD is still far from being a general-purpose development methodology, and therefore many of the statements in Subsection 1.1 remain unsubstantiated. In other words, I suggest that the authors revise the claims in that subsection and focus more on the specific contributions of the paper.} \\

\res{We agree that the initial wording in Introduction could be interpreted as claiming broader generality than currently demonstrated. We revised the text to reduce general-purpose claims and also better emphasize that MAMAD contributes to bridging AOSE and MARL automation rather than replacing generic AOSE methods.}

\vspace{1em}

\rev{2. Moreover, I do not agree with the assertion that AOSE faces a major limitation in achieving a fully end-to-end automated design process. In my view, the main limitation of AOSE lies in the gap between the conceptual model and the design model, which stems from the fragmentation of languages and conceptual proposals related to MAS development. An end-to-end automated development process may come later and, in any case, is not always necessarily desirable. Nonetheless, I appreciate the effort made in the paper to move in this direction.} \\

\res{We thank the reviewer for this insightful perspective. We clarified that:

    \begin{itemize}
        \item AOSE faces multiple challenges, including language fragmentation and the conceptual-to-design gap,
        \item end-to-end automation is \textbf{not positioned as a universal requirement}, but as an important research direction for data-driven MAS engineering,
        \item our contribution explores this direction for MAS adopting organizational abstractions.
    \end{itemize}
    This nuance is now included in Subsection 1.1.}

\vspace{1em}

\rev{3. The MAMAD Method, however, does not seem fully aligned with the introduction of the paper. The combination of MOISE and MARL may be suitable for specific development settings, but it does not constitute a general-purpose approach to MAS. Subsection 1.3 clearly states that the key activities are modeling, training, analysis, and transfer. Not all of these are required in a generic MAS development process unless MOISE+MARL were to be adopted as a standard framework.} \\

\res{We agree that the introduction should make clear that MAMAD is not intended as a universal MAS engineering methodology. We revised Subsection 1.3 to clarify that the four activities (modelling, training, analysis, and transfer) are a practical decomposition of the design workflow used in this work, offered as one possible, non-prescriptive approach rather than a mandatory template for all MAS development.}

\vspace{1em}

\rev{4. The paper is difficult to follow. The theoretical background is not presented in an intuitive way, and the mathematical representations are hard to read. I suggest introducing the proposal in a more informal and intuitive manner before presenting the formal mathematical definitions.} \\

\res{Thank you for this suggestion. We tried to better introduce our proposoals especially in Section 4 when introducing the MAMAD method by adding more intuitive explanation before formal definitions.}

\vspace{1em}

\rev{5. The writing style is also somewhat difficult to follow. Figure captions are excessively long, and it is sometimes unclear where the main text resumes (for example, in Subsection 3.3, where the caption for Figure 1 extends across two pages).} \\

\res{We moved excessively long figure captions into the main text to improve readability.}

\vspace{1em}

\rev{6. There are also too many listings, some of which do not significantly aid understanding.} \\

\res{We operated a thorough revision of the section 4 that introduces the MAMAD method by giving more intuitive explanations walking through the algorithms.}

\vspace{1em}

\rev{7. Additionally, the items in these listings often end with "; ; " instead of "; "; why?} \\

\res{We corrected all list items to end with a single semicolon for consistency. The double semicolons were due to Latex formatting issues.}

\vspace{1em}

\rev{8. The formulas are frequently presented in multiple colors (black, red, and blue); is there a specific reason for this? The use of multiple colors makes the formulas harder to read and can confuse the reader.} \\

\res{We apologize for the confusion caused by the color-coded formulas. The colors were initially intended to highlight different components, but we understand that this may have hindered readability. We have revised the manuscript to better state that red color is for roles, while blue color is for goals and missions in Section 3.3.}

\vspace{1em}

\rev{9. Finally, the frequent use of abbreviations and excessive boldface also reduce readability.} \\

\res{We thank the reviewer for pointing out these stylistic issues. We updated the manuscript to reduce boldface to essential technical terms only, and expand abbreviations at first occurrence and avoid unnecessary acronym use.}

\vspace{1em}

\rev{10. Examples are missing, and they would greatly improve the presentation.} \\

\res{We agree that examples help readability. We added a brief illustrative example in Subsection 3.3 (introducing MOISE+MARL) to describe how roles and goals are concretely defined. Additionally, in Section 6.3, we included examples of figures generated by TEMM/Auto-TEMM to illustrate inferred organizational structures in the Overcooked-AI environment.}

\vspace{1em}

\rev{11. The evaluation section includes a sequence of interesting problems, but there is no opportunity to examine the actual results of the development process.} \\

\res{Thank you for this remark. We acknowledge that the manuscript presents organisational specifications at an informal level and focuses on illustrating the types of rules used to define roles and goals rather than giving every formal implementation detail. For readers interested in the concrete specifications and implementations, please see our code repository and project wiki: \href{https://github.com/julien6/CybMASDE}{https://github.com/julien6/CybMASDE} and \href{https://julien6.github.io/CybMASDE/}{https://julien6.github.io/CybMASDE/}.}


\clearpage

\section*{References}

\bibliographystyle{unsrt}
\bibliography{references}

\end{document}

\end{document}
