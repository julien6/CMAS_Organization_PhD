\documentclass[11pt]{article}

\usepackage{hyperref}
\usepackage[T1]{fontenc}
\usepackage{graphicx}
\usepackage{color}
% \renewcommand\UrlFont{\color{blue}\rmfamily}

\usepackage{tabularx}
\usepackage{caption}
\usepackage{listings}
\usepackage{stfloats}
\usepackage{titlesec}
\usepackage{ragged2e}
% \usepackage[hyphens]{url}
\usepackage{float}
\usepackage[english]{babel}
\addto\extrasenglish{  
    \def\figureautorefname{Figure}
    \def\tableautorefname{Table}
    \def\algorithmautorefname{Algorithm}
    \def\sectionautorefname{Section}
    \def\subsectionautorefname{Subsection}
    \def\proofoutlineautorefname{Proof Outline}
}

\usepackage[utf8]{inputenc}
\usepackage{enumitem}
\usepackage{hyperref}
\usepackage{natbib}
\usepackage[a4paper, margin=2.5cm]{geometry}

\title{Information Sheet -- JAAMAS Submission}
\date{}
\begin{document}
\maketitle

\section*{1. Main claim and contribution}

This paper presents \textbf{MAMAD} (MOISE+MARL Assisted MAS Design), a novel method for automating the design of Multi-Agent Systems (MAS) by leveraging Multi-Agent Reinforcement Learning (MARL) within the Agent-Oriented Software Engineering (AOSE). The key contribution is to formalize MAS design as an iterative optimization problem under organizational constraints, combining learning-based methods with symbolic specifications. This approach not only accelerates design but also enhances compliance and interpretability at the organizational level. By integrating the $\mathcal{M}OISE^+$ organizational model into each step of the MAS design lifecycle, MAMAD enables structured guidance during training, automatic inference of roles and goals, and progressive refinement of organizational policies. This work contributes to addressing four long-standing gaps in the literature: (i) lack of structured AOSE-MARL integration, (ii) difficulty in interpreting emergent behaviors, (iii) limited mechanisms for guiding learning, and (iv) absence of end-to-end automation in MAS design.

\section*{2. Supporting evidence}

The claim is supported through a fully implemented pipeline and comprehensive empirical validation across four benchmark environments (Overcooked-AI, Predator-Prey, Warehouse Management, and Cyber-Defense). We compare MAMAD to classical baselines with and without manually specified organizational constraints. Quantitative results show that MAMAD improves (i) automation (lower design time and knowledge input), (ii) efficiency (comparable or higher cumulative rewards, faster convergence, greater robustness), (iii) compliance (lower constraint violations and alignment with predefined specifications), and (iv) explainability (clearer role/goal structures and higher organizational fit). Ablation studies further demonstrate the contribution of each component. The method also shows generalization and interpretability with minimal prior expert knowledge.

\section*{3. Related work}

In the context of explainability, some methods promote transparency by requiring agents to predict human-interpretable concepts before acting~\citep{zabounidis2023concept}, while others combine recurrent networks and decision trees to yield interpretable policies~\citep{liu2025}. Post-hoc techniques such as relevance backpropagation, activation patching~\citep{poupart2025perspectives}, and Shapley-value-based policy transformation~\citep{li2025from} extract explanations from trained models. However, these approaches focus on individual decisions, offering limited insight into collective or organizational dynamics.

Other efforts aim to infer roles or goals from behavioral traces. Some extract latent roles from interaction patterns~\citep{serrino2019finding}, though often without formal alignment with symbolic organizational models. Others enhance coordination via dependency modeling~\citep{berenji2000learning} or Bayesian reasoning~\citep{yusuf2020inferential}, yet without producing reusable, structured specifications.

The AOSE community has proposed frameworks that explicitly define agent roles, goals, and interactions~\citep{gaia1998,Castro2002,Pavon2003}. While these offer explainability and control by design, they lack integration with learning-based mechanisms and support for simulation-based refinement.
%
In contrast, the approach presented here bridges learning and symbolic modeling by enabling organizational structures to be inferred and refined through interaction in simulated environments.


\section*{4. Relation to prior publications}

This paper extends the work presented at AAMAS 2025 in the paper \textit{“An Organizationally-Oriented Approach to Enhancing Explainability and Control in Multi-Agent Reinforcement Learning”}~\citep{soule2025moisemarl}. The AAMAS version introduced MOISE+MARL, a framework for embedding organizational constraints into MARL and analyzing emergent behaviors through role and goal inference. The present journal submission significantly expands upon this by:

\begin{itemize}[noitemsep, topsep=2pt]
    \item Wrapping MOISE+MARL into a complete MAS design method (MAMAD) with a formalized iterative design loop.
    \item Adding two new activities: Modeling and Transferring, to form a full pipeline from raw data to deployment.
    \item Providing a formal pseudo-code for the entire method and for each internal operation.
    \item Introducing a comprehensive evaluation protocol, new metrics, and a experimental baselines (including ablation-style comparisons).
    \item Demonstrating the implementation of MAMAD in a complete software tool (CybMASDE).
\end{itemize}

\noindent We aim this contribution to go beyond the original AAMAS conference paper.

\setlength{\bibsep}{4pt}
\bibliographystyle{abbrv}
\bibliography{references}

\end{document}
