\documentclass[9pt, aspectratio=169]{beamer}
% \documentclass[10pt]{beamer}
\usepackage[utf8]{inputenc}
\usepackage[T1]{fontenc}
\usepackage[english]{babel}
\usetheme{Frankfurt}

\usepackage[backend=biber, style=authoryear]{biblatex}
\addbibresource{local_references.bib}

%\usepackage{lmodern}
\usepackage{amsfonts,amssymb,amsmath}
\usepackage[english]{babel}
\usetheme{Frankfurt}

\usepackage{csquotes}
\usepackage{setspace}

\usepackage{colortbl}
\usepackage{tabularx}
\renewcommand\tabularxcolumn[1]{m{#1}}

% --- Tickz
\usepackage{physics}
\usepackage{amsmath}
\usepackage{tikz}
\usepackage{mathdots}
\usepackage{yhmath}
\usepackage{cancel}
\usepackage{color}
\usepackage{siunitx}
\usepackage{array}
\usepackage{multirow}
\usepackage{amssymb}
\usepackage{gensymb}
\usepackage{tabularx}
\usepackage{extarrows}
\usepackage{booktabs}
\usetikzlibrary{fadings}
\usetikzlibrary{patterns}
\usetikzlibrary{shadows.blur}
\usetikzlibrary{shapes}

% ---------

\usepackage{booktabs}
\usepackage{setspace}
\usepackage{amssymb}
\usepackage{adjustbox}
\usepackage{pifont}
\usepackage[inkscapeformat=png]{svg}
\usepackage{graphicx}
\usepackage{times}
\setbeamertemplate{caption}[numbered]
% % \setbeamertemplate{bibliography item}{[\theenumiv]}

\setbeamerfont{bibliography item}{size=\tiny}
\setbeamerfont{bibliography entry author}{size=\tiny}
\setbeamerfont{bibliography entry title}{size=\tiny}
\setbeamerfont{bibliography entry location}{size=\tiny}
\setbeamerfont{bibliography entry note}{size=\tiny}

\setbeamerfont{frametitle}{size=\large}

\usepackage{caption}
\usepackage{float}
\usepackage{xcolor}
\usepackage{listings}
\usepackage{animate}

\definecolor{codegreen}{rgb}{0,0.6,0}
\definecolor{codegray}{rgb}{0.5,0.5,0.5}
\definecolor{codepurple}{rgb}{0.58,0,0.82}
\definecolor{backcolour}{rgb}{0.95,0.95,0.92}
 
\lstdefinestyle{mystyle}{
    backgroundcolor=\color{backcolour},   
    commentstyle=\color{codegreen},
    keywordstyle=\color{magenta},
    numberstyle=\tiny\color{codegray},
    stringstyle=\color{codepurple},
    basicstyle=\footnotesize,
    breakatwhitespace=false,         
    breaklines=true,                 
    captionpos=b,                    
    keepspaces=true,                 
    numbers=left,                    
    numbersep=5pt,                  
    showspaces=false,                
    showstringspaces=false,
    showtabs=false,                  
    tabsize=2
}
 
\lstset{style=mystyle}

\usepackage{ragged2e}
\setbeamercolor{section in foot}{fg=white,bg=darkorange}
\setbeamercolor{subsection in foot}{fg=white,bg=darkorange}
\setbeamercolor{frametitle}{fg=white, bg=darkorange}
\setbeamercolor{title}{fg=white, bg=darkorange}
\setbeamercolor{frame}{bg=darkorange}
\setbeamercolor{block title}{bg=darkorange,fg=white}

\setbeamercolor{item}{fg=darkorange}

% \definecolor{darkorange}{rgb}{0.81, 0.52, 0.05}
\definecolor{darkorange}{rgb}{1,0.5,0}
\definecolor{darkorange2}{rgb}{1, 0.64, 0.2}
\definecolor{honeydew}{rgb}{1, 0.85, 0.45}


\newenvironment{variableblock}[3]{%
  \setbeamercolor{block body}{#2}
  \setbeamercolor{block title}{#3}
  \begin{block}{#1}}{\end{block}}

\newenvironment{prosblock}[1]{%
  % \setbeamercolor{block body}{bg=blue,fg=white}
  \setbeamercolor{block title}{bg=blue,fg=white}
  \begin{block}{#1}}{\end{block}}

\newenvironment{consblock}[1]{%
  % \setbeamercolor{block body}{bg=red,fg=white}
  \setbeamercolor{block title}{bg=red,fg=white}
  \begin{block}{#1}}{\end{block}}

\newcommand{\cmark}{\ding{51}}%
\newcommand{\xmark}{\ding{55}}%

\renewcommand{\arraystretch}{1.5}

% Please add the following required packages to your document preamble:
\usepackage{booktabs}
\usepackage{multirow}
\usepackage{colortbl}
% Beamer presentation requires \usepackage{colortbl} instead of \usepackage[table,xcdraw]{xcolor}

\usepackage{tabularray}\UseTblrLibrary{varwidth}
\usepackage{xcolor}
\def\BibTeX{{\rm B\kern-.05em{\sc i\kern-.025em b}\kern-.08em
    T\kern-.1667em\lower.7ex\hbox{E}\kern-.125emX}}
% \usepackage{cite}
\usepackage{amsmath}
\newcommand{\probP}{\text{I\kern-0.15em P}}
\usepackage{etoolbox}
\patchcmd{\thebibliography}{\section*{\refname}}{}{}{}

\setlength\tabcolsep{0.5pt}

\renewcommand{\arraystretch}{0.9}
\setlength{\tabcolsep}{2pt}

\usepackage{pgffor}

\setbeamerfont{bibliography item}{size=\tiny}
\setbeamerfont{bibliography entry author}{size=\tiny}
\setbeamerfont{bibliography entry title}{size=\tiny}
\setbeamerfont{bibliography entry location}{size=\tiny}
\setbeamerfont{bibliography entry note}{size=\tiny}

\setbeamerfont{bibliography entry author}{shape=\upshape,series=\mdseries,size=\footnotesize}
\setbeamerfont{bibliography entry title}{shape=\slshape,series=\mdseries,size=\footnotesize}
\setbeamerfont{bibliography entry journal}{shape=\upshape,series=\mdseries,size=\footnotesize}
\setbeamerfont{bibliography entry note}{shape=\upshape,series=\mdseries,size=\footnotesize}

\renewcommand*{\bibfont}{\scriptsize}

\newenvironment<>{varblock}[2][.9\textwidth]{%
  \setlength{\textwidth}{#1}
  \begin{actionenv}#3%
    \def\insertblocktitle{#2}%
    \par%
    \usebeamertemplate{block begin}}
  {\par%
    \usebeamertemplate{block end}%
  \end{actionenv}}

% \setbeamertemplate{footline}[frame number]

\setbeamertemplate{footline}{
  \leavevmode%
  \hfill
  \usebeamercolor[fg]{page number in head/foot}%
  \scriptsize%
  \ifnum\value{framenumber}>19%
    Appendix \number\numexpr\value{framenumber}-19\relax/32%
  \else%
    \ifnum\value{framenumber}>16%
      %
    \else
      \number\numexpr\value{framenumber}\relax/16%
    \fi

  \fi%
  \hspace{1em}
}


\begin{document}

\author{\textbf{Julien Soulé$^{1,2}$}, Jean-Paul Jamont$^1$, Michel Occello$^1$, Louis-Marie Traonouez$^2$, Paul Théron$^3$}

\title{\textbf{Towards Assisted MAS Design: A Library for
Explainable MARL with Organizational Model}}

\subtitle{ECAI 2024 Demo Presentation}

% \logo{\includegraphics[scale=0.01]{figures/grenoble-inp_logo.png}}

\institute{\footnotesize \textit{University Grenoble Alpes, Grenoble INP, LCIS, 26000, Valence, France \\
$^1$\{julien.soule, jean-paul.jamont, michel.occello\}@lcis.grenoble-inp.fr \\ \phantom{U} \\
Thales Land and Air Systems, BL IAS, 35000, Rennes, France \\
$^2$\{julien.soule, louis-marie.traonouez\}@thalesgroup.com \\ \phantom{U} \\
AICA IWG, La Guillermie, France \\
$^3$paul.theron@orange.fr}}


\date{\textit{\footnotesize May 9, 2024}}

%\subject{}
\setbeamercovered{transparent}
%\setbeamertemplate{navigation symbols}{}
\begin{frame}[plain]
	\maketitle\vspace{-0.8cm}
	\begin{figure}[ht!]
		\centering
            \includegraphics[height=0.8cm]{figures/la-ruche_logo.png}
            \hspace{0.8cm}
            \includegraphics[height=0.8cm]{figures/lcis_logo.png}
            \hspace{0.8cm}
		\includegraphics[height=0.8cm]{figures/grenoble-inp_logo.png}
            \hspace{0.8cm}
            \includegraphics[height=0.8cm]{figures/uga_logo.jpg}
	\end{figure}
\end{frame}

% \begin{frame}{Content}
%   \tableofcontents
% \end{frame}

\addtocounter{framenumber}{-1}

\section{Introduction}

\begin{frame}{Un sens émergent de l'organisation dans le MARL ?}

  \begin{columns}[T]

    \hspace{-0.5cm}
    \begin{column}{0.3\textwidth}

      \vspace{1cm}
      \begin{itemize}
        \item Même sans instructions explicites, certains motifs comportementaux distincts émergent.
        \item Dans certaines configurations, ces comportements ressemblent à des \textbf{rôles} (et \textbf{objectifs}) définis par des humains.
        \item Mais dans autres cas, coordination reste ambiguë.
      \end{itemize}

    \end{column}

    \begin{column}{0.8\textwidth}

      \begin{columns}[T]
        \begin{column}{0.48\textwidth}
          \centering
          \animategraphics[autoplay,loop,width=0.8\linewidth]{8}{figures/overcooked_asymmetric_advantage/frame}{0}{33} \\
          \small{\textbf{Avantage Asymétrique}}\\
          \vspace{0.3em}
          \animategraphics[autoplay,loop,width=0.7\linewidth]{8}{figures/overcooked_coordination_ring/frame}{0}{33} \\
          \small{\textbf{Anneau de Coordination}}
        \end{column}
        \begin{column}{0.48\textwidth}
          \centering
          \animategraphics[autoplay,loop,width=0.75\linewidth]{8}{figures/overcooked_counter_circuit/frame}{0}{33} \\
          \small{\textbf{Circuit de Comptoir}}\\
          \vspace{0.3em}
          \animategraphics[autoplay,loop,width=0.7\linewidth]{8}{figures/overcooked_forced_coordination/frame}{0}{33} \\
          \small{\textbf{Coordination Forcée}}
        \end{column}
      \end{columns}

    \end{column}

  \end{columns}

  \hspace{-1cm}
  \vspace{-0.5cm}
  {\tiny \textit{Overcooked-AI~\autocite{overcookedai}}}

\end{frame}

\begin{frame}{De motifs émergents à l'adéquation organisationnelle}

  \begin{columns}[c]
    % COLONNE GAUCHE : comportement appris
    \begin{column}{0.48\textwidth}
      \centering
      \animategraphics[autoplay,loop,width=0.95\linewidth]{8}{figures/overcooked_asymmetric_advantage/frame}{0}{33} \\
      \small\textbf{Comportement Appris (après MARL)}
    \end{column}

    % COLONNE DROITE : comportement idéal défini par l'humain
    \begin{column}{0.48\textwidth}
      \centering
      \animategraphics[autoplay,loop,width=0.95\linewidth]{8}{figures/clean_overcooked_asymmetric_advantage/frame}{0}{33} \\
      \small\textbf{Organisation Régulière (conçue par des humains)}
    \end{column}
  \end{columns}

  \vspace{1em}

  \begin{itemize}
    \item Agents entraînés peuvent manifester régularités comportementales proches d'une \textbf{organisation implicite} comprenant spécifications \textquote{débruitées}:
          \begin{itemize}
            \item \textbf{Structurelles} — rôles réguliers
            \item \textbf{Fonctionnelles} — objectifs réguliers
          \end{itemize}
  \end{itemize}

  \begin{block}{\textbf{Adéquation organisationnelle} : un concept pratique proposé comme\dots}
    La \textquote{distance} entre comportements des agents entraînés et ceux de organisation implicite.
  \end{block}

\end{frame}

\begin{frame}{Jusqu'où peut-on pousser le concept d'adéquation organisationnelle ?}

  \begin{block}{\textbf{Voir les agents MARL à travers le prisme organisationnel ?}}

    Deux questions de recherche centrales :

    \begin{enumerate}
      \item \textbf{Comment peut-on évaluer l'adéquation organisationnelle ?} \\
            Mesurer dans quelle mesure les comportements des agents correspondent à une organisation fonctionnelle et structurée ?

            \vspace{0.8em}
      \item \textbf{Comment peut-on contrôler l'adéquation organisationnelle ?} \\
            Guider l'entraînement des agents pour qu'ils s'alignent avec des rôles et objectifs prédéfinis ?
    \end{enumerate}

  \end{block}

\end{frame}

\section{Travaux connexes \& Contribution}

\begin{frame}{Travaux connexes \& Contribution}{Évaluation/Contrôle de l'adéquation organisationnelle ?}

  \begin{columns}
    \vspace{-0.7cm}
    \begin{column}{0.35\textwidth}
      \centering
      \begin{block}{Limitations}
        \begin{footnotesize}
          \begin{itemize}
            \item Aucun équivalent direct du concept d'adéquation organisationnelle.
            \item Aucun moyen systématique de quantifier dans quelle mesure les comportements appris s'alignent avec une organisation implicite ou donnée.
            \item Aucun cadre général pour contrôler ou guider les agents de manière organisationnelle.
          \end{itemize}
        \end{footnotesize}
      \end{block}
      \begin{exampleblock}{Contribution}
        Modèle organisationnel + MARL ? \\ \ \ \ \ $\rightarrow$ \textbf{MOISE+MARL}
      \end{exampleblock}
    \end{column}
    \hspace{0.1cm}
    \begin{column}{0.7\textwidth}
      \centering
      \begin{footnotesize}
        \textquote{Contrôle de l'adéquation organisationnelle} ?
        \begin{itemize}
          \item Wilson et al.~\autocite{wilson2008learning} : Transfert de rôles entre environnements dans des MDP multi-agents, mais manque d'abstraction.
          \item Berenji et Vengerov~\autocite{berenji2000learning} : Coordination et inférence de rôles dans des UAVs — spécifique à une tâche, non généralisable.
          \item Yusuf et Baber~\autocite{yusuf2020inferential} : Inférence bayésienne pour la coordination de tâches — sans métriques d'alignement organisationnel.
          \item Serrino et al.~\autocite{serrino2019finding} : Inférence de rôles sociaux à partir d'interactions — limité à des rôles opérationnels à court terme.
        \end{itemize}

        \vspace{0.5cm}

        \textquote{Évaluation de l'adéquation organisationnelle} ?
        \begin{itemize}
          \item CPO~\autocite{achiam2017cpo} : Contraintes de sûreté pour régulariser l'apprentissage — pas d'extension aux contraintes organisationnelles.
          \item Ray et al.~\autocite{ray2019benchmarking} : Shaping de récompense lagrangienne — pas de contraintes sur des espaces d'action dynamiques.
          \item \textit{Safe exploration}~\autocite{garcia2015comprehensive,alshiekh2018safe} : Vise à éviter les états dangereux — sans respect des rôles.
          \item HRL~\autocite{ghavamzadeh2006hrl} : Décomposition de tâches alignée avec des hiérarchies — mais imposée de manière interne.
        \end{itemize}
      \end{footnotesize}
    \end{column}
  \end{columns}

\end{frame}

\section{Cadre MOISE+MARL}

\begin{frame}{Cadre MOISE+MARL}{Le cadre $\mathcal{M}OISE^+$ (1/2)}

  \begin{figure}
    \centering
    \includegraphics[width=0.75\linewidth]{figures/moise_model.png}
  \end{figure}

  \begin{spacing}{0.25}
    {\tiny Hübner, J. F., Sichman, J. S., and Boissier, O. (2002).
      Un modèle pour la spécification structurelle, fonctionnelle et déontique
      des organisations dans les systèmes multi-agents.
      In Bittencourt, G. et Ramalho, G. L., éditeurs, Actes du 16ème Symposium Brésilien sur l'Intelligence Artificielle (SBIA'02), volume 2507 de LNAI, pages 118–128, Berlin. Springer.}
  \end{spacing}

\end{frame}

\begin{frame}{Le cadre MOISE+MARL}{Modèle markovien}

  \begin{columns}

    \hspace{-2ex}

    \begin{column}{0.4\textwidth}

      \begin{figure}
        \includegraphics[width=\linewidth]{figures/marl_basics.png}
      \end{figure}

    \end{column}

    \begin{column}{0.7\textwidth}
      \vspace{-2ex}

      \begin{center}
        \begin{minipage}{0.95\linewidth}
          \centering
          \begin{block}{Modèles markoviens pour le MARL : Dec-POMDP}
            {\small
              Processus Décisionnel de Markov Partiellement Observable et Décentralisé (Dec-POMDP)~\autocite{Oliehoek2016}
              \begin{itemize}
                \item considère plusieurs agents dans une approche MAS similaire ;
                \item processus stochastiques pour modéliser l'incertitude des changements environnementaux, y compris les observations ;
                \item fonction de récompense commune aux agents, favorisant l'apprentissage d'actions collaboratives~\autocite{Beynier2013}
              \end{itemize}
            }

            { \scriptsize

              $(S,\{A_i\},T,R,\{\Omega_i\},O,\gamma)$ , où
              \begin{itemize}
                \item $S = \{s_1, ..s_{|S|}\}$ : ensemble des états possibles ;
                \item $A_{i} = \{a_{1}^{i},..,a_{|A_{i}|}^{i}\}$ : ensemble des actions possibles pour l'agent $i$ ;
                \item $T$ tel que $T(s,a,s') = \probP{(s'|s,a)}$ : probabilités de transition conditionnelles ;
                \item $R: S \times A \times S \rightarrow \mathbb{R}$ : fonction de récompense ;
                \item $\Omega_{i} = \{o_{1}^{i},..,o_{|\Omega_{i}|}^{i}\}$ : ensemble des observations pour l'agent $ag_i$ ;
                \item $O$ tel que $O(s',a,o) = \probP{(o|s',a)}$ : probabilités conditionnelles d'observation ;
                \item $\gamma \in [0,1]$ : facteur d'actualisation.
              \end{itemize}

            }

          \end{block}

        \end{minipage}
      \end{center}

    \end{column}

  \end{columns}

\end{frame}

\begin{frame}{Le cadre MOISE+MARL}{Approche}

  \begin{columns}[c]

    \begin{column}{0.5\textwidth}

      \begin{itemize}
        \item Combiner Dec-POMDP avec le modèle organisationnel MOISE+.
        \item Les agents se voient attribuer des rôles et des missions.
        \item Utiliser des guides de contraintes pour ajuster :
              \begin{itemize}
                \item \textbf{Actions} via \textbf{Guides d'Actions par Rôle} (RAG)
                \item \textbf{Récompenses} via \textbf{Guides de Récompense par Rôle} (RRG) et \textbf{Guides de Récompense par Objectif} (GRG)
              \end{itemize}
      \end{itemize}

      \noindent
      \hspace{0.5cm}
      \begin{minipage}{\linewidth}
        \begin{varblock}[5.5cm]{Implémentation des \textbf{rôles} et \textbf{objectifs} $\sim$ \textbf{Guides de Contrainte}}

          \begin{itemize}
            \item $rag: H \times \Omega \rightarrow \mathcal{P}(A \times \mathbb{R})$
            \item $rrg: H \times \Omega \times A \to \mathbb{R}$
            \item $grg: H \rightarrow \mathbb{R}$
          \end{itemize}

        \end{varblock}
      \end{minipage}

    \end{column}

    \hspace{-0.2cm}

    \begin{column}{0.6\textwidth}
      \begin{figure}
        \centering
        \includegraphics[width=1.\linewidth]{figures/mm_simple_representation.png}
      \end{figure}
    \end{column}
  \end{columns}
\end{frame}

\begin{frame}{Le cadre MOISE+MARL}{Approche}

  \begin{figure}
    \centering
    \includegraphics[width=1.07\linewidth]{figures/modified_state_value_function.png}

    \

    \phantom{X}\\

    \begin{tikzpicture}
      \fill[black] (0,0) rectangle +(0.3,0.3);
      \node[right=0.4cm] at (0.3,0.15) {\small \textbf{MARL standard (fonction de valeur d'état originale)}};
      \fill[red] (0,-0.8) rectangle +(0.3,0.3);
      \node[right=0.4cm] at (0.3,-0.65) {\small \textbf{Guidage par rôle (RAG, RRG)}};
      \fill[blue] (0,-1.6) rectangle +(0.3,0.3);
      \node[right=0.4cm] at (0.3,-1.45) {\small \textbf{Guidage par objectif (GRG)}};
    \end{tikzpicture}

  \end{figure}

  \begin{tikzpicture}[remember picture, overlay]

    \node[anchor=north west, draw=red, fill=red, text=white, rounded corners=1pt, inner sep=2pt]
    at ([xshift=1.3cm,yshift=-3.69cm]current page.north west) {\small \textit{RAG}};

    \node[anchor=north west, draw=red, fill=red, text=white, rounded corners=1pt, inner sep=2pt]
    at ([xshift=11.5cm,yshift=-3.69cm]current page.north west) {\small \textit{RRG}};

    \node[anchor=north west, draw=blue, fill=blue, text=white, rounded corners=1pt, inner sep=2pt]
    at ([xshift=9.1cm,yshift=-3.69cm]current page.north west) {\small \textit{GRG}};

  \end{tikzpicture}

\end{frame}

\begin{frame}{Le cadre MOISE+MARL}{Présentation de la méthode TEMM}

  \textbf{Évaluation basée sur les trajectoires dans MOISE+MARL (TEMM)}
  \begin{itemize}
    \item \textbf{Objectif} : Fournir une interprétation organisationnelle du comportement des agents entraînés.
  \end{itemize}

  \vspace{1em}
  \textbf{Hypothèses sous-jacentes :}
  \begin{itemize}
    \item \textbf{Rôles} $\sim$ motifs fréquents de transitions \emph{(observation, action)} dans les trajectoires des agents.
    \item \textbf{Objectifs} $\sim$ observations reçues fréquemment dans les trajectoires.
  \end{itemize}

  \begin{center}
    \begin{columns}[c]

      \begin{column}{0.4\textwidth}
        \centering
        


\tikzset{every picture/.style={line width=0.75pt}} %set default line width to 0.75pt        

\begin{tikzpicture}[x=0.75pt,y=0.75pt,yscale=-1,xscale=1]
%uncomment if require: \path (0,1974); %set diagram left start at 0, and has height of 1974

%Shape: Rectangle [id:dp21508130618742183] 
\draw  [fill={rgb, 255:red, 255; green, 255; blue, 255 }  ,fill opacity=1 ] (27.9,1694.11) -- (180,1694.11) -- (180,1780) -- (27.9,1780) -- cycle ;
%Straight Lines [id:da4715774117452397] 
\draw [color={rgb, 255:red, 208; green, 2; blue, 27 }  ,draw opacity=1 ]   (146.57,1706.84) -- (128.18,1713.49) -- (91.16,1728.41) -- (112.58,1740.12) -- (97.43,1737.36) -- (90.48,1739.01) -- (90.48,1748.77) -- (85.95,1752.07) -- (85.12,1752.67) -- (82.55,1750.8) -- (74.41,1744.86) -- (58.34,1744.86) -- (60.9,1746.73) -- (52.99,1748.77) -- (55.75,1752.79) -- (42.28,1764.38) ;
\draw [shift={(149.39,1705.82)}, rotate = 160.12] [fill={rgb, 255:red, 208; green, 2; blue, 27 }  ,fill opacity=1 ][line width=0.08]  [draw opacity=0] (3.57,-1.72) -- (0,0) -- (3.57,1.72) -- cycle    ;
%Straight Lines [id:da12621479480258224] 
\draw [color={rgb, 255:red, 80; green, 227; blue, 194 }  ,draw opacity=1 ]   (147.37,1704.13) -- (138.68,1713.63) -- (117.26,1713.63) -- (117.26,1721.44) -- (90.48,1729.25) -- (95.83,1733.15) -- (101.19,1740.96) -- (85.12,1737.06) -- (90.48,1744.86) -- (85.12,1744.86) -- (90.48,1752.67) -- (79.77,1752.67) -- (71.84,1744.94) -- (69.05,1750.72) -- (47.63,1739.01) -- (63.7,1752.67) -- (47.63,1744.86) -- (52.99,1752.67) -- (47.63,1768.29) ;
\draw [shift={(149.39,1701.92)}, rotate = 132.45] [fill={rgb, 255:red, 80; green, 227; blue, 194 }  ,fill opacity=1 ][line width=0.08]  [draw opacity=0] (3.57,-1.72) -- (0,0) -- (3.57,1.72) -- cycle    ;
%Straight Lines [id:da7064890769152855] 
\draw [color={rgb, 255:red, 248; green, 231; blue, 28 }  ,draw opacity=1 ]   (157.13,1710.14) -- (130.73,1713.75) -- (127.97,1717.54) -- (109.3,1721.56) -- (95.83,1729.25) -- (97.43,1737.36) -- (93.24,1741.08) -- (85.12,1733.15) -- (95.83,1748.77) -- (95.83,1752.67) -- (85.12,1752.67) -- (63.89,1745.06) -- (61.1,1750.84) -- (45.04,1752.79) -- (39.68,1768.41) ;
\draw [shift={(160.1,1709.73)}, rotate = 172.2] [fill={rgb, 255:red, 248; green, 231; blue, 28 }  ,fill opacity=1 ][line width=0.08]  [draw opacity=0] (3.57,-1.72) -- (0,0) -- (3.57,1.72) -- cycle    ;
%Straight Lines [id:da7638374972620724] 
\draw [color={rgb, 255:red, 144; green, 19; blue, 254 }  ,draw opacity=1 ]   (164.12,1713.39) -- (169.74,1714.41) -- (165.46,1709.73) -- (161.17,1706.61) -- (168.76,1709.73) -- (174.03,1714.41) -- (165.46,1719.1) -- (170.81,1725.34) -- (165.46,1733.15) -- (170.81,1737.06) -- (165.46,1748.77) -- (165.46,1764.38) -- (149.39,1760.48) -- (138.68,1760.48) -- (132.88,1759.07) -- (129.82,1758.33) -- (125.47,1757.27) -- (122.61,1756.58) -- (111.86,1755.71) -- (103.33,1755.01) -- (99.13,1754.25) -- (90.48,1752.67) -- (79.77,1752.67) -- (74.41,1756.58) -- (63.7,1760.48) -- (63.7,1768.29) ;
\draw [shift={(161.17,1712.85)}, rotate = 10.33] [fill={rgb, 255:red, 144; green, 19; blue, 254 }  ,fill opacity=1 ][line width=0.08]  [draw opacity=0] (3.57,-1.72) -- (0,0) -- (3.57,1.72) -- cycle    ;
%Straight Lines [id:da2586635599751653] 
\draw [color={rgb, 255:red, 65; green, 117; blue, 5 }  ,draw opacity=1 ]   (163.05,1714.17) -- (168.67,1715.19) -- (164.39,1710.51) -- (160.1,1707.39) -- (167.69,1710.51) -- (172.95,1715.19) -- (165.46,1723.78) -- (169.74,1726.13) -- (167.6,1737.84) -- (154.75,1733.15) -- (165.46,1739.4) -- (178.31,1751.89) -- (161.17,1742.52) -- (164.39,1749.55) -- (167.6,1761.26) -- (156.89,1764.38) -- (139.75,1758.14) -- (126.9,1758.14) -- (120.47,1758.14) -- (114.04,1756.58) -- (111.9,1759.7) -- (107.62,1756.58) -- (109.76,1761.26) -- (98.06,1755.03) -- (89.41,1753.45) -- (78.69,1753.45) -- (73.34,1757.36) -- (62.63,1761.26) -- (62.63,1769.07) ;
\draw [shift={(160.1,1713.63)}, rotate = 10.33] [fill={rgb, 255:red, 65; green, 117; blue, 5 }  ,fill opacity=1 ][line width=0.08]  [draw opacity=0] (3.57,-1.72) -- (0,0) -- (3.57,1.72) -- cycle    ;
%Shape: Polygon Curved [id:ds11407168049221061] 
\draw  [color={rgb, 255:red, 74; green, 144; blue, 226 }  ,draw opacity=1 ][fill={rgb, 255:red, 74; green, 144; blue, 226 }  ,fill opacity=0.5 ] (31.11,1764.38) .. controls (34.28,1759.39) and (40.53,1757.99) .. (46.46,1758.18) .. controls (50.95,1758.33) and (55.26,1759.39) .. (57.89,1760.48) .. controls (64,1763.02) and (69.46,1762.63) .. (68.6,1768.29) .. controls (67.74,1773.95) and (60.78,1773.17) .. (52.53,1772.19) .. controls (44.29,1771.22) and (25.54,1773.17) .. (31.11,1764.38) -- cycle ;
%Shape: Polygon Curved [id:ds9280877944894264] 
\draw  [color={rgb, 255:red, 208; green, 2; blue, 27 }  ,draw opacity=1 ][fill={rgb, 255:red, 208; green, 2; blue, 27 }  ,fill opacity=0.5 ] (143.58,1705.82) .. controls (149.15,1697.04) and (151.91,1697.59) .. (148.94,1701.92) .. controls (145.97,1706.25) and (158.58,1699.87) .. (157.72,1705.53) .. controls (156.86,1711.19) and (173.25,1714.61) .. (165,1713.63) .. controls (156.76,1712.66) and (138.01,1714.61) .. (143.58,1705.82) -- cycle ;


% Text Node
\draw (109.47,1704.35) node  [font=\tiny,color={rgb, 255:red, 202; green, 52; blue, 69 }  ,opacity=1 ] [align=left] {$\displaystyle g_{*} =\Omega _{goal}$};
% Text Node
\draw (77.33,1773.35) node  [font=\tiny,color={rgb, 255:red, 74; green, 144; blue, 226 }  ,opacity=1 ] [align=left] {$\displaystyle \Omega _{init}$};
% Text Node
\draw (37,1704) node  [font=\scriptsize] [align=left] {$\displaystyle \Omega $};
% Text Node
\draw (96.51,1791) node   [align=left] {{\tiny \textit{An abstract visualization of}}};
\draw (96.51,1800) node   [align=left] {{\tiny \textit{observations in trajectories}}};

\end{tikzpicture}
      \end{column}

      \begin{column}{0.1\textwidth}
      \end{column}

      \begin{column}{0.4\textwidth}
        \centering
        


\tikzset{every picture/.style={line width=0.75pt}} %set default line width to 0.75pt        

\begin{tikzpicture}[x=0.75pt,y=0.75pt,yscale=-1,xscale=1]
%uncomment if require: \path (0,1974); %set diagram left start at 0, and has height of 1974

%Shape: Rectangle [id:dp10581972605309897] 
\draw  [fill={rgb, 255:red, 255; green, 255; blue, 255 }  ,fill opacity=1 ] (189.9,1694) -- (342,1694) -- (342,1779.89) -- (189.9,1779.89) -- cycle ;
%Straight Lines [id:da15954508078344698] 
\draw [color={rgb, 255:red, 208; green, 2; blue, 27 }  ,draw opacity=1 ]   (308.57,1706.73) -- (290.18,1713.38) -- (253.16,1728.29) -- (292,1737.89) -- (274,1745.89) -- (262,1749.89) -- (252.48,1748.66) -- (246,1737.89) -- (240,1737.89) -- (236,1739.89) -- (236.41,1744.75) -- (220.34,1744.75) -- (222.9,1746.61) -- (214.99,1748.66) -- (217.75,1752.68) -- (204.28,1764.27) ;
\draw [shift={(311.39,1705.71)}, rotate = 160.12] [fill={rgb, 255:red, 208; green, 2; blue, 27 }  ,fill opacity=1 ][line width=0.08]  [draw opacity=0] (3.57,-1.72) -- (0,0) -- (3.57,1.72) -- cycle    ;
%Straight Lines [id:da4456358262502481] 
\draw [color={rgb, 255:red, 80; green, 227; blue, 194 }  ,draw opacity=1 ]   (309.37,1704.02) -- (300.68,1713.52) -- (279.26,1713.52) -- (279.26,1721.33) -- (252.48,1729.14) -- (250,1731.89) -- (248,1733.89) -- (247.12,1736.94) -- (252.48,1744.75) -- (247.12,1744.75) -- (242,1741.89) -- (242,1743.89) -- (233.84,1744.83) -- (231.05,1750.61) -- (209.63,1738.9) -- (225.7,1752.56) -- (209.63,1744.75) -- (214.99,1752.56) -- (209.63,1768.18) ;
\draw [shift={(311.39,1701.81)}, rotate = 132.45] [fill={rgb, 255:red, 80; green, 227; blue, 194 }  ,fill opacity=1 ][line width=0.08]  [draw opacity=0] (3.57,-1.72) -- (0,0) -- (3.57,1.72) -- cycle    ;
%Straight Lines [id:da9735384751024792] 
\draw [color={rgb, 255:red, 248; green, 231; blue, 28 }  ,draw opacity=1 ]   (319.13,1710.02) -- (292.73,1713.64) -- (289.97,1717.42) -- (271.3,1721.45) -- (257.83,1729.14) -- (280,1725.89) -- (284,1727.89) -- (290,1737.89) -- (257.83,1748.66) -- (257.83,1752.56) -- (247.12,1752.56) -- (225.89,1744.95) -- (223.1,1750.73) -- (207.04,1752.68) -- (201.68,1768.3) ;
\draw [shift={(322.1,1709.62)}, rotate = 172.2] [fill={rgb, 255:red, 248; green, 231; blue, 28 }  ,fill opacity=1 ][line width=0.08]  [draw opacity=0] (3.57,-1.72) -- (0,0) -- (3.57,1.72) -- cycle    ;
%Straight Lines [id:da9773846199264294] 
\draw [color={rgb, 255:red, 144; green, 19; blue, 254 }  ,draw opacity=1 ]   (326.12,1713.28) -- (331.74,1714.3) -- (327.46,1709.62) -- (323.17,1706.49) -- (330.76,1709.62) -- (336.03,1714.3) -- (327.46,1718.99) -- (332.81,1725.23) -- (324,1733.89) -- (330,1739.89) -- (312,1769.89) -- (320,1771.89) -- (306,1777.89) -- (306,1739.89) -- (300,1747.89) -- (311.39,1760.37) -- (300.68,1760.37) -- (294.88,1758.96) -- (291.82,1758.21) -- (287.47,1757.16) -- (284.61,1756.46) -- (273.86,1755.59) -- (265.33,1754.9) -- (261.13,1754.14) -- (250,1755.89) -- (241.77,1752.56) -- (236.41,1756.46) -- (225.7,1760.37) -- (225.7,1768.18) ;
\draw [shift={(323.17,1712.74)}, rotate = 10.33] [fill={rgb, 255:red, 144; green, 19; blue, 254 }  ,fill opacity=1 ][line width=0.08]  [draw opacity=0] (3.57,-1.72) -- (0,0) -- (3.57,1.72) -- cycle    ;
%Straight Lines [id:da5869208077424531] 
\draw [color={rgb, 255:red, 65; green, 117; blue, 5 }  ,draw opacity=1 ]   (325.05,1714.06) -- (330.67,1715.08) -- (326.39,1710.4) -- (322.1,1707.27) -- (329.69,1710.4) -- (334.95,1715.08) -- (327.46,1723.67) -- (331.74,1726.01) -- (329.6,1737.72) -- (316.75,1733.04) -- (327.46,1739.29) -- (314,1769.89) -- (318,1773.89) -- (310,1775.89) -- (304,1741.89) -- (298,1745.89) -- (301.75,1758.03) -- (288.9,1758.03) -- (282.47,1758.03) -- (276.04,1756.46) -- (273.9,1759.59) -- (269.62,1756.46) -- (271.76,1761.15) -- (260.06,1754.92) -- (251.41,1753.34) -- (240.69,1753.34) -- (235.34,1757.24) -- (224.63,1761.15) -- (224.63,1768.96) ;
\draw [shift={(322.1,1713.52)}, rotate = 10.33] [fill={rgb, 255:red, 65; green, 117; blue, 5 }  ,fill opacity=1 ][line width=0.08]  [draw opacity=0] (3.57,-1.72) -- (0,0) -- (3.57,1.72) -- cycle    ;


% Text Node
\draw (209.73,1703.84) node  [font=\scriptsize] [align=left] {$\displaystyle \Omega \times A$};
% Text Node
\draw (267.61,1790.89) node   [align=left] {{\tiny \textit{An abstract visualization of}}};
\draw (267.61,1800) node   [align=left] {{\tiny \textit{transitions in trajectories}}};

\end{tikzpicture}
      \end{column}
    \end{columns}
  \end{center}

  \begin{itemize}
    \item[ ] \phantom{\text{adéquation organisationnelle Structurel (SOF) = score de déviation normalisé} ($clusters$)}
    \item Comment y parvenir à partir des trajectoires collectées ?
    \item[ ] \phantom{adéquation organisationnelle = $\frac{SOF + FOF}{2}$}
  \end{itemize}

\end{frame}

\begin{frame}{Le cadre MOISE+MARL}{Présentation de la méthode TEMM}

  \textbf{Évaluation basée sur les trajectoires dans MOISE+MARL (TEMM)}
  \begin{itemize}
    \item \textbf{Objectif} : Fournir une interprétation organisationnelle du comportement des agents entraînés.
  \end{itemize}

  \vspace{1em}
  \textbf{Hypothèses sous-jacentes :}
  \begin{itemize}
    \item \textbf{Rôles} $\sim$ motifs fréquents de transitions \emph{(observation, action)} dans les trajectoires des agents.
    \item \textbf{Objectifs} $\sim$ observations fréquentes reçues dans les trajectoires.
  \end{itemize}

  \begin{center}
    \begin{columns}[c]

      \begin{column}{0.4\textwidth}
        \centering
        


\tikzset{every picture/.style={line width=0.75pt}} %set default line width to 0.75pt        

\begin{tikzpicture}[x=0.75pt,y=0.75pt,yscale=-1,xscale=1]
    %uncomment if require: \path (0,1974); %set diagram left start at 0, and has height of 1974

    %Shape: Rectangle [id:dp9996076613305621] 
    \draw  [fill={rgb, 255:red, 255; green, 255; blue, 255 }  ,fill opacity=1 ] (24,1558.11) -- (176.1,1558.11) -- (176.1,1644) -- (24,1644) -- cycle ;
    %Straight Lines [id:da05824332013205091] 
    \draw [color={rgb, 255:red, 208; green, 2; blue, 27 }  ,draw opacity=1 ]   (142.67,1570.84) -- (124.28,1577.49) -- (87.26,1592.41) -- (108.68,1604.12) -- (93.53,1601.36) -- (86.58,1603.01) -- (86.58,1612.77) -- (82.05,1616.07) -- (81.22,1616.67) -- (78.65,1614.8) -- (70.51,1608.86) -- (54.44,1608.86) -- (57,1610.73) -- (49.09,1612.77) -- (51.85,1616.79) -- (38.38,1628.38) ;
    \draw [shift={(145.49,1569.82)}, rotate = 160.12] [fill={rgb, 255:red, 208; green, 2; blue, 27 }  ,fill opacity=1 ][line width=0.08]  [draw opacity=0] (3.57,-1.72) -- (0,0) -- (3.57,1.72) -- cycle    ;
    %Straight Lines [id:da9249559779542824] 
    \draw [color={rgb, 255:red, 80; green, 227; blue, 194 }  ,draw opacity=1 ]   (143.47,1568.13) -- (134.78,1577.63) -- (113.36,1577.63) -- (113.36,1585.44) -- (86.58,1593.25) -- (91.93,1597.15) -- (97.29,1604.96) -- (81.22,1601.06) -- (86.58,1608.86) -- (81.22,1608.86) -- (86.58,1616.67) -- (75.87,1616.67) -- (67.94,1608.94) -- (65.16,1614.72) -- (43.73,1603.01) -- (59.8,1616.67) -- (43.73,1608.86) -- (49.09,1616.67) -- (43.73,1632.29) ;
    \draw [shift={(145.49,1565.92)}, rotate = 132.45] [fill={rgb, 255:red, 80; green, 227; blue, 194 }  ,fill opacity=1 ][line width=0.08]  [draw opacity=0] (3.57,-1.72) -- (0,0) -- (3.57,1.72) -- cycle    ;
    %Straight Lines [id:da17118391857757054] 
    \draw [color={rgb, 255:red, 248; green, 231; blue, 28 }  ,draw opacity=1 ]   (153.23,1574.14) -- (126.83,1577.75) -- (124.07,1581.54) -- (105.41,1585.56) -- (91.93,1593.25) -- (93.53,1601.36) -- (89.34,1605.08) -- (81.22,1597.15) -- (91.93,1612.77) -- (91.93,1616.67) -- (81.22,1616.67) -- (59.99,1609.06) -- (57.21,1614.84) -- (41.14,1616.79) -- (35.78,1632.41) ;
    \draw [shift={(156.2,1573.73)}, rotate = 172.2] [fill={rgb, 255:red, 248; green, 231; blue, 28 }  ,fill opacity=1 ][line width=0.08]  [draw opacity=0] (3.57,-1.72) -- (0,0) -- (3.57,1.72) -- cycle    ;
    %Straight Lines [id:da6427777277243145] 
    \draw [color={rgb, 255:red, 144; green, 19; blue, 254 }  ,draw opacity=1 ]   (160.23,1577.39) -- (165.84,1578.41) -- (161.56,1573.73) -- (157.27,1570.61) -- (164.86,1573.73) -- (170.13,1578.41) -- (161.56,1583.1) -- (166.91,1589.34) -- (161.56,1597.15) -- (166.91,1601.06) -- (161.56,1612.77) -- (161.56,1628.38) -- (145.49,1624.48) -- (134.78,1624.48) -- (128.99,1623.07) -- (125.92,1622.33) -- (121.57,1621.27) -- (118.71,1620.58) -- (107.96,1619.71) -- (99.43,1619.01) -- (95.23,1618.25) -- (86.58,1616.67) -- (75.87,1616.67) -- (70.51,1620.58) -- (59.8,1624.48) -- (59.8,1632.29) ;
    \draw [shift={(157.27,1576.85)}, rotate = 10.33] [fill={rgb, 255:red, 144; green, 19; blue, 254 }  ,fill opacity=1 ][line width=0.08]  [draw opacity=0] (3.57,-1.72) -- (0,0) -- (3.57,1.72) -- cycle    ;
    %Straight Lines [id:da7390021320622445] 
    \draw [color={rgb, 255:red, 65; green, 117; blue, 5 }  ,draw opacity=1 ]   (159.15,1578.17) -- (164.77,1579.19) -- (160.49,1574.51) -- (156.2,1571.39) -- (163.79,1574.51) -- (169.06,1579.19) -- (161.56,1587.78) -- (165.84,1590.13) -- (163.7,1601.84) -- (150.85,1597.15) -- (161.56,1603.4) -- (174.41,1615.89) -- (157.27,1606.52) -- (160.49,1613.55) -- (163.7,1625.26) -- (152.99,1628.38) -- (135.85,1622.14) -- (123,1622.14) -- (116.57,1622.14) -- (110.14,1620.58) -- (108,1623.7) -- (103.72,1620.58) -- (105.86,1625.26) -- (94.16,1619.03) -- (85.51,1617.45) -- (74.8,1617.45) -- (69.44,1621.36) -- (58.73,1625.26) -- (58.73,1633.07) ;
    \draw [shift={(156.2,1577.63)}, rotate = 10.33] [fill={rgb, 255:red, 65; green, 117; blue, 5 }  ,fill opacity=1 ][line width=0.08]  [draw opacity=0] (3.57,-1.72) -- (0,0) -- (3.57,1.72) -- cycle    ;
    %Shape: Ellipse [id:dp30050508180239144] 
    \draw  [draw opacity=0][fill={rgb, 255:red, 208; green, 2; blue, 27 }  ,fill opacity=0.62 ] (46.49,1615.89) .. controls (46.49,1614.6) and (47.93,1613.55) .. (49.71,1613.55) .. controls (51.48,1613.55) and (52.92,1614.6) .. (52.92,1615.89) .. controls (52.92,1617.19) and (51.48,1618.23) .. (49.71,1618.23) .. controls (47.93,1618.23) and (46.49,1617.19) .. (46.49,1615.89) -- cycle ;
    %Shape: Ellipse [id:dp15311501498248647] 
    \draw  [draw opacity=0][fill={rgb, 255:red, 208; green, 2; blue, 27 }  ,fill opacity=0.62 ] (90.49,1619.03) .. controls (90.49,1617.74) and (91.93,1616.69) .. (93.71,1616.69) .. controls (95.48,1616.69) and (96.92,1617.74) .. (96.92,1619.03) .. controls (96.92,1620.32) and (95.48,1621.37) .. (93.71,1621.37) .. controls (91.93,1621.37) and (90.49,1620.32) .. (90.49,1619.03) -- cycle ;
    %Shape: Ellipse [id:dp19167487081496637] 
    \draw  [draw opacity=0][fill={rgb, 255:red, 208; green, 2; blue, 27 }  ,fill opacity=0.62 ] (161.11,1606.52) .. controls (161.11,1605.23) and (162.54,1604.18) .. (164.32,1604.18) .. controls (166.09,1604.18) and (167.53,1605.23) .. (167.53,1606.52) .. controls (167.53,1607.82) and (166.09,1608.86) .. (164.32,1608.86) .. controls (162.54,1608.86) and (161.11,1607.82) .. (161.11,1606.52) -- cycle ;
    %Shape: Ellipse [id:dp9201279867822619] 
    \draw  [draw opacity=0][fill={rgb, 255:red, 208; green, 2; blue, 27 }  ,fill opacity=0.62 ] (120.4,1622.92) .. controls (120.4,1621.62) and (121.84,1620.58) .. (123.62,1620.58) .. controls (125.39,1620.58) and (126.83,1621.62) .. (126.83,1622.92) .. controls (126.83,1624.21) and (125.39,1625.26) .. (123.62,1625.26) .. controls (121.84,1625.26) and (120.4,1624.21) .. (120.4,1622.92) -- cycle ;
    %Shape: Ellipse [id:dp3048334813609519] 
    \draw  [draw opacity=0][fill={rgb, 255:red, 208; green, 2; blue, 27 }  ,fill opacity=0.62 ] (161.11,1590.91) .. controls (161.11,1589.61) and (162.54,1588.56) .. (164.32,1588.56) .. controls (166.09,1588.56) and (167.53,1589.61) .. (167.53,1590.91) .. controls (167.53,1592.2) and (166.09,1593.25) .. (164.32,1593.25) .. controls (162.54,1593.25) and (161.11,1592.2) .. (161.11,1590.91) -- cycle ;
    %Shape: Ellipse [id:dp7290465976812913] 
    \draw  [draw opacity=0][fill={rgb, 255:red, 208; green, 2; blue, 27 }  ,fill opacity=0.62 ] (86.13,1603.4) .. controls (86.13,1602.11) and (87.56,1601.06) .. (89.34,1601.06) .. controls (91.11,1601.06) and (92.55,1602.11) .. (92.55,1603.4) .. controls (92.55,1604.69) and (91.11,1605.74) .. (89.34,1605.74) .. controls (87.56,1605.74) and (86.13,1604.69) .. (86.13,1603.4) -- cycle ;
    %Shape: Ellipse [id:dp6154487622646608] 
    \draw  [draw opacity=0][fill={rgb, 255:red, 208; green, 2; blue, 27 }  ,fill opacity=0.62 ] (109.69,1583.1) .. controls (109.69,1581.8) and (111.13,1580.76) .. (112.9,1580.76) .. controls (114.68,1580.76) and (116.12,1581.8) .. (116.12,1583.1) .. controls (116.12,1584.39) and (114.68,1585.44) .. (112.9,1585.44) .. controls (111.13,1585.44) and (109.69,1584.39) .. (109.69,1583.1) -- cycle ;
    %Shape: Ellipse [id:dp6108483574180856] 
    \draw  [draw opacity=0][fill={rgb, 255:red, 189; green, 16; blue, 224 }  ,fill opacity=0.8 ] (77.56,1615.89) .. controls (77.56,1614.6) and (79,1613.55) .. (80.77,1613.55) .. controls (82.55,1613.55) and (83.98,1614.6) .. (83.98,1615.89) .. controls (83.98,1617.19) and (82.55,1618.23) .. (80.77,1618.23) .. controls (79,1618.23) and (77.56,1617.19) .. (77.56,1615.89) -- cycle ;
    %Shape: Ellipse [id:dp08863924891219843] 
    \draw  [draw opacity=0][fill={rgb, 255:red, 208; green, 2; blue, 27 }  ,fill opacity=0.62 ] (84.52,1609.06) .. controls (84.52,1607.77) and (85.96,1606.72) .. (87.73,1606.72) .. controls (89.51,1606.72) and (90.95,1607.77) .. (90.95,1609.06) .. controls (90.95,1610.35) and (89.51,1611.4) .. (87.73,1611.4) .. controls (85.96,1611.4) and (84.52,1610.35) .. (84.52,1609.06) -- cycle ;
    %Shape: Ellipse [id:dp49807154634681794] 
    \draw  [draw opacity=0][fill={rgb, 255:red, 208; green, 2; blue, 27 }  ,fill opacity=0.62 ] (91.21,1601.64) .. controls (91.21,1600.35) and (92.65,1599.3) .. (94.43,1599.3) .. controls (96.2,1599.3) and (97.64,1600.35) .. (97.64,1601.64) .. controls (97.64,1602.94) and (96.2,1603.98) .. (94.43,1603.98) .. controls (92.65,1603.98) and (91.21,1602.94) .. (91.21,1601.64) -- cycle ;
    %Shape: Ellipse [id:dp17062416794692736] 
    \draw  [draw opacity=0][fill={rgb, 255:red, 208; green, 2; blue, 27 }  ,fill opacity=0.62 ] (100.59,1619.41) .. controls (100.59,1618.11) and (102.03,1617.06) .. (103.8,1617.06) .. controls (105.57,1617.06) and (107.01,1618.11) .. (107.01,1619.41) .. controls (107.01,1620.7) and (105.57,1621.75) .. (103.8,1621.75) .. controls (102.03,1621.75) and (100.59,1620.7) .. (100.59,1619.41) -- cycle ;
    %Shape: Ellipse [id:dp05427293190477478] 
    \draw  [draw opacity=0][fill={rgb, 255:red, 189; green, 16; blue, 224 }  ,fill opacity=0.8 ] (152.54,1626.82) .. controls (152.54,1625.53) and (153.98,1624.48) .. (155.75,1624.48) .. controls (157.52,1624.48) and (158.96,1625.53) .. (158.96,1626.82) .. controls (158.96,1628.12) and (157.52,1629.16) .. (155.75,1629.16) .. controls (153.98,1629.16) and (152.54,1628.12) .. (152.54,1626.82) -- cycle ;
    %Shape: Ellipse [id:dp0565658994925915] 
    \draw  [draw opacity=0][fill={rgb, 255:red, 208; green, 2; blue, 27 }  ,fill opacity=0.62 ] (158.96,1616.67) .. controls (158.96,1615.38) and (160.4,1614.33) .. (162.18,1614.33) .. controls (163.95,1614.33) and (165.39,1615.38) .. (165.39,1616.67) .. controls (165.39,1617.97) and (163.95,1619.01) .. (162.18,1619.01) .. controls (160.4,1619.01) and (158.96,1617.97) .. (158.96,1616.67) -- cycle ;
    %Shape: Ellipse [id:dp5007110255270828] 
    \draw  [draw opacity=0][fill={rgb, 255:red, 189; green, 16; blue, 224 }  ,fill opacity=0.8 ] (57,1610.73) .. controls (57,1609.43) and (58.44,1608.38) .. (60.21,1608.38) .. controls (61.99,1608.38) and (63.43,1609.43) .. (63.43,1610.73) .. controls (63.43,1612.02) and (61.99,1613.07) .. (60.21,1613.07) .. controls (58.44,1613.07) and (57,1612.02) .. (57,1610.73) -- cycle ;
    %Shape: Ellipse [id:dp22598728144573377] 
    \draw  [draw opacity=0][fill={rgb, 255:red, 208; green, 2; blue, 27 }  ,fill opacity=0.62 ] (88.72,1595.59) .. controls (88.72,1594.3) and (90.16,1593.25) .. (91.93,1593.25) .. controls (93.71,1593.25) and (95.15,1594.3) .. (95.15,1595.59) .. controls (95.15,1596.88) and (93.71,1597.93) .. (91.93,1597.93) .. controls (90.16,1597.93) and (88.72,1596.88) .. (88.72,1595.59) -- cycle ;
    %Shape: Ellipse [id:dp14749486568088088] 
    \draw  [draw opacity=0][fill={rgb, 255:red, 189; green, 16; blue, 224 }  ,fill opacity=0.8 ] (93.54,1589.34) .. controls (93.54,1588.05) and (94.98,1587) .. (96.75,1587) .. controls (98.53,1587) and (99.97,1588.05) .. (99.97,1589.34) .. controls (99.97,1590.64) and (98.53,1591.69) .. (96.75,1591.69) .. controls (94.98,1591.69) and (93.54,1590.64) .. (93.54,1589.34) -- cycle ;
    %Shape: Polygon Curved [id:ds6643267525526769] 
    \draw  [color={rgb, 255:red, 74; green, 144; blue, 226 }  ,draw opacity=1 ][fill={rgb, 255:red, 74; green, 144; blue, 226 }  ,fill opacity=0.5 ] (27.21,1628.38) .. controls (30.38,1623.39) and (36.63,1621.99) .. (42.56,1622.18) .. controls (47.05,1622.33) and (51.36,1623.39) .. (53.99,1624.48) .. controls (60.1,1627.02) and (65.56,1626.63) .. (64.7,1632.29) .. controls (63.85,1637.95) and (56.88,1637.17) .. (48.64,1636.19) .. controls (40.39,1635.22) and (21.64,1637.17) .. (27.21,1628.38) -- cycle ;
    %Shape: Polygon Curved [id:ds9461514343962948] 
    \draw  [color={rgb, 255:red, 208; green, 2; blue, 27 }  ,draw opacity=1 ][fill={rgb, 255:red, 208; green, 2; blue, 27 }  ,fill opacity=0.5 ] (139.68,1569.82) .. controls (145.25,1561.04) and (148.01,1561.59) .. (145.04,1565.92) .. controls (142.07,1570.25) and (154.68,1563.87) .. (153.82,1569.53) .. controls (152.97,1575.19) and (169.35,1578.61) .. (161.11,1577.63) .. controls (152.86,1576.66) and (134.11,1578.61) .. (139.68,1569.82) -- cycle ;
    %Shape: Ellipse [id:dp06406072166611776] 
    \draw  [draw opacity=0][fill={rgb, 255:red, 208; green, 2; blue, 27 }  ,fill opacity=0.62 ] (71.13,1617.45) .. controls (71.13,1616.16) and (72.57,1615.11) .. (74.34,1615.11) .. controls (76.12,1615.11) and (77.56,1616.16) .. (77.56,1617.45) .. controls (77.56,1618.75) and (76.12,1619.8) .. (74.34,1619.8) .. controls (72.57,1619.8) and (71.13,1618.75) .. (71.13,1617.45) -- cycle ;
    %Shape: Ellipse [id:dp049221150011381054] 
    \draw  [draw opacity=0][fill={rgb, 255:red, 189; green, 16; blue, 224 }  ,fill opacity=0.8 ] (56.59,1624.48) .. controls (56.59,1623.19) and (58.03,1622.14) .. (59.8,1622.14) .. controls (61.58,1622.14) and (63.01,1623.19) .. (63.01,1624.48) .. controls (63.01,1625.77) and (61.58,1626.82) .. (59.8,1626.82) .. controls (58.03,1626.82) and (56.59,1625.77) .. (56.59,1624.48) -- cycle ;


    % Text Node
    \draw (68.95,1583.75) node  [font=\tiny,color={rgb, 255:red, 189; green, 16; blue, 224 }  ,opacity=1 ] [align=left] {$\displaystyle g_{5} =\{\omega _{21} \dotsc \}$};
    % Text Node
    \draw (82.63,1627.35) node  [font=\tiny,color={rgb, 255:red, 189; green, 16; blue, 224 }  ,opacity=1 ] [align=left] {$\displaystyle g_{2} =\{\omega _{5}\}$};
    % Text Node
    \draw (152.91,1586.86) node  [font=\tiny,color={rgb, 255:red, 189; green, 16; blue, 224 }  ,opacity=1 ] [align=left] {$\displaystyle ...$};
    % Text Node
    \draw (101.5,1575.93) node  [font=\tiny,color={rgb, 255:red, 189; green, 16; blue, 224 }  ,opacity=1 ] [align=left] {$\displaystyle ...$};
    % Text Node
    \draw (136.45,1636.35) node  [font=\tiny,color={rgb, 255:red, 189; green, 16; blue, 224 }  ,opacity=1 ] [align=left] {$\displaystyle g_{4} =\{\omega _{301} ,\omega _{302}\}$};
    % Text Node
    \draw (113.58,1612.35) node  [font=\tiny,color={rgb, 255:red, 189; green, 16; blue, 224 }  ,opacity=1 ] [align=left] {$\displaystyle g_{3} =\{\omega _{10}\}$};
    % Text Node
    \draw (55.11,1600.35) node  [font=\tiny,color={rgb, 255:red, 189; green, 16; blue, 224 }  ,opacity=1 ] [align=left] {$\displaystyle g_{1} =\{\omega _{1}\}$};
    % Text Node
    \draw (105.58,1568.35) node  [font=\tiny,color={rgb, 255:red, 202; green, 52; blue, 69 }  ,opacity=1 ] [align=left] {$\displaystyle g_{*} =\Omega _{goal}$};
    % Text Node
    \draw (73.43,1637.35) node  [font=\tiny,color={rgb, 255:red, 74; green, 144; blue, 226 }  ,opacity=1 ] [align=left] {$\displaystyle \Omega _{init}$};
    % Text Node
    \draw (32.91,1567.84) node  [font=\scriptsize] [align=left] {$\displaystyle \Omega $};
    % Text Node
    \draw (93.61,1653) node   [align=left] {{\tiny \textit{Une visualisation abstraite des}}};
    \draw (93.61,1662) node   [align=left] {{\tiny \textit{observations dans les trajectoires}}};

\end{tikzpicture}
      \end{column}

      \begin{column}{0.1\textwidth}
      \end{column}

      \begin{column}{0.4\textwidth}
        \centering
        


\tikzset{every picture/.style={line width=0.75pt}} %set default line width to 0.75pt        

\begin{tikzpicture}[x=0.75pt,y=0.75pt,yscale=-1,xscale=1]
%uncomment if require: \path (0,1974); %set diagram left start at 0, and has height of 1974

%Shape: Rectangle [id:dp5335676631264512] 
\draw  [fill={rgb, 255:red, 255; green, 255; blue, 255 }  ,fill opacity=1 ] (190,1560.11) -- (342.1,1560.11) -- (342.1,1646) -- (190,1646) -- cycle ;
%Straight Lines [id:da6623576988919416] 
\draw [color={rgb, 255:red, 208; green, 2; blue, 27 }  ,draw opacity=1 ]   (308.67,1572.84) -- (290.28,1579.49) -- (253.26,1594.41) -- (292.1,1604) -- (274.1,1612) -- (262.1,1616) -- (252.58,1614.77) -- (246.1,1604) -- (240.1,1604) -- (236.1,1606) -- (236.51,1610.86) -- (220.44,1610.86) -- (223,1612.73) -- (215.09,1614.77) -- (217.85,1618.79) -- (204.38,1630.38) ;
\draw [shift={(311.49,1571.82)}, rotate = 160.12] [fill={rgb, 255:red, 208; green, 2; blue, 27 }  ,fill opacity=1 ][line width=0.08]  [draw opacity=0] (3.57,-1.72) -- (0,0) -- (3.57,1.72) -- cycle    ;
%Straight Lines [id:da5424854363807742] 
\draw [color={rgb, 255:red, 80; green, 227; blue, 194 }  ,draw opacity=1 ]   (309.47,1570.13) -- (300.78,1579.63) -- (279.36,1579.63) -- (279.36,1587.44) -- (252.58,1595.25) -- (250.1,1598) -- (248.1,1600) -- (247.22,1603.06) -- (252.58,1610.86) -- (247.22,1610.86) -- (242.1,1608) -- (242.1,1610) -- (233.94,1610.94) -- (231.16,1616.72) -- (209.73,1605.01) -- (225.8,1618.67) -- (209.73,1610.86) -- (215.09,1618.67) -- (209.73,1634.29) ;
\draw [shift={(311.49,1567.92)}, rotate = 132.45] [fill={rgb, 255:red, 80; green, 227; blue, 194 }  ,fill opacity=1 ][line width=0.08]  [draw opacity=0] (3.57,-1.72) -- (0,0) -- (3.57,1.72) -- cycle    ;
%Straight Lines [id:da21186841526109945] 
\draw [color={rgb, 255:red, 248; green, 231; blue, 28 }  ,draw opacity=1 ]   (319.23,1576.14) -- (292.83,1579.75) -- (290.07,1583.54) -- (271.41,1587.56) -- (257.93,1595.25) -- (280.1,1592) -- (284.1,1594) -- (290.1,1604) -- (257.93,1614.77) -- (257.93,1618.67) -- (247.22,1618.67) -- (225.99,1611.06) -- (223.21,1616.84) -- (207.14,1618.79) -- (201.78,1634.41) ;
\draw [shift={(322.2,1575.73)}, rotate = 172.2] [fill={rgb, 255:red, 248; green, 231; blue, 28 }  ,fill opacity=1 ][line width=0.08]  [draw opacity=0] (3.57,-1.72) -- (0,0) -- (3.57,1.72) -- cycle    ;
%Straight Lines [id:da6313290732282947] 
\draw [color={rgb, 255:red, 144; green, 19; blue, 254 }  ,draw opacity=1 ]   (326.23,1579.39) -- (331.84,1580.41) -- (327.56,1575.73) -- (323.27,1572.61) -- (330.86,1575.73) -- (336.13,1580.41) -- (327.56,1585.1) -- (332.91,1591.34) -- (324.1,1600) -- (330.1,1606) -- (312.1,1636) -- (320.1,1638) -- (306.1,1644) -- (306.1,1606) -- (300.1,1614) -- (311.49,1626.48) -- (300.78,1626.48) -- (294.99,1625.07) -- (291.92,1624.33) -- (287.57,1623.27) -- (284.71,1622.58) -- (273.96,1621.71) -- (265.43,1621.01) -- (261.23,1620.25) -- (250.1,1622) -- (241.87,1618.67) -- (236.51,1622.58) -- (225.8,1626.48) -- (225.8,1634.29) ;
\draw [shift={(323.27,1578.85)}, rotate = 10.33] [fill={rgb, 255:red, 144; green, 19; blue, 254 }  ,fill opacity=1 ][line width=0.08]  [draw opacity=0] (3.57,-1.72) -- (0,0) -- (3.57,1.72) -- cycle    ;
%Straight Lines [id:da1305524961942589] 
\draw [color={rgb, 255:red, 65; green, 117; blue, 5 }  ,draw opacity=1 ]   (325.15,1580.17) -- (330.77,1581.19) -- (326.49,1576.51) -- (322.2,1573.39) -- (329.79,1576.51) -- (335.06,1581.19) -- (327.56,1589.78) -- (331.84,1592.13) -- (329.7,1603.84) -- (316.85,1599.15) -- (327.56,1605.4) -- (314.1,1636) -- (318.1,1640) -- (310.1,1642) -- (304.1,1608) -- (298.1,1612) -- (301.85,1624.14) -- (289,1624.14) -- (282.57,1624.14) -- (276.14,1622.58) -- (274,1625.7) -- (269.72,1622.58) -- (271.86,1627.26) -- (260.16,1621.03) -- (251.51,1619.45) -- (240.8,1619.45) -- (235.44,1623.36) -- (224.73,1627.26) -- (224.73,1635.07) ;
\draw [shift={(322.2,1579.63)}, rotate = 10.33] [fill={rgb, 255:red, 65; green, 117; blue, 5 }  ,fill opacity=1 ][line width=0.08]  [draw opacity=0] (3.57,-1.72) -- (0,0) -- (3.57,1.72) -- cycle    ;
%Shape: Polygon Curved [id:ds29559681347985167] 
\draw  [color={rgb, 255:red, 184; green, 233; blue, 134 }  ,draw opacity=0 ][fill={rgb, 255:red, 74; green, 144; blue, 226 }  ,fill opacity=0.75 ] (203.31,1627.26) .. controls (208.88,1618.48) and (203.46,1612.19) .. (210,1608) .. controls (216.54,1603.81) and (229.5,1600.53) .. (248,1604) .. controls (266.5,1607.47) and (269.83,1605.44) .. (280,1604) .. controls (290.17,1602.56) and (250.44,1601.87) .. (252,1594) .. controls (253.56,1586.13) and (258.5,1591.77) .. (262,1588) .. controls (265.5,1584.23) and (314.2,1570.33) .. (316,1570) .. controls (317.8,1569.67) and (320.91,1572.43) .. (314,1576) .. controls (307.09,1579.57) and (294.2,1581.95) .. (288,1584) .. controls (281.8,1586.05) and (277.13,1589.32) .. (276,1590) .. controls (274.87,1590.68) and (280.1,1589.85) .. (284,1592) .. controls (287.9,1594.15) and (295.82,1601.53) .. (296,1602) .. controls (296.18,1602.47) and (264.78,1615.69) .. (264,1616) .. controls (263.22,1616.31) and (249.54,1612.38) .. (244,1612) .. controls (238.46,1611.62) and (217.32,1621.29) .. (214,1626) .. controls (210.68,1630.71) and (210.87,1632.35) .. (210,1636) .. controls (209.13,1639.65) and (197.74,1636.04) .. (203.31,1627.26) -- cycle ;
%Shape: Polygon Curved [id:ds2928272635642186] 
\draw  [color={rgb, 255:red, 208; green, 2; blue, 27 }  ,draw opacity=0 ][fill={rgb, 255:red, 208; green, 2; blue, 27 }  ,fill opacity=0.5 ] (304,1644) .. controls (302.69,1641.5) and (306.85,1644.5) .. (304,1634) .. controls (301.15,1623.5) and (270.08,1628.81) .. (266,1628) .. controls (261.92,1627.19) and (254.9,1623.42) .. (244,1624) .. controls (233.1,1624.58) and (230.1,1635.78) .. (226,1638) .. controls (221.9,1640.22) and (223.11,1630.38) .. (222,1630) .. controls (220.89,1629.62) and (222.67,1626.54) .. (226,1624) .. controls (229.33,1621.46) and (236.24,1618.61) .. (236,1618) .. controls (235.76,1617.39) and (243.31,1617.27) .. (250,1618) .. controls (256.69,1618.73) and (262.53,1620.62) .. (264,1620) .. controls (265.47,1619.38) and (296.11,1616.11) .. (296,1614) .. controls (295.89,1611.89) and (301.99,1602.86) .. (304,1602) .. controls (306.01,1601.14) and (317.99,1622.88) .. (320,1616) .. controls (322.01,1609.12) and (321.73,1568.73) .. (326,1570) .. controls (330.27,1571.27) and (339.87,1563.9) .. (338,1584) .. controls (336.13,1604.1) and (324.06,1639.29) .. (320,1642) .. controls (315.94,1644.71) and (305.31,1646.5) .. (304,1644) -- cycle ;


% Text Node
\draw (268.2,1637.75) node  [font=\tiny,color={rgb, 255:red, 189; green, 16; blue, 224 }  ,opacity=1 ] [align=left] {$\displaystyle \rho _{2} =\{( \omega _{11} ,a_{11}) \dotsc \}$};
% Text Node
\draw (231.2,1581.75) node  [font=\tiny,color={rgb, 255:red, 189; green, 16; blue, 224 }  ,opacity=1 ] [align=left] {$\displaystyle \rho _{1} =\{( \omega _{21} ,a_{21}) \dotsc \}$};
% Text Node
\draw (209.5,1570) node  [font=\scriptsize] [align=left] {$\displaystyle \Omega \times A$};
% Text Node
\draw (267.61,1655) node   [align=left] {{\tiny \textit{An abstract visualization of}}};
\draw (267.61,1665) node   [align=left] {{\tiny \textit{transitions in trajectories}}};

\end{tikzpicture}
      \end{column}
    \end{columns}
  \end{center}

  \begin{tikzpicture}[remember picture, overlay]
    \node[anchor=north west, text=black]
    at ([xshift=5.8cm,yshift=-4.4cm]current page.north west) {\small
      \begin{minipage}{0.3\linewidth}
        {\small \hspace{1.3cm} \textit{\textbf{Idées\dots}}
          \begin{itemize}
            \item  \textit{Trajectoires comme vecteurs;}
            \item  \textit{Distance: Smith-Waterman, LCS, Euclidean\dots;}
            \item  \textit{Clustering + Centroides \\ \ \ \ $\rightarrow$ roles/objectifs.}
          \end{itemize}}
      \end{minipage}

    };
  \end{tikzpicture}

  \vspace{-0.5cm}
  {
    \small

    \textit{Ainsi:}
    \begin{itemize}
      \item \textit{Adéquation organisationnelle structurelle (SOF) = normalized deviation score} ($clusters_{trans}$, $centroids_{trans}$)
      \item \textit{Adéquation organisationnelle fonctionnelle (FOF) = normalized deviation score} ($clusters_{obs}$, $centroids_{obs}$)
      \item \textit{Adéquation organisationnelle = $\frac{1}{2} \times (SOF + FOF)$}
    \end{itemize}
  }
\end{frame}

\begin{frame}{Le cadre MOISE+MARL}{Présentation de la méthode TEMM}
  


\tikzset{every picture/.style={line width=0.75pt}} %set default line width to 0.75pt        

\begin{tikzpicture}[x=0.75pt,y=0.75pt,yscale=-1,xscale=1]
    %uncomment if require: \path (0,3196); %set diagram left start at 0, and has height of 3196

    %Shape: Rectangle [id:dp6446774683746603] 
    \draw  [fill={rgb, 255:red, 184; green, 233; blue, 134 }  ,fill opacity=1 ] (184,2054) -- (554,2054) -- (554,2110) -- (184,2110) -- cycle ;
    %Shape: Rectangle [id:dp11954381835057803] 
    \draw  [fill={rgb, 255:red, 80; green, 227; blue, 194 }  ,fill opacity=1 ] (214,2062) -- (330,2062) -- (330,2102) -- (214,2102) -- cycle ;
    %Shape: Rectangle [id:dp16354412692751719] 
    \draw  [fill={rgb, 255:red, 245; green, 166; blue, 35 }  ,fill opacity=1 ] (252.76,2083.89) -- (280,2083.89) -- (280,2095.6) -- (252.76,2095.6) -- cycle ;
    %Shape: Rectangle [id:dp2654297895337838] 
    \draw  [fill={rgb, 255:red, 139; green, 87; blue, 42 }  ,fill opacity=1 ] (298.76,2083.87) -- (326,2083.87) -- (326,2095.59) -- (298.76,2095.59) -- cycle ;
    %Shape: Rectangle [id:dp8985818084992432] 
    \draw  [fill={rgb, 255:red, 126; green, 211; blue, 33 }  ,fill opacity=1 ] (218.76,2084.16) -- (246,2084.16) -- (246,2095.87) -- (218.76,2095.87) -- cycle ;
    %Shape: Rectangle [id:dp48712472547976826] 
    \draw  [fill={rgb, 255:red, 255; green, 255; blue, 255 }  ,fill opacity=1 ] (242,2131.08) .. controls (242,2128.32) and (244.24,2126.08) .. (247,2126.08) -- (301,2126.08) .. controls (303.76,2126.08) and (306,2128.32) .. (306,2131.08) -- (306,2143) .. controls (306,2145.76) and (303.76,2148) .. (301,2148) -- (247,2148) .. controls (244.24,2148) and (242,2145.76) .. (242,2143) -- cycle ;

    %Shape: Rectangle [id:dp9481830819072239] 
    \draw  [fill={rgb, 255:red, 184; green, 233; blue, 134 }  ,fill opacity=1 ] (220,2164) -- (330,2164) -- (330,2229) -- (220,2229) -- cycle ;
    %Shape: Rectangle [id:dp41481109128799265] 
    \draw  [fill={rgb, 255:red, 80; green, 227; blue, 194 }  ,fill opacity=1 ] (226,2169) -- (265,2169) -- (265,2224) -- (226,2224) -- cycle ;
    %Shape: Rectangle [id:dp36210081179115716] 
    \draw  [fill={rgb, 255:red, 139; green, 87; blue, 42 }  ,fill opacity=1 ] (232.76,2209) -- (260,2209) -- (260,2220.71) -- (232.76,2220.71) -- cycle ;
    %Shape: Rectangle [id:dp9574972109935381] 
    \draw  [fill={rgb, 255:red, 126; green, 211; blue, 33 }  ,fill opacity=1 ] (232.48,2184.62) -- (259.72,2184.62) -- (259.72,2196.33) -- (232.48,2196.33) -- cycle ;
    %Shape: Rectangle [id:dp6763939878516604] 
    \draw  [fill={rgb, 255:red, 80; green, 227; blue, 194 }  ,fill opacity=1 ] (285,2169) -- (324,2169) -- (324,2224) -- (285,2224) -- cycle ;
    %Shape: Rectangle [id:dp7254457331732914] 
    \draw  [fill={rgb, 255:red, 139; green, 87; blue, 42 }  ,fill opacity=1 ] (291.76,2209) -- (319,2209) -- (319,2220.71) -- (291.76,2220.71) -- cycle ;
    %Shape: Rectangle [id:dp09098875701220999] 
    \draw  [fill={rgb, 255:red, 126; green, 211; blue, 33 }  ,fill opacity=1 ] (291.48,2184.62) -- (318.72,2184.62) -- (318.72,2196.33) -- (291.48,2196.33) -- cycle ;
    %Straight Lines [id:da310649499986848] 
    \draw    (220.53,2319) -- (220.53,2303.39) -- (242.33,2303.39) -- (242.33,2319) ;
    %Straight Lines [id:da4779343766066969] 
    \draw    (231.43,2303.39) -- (231.43,2287.77) -- (264.12,2287.77) -- (264.12,2319) ;
    %Straight Lines [id:da5756029666044655] 
    \draw    (242.33,2272.16) -- (242.33,2264.35) -- (318.61,2264.35) -- (318.61,2272.16) ;
    %Straight Lines [id:da000745643490450365] 
    \draw    (307.71,2319) -- (307.71,2303.39) -- (329.51,2303.39) -- (329.51,2319) ;
    %Shape: Ellipse [id:dp7156214883304655] 
    \draw  [line width=2.25]  (220.53,2319) .. controls (220.53,2318.79) and (220.78,2318.61) .. (221.08,2318.61) .. controls (221.38,2318.61) and (221.62,2318.79) .. (221.62,2319) .. controls (221.62,2319.22) and (221.38,2319.39) .. (221.08,2319.39) .. controls (220.78,2319.39) and (220.53,2319.22) .. (220.53,2319) -- cycle ;
    %Shape: Ellipse [id:dp3858567279639801] 
    \draw  [line width=2.25]  (242.33,2319) .. controls (242.33,2318.79) and (242.57,2318.61) .. (242.87,2318.61) .. controls (243.17,2318.61) and (243.42,2318.79) .. (243.42,2319) .. controls (243.42,2319.22) and (243.17,2319.39) .. (242.87,2319.39) .. controls (242.57,2319.39) and (242.33,2319.22) .. (242.33,2319) -- cycle ;
    %Shape: Ellipse [id:dp06345412404985873] 
    \draw  [line width=2.25]  (264.12,2319) .. controls (264.12,2318.79) and (264.37,2318.61) .. (264.67,2318.61) .. controls (264.97,2318.61) and (265.21,2318.79) .. (265.21,2319) .. controls (265.21,2319.22) and (264.97,2319.39) .. (264.67,2319.39) .. controls (264.37,2319.39) and (264.12,2319.22) .. (264.12,2319) -- cycle ;
    %Shape: Ellipse [id:dp23438755375160103] 
    \draw  [line width=2.25]  (307.17,2319.39) .. controls (307.17,2319.18) and (307.41,2319) .. (307.71,2319) .. controls (308.01,2319) and (308.26,2319.18) .. (308.26,2319.39) .. controls (308.26,2319.61) and (308.01,2319.78) .. (307.71,2319.78) .. controls (307.41,2319.78) and (307.17,2319.61) .. (307.17,2319.39) -- cycle ;
    %Shape: Ellipse [id:dp9125729527443308] 
    \draw  [line width=2.25]  (329.51,2319) .. controls (329.51,2318.79) and (329.75,2318.61) .. (330.05,2318.61) .. controls (330.35,2318.61) and (330.6,2318.79) .. (330.6,2319) .. controls (330.6,2319.22) and (330.35,2319.39) .. (330.05,2319.39) .. controls (329.75,2319.39) and (329.51,2319.22) .. (329.51,2319) -- cycle ;
    %Shape: Ellipse [id:dp9368497404716122] 
    \draw  [line width=2.25]  (231.43,2303.39) .. controls (231.43,2303.17) and (231.67,2303) .. (231.97,2303) .. controls (232.27,2303) and (232.52,2303.17) .. (232.52,2303.39) .. controls (232.52,2303.6) and (232.27,2303.78) .. (231.97,2303.78) .. controls (231.67,2303.78) and (231.43,2303.6) .. (231.43,2303.39) -- cycle ;
    %Shape: Ellipse [id:dp9082840601009624] 
    \draw  [line width=2.25]  (242.33,2288.16) .. controls (242.33,2287.95) and (242.57,2287.77) .. (242.87,2287.77) .. controls (243.17,2287.77) and (243.42,2287.95) .. (243.42,2288.16) .. controls (243.42,2288.38) and (243.17,2288.55) .. (242.87,2288.55) .. controls (242.57,2288.55) and (242.33,2288.38) .. (242.33,2288.16) -- cycle ;
    %Shape: Ellipse [id:dp9572066005117809] 
    \draw  [line width=2.25]  (285.37,2264.74) .. controls (285.37,2264.52) and (285.62,2264.35) .. (285.92,2264.35) .. controls (286.22,2264.35) and (286.46,2264.52) .. (286.46,2264.74) .. controls (286.46,2264.95) and (286.22,2265.13) .. (285.92,2265.13) .. controls (285.62,2265.13) and (285.37,2264.95) .. (285.37,2264.74) -- cycle ;
    %Shape: Ellipse [id:dp739566639462307] 
    \draw  [line width=2.25]  (318.06,2303.78) .. controls (318.06,2303.56) and (318.31,2303.39) .. (318.61,2303.39) .. controls (318.91,2303.39) and (319.15,2303.56) .. (319.15,2303.78) .. controls (319.15,2303.99) and (318.91,2304.17) .. (318.61,2304.17) .. controls (318.31,2304.17) and (318.06,2303.99) .. (318.06,2303.78) -- cycle ;
    %Straight Lines [id:da36201621077359325] 
    \draw  [dash pattern={on 0.84pt off 2.51pt}]  (242.33,2272.16) -- (242.33,2287.77) ;
    %Straight Lines [id:da3740526131455012] 
    \draw  [dash pattern={on 0.84pt off 2.51pt}]  (318.61,2272.16) -- (318.61,2303.39) ;
    %Straight Lines [id:da33114151268319136] 
    \draw  [dash pattern={on 0.84pt off 2.51pt}]  (285.92,2264.35) -- (285.92,2311.19) ;
    %Straight Lines [id:da9611554132027019] 
    \draw    (407.49,2320.66) -- (407.49,2305.04) -- (429.28,2305.04) -- (429.28,2320.66) ;
    %Straight Lines [id:da5452608062724974] 
    \draw    (418.38,2305.04) -- (418.38,2289.42) -- (451.08,2289.42) -- (451.08,2320.66) ;
    %Straight Lines [id:da5268642881289671] 
    \draw    (429.28,2273.81) -- (429.28,2266) -- (505.56,2266) -- (505.56,2273.81) ;
    %Straight Lines [id:da05744469340801017] 
    \draw    (494.67,2320.66) -- (494.67,2305.04) -- (516.46,2305.04) -- (516.46,2320.66) ;
    %Shape: Ellipse [id:dp616447935630148] 
    \draw  [line width=2.25]  (407.49,2320.66) .. controls (407.49,2320.44) and (407.73,2320.27) .. (408.03,2320.27) .. controls (408.33,2320.27) and (408.58,2320.44) .. (408.58,2320.66) .. controls (408.58,2320.87) and (408.33,2321.05) .. (408.03,2321.05) .. controls (407.73,2321.05) and (407.49,2320.87) .. (407.49,2320.66) -- cycle ;
    %Shape: Ellipse [id:dp3777096214669222] 
    \draw  [line width=2.25]  (429.28,2320.66) .. controls (429.28,2320.44) and (429.52,2320.27) .. (429.83,2320.27) .. controls (430.13,2320.27) and (430.37,2320.44) .. (430.37,2320.66) .. controls (430.37,2320.87) and (430.13,2321.05) .. (429.83,2321.05) .. controls (429.52,2321.05) and (429.28,2320.87) .. (429.28,2320.66) -- cycle ;
    %Shape: Ellipse [id:dp8138542385543501] 
    \draw  [line width=2.25]  (451.08,2320.66) .. controls (451.08,2320.44) and (451.32,2320.27) .. (451.62,2320.27) .. controls (451.92,2320.27) and (452.17,2320.44) .. (452.17,2320.66) .. controls (452.17,2320.87) and (451.92,2321.05) .. (451.62,2321.05) .. controls (451.32,2321.05) and (451.08,2320.87) .. (451.08,2320.66) -- cycle ;
    %Shape: Ellipse [id:dp322481737659716] 
    \draw  [line width=2.25]  (494.12,2321.05) .. controls (494.12,2320.83) and (494.36,2320.66) .. (494.67,2320.66) .. controls (494.97,2320.66) and (495.21,2320.83) .. (495.21,2321.05) .. controls (495.21,2321.26) and (494.97,2321.44) .. (494.67,2321.44) .. controls (494.36,2321.44) and (494.12,2321.26) .. (494.12,2321.05) -- cycle ;
    %Shape: Ellipse [id:dp7676517098779699] 
    \draw  [line width=2.25]  (516.46,2320.66) .. controls (516.46,2320.44) and (516.7,2320.27) .. (517.01,2320.27) .. controls (517.31,2320.27) and (517.55,2320.44) .. (517.55,2320.66) .. controls (517.55,2320.87) and (517.31,2321.05) .. (517.01,2321.05) .. controls (516.7,2321.05) and (516.46,2320.87) .. (516.46,2320.66) -- cycle ;
    %Shape: Ellipse [id:dp1585224562411227] 
    \draw  [line width=2.25]  (418.38,2305.04) .. controls (418.38,2304.82) and (418.63,2304.65) .. (418.93,2304.65) .. controls (419.23,2304.65) and (419.47,2304.82) .. (419.47,2305.04) .. controls (419.47,2305.26) and (419.23,2305.43) .. (418.93,2305.43) .. controls (418.63,2305.43) and (418.38,2305.26) .. (418.38,2305.04) -- cycle ;
    %Shape: Ellipse [id:dp02834033002913927] 
    \draw  [line width=2.25]  (429.28,2289.81) .. controls (429.28,2289.6) and (429.52,2289.42) .. (429.83,2289.42) .. controls (430.13,2289.42) and (430.37,2289.6) .. (430.37,2289.81) .. controls (430.37,2290.03) and (430.13,2290.2) .. (429.83,2290.2) .. controls (429.52,2290.2) and (429.28,2290.03) .. (429.28,2289.81) -- cycle ;
    %Shape: Ellipse [id:dp9471335251053217] 
    \draw  [line width=2.25]  (472.33,2266.39) .. controls (472.33,2266.17) and (472.57,2266) .. (472.87,2266) .. controls (473.17,2266) and (473.42,2266.17) .. (473.42,2266.39) .. controls (473.42,2266.61) and (473.17,2266.78) .. (472.87,2266.78) .. controls (472.57,2266.78) and (472.33,2266.61) .. (472.33,2266.39) -- cycle ;
    %Shape: Ellipse [id:dp8036746049983543] 
    \draw  [line width=2.25]  (505.02,2305.43) .. controls (505.02,2305.21) and (505.26,2305.04) .. (505.56,2305.04) .. controls (505.86,2305.04) and (506.11,2305.21) .. (506.11,2305.43) .. controls (506.11,2305.65) and (505.86,2305.82) .. (505.56,2305.82) .. controls (505.26,2305.82) and (505.02,2305.65) .. (505.02,2305.43) -- cycle ;
    %Straight Lines [id:da4862783270480875] 
    \draw  [dash pattern={on 0.84pt off 2.51pt}]  (429.28,2273.81) -- (429.28,2289.42) ;
    %Straight Lines [id:da33705102258229946] 
    \draw  [dash pattern={on 0.84pt off 2.51pt}]  (505.56,2273.81) -- (505.56,2305.04) ;
    %Straight Lines [id:da7632784737841659] 
    \draw  [dash pattern={on 0.84pt off 2.51pt}]  (472.87,2266) -- (472.87,2312.85) ;
    %Shape: Rectangle [id:dp8486981747751949] 
    \draw  [fill={rgb, 255:red, 184; green, 233; blue, 134 }  ,fill opacity=1 ] (404,2164) -- (514,2164) -- (514,2229) -- (404,2229) -- cycle ;
    %Shape: Rectangle [id:dp8662345148113237] 
    \draw  [fill={rgb, 255:red, 80; green, 227; blue, 194 }  ,fill opacity=1 ] (410,2169) -- (449,2169) -- (449,2224) -- (410,2224) -- cycle ;
    %Shape: Rectangle [id:dp7401789100724447] 
    \draw  [fill={rgb, 255:red, 139; green, 87; blue, 42 }  ,fill opacity=1 ] (416.76,2209) -- (444,2209) -- (444,2220.71) -- (416.76,2220.71) -- cycle ;
    %Shape: Rectangle [id:dp732580883063841] 
    \draw  [fill={rgb, 255:red, 126; green, 211; blue, 33 }  ,fill opacity=1 ] (416.48,2184.62) -- (443.72,2184.62) -- (443.72,2196.33) -- (416.48,2196.33) -- cycle ;
    %Shape: Rectangle [id:dp9235809941782823] 
    \draw  [fill={rgb, 255:red, 80; green, 227; blue, 194 }  ,fill opacity=1 ] (469,2169) -- (508,2169) -- (508,2224) -- (469,2224) -- cycle ;
    %Shape: Rectangle [id:dp6899402027382165] 
    \draw  [fill={rgb, 255:red, 139; green, 87; blue, 42 }  ,fill opacity=1 ] (475.76,2209) -- (503,2209) -- (503,2220.71) -- (475.76,2220.71) -- cycle ;
    %Shape: Rectangle [id:dp17139392675097032] 
    \draw  [fill={rgb, 255:red, 126; green, 211; blue, 33 }  ,fill opacity=1 ] (475.48,2184.62) -- (502.72,2184.62) -- (502.72,2196.33) -- (475.48,2196.33) -- cycle ;
    %Straight Lines [id:da5226422691270944] 
    \draw [color={rgb, 255:red, 65; green, 117; blue, 5 }  ,draw opacity=1 ][line width=2.25]    (231.77,2320.33) -- (231.81,2303.78) ;
    %Shape: Ellipse [id:dp5047419290586512] 
    \draw  [color={rgb, 255:red, 65; green, 117; blue, 5 }  ,draw opacity=1 ][line width=2.25]  (230.92,2319.42) .. controls (230.92,2318.91) and (231.3,2318.5) .. (231.77,2318.5) .. controls (232.24,2318.5) and (232.63,2318.91) .. (232.63,2319.42) .. controls (232.63,2319.92) and (232.24,2320.33) .. (231.77,2320.33) .. controls (231.3,2320.33) and (230.92,2319.92) .. (230.92,2319.42) -- cycle ;
    %Straight Lines [id:da40145914360461865] 
    \draw [color={rgb, 255:red, 65; green, 117; blue, 5 }  ,draw opacity=1 ][line width=2.25]    (251.94,2319) -- (251.94,2288.17) ;
    %Shape: Ellipse [id:dp8858416104743969] 
    \draw  [color={rgb, 255:red, 65; green, 117; blue, 5 }  ,draw opacity=1 ][line width=2.25]  (251.08,2319.92) .. controls (251.08,2319.41) and (251.47,2319) .. (251.94,2319) .. controls (252.41,2319) and (252.79,2319.41) .. (252.79,2319.92) .. controls (252.79,2320.42) and (252.41,2320.83) .. (251.94,2320.83) .. controls (251.47,2320.83) and (251.08,2320.42) .. (251.08,2319.92) -- cycle ;
    %Straight Lines [id:da6353184164708716] 
    \draw [color={rgb, 255:red, 65; green, 117; blue, 5 }  ,draw opacity=1 ][line width=2.25]    (318.77,2320.67) -- (318.81,2304.11) ;
    %Shape: Ellipse [id:dp9629746335106062] 
    \draw  [color={rgb, 255:red, 65; green, 117; blue, 5 }  ,draw opacity=1 ][line width=2.25]  (317.92,2319.75) .. controls (317.92,2319.24) and (318.3,2318.83) .. (318.77,2318.83) .. controls (319.24,2318.83) and (319.63,2319.24) .. (319.63,2319.75) .. controls (319.63,2320.26) and (319.24,2320.67) .. (318.77,2320.67) .. controls (318.3,2320.67) and (317.92,2320.26) .. (317.92,2319.75) -- cycle ;
    %Straight Lines [id:da7403569117479297] 
    \draw [color={rgb, 255:red, 65; green, 117; blue, 5 }  ,draw opacity=1 ][line width=2.25]    (418.7,2321.94) -- (418.74,2305.38) ;
    %Shape: Ellipse [id:dp9693987957556999] 
    \draw  [color={rgb, 255:red, 65; green, 117; blue, 5 }  ,draw opacity=1 ][line width=2.25]  (417.84,2321.02) .. controls (417.84,2320.52) and (418.23,2320.11) .. (418.7,2320.11) .. controls (419.17,2320.11) and (419.56,2320.52) .. (419.56,2321.02) .. controls (419.56,2321.53) and (419.17,2321.94) .. (418.7,2321.94) .. controls (418.23,2321.94) and (417.84,2321.53) .. (417.84,2321.02) -- cycle ;
    %Straight Lines [id:da0840318719732559] 
    \draw [color={rgb, 255:red, 65; green, 117; blue, 5 }  ,draw opacity=1 ][line width=2.25]    (438.87,2320.61) -- (438.87,2289.77) ;
    %Shape: Ellipse [id:dp38103443949256577] 
    \draw  [color={rgb, 255:red, 65; green, 117; blue, 5 }  ,draw opacity=1 ][line width=2.25]  (438.01,2321.52) .. controls (438.01,2321.02) and (438.39,2320.61) .. (438.87,2320.61) .. controls (439.34,2320.61) and (439.72,2321.02) .. (439.72,2321.52) .. controls (439.72,2322.03) and (439.34,2322.44) .. (438.87,2322.44) .. controls (438.39,2322.44) and (438.01,2322.03) .. (438.01,2321.52) -- cycle ;
    %Straight Lines [id:da5025807566153033] 
    \draw [color={rgb, 255:red, 65; green, 117; blue, 5 }  ,draw opacity=1 ][line width=2.25]    (505.7,2322.27) -- (505.74,2305.72) ;
    %Shape: Ellipse [id:dp2054322761052696] 
    \draw  [color={rgb, 255:red, 65; green, 117; blue, 5 }  ,draw opacity=1 ][line width=2.25]  (504.84,2321.36) .. controls (504.84,2320.85) and (505.23,2320.44) .. (505.7,2320.44) .. controls (506.17,2320.44) and (506.56,2320.85) .. (506.56,2321.36) .. controls (506.56,2321.86) and (506.17,2322.27) .. (505.7,2322.27) .. controls (505.23,2322.27) and (504.84,2321.86) .. (504.84,2321.36) -- cycle ;
    %Shape: Rectangle [id:dp18036074660325163] 
    \draw   (222.17,2364) -- (284,2364) -- (284,2460) -- (222.17,2460) -- cycle ;
    %Shape: Rectangle [id:dp3757874344168871] 
    \draw   (450,2364) -- (511.83,2364) -- (511.83,2458.97) -- (450,2458.97) -- cycle ;
    %Straight Lines [id:da4807858714110206] 
    \draw    (514.51,2371.91) -- (517.2,2371.91) -- (517.2,2403.57) -- (522,2412) -- (517.2,2423.36) -- (517.2,2451.06) -- (514.51,2451.06) ;
    %Shape: Rectangle [id:dp6563270106594438] 
    \draw   (288,2412) -- (364,2412) -- (364,2460) -- (288,2460) -- cycle ;

    %Shape: Rectangle [id:dp3825654404747878] 
    \draw  [fill={rgb, 255:red, 80; green, 227; blue, 194 }  ,fill opacity=1 ] (406,2062) -- (522,2062) -- (522,2102) -- (406,2102) -- cycle ;
    %Shape: Rectangle [id:dp973718882488827] 
    \draw  [fill={rgb, 255:red, 245; green, 166; blue, 35 }  ,fill opacity=1 ] (444.76,2083.89) -- (472,2083.89) -- (472,2095.6) -- (444.76,2095.6) -- cycle ;
    %Shape: Rectangle [id:dp0248437668015975] 
    \draw  [fill={rgb, 255:red, 139; green, 87; blue, 42 }  ,fill opacity=1 ] (490.76,2083.87) -- (518,2083.87) -- (518,2095.59) -- (490.76,2095.59) -- cycle ;
    %Shape: Rectangle [id:dp3173994637780163] 
    \draw  [fill={rgb, 255:red, 126; green, 211; blue, 33 }  ,fill opacity=1 ] (410.76,2084.16) -- (438,2084.16) -- (438,2095.87) -- (410.76,2095.87) -- cycle ;
    %Straight Lines [id:da44145788537292463] 
    \draw    (274,2110) -- (274,2124) ;
    \draw [shift={(274,2126)}, rotate = 270] [color={rgb, 255:red, 0; green, 0; blue, 0 }  ][line width=0.75]    (6.56,-1.97) .. controls (4.17,-0.84) and (1.99,-0.18) .. (0,0) .. controls (1.99,0.18) and (4.17,0.84) .. (6.56,1.97)   ;
    %Straight Lines [id:da006646172117977911] 
    \draw    (274,2148) -- (274,2162) ;
    \draw [shift={(274,2164)}, rotate = 270] [color={rgb, 255:red, 0; green, 0; blue, 0 }  ][line width=0.75]    (6.56,-1.97) .. controls (4.17,-0.84) and (1.99,-0.18) .. (0,0) .. controls (1.99,0.18) and (4.17,0.84) .. (6.56,1.97)   ;
    %Shape: Rectangle [id:dp5947531945722148] 
    \draw  [fill={rgb, 255:red, 255; green, 255; blue, 255 }  ,fill opacity=1 ] (418,2131) .. controls (418,2128.24) and (420.24,2126) .. (423,2126) -- (499,2126) .. controls (501.76,2126) and (504,2128.24) .. (504,2131) -- (504,2142.92) .. controls (504,2145.68) and (501.76,2147.92) .. (499,2147.92) -- (423,2147.92) .. controls (420.24,2147.92) and (418,2145.68) .. (418,2142.92) -- cycle ;

    %Straight Lines [id:da1469916019080323] 
    \draw    (460,2110) -- (460,2124) ;
    \draw [shift={(460,2126)}, rotate = 270] [color={rgb, 255:red, 0; green, 0; blue, 0 }  ][line width=0.75]    (6.56,-1.97) .. controls (4.17,-0.84) and (1.99,-0.18) .. (0,0) .. controls (1.99,0.18) and (4.17,0.84) .. (6.56,1.97)   ;
    %Straight Lines [id:da48571688709823413] 
    \draw    (460,2148) -- (460,2162) ;
    \draw [shift={(460,2164)}, rotate = 270] [color={rgb, 255:red, 0; green, 0; blue, 0 }  ][line width=0.75]    (6.56,-1.97) .. controls (4.17,-0.84) and (1.99,-0.18) .. (0,0) .. controls (1.99,0.18) and (4.17,0.84) .. (6.56,1.97)   ;
    %Straight Lines [id:da10082879672889511] 
    \draw    (274,2244) -- (274,2258) ;
    \draw [shift={(274,2260)}, rotate = 270] [color={rgb, 255:red, 0; green, 0; blue, 0 }  ][line width=0.75]    (6.56,-1.97) .. controls (4.17,-0.84) and (1.99,-0.18) .. (0,0) .. controls (1.99,0.18) and (4.17,0.84) .. (6.56,1.97)   ;
    %Straight Lines [id:da6259013142611769] 
    \draw    (460,2244) -- (460,2258) ;
    \draw [shift={(460,2260)}, rotate = 270] [color={rgb, 255:red, 0; green, 0; blue, 0 }  ][line width=0.75]    (6.56,-1.97) .. controls (4.17,-0.84) and (1.99,-0.18) .. (0,0) .. controls (1.99,0.18) and (4.17,0.84) .. (6.56,1.97)   ;
    %Straight Lines [id:da37112437512887253] 
    \draw    (274,2344) -- (274,2362) ;
    \draw [shift={(274,2364)}, rotate = 270] [color={rgb, 255:red, 0; green, 0; blue, 0 }  ][line width=0.75]    (6.56,-1.97) .. controls (4.17,-0.84) and (1.99,-0.18) .. (0,0) .. controls (1.99,0.18) and (4.17,0.84) .. (6.56,1.97)   ;
    %Straight Lines [id:da7692948858201427] 
    \draw    (460,2344) -- (460,2362) ;
    \draw [shift={(460,2364)}, rotate = 270] [color={rgb, 255:red, 0; green, 0; blue, 0 }  ][line width=0.75]    (6.56,-1.97) .. controls (4.17,-0.84) and (1.99,-0.18) .. (0,0) .. controls (1.99,0.18) and (4.17,0.84) .. (6.56,1.97)   ;
    %Straight Lines [id:da08711296960205606] 
    \draw    (219.42,2372) -- (217.17,2372) -- (217.17,2403.66) -- (213.36,2414) -- (217.17,2423.44) -- (217.17,2451.14) -- (219.42,2451.14) ;
    %Straight Lines [id:da7520506466890287] 
    \draw    (328,2344) -- (328,2410) ;
    \draw [shift={(328,2412)}, rotate = 270] [color={rgb, 255:red, 0; green, 0; blue, 0 }  ][line width=0.75]    (6.56,-1.97) .. controls (4.17,-0.84) and (1.99,-0.18) .. (0,0) .. controls (1.99,0.18) and (4.17,0.84) .. (6.56,1.97)   ;
    %Straight Lines [id:da2882853294833533] 
    \draw    (408,2344) -- (408,2410) ;
    \draw [shift={(408,2412)}, rotate = 270] [color={rgb, 255:red, 0; green, 0; blue, 0 }  ][line width=0.75]    (6.56,-1.97) .. controls (4.17,-0.84) and (1.99,-0.18) .. (0,0) .. controls (1.99,0.18) and (4.17,0.84) .. (6.56,1.97)   ;
    %Shape: Rectangle [id:dp6985462910267374] 
    \draw   (368,2412) -- (446,2412) -- (446,2460) -- (368,2460) -- cycle ;



    % Text Node
    \draw (198.5,2410) node  [font=\scriptsize] [align=left] {$\displaystyle g_{i} ...$};
    % Text Node
    \draw (506.31,2090) node  [font=\tiny] [align=left] {$\displaystyle h_{z,n}$};
    % Text Node
    \draw (459.81,2089.87) node  [font=\tiny] [align=left] {$\displaystyle h_{z,2}$};
    % Text Node
    \draw (424,2090) node  [font=\tiny] [align=left] {$\displaystyle h_{z,1}$};
    % Text Node
    \draw (464.5,2070) node  [font=\tiny] [align=left] {$\displaystyle h_{joint,z}$};
    % Text Node
    \draw (473.76,2077.87) node [anchor=north west][inner sep=0.75pt]  [font=\footnotesize] [align=left] {\begin{minipage}[lt]{9.53pt}\setlength\topsep{0pt}
            \begin{flushright}
                ...
            \end{flushright}

        \end{minipage}};
    % Text Node
    \draw (480.89,2450.23) node  [font=\tiny] [align=left] {$\displaystyle \left( R'\ \leftarrow b\times R\right)$};
    % Text Node
    \draw (485.18,2422.87) node  [font=\tiny] [align=left] {$\displaystyle ...$};
    % Text Node
    \draw (484.08,2434.13) node  [font=\tiny] [align=left] {$\displaystyle \Omega _{j,k}\rightarrow b$};
    % Text Node
    \draw (484.28,2408.5) node  [font=\tiny] [align=left] {$\displaystyle \Omega _{j,2}\rightarrow 2b/k$};
    % Text Node
    \draw (484.81,2390.97) node  [font=\tiny] [align=left] {$\displaystyle \Omega _{j,1}\rightarrow b/k$};
    % Text Node
    \draw (255.17,2429.74) node  [font=\tiny] [align=left] {$\displaystyle ...$};
    % Text Node
    \draw (254.07,2440.88) node  [font=\tiny] [align=left] {$\displaystyle \Omega _{i,q}\rightarrow A_{i,q}$};
    % Text Node
    \draw (254.27,2415.52) node  [font=\tiny] [align=left] {$\displaystyle \Omega _{i,2}\rightarrow A_{2}$};
    % Text Node
    \draw (254.8,2398.16) node  [font=\tiny] [align=left] {$\displaystyle \Omega _{i,1}\rightarrow A_{1}$};
    % Text Node
    \draw (537,2410) node  [font=\scriptsize] [align=left] {$\displaystyle \rho _{j} ...$};
    % Text Node
    \draw (252.79,2374.51) node  [font=\scriptsize] [align=left] {$\displaystyle GRG_{i}$};
    % Text Node
    \draw (480.89,2372.97) node  [font=\scriptsize] [align=left] {$\displaystyle RAG_{j}$};
    % Text Node
    \draw (460.18,2234.5) node   [align=left] {{\tiny \textit{trajectoires complètes}}};
    % Text Node
    \draw (275.5,2234.5) node   [align=left] {{\tiny \textit{trajectoires d'observations}}};
    % Text Node
    \draw (448.99,2195.53) node [anchor=west] [inner sep=0.75pt]  [font=\footnotesize] [align=left] {\begin{minipage}[lt]{9.53pt}\setlength\topsep{0pt}
            \begin{flushright}
                ...
            \end{flushright}

        \end{minipage}};
    % Text Node
    \draw (490.5,2216.03) node  [font=\tiny] [align=left] {$\displaystyle h_{z,n}$};
    % Text Node
    \draw (489.45,2190.24) node  [font=\tiny] [align=left] {$\displaystyle h_{z,1}$};
    % Text Node
    \draw (488.23,2177.83) node  [font=\tiny] [align=left] {$\displaystyle h_{joint,z}$};
    % Text Node
    \draw (499,2194) node [anchor=north west][inner sep=0.75pt]  [font=\footnotesize,rotate=-90] [align=left] {\begin{minipage}[lt]{9.53pt}\setlength\topsep{0pt}
            \begin{flushright}
                ...
            \end{flushright}

        \end{minipage}};
    % Text Node
    \draw (430.38,2214.86) node  [font=\tiny] [align=left] {$\displaystyle h_{1,n}$};
    % Text Node
    \draw (430.69,2190.24) node  [font=\tiny] [align=left] {$\displaystyle h_{1,1}$};
    % Text Node
    \draw (429.22,2177.83) node  [font=\tiny] [align=left] {$\displaystyle h_{joint,1}$};
    % Text Node
    \draw (440,2194) node [anchor=north west][inner sep=0.75pt]  [font=\footnotesize,rotate=-90] [align=left] {\begin{minipage}[lt]{9.53pt}\setlength\topsep{0pt}
            \begin{flushright}
                ...
            \end{flushright}

        \end{minipage}};
    % Text Node
    \draw (109,2376) node [anchor=north west][inner sep=0.75pt]   [align=left] {{\footnotesize \textit{4) Extraire}}\\{\footnotesize \textit{Guides de}}\\{\footnotesize \textit{Contraintes}}\\{\footnotesize (Roles \&}\\{\footnotesize Objectifs)}};
    % Text Node
    \draw (109,2269) node [anchor=north west][inner sep=0.75pt]   [align=left] {{\footnotesize \textit{3) Clustering}}\\{\footnotesize \textit{hierarchique}}\\{\footnotesize \textit{\& Calcul}}\\{\footnotesize \textit{\textcolor[rgb]{0.25,0.46,0.02}{\textbf{centroides}}}}};
    % Text Node
    \draw (109,2181) node [anchor=north west][inner sep=0.75pt]   [align=left] {{\footnotesize \textit{2) Preparer}}\\{\footnotesize \textit{données}}};
    % Text Node
    \draw (109,2065) node [anchor=north west][inner sep=0.75pt]   [align=left] {{\footnotesize \textit{1) Collecter}}\\{\footnotesize \textit{trajectoires}}};
    % Text Node
    \draw (346,2292.59) node  [font=\tiny] [align=left] {$\displaystyle h_{( z,n) ,( z,n-1)}^{\Omega }$};
    % Text Node
    \draw (266,2264.27) node [anchor=north west][inner sep=0.75pt]  [font=\tiny] [align=left] {$\displaystyle h_{*}^{\Omega }$};
    % Text Node
    \draw (209,2272.59) node  [font=\tiny] [align=left] {$\displaystyle h_{(( 1,1) ,( 1,2)) ,( 1,3)}^{\Omega }$};
    % Text Node
    \draw (210.5,2292.59) node  [font=\tiny] [align=left] {$\displaystyle h_{( 1,1) ,( 1,2)}^{\Omega }$};
    % Text Node
    \draw (341.01,2329.5) node  [font=\tiny] [align=left] {$\displaystyle h_{z,n}^{\Omega }$};
    % Text Node
    \draw (313.32,2329.5) node  [font=\tiny] [align=left] {$\displaystyle h_{z,n-1}^{\Omega }$};
    % Text Node
    \draw (287.82,2327.74) node  [font=\tiny] [align=left] {$\displaystyle ...$};
    % Text Node
    \draw (269.09,2329.5) node  [font=\tiny] [align=left] {$\displaystyle h_{1,3}^{\Omega }$};
    % Text Node
    \draw (247.3,2329.5) node  [font=\tiny] [align=left] {$\displaystyle h_{1,2}^{\Omega }$};
    % Text Node
    \draw (225.5,2329.5) node  [font=\tiny] [align=left] {$\displaystyle h_{1,1}^{\Omega }$};
    % Text Node
    \draw (534,2296.84) node  [font=\tiny] [align=left] {$\displaystyle h_{( z,n) ,( z,n-1)}$};
    % Text Node
    \draw (464.26,2272.93) node  [font=\tiny] [align=left] {$\displaystyle h_{*}$};
    % Text Node
    \draw (401,2278) node  [font=\tiny] [align=left] {$\displaystyle h_{(( 1,1) ,( 1,2)) ,( 1,3)}$};
    % Text Node
    \draw (396.71,2296.07) node  [font=\tiny] [align=left] {$\displaystyle h_{( 1,1) ,( 1,2)}$};
    % Text Node
    \draw (527.28,2329.77) node  [font=\tiny] [align=left] {$\displaystyle h_{z,n}$};
    % Text Node
    \draw (499.58,2329.77) node  [font=\tiny] [align=left] {$\displaystyle h_{z,n-1}$};
    % Text Node
    \draw (474.09,2329.39) node  [font=\tiny] [align=left] {$\displaystyle ...$};
    % Text Node
    \draw (455.35,2329.77) node  [font=\tiny] [align=left] {$\displaystyle h_{1,3}$};
    % Text Node
    \draw (433.56,2329.77) node  [font=\tiny] [align=left] {$\displaystyle h_{1,2}$};
    % Text Node
    \draw (411.76,2329.77) node  [font=\tiny] [align=left] {$\displaystyle h_{1,1}$};
    % Text Node
    \draw (264.99,2195.53) node [anchor=west] [inner sep=0.75pt]  [font=\footnotesize] [align=left] {\begin{minipage}[lt]{9.53pt}\setlength\topsep{0pt}
            \begin{flushright}
                ...
            \end{flushright}

        \end{minipage}};
    % Text Node
    \draw (305.38,2214.86) node  [font=\tiny] [align=left] {$\displaystyle h_{z,n}^{\Omega }$};
    % Text Node
    \draw (305.45,2190.24) node  [font=\tiny] [align=left] {$\displaystyle h_{z,1}^{\Omega }$};
    % Text Node
    \draw (304.23,2177.83) node  [font=\tiny] [align=left] {$\displaystyle h_{joint,z}^{\Omega }$};
    % Text Node
    \draw (315,2194) node [anchor=north west][inner sep=0.75pt]  [font=\footnotesize,rotate=-90] [align=left] {\begin{minipage}[lt]{9.53pt}\setlength\topsep{0pt}
            \begin{flushright}
                ...
            \end{flushright}

        \end{minipage}};
    % Text Node
    \draw (248,2215) node  [font=\tiny] [align=left] {$\displaystyle h_{1,n}^{\Omega }$};
    % Text Node
    \draw (246.69,2190.24) node  [font=\tiny] [align=left] {$\displaystyle h_{1,1}^{\Omega }$};
    % Text Node
    \draw (245.22,2177.83) node  [font=\tiny] [align=left] {$\displaystyle h_{joint,1}^{\Omega }$};
    % Text Node
    \draw (256,2194) node [anchor=north west][inner sep=0.75pt]  [font=\footnotesize,rotate=-90] [align=left] {\begin{minipage}[lt]{9.53pt}\setlength\topsep{0pt}
            \begin{flushright}
                ...
            \end{flushright}

        \end{minipage}};
    % Text Node
    \draw (314.31,2090) node  [font=\tiny] [align=left] {$\displaystyle h_{1,n}$};
    % Text Node
    \draw (267.81,2089.87) node  [font=\tiny] [align=left] {$\displaystyle h_{1,2}$};
    % Text Node
    \draw (232,2090) node  [font=\tiny] [align=left] {$\displaystyle h_{1,1}$};
    % Text Node
    \draw (272.5,2070) node  [font=\tiny] [align=left] {$\displaystyle h_{joint,1}$};
    % Text Node
    \draw (281.76,2077.87) node [anchor=north west][inner sep=0.75pt]  [font=\footnotesize] [align=left] {\begin{minipage}[lt]{9.53pt}\setlength\topsep{0pt}
            \begin{flushright}
                ...
            \end{flushright}

        \end{minipage}};
    % Text Node
    \draw (369,2081.5) node  [font=\footnotesize] [align=left] {\begin{minipage}[lt]{9.53pt}\setlength\topsep{0pt}
            \begin{flushright}
                ...
            \end{flushright}

        \end{minipage}};
    % Text Node
    \draw (407,2436) node  [font=\tiny] [align=left] {$\displaystyle  \begin{array}{{>{\displaystyle}l}}
                FOF=                           \\
                \ \ \ \ deviation(             \\
                \ \ \ \ \ \ \ \ clusters_{obs} \\
                \ \ \ \ \ \ \ \ \dotsc )
            \end{array}$};
    % Text Node
    \draw (461,2136.96) node  [font=\scriptsize] [align=left] {{\footnotesize one-hot encoding}\\{\footnotesize sur actions}};
    % Text Node
    \draw (326,2436) node  [font=\tiny] [align=left] {$\displaystyle  \begin{array}{{>{\displaystyle}l}}
                SOF=                             \\
                \ \ \ \ deviation(               \\
                \ \ \ \ \ \ \ \ clusters_{trans} \\
                \ \ \ \ \ \ \ \ \dotsc )
            \end{array}$};
    % Text Node
    \draw (274,2137.04) node  [font=\scriptsize] [align=left] {{\footnotesize extraire}\\{\footnotesize observations}};


\end{tikzpicture}
\end{frame}


\section{Configuration expérimentale}

\begin{frame}{Configuration expérimentale}{Métriques d'évaluation \& protocole}

  \begin{block}{Objectifs du cadre expérimental}
    \begin{itemize}
      \item Évaluer si les contraintes organisationnelles améliorent :
            \begin{itemize}
              \item \textbf{L'alignement comportemental} avec des rôles et objectifs structurés.
              \item \textbf{La stabilité de l'apprentissage et la convergence}.
              \item \textbf{L'explicabilité et la robustesse} des agents entraînés.
            \end{itemize}
    \end{itemize}
  \end{block}

  \vspace{0.5em}

  \begin{block}{Métriques d'évaluation}
    Récompense Cumulative, Écart-type de la Récompense, Taux de Convergence, Taux de Violation des Contraintes, Score de Cohérence, Score de Robustesse, Niveau d'adéquation organisationnelle
  \end{block}

  \vspace{0.5em}

  \begin{block}{Protocole d'évaluation}
    \begin{itemize}
      \item \textbf{Référence de base (RB)} : entraînement MARL sans contraintes organisationnelles.
      \item \textbf{Base organisationnelle (OB)} : entraînement avec spécifications MOISE+ prédéfinies.
      \item Les métriques sont calculées et comparées entre environnements.
    \end{itemize}
  \end{block}

\end{frame}



\begin{frame}{Configuration expérimentale}{Environnements \& Spécifications organisationnelles}

  \vspace{-0cm}

  \begin{columns}[c]

    \hspace{-1cm}

    \begin{column}{0.5\textwidth}

      \begin{itemize}
        \item \textbf{Predator-Prey}~\autocite{lowe2017multi}.
              \begin{itemize}
                \item \textit{Rôles} : poursuivant, bloqueur
                \item \textit{Objectifs} : encercler la proie, empêcher la fuite
              \end{itemize}
        \item \textbf{Overcooked-AI}~\autocite{overcookedai}.
              \begin{itemize}
                \item \textit{Rôles} : chef, assistant, serveur
                \item \textit{Objectifs} : livrer des plats, éviter les collisions
              \end{itemize}
        \item \textbf{Warehouse Management (personnalisé)}
              \begin{itemize}
                \item \textit{Rôles} : préparateur, transporteur, réapprovisionneur
                \item \textit{Objectifs} : déplacer des objets, réapprovisionner les étagères
              \end{itemize}
        \item \textbf{CybORG}~\autocite{Maxwell2021}.
              \begin{itemize}
                \item \textit{Rôles} : IDS, intervenant, opérateur pare-feu
                \item \textit{Objectifs} : détecter les intrusions, restaurer les hôtes
              \end{itemize}
      \end{itemize}

    \end{column}

    \hspace{-1.5cm}

    \begin{column}{0.5\textwidth}
      \begin{tabular}{@{}c@{\hspace{1cm}}c@{}}
        \makebox[.48\textwidth][c]{\animategraphics[loop,autoplay,scale=0.15]{8}{figures/wm/frame}{0}{33}}   &
        \vspace{0.1cm} \makebox[.48\textwidth][c]{\animategraphics[loop,autoplay,scale=0.18]{8}{figures/overcooked_asymmetric_advantage/frame}{0}{66}} \\
        \small{Warehouse Management}                                                                           & \vspace{0.1cm} \small{Overcooked-AI}    \\
        \makebox[.48\textwidth][c]{\animategraphics[loop,autoplay,scale=0.135]{8}{figures/mpe/frame}{0}{25}} &
        \makebox[.48\textwidth][c]{\animategraphics[loop,autoplay,scale=0.135]{8}{figures/cyborg/frame}{0}{33}}                                        \\
        \small{Predator-Prey}                                                                                & \small{CybORG}                          \\
      \end{tabular}
    \end{column}
  \end{columns}
\end{frame}

\section{Résultats clés}

\begin{frame}{Résultats}{adéquation organisationnelle, performance et comparaison}

  \begin{itemize}
    \item \textbf{adéquation organisationnelle} ↑ comme attendu :
          \begin{itemize}
            \item Predator-Prey : +44\% (0.43 → 0.87), Overcooked-AI : +89\%, Entrepôt : +80\%
          \end{itemize}

    \item \textbf{Performance \& Stabilité} :
          \begin{itemize}
            \item Écart-type des récompenses ↓ (Overcooked-AI \& MAPPO : 15.6 → 10.4, Predator-Prey \& MADDPG : 21.5 → 15.2)
            \item Convergence ↑ (CybORG \& COMA : 0.70 → 0.86, Entrepôt \& Q-Mix : 0.74 → 0.88)
            \item Robustesse ↑ (Overcooked-AI \& MAPPO : 0.71 → 0.89) (\textit{biais introduit})
            \item Aucune violation de contrainte même avec un niveau de difficulté élevé
          \end{itemize}

    \item \textbf{Comparé à AGR+MARL} (sans \textquote{objectifs}) :
          \begin{itemize}
            \item adéquation organisationnelle : +33\% (Overcooked-AI), Robustesse : +0.14 (Predator-Prey)
            \item Récompense ↑ (Entrepôt : 307.1 contre 278.6)
            \item[] $\rightarrow$ \textbf{Gains dus aux objectifs intermédiaires}
          \end{itemize}

    \item \textbf{Cohérence globale} $\geq$ 0.76 \quad $\rightarrow$ \textbf{Les comportements correspondent aux rôles/objectifs} (+ inspection manuelle)

    \item \textit{Résultats préliminaires : le temps d'apprentissage semble évoluer de manière \textbf{quasi-linéaire} avec un faible nombre de contraintes}

  \end{itemize}

\end{frame}



\section{Conclusion}


\begin{frame}{Conclusion et perspectives}

  \begin{block}{Conclusion : relier les modèles symboliques/sub-symboliques aux politiques}
    \begin{itemize}
      \item Un cadre permettant d'injecter des connaissances humaines sous forme de rôles et d'objectifs durant l'apprentissage/l'exécution ;
      \item Une méthode empirique pour évaluer l'organisation \textquote{implicite} à partir des comportements des agents.
    \end{itemize}
  \end{block}

  \vspace{0.5em}

  \begin{block}{Perspectives}
    \begin{itemize}
      \item Permettre des contraintes organisationnelles \textbf{dynamiques} durant l'apprentissage ;
      \item Explorer l'utilisation de \textbf{LLMs} dans TEMM pour aider à identifier rôles et objectifs ;
      \item Passer à Jax/JaxMARL~\autocite{flair2023jaxmarl} ;
    \end{itemize}
  \end{block}

\end{frame}


\appendix
%\setbeamertemplate{headline}{}
\setbeamertemplate{mini frames}{}

% \AtBeginSection[]{
% 	\begin{frame}
% 		\frametitle{}
% 		\tableofcontents[currentsection]
% 	\end{frame}
% }

% %%%%%%%%%%%%%%%%%%%%%%%%%%%%%%%%%%%%

\section*{\phantom{Thanks}}

\begin{frame}{}

  \vspace{6ex}

  \centering
  {
    \Huge
    \emph{Thank You}
  }

  \vspace{6ex}

  \begin{columns}

    \hspace{-27ex}

    \begin{column}{0.5\textwidth}
      \raggedleft
      {\Large Demo video $\Longrightarrow$}
    \end{column}

    \hspace{-12ex}

    \begin{column}{0.5\textwidth}
      \includegraphics[width=0.5\linewidth]{figures/demo_qr_code.png}
    \end{column}

  \end{columns}

  \vspace{3ex}

  \centering
  {\Large
    \url{https://t.ly/4JBxr}
  }

\end{frame}

% \AtBeginSection[]{
% 	\begin{frame}
% 		\frametitle{}
% 		\tableofcontents[currentsection]
% 	\end{frame}
% }

% %%%%%%%%%%%%%%%%%%%%%%%%%%%%%%%%%%%%

\section*{\phantom{References}}

\begin{frame}[allowframebreaks]{References}{}

    % \bibliographystyle{plain}
    % \bibliography{local_references}
    \printbibliography

\end{frame}

\newcounter{mainframenumber}
\setcounter{mainframenumber}{\value{framenumber}}

\begin{frame}{Annexes}
    {Context}

    \begin{block}{Multi-Agent Systems (MAS) paradigm for complex \& distributed problems}
        \begin{itemize}
            \item \textbf{task decomposition}: missions delegated to agents achieved through cooperation~\cite{Raileanu2023};
            \item \textbf{benefits}: handle conflicting goals, parallel computation, system robustness, scalability\dots
        \end{itemize}
    \end{block}

    \begin{block}{\textbf{Organization}: key for MAS designing}
        \begin{itemize}
            \item \textbf{coordination}: how to collaboratively achieve a common goal~\cite{Hubner2007};
            \item \textbf{dynamic \& uncertain environments}: flexible runtime behavior to adapt~\cite{Kathleen2020};
        \end{itemize}
    \end{block}

    \begin{block}{Methods and practice for MAS design}
        \begin{itemize}
            \item \textbf{approach + organizational model}: methods rely on designers' experience to hand-craft agents' \textbf{policies} so resulting MAS achieve goals;
                  %   \begin{itemize}
                  %       \item Examples: \emph{GAIA}~\cite{Wooldridge2000,Cernuzzi2014}, \emph{ADELFE}~\cite{Mefteh2015}, or \emph{DIAMOND}~\cite{Jamont2015}, \emph{KB-ORG}~\cite{Sims2008}
                  %   \end{itemize}
            \item \textbf{simulation to reality}: 1) safe \& efficient MAS design in high fidelity simulated environment; \quad 2) transfer to real environment to perform adequately~\cite{Schon2021}.
        \end{itemize}
        \vspace{1ex}
        \quad $\Longrightarrow$ \textbf{Iterative process proceeding by trial and error}

    \end{block}

\end{frame}

\begin{frame}{Annexes}
    {MAS basics}

    \begin{block}{Keywords}
        \begin{itemize}
            \item \textbf{Agent}: entity immersed in an environment perceiving observation and making decision autonomously to achieve some goals;
            \item \textbf{MAS}: a set of agents collaborating with self/re-organizing mechanisms to achieve their goal;
            \item \textbf{Organization}: the agents' interactions even though it may be implicit;
            \item \textbf{Organizational Model (OM)}: medium to formally describe an explicit/implicit organization;
            \item \textbf{Organizational Specifications (OS)}: components of an OM to characterize an organization
        \end{itemize}
    \end{block}

    \begin{block}{Organizational model: $\mathcal{M}OISE^+$}
        \begin{itemize}
            \item more complex than \emph{Agent Group Roles} (integration of standards);
            \item takes into account the social aspects between agents explicitly;
            \item possible to link agents' policies to organizational specifications.
        \end{itemize}
    \end{block}

\end{frame}

\begin{frame}{Annexes}
    {MARL basics}

    \begin{block}{Keywords}
        \begin{itemize}
            \item \textbf{Policy}: the \textquote{logic} to choose next action according to observation for an agent;
            \item \textbf{History/trajectory}: the tuple of (observation, action) couples over an episode;
            \item \textbf{Joint-policy / Joint-history}: all of the agents' policies / histories as tuples;
            \item \textbf{Reinforcement learning}: an agent updates its policy to maximize a cumulative reward;
            \item \textbf{Multi-Agent Reinforcement Learning (MARL)}: extends to multiple agents that learn while considering the actions of other agents;
        \end{itemize}
    \end{block}

\end{frame}



\end{document}
