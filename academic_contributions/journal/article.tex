\documentclass[runningheads]{llncs}

\usepackage[T1]{fontenc}
\usepackage{graphicx}
%\usepackage{color}
%\renewcommand\UrlFont{\color{blue}\rmfamily}

\usepackage{amsmath,amssymb,amsfonts}
\usepackage[inline, shortlabels]{enumitem}
\usepackage{tabularx}
\usepackage{caption}
% \usepackage{titlesec}
\usepackage[english]{babel}
\captionsetup{font=it}
\usepackage{ragged2e}
\usepackage{hyperref}
\usepackage{pifont}
\usepackage{footmisc}
\usepackage{multirow}
\usepackage{algorithm2e}

% --- Tickz
\usepackage{physics}
\usepackage{amsmath}
\usepackage{tikz}
\usepackage{mathdots}
\usepackage{yhmath}
\usepackage{cancel}
\usepackage{color}
\usepackage{siunitx}
\usepackage{array}
\usepackage{multirow}
\usepackage{amssymb}
\usepackage{gensymb}
\usepackage{tabularx}
\usepackage{extarrows}
\usepackage{booktabs}
\usetikzlibrary{fadings}
\usetikzlibrary{patterns}
\usetikzlibrary{shadows.blur}
\usetikzlibrary{shapes}

% ---------

\usepackage{pdfpages}
\usepackage{booktabs}
\usepackage{csquotes}
\usepackage{lipsum}  
\usepackage{arydshln}
\usepackage{smartdiagram}
\usepackage[inkscapeformat=png]{svg}
\usepackage{textcomp}
\usepackage{tabularray}\UseTblrLibrary{varwidth}
\usepackage{xcolor}
\def\BibTeX{{\rm B\kern-.05em{\sc i\kern-.025em b}\kern-.08em
    T\kern-.1667em\lower.7ex\hbox{E}\kern-.125emX}}
\usepackage{cite}
\usepackage{amsmath}
\newcommand{\probP}{\text{I\kern-0.15em P}}
\usepackage{etoolbox}
\patchcmd{\thebibliography}{\section*{\refname}}{}{}{}

\setlength{\extrarowheight}{2.5pt}

\renewcommand{\arraystretch}{1.7}

\setlength{\extrarowheight}{2.5pt}
\renewcommand{\arraystretch}{0.2}
\renewcommand{\arraystretch}{1.7}

% --------------
% \titleclass{\subsubsubsection}{straight}[\subsection]

% \newcounter{subsubsubsection}[subsubsection]
% \renewcommand\thesubsubsubsection{\thesubsubsection.\arabic{subsubsubsection}}
% \renewcommand\theparagraph{\thesubsubsubsection.\arabic{paragraph}} % optional; useful if paragraphs are to be numbered

% \titleformat{\subsubsubsection}
%   {\normalfont\normalsize\bfseries}{\thesubsubsubsection}{1em}{}
% \titlespacing*{\subsubsubsection}
% {0pt}{3.25ex plus 1ex minus .2ex}{1.5ex plus .2ex}

% \makeatletter
% \renewcommand\paragraph{\@startsection{paragraph}{5}{\z@}%
%   {3.25ex \@plus1ex \@minus.2ex}%
%   {-1em}%
%   {\normalfont\normalsize\bfseries}}
% \renewcommand\subparagraph{\@startsection{subparagraph}{6}{\parindent}%
%   {3.25ex \@plus1ex \@minus .2ex}%
%   {-1em}%
%   {\normalfont\normalsize\bfseries}}
% \def\toclevel@subsubsubsection{4}
% \def\toclevel@paragraph{5}
% \def\toclevel@paragraph{6}
% \def\l@subsubsubsection{\@dottedtocline{4}{7em}{4em}}
% \def\l@paragraph{\@dottedtocline{5}{10em}{5em}}
% \def\l@subparagraph{\@dottedtocline{6}{14em}{6em}}
% \makeatother

% \setcounter{secnumdepth}{4}
% \setcounter{tocdepth}{4}
% --------------

\newcommand{\before}[1]{\textcolor{red}{#1}}
\newcommand{\after}[1]{\textcolor{green}{#1}}

\newcommand{\old}[1]{\textcolor{orange}{#1}}
\newcommand{\rem}[1]{\textcolor{red}{#1}}
\newcommand{\todo}[1]{\textcolor{orange}{\newline \textit{\textbf{TODO:} #1}} \newline \newline }



\newcounter{relation}
\setcounter{relation}{0}
\renewcommand{\therelation}{\arabic{relation}}
\newcommand{\relationautorefname}{Relation}

\newenvironment{relation}[1][]{%
    \refstepcounter{relation}%
    \noindent \raggedright \textit{\textbf{Relation. \therelation}} \hfill$}
{%
$ \hfill \phantom{x}

}

\newcounter{proof}
\setcounter{proof}{0}
\renewcommand{\theproof}{\arabic{proof}}
\newcommand{\proofautorefname}{Proof}

\renewenvironment{proof}[1][]{
    \refstepcounter{proof}
    \noindent \raggedright \textit{\textbf{Proof. \theproof}}

    \setlength{\leftskip}{1em}

}
{

\
\setlength{\leftskip}{0pt}
}

% --------------------------------
%             DOCUMENT
% --------------------------------

\begin{document}
%
\title{An Assisted-Engineering Method for Developing Cyberdefense Multi-Agent Systems}
%
%\titlerunning{Abbreviated paper title}
% If the paper title is too long for the running head, you can set
% an abbreviated paper title here
%
\author{Julien Soulé\inst{1}\orcidID{0000-1111-2222-3333} \and
    Jean-Paul Jamont\inst{1}\orcidID{1111-2222-3333-4444} \and
    Michel Occelo\inst{1}\orcidID{2222--3333-4444-5555} \and
    Louis-Marie Traonouez\inst{2}\orcidID{2222--3333-4444-5555} \and
    Paul Théron\inst{3}\orcidID{2222--3333-4444-5555}}
%
\authorrunning{J. Soulé et al.}
% First names are abbreviated in the running head.
% If there are more than two authors, 'et al.' is used.
%
\institute{Univ. Grenoble Alpes, Grenoble INP, LCIS, 26000, Valence, France
    \email{\{julien.soule, jean-paul.jamont, michel.occello\}@lcis.grenoble-inp.fr}
    \and
    Thales Land and Air Systems, BU IAS, Rennes, France
    \email{louis-marie.traonouez@thalesgroup.com}
    \and
    AICA IWG, La Guillermie, France \\
    \email{paul.theron@orange.fr}
}

\maketitle              % typeset the header of the contribution

% TODO:
%  - Se focaliser principalement sur l'explicabilité

\begin{abstract}
    Current Multi-Agent Reinforcement Learning (MARL) algorithms, such as MADDPG and MAPPO, are quantitatively effective in achieving objectives but lack the capability to explain their decision-making processes and ensure compliance with strong constraints. This paper proposes a hybrid approach that integrates MARL with an organizational model derived from the MOISE+ framework. In this approach, organizational specifications, such as roles and missions, are imposed on agents through obligations and permissions. These specifications are associated with expected behaviors, theoretically formalized as sets of histories. This formalization not only links generated histories to known specifications but also constrains agent training to expected behaviors. Consequently, agents constrained by roles and missions are limited in the policies they can develop during training, ensuring that certain constraints are respected, thereby increasing control and safety in the multi-agent system (MAS). Additionally, this approach enhances the interpretability of trained MAS behavior by associating generated histories with known specifications.

    If the generated histories do not correspond to known specifications, we propose the General Organizational Specification Inference Approach (GOSIA). GOSIA infers new organizational specifications, such as roles or missions, from generated histories. To determine roles, it uses hierarchical clustering by measuring how much sequences in the histories share the longest common subsequence (LCS). Policies that produce histories with significant LCS are likely to be abstractly understood as arising from the same role. For missions, GOSIA identifies intermediate states (in terms of observations) that agents pass through before reaching their final states. The idea is to view objectives as intermediate states that all agents effective in reaching the final objective pass through. By collecting joint observations of effective agents, GOSIA identifies common trajectories (via KNN, PCA) by measuring the proximity of observations.

    We demonstrate the effectiveness of this hybrid approach in a cyber environment application, specifically a scenario involving a swarm of drones attacked by malware, based on the CAGE Challenge No. 3.

\end{abstract}

\section{Introduction}

Multi-Agent Reinforcement Learning (MARL) has shown great promise in solving complex problems involving multiple interacting agents. However, a significant limitation of current MARL algorithms, such as MADDPG and MAPPO, is the lack of interpretability and the inability to guarantee that agents respect strong constraints. In safety-critical applications like cybersecurity, these limitations can undermine trust in the system's behavior.

This paper introduces a hybrid approach that combines MARL with an organizational model inspired by the MOISE+ framework. The core idea is to impose organizational specifications, such as roles and missions, on agents through obligations and permissions, thereby ensuring that certain constraints are respected during training and deployment. Additionally, we propose the General Organizational Specification Inference Approach (GOSIA) to infer new organizational specifications from generated histories when known specifications are insufficient.

\section{Related Work}

Discuss existing MARL algorithms (e.g., MADDPG, MAPPO), their applications, and limitations in terms of interpretability and constraint satisfaction. Additionally, review existing organizational models for multi-agent systems, particularly MOISE+ and its derivatives.

\section{Hybrid Approach: MARL with Organizational Constraints}

\subsection{Organizational Model Based on MOISE+}
Describe the MOISE+ framework and how it is adapted to impose roles and missions on agents in a MARL environment. Explain the concept of organizational specifications, including obligations and permissions, and how they are associated with expected behaviors.

\subsection{Constraining Agent Training with Organizational Specifications}
Detail how the integration of organizational constraints limits the policies that agents can develop during training. Discuss how this approach increases control and safety in the resulting multi-agent system.

\section{General Organizational Specification Inference Approach (GOSIA)}

\subsection{Inferring Roles from Histories}
Explain the hierarchical clustering technique used to infer roles based on the longest common subsequence (LCS) in generated histories. Discuss the rationale behind considering policies with significant LCS as belonging to the same role.

\subsection{Inferring Missions from Histories}
Describe how GOSIA identifies intermediate states (observations) that agents pass through before reaching their final states. Detail the use of K-Nearest Neighbors (KNN) and Principal Component Analysis (PCA) to determine common trajectories and infer missions.

\section{Case Study: Cyber Defense Scenario}

\subsection{CAGE Challenge No. 3: Swarm of Drones Attacked by Malware}
Provide an overview of the CAGE Challenge No. 3, which involves a swarm of drones under cyber attack. Explain how the hybrid approach was applied to this scenario and the results achieved.

\subsection{Experimental Results}
Present experimental results demonstrating the effectiveness of the hybrid approach in constraining agent behavior, improving safety, and enhancing interpretability in the cyber defense scenario.

\section{Conclusion}

Summarize the key contributions of the hybrid approach and GOSIA. Discuss the implications for developing safe and interpretable multi-agent systems, particularly in cybersecurity applications. Suggest potential future work, including further refinement of the GOSIA technique and exploration of additional application domains.

\section*{Acknowledgment}

This work was supported by \emph{Thales Land Air Systems} within the framework of the \emph{Cyb'Air} chair and the \emph{AICA IWG}


\section*{References}

% \bibliographystyle{abbrv}
\bibliographystyle{splncs04}

\bibliography{references}

% \newpage

% \section*{Annexes}

% \subsection*{Action constraining during training implies result joint-policy constraining}
% \begin{proofoutline}\label{proof:jpc_to_ac}

    We provide an overview of our approach to constrain the possible policies of trained agents through a simple abstract example. While this example is somewhat artificial, it serves to illustrate the general principle of our approach and gives insights into why it is indeed effective in constraining policies.

    \noindent Let's consider an example with this initial configuration:

    \begin{itemize}
        \item $d=\langle S,A,T,R,\Omega, O, \gamma \rangle \in D$, the Dec-POMDP to solve (i.e maximizing $R$);
        \item $\mathcal{A}, |\mathcal{A}| = n \in \mathbb{N}$, the  $n$ agents involved in the Dec-POMDP;
        \item $s \in \mathbb{R}$, the cumulative reward expectancy to reach;
        \item $\pi_{joint} \in \Pi_{joint}, \allowbreak \pi_{joint} = \{\pi_1..\pi_n\}, \pi_k \in \Pi (k \leq n)$, the joint-policy to update;
        \item $ep_{max}$, the maximum number of episodes;
        \item $step_{max}$, the maximum number of steps per episode;
        \item $u_{marl}: \Pi_{joint} \times H_{joint} \times R_{joint} \rightarrow \Pi_{joint}$, the MARL algorithm that uses the joint-reward and joint-history to update a joint-policy;
    \end{itemize}
    %
    \noindent We assume some organizational specifications are defined, applied to agents, and associated with matching history subsets (at least from a theoretical point view):
    \item $os \in \mathcal{OS}$, the organizational specifications containing: $\mathcal{R}$, the roles that agents may be constrained to; $\mathcal{M}$, the missions that agents may be committed to; $\mathcal{OBL}$, the obligations indicating whether an agent playing a role $\rho \in \mathcal{R}$ is obligated to commit on mission $m \in \mathcal{M}$. In this example, we do not consider permissions; $rh: \mathcal{R} \rightarrow \mathcal{P}(H)$: gives the expected history subset for a role; $mh: \mathcal{M} \rightarrow \mathcal{P}(H)$: gives the expected history subset for a mission; $da: \mathcal{OBL} \rightarrow \mathcal{P}(A)$: gives the agents constrained to a role and obligated to commit on a mission.

    \

    \noindent We suppose there exists a set of joint-policies $S\Pi_{joint} = \{s\pi_{joint,1}.. s\pi_{joint,d}\} \allowbreak (d \in \mathbb{N})$, that enables reaching at least the $s$ cumulative reward expectancy.

    \noindent We suppose there exists a set of joint-policies $O\Pi_{joint} = \{o\pi_{joint,1}.. o\pi_{joint,d'}\} (d' \in \mathbb{N})$ that satisfy the applied organizational specifications, so that an agent playing role $\rho \in \mathcal{R}$ and obligated to commit on mission $m \in \mathcal{M}$ should have its policy $o\pi_{joint,i} \ (i \leq d')$ to generate any matching history $h \in (rh(\rho) \cap mh(m))$.

    \noindent We assume there exists a non-empty set of joint-policies $P\Pi = S\Pi \cap O\Pi \allowbreak = \{p\pi_{joint,1}..p\pi_{joint,q}\}, q \in \mathbb{N}$ that both reach at least the $s$ cumulative reward expectancy and satisfy the organizational specifications $os$.

    \

    Based on these assumptions and initial data, we apply PRAHOM on the first iterations and generalize it to indefinite number of iteration, in order to determine whether it enables building a policy that does belong to $P\Pi$. Although all constraints integration modes are effective in constraining policies, in this example, we chose the $correct\_policy$ mode to apply our algorithm for it offers a clear way to understand the proof outline.
    We consider the first episode. Initially, a constrained policy $\pi_{joint} = \pi_{joint,c}$ built from the initial policy $\pi_{joint,init,0}$ and the observable policy constraint $c\pi_{joint}$.

    At first step, agents have an empty history $h_{joint} = \langle \rangle$, null rewards $rh_{joint} = \langle (0)^n \rangle $. Thus, the initial policies $\pi_{joint,0} \in \Pi_{joint}$ are not updated for now. Receiving the initial observations for each agents $\omega_{joint,0} \in \Omega_{joint}$, agents choose their respective next actions $a_{joint,0}$ using their policies $\allowbreak \pi_{joint,0}$. The observations and actions are stored in history $h_{joint} \allowbreak = \allowbreak \langle \allowbreak (\omega_{joint,0}, \allowbreak a_{joint,0}) \rangle$. Then, the action are applied, hence generating new observations $\omega_{joint,1}$ and rewards $r_{joint,1}$ stored in $rh_{joint}$ for the next step.

    % \

    % At second step, agents have the current history $h_{joint} = \langle (\omega_{joint,0}, a_{joint,0}) \rangle$, and rewards $rh_{joint} = \langle (0)^n, r_{joint,1} \rangle$. Thus, the policies are updated accordingly $\pi_{joint,1} = u_{marl}(\pi_{joint,0},h_{joint},rh_{joint})$. From received observation $\omega_{joint,1}$, agents choose their next actions $a_{joint,1}$ using their policies $\pi_{joint,1}$. The observations and actions are stored in the joint-histories $h_{joint} = \langle (\omega_{joint,0},a_{joint,0}), (\omega_{joint,1},a_{joint,1}) \rangle$. Then, the actions are applied, hence generating new observations $\omega_{joint,2}$ and rewards $r_{joint,2}$ stored in $rh_{joint}$ for the next step.

    Generalizing until the $p < step_{max}$ step, agents have the current history $h_{joint} = \langle (\omega_{joint,0}, \allowbreak a_{joint,0}), (\omega_{joint,1}, a_{joint,1})..(\omega_{joint,p-1}, a_{joint,p-1}) \rangle$, and rewards $rh_{joint} = \langle (0)^n, r_{joint,1}, r_{joint,2}..r_{joint,p} \rangle$. Thus, the policies are updated accordingly $\pi_{joint,p} = u_{marl}(\pi_{joint,p-1},h_{joint},rh_{joint})$. From received observation $\omega_{joint,p}$, agents choose their next actions $a_{joint,p}$ using their policies $\pi_{joint,p}$. The observations and actions are stored in history $h_{joint} \allowbreak = \allowbreak \langle \allowbreak (\omega_{joint,0},a_{joint,0}), \allowbreak (\omega_{joint,1},a_{joint,1}), (\omega_{joint,2},a_{joint,2})..(\omega_{joint,p},a_{joint,p}) \rangle$. Then, the actions are applied, hence generating new observations $\omega_{joint,p+1}$ and rewards $r_{joint,p+1}$ stored in $rh_{joint}$ for the next step.

    \

    When $p = step_{max}$, the episode is finished, we assume the cumulative reward reaches at least $s$. The generated histories are $h_{joint} = \langle (\omega_{joint,0}, \allowbreak a_{joint,0}) .. \allowbreak (\omega_{joint,step_{max}},a_{joint,step_{max}}) \rangle$. Throughout all steps, it is built using the $\pi_{joint,k}, \allowbreak k < step_{max}$ and $\pi_{joint,k} \allowbreak = \allowbreak \{sample(c\pi_{joint}(\omega_{joint})) \allowbreak \ \allowbreak if \allowbreak \ \allowbreak \omega_{joint} \in Dom(c\pi_{joint}) \allowbreak \ \allowbreak else \allowbreak \ \allowbreak \pi_{joint,k}(\omega_{joint,init,k})\}$.
    %
    By definition, $\langle (\omega_{joint,j}, \allowbreak sample(c\pi_{joint}(\omega_{joint,j}))), \allowbreak j < step_{max} \rangle, \allowbreak \omega_{joint,j} \allowbreak \in \Omega_{joint}$, the joint-history generated using the observable constrained policy satisfy the organizational specifications. Thus, the policy represented by $sample(c\pi_{joint}(\omega_{joint,j}))$ belongs to $O\Pi_{joint}$ and possibly $S\Pi_{joint}$.
    %
    By construction, $\langle (\pi_{joint,init,k}(\omega_{joint,k}))_{k < step_{max}} \rangle$, the joint-history generated using the initial policy trained over $k$ steps so that the cumulative reward reach at least $s$. Thus, the policy $\pi_{joint,init,step_{max}}$ belong to $S\Pi_{joint}$.

    Considering several episodes, $s$ is reach for a policy in $\allowbreak S\Pi_{joint,ep_{max},step_{max}}$. Moreover, since a history $h_{joint}$ belongs, at least, to histories generated by a policy in $O\Pi_{joint}$. Thus, $\pi_{joint,ep_{max},step_{max}} \in S\Pi \cap O\Pi, \pi_{joint,ep_{max},step_{max}} \in P\Pi$. So, built policies indeed satisfy organizational specifications while reaching sufficient cumulative reward expectancy.

    \

    As for the other constraint integration modes, we briefly outline the main ideas supporting why it is also effective as well as the $correct\_policy$ mode:
    $\mathbf{correct}$, corrects the action according to an observable policy constraint after the initial policy has chosen it. Without other consideration, it can be modeled by building a constrained policy $\pi_c$ encompassing both the observable policy constraint and the initial policy. Therefore, this goes back to the $correct\_policy$ case;
    $\mathbf{penalize}$, adjust the reward comparing chosen action by the policy and the expected ones according to an observable policy constraint. We assume that the policy can be updated according to rewards so that it asymptotically tends be equal to any constrained policy formed from the current policy and the observable policy constraint. Therefore, it also goes back to the $correct\_policy$ case.

\end{proofoutline}

\end{document}
