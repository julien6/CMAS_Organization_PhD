\RequirePackage[2020-02-02]{latexrelease}
%Version 3 October 2023
% See section 11 of the User Manual for version history
%
%%%%%%%%%%%%%%%%%%%%%%%%%%%%%%%%%%%%%%%%%%%%%%%%%%%%%%%%%%%%%%%%%%%%%%
%%                                                                 %%
%% Please do not use \input{...} to include other tex files.       %%
%% Submit your LaTeX manuscript as one .tex document.              %%
%%                                                                 %%
%% All additional figures and files should be attached             %%
%% separately and not embedded in the \TeX\ document itself.       %%
%%                                                                 %%
%%%%%%%%%%%%%%%%%%%%%%%%%%%%%%%%%%%%%%%%%%%%%%%%%%%%%%%%%%%%%%%%%%%%%

%%\documentclass[referee,sn-basic]{sn-jnl}% referee option is meant for double line spacing

%%=======================================================%%
%% to print line numbers in the margin use lineno option %%
%%=======================================================%%

%%\documentclass[lineno,sn-basic]{sn-jnl}% Basic Springer Nature Reference Style/Chemistry Reference Style

%%======================================================%%
%% to compile with pdflatex/xelatex use pdflatex option %%
%%======================================================%%

%%\documentclass[pdflatex,sn-basic]{sn-jnl}% Basic Springer Nature Reference Style/Chemistry Reference Style


%%Note: the following reference styles support Namedate and Numbered referencing. By default the style follows the most common style. To switch between the options you can add or remove �Numbered� in the optional parenthesis. 
%%The option is available for: sn-basic.bst, sn-vancouver.bst, sn-chicago.bst%  
 
%%\documentclass[sn-nature]{sn-jnl}% Style for submissions to Nature Portfolio journals
%%\documentclass[sn-basic]{sn-jnl}% Basic Springer Nature Reference Style/Chemistry Reference Style
\documentclass[sn-mathphys-num]{sn-jnl}% Math and Physical Sciences Numbered Reference Style 
%%\documentclass[sn-mathphys-ay]{sn-jnl}% Math and Physical Sciences Author Year Reference Style
%%\documentclass[sn-aps]{sn-jnl}% American Physical Society (APS) Reference Style
%%\documentclass[sn-vancouver,Numbered]{sn-jnl}% Vancouver Reference Style
%%\documentclass[sn-apa]{sn-jnl}% APA Reference Style 
%%\documentclass[sn-chicago]{sn-jnl}% Chicago-based Humanities Reference Style

%%%% Standard Packages
%%<additional latex packages if required can be included here>

\usepackage{graphicx}%
\usepackage{multirow}%
\usepackage{amsmath,amssymb,amsfonts}%
\usepackage{amsthm}%
\usepackage{mathrsfs}%
\usepackage[title]{appendix}%
\usepackage{xcolor}%
\usepackage{textcomp}%
\usepackage{manyfoot}%
\usepackage{booktabs}%
\usepackage{algorithm}%
\usepackage{algorithmicx}%
\usepackage{algpseudocode}%
\usepackage{listings}%
%%%%

%%%%%=============================================================================%%%%
%%%%  Remarks: This template is provided to aid authors with the preparation
%%%%  of original research articles intended for submission to journals published 
%%%%  by Springer Nature. The guidance has been prepared in partnership with 
%%%%  production teams to conform to Springer Nature technical requirements. 
%%%%  Editorial and presentation requirements differ among journal portfolios and 
%%%%  research disciplines. You may find sections in this template are irrelevant 
%%%%  to your work and are empowered to omit any such section if allowed by the 
%%%%  journal you intend to submit to. The submission guidelines and policies 
%%%%  of the journal take precedence. A detailed User Manual is available in the 
%%%%  template package for technical guidance.
%%%%%=============================================================================%%%%

\def\BibTeX{{\rm B\kern-.05em{\sc i\kern-.025em b}\kern-.08em
    T\kern-.1667em\lower.7ex\hbox{E}\kern-.125emX}}

\usepackage[english]{babel}
\addto\extrasenglish{  
    \def\figureautorefname{Figure}
    \def\tableautorefname{Table}
    \def\algorithmautorefname{Algorithm}
    \def\sectionautorefname{Section}
    \def\subsectionautorefname{Subsection}
}

\newcommand{\supertiny}{\fontsize{1}{2}\selectfont}

\usepackage{catoptions}
\makeatletter

\def\Autoref#1{%
  \begingroup
  \edef\reserved@a{\cpttrimspaces{#1}}%
  \ifcsndefTF{r@#1}{%
    \xaftercsname{\expandafter\testreftype\@fourthoffive}
      {r@\reserved@a}.\\{#1}%
  }{%
    \ref{#1}%
  }%
  \endgroup
}
\def\testreftype#1.#2\\#3{%
  \ifcsndefTF{#1autorefname}{%
    \def\reserved@a##1##2\@nil{%
      \uppercase{\def\ref@name{##1}}%
      \csn@edef{#1autorefname}{\ref@name##2}%
      \autoref{#3}%
    }%
    \reserved@a#1\@nil
  }{%
    \autoref{#3}%
  }%
}
\makeatother


%% as per the requirement new theorem styles can be included as shown below
\theoremstyle{thmstyleone}%
\newtheorem{theorem}{Theorem}%  meant for continuous numbers
%%\newtheorem{theorem}{Theorem}[section]% meant for sectionwise numbers
%% optional argument [theorem] produces theorem numbering sequence instead of independent numbers for Proposition
\newtheorem{proposition}[theorem]{Proposition}% 
%%\newtheorem{proposition}{Proposition}% to get separate numbers for theorem and proposition etc.

\theoremstyle{thmstyletwo}%
\newtheorem{example}{Example}%
\newtheorem{remark}{Remark}%

\theoremstyle{thmstylethree}%
\newtheorem{definition}{Definition}%

\raggedbottom
%%\unnumbered% uncomment this for unnumbered level heads

\begin{document}

\title[Explainability in MARL]{A History-based Approach for Organizational Explainability in MARL}

%%=============================================================%%
%% GivenName	-> \fnm{Joergen W.}
%% Particle	-> \spfx{van der} -> surname prefix
%% FamilyName	-> \sur{Ploeg}
%% Suffix	-> \sfx{IV}
%% \author*[1,2]{\fnm{Joergen W.} \spfx{van der} \sur{Ploeg} 
%%  \sfx{IV}}\email{iauthor@gmail.com}
%%=============================================================%%

\author*[1,2]{\fnm{Julien} \sur{Soulé}}\email{julien.soule@lcis.grenoble-inp.fr}

\author[1]{\fnm{Jean-Paul} \sur{Jamont}}\email{jean-paul.jamont@lcis.grenoble-inp.fr}

\author[1]{\fnm{Michel} \sur{Occello}}\email{michel.occello@lcis.grenoble-inp.fr}
% \equalcont{These authors contributed equally to this work.}

\author[2]{\fnm{Louis-Marie} \sur{Traonouez}}\email{louis-marie.traonouez@thalesgroup.com}
% \equalcont{These authors contributed equally to this work.}

\author[3]{\fnm{Paul} \sur{Theron}}\email{paul.theron@orange.fr}

\affil*[1]{\orgdiv{Univ. Grenoble Alpes, Grenoble INP}, \orgname{LCIS}, \orgaddress{\street{50 Rue Barthélémy de Laffemas}, \city{Valence}, \postcode{26000}, \state{Auvergne-Rhône-Alpes}, \country{France}}}

\affil[2]{\orgdiv{Thales LAS / IAS / La Ruche}, \orgaddress{\city{Rennes}, \country{France}}}

\affil[3]{\orgdiv{AICA IWG}, \orgaddress{\city{La Guillermie}, \country{France}}}

%%==================================%%
%% Sample for unstructured abstract %%
%%==================================%%

\abstract{The issue of explainability in Multi-Agent Reinforcement Learning is critical as we move toward real-world applications in complex environments. Ensuring safe and reliable deployment of MARL systems requires clear insights into how agents operate according to defined requirements.
    %
    This paper presents an approach to characterizing the behaviors of successfully trained agents by translating their histories into organizational specifications, such as roles or missions, that are interpretable at the collective level. Our method comprises two approaches: first, it uses given pattern and rule-based logic to identify pre-defined organizational specifications from histories; second, it uses general definitions of organizational specifications related to histories to infer new roles or missions, applying techniques like hierarchical clustering or dimensionality reduction.
    %
    We assess our approach in a predator-prey scenario to explicitize recurrent collective strategies in hunting and defense. It also comprises a comparative analysis of various parameters, including similarity measures and clustering techniques in order to better infer new organizational specifications. The results align with the expected roles and missions, demonstrating that our approach offers insights into the collective behavior of agents and aids in the design of Multi-Agent Systems.}

\keywords{Explainable AI, Multi-Agent Reinforcement Learning, Organizational Models, Multi-Agent System Design, MOISE+}

%%\pacs[JEL Classification]{D8, H51}

%%\pacs[MSC Classification]{35A01, 65L10, 65L12, 65L20, 65L70}

\maketitle

\section{Introduction}
\label{sec:intro}

% Context
Explainable Artificial Intelligence (XAI) has become a crucial factor for the broader acceptance of AI systems, especially in multi-agent settings where multiple agents collaborate to achieve complex objectives~\citep{doshivelez2017rigorous,gunning2019xai}. While substantial advancements have been made in explaining the behavior of individual agents~\citep{ribeiro2016classifier,lundberg2017unified}, the challenge of elucidating cooperative strategies and emergent organizational structures in Multi-Agent Reinforcement Learning (MARL) remains underexplored~\citep{busoniu2008survey}. This gap is significant given the growing importance of XAI in MARL as we move towards deploying MARL-based systems, where ensuring safety, reliability, and trustworthiness is paramount. To facilitate understanding of trained agents' behaviors, various approaches can be explored. In this paper, we propose investigating organizational models as a promising approach to provide a structured view on the collective actions of agents through organizational specifications.

% Problem
In MARL, a set of agents must learn to achieve goals that often require implicit cooperation and coordination. However, few studies have attempted to analyze the policies of these trained agents to explicitly define their cooperation through organizational specifications~\citep{albrecht2018survey,perolat2017pool}. This lack of explainability introduces three key theoretical and technical challenges in the field:
\begin{enumerate}
    \item The absence of systematic approaches for linking organizational specifications to agent behavior in MARL.
    \item The need for robust methods to infer organizational structures from agent trajectories.
    \item The difficulty in generalizing learned behaviors to new organizational contexts.
\end{enumerate}

% Contribution
To address the aforementioned challenges, we propose a dual approach that leverages the $\mathcal{M}OISE^+$ organizational model~\citep{hubner2007} as a foundational framework to make the cooperative aspects of observed agent behaviors explicit. The core idea is to establish links between the agents' histories and organizational specifications, such as roles and missions, enabling the translation of a joint history into a $\mathcal{M}OISE^+$ organizational description.

Within this framework, we introduce two complementary approaches to facilitate this translation:
\begin{itemize}
    \item \textbf{KOSIA (Knowledge Organizational Specification Identification Approach)}, which aligns agent histories with known organizational specifications using pattern-matching techniques.
    \item \textbf{GOSIA (General Organizational Specification Inference Approach)}, which infers new organizational specifications by leveraging general definitions of roles and missions, utilizing hierarchical clustering, representation techniques, and Large Language Models (LLMs).
\end{itemize}

Given that GOSIA was refined through empirical iterations, our contribution also includes an analysis of the various techniques and metrics considered during development, and an outline of the final settings based on our experimental criteria.

We applied both approaches to a series of scenarios in a predator-prey environment, where agents are required to adopt collective strategies for hunting or defense. The results demonstrate that our general approach enables getting description of roles, missions and other organizational specifications that align with expected ones for each scenario. Our approach does enhance the interpretability of agent behaviors at the collective level for this scenario. Furthermore, we discussed the generalization of our approach to various environments and next steps in that purpose.

% Outline
The remainder is organized as follows: \Autoref{sec:related} gives an overview of related works for XAI in MARL. \Autoref{sec:background} provides a fundamental background in the MARL and organizational to introduce the framework we proposed to link the MARL and organizational model for KOSIA and GOSIA. \Autoref{sec:kosia} introduces KOSIA and associated means available for users. \Autoref{sec:gosia} introduces GOSIA mainly as for the theoretical aspects. \Autoref{sec:experiment} describes the conducted experiments and discusses the first results of our empirical evaluation. \Autoref{sec:conclusion} concludes the paper and outlines future research directions.

\section{Related Works}
\label{sec:related}

Explainability in Reinforcement Learning (RL) has traditionally focused on single-agent settings, where the goal is to understand the decision-making process of an individual agent. Techniques such as model distillation, feature importance analysis, and policy summarization have been widely studied. However, in Multi-Agent Systems (MAS), where multiple agents interact and coordinate, the challenge of explainability becomes more complex due to the emergent behaviors that arise from agent interactions.

Organizational models, such as MOISE+, provide a framework for defining the roles, missions, and interactions among agents in MAS. These models have been used to guide the design and analysis of MAS, but their integration with MARL is less explored. Few works have attempted to align the organizational structures defined in models like MOISE+ with the learned behaviors of agents in MARL settings.

In the context of MARL, explicability is still an emerging field. Existing approaches often focus on post-hoc analysis, where the behaviors of trained agents are analyzed to identify patterns or strategies. However, these methods do not provide a systematic way to link these behaviors to higher-level organizational specifications, nor do they offer a framework for generalizing these behaviors to new contexts.

Our proposed approaches, KOSIA and GOSIA, aim to bridge this gap by offering systematic methods for linking agent behaviors to organizational specifications and for inferring new organizational structures from agent trajectories.

\section{MARL and Organizational Background}\label{sec:background}
TODO

\section{The KOSIA Approach}
\label{sec:kosia}

The KOSIA approach is designed to leverage known organizational specifications to interpret the behaviors of trained agents in MARL environments. The core idea is to use pattern-matching techniques to identify pre-defined roles, missions, and interactions within the agent histories.

\subsection{Conceptual Framework}

KOSIA operates on the assumption that the agent behaviors observed in the environment can be mapped to known organizational specifications. These specifications might include roles that agents are expected to play, missions they are tasked with, or social links that define their interactions. By identifying patterns within the agents' histories that match these specifications, KOSIA provides a way to interpret the learned behaviors in terms of the organizational model.

\subsection{Methodology}

The implementation of KOSIA involves several steps:
1. Extracting the histories of agents from the MARL environment.
2. Applying pattern-matching techniques to these histories to identify sequences of actions or states that correspond to known organizational roles or missions.
3. Mapping these identified patterns to the organizational model to provide an interpretation of the agent behaviors.

\subsection{Strengths and Limitations}

The primary strength of KOSIA lies in its ability to provide a clear, interpretable mapping between agent behaviors and organizational specifications. However, it is limited by the need for pre-defined specifications. If the organizational model does not cover all possible behaviors or if agents learn novel strategies that were not anticipated, KOSIA may fail to provide a comprehensive explanation.

\section{The GOSIA Approach}
\label{sec:gosia}

While KOSIA is effective when organizational specifications are known, GOSIA is designed to handle situations where these specifications are not predefined or need to be inferred from agent behaviors. GOSIA generalizes the concept of organizational specifications by using clustering and dimensionality reduction techniques to identify new roles or missions from the trajectories of agents.

\subsection{Conceptual Framework}

GOSIA builds on the idea that the behaviors of agents, as captured in their trajectories, can reveal underlying organizational structures even when these are not explicitly defined. By analyzing the similarities and differences between trajectories, GOSIA infers roles, missions, and interactions that are consistent with the observed behaviors.

\subsection{Methodology}

The GOSIA approach involves several key steps:
1. Representation of agent trajectories using techniques such as PCA, t-SNE, or autoencoders to capture the essential features of the behaviors.
2. Application of similarity measures, such as Dynamic Time Warping or Longest Common Subsequence, to compare different trajectories.
3. Clustering of similar trajectories using methods like Hierarchical Clustering or K-Means to identify common roles or missions.
4. Inference of organizational specifications based on the identified clusters, providing a generalized description of the agent behaviors.

\subsection{Analysis of Techniques for GOSIA}

Different techniques offer various advantages and challenges:
- Clustering techniques like Hierarchical Clustering are effective for identifying hierarchical roles but may struggle with noisy data.
- Similarity measures such as Dynamic Time Warping provide robust comparisons but can be computationally expensive.
- Representation techniques like autoencoders capture complex, non-linear relationships but require careful tuning.

The choice of techniques in GOSIA depends on the specific characteristics of the MARL environment and the desired level of granularity in the inferred organizational specifications.

\section{Experimental Setup}
\label{sec:experiment}

To evaluate the effectiveness of the KOSIA and GOSIA approaches, we conducted experiments in a predator-prey scenario within the PettingZoo framework. This environment was chosen because it presents clear roles (predator, prey) and interactions (hunting, evasion) that can be mapped to organizational specifications.

\subsection{Environment and Framework}

The predator-prey scenario involves multiple predators coordinating to capture prey, which in turn use various strategies to avoid capture. We integrated the KOSIA and GOSIA approaches within this framework to assess how well they could identify and infer the roles and missions of the agents.

\subsection{Evaluation Metrics}

We used a combination of quantitative and qualitative metrics to evaluate our approaches:
- Performance improvements: Measured by the cumulative reward achieved by the agents.
- Explainability: Assessed by the clarity and accuracy of the inferred organizational specifications.
- Generalizability: Tested by applying the inferred specifications to new, unseen environments.

\subsection{Results}

The results of our experiments showed that both KOSIA and GOSIA were able to effectively identify and infer organizational specifications that aligned with the expected roles and missions. KOSIA provided clear mappings when specifications were known, while GOSIA successfully inferred new structures in more complex scenarios.

\subsection{Discussion}

The experimental results highlight the strengths of KOSIA in providing direct interpretations of agent behaviors and the flexibility of GOSIA in discovering new organizational structures. However, both approaches have limitations: KOSIA's dependency on pre-defined specifications and GOSIA's sensitivity to the choice of clustering and similarity measures.

\section{Conclusion and Future Work}
\label{sec:conclusion}

In this paper, we introduced two approaches, KOSIA and GOSIA, to enhance the explainability of MARL systems by linking agent behaviors to organizational specifications. KOSIA excels in environments with well-defined roles and missions, while GOSIA offers a powerful tool for inferring new organizational structures from agent trajectories.

Future work will explore more sophisticated techniques for trajectory analysis, such as deep learning-based sequence models, and extend our approaches to more complex, real-world scenarios. We also plan to investigate the integration of these methods into the design process of Multi-Agent Systems to improve both their performance and interpretability.


%% References
\bibliography{references}

%%===========================================================================================%%
%% If you are submitting to one of the Nature Portfolio journals, using the eJP submission   %%
%% system, please include the references within the manuscript file itself. You may do this  %%
%% by copying the reference list from your .bbl file, paste it into the main manuscript .tex %%
%% file, and delete the associated \verb+\bibliography+ commands.                            %%
%%===========================================================================================%%

%% if required, the content of .bbl file can be included here once bbl is generated
%%\input sn-article.bbl

\end{document}
