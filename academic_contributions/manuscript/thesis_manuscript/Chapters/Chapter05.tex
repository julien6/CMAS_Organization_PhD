%************************************************
\chapter{Experimentation et évaluation}\label{ch:case_studies} % $\mathbb{ZNR}$
%************************************************

\begin{itemize}
    \item \textbf{Ici l'idée est de créer un modèle AICA intégré dans notre méthode et qui pourra être utilisé comme un SMA polyvalent capable d'être ajusté pour des scénarios différents}
\end{itemize}


\section{Intégration d'un agent AICA dans CybSMADE}
\begin{itemize}
    \item Traduire au moins dans l'idée les composants de SMACARA comme des spécifications de l'organisation qui seront prises en compte pour générer une organisation dans CybSMADA
    \item Expliquer comment nous avons pris en compte l'AICA dans l'outil afin qu'il soit utilisable directement
    \item Problèmes d'implémentation dus à la complexité de l'architecture
\end{itemize}


\section{Expériences à travers trois études de cas}
\begin{itemize}
    \item Présentation de 3 cas d'études : Essaim de drones, réseau d'entreprise et Kubernetes
    \item Sur le modèle du tutoriel en utilisant l'outil CybSMADE
    \item Montrer comment nous utilisons concrètement notre modèle avec CybSMADE pour comprendre et définir le problème dans chaque étude de cas
\end{itemize}

\subsection{Un scénario d'infrastructure d'entreprise}
\subsection{Un scénario d'essaim de drones}
\subsection{Un scénario d'orchestration Kubernetes}


\section{Résultats et discussion}
\begin{itemize}

    \item Montrer comment nous utilisons CybSMADE pour résoudre le problème
          \begin{itemize}
              \item En utilisant également le modèle AICA
          \end{itemize}
\end{itemize}

\section{Tendances générales qui ressortent des résultats et de la synthèse}
\begin{itemize}
    \item Discuter le niveau de l'impact de la méthode
          \begin{itemize}
              \item Sans spécifications initiales
              \item Avec spécifications initiales (incluant l'AICA)
          \end{itemize}
    \item Discuter des organisations qui semblent pertinentes avec une analyse quantitative sur les 3 cas d'étude
    \item Discuter des résultats et la cohérence de ces-derniers par rapport à d'autres résultats
\end{itemize}