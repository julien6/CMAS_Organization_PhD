%************************************************
\chapter{Introduction}\label{ch:introduction}
%************************************************

\section{An Increasing Need for Decentralized Cyberdefense}
\begin{itemize}
    \item AICA, besoins nouveaux, IoT/IoBT, etc.
    \item Approche centralisée peu adaptée, etc. pour des raisons d’interruptions de communication, hétérogénéité des SI, etc.
    \item Une approche MA pourrait être appliquée -> Système Multi-Agents de Cyberdéfense (SMAC)
\end{itemize}

\section{The Challenges of a Multi-Agent Cyberdefense}
\begin{itemize}
    \item Mais sujet nouveau : pas de modélisation, travaux formels…
\end{itemize}

\section{Research Question and Objectives}
\begin{itemize}
    \item Quelle méthode pour concevoir un SMAC qui atteint ses objectifs de cyberdéfense tout en satisfaisant les contraintes de déploiement et opérationelles du système hôte qu'il doit défendre ?
\end{itemize}

\section{Positioning and Contribution of this thesis}
\begin{itemize}
    \item Un ensemble d’agents collaboratifs répond effectivement aux nouveaux besoins, etc. mieux que des solutions centralisées
    \item Une méthode modélisant le "domaine" (environnement réseau + actions/observations possibles des red/blue/green teams), du "problème" (objectif de cyberdéfense / blue team + contraintes opérationelles/déploiement) sous forme d'un **problème d'optimisation sous-contraintes*\item ; permet de fournir des moyens objectifs d’évaluer de façon consistante si le SMA tient ses promesses dans plusieurs scénarios d’attaque…
\end{itemize}

\section{Manuscript Organization}

The remaining parts of the manuscript are as follows, Chapter 2 begins by an introduction of the concepts of multi-agent systems of embed-ded agents and security architecture. It is continued by a description of the first contribution of this thesis, the systematic mapping study.
This chapter includes a detailed presentation of the employed method-ology, the results which are answers to the three research questions:
(i) “What are the main security properties studied in multi-embed-ded-agent systems?”; (ii) “What are the specific technical solutions for securing multi-embedded-agent systems?”; (iii) “Which parts of a global security architecture for multi-embedded-agent systems are studied?”, and a discussion about the implications of these results on this thesis. In particular, it led us to focus on key management as the literature lacks relevant solutions related to this topic.

Before addressing the remaining contributions, we first present in Chapter 3 a second contribution, a simulation tool we specifically developed to allow us to prototype and validate our approach. After presenting a comparison study of the current tools that are available and why they do not suit our context, we explain the implementation choices made and the concepts we followed to create such a tool.

Chapter 4 details the third and main contribution, Multi-Agent Key Infrastructure (MAKI), a PKI designed for multi-agent systems of embedded agents. The chapter consists of a literature review of PKI used in decentralized systems, a description of the infrastructure we propose and results of validation using both simulation and model-checking approaches.
The fourth contribution is presented in Chapter 5. It aims at mitigat-ing the absence of a central authority sharing the certificates between the agents by deploying a blockchain tailored to the needs of multi-agent systems of embedded agents. Like the previous one, this chapter includes a literature review of the related works concerning the estab-lishment of a shared database in a decentralized system, a description of the contribution and validation results.

The manuscript ends with Chapter 6. This last chapter summarizes the main achievements of the presented works and offers several future works and research perspectives.