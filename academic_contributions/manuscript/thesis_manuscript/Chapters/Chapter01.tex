%************************************************
\chapter{Introduction}\label{ch:introduction}
%************************************************

\section{Genèse / contexte du sujet}

\begin{itemize}
    \item AICA, besoins nouveaux, IoT/IoBT, etc.
    \item Approche centralisée peu adaptée, etc. pour des raisons d’interruptions de communication, hétérogénéité des SI, etc.
    \item Une approche MA pourrait être appliquée -> Système Multi-Agents de Cyberdéfense (SMAC)
    \item Mais sujet nouveau : pas de modélisation, travaux formels…
\end{itemize}

\section{Problématique générale}
\begin{itemize}

    \item Quelle méthode pour concevoir un SMAC qui atteint ses objectifs de cyberdéfense tout en satisfaisant les contraintes de déploiement et opérationelles du système hôte qu'il doit défendre ?
\end{itemize}

\section{Positionnement et thèse défendue}
\begin{itemize}

    \item Un ensemble d’agents collaboratifs répond effectivement aux nouveaux besoins, etc. mieux que des solutions centralisées
    \item Une méthode modélisant le "domaine" (environnement réseau + actions/observations possibles des red/blue/green teams), du "problème" (objectif de cyberdéfense / blue team + contraintes opérationelles/déploiement) sous forme d'un **problème d'optimisation sous-contraintes*\item ; permet de fournir des moyens objectifs d’évaluer de façon consistante si le SMA tient ses promesses dans plusieurs scénarios d’attaque…
\end{itemize}
