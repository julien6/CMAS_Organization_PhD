%************************************************
\chapter{Introduction}\label{ch:introduction}
%************************************************

\section{A Cyberdefense Context With Future And New Challenges}

In today's increasingly connected world, the complexity and scope of cyber threats are evolving at an unprecedented rate. The transition to decentralized and distributed systems, driven by advancements in the Internet of Things (IoT), cloud computing, and mobile networks, has opened up new vulnerabilities that traditional centralized defense mechanisms are ill-equipped to handle. This new paradigm demands innovative approaches to Cyberdefense that are both agile and resilient in the face of diverse and sophisticated attacks.

\subsection{The Rise of Decentralized Cyber Threats}

As the digital landscape expands, so does the attack surface available to malicious actors. Modern cyber threats are no longer confined to isolated incidents targeting specific systems; instead, they can propagate across vast networks, affecting multiple systems simultaneously. This shift is largely due to the decentralization of both infrastructure and attacks. Decentralized systems, such as blockchain networks, peer-to-peer architectures, and distributed cloud platforms, have become increasingly common in both civilian and military contexts. While these systems offer enhanced robustness and scalability, they also introduce new security challenges.

For example, in a decentralized financial system based on blockchain technology, a single security breach can compromise an entire network of transactions across the globe. Similarly, attacks on decentralized IoT networks, where millions of devices communicate autonomously, can cause widespread disruptions, ranging from critical infrastructure failures to breaches in personal data privacy. The WannaCry ransomware attack of 2017, which exploited vulnerabilities in a decentralized network of systems, highlighted the devastating impact of such distributed attacks on both public and private sectors.

The rise of decentralized cyber threats is further complicated by the increasing use of sophisticated attack vectors such as hardware Trojans, advanced persistent threats (APTs), and zero-day vulnerabilities. These attacks can remain dormant within decentralized networks for extended periods, waiting for the opportune moment to strike. Traditional centralized defense mechanisms, which rely on a single point of control, are often too slow to respond to such dynamic and distributed threats. This highlights the need for new defense strategies that can operate in a decentralized manner, responding autonomously and in real-time to emerging threats.

\subsection{Challenges in Modern Cyberdefense}

In this context, several key challenges must be addressed to ensure the security of decentralized systems. These challenges can be categorized into the following criteria, each representing a critical aspect of modern Cyberdefense:

\paragraph{C1) Decentralization:}
The shift towards decentralized systems requires a corresponding shift in defense strategies. Centralized security architectures, where decisions are made by a single authority, are no longer sufficient. Instead, security must be distributed across the network, with individual components capable of autonomous decision-making. This decentralization of defense introduces challenges related to coordination, communication, and trust among autonomous agents.

\paragraph{C2) Distribution:}
In a distributed system, components are geographically and logically dispersed. This distribution complicates traditional security measures, as it is difficult to monitor and control all parts of the network simultaneously. Ensuring the integrity and availability of data across distributed networks requires new approaches to threat detection and response that can operate at scale and in real-time.

\paragraph{C3) Autonomy:}
As systems become more complex and distributed, the need for autonomous Cyberdefense mechanisms grows. Autonomous agents must be capable of detecting and responding to threats without relying on human intervention. However, designing such autonomous systems introduces challenges related to decision-making under uncertainty, as well as ensuring that these systems can adapt to evolving threats.

\paragraph{C4) Speed and Scalability:}
Cyberattacks are becoming faster and more aggressive, capable of compromising systems within seconds. Defense mechanisms must therefore be able to respond at an equally rapid pace. Moreover, as networks grow in size and complexity, security solutions must be scalable, ensuring that they can protect large, distributed infrastructures without becoming a bottleneck themselves.

\paragraph{C5) Security and Safety:}
Ensuring the safety and security of decentralized systems requires a holistic approach that considers not only technical vulnerabilities but also human factors. Phishing, social engineering, and insider threats remain significant concerns, particularly in environments where trust is decentralized and spread across multiple entities.

\paragraph{C6) Machine Learning and Adaptability:}
The use of machine learning (ML) and artificial intelligence (AI) in Cyberdefense is on the rise, offering new capabilities for threat detection and prediction. However, these technologies also introduce new challenges. ML models must be trained on vast amounts of data, and they must be capable of adapting to new and previously unseen attack vectors. Additionally, adversarial attacks against ML models pose a growing threat, as attackers seek to manipulate the very algorithms designed to protect systems.

\subsection{Emerging Threats and the Need for Innovative Defense Mechanisms}

Looking to the future, the landscape of cyber threats is expected to become even more complex, driven by advancements in technology and the increasing interconnectivity of global systems. Quantum computing, for example, poses a significant threat to current cryptographic standards, potentially rendering many existing security protocols obsolete. Similarly, the proliferation of AI-driven attacks, where adversaries use AI to automate and optimize their attack strategies, will require equally sophisticated defense mechanisms.

In this rapidly evolving environment, it is clear that traditional centralized defense strategies are no longer sufficient. Instead, there is a growing need for decentralized, distributed, and autonomous defense mechanisms that can operate at scale and adapt to new threats in real-time. Multi-Agent Systems (MAS), with their ability to coordinate the actions of multiple autonomous agents, offer a promising solution to these challenges.

The concept of a Cyberdefense Multi-Agent System (MAS) will be explored in the following sections, with a focus on how such systems can address the challenges outlined above and provide a robust, scalable, and adaptive defense against future cyber threats.



\section{The Idea Of A Cyberdefense Multi-Agent System}

\subsection{The AICA Initiative: A Pioneer in Cyberdefense Automation}

The increasing complexity and frequency of cyber threats, particularly in military and critical infrastructure domains, have driven the development of more advanced defense systems. One pioneering initiative in this field is the Autonomous Intelligent Cyberdefense Agent (AICA). AICA represents an innovative approach to Cyberdefense, focusing on the creation of autonomous, intelligent agents capable of detecting, responding to, and mitigating cyber threats in real-time.

AICA agents are designed to operate in highly dynamic and contested environments, such as those found in military operations, where the cyber threat landscape is constantly evolving. The core idea behind AICA is to create a distributed network of autonomous agents that work collaboratively to protect critical systems. Each agent is equipped with advanced artificial intelligence (AI) capabilities, allowing it to autonomously assess its environment, identify potential threats, and take appropriate action to neutralize them.

The decentralized nature of AICA agents is a key advantage in environments where centralized control is impractical or vulnerable to attack. Instead of relying on a central authority, AICA agents communicate with one another to share information and coordinate their actions. This distributed approach enhances the system's resilience, as it reduces the likelihood that the entire defense system can be compromised through a single point of failure.

AICA's design principles are based on the notion of autonomous decision-making. Each agent can independently monitor the system, detect anomalies, and implement countermeasures, all while adapting to the evolving threat environment. This adaptability is crucial in modern cyber warfare, where attackers constantly develop new techniques to bypass traditional security measures. By leveraging AI and machine learning, AICA agents can evolve their strategies over time, ensuring that the defense system remains effective against emerging threats.

\subsection{From AICA to Multi-Agent Systems in Cyberdefense}

While AICA provides a specific implementation of an autonomous Cyberdefense system, the broader concept of Multi-Agent Systems (MAS) offers a general framework that can be applied across various domains of Cyberdefense. MAS are systems composed of multiple interacting autonomous agents that collaborate to achieve a common goal. In the context of Cyberdefense, this goal is the protection of critical systems and networks from cyber threats.

The generalization from AICA to MAS allows for greater flexibility in the design and deployment of Cyberdefense systems. MAS can be tailored to specific environments, whether it be military networks, corporate infrastructures, or decentralized IoT ecosystems. The principles that guide the operation of AICA—decentralization, autonomy, adaptability—are also at the core of MAS, making them an ideal framework for addressing the challenges of modern Cyberdefense.

In a MAS for Cyberdefense, each agent operates as an independent entity, capable of making decisions based on local information. However, these agents are not isolated; they communicate with one another to share threat intelligence and coordinate their defensive actions. This communication is crucial in distributed environments, where threats can emerge simultaneously in multiple locations, requiring a coordinated response.

The decentralized nature of MAS is particularly well-suited to defending large-scale, distributed networks. Unlike centralized systems, which rely on a single point of control, MAS distribute the responsibility for defense across the entire network. This distribution not only enhances the system's resilience but also allows it to scale more effectively. As the network grows, additional agents can be deployed without the need for significant reconfiguration of the overall system.

\subsection{How MAS Address the Challenges of Cyberdefense}

The challenges of modern Cyberdefense, as outlined in the previous section, can be effectively addressed through the key features of MAS. These features, which include decentralization, distribution, autonomy, speed, security, and adaptability, align closely with the needs of a robust and scalable defense system. Here, we explore how MAS can tackle each challenge in detail.

\paragraph{Decentralization and Distribution:}
MAS inherently operate in a decentralized manner, with no single point of failure. This makes them particularly effective in distributed environments where attacks can occur at multiple points simultaneously. Each agent in a MAS can independently monitor and protect a specific segment of the network, ensuring that even if one part of the system is compromised, the rest remains secure.

\paragraph{Autonomy and Speed:}
The autonomy of MAS agents allows them to detect and respond to threats in real-time, without waiting for instructions from a central authority. This speed is critical in Cyberdefense, where the time between detecting an attack and responding to it can determine the extent of the damage. By operating autonomously, MAS agents can neutralize threats as soon as they are identified, minimizing the impact on the system.

\paragraph{Security and Resilience:}
MAS enhance the security of distributed systems by decentralizing defensive capabilities. This decentralization makes it more difficult for attackers to compromise the entire system, as they would need to defeat multiple independent agents rather than a single centralized defense. Additionally, the collaboration between agents allows for more comprehensive threat detection, as each agent contributes its local knowledge to the global defense effort.

\paragraph{Adaptability and Learning:}
One of the most significant advantages of MAS in Cyberdefense is their ability to adapt to new threats. Through the use of machine learning and AI, MAS agents can continuously improve their threat detection and response strategies. This adaptability is crucial in the face of evolving cyber threats, where attackers are constantly developing new methods to breach security systems. MAS agents can learn from past experiences and adjust their behavior accordingly, ensuring that the defense system remains effective over time.

\subsection{Towards a General Cyberdefense MAS Framework}

Building on the principles established by AICA, a general framework for MAS in Cyberdefense can be developed. This framework would consist of multiple layers of defense, with agents operating at different levels of the system, from individual devices to entire network segments. Each layer would have its own set of agents responsible for monitoring, detecting, and responding to threats within their specific domain.

At the lowest level, agents could be deployed on individual devices, monitoring for signs of compromise such as unusual network traffic or unauthorized access attempts. These agents would communicate with one another to share information and coordinate their responses. At higher levels, agents could be responsible for overseeing larger segments of the network, ensuring that any threats detected at the device level do not propagate further into the system.

This layered approach to defense allows for more comprehensive coverage of the entire network. By deploying agents at multiple levels, the system can detect and respond to threats more quickly and effectively than a traditional centralized security system. Furthermore, the adaptability of MAS agents ensures that the system can evolve in response to new threats, maintaining its effectiveness over time.

\subsection{Conclusion: The Future of MAS in Cyberdefense}

The concept of a Cyberdefense Multi-Agent System represents a significant advancement in the field of cybersecurity. By leveraging the principles of decentralization, autonomy, and adaptability, MAS offer a robust and scalable solution to the challenges of modern Cyberdefense. The AICA initiative serves as a concrete example of how these principles can be applied in practice, providing valuable insights into the design and deployment of MAS in critical environments.

As cyber threats continue to evolve, the need for decentralized, autonomous defense systems will only grow. MAS provide a flexible framework that can be adapted to a wide range of applications, from military operations to corporate networks and beyond. The following chapters will explore the theoretical foundations and practical applications of MAS in greater detail, offering a comprehensive guide to developing and deploying these systems in real-world Cyberdefense scenarios.



\section{Concepts in Multi-Agent Systems And Organization}

\subsection{Introduction to Multi-Agent Systems (MAS)}

Multi-Agent Systems (MAS) are a class of distributed systems composed of multiple interacting autonomous agents. Each agent in a MAS is an independent entity that can perceive its environment, make decisions, and act upon its environment to achieve specific goals. The agents in a MAS typically collaborate, compete, or coexist to achieve both individual and collective objectives.

The concept of MAS draws from various disciplines, including artificial intelligence, distributed computing, and game theory. MAS are used in a wide range of applications, from robotic systems and automated trading to network management and, most importantly in the context of this work, Cyberdefense. The decentralized nature of MAS makes them particularly suitable for complex, dynamic, and distributed environments where centralized control may be impractical or ineffective.

\subsection{Key Characteristics of Multi-Agent Systems}

A MAS can be characterized by several key features that distinguish it from other types of systems. These features include autonomy, distribution, communication, coordination, and adaptability. Below, we explore each of these characteristics in detail:

\paragraph{Autonomy:}
Each agent in a MAS operates autonomously, meaning that it can make decisions independently of other agents. This autonomy is crucial in environments where agents must respond to local conditions without waiting for instructions from a central authority. Autonomy enables agents to function effectively even in the absence of global knowledge or communication with other agents.

\paragraph{Distribution:}
In a MAS, the agents are distributed across the environment, both geographically and logically. This distribution allows the system to scale effectively, as additional agents can be deployed to cover new areas or tasks. It also enhances the system's resilience, as the failure of a single agent does not necessarily compromise the entire system. In the context of Cyberdefense, this distribution is particularly important for monitoring large networks and responding to threats in real-time.

\paragraph{Communication:}
Although agents in a MAS operate autonomously, they often need to communicate with one another to achieve their goals. Communication can take many forms, including direct message passing, broadcast communication, or even indirect communication through the environment (e.g., stigmergy). Effective communication protocols are essential for coordination and cooperation among agents, enabling them to share information, negotiate, and synchronize their actions.

\paragraph{Coordination:}
Coordination refers to the process by which agents align their actions to achieve a common goal. In a MAS, coordination can be achieved through various mechanisms, including centralized planning, distributed negotiation, or emergent behavior. Coordination is particularly important in environments where agents must work together to solve complex problems, such as defending a network against a coordinated cyber attack.

\paragraph{Adaptability:}
One of the strengths of MAS is their ability to adapt to changing environments. Agents can learn from their experiences, adjust their strategies, and evolve their behavior over time. This adaptability is especially valuable in dynamic environments, such as those found in cybersecurity, where threats are constantly evolving and new challenges arise regularly. Techniques such as reinforcement learning, genetic algorithms, and adaptive control are often employed to enhance the adaptability of agents in a MAS.

\subsection{Types of Multi-Agent Systems}

MAS can be classified into different types based on the nature of the agents and the interactions among them. Below are some common classifications of MAS:

\paragraph{Homogeneous vs. Heterogeneous MAS:}
In a homogeneous MAS, all agents are identical in terms of their capabilities, goals, and behavior. This simplicity can make coordination easier, as all agents follow the same rules and can substitute for one another. In contrast, a heterogeneous MAS consists of agents with different capabilities, roles, and goals. Heterogeneous MAS are often more powerful and flexible, as they can leverage the diverse skills of different agents to tackle complex problems. However, coordination in heterogeneous MAS can be more challenging due to the varying interests and abilities of the agents.

\paragraph{Cooperative vs. Competitive MAS:}
In a cooperative MAS, agents work together to achieve a shared goal. Cooperation is often facilitated through communication and coordination mechanisms that allow agents to share information and synchronize their actions. Cooperative MAS are commonly used in applications such as robotics, where agents need to collaborate to accomplish tasks like exploration or object manipulation. In contrast, a competitive MAS involves agents with conflicting goals. In this setting, agents compete for resources or influence, and game-theoretic strategies are often used to model their interactions. Competitive MAS are prevalent in domains such as economics, security, and gaming.

\paragraph{Static vs. Dynamic MAS:}
A static MAS operates in an environment where the agents and their relationships remain constant over time. This stability simplifies the design and analysis of the system, as the interactions among agents can be predicted and optimized in advance. In contrast, a dynamic MAS operates in an environment where agents can enter, leave, or change their roles and relationships over time. Dynamic MAS require more flexible and adaptive strategies, as agents must continuously adjust their behavior to accommodate changes in the environment. Cybersecurity is a prime example of a dynamic environment, where new threats emerge and existing threats evolve, requiring agents to adapt in real-time.

\subsection{Organization in Multi-Agent Systems}

The organization of agents within a MAS is a critical factor that influences the system's performance, scalability, and resilience. The term "organization" refers to the structure and coordination mechanisms that govern the interactions among agents. Different organizational structures are appropriate for different types of MAS, depending on the complexity of the environment, the nature of the agents, and the goals of the system.

\paragraph{Hierarchical Organization:}
In a hierarchical organization, agents are arranged in a tree-like structure, with higher-level agents overseeing the activities of lower-level agents. This type of organization is commonly used in situations where a central authority is needed to coordinate the actions of subordinate agents. Hierarchical organizations are relatively easy to manage and can be efficient in static environments where the relationships among agents do not change frequently. However, they can be less effective in dynamic environments, where the centralized control may become a bottleneck or a single point of failure.

\paragraph{Flat Organization:}
In a flat organization, all agents are considered equal, with no hierarchical structure. Agents communicate and coordinate directly with one another, making decisions based on local information. Flat organizations are highly decentralized and can be more resilient than hierarchical structures, as they do not rely on a central authority. However, coordinating the actions of agents in a flat organization can be more challenging, especially in large-scale systems where communication and decision-making need to be synchronized across the network.

\paragraph{Holonic Organization:}
Holonic organization is a hybrid structure that combines elements of both hierarchical and flat organizations. In a holonic MAS, agents are grouped into clusters, called holons, which operate as autonomous units. Each holon can be organized hierarchically internally, but the interactions among holons are more decentralized. This approach allows for greater flexibility and scalability, as holons can operate independently while still contributing to the overall goals of the system. Holonic organizations are particularly useful in dynamic environments, where different parts of the system may need to adapt to changing conditions at different rates.

\subsection{Coordination Mechanisms in MAS}

Coordination is a fundamental aspect of MAS, enabling agents to align their actions to achieve common objectives. Several coordination mechanisms have been developed to facilitate this process, each with its strengths and limitations. Below, we discuss some of the most commonly used coordination mechanisms in MAS:

\paragraph{Centralized Coordination:}
In centralized coordination, a single agent (or a small group of agents) is responsible for making decisions and assigning tasks to other agents. This approach can be efficient in environments where global knowledge is available and decisions need to be optimized for the entire system. However, centralized coordination can become a bottleneck in large-scale or dynamic environments, where the central agent may struggle to process all the necessary information in a timely manner.

\paragraph{Distributed Coordination:}
Distributed coordination relies on decentralized decision-making, where each agent makes decisions based on local information and interacts with other agents to achieve overall system goals. This approach is more scalable and resilient than centralized coordination, as it does not rely on a single point of control. However, distributed coordination can be more complex to implement, as it requires sophisticated communication and negotiation protocols to ensure that agents work together effectively.

\paragraph{Negotiation-Based Coordination:}
In negotiation-based coordination, agents communicate with one another to reach agreements on how to allocate tasks and resources. Negotiation can take many forms, including auction-based mechanisms, contract net protocols, and bargaining strategies. This approach is particularly useful in environments where agents have competing interests or where the allocation of resources needs to be optimized dynamically. Negotiation-based coordination is often used in competitive MAS, where agents must balance cooperation with competition.

\paragraph{Emergent Coordination:}
Emergent coordination occurs when the behavior of the MAS as a whole emerges from the interactions among individual agents, without the need for explicit coordination mechanisms. This approach leverages the principles of self-organization, where agents follow simple local rules that lead to complex global behavior. Emergent coordination is often seen in swarm systems, where agents exhibit collective behavior such as flocking, foraging, or formation control. While emergent coordination can be highly efficient in some environments, it can also be unpredictable and difficult to control, especially in systems where precise outcomes are required.

\subsection{Applications of MAS in Cyberdefense}

Multi-Agent Systems have been widely applied in the field of cybersecurity, where their decentralized nature and adaptability make them well-suited for defending against complex and evolving threats. Below, we highlight some key applications of MAS in Cyberdefense:

\paragraph{Intrusion Detection and Response:}
MAS are often used in intrusion detection systems (IDS) to monitor network traffic and detect anomalous behavior that may indicate a cyber attack. Agents in the IDS can be deployed at different points in the network, each responsible for monitoring a specific segment of traffic. When an intrusion is detected, agents can collaborate to identify the source of the attack and implement appropriate countermeasures. The decentralized nature of MAS allows for more robust and scalable intrusion detection, as each agent can operate independently while still contributing to the overall security of the network.

\paragraph{Malware Detection and Mitigation:}
MAS can also be used to detect and mitigate malware infections in distributed systems. Agents can monitor the behavior of devices and applications, looking for signs of malware activity such as unusual resource usage, unauthorized access attempts, or suspicious network communications. When malware is detected, agents can take action to isolate the infected device, remove the malware, or block further communication from the malicious source. The distributed nature of MAS ensures that malware detection and mitigation can occur across the entire network, even in the presence of sophisticated and stealthy threats.

\paragraph{Distributed Denial of Service (DDoS) Protection:}
DDoS attacks, where multiple compromised devices are used to overwhelm a target with traffic, are a significant threat to modern networks. MAS can be used to protect against DDoS attacks by distributing defensive agents across the network. These agents can detect the onset of a DDoS attack by monitoring traffic patterns and identifying abnormal spikes in traffic. Once an attack is detected, the agents can collaborate to mitigate the impact by blocking or rerouting malicious traffic, adjusting resource allocations, and coordinating with other agents to ensure the continued availability of critical services.

\paragraph{Cyber-Physical Systems Security:}
Cyber-physical systems (CPS), such as smart grids, industrial control systems, and autonomous vehicles, are increasingly targeted by cyber attacks. MAS can play a critical role in securing these systems by providing real-time monitoring, threat detection, and response capabilities. In a CPS environment, agents can be deployed at various points in the system to monitor both the physical and cyber aspects of the system's operation. When a threat is detected, agents can coordinate their responses to ensure that both the physical and cyber components of the system remain secure and operational.

\subsection{Challenges and Future Directions}

Despite the many advantages of MAS, there are several challenges that need to be addressed to fully realize their potential in Cyberdefense. These challenges include:

\paragraph{Scalability:}
As MAS are deployed in larger and more complex environments, scalability becomes a critical concern. Ensuring that agents can operate effectively at scale requires efficient communication, coordination, and resource management mechanisms. Future research in MAS should focus on developing scalable architectures that can support thousands or even millions of agents without degrading system performance.

\paragraph{Security of MAS:}
While MAS are used to enhance security, they are also vulnerable to attacks. Compromised agents can be used to disrupt the system, manipulate information, or coordinate attacks against other agents. Ensuring the security of MAS themselves is a significant challenge, requiring robust authentication, encryption, and trust mechanisms. Additionally, the decentralized nature of MAS can make it difficult to detect and respond to insider threats, where compromised agents behave maliciously while remaining within the system's normal operating parameters.

\paragraph{Coordination in Dynamic Environments:}
Dynamic environments, such as those found in Cyberdefense, present unique challenges for coordination in MAS. Agents must be able to adapt to changing conditions, such as new threats, evolving attack strategies, and shifting network topologies. Developing coordination mechanisms that are both flexible and efficient in dynamic environments is an ongoing research challenge.

\paragraph{Human-Agent Interaction:}
As MAS are increasingly integrated into cybersecurity operations, the interaction between human operators and autonomous agents becomes more important. Ensuring that agents can provide meaningful explanations of their actions, decisions, and recommendations is critical for building trust and enabling effective human-agent collaboration. Future research should explore methods for improving the transparency, interpretability, and usability of MAS in Cyberdefense contexts.



\section{State-of-the-Art in Distributed and Decentralized Cyberdefense}

\subsection{Introduction to Distributed and Decentralized Cyberdefense}

In response to the increasing complexity and sophistication of cyber threats, the field of Cyberdefense has seen a shift from centralized security architectures to distributed and decentralized models. Traditional centralized approaches, which rely on a single point of control for decision-making and threat mitigation, have proven inadequate in the face of distributed denial-of-service (DDoS) attacks, ransomware, and other multi-vector threats that can simultaneously target multiple parts of a system.

Distributed and decentralized Cyberdefense systems, on the other hand, offer greater resilience by spreading defensive capabilities across the network. This approach reduces the reliance on a central authority, minimizes single points of failure, and allows for faster, localized responses to threats. In this section, we explore the current state of research in distributed and decentralized Cyberdefense, focusing on key approaches, technologies, and applications.

\subsection{Centralized vs. Decentralized Cyberdefense Architectures}

Centralized Cyberdefense architectures have been the traditional choice for securing networks and systems. In these architectures, a central security node or server manages all aspects of security, including threat detection, incident response, and policy enforcement. While this approach offers simplicity and ease of management, it suffers from several limitations:

\begin{itemize}
    \item \textbf{Single Point of Failure:} The central security node becomes a critical point of failure. If compromised, the entire network is at risk.
    \item \textbf{Scalability Issues:} As networks grow larger, centralized systems struggle to manage the increased volume of data and security events.
    \item \textbf{Latency and Response Time:} Centralized systems can introduce latency in threat detection and response, particularly in geographically dispersed networks.
\end{itemize}

In contrast, decentralized architectures distribute security functions across multiple nodes in the network. Each node is responsible for monitoring its own environment, detecting threats, and taking action independently or in collaboration with other nodes. Decentralized Cyberdefense offers several advantages:

\begin{itemize}
    \item \textbf{Improved Resilience:} The absence of a single point of failure enhances the overall resilience of the system.
    \item \textbf{Scalability:} Decentralized systems can scale more effectively as new nodes are added to the network.
    \item \textbf{Faster Response Times:} Localized threat detection and response enable faster mitigation of attacks.
\end{itemize}

\subsection{Key Approaches in Distributed and Decentralized Cyberdefense}

The development of distributed and decentralized Cyberdefense strategies has led to the emergence of several key approaches, including collaborative intrusion detection systems, blockchain-based security, peer-to-peer networks, and the use of multi-agent systems (MAS). Below, we explore these approaches in greater detail.

\subsubsection{Collaborative Intrusion Detection Systems (CIDS)}

Collaborative Intrusion Detection Systems (CIDS) represent a decentralized approach to network security, where multiple intrusion detection systems (IDS) work together to detect and mitigate threats. In a CIDS, each IDS monitors a specific segment of the network and shares information with other IDS nodes. This collaboration allows the system to identify distributed attacks that may not be detectable by a single IDS operating in isolation.

CIDS can be organized in different topologies, such as peer-to-peer (P2P) networks, hierarchical networks, or hybrid models. In a P2P CIDS, all nodes are equal, and information is shared directly between them. In a hierarchical CIDS, nodes are organized in tiers, with higher-level nodes aggregating and analyzing data from lower-level nodes. Hybrid CIDS combine elements of both P2P and hierarchical models to achieve a balance between decentralization and efficiency.

One of the main challenges in CIDS is ensuring the timely and accurate sharing of information between nodes, particularly in large and dynamic networks. Additionally, issues such as trust management, false positives, and data privacy must be addressed to ensure the effectiveness of CIDS in real-world applications.

\subsubsection{Blockchain-Based Security}

Blockchain technology, originally developed for cryptocurrency, has gained significant attention as a potential solution for decentralized Cyberdefense. Blockchain's decentralized and immutable nature makes it an attractive option for securing distributed systems and ensuring the integrity of transactions, data, and communications.

In the context of Cyberdefense, blockchain can be used to secure communications between nodes, authenticate users and devices, and maintain an audit trail of security events. One key advantage of blockchain is its resistance to tampering, as any attempt to alter data on the blockchain would require consensus from the majority of nodes in the network.

Blockchain-based security has been proposed for several applications, including securing IoT networks, protecting supply chains, and enhancing the security of cloud services. However, challenges such as scalability, energy consumption, and latency need to be addressed before blockchain can be widely adopted in Cyberdefense.

\subsubsection{Peer-to-Peer (P2P) Networks}

Peer-to-peer (P2P) networks are inherently decentralized, with each node in the network acting as both a client and a server. P2P networks have been used in various Cyberdefense applications, particularly in distributed file sharing, content distribution, and secure communication.

In a P2P network, each node is responsible for managing its own security, including detecting threats and protecting against attacks. Nodes can share information with one another to coordinate their defenses, but there is no central authority controlling the network. This decentralized approach offers significant resilience, as the compromise of a single node does not necessarily threaten the entire network.

However, P2P networks also face challenges in terms of security and trust. Since there is no central authority, nodes must establish trust relationships with one another, which can be difficult in open networks where malicious actors may be present. Additionally, P2P networks can be vulnerable to attacks such as Sybil attacks, where an attacker creates multiple fake identities to influence the network.

\subsubsection{Multi-Agent Systems (MAS)}

Multi-Agent Systems (MAS) have emerged as a powerful approach to distributed Cyberdefense. In a MAS, multiple autonomous agents work together to monitor, detect, and respond to threats in a decentralized manner. Each agent operates independently, but can collaborate with other agents to achieve a common goal, such as protecting a network from cyber attacks.

MAS are particularly well-suited for dynamic and distributed environments, where threats can emerge in multiple locations simultaneously. The agents in a MAS can be deployed at different points in the network, each responsible for monitoring its own environment and sharing information with other agents. This decentralized approach allows the system to scale effectively and respond quickly to new threats.

In the context of Cyberdefense, MAS have been applied to various tasks, including intrusion detection, malware mitigation, and network monitoring. The adaptability and scalability of MAS make them an ideal solution for protecting large and complex networks.

\subsection{Technologies Enabling Distributed Cyberdefense}

Several technologies have played a key role in enabling distributed and decentralized Cyberdefense systems. These technologies provide the foundational infrastructure and tools needed to support decentralized architectures and enable effective communication, coordination, and decision-making among distributed nodes. Below, we highlight some of the most important enabling technologies.

\subsubsection{Edge Computing}

Edge computing brings computation and data storage closer to the location where it is needed, reducing latency and bandwidth usage. In a decentralized Cyberdefense system, edge computing enables security functions to be distributed across the network, with each edge node responsible for monitoring and defending its local environment.

Edge computing is particularly valuable in IoT networks, where devices generate large volumes of data that need to be processed in real-time. By distributing security functions to the edge, the system can detect and respond to threats more quickly, without relying on a central server to process all the data.

\subsubsection{Fog Computing}

Fog computing extends the concept of edge computing by creating a hierarchical architecture that includes edge devices, fog nodes, and cloud servers. In a fog computing environment, fog nodes act as intermediaries between the edge devices and the cloud, providing additional processing power and storage closer to the data source.

Fog computing can enhance the resilience of decentralized Cyberdefense systems by providing redundancy and backup capabilities. If an edge node is compromised or becomes unavailable, the fog nodes can take over its functions, ensuring that the system remains operational.

\subsubsection{Software-Defined Networking (SDN)}

Software-Defined Networking (SDN) is an architectural approach that separates the control plane from the data plane in a network. This separation allows network administrators to programmatically manage and optimize the network, making it more flexible and responsive to changing conditions.

In a decentralized Cyberdefense system, SDN can be used to dynamically reconfigure the network in response to security threats. For example, SDN controllers can detect a DDoS attack and automatically reroute traffic to mitigate its impact. SDN also enables more granular control over network traffic, allowing security policies to be enforced at a finer level of detail.

\subsubsection{Artificial Intelligence and Machine Learning}

Artificial intelligence (AI) and machine learning (ML) have become integral to modern Cyberdefense systems, particularly in distributed and decentralized environments. AI and ML can be used to analyze vast amounts of data, identify patterns, and detect anomalies that may indicate a cyber attack.

In decentralized systems, AI and ML algorithms can be deployed at various points in the network to enhance threat detection and response. For example, machine learning models can be trained to identify malicious behavior based on historical data, allowing the system to detect new and previously unseen threats. AI-powered agents can also collaborate with one another to share insights and coordinate their defenses, improving the overall effectiveness of the system.

\subsection{Challenges in Distributed and Decentralized Cyberdefense}

Despite the advantages of distributed and decentralized Cyberdefense, there are several challenges that need to be addressed to fully realize the potential of these systems. These challenges include:

\paragraph{Coordination and Communication:}
In a decentralized system, coordination and communication among nodes are critical to ensuring effective defense. However, managing communication in a large and dynamic network can be challenging, particularly when nodes are geographically dispersed or operate in environments with limited bandwidth. Ensuring that nodes can share information in a timely and reliable manner is essential for detecting and responding to threats.

\paragraph{Trust Management:}
In decentralized systems, establishing trust between nodes is a significant challenge. Unlike centralized systems, where a single authority can enforce security policies, decentralized systems rely on nodes to make their own security decisions. Ensuring that nodes can trust one another to share accurate and reliable information is critical to the effectiveness of the system. Trust management frameworks, such as reputation-based systems, are often used to address this challenge, but they can be vulnerable to attacks such as collusion or false reporting.

\paragraph{Scalability:}
As decentralized Cyberdefense systems grow in size, scalability becomes a major concern. Ensuring that the system can handle an increasing number of nodes, devices, and security events without degrading performance is a significant challenge. Technologies such as blockchain and distributed databases offer potential solutions, but they also introduce new challenges in terms of resource consumption and latency.

\paragraph{Security of the System Itself:}
Decentralized systems are not immune to attacks, and ensuring the security of the system itself is a critical challenge. Malicious actors may attempt to compromise individual nodes, disrupt communication between nodes, or exploit vulnerabilities in the decentralized architecture. Developing robust security protocols that protect the system from both external and internal threats is essential for maintaining the integrity of the defense system.

\subsection{Case Studies and Applications}

To better understand the state of the art in distributed and decentralized Cyberdefense, it is useful to examine real-world case studies and applications. Below, we highlight several notable examples of distributed Cyberdefense systems in practice.

\subsubsection{Case Study 1: Botnet Mitigation Using Collaborative Intrusion Detection}

Botnets are a major threat to modern networks, as they can be used to launch large-scale DDoS attacks, steal data, and distribute malware. A collaborative intrusion detection system (CIDS) was deployed in a large enterprise network to mitigate botnet activity. The CIDS nodes were distributed across different segments of the network, each responsible for monitoring traffic and detecting signs of botnet communication.

When a botnet was detected in one part of the network, the CIDS nodes collaborated to identify other infected devices and block their communication with the botnet command-and-control servers. The decentralized nature of the CIDS allowed the system to respond quickly to the botnet, minimizing its impact on the network.

\subsubsection{Case Study 2: Blockchain-Based Security for IoT Networks}

A decentralized IoT network was secured using blockchain technology to authenticate devices and maintain an immutable record of security events. Each IoT device was registered on the blockchain, and any attempt to modify device settings or firmware required consensus from a majority of nodes in the network.

This blockchain-based security system ensured that unauthorized modifications to IoT devices were detected and prevented, even if some devices were compromised. The decentralized nature of the blockchain made it difficult for attackers to alter the security records, enhancing the overall security of the network.

\subsubsection{Case Study 3: Distributed Denial-of-Service (DDoS) Protection Using Multi-Agent Systems}

A large cloud service provider deployed a multi-agent system (MAS) to protect against DDoS attacks. The MAS agents were distributed across the provider's data centers, each responsible for monitoring incoming traffic and detecting signs of a DDoS attack.

When a DDoS attack was detected, the agents coordinated to block malicious traffic and reroute legitimate traffic to unaffected parts of the network. The decentralized nature of the MAS allowed the system to scale effectively and respond to the DDoS attack in real-time, ensuring that the cloud services remained available to customers.

\subsection{Future Directions in Distributed and Decentralized Cyberdefense}

The field of distributed and decentralized Cyberdefense is rapidly evolving, with new technologies and approaches being developed to address emerging threats. Some of the key areas for future research and development include:

\begin{itemize}
    \item \textbf{AI-Driven Cyberdefense:} The integration of artificial intelligence and machine learning into decentralized Cyberdefense systems will continue to be a major area of focus. AI-driven agents that can autonomously detect and respond to threats will be critical for defending against increasingly sophisticated attacks.
    
    \item \textbf{Quantum-Resistant Security:} As quantum computing advances, there is a growing need to develop decentralized Cyberdefense systems that are resistant to quantum-based attacks. This includes the development of quantum-resistant cryptographic algorithms and protocols that can be used in decentralized architectures.
    
    \item \textbf{Secure Communication Protocols:} Ensuring secure communication between nodes in decentralized systems will remain a priority. Future research will focus on developing lightweight, efficient, and secure communication protocols that can operate in resource-constrained environments, such as IoT networks.
    
    \item \textbf{Autonomous Decision-Making:} Enhancing the autonomy of decentralized Cyberdefense systems will be critical for improving their ability to respond to emerging threats. Research in this area will focus on developing decision-making frameworks that allow agents to operate independently, while still coordinating effectively with other agents.
\end{itemize}

\subsection{Conclusion}

Distributed and decentralized Cyberdefense systems represent a significant advancement in the field of cybersecurity. By distributing security functions across the network and enabling localized threat detection and response, these systems offer greater resilience, scalability, and adaptability in the face of modern cyber threats. However, challenges such as coordination, trust management, and scalability must be addressed to fully realize the potential of decentralized Cyberdefense.

As new technologies such as AI, blockchain, and quantum computing continue to evolve, the field of distributed Cyberdefense will continue to expand. Future research and development will focus on integrating these technologies into decentralized architectures, ensuring that they can provide robust and scalable security for a wide range of applications.




\section{Synthesis of Current Knowledge and Identification of Research Gaps}

\subsection{Synthesis of Current Knowledge in Distributed and Decentralized Cyberdefense}

The field of distributed and decentralized Cyberdefense has seen significant advancements over the past decade. With the increasing complexity of cyber threats and the limitations of traditional centralized defense mechanisms, researchers have turned their attention to more resilient and scalable solutions. In this section, we synthesize the current knowledge on key approaches, technologies, and methodologies in decentralized Cyberdefense, focusing on multi-agent systems (MAS), collaborative intrusion detection systems (CIDS), blockchain-based security, and other enabling technologies.

\subsubsection{Multi-Agent Systems (MAS) in Cyberdefense}

Multi-Agent Systems (MAS) have emerged as one of the most promising approaches for distributed Cyberdefense. As discussed in previous sections, MAS leverage the autonomy and collaboration of multiple agents to monitor, detect, and respond to cyber threats in a decentralized manner. The key advantages of MAS include:

\begin{itemize}
    \item \textbf{Scalability:} MAS can scale effectively by adding more agents to the network as needed, without the need for a central authority to manage the entire system.
    \item \textbf{Resilience:} The decentralized nature of MAS reduces the risk of a single point of failure, enhancing the overall resilience of the system.
    \item \textbf{Adaptability:} MAS are highly adaptable, capable of learning from their environment and adjusting their strategies to counter new and emerging threats.
\end{itemize}

Despite these advantages, MAS face challenges related to coordination, communication, and trust management among agents. Current research has focused on improving the efficiency of coordination mechanisms, developing robust communication protocols, and addressing issues of trust and security within MAS. However, there remain several open questions regarding the optimization of agent collaboration in highly dynamic environments and the development of more sophisticated learning algorithms to enhance agent autonomy.

\subsubsection{Collaborative Intrusion Detection Systems (CIDS)}

Collaborative Intrusion Detection Systems (CIDS) represent a decentralized approach to threat detection, where multiple IDS nodes work together to monitor network traffic and detect anomalies. The key strength of CIDS lies in their ability to aggregate information from different parts of the network, allowing for the detection of distributed attacks that may not be visible to a single IDS node.

Recent advancements in CIDS have focused on improving the accuracy of threat detection through the use of machine learning and AI-based anomaly detection techniques. Additionally, research has explored the use of hierarchical and peer-to-peer (P2P) topologies to enhance the scalability and resilience of CIDS.

However, challenges remain in ensuring the timely and accurate sharing of information between CIDS nodes, particularly in large and dynamic networks. Trust management is also a critical issue, as the effectiveness of a CIDS depends on the reliability of the information shared between nodes. Addressing these challenges requires further research into trust models, secure communication protocols, and the integration of advanced AI techniques.

\subsubsection{Blockchain-Based Security}

Blockchain technology has gained attention as a potential solution for securing decentralized systems, particularly in IoT networks and supply chain security. The decentralized and immutable nature of blockchain makes it an attractive option for ensuring the integrity of data and transactions in distributed environments.

Recent research has explored the use of blockchain for securing communications between devices, authenticating users, and maintaining an audit trail of security events. One of the key benefits of blockchain is its resistance to tampering, as any attempt to alter data requires consensus from the majority of nodes in the network.

However, blockchain-based security faces challenges related to scalability, energy consumption, and latency. As blockchain networks grow larger, the consensus mechanism can become a bottleneck, slowing down transaction processing times. Additionally, the energy-intensive nature of blockchain mining presents a barrier to widespread adoption, particularly in resource-constrained environments such as IoT networks. Ongoing research is focused on developing more efficient consensus algorithms and exploring alternative blockchain architectures, such as permissioned blockchains, to address these challenges.

\subsubsection{Edge and Fog Computing in Cyberdefense}

Edge and fog computing have emerged as key technologies for enabling distributed Cyberdefense, particularly in IoT networks and other decentralized environments. By bringing computation and data storage closer to the location where it is needed, edge and fog computing reduce latency and bandwidth usage, enabling real-time threat detection and response.

Edge computing allows security functions to be distributed across the network, with each edge node responsible for monitoring and defending its local environment. Fog computing extends this concept by creating a hierarchical architecture that includes edge devices, fog nodes, and cloud servers. Fog nodes act as intermediaries, providing additional processing power and storage closer to the data source.

Recent research in edge and fog computing has focused on optimizing the deployment of security functions, improving the efficiency of data processing, and enhancing the resilience of the system. However, challenges remain in managing the complexity of large-scale edge and fog networks, ensuring secure communication between nodes, and developing adaptive security mechanisms that can respond to emerging threats in real-time.

\subsection{Identification of Research Gaps in Distributed and Decentralized Cyberdefense}

While significant progress has been made in the development of distributed and decentralized Cyberdefense systems, several research gaps remain. These gaps represent critical areas where further investigation is needed to fully realize the potential of decentralized architectures in cybersecurity. Below, we identify and discuss key research gaps in the field.

\subsubsection{Gap 1: Trust and Security in Decentralized Systems}

One of the most significant challenges in decentralized Cyberdefense is ensuring trust and security among distributed nodes. Unlike centralized systems, where a single authority can enforce security policies and verify the integrity of the system, decentralized systems rely on individual nodes to make their own security decisions.

This decentralization introduces several challenges related to trust management, particularly in open and dynamic environments where nodes may not have pre-established trust relationships. Existing trust models, such as reputation-based systems, are vulnerable to attacks such as collusion, false reporting, and Sybil attacks, where an attacker creates multiple fake identities to manipulate the system.

Research is needed to develop more robust trust management frameworks that can operate effectively in decentralized environments. This includes exploring new trust models, developing secure communication protocols, and implementing mechanisms for detecting and mitigating insider threats.

\subsubsection{Gap 2: Scalability of Distributed Cyberdefense Architectures}

Scalability is a critical concern in decentralized Cyberdefense systems, particularly as networks grow larger and more complex. Ensuring that the system can handle an increasing number of nodes, devices, and security events without degrading performance is a significant challenge.

Current research has explored various approaches to improving the scalability of decentralized systems, including the use of hierarchical architectures, blockchain, and distributed databases. However, these solutions often introduce new challenges, such as increased resource consumption, latency, and complexity.

Further research is needed to develop scalable architectures that can support large-scale decentralized Cyberdefense systems. This includes exploring new technologies, such as quantum computing, that may offer new opportunities for improving scalability, as well as optimizing existing architectures to reduce resource consumption and improve performance.

\subsubsection{Gap 3: Coordination and Communication Among Distributed Nodes}

Effective coordination and communication are essential for the success of decentralized Cyberdefense systems. However, managing communication in a large and dynamic network can be challenging, particularly when nodes are geographically dispersed or operate in environments with limited bandwidth.

Research is needed to develop more efficient communication protocols that can operate in resource-constrained environments and ensure the timely and accurate sharing of information between nodes. Additionally, new coordination mechanisms are needed to optimize the collaboration between nodes and enhance the overall effectiveness of the system.

\subsubsection{Gap 4: Adaptability and Learning in Dynamic Environments}

One of the key advantages of decentralized Cyberdefense systems is their ability to adapt to changing environments. However, current systems often struggle to respond effectively to new and emerging threats, particularly in highly dynamic environments where attack strategies are constantly evolving.

Research is needed to develop more sophisticated learning algorithms that can enhance the adaptability of decentralized systems. This includes exploring new approaches to machine learning, reinforcement learning, and AI, as well as developing mechanisms for continuous learning and adaptation in real-time.

\subsubsection{Gap 5: Integration of Emerging Technologies}

Emerging technologies, such as quantum computing, AI, and blockchain, have the potential to transform decentralized Cyberdefense systems. However, integrating these technologies into existing architectures presents several challenges.

For example, quantum computing offers the potential to significantly enhance the security of decentralized systems, but it also introduces new vulnerabilities that must be addressed. Similarly, AI and machine learning can enhance threat detection and response, but they also introduce challenges related to transparency, explainability, and trust.

Research is needed to explore the integration of emerging technologies into decentralized Cyberdefense systems and to address the new challenges they introduce. This includes developing new frameworks for assessing the impact of these technologies on security and resilience, as well as exploring new opportunities for leveraging these technologies to enhance the effectiveness of decentralized architectures.

\subsection{Conclusion}

The synthesis of current knowledge in distributed and decentralized Cyberdefense highlights the significant progress that has been made in recent years. Multi-agent systems, collaborative intrusion detection systems, blockchain, and edge computing have all contributed to the development of more resilient and scalable Cyberdefense architectures. However, several research gaps remain, particularly in the areas of trust management, scalability, coordination, adaptability, and the integration of emerging technologies.

Addressing these gaps will require continued research and innovation in the field. By exploring new approaches to trust and security, improving the scalability of decentralized systems, and developing more sophisticated learning algorithms, researchers can help ensure that decentralized Cyberdefense architectures are capable of meeting the challenges of modern cybersecurity. Furthermore, the integration of emerging technologies offers new opportunities for enhancing the effectiveness and resilience of decentralized systems, making them a critical area for future research.



\section{Research Questions}

\subsection{Introduction}

The synthesis of current knowledge in distributed and decentralized Cyberdefense, as well as the identification of research gaps, has highlighted several critical challenges that need to be addressed to advance the field. These challenges revolve around trust management, scalability, coordination, adaptability, and the integration of emerging technologies into decentralized architectures. To address these challenges, the following research questions have been formulated. These questions will guide the direction of this research and shape the development of new methodologies, models, and systems for distributed and decentralized Cyberdefense.

\subsection{Research Question 1: How Can Trust Be Effectively Managed in Decentralized Cyberdefense Systems?}

One of the most significant challenges in decentralized Cyberdefense is ensuring trust among distributed nodes. Unlike centralized systems, where a single authority can enforce security policies and verify the integrity of the system, decentralized systems rely on individual nodes to make security decisions. This decentralization introduces vulnerabilities, such as the potential for malicious nodes to disrupt the system, collusion attacks, and false reporting.

\textbf{Research Question 1 (Q1):} \textit{How can trust be effectively managed in decentralized Cyberdefense systems to ensure secure and reliable communication between nodes, while mitigating the risks posed by malicious actors and ensuring the overall integrity of the system?}

This question aims to explore new models and frameworks for trust management in decentralized environments. It seeks to address the challenges of trust establishment, maintenance, and recovery in dynamic and potentially adversarial settings. The focus will be on developing mechanisms that enable nodes to assess the trustworthiness of their peers and make informed decisions about collaboration and information sharing.

\subsection{Research Question 2: How Can Decentralized Cyberdefense Systems Be Scaled Effectively?}

Scalability is a critical concern in decentralized Cyberdefense systems, especially as networks grow larger and more complex. Ensuring that the system can handle an increasing number of nodes, devices, and security events without degrading performance is a significant challenge. Current research has explored various approaches, such as hierarchical architectures and blockchain-based solutions, but these often introduce new issues related to resource consumption and latency.

\textbf{Research Question 2 (Q2):} \textit{What architectural and algorithmic innovations are needed to ensure the effective scalability of decentralized Cyberdefense systems, while maintaining performance, security, and resource efficiency in large-scale networks?}

This question seeks to investigate new approaches for scaling decentralized systems, focusing on optimizing communication, coordination, and resource allocation in large-scale environments. The research will explore both the technical and theoretical aspects of scalability, aiming to develop solutions that can be applied across various decentralized architectures.

\subsection{Research Question 3: What Coordination Mechanisms Are Most Effective in Distributed Cyberdefense?}

Coordination is essential for the success of decentralized Cyberdefense systems, particularly in dynamic environments where threats can emerge in multiple locations simultaneously. However, managing coordination in a large and distributed network can be challenging, especially when nodes are geographically dispersed or operate in environments with limited bandwidth.

\textbf{Research Question 3 (Q3):} \textit{What coordination mechanisms and communication protocols are most effective in distributed Cyberdefense systems, and how can they be optimized to ensure timely and accurate responses to emerging threats in dynamic and resource-constrained environments?}

This question focuses on identifying and developing coordination strategies that enhance the effectiveness of decentralized Cyberdefense systems. The research will explore both centralized and decentralized coordination mechanisms, as well as hybrid approaches that balance the trade-offs between flexibility and efficiency. Special attention will be given to the challenges of coordination in resource-constrained environments, such as IoT networks and mobile ad-hoc networks (MANETs).

\subsection{Research Question 4: How Can Decentralized Cyberdefense Systems Be Made More Adaptive to Emerging Threats?}

One of the key advantages of decentralized Cyberdefense systems is their potential for adaptability in dynamic environments. However, current systems often struggle to respond effectively to new and emerging threats, particularly as attack strategies evolve and become more sophisticated. There is a need for decentralized systems to continuously learn and adapt their defense strategies to keep pace with the evolving threat landscape.

\textbf{Research Question 4 (Q4):} \textit{How can decentralized Cyberdefense systems be made more adaptive to emerging threats, and what machine learning and AI techniques can be leveraged to enhance the system's ability to learn from its environment and evolve its defense strategies over time?}

This question aims to explore new approaches to adaptability in decentralized systems, focusing on the integration of machine learning, reinforcement learning, and AI techniques. The research will investigate how these technologies can be applied to enhance the system's ability to detect and respond to novel attack patterns, while continuously improving its performance over time.

\subsection{Research Question 5: How Can Emerging Technologies Be Integrated into Decentralized Cyberdefense Architectures?}

Emerging technologies such as quantum computing, blockchain, and advanced AI have the potential to transform decentralized Cyberdefense systems. However, integrating these technologies into existing architectures presents significant challenges, particularly in terms of security, scalability, and performance.

\textbf{Research Question 5 (Q5):} \textit{What are the most effective methods for integrating emerging technologies, such as quantum computing and blockchain, into decentralized Cyberdefense architectures, and how can these technologies be leveraged to enhance the security and resilience of the system?}

This question seeks to explore the potential of emerging technologies to address the challenges faced by decentralized Cyberdefense systems. The research will focus on identifying the opportunities and limitations of these technologies, as well as developing frameworks for their integration into existing and future architectures.

\subsection{Conclusion}

The research questions outlined in this section are designed to address the key challenges identified in the field of distributed and decentralized Cyberdefense. By exploring new approaches to trust management, scalability, coordination, adaptability, and the integration of emerging technologies, this research aims to contribute to the development of more resilient and effective Cyberdefense systems. The answers to these questions will form the foundation of the subsequent chapters, guiding both the theoretical and practical contributions of this thesis.



% \section{An Increasing Need for Decentralized Cyberdefense}
% \begin{itemize}
%     \item AICA, besoins nouveaux, IoT/IoBT, etc.
%     \item Approche centralisée peu adaptée, etc. pour des raisons d’interruptions de communication, hétérogénéité des SI, etc.
%     \item Une approche MA pourrait être appliquée -> Système Multi-Agents de Cyberdéfense (SMAC)
% \end{itemize}

% \section{The Challenges of a Multi-Agent Cyberdefense}
% \begin{itemize}
%     \item Mais sujet nouveau : pas de modélisation, travaux formels…
% \end{itemize}

% \section{Research Question and Objectives}
% \begin{itemize}
%     \item Quelle méthode pour concevoir un SMAC qui atteint ses objectifs de cyberdéfense tout en satisfaisant les contraintes de déploiement et opérationelles du système hôte qu'il doit défendre ?
% \end{itemize}

% \section{Positioning and Contribution of this thesis}
% \begin{itemize}
%     \item Un ensemble d’agents collaboratifs répond effectivement aux nouveaux besoins, etc. mieux que des solutions centralisées
%     \item Une méthode modélisant le "domaine" (environnement réseau + actions/observations possibles des red/blue/green teams), du "problème" (objectif de cyberdéfense / blue team + contraintes opérationelles/déploiement) sous forme d'un **problème d'optimisation sous-contraintes*\item ; permet de fournir des moyens objectifs d’évaluer de façon consistante si le SMA tient ses promesses dans plusieurs scénarios d’attaque…
% \end{itemize}

\section{Manuscript Organization}

The remaining parts of the manuscript are as follows, Chapter 2 begins by an introduction of the concepts of multi-agent systems of embed-ded agents and security architecture. It is continued by a description of the first contribution of this thesis, the systematic mapping study.
This chapter includes a detailed presentation of the employed method-ology, the results which are answers to the three research questions:
(i) “What are the main security properties studied in multi-embed-ded-agent systems?”; (ii) “What are the specific technical solutions for securing multi-embedded-agent systems?”; (iii) “Which parts of a global security architecture for multi-embedded-agent systems are studied?”, and a discussion about the implications of these results on this thesis. In particular, it led us to focus on key management as the literature lacks relevant solutions related to this topic.

Before addressing the remaining contributions, we first present in Chapter 3 a second contribution, a simulation tool we specifically developed to allow us to prototype and validate our approach. After presenting a comparison study of the current tools that are available and why they do not suit our context, we explain the implementation choices made and the concepts we followed to create such a tool.

Chapter 4 details the third and main contribution, Multi-Agent Key Infrastructure (MAKI), a PKI designed for multi-agent systems of embedded agents. The chapter consists of a literature review of PKI used in decentralized systems, a description of the infrastructure we propose and results of validation using both simulation and model-checking approaches.
The fourth contribution is presented in Chapter 5. It aims at mitigat-ing the absence of a central authority sharing the certificates between the agents by deploying a blockchain tailored to the needs of multi-agent systems of embedded agents. Like the previous one, this chapter includes a literature review of the related works concerning the estab-lishment of a shared database in a decentralized system, a description of the contribution and validation results.

The manuscript ends with Chapter 6. This last chapter summarizes the main achievements of the presented works and offers several future works and research perspectives.