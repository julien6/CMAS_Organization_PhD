%************************************************
\chapter{Introduction}\label{ch:introduction}
%************************************************

\section{A Cyberdefense Context With Future And New Challenges}

In today's increasingly connected world, the complexity and scope of cyber threats are evolving at an unprecedented rate. The transition to decentralized and distributed systems, driven by advancements in the Internet of Things (IoT), cloud computing, and mobile networks, has opened up new vulnerabilities that traditional centralized defense mechanisms are ill-equipped to handle\cite{sun2014data}. This new paradigm demands innovative approaches to Cyberdefense that are both agile and resilient in the face of diverse and sophisticated attacks\cite{taddeo2019trusting}.

\subsection{The Rise of Decentralized Cyber Threats}

As the digital landscape expands, so does the attack surface available to malicious actors. Modern cyber threats are no longer confined to isolated incidents targeting specific systems; instead, they can propagate across vast networks, affecting multiple systems simultaneously. This shift is largely due to the decentralization of both infrastructure and attacks\cite{li2020survey}. Decentralized systems, such as blockchain networks, peer-to-peer architectures, and distributed cloud platforms, have become increasingly common in both civilian and military contexts. While these systems offer enhanced robustness and scalability, they also introduce new security challenges\cite{sun2014data}.

For example, in a decentralized financial system based on blockchain technology, a single security breach can compromise an entire network of transactions across the globe\cite{li2020survey}. Similarly, attacks on decentralized IoT networks, where millions of devices communicate autonomously, can cause widespread disruptions, ranging from critical infrastructure failures to breaches in personal data privacy\cite{sun2014data}. The WannaCry ransomware attack of 2017, which exploited vulnerabilities in a decentralized network of systems, highlighted the devastating impact of such distributed attacks on both public and private sectors\cite{mohurle2017wannacry}.

The rise of decentralized cyber threats is further complicated by the increasing use of sophisticated attack vectors such as hardware Trojans, advanced persistent threats (APTs), and zero-day vulnerabilities\cite{chen2014study}. These attacks can remain dormant within decentralized networks for extended periods, waiting for the opportune moment to strike. Traditional centralized defense mechanisms, which rely on a single point of control, are often too slow to respond to such dynamic and distributed threats\cite{vasilomanolakis2015taxonomy}. This highlights the need for new defense strategies that can operate in a decentralized manner, responding autonomously and in real-time to emerging threats\cite{vasilomanolakis2015taxonomy}.

\subsection{Challenges in Modern Cyberdefense}

In this context, several key challenges must be addressed to ensure the security of decentralized systems. These challenges can be categorized into the following criteria, each representing a critical aspect of modern Cyberdefense:

\paragraph{C1) Decentralization:}
The shift towards decentralized systems requires a corresponding shift in defense strategies. Centralized security architectures, where decisions are made by a single authority, are no longer sufficient. Instead, security must be distributed across the network, with individual components capable of autonomous decision-making\cite{xu2021decentralized}. This decentralization of defense introduces challenges related to coordination, communication, and trust among autonomous agents\cite{tosh2018evolutionary}.

\paragraph{C2) Distribution:}
In a distributed system, components are geographically and logically dispersed. This distribution complicates traditional security measures, as it is difficult to monitor and control all parts of the network simultaneously\cite{vasilomanolakis2015taxonomy}. Ensuring the integrity and availability of data across distributed networks requires new approaches to threat detection and response that can operate at scale and in real-time\cite{husak2019survey}.

\paragraph{C3) Autonomy:}
As systems become more complex and distributed, the need for autonomous Cyberdefense mechanisms grows. Autonomous agents must be capable of detecting and responding to threats without relying on human intervention\cite{taddeo2019trusting}. However, designing such autonomous systems introduces challenges related to decision-making under uncertainty, as well as ensuring that these systems can adapt to evolving threats\cite{sarker2020cybersecurity}.

\paragraph{C4) Speed and Scalability:}
Cyberattacks are becoming faster and more aggressive, capable of compromising systems within seconds. Defense mechanisms must therefore be able to respond at an equally rapid pace\cite{khraisat2019survey}. Moreover, as networks grow in size and complexity, security solutions must be scalable, ensuring that they can protect large, distributed infrastructures without becoming a bottleneck themselves\cite{berman2019survey}.

\paragraph{C5) Security and Safety:}
Ensuring the safety and security of decentralized systems requires a holistic approach that considers not only technical vulnerabilities but also human factors. Phishing, social engineering, and insider threats remain significant concerns, particularly in environments where trust is decentralized and spread across multiple entities\cite{salahdine2019social}.

\paragraph{C6) Machine Learning and Adaptability:}
The use of machine learning (ML) and artificial intelligence (AI) in Cyberdefense is on the rise, offering new capabilities for threat detection and prediction. However, these technologies also introduce new challenges. ML models must be trained on vast amounts of data, and they must be capable of adapting to new and previously unseen attack vectors\cite{xin2018machine}. Additionally, adversarial attacks against ML models pose a growing threat, as attackers seek to manipulate the very algorithms designed to protect systems\cite{apruzzese2018effectiveness}.

\subsection{Emerging Threats and the Need for Innovative Defense Mechanisms}

Looking to the future, the landscape of cyber threats is expected to become even more complex, driven by advancements in technology and the increasing interconnectivity of global systems. Quantum computing, for example, poses a significant threat to current cryptographic standards, potentially rendering many existing security protocols obsolete\cite{mosca2018cybersecurity}\cite{bernstein2017post}. Similarly, the proliferation of AI-driven attacks, where adversaries use AI to automate and optimize their attack strategies, will require equally sophisticated defense mechanisms\cite{brundage2018malicious}.

In this rapidly evolving environment, it is clear that traditional centralized defense strategies are no longer sufficient. Instead, there is a growing need for decentralized, distributed, and autonomous defense mechanisms that can operate at scale and adapt to new threats in real-time\cite{moustafa2019holistic}. Multi-Agent Systems (MAS), with their ability to coordinate the actions of multiple autonomous agents, offer a promising solution to these challenges\cite{shakarian2015cyber}.

The concept of a Cyberdefense Multi-Agent System (MAS) will be explored in the following sections, with a focus on how such systems can address the challenges outlined above and provide a robust, scalable, and adaptive defense against future cyber threats\cite{kotenko2017cyber}.


\section{The Idea Of A Cyberdefense Multi-Agent System}

\subsection{AICA: A Pioneer in Autonomous Cyberdefense}

The increasing complexity and frequency of cyber threats, particularly in military and critical infrastructure domains, have driven the development of more advanced defense systems. One pioneering initiative in this field is the Autonomous Intelligent Cyberdefense Agent (AICA). AICA represents an innovative approach to Cyberdefense, focusing on the creation of autonomous, intelligent agents capable of detecting, responding to, and mitigating cyber threats in real-time \cite{jajodia2017autonomous}.

AICA agents are designed to operate in highly dynamic and contested environments, such as those found in military operations, where the cyber threat landscape is constantly evolving \cite{kott2018towards}. The core idea behind AICA is to create a distributed network of autonomous agents that work collaboratively to protect critical systems. Each agent is equipped with advanced artificial intelligence (AI) capabilities, allowing it to autonomously assess its environment, identify potential threats, and take appropriate action to neutralize them \cite{kott2023autonomous}.

The decentralized nature of AICA agents is a key advantage in environments where centralized control is impractical or vulnerable to attack. Instead of relying on a central authority, AICA agents communicate with one another to share information and coordinate their actions \cite{kott2018autonomous}. This distributed approach enhances the system's resilience, as it reduces the likelihood that the entire defense system can be compromised through a single point of failure \cite{kott2023autonomous}.

AICA's design principles are based on the notion of autonomous decision-making. Each agent can independently monitor the system, detect anomalies, and implement countermeasures, all while adapting to the evolving threat environment \cite{kott2018towards}. This adaptability is crucial in modern cyber warfare, where attackers constantly develop new techniques to bypass traditional security measures. By leveraging AI and machine learning, AICA agents can evolve their strategies over time, ensuring that the defense system remains effective against emerging threats \cite{chatterjee2023collaboration}.

\subsection{From AICA to MAS in Cyberdefense}

While AICA provides a specific implementation of an autonomous Cyberdefense system, the broader concept of Multi-Agent Systems (MAS) offers a general framework that can be applied across various domains of Cyberdefense. MAS are systems composed of multiple interacting autonomous agents that collaborate to achieve a common goal \cite{kott2018towards}. In the context of Cyberdefense, this goal is the protection of critical systems and networks from cyber threats \cite{jajodia2017autonomous}.

The generalization from AICA to MAS allows for greater flexibility in the design and deployment of Cyberdefense systems. MAS can be tailored to specific environments, whether it be military networks, corporate infrastructures, or decentralized IoT ecosystems \cite{kott2023autonomous}. The principles that guide the operation of AICA—decentralization, autonomy, adaptability—are also at the core of MAS, making them an ideal framework for addressing the challenges of modern Cyberdefense \cite{kolias2011swarm}.

In a MAS for Cyberdefense, each agent operates as an independent entity, capable of making decisions based on local information. However, these agents are not isolated; they communicate with one another to share threat intelligence and coordinate their defensive actions \cite{kolias2016swarm}. This communication is crucial in distributed environments, where threats can emerge simultaneously in multiple locations, requiring a coordinated response \cite{bace2001intrusion}.

The decentralized nature of MAS is particularly well-suited to defending large-scale, distributed networks. Unlike centralized systems, which rely on a single point of control, MAS distribute the responsibility for defense across the entire network \cite{shamshirband2014cooperative}. This distribution not only enhances the system's resilience but also allows it to scale more effectively. As the network grows, additional agents can be deployed without the need for significant reconfiguration of the overall system \cite{shamshirband2018computational}.

\subsection{MAS to Meet the Cyberdefense Challenges}

The challenges of modern Cyberdefense, as outlined in the previous section, can be effectively addressed through the key features of MAS. These features, which include decentralization, distribution, autonomy, speed, security, and adaptability, align closely with the needs of a robust and scalable defense system. Here, we explore how MAS can tackle each challenge in detail.

\paragraph{Decentralization and Distribution:}
MAS inherently operate in a decentralized manner, with no single point of failure. This makes them particularly effective in distributed environments where attacks can occur at multiple points simultaneously \cite{kolias2016swarm}. Each agent in a MAS can independently monitor and protect a specific segment of the network, ensuring that even if one part of the system is compromised, the rest remains secure \cite{shakarian2015cyber}.

\paragraph{Autonomy and Speed:}
The autonomy of MAS agents allows them to detect and respond to threats in real-time, without waiting for instructions from a central authority \cite{shamshirband2014cooperative}. This speed is critical in Cyberdefense, where the time between detecting an attack and responding to it can determine the extent of the damage \cite{liao2013intrusion}. By operating autonomously, MAS agents can neutralize threats as soon as they are identified, minimizing the impact on the system \cite{shamshirband2018computational}.

\paragraph{Security and Resilience:}
MAS enhance the security of distributed systems by decentralizing defensive capabilities. This decentralization makes it more difficult for attackers to compromise the entire system, as they would need to defeat multiple independent agents rather than a single centralized defense \cite{jajodia2005topological}. Additionally, the collaboration between agents allows for more comprehensive threat detection, as each agent contributes its local knowledge to the global defense effort \cite{buczak2016survey}.

\paragraph{Adaptability and Learning:}
One of the most significant advantages of MAS in Cyberdefense is their ability to adapt to new threats. Through the use of machine learning and AI, MAS agents can continuously improve their threat detection and response strategies \cite{buczak2016survey}. This adaptability is crucial in the face of evolving cyber threats, where attackers are constantly developing new methods to breach security systems. MAS agents can learn from past experiences and adjust their behavior accordingly, ensuring that the defense system remains effective over time \cite{debar1999towards}.

\subsection{Towards a General Cyberdefense MAS}

Building on the principles established by AICA, a general framework for MAS in Cyberdefense can be developed. This framework would consist of multiple layers of defense, with agents operating at different levels of the system, from individual devices to entire network segments. Each layer would have its own set of agents responsible for monitoring, detecting, and responding to threats within their specific domain \cite{bou-harb2017cyber}.

At the lowest level, agents could be deployed on individual devices, monitoring for signs of compromise such as unusual network traffic or unauthorized access attempts \cite{kolias2016swarm}. These agents would communicate with one another to share information and coordinate their responses \cite{shamshirband2018computational}. At higher levels, agents could be responsible for overseeing larger segments of the network, ensuring that any threats detected at the device level do not propagate further into the system \cite{hu2018mimic}.

This layered approach to defense allows for more comprehensive coverage of the entire network. By deploying agents at multiple levels, the system can detect and respond to threats more quickly and effectively than a traditional centralized security system \cite{bou-harb2017cyber}. Furthermore, the adaptability of MAS agents ensures that the system can evolve in response to new threats, maintaining its effectiveness over time \cite{haider2020artificial}.

\subsection{Conclusion: The Place of MAS in Cyberdefense}

The concept of a Cyberdefense Multi-Agent System represents a significant advancement in the field of cybersecurity. By leveraging the principles of decentralization, autonomy, and adaptability, MAS offer a robust and scalable solution to the challenges of modern Cyberdefense \cite{kolias2011swarm}. The AICA initiative serves as a concrete example of how these principles can be applied in practice, providing valuable insights into the design and deployment of MAS in critical environments \cite{bou-harb2014cyber}.

As cyber threats continue to evolve, the need for decentralized, autonomous defense systems will only grow. MAS provide a flexible framework that can be adapted to a wide range of applications, from military operations to corporate networks and beyond \cite{jahanbin2013computer}. The following chapters will explore the theoretical foundations and practical applications of MAS in greater detail, offering a comprehensive guide to developing and deploying these systems in real-world Cyberdefense scenarios.



\section{Manuscript Organization}

The structure of this thesis is organized into six chapters, each building on the foundations laid by the previous one to systematically address the research questions and objectives outlined earlier.

Chapter 1 provides an introduction to the problem domain, emphasizing the increasing complexity of cyber threats and the inadequacy of centralized defense mechanisms. It introduces the concept of Multi-Agent Systems (MAS) for cyberdefense, setting the stage for the detailed exploration of this approach in the subsequent chapters.

Chapter 2 delves into the theoretical background and state-of-the-art in distributed and decentralized cyberdefense. It discusses the key concepts, challenges, and existing solutions in the field, with a particular focus on the role of MAS. This chapter also synthesizes current knowledge and identifies critical research gaps that this thesis aims to address.

Chapter 3 presents a detailed overview of the problem statement, defining the key research questions and hypotheses. It also situates the research within a theoretical framework, providing a comprehensive analysis of the existing literature and identifying the specific areas where this research contributes new insights.

Chapter 4 is dedicated to the design and development of the Cyberdefense Multi-Agent System (CybMAS) proposed in this thesis. It introduces the formal model (CybMASFM) and development approach (CybMASDA) that underpin the design of the MAS, along with the specific methodologies employed in the implementation of the system.

Chapter 5 discusses the experimental setup and validation of the CybMAS. It includes detailed case studies that demonstrate the application of the system in various scenarios, such as corporate infrastructure, drone swarms, and Kubernetes orchestration. The chapter also analyzes the results, highlighting the effectiveness and limitations of the proposed system.

Finally, Chapter 6 concludes the thesis by summarizing the main findings and contributions of the research. It also outlines potential future research directions and the broader implications of the work for the field of cyberdefense.


% \section{An Increasing Need for Decentralized Cyberdefense}
% \begin{itemize}
%     \item AICA, besoins nouveaux, IoT/IoBT, etc.
%     \item Approche centralisée peu adaptée, etc. pour des raisons d’interruptions de communication, hétérogénéité des SI, etc.
%     \item Une approche MA pourrait être appliquée -> Système Multi-Agents de Cyberdéfense (SMAC)
% \end{itemize}

% \section{The Challenges of a Multi-Agent Cyberdefense}
% \begin{itemize}
%     \item Mais sujet nouveau : pas de modélisation, travaux formels…
% \end{itemize}

% \section{Research Question and Objectives}
% \begin{itemize}
%     \item Quelle méthode pour concevoir un SMAC qui atteint ses objectifs de cyberdéfense tout en satisfaisant les contraintes de déploiement et opérationelles du système hôte qu'il doit défendre ?
% \end{itemize}

% \section{Positioning and Contribution of this thesis}
% \begin{itemize}
%     \item Un ensemble d’agents collaboratifs répond effectivement aux nouveaux besoins, etc. mieux que des solutions centralisées
%     \item Une méthode modélisant le "domaine" (environnement réseau + actions/observations possibles des red/blue/green teams), du "problème" (objectif de cyberdéfense / blue team + contraintes opérationelles/déploiement) sous forme d'un **problème d'optimisation sous-contraintes*\item ; permet de fournir des moyens objectifs d’évaluer de façon consistante si le SMA tient ses promesses dans plusieurs scénarios d’attaque…
% \end{itemize}