%************************************************
\chapter{Introduction}\label{ch:introduction}
%************************************************

\section{Un contexte de Cyberdéfense avec des défis futurs et nouveaux}

Dans le monde de plus en plus connecté d'aujourd'hui, la complexité et la portée des cybermenaces évoluent à un rythme sans précédent. La transition vers des systèmes décentralisés et distribués, portée par les progrès de l'Internet des objets (IoT), du cloud computing et des réseaux mobiles, a ouvert la voie à de nouveaux défis. de nouvelles vulnérabilités que les mécanismes de défense centralisés traditionnels ne sont pas en mesure de gérer\cite{sun2014data}. Ce nouveau paradigme exige des approches innovantes en matière de Cyberdéfense qui soient à la fois agiles et résilientes face à des attaques diverses et sophistiquées\cite{taddeo2019trusting}.

\subsection{La montée des cybermenaces décentralisées}

À mesure que le paysage numérique s’étend, la surface d’attaque à laquelle les acteurs malveillants peuvent avoir accès s’élargit. Les cybermenaces modernes ne se limitent plus à des incidents isolés ciblant des systèmes spécifiques ; elles peuvent désormais se propager sur de vastes réseaux, affectant plusieurs systèmes simultanément. Cette évolution est en grande partie due à la décentralisation des infrastructures et des attaques. Les systèmes décentralisés, tels que les réseaux blockchain, les architectures peer-to-peer et les plateformes cloud distribuées, sont de plus en plus courants dans les contextes civils et militaires. Bien que ces systèmes offrent une robustesse et une évolutivité accrues, ils introduisent également de nouveaux défis en matière de sécurité.

Par exemple, dans un système financier décentralisé basé sur la technologie blockchain, une seule faille de sécurité peut compromettre l’ensemble d’un réseau de transactions à travers le monde\cite{li2020survey}. De même, les attaques sur les réseaux IoT décentralisés, où des millions d’appareils communiquent de manière autonome, peuvent provoquer des perturbations généralisées, allant de défaillances d’infrastructures critiques à des atteintes à la confidentialité des données personnelles\cite{sun2014data}. L’attaque par ransomware WannaCry de 2017, qui a exploité les vulnérabilités d’un réseau décentralisé de systèmes, a mis en évidence l’impact dévastateur de telles attaques distribuées sur les secteurs public et privé\cite{mohurle2017wannacry}.

L'essor des cybermenaces décentralisées est encore compliqué par l'utilisation croissante de vecteurs d'attaque sophistiqués tels que les chevaux de Troie matériels, les menaces persistantes avancées (APT) et les vulnérabilités zero-day. Ces attaques peuvent rester latentes au sein des réseaux décentralisés pendant de longues périodes, en attendant le moment opportun pour frapper. Les mécanismes de défense centralisés traditionnels, qui s'appuient sur un point de contrôle unique, sont souvent trop lents à réagir à ces menaces dynamiques et distribuées. Cela souligne la nécessité de nouvelles stratégies de défense capables de fonctionner de manière décentralisée, en répondant de manière autonome et en temps réel aux menaces émergentes.

\subsection{Les défis de la Cyberdéfense moderne}

Dans ce contexte, plusieurs défis majeurs doivent être relevés pour assurer la sécurité des systèmes décentralisés. Ces défis peuvent être classés selon les critères suivants, chacun représentant un aspect critique de la Cyberdéfense moderne :

\paragraph{C1) Décentralisation :}
L'évolution vers des systèmes décentralisés nécessite un changement correspondant dans les stratégies de défense. Les architectures de sécurité centralisées, où les décisions sont prises par une seule autorité, ne suffisent plus. Au lieu de cela, la sécurité doit être distribuée sur l'ensemble du réseau, avec des composants individuels capables de prendre des décisions de manière autonome. Cette décentralisation de la défense introduit des défis liés à la coordination, à la communication et à la confiance entre les agents autonomes.

\paragraph{C2) Distribution :}
Dans un système distribué, les composants sont dispersés géographiquement et logiquement. Cette répartition complique les mesures de sécurité traditionnelles, car il est difficile de surveiller et de contrôler simultanément toutes les parties du réseau. Pour garantir l'intégrité et la disponibilité des données sur les réseaux distribués, il faut de nouvelles approches de détection et de réponse aux menaces qui peuvent fonctionner à grande échelle et en temps réel.

\paragraph{C3) Autonomie :}
À mesure que les systèmes deviennent plus complexes et distribués, le besoin de mécanismes de Cyberdéfense autonomes augmente. Les agents autonomes doivent être capables de détecter et de répondre aux menaces sans avoir recours à l'intervention humaine. Cependant, la conception de tels systèmes autonomes pose des défis liés à la prise de décision dans des conditions d'incertitude, ainsi qu'à la nécessité de garantir que ces systèmes peuvent s'adapter à l'évolution des menaces.

\paragraph{C4) Rapidité et évolutivité :}
Les cyberattaques deviennent de plus en plus rapides et agressives, capables de compromettre les systèmes en quelques secondes. Les mécanismes de défense doivent donc être capables de réagir à un rythme tout aussi rapide. De plus, à mesure que les réseaux augmentent en taille et en complexité, les solutions de sécurité doivent être évolutives, garantissant qu'elles peuvent protéger de grandes infrastructures distribuées sans devenir elles-mêmes un goulot d'étranglement.

\paragraph{C5) Sécurité et sûreté :}
Assurer la sécurité des systèmes décentralisés nécessite une approche globale qui prend en compte non seulement les vulnérabilités techniques mais aussi les facteurs humains. Le phishing, l'ingénierie sociale et les menaces internes restent des préoccupations importantes, en particulier dans les environnements où la confiance est décentralisée et répartie sur plusieurs entités.

\paragraph{C6) Apprentissage automatique et adaptabilité :}
L’utilisation de l’apprentissage automatique (ML) et de l’intelligence artificielle (IA) dans la Cyberdéfense est en plein essor, offrant de nouvelles capacités de détection et de prédiction des menaces. Cependant, ces technologies introduisent également de nouveaux défis. Les modèles ML doivent être entrainés sur de vastes quantités de données et doivent être capables de s’adapter à des vecteurs d’attaque nouveaux et inédits. En outre, les attaques adverses contre les modèles ML représentent une menace croissante, car les attaquants cherchent à manipuler les algorithmes mêmes conçus pour protéger les systèmes.

\subsection{Menaces émergentes et nécessité de mécanismes de défense innovants}

À l’avenir, le paysage des cybermenaces devrait devenir encore plus complexe, en raison des progrès technologiques et de l’interconnectivité croissante des systèmes mondiaux. L’informatique quantique, par exemple, représente une menace importante pour les normes cryptographiques actuelles, rendant potentiellement obsolètes de nombreux protocoles de sécurité existants\cite{mosca2018cybersecurity}\cite{bernstein2017post}. De même, la prolifération des attaques pilotées par l’IA, où les adversaires utilisent l’IA pour automatiser et optimiser leurs stratégies d’attaque, nécessitera des mécanismes de défense tout aussi sophistiqués\cite{brundage2018malicious}.

Dans cet environnement en évolution rapide, il est clair que les stratégies de défense centralisées traditionnelles ne suffisent plus. Au lieu de cela, il existe un besoin croissant de mécanismes de défense décentralisés, distribués et autonomes capables de fonctionner à grande échelle et de s'adapter aux nouvelles menaces en temps réel. Les systèmes multi-agents (SMA), avec leur capacité à coordonner les actions de plusieurs agents autonomes, offrent une solution prometteuse à ces défis.

Le concept d'un système multi-agent de Cyberdéfense (SMA) sera exploré dans les sections suivantes, en mettant l'accent sur la manière dont ces systèmes peuvent relever les défis décrits ci-dessus et fournir une défense robuste, évolutive et adaptative contre les futures cybermenaces.


\section{L'idée d'un système multi-agents de Cyberdéfense}

\subsection{AICA : un pionnier de la Cyberdéfense autonome}

La complexité et la fréquence croissantes des cybermenaces, notamment dans les domaines militaire et des infrastructures critiques, ont conduit au développement de systèmes de défense plus avancés. L'une des initiatives pionnières dans ce domaine est l'agent de Cyberdéfense intelligent autonome (AICA). L'AICA représente une approche innovante de la Cyberdéfense, axée sur la création d'agents autonomes et intelligents capables de détecter, de répondre et d'atténuer les cybermenaces en temps réel.

Les agents AICA sont conçus pour fonctionner dans des environnements hautement dynamiques et contestés, tels que ceux que l'on trouve dans les opérations militaires, où le paysage des cybermenaces évolue constamment. L'idée principale derrière AICA est de créer un réseau distribué d'agents autonomes qui travaillent en collaboration pour protéger les systèmes critiques. Chaque agent est équipé de capacités avancées d'intelligence artificielle (IA), lui permettant d'évaluer de manière autonome son environnement, d'identifier les menaces potentielles et de prendre les mesures appropriées pour les neutraliser.

La nature décentralisée des agents AICA constitue un avantage clé dans les environnements où le contrôle centralisé est peu pratique ou vulnérable aux attaques. Au lieu de s'appuyer sur une autorité centrale, les agents AICA communiquent entre eux pour partager des informations et coordonner leurs actions. Cette approche distribuée améliore la résilience du système, car elle réduit la probabilité que l'ensemble du système de défense puisse être compromis par un seul point de défaillance.

Les principes de conception d'AICA reposent sur la notion de prise de décision autonome. Chaque agent peut surveiller le système de manière indépendante, détecter les anomalies et mettre en œuvre des contre-mesures, tout en s'adaptant à l'évolution de l'environnement des menaces. Cette adaptabilité est cruciale dans la cyberguerre moderne, où les attaquants développent constamment de nouvelles techniques pour contourner les mesures de sécurité traditionnelles. En tirant parti de l'IA et de l'apprentissage automatique, les agents d'AICA peuvent faire évoluer leurs stratégies au fil du temps, garantissant ainsi que le système de défense reste efficace contre les menaces émergentes.

\subsection{De l'AICA au SMA en Cyberdéfense}

Alors que l'AICA fournit une implémentation spécifique d'un système de Cyberdéfense autonome, le concept plus large de systèmes multi-agents (SMA) offre un cadre général qui peut être appliqué à divers domaines de la Cyberdéfense. Les SMA sont des systèmes composés de plusieurs agents autonomes en interaction qui collaborent pour atteindre un objectif commun \cite{kott2018towards}. Dans le contexte de la Cyberdéfense, cet objectif est la protection des systèmes et réseaux critiques contre les cybermenaces \cite{jajodia2017autonomous}.

La généralisation de l'AICA au SMA permet une plus grande flexibilité dans la conception et le déploiement des systèmes de Cyberdéfense. Le SMA peut être adapté à des environnements spécifiques, qu'il s'agisse de réseaux militaires, d'infrastructures d'entreprise ou d'écosystèmes IoT décentralisés \cite{kott2023autonomous}. Les principes qui guident le fonctionnement de l'AICA (décentralisation, autonomie, adaptabilité) sont également au cœur du SMA, ce qui en fait un cadre idéal pour relever les défis de la Cyberdéfense moderne \cite{kolias2011swarm}.

Dans un SMA de Cyberdéfense, chaque agent fonctionne comme une entité indépendante, capable de prendre des décisions en fonction des informations locales. Cependant, ces agents ne sont pas isolés ; ils communiquent entre eux pour partager des renseignements sur les menaces et coordonner leurs actions défensives \cite{kolias2011swarm}. Cette communication est cruciale dans les environnements distribués, où les menaces peuvent émerger simultanément à plusieurs endroits, nécessitant une réponse coordonnée \cite{bace2001intrusion}.

La nature décentralisée du SMA est particulièrement adaptée à la défense des réseaux distribués à grande échelle. Contrairement aux systèmes centralisés, qui reposent sur un point de contrôle unique, le SMA répartit la responsabilité de la défense sur l'ensemble du réseau. Cette répartition améliore non seulement la résilience du système, mais lui permet également d'évoluer plus efficacement. À mesure que le réseau se développe, des agents supplémentaires peuvent être déployés sans qu'il soit nécessaire de reconfigurer de manière significative le système global.

\subsection{Un SMA pour relever les défis de la Cyberdéfense}

Les défis de la Cyberdéfense moderne, tels que décrits dans la section précédente, peuvent être relevés efficacement grâce aux fonctionnalités clés du SMA. Ces fonctionnalités, qui incluent la décentralisation, la distribution, l'autonomie, la rapidité, la sécurité et l'adaptabilité, correspondent étroitement aux besoins d'un système de défense robuste et évolutif. Nous explorons ici en détail comment le SMA peut relever chaque défi.

\paragraph{Décentralisation et distribution :}
Les SMA fonctionnent de manière décentralisée, sans point de défaillance unique. Cela les rend particulièrement efficaces dans les environnements distribués où les attaques peuvent survenir à plusieurs endroits simultanément \cite{kolias2011swarm}. Chaque agent d'un SMA peut surveiller et protéger indépendamment un segment spécifique du réseau, garantissant que même si une partie du système est compromise, le reste reste sécurisé \cite{shakarian2015cyber}.

\paragraph{Autonomie et rapidité :}
L'autonomie des SMA leur permet de détecter et de répondre aux menaces en temps réel, sans attendre les instructions d'une autorité centrale \cite{shamshirband2014cooperative}. Cette rapidité est critique en Cyberdéfense, où le temps entre la détection d'une attaque et la réponse à celle-ci peut déterminer l'étendue des dégâts \cite{liao2013intrusion}. En fonctionnant de manière autonome, les agents SMA peuvent neutraliser les menaces dès qu'elles sont identifiées, minimisant ainsi l'impact sur le système \cite{shamshirband2018computational}.

\paragraph{Sécurité et résilience :}
Les SMA améliorent la sécurité des systèmes distribués en décentralisant les capacités défensives. Cette décentralisation rend plus difficile pour les attaquants de compromettre l'ensemble du système, car ils devraient vaincre plusieurs agents indépendants plutôt qu'une seule défense centralisée. De plus, la collaboration entre les agents permet une détection des menaces plus complète, car chaque agent apporte ses connaissances locales à l'effort de défense mondial.

\paragraph{Adaptabilité et apprentissage :}
L'un des principaux avantages des SMA en Cyberdéfense est leur capacité à s'adapter aux nouvelles menaces. Grâce à l'utilisation de l'apprentissage automatique et de l'IA, les agents SMA peuvent continuellement améliorer leurs stratégies de détection et de réponse aux menaces. Cette adaptabilité est cruciale face à l'évolution des cybermenaces, où les attaquants développent constamment de nouvelles méthodes pour violer les systèmes de sécurité. Les agents SMA peuvent tirer des leçons des expériences passées et ajuster leur comportement en conséquence, garantissant ainsi que le système de défense reste efficace au fil du temps.

\subsection{Vers un SMA de Cyberdéfense général}

En s'appuyant sur les principes établis par l'AICA, un cadre général pour le SMA en Cyberdéfense peut être développé. Ce cadre comprendrait plusieurs couches de défense, avec des agents opérant à différents niveaux du système, des appareils individuels aux segments de réseau entiers. Chaque couche aurait son propre ensemble d'agents chargés de surveiller, de détecter et de répondre aux menaces dans leur domaine spécifique \cite{bou-harb2017cyber}.

Au niveau le plus bas, les agents peuvent être déployés sur des appareils individuels, surveillant les signes de compromission tels qu'un trafic réseau inhabituel ou des tentatives d'accès non autorisées \cite{kolias2011swarm}. Ces agents communiqueraient entre eux pour partager des informations et coordonner leurs réponses \cite{shamshirband2018computational}. À des niveaux plus élevés, les agents pourraient être chargés de superviser des segments plus importants du réseau, garantissant que les menaces détectées au niveau de l'appareil ne se propagent pas davantage dans le système \cite{hu2018mimic}.

Cette approche de défense en couches permet une couverture plus complète de l'ensemble du réseau. En déployant des agents à plusieurs niveaux, le système peut détecter et répondre aux menaces plus rapidement et plus efficacement qu'un système de sécurité centralisé traditionnel. De plus, l'adaptabilité des agents SMA garantit que le système peut évoluer en réponse aux nouvelles menaces, en maintenant son efficacité au fil du temps.

\subsection{Conclusion : La place du SMA dans la Cyberdéfense}

Le concept de système multi-agents de Cyberdéfense représente une avancée significative dans le domaine de la cybersécurité. En s'appuyant sur les principes de décentralisation, d'autonomie et d'adaptabilité, les SMA offrent une solution robuste et évolutive aux défis de la Cyberdéfense moderne \cite{kolias2011swarm}. L'initiative AICA sert d'exemple concret de la manière dont ces principes peuvent être appliqués dans la pratique, en fournissant des informations précieuses sur la conception et le déploiement de SMA dans des environnements critiques \cite{bou-harb2014cyber}.

À mesure que les cybermenaces continuent d'évoluer, le besoin de systèmes de défense décentralisés et autonomes ne fera que croître. Les SMA offrent un cadre flexible qui peut être adapté à une large gamme d'applications, des opérations militaires aux réseaux d'entreprise et au-delà. Les chapitres suivants exploreront plus en détail les fondements théoriques et les applications pratiques des SMA, offrant un guide complet pour le développement et le déploiement de ces systèmes dans des scénarios de Cyberdéfense réels.


\section{Question de recherche et objectifs de la thèse}

Considérant la nécessité d'explorer des approches de défense décentralisées et autonomes pour répartir la prise de décision sur plusieurs entités collaboratives et autonomes, les systèmes multi-agents (SMA) apparaissent comme une solution prometteuse pour assurer la Cyberdéfense dans des environnements complexes et évolutifs. L'architecture AICA propose un cadre où chaque agent, doté d'une intelligence artificielle, est capable de réagir de manière autonome aux cybermenaces en temps réel. Cependant, la conception et l'organisation de tels systèmes posent des défis majeurs, car les agents doivent non seulement s'adapter à un environnement en constante évolution, mais aussi garantir la sûreté de leur fonctionnement.

La question centrale de cette thèse émerge directement de ces défis de conception dans le contexte de la Cyberdéfense moderne. Il s'agit de déterminer comment organiser efficacement ces agents autonomes pour qu'ils puissent assurer une protection robuste face à des attaques sophistiquées, tout en respectant les contraintes spécifiques des environnements dans lesquels ils seront déployés.

Ainsi, la question de recherche se formule comme suit :

\begin{quote}
    "Dans un contexte où les cyberattaques sont de plus en plus sophistiquées, comment peut-on organiser efficacement un système multi-agent de Cyberdéfense autonome intelligent, tel que les agents AICA, pour assurer une protection dynamique et adaptative face aux contraintes techniques, environnementales et aux malwares évolués ?"
\end{quote}

Cette question met en lumière plusieurs problématiques sous-jacentes qu'il convient de traiter pour proposer une réponse viable :

\textbf{Complexité et évolution des cyberattaques} : Les cyberattaques modernes sont à la fois imprévisibles et sophistiquées. Tester directement des agents AICA sur des réseaux réels, comme ceux des entreprises, pourrait entraîner des conséquences graves, allant de failles de sécurité à des interruptions de service majeures. Simuler l’environnement cible permet de modéliser ces menaces sans risque pour les infrastructures réelles, offrant ainsi un environnement sûr pour expérimenter, ajuster et optimiser les agents.

\textbf{Adéquation aux contraintes environnementales} : Les environnements de cybersécurité réels sont souvent complexes, présentant des contraintes techniques spécifiques (topologie de réseau, infrastructures de sécurité existantes, etc.). Ces contraintes rendent les expérimentations directes à la fois risquées et inefficaces, car elles n'offrent pas la flexibilité nécessaire pour tester un large éventail de scénarios ou configurations. La simulation permet de recréer ces contraintes et d’évaluer les capacités des agents AICA dans divers scénarios, tout en facilitant leur adaptation.

\textbf{Apprentissage sans supervision humaine directe} : La complexité croissante des cybermenaces rend difficile, voire impossible, le développement manuel de stratégies de défense optimales. Le Multi-Agent Reinforcement Learning (MARL) permet aux agents de découvrir et d’apprendre par eux-mêmes des stratégies adaptées au contexte. Ce processus résout les défis liés à l'imprévisibilité des cyberattaques évolutives, permettant aux agents de s'entraîner dans des environnements simulés et d'améliorer leur comportement de manière autonome.

\textbf{Explicabilité et sûreté} : Les systèmes autonomes, tels que les agents AICA, sont souvent perçus comme des "boîtes noires", rendant difficile leur compréhension par les humains. Pour assurer une sécurité optimale, il est essentiel que les actions et décisions des agents soient interprétables et compréhensibles. En simulant et en testant ces agents, puis en traduisant leur comportement en termes de rôles et de missions, on garantit la sûreté du système avant son déploiement réel, tout en fournissant aux concepteurs humains des indicateurs clairs pour évaluer la pertinence et la sécurité des stratégies mises en place.

\textbf{Garantie de non-dommage à l'environnement réel} : Un objectif clé de l’approche basée sur la simulation est de garantir que les agents AICA n’introduiront pas de nouveaux risques ou dysfonctionnements lorsqu'ils seront déployés dans des environnements réels. Cela revêt une importance particulière dans des contextes critiques tels que les réseaux d'entreprise ou les infrastructures militaires. En validant d’abord le comportement des agents dans un environnement simulé, il est possible d’identifier et de corriger d’éventuelles erreurs avant qu’elles ne causent des dommages irréversibles dans le monde réel.

\vspace{0.5cm}

Compte tenu de ces propriétés essentielles, nous avons orienté l’objectif général de cette thèse vers l'établissement d'une \textbf{méthode de conception} d’un système multi-agent de Cyberdéfense. Cette méthode doit intégrer les contraintes spécifiques des environnements techniques et répondre aux objectifs de Cyberdéfense, tout en fournissant des informations explicites et compréhensibles ainsi qu'une instance déployable d'un SMA adéquat, capable d’être facilement mis à jour.

Pour atteindre cet objectif, notre méthode repose sur quatre étapes successives qui constituent les principaux axes de nos contributions :

\begin{enumerate}
    \item \textbf{Reproduction en simulation} : Création d’un environnement de simulation fidèle à l’environnement cible, dans lequel la logique des agents attaquants est prédéfinie, mais celle des agents défenseurs reste à établir. Des outils d’évaluation des performances vis-à-vis des objectifs de Cyberdéfense seront également intégrés. Cette étape correspond à la modélisation du problème (nommée CybMASFM) qui sera résolue les étapes suivantes que constituent l'approche de résolution (nommée CybMASDA).
    
    \item \textbf{Entraînement en simulation} : Les agents défenseurs apprennent à contrer les attaques et à s’adapter à divers scénarios. Cette étape représente la résolution d’un problème d'optimisation sous contraintes, où les logiques des agents défenseurs doivent être découvertes.
    
    \item \textbf{Explicitation en spécifications organisationnelles} : Une fois entrainés, les comportements des agents défenseurs sont traduits en spécifications organisationnelles pertinentes pour la conception d’un SMA déployable dans une copie émulée de l’environnement cible. Cette étape vise à rendre la solution explicable afin d’assister ou d’automatiser la conception du SMA.
    
    \item \textbf{Validation et transfert} : Une fois le SMA affiné en émulation et validé, il est prêt à être déployé dans l’environnement réel cible. Cette phase permet de vérifier que le SMA atteint ses objectifs de Cyberdéfense sans nuire à la sûreté de l’environnement réel avant son déploiement final.
\end{enumerate}

En résumé, cette thèse vise à répondre à la question de recherche en proposant une solution combinant simulation, émulation, apprentissage autonome et explicabilité, afin de développer un SMA de Cyberdéfense à la fois efficace et sécurisé. Les chapitres suivants détailleront les contributions proposées pour atteindre cet objectif, ainsi que les résultats obtenus lors de la phase de simulation et d’expérimentation.


\section{Organisation du manuscrit}

La structure de cette thèse est organisée en six chapitres, chacun s'appuyant sur les fondations posées par le précédent pour répondre systématiquement aux questions et objectifs de recherche décrits précédemment.

% \autoref{ch:introduction} fournit une introduction au domaine problématique, en soulignant la complexité croissante des cybermenaces et l'insuffisance des mécanismes de défense centralisés. Il introduit le concept de systèmes multi-agents (SMA) pour la Cyberdéfense, ouvrant la voie à l'exploration détaillée de cette approche dans les chapitres suivants.

\autoref{ch:towards_cSMA} se penche sur le contexte théorique de la Cyberdéfense distribuée et décentralisée. Il aborde les concepts clés, les défis et les solutions existantes dans le domaine, en mettant l'accent sur le rôle des systèmes multi-agents (SMA). Ce chapitre identifie les verrous dans la réalisation d'un SMA de Cyberdéfense pleinement opérationnel en tant que questions de recherche. 

\autoref{ch:problem} fait la synthèse des différents verrous théoriques identifiés pour exprimer les problèmes que nous avons considéré. Ce chapitre propose plusieurs hypothèses proposées pour établir notre méthode de développement pour les SMA de Cyberdéfense qui constitue notre contribution générale de la thèse.

\autoref{ch:cybSMAdm} présente la méthode proposée pour la conception et le développement d'un système multi-agents de Cyberdéfense. Il commence par présenter le cadre permettant de modéliser rigoureusement le problème de conception tout en incorporant diverses techniques d'IA pour le résoudre. Pour résoudre le problème de conception exprimé dans ce cadre, nous proposons une approche de développement assisté exploitant l'apprentissage par renforcement multi-agents (MARL) et un modèle organisationnel (OM) pour guider la génération d'un SMA de Cyberdéfense efficace, tout en fournissant des informations de conception explicites.

\autoref{ch:case_studies} présente la configuration expérimentale et la validation du SMA de Cyberdéfense. Il comprend des études de cas détaillées qui démontrent l'application de la méthode de développement dans divers scénarios, tels que l'infrastructure d'entreprise, les essaims de drones et l'orchestration de Kubernetes. Le chapitre analyse également les résultats, soulignant l'efficacité et les limites de la méthode de développement proposée.

Enfin, \autoref{ch:conclusion} conclut la thèse en résumant les principales contributions de la recherche et en soulignant les résultats obtenus à partir de nos expériences. Il suggère également des orientations de recherche futures potentielles et discute des implications plus larges du travail pour le domaine de la Cyberdéfense.


% \section{Un besoin croissant de Cyberdéfense décentralisée}
% \begin{itemize}
% \item AICA, besoins nouveaux, IoT/IoBT, etc.
% \item Approche centralisée peu adaptée, etc. pour des raisons d'interruptions de communication, hétérogénéité des SI, etc.
% \item Une approche MA pourrait être appliquée -> Système Multi-Agents de Cyberdéfense (SMAC)
% \end{itemize}

% \section{Les défis d'une Cyberdéfense multi-agents}
% \begin{itemize}
% \item Mais sujet nouveau : pas de modélisation, travaux formels…
% \end{itemize}

% \section{Question de recherche et objectifs}
% \begin{itemize}
% \item Quelle méthode pour concevoir un SMAC qui atteint ses objectifs de Cyberdéfense tout en satisfaisant les contraintes de déploiement et opérationnelles du système hôte qu'il doit défendre ?
% \end{itemize}

% \section{Positionnement et apport de cette thèse}
% \begin{itemize}
% \item Un ensemble d'agents collaboratifs répond efficacement aux nouveaux besoins, etc. mieux que des solutions centralisées
% \item Une méthode modélisant le "domaine" (environnement réseau + actions/observations possibles des red/blue/green teams), du "problème" (objectif de Cyberdéfense / blue team + contraintes opérationnelles/déploiement) sous forme d'un * *problème d'optimisation sous-contraintes*\item ; permet de fournir des moyens objectifs d'évaluer de façon cohérente si le SMA tient ses promesses dans plusieurs scénarios d'attaque…
% \end{itemize}