%************************************************
\chapter{Méthode de conception de systèmes multi-agents de Cyberdéfense}\label{ch:cybSMAdm} % $\mathbb{ZNR}$
%************************************************

\section{CybSMAFM : Modèle formel des systèmes multi-agents de Cyberdéfense}\label{sec:cybSMAfm}

\begin{itemize}
    \item \textbf{Ici, on cherche à poser les bases pour un modèle qui exprime de façon cohérente et sans-ambiguïtés/formellement les éléments dont nous avons besoin pour le domaine et le problème}
    \item \textbf{Cela est fondamental pour la suite car l'approche et l'outil se repose sur CybSMAFM}
\end{itemize}

\subsection{Approches de modélisation pour un système multi-agents de Cyberdéfense}
\begin{itemize}

    \item Modèles partiels (Arbre attaque-défense, réseaux de Pétri, etc)
    \item Modèles de la théorie des jeux (POSG, Dec-POMDP, etc.)
\end{itemize}

\subsection{Comparaison des modèles pour le cadre théorique}
\begin{itemize}

    \item Tableau comparatif, discussion…
    \item Motivations pour le Dec-PODMP d'après le rapprochement entre incertitude des observations, conditions pour les actions, dynamique de l'env, récompense collective et non individuelle, etc.
    \item Le Dec-POMDP reste assez générique : plusieurs approches d'organisation / mécanismes / algos d'IA sont envisageables dans ce cadre formel
\end{itemize}

\subsection{Un modèle formel pour le cadre théorique}
\begin{itemize}

    \item Montrer comment le cadre théorique non formel (environnement + équipes) peut être formalisé dans le Dec-POMDP
          \begin{itemize}
              \item Le domaine doit aussi inclure la question de l'organisation dans l'équipe bleu (+ rouge éventuellement)
          \end{itemize}
    \item Le modèle formel du domaine (ie le modèle formel de l'environnement et le modèle formel de l'organisation) doit permettre d'exprimer formellement la question de l'organisation comme un problème d'optimisation sous contraintes (ie de l'organisation). environnement et de l'organisation voulue par le concepteur càd ses spécifications)
\end{itemize}

\subsection{Le modèle d'environnement}
\begin{itemize}
    \item Montrer comment lier le Dec-POMDP avec les connaissances liées aux attaques (MITRE ATTACK) et où on peut mettre en place des contre-mesures (MITRE DEFEND)
    \item À FAIRE...
\end{itemize}

\subsection{Le modèle d'organisation SMA}
\begin{itemize}

    \item Expliquer comment nous envisageons les organisations possibles au travers de : Conscience/inconscience de l'organisation + Centré agent / organisation.
          \begin{itemize}
              \item Positionnement par rapport à la littérature
          \end{itemize}
    \item Expliquer notre vision où une organisation dans un SMA peut être résumée comme : le résultat d'une recherche dans un espace des organisations contraint par les contraintes liées à l'environnement et celles du concepteur (spécifications initiales du concepteur)
    \item On peut illustrer cela dans les 3 cas ci-dessous
    \item **Les organisations "totalement prédéfinies"*\item :
          \begin{itemize}
              \item Les spécifications initiales du concepteur contraignent totalement le processus de conception à une seule organisation possible
              \item 1 seul épisode : chaque agent suit une liste de règles (associant un ensemble d'observation à une action) qui ne changent jamais (sans apprentissage mais basé sur la connaissance/expertise du concepteur) : hiérarchie, mécanisme d'enchère, coalition , etc.
                    Système multi-agents basé sur une coalition AICA (CSMAA), système multi-agents basé sur le marché AICA (MSMAA), etc.
          \end{itemize}
    \item \textbf{Les organisations "totalement indéfinies"}
          \begin{itemize}
              \item Les spécifications initiales du concepteur n'ont aucun impact sur la façon dont on peut concevoir les agents (le concepteur humain ou MARL peut choisir librement comment concevoir les règles des agents)
              \item Plusieurs épisodes : chaque agent voit ses propres règles changer quand cela maximise la récompense (QLearning, DQN en full automatique ou bien le concepteur humain)
              \item Aboutit à une solution locale (ensemble de politique) mais pas facilement explicable en MARL (d'où le besoin d'avoir des spécifications en même temps)
                    Système multi-agent basé sur QLearning AICA (QSMA), etc.
          \end{itemize}
    \item \textbf{Les organisations "semi-définies"}
    \item Les spécifications initiales du concepteur couvrent partiellement l'espace des organisations possibles
          \begin{itemize}
              \item Par exemple : trouver des SMA satisfaisant l'architecture hiérarchique mais sans savoir précisément quel agent doit adopter quel rôle, etc.
          \end{itemize}
    \item Mécanisme d'organisation déjà en place mais les hyper-paramètres doivent être ajustés avec un apprentissage
    \item Mix entre agents aux politiques définies et indéfinies
    \item D'abord indéfini avec recherche du mécanisme d'organisation défini optimal pour le scénario donné
    \item Adaptive Multi Agent System AICA (ASMAA) : Une architecture polyvalente pour un SMA de Cyberdéfense, etc.
\end{itemize}

\subsection{Exprimer formellement le problème}
\textbf{Ici, j'utilise les modèles formels précédents pour poser le problème de façon formelle}

\textbf{ici, on ne doit pas encore comprendre que le MARL sera choisi pour la suite en combinaison des modèles d'organisation de SMA comme Moise+}

\begin{itemize}
    \item L'environnement peut être traduit en contraintes qui réduisent l'espace des organisations (ie politique commune) à celles qui sont réellement possibles (cf. Moise+)
    \item Les spécifications appliquées à l'OE contraignent également l'espace des organisations (ie joint-policy) à celles qui sont prescrites par le concepteur
    \item Les objectifs de Cyberdéfense peuvent être traduits en une fonction à maximiser
    \item \textbf{Problème} : Trouver l'ensemble des politiques (joint-policy) respectant les contraintes telles que sur un épisode maximal, la récompense cumulée des agents bleus soit / supérieure à un seuil
    \item \begin{itemize}
              \item Pour une fonction de récompense donnée Rew, les variables inconnues à maximiser sont les politiques PIj (1<j<nb agents) :
                    $max (Somme{1, .., i, .., nb\_it} Rew(PI1, PI2, …, PIn))$
              \item De plus, la joint-policy obtenue doit être associée à des spécifications compréhensibles pour un être humain
                    \begin{itemize}
                        \item $Design(Env., Spécifications\_init, JointPolicy\_init) = (Spécifications\_opt, JointPolicy\_opt)$
                    \end{itemize}
          \end{itemize}
    \item \textbf{Maintenant que l'on a posé le problème il faut le résoudre...}
\end{itemize}

\section{CybSMADA : Approche de développement de systèmes multi-agents de Cyberdéfense}\label{sec:cybSMAda}

\begin{itemize}
    \item \textbf{Ici, on se positione au niveau du concepteur et du développeur qui veut un résultat tangible à la fin}
    \item \textbf{On propose une approche qui utilise le problème formel décrit précédement pour concevoir l'organisation du SMA en simulation puis on l'implémente réellement en émulation}
\end{itemize}

\subsection{Vers l'exploitation de MARL pour la génération d'organisations SMA}

\begin{itemize}

    \item Présenter les travaux qui visent à passer des joint-policy du MARL aux spécifications du modèle organisationnel Moise+
          \begin{itemize}

              \item Travaux sur les résultats obtenus en MARL après entraînement pour en extraire les spécifications en Moise+
              \item Travaux pour contraindre l'entraînement MARL à respecter des spécifications décrites avec Moise+
          \end{itemize}
    \item La définition du problème autorise au moins 2 cas différents qui seront présentés sous forme d'exemples :
          \begin{itemize}
              \item \textbf{Spec\_Init vide, joint-policy -> Moise+}: Moise+ intervient après l'entraînement en MARL
              \item \begin{itemize}
                        \item La joint-policy est obtenue après entraînement en MARL (qui inclut implicitement les contraintes de l'environnement) puis il faut faire un travail d'analyse (au moins partiellement automatisable) pour en extraire les spécifications de l'organisation de Moise+
                    \end{itemize}
              \item \textbf{Spec\_Init non vide, joint-policy -> Moise+}: Moise+ intervient après et pendant l'entraînement en MARL
              \item \begin{itemize}
                        \item La joint-policy est obtenue après entraînement en MARL qui doit prendre en compte les contraintes des Spec\_Init (en plus des contraintes de l'environnement) puis il faut faire un travail d'analyse (au moins partiellement automatisable) pour en extraire les spécifications de l'organisation de Moise+
                    \end{itemize}
          \end{itemize}
    \item Montrer qu'on peut aussi imaginer d'autres exemples...
          \begin{itemize}
              \item On a défini "à la main" une joint-policy et on veut connaître les spécifications de l'organisation en Moise+
              \item On a des spécifications initiales qui contraignent le MARL à converger vers une seule organisation possible et on veut déterminer la joint-policy correspondante
          \end{itemize}
    \item \textbf{L'attendu de cette approche est que le concepteur doit être en mesure de posséder une joint-policy / des joint-policies avec les spécifications de l'organisation associée qui sont suffisamment performantes et respectent les contraintes}
\end{itemize}

\subsection{Une approche reposant sur le couplage simulation-émulation}

\subsubsection{Tirer parti de la simulation pour concevoir des SMA de Cyberdéfense}
\begin{itemize}

    \item Présentations des travaux correspondant au mieux aux besoins du SMA et de la Cyberdéfense
    \item Comparaison et aboutissement sur l'idée d'étendre l'environnement CybORG du framework PettingZoo (code libre, issu d'un travail de recherche précédent publié à IJCAI, contexte d'application très proche et pertinent, compatibilité avec le modèle Dec-POMDP précédent, etc.)
    \item L'approche de conception de l'organisation peut être de façon sûre appliquée en simulation car il n'y a pas de risque d'endommager le système cible
    \item Possibilité de faire du "system identification" pour créer le modèle de simulation et ainsi réduire l'écart entre émulation et simulation
          \begin{itemize}
              \item Cela permettra de ne pas utiliser CybORG
          \end{itemize}
    \item + autres avantages de la simulation
\end{itemize}

\subsubsection{Évaluation du SMA conçu en émulation}
\begin{itemize}

    \item Reproduction du système cible sous une forme émulée (avec conteneur)
    \item Mise en place d'un dispositif expérimental pour transférer les agents de la simulation en émulation
    \item Validation des candidats SMAC et implémentation dans le système cible

\end{itemize}

\section{CybSMADE : Environnement de développement de systèmes multi-agents de Cyberdéfense}\label{sec:cybSMAde}

\begin{itemize}

    \item Montrer comment CybSMADE peut être utilisé de façon systématique/consistante pour définir un scénario (env, red team, green team) + une blue team (ie un SMA de Cyberdéfense) en définissant soit même les politiques des agents
    \item Montrer que les modèles de la simulation peuvent être mappés à des modèles émulés afin de vérifier la véritable performance des SMA de Cyberdéfense proposé AVEC l'intérêt du transfert d'apprentissage pour l'apprentissage dans la simulation (car rapide et léger) et vérification dans l'émulation (machines virtuelles)
\end{itemize}