%************************************************
\chapter{CybMASDA: Cyberdefense Multi-Agent Systems Developpment Approach}\label{ch:cybmasda} % $\mathbb{ZNR}$
%************************************************

\begin{itemize}
  \item \textbf{Ici, on se positione au niveau du concepteur et du développeur qui veut un résultat tangible à la fin}
  \item \textbf{On propose une approche qui utilise le problème formel décrit précédement pour concevoir l'organisation du MAS en simulation puis on l'implémente réellement en émulation}
\end{itemize}

\section{Approche de conception théorique combinant MARL et modèle d'organisation de MAS}

\begin{itemize}

  \item Présenter les travaux qui visent à passer des joint-policy du MARL aux spécifications du modèle organisationel Moise+
        \begin{itemize}

          \item Travaux sur les résultats obtenus en MARL après entrainement pour en extraire les spécifications en Moise+
          \item Travaux pour contraindre l'entrainement MARL à respecter des spécifications décrites avec Moise+
        \end{itemize}
  \item La définition du problème autorise au moins 2 cas différents qui seront présentés sous forme d'exemples :
        \begin{itemize}
          \item \textbf{Spec\_Init vide, joint-policy -> Moise+}: Moise+ intervient après l'entrainement en MARL
          \item \begin{itemize}
                  \item La joint-policy est obtenue après entrainement en MARL (qui inclut implicitement les contraintes de l'environnement) puis il faut faire un travail d'analyse (au moins partiellement automatisable) pour en extraire les spécifications de l'organisation de Moise+
                \end{itemize}
          \item \textbf{Spec\_Init non vide, joint-policy -> Moise+}: Moise+ intervient après et pendant l'entrainement en MARL
          \item \begin{itemize}
                  \item La joint-policy est obtenue après entrainement en MARL qui doit prendre en compte les contraintes des Spec\_Init (en plus des contraintes de l'environnement) puis il faut faire un travail d'analyse (au moins partiellement automatisable) pour en extraire les spécifications de l'organisation de Moise+
                \end{itemize}
        \end{itemize}
  \item Montrer qu'on peut aussi imaginer d'autres exemples...
        \begin{itemize}
          \item On a définit "à la main" une joint-policy et on veut savoir les spécifications de l'organisation en Moise+
          \item On a des spécifications initiales qui contraignent le MARL à converger vers une seule organisation possible et on veut determiner la joint-policy correspondante
        \end{itemize}
  \item \textbf{L'attendu de cette approche est que le concepteur doit être en mesure de posseder une joint-policy / des joint-policies avec les spécifications de l'organisation associées qui sont suffisament performantes et respectent les contraintes}
\end{itemize}

\section{Approche de développement basée sur des cycles de simulation et émulation}

\section{La simulation pour la conception d'organisation de MAS candidates}
\begin{itemize}

  \item Présentations des travaux correspondant au mieux aux besoins des SMA et de la cyberdéfense
  \item Comparaison et aboutissement sur l’idée d’étendre l’environnement CybORG du framework PettingZoo (code libre, issu d’un travail de recherche précédent publié à IJCAI, contexte d’application très proche et pertinent, compatibilité avec le modèle Dec-POMDP précédent, etc.)
  \item L'approche de conception de l'organisation peut être de façon sûre appliquée en simulation car il n'y a pas de risque d'endommager le système cible
  \item Possibilité de faire du "system identification" pour créer le modèle de simulation et ainsi réduire le gap entre émulation et simulation
        \begin{itemize}
          \item Cela permettrait de ne pas utiliser CybORG
        \end{itemize}
  \item + autres avantages de la simulation
\end{itemize}

\section{L'émulation pour valider un SMAC candidat}
\begin{itemize}

  \item Reproduction du système cible sous une forme émulée (avec container)
  \item Mise en place d'un dispositif experimental pour transferer les agents de la simulation en émulation
  \item Validation des SMAC candidats et implémentation dans le système cible

\end{itemize}
