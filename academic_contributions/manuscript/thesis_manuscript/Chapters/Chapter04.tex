%************************************************
\chapter{Setting Up an AICA Agent Model}\label{ch:case_studies} % $\mathbb{ZNR}$
%************************************************

\begin{itemize}
    \item \textbf{Ici l'idée est de créer un modèle AICA integré dans notre méthode et qui pourra être utilisé comme un SMA polyvalent capable d'être ajusté pour des scénarios différents}
\end{itemize}


\section{Integrating an AICA agent into CybMASDE}
\begin{itemize}
    \item Traduire au moins dans l'idée les composants de MASCARA comme des spécifications de l'organisations qui seront pris en compte pour generer une organisation dans CybMASDA
    \item Expliquer comment nous avons pris en compte l'AICA dans l'outil afin qu'il soit utilisable directement
    \item Problématiques d'implémentation dues à la complexité de l'architecture
\end{itemize}


\section{Experiments through Three Case Studies}
\begin{itemize}
    \item Présentation de 3 cas d’études : Drone swarm, company network et Kubernetes
    \item Sur le modèle du tutoriel en utilisant l'outil CybMASDE
    \item Montrer comment nous utilisons notre modèle concrètement avec CybMASDE pour comprendre et définir le problème dans chaque étude de cas
\end{itemize}

\subsection{A Company Infrastructure Scenario}
\subsection{A Drone Swarm Scenario}
\subsection{A Kubernetes Orchestration Scenario}


\section{Results and Discussion}
\begin{itemize}

    \item Montrer comment nous utilisons CybMASDE pour résoudre le problème
          \begin{itemize}
              \item En utilisant également le modèle AICA
          \end{itemize}
\end{itemize}

\section{General Trends Emerging from Results and Synthesis}
\begin{itemize}
    \item Discuter le niveau de l'impact de la méthode
          \begin{itemize}
              \item Sans spécifications initiales
              \item Avec spécifications initiales (incluant l'AICA)
          \end{itemize}
    \item Discuter des organisations qui semblent pertinentes avec une analyse quantitative sur les 3 cas d'étude
    \item Discuter des résultats et la cohérence de ces-derniers par rapport à d'autres résultats
\end{itemize}
