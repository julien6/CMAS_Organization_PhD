%************************************************
\chapter{Math Test Chapter}\label{ch:mathtest} % $\mathbb{ZNR}$
%************************************************
Ei choro aeterno antiopam mea, labitur bonorum pri no. His no decore
nemore graecis. In eos meis nominavi, liber soluta vim cu. Sea commune
suavitate interpretaris eu, vix eu libris efficiantur.


\section{CybMASFM: Cyberdefense Multi-Agent Systems Formal Model}
\begin{itemize}

    \item \textbf{Ici, on cherche à poser les bases pour un modèle qui exprime de façon cohérente et sans-ambiguités/formellement les éléments dont nous avons besoin pour le domaine et le problème}
    \item \textbf{Cela est fondammental pour la suite car l'approche et l'outil se repose sur CybMASFM}
\end{itemize}

\subsection{Approches de modélisation pour un SMA de Cyberdéfense}
\begin{itemize}

    \item Modèles partiels (Attack-defense tree, petri nets, etc)
    \item Modèles de la théorie des jeux (POSG, Dec-POMDP, etc.)
\end{itemize}

\subsection{Comparaison et choix du modèle}
\begin{itemize}

    \item Tableau comparatifs, discussion…
    \item Motivations pour le Dec-PODMP d’après le rapprochement entre incertitude des observations, conditions pour les actions, dynamique de l’env, récompense collective et non individuelle, etc.
    \item Le Dec-POMDP reste assez générique : plusieurs approche d’organisation / mécanismes / algos d’IA sont envisageable dans ce cadre formel
\end{itemize}

\subsection{Modèle formel du domaine}
\begin{itemize}

    \item Montrer comment le cadre théorique non-formel (environnement + teams) peut être formalisé dans le Dec-POMDP
          \begin{itemize}
              \item Le domaine doit aussi inclure la question de l'organisation dans l'équipe bleu (+ rouge éventuellement)
          \end{itemize}
    \item Le modèle formel du domaine (i.e le modèle formel de l'environnement et le modèle formel de l'organisation) doit permettre d'exprimer formellement la question de l'organisation comme un problème d'optimisation sous contraintes (i.e de l'environnement et de l'organisation voulue par le concepteur càd ses spécifications)
\end{itemize}

\subsection{Modèle de l'environnement}
\begin{itemize}

    \item Montrer comment lier le Dec-POMDP avec les connaissances liées aux attaques (MITRE ATTACK) et où on peut mettre en place des contre-mesures (MITRE DEFEND)
    \item TODO...
\end{itemize}

\subsection{Modèle d'une organisation dans les SMA}
\begin{itemize}

    \item Expliquer comment nous envisageons les organisations possibles au travers de : Conscience/inconscience de l’organisation + Centré agent / organisation.
          \begin{itemize}
              \item Positionnement par rapport à la littérature
          \end{itemize}
    \item Expliquer notre vision où une organisation dans un MAS peut être résumée comme : le résultat d'une recherche dans un espace des organisations contraint par les contraintes liées à l'environnement et celles du concepteur (spécifications initiales du concepteur)
    \item On peut illustrer cela dans les 3 cas ci-dessous
    \item **Les organisations "totalement prédéfinis"*\item :
          \begin{itemize}
              \item Les spécifications initiales du concepteur contraignent totalement le processus de conception à une seule organisation possible
              \item 1 seul épisode : chaque agent suit une liste de règles (associant un ensemble d’observation à une action) qui ne changent jamais (sans apprentissage mais basé sur la connaissance/expertise du concepteur) : hiérarchie, mécanisme d’enchère, coalition, etc.
              \item Coalition based Multi Agent System AICA (CMASA), Market based Multi Agent System AICA (MMASA), etc.
          \end{itemize}
    \item \textbf{Les organisations "totalement indéfinies"}
          \begin{itemize}
              \item Les spécifications initiales du concepteur n'ont aucun impact sur la façon dont on peut concevoir les agents (le concepteur humain ou MARL peut choisir librement comment concevoir les règles des agents)
              \item Plusieurs épisodes : chaque agent voit ses propres règles changer quand cela maximise la récompense (QLearning, DQN en full automatique ou bien le concepteur humain)
              \item Aboutit à une solution local (ensemble de politique) mais pas facilement explicable en MARL (d'où le besoin d'avoir des spécifications en même temps)
              \item QLearning based Multi Agent System AICA (QMAS), etc.
          \end{itemize}
    \item \textbf{Les organisation "semi-définies"}
    \item Les spécifications initiales du concepteur couvrent partiellement l'espace des organisations possibles
          \begin{itemize}
              \item Par exemple : trouver des SMA satisfaisant l'architecture hierarchique mais sans savoir précisement quel agent doit adopter quel rôle, etc.
          \end{itemize}
    \item Mécanisme d’organisation déjà en place mais les hyper-paramètres doivent être ajustés avec un apprentissage
    \item Mix entre agents aux politiques définis et indéfinis
    \item D’abord indéfini avec recherche du mécanisme d’organisation défini optimal pour le scénario donné
    \item Adaptive Multi Agent System AICA (AMASA) : Une architecture polyvalente pour un SMA de Cyberdéfense, etc.
\end{itemize}

\subsection{Expression du problème}
\textbf{Ici, j'utilise les modèles formels précédents pour poser le problème de façon formelle}

\textbf{ici, on ne doit pas encore comprendre que le MARL sera choisi pour la suite en combinaison des modèles d'organisation de SMA comme Moise+}

\begin{itemize}

    \item L'environnement peut être traduit en contraintes qui réduit l'espace des organisations (i.e joint-policy) à celles qui sont réellement possibles (cf. Moise+)
    \item Les spécifications appliquées à l'OE contraignent également l'espace des organisations (i.e joint-policy) à celles qui sont spécifiées par le concepteur
    \item Les objectifs de cyberdéfense peuvent être traduits en une fonction à maximiser
    \item \textbf{Problème} : Trouver l’ensemble des politiques (joint-policy) respectant les contraintes telles que sur un épisode, la récompense cumulée des agents bleus soit maximale / supérieur à un seuil
    \item \begin{itemize}
              \item Pour une fonction de récompense donnée Rew, les variables inconnues à maximiser sont les politiques PIj (1<j<nb agents):
                    $max (Sum{1, .., i, .., nb\_it} Rew(PI1, PI2, …, PIn))$
              \item De plus, la joint-policy obtenue doit être associé à des specifications compréhensibles pour un être humain
                    \begin{itemize}
                        \item $Design(Env., Specs\_init, JointPolicy\_init) = (Specs\_opt, JointPolicy\_opt)$
                    \end{itemize}
          \end{itemize}
    \item \textbf{Maintenant que l'on a posé le problème il faut le résoudre...}
\end{itemize}

% =====================================

\section{Some Formulas}
Due to the statistical nature of ionisation energy loss, large
fluctuations can occur in the amount of energy deposited by a particle
traversing an absorber element\footnote{Examples taken from Walter
Schmidt's great gallery: \\
\url{http://home.vrweb.de/~was/mathfonts.html}}.  Continuous processes
such as multiple
scattering and energy loss play a relevant role in the longitudinal
and lateral development of electromagnetic and hadronic
showers, and in the case of sampling calorimeters the
measured resolution can be significantly affected by such fluctuations
in their active layers.  The description of ionisation fluctuations is
characterised by the significance parameter $\kappa$, which is
proportional to the ratio of mean energy loss to the maximum allowed
energy transfer in a single collision with an atomic electron:
\graffito{You might get unexpected results using math in chapter or
section heads. Consider the \texttt{pdfspacing} option.}
\begin{equation}
\kappa =\frac{\xi}{E_{\textrm{max}}} %\mathbb{ZNR}
\end{equation}
$E_{\textrm{max}}$ is the maximum transferable energy in a single
collision with an atomic electron.
\[
E_{\textrm{max}} =\frac{2 m_{\textrm{e}} \beta^2\gamma^2 }{1 +
2\gamma m_{\textrm{e}}/m_{\textrm{x}} + \left ( m_{\textrm{e}}
/m_{\textrm{x}}\right)^2}\ ,
\]
where $\gamma = E/m_{\textrm{x}}$, $E$ is energy and
$m_{\textrm{x}}$ the mass of the incident particle,
$\beta^2 = 1 - 1/\gamma^2$ and $m_{\textrm{e}}$ is the electron mass.
$\xi$ comes from the Rutherford scattering cross section
and is defined as:
\begin{eqnarray*} \xi  = \frac{2\pi z^2 e^4 N_{\textrm{Av}} Z \rho
\delta x}{m_{\textrm{e}} \beta^2 c^2 A} =  153.4 \frac{z^2}{\beta^2}
\frac{Z}{A}
  \rho \delta x \quad\textrm{keV},
\end{eqnarray*}
where

\begin{tabular}{ll}
$z$          & charge of the incident particle \\
$N_{\textrm{Av}}$     & Avogadro's number \\
$Z$          & atomic number of the material \\
$A$          & atomic weight of the material \\
$\rho$       & density \\
$ \delta x$  & thickness of the material \\
\end{tabular}

$\kappa$ measures the contribution of the collisions with energy
transfer close to $E_{\textrm{max}}$.  For a given absorber, $\kappa$
tends
towards large values if $\delta x$ is large and/or if $\beta$ is
small.  Likewise, $\kappa$ tends towards zero if $\delta x $ is small
and/or if $\beta$ approaches $1$.

The value of $\kappa$ distinguishes two regimes which occur in the
description of ionisation fluctuations:

\begin{enumerate}
\item A large number of collisions involving the loss of all or most
  of the incident particle energy during the traversal of an absorber.

  As the total energy transfer is composed of a multitude of small
  energy losses, we can apply the central limit theorem and describe
  the fluctuations by a Gaussian distribution.  This case is
  applicable to non-relativistic particles and is described by the
  inequality $\kappa > 10 $ (\ie, when the mean energy loss in the
  absorber is greater than the maximum energy transfer in a single
  collision).

\item Particles traversing thin counters and incident electrons under
  any conditions.

  The relevant inequalities and distributions are $ 0.01 < \kappa < 10
  $,
  Vavilov distribution, and $\kappa < 0.01 $, Landau distribution.
\end{enumerate}


\section{Various Mathematical Examples}
If $n > 2$, the identity
\[
  t[u_1,\dots,u_n] = t\bigl[t[u_1,\dots,u_{n_1}], t[u_2,\dots,u_n]
  \bigr]
\]
defines $t[u_1,\dots,u_n]$ recursively, and it can be shown that the
alternative definition
\[
  t[u_1,\dots,u_n] = t\bigl[t[u_1,u_2],\dots,t[u_{n-1},u_n]\bigr]
\]
gives the same result.  

%*****************************************
%*****************************************
%*****************************************
%*****************************************
%*****************************************
