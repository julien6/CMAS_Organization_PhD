%************************************************
\chapter{Problèmes détaillées et hypothèses de travail}\label{ch:problem}
%************************************************


\section{Problèmes à considérer}

Dans le cadre de cette thèse, plusieurs défis et problèmes clés émergent de la littérature sur les systèmes multi-agents (SMA), l’apprentissage par renforcement multi-agent (MARL), l'explicabilité des modèles organisationnels, et la cyberdéfense. Ces problèmes doivent être pris en compte pour répondre efficacement à la question de recherche.

\paragraph{Problème 1 : Complexité et évolutivité des environnements cyberdéfensifs.}
Les environnements de cybersécurité, tels que les réseaux d'entreprise ou les infrastructures critiques, sont souvent dynamiques, évolutifs et très complexes \cite{Calo2017, Kott2019}. Cette complexité pose un problème majeur pour la conception d'un SMA capable de s'adapter en temps réel aux cyberattaques sophistiquées. Il n'existe pas de méthode unifiée dans la littérature pour organiser les agents de manière à ce qu'ils puissent efficacement prendre des décisions autonomes dans ces environnements sous forte contrainte.

\paragraph{Problème 2 : Entraînement sans supervision des agents.}
L'entraînement des agents AICA repose principalement sur des processus d'apprentissage autonome, notamment à travers le MARL, qui permet aux agents de découvrir et d'apprendre des stratégies de défense optimisées \cite{Jamont2015, Theron2020}. Toutefois, les scénarios actuels dans la littérature montrent une lacune dans la manière dont ces agents peuvent s'adapter à des environnements divers et hautement évolutifs, tels que ceux soumis à des malwares intelligents. De même, il y a peu de travaux qui traitent de moyens pour les agents d'apprendre de nouvelles stratégies dans un cadre contraignant arbitrairement défini en plus de l'environnement de déploiement.

\paragraph{Problème 3 : Explicabilité des agents et des systèmes multi-agents.}
Les systèmes multi-agents sont souvent perçus comme des "boîtes noires" en raison de la complexité des décisions qu’ils prennent de manière décentralisée \cite{Theron2018}. Cela soulève des questions cruciales sur la capacité à comprendre et à expliciter les comportements des agents après leur entraînement. Il est essentiel de rendre explicite l'organisation des agents (rôles, missions) pour garantir que leurs décisions ne compromettent pas la sûreté de l'environnement cible.

\paragraph{Problème 4 : Risques liés à l'expérimentation directe en environnement réel.}
Tester directement des agents dans des environnements réels (réseaux d'entreprise, infrastructures critiques) pourrait causer des dommages irréversibles si les agents ne fonctionnent pas correctement \cite{Calo2017}. La nécessité de simuler ces environnements et de reproduire des cyberattaques réalistes devient donc impérative pour éviter les risques liés à l’expérimentation directe.

Ces problèmes identifiés dans la littérature justifient l'exploration de nouvelles solutions pour organiser et entraîner des agents AICA dans des SMA de cyberdéfense adaptatifs.


\section{Hypothèses et contributions proposées}

Pour répondre aux problèmes identifiés, nous proposons les hypothèses suivantes, qui guideront le développement de notre méthode de conception et d'organisation des agents dans un SMA de cyberdéfense.

\paragraph{Hypothèse 1 : Reproduction en simulation des environnements et des attaques.}
Nous postulons qu'en reproduisant fidèlement les environnements cibles et les scénarios d'attaques dans une simulation, nous pouvons entraîner les agents sans risque pour l’environnement réel. Cette méthode permet d'explorer en toute sécurité des stratégies de défense autonomes et d'analyser les réponses des agents face à des cybermenaces complexes.

\paragraph{Hypothèse 2 : Entraînement des agents à travers le MARL pour découvrir des stratégies optimisées.}
Nous faisons l'hypothèse que le MARL est une méthode efficace pour permettre aux agents d’apprendre à se défendre de manière autonome et collaborative contre des attaques sophistiquées. Le MARL permet de surmonter le problème d’entraînement non supervisé en exploitant les interactions entre agents pour générer des stratégies de défense robustes. De plus, il y a un interêt en MARL pour la possibilité de diriger ou contraindre les agents à respecter des contraintes supplémentaires au cours de l'apprentissage, ce qui correspondrait aux exigences de conception.

\paragraph{Hypothèse 3 : Explicabilité des stratégies des agents en termes organisationnels.}
Nous postulons qu'une fois que les agents ont été entraînés, leur comportement peut être expliqué en termes organisationnels issus d'un modèle existant, tels que des rôles ou des missions, ce qui rend les stratégies développées compréhensibles pour les concepteurs humains. Cela répond à la nécessité de garantir que les actions des agents sont sûres et interprétables avant leur déploiement.

\paragraph{Hypothèse 4 : Validation et ajustement des SMA avant déploiement réel.}
Nous postulons que la validation du SMA dans une copie émulée de l’environnement cible permettra de vérifier l'efficacité des agents et d'ajuster leur organisation avant leur déploiement final. Cette validation est cruciale pour minimiser les risques de dysfonctionnement dans un environnement réel.

Ces hypothèses seront testées et validées tout au long de cette thèse, à travers une série d’expérimentations basées sur la simulation d’environnements de cyberdéfense complexes et évolutifs. Les contributions proposées viseront à répondre aux lacunes identifiées dans la littérature et à développer un SMA de cyberdéfense autonome, efficace et explicable.
