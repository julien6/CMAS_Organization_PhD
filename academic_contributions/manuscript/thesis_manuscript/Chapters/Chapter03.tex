%************************************************
\chapter{Problèmes détaillées et hypothèses de travail}\label{ch:problem}
%************************************************

\section{Un contexte de Cyberdéfense avec des défis futurs et nouveaux}

Dans le monde de plus en plus connecté d'aujourd'hui, la complexité et la portée des cybermenaces évoluent à un rythme sans précédent. La transition vers des systèmes décentralisés et distribués, portée par les progrès de l'Internet des objets (IoT), du cloud computing et des réseaux mobiles, a ouvert la voie à de nouveaux défis. de nouvelles vulnérabilités que les mécanismes de défense centralisés traditionnels ne sont pas en mesure de gérer\cite{sun2014data}. Ce nouveau paradigme exige des approches innovantes en matière de Cyberdéfense qui soient à la fois agiles et résilientes face à des attaques diverses et sophistiquées\cite{taddeo2019trusting}.