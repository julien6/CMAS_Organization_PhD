%************************************************
\chapter{Vers un système multi-agents de cyberdéfense}\label{ch:towards_cSMA}
%************************************************

\section{Concepts dans les systèmes multi-agents et l'organisation}

Les systèmes multi-agents (SMA) sont une classe de systèmes distribués composés de plusieurs agents autonomes en interaction. Chaque agent d'un SMA est une entité indépendante qui peut percevoir son environnement, prendre des décisions et agir sur son environnement pour atteindre ses objectifs. objectifs spécifiques \cite{ferber1999multi}. Les agents d'un SMA collaborent, rivalisent ou coexistent généralement pour atteindre des objectifs individuels et collectifs \cite{weiss1999multiagent}.

Le concept de SMA s'inspire de diverses disciplines, notamment l'intelligence artificielle, l'informatique distribuée et la théorie des jeux \cite{shoham2008multiagent}. Les SMA sont utilisés dans une large gamme d'applications, des systèmes robotiques et du trading automatisé à la gestion de réseau et, plus important encore dans le contexte de ce travail, à la cyberdéfense \cite{jennings1998applications, shakarian2015cyber}. La nature décentralisée des SMA les rend particulièrement adaptés aux environnements complexes, dynamiques et distribués où le contrôle centralisé peut être peu pratique ou inefficace \cite{sycara1998multiagent}.

\subsection{Caractéristiques clés des systèmes multi-agents}

Un SMA peut être caractérisé par plusieurs caractéristiques clés qui le distinguent des autres types de systèmes. Ces caractéristiques comprennent l'autonomie, la distribution, la communication, la coordination et l'adaptabilité. Ci-dessous, nous explorons chacune de ces caractéristiques en détail :

\paragraph{Autonomie :}
Chaque agent d'un SMA fonctionne de manière autonome, ce qui signifie qu'il peut prendre des décisions indépendamment des autres agents \cite{jennings1998applications}. Cette autonomie est cruciale dans les environnements où les agents doivent réagir aux conditions locales sans attendre les instructions d'une autorité centrale. L'autonomie permet aux agents de fonctionner efficacement même en l'absence de connaissances globales ou de communication avec d'autres agents \cite{russell2016artificial}.

\paragraph{Distribution :}
Dans un SMA, les agents sont répartis dans l'environnement, à la fois géographiquement et logiquement \cite{ferber1999multi}. Cette distribution permet au système de s'adapter efficacement, car des agents supplémentaires peuvent être déployés pour couvrir de nouvelles zones ou tâches. Elle améliore également la résilience du système, car la défaillance d'un seul agent ne compromet pas nécessairement l'ensemble du système \cite{durfee1999distributed}. Dans le contexte de la cyberdéfense, cette distribution est particulièrement importante pour surveiller les grands réseaux et répondre aux menaces en temps réel \cite{shakarian2015cyber}.

\paragraph{Communication :}
Bien que les agents d'un SMA fonctionnent de manière autonome, ils ont souvent besoin de communiquer entre eux pour atteindre leurs objectifs \cite{huhns1999multiagent}. La communication peut prendre de nombreuses formes, notamment la transmission directe de messages, la communication par diffusion ou même la communication indirecte via l'environnement (par exemple, la stigmergie) \cite{dorigo2000ant}. Des protocoles de communication efficaces sont essentiels à la coordination et à la coopération entre les agents, leur permettant de partager des informations, de négocier et de synchroniser leurs actions.

\paragraph{Coordination :}
La coordination fait référence au processus par lequel les agents alignent leurs actions pour atteindre un objectif commun. Dans un SMA, la coordination peut être obtenue grâce à divers mécanismes, notamment la planification centralisée, la négociation distribuée ou le comportement émergent. La coordination est particulièrement importante dans les environnements où les agents doivent travailler ensemble pour résoudre des problèmes complexes, comme la défense d'un réseau contre une cyberattaque coordonnée.

\paragraph{Adaptabilité :}
L’une des forces des SMA est leur capacité à s’adapter à des environnements changeants. Les agents peuvent apprendre de leurs expériences, ajuster leurs stratégies et faire évoluer leur comportement au fil du temps. Cette adaptabilité est particulièrement précieuse dans les environnements dynamiques, tels que ceux que l’on trouve dans la cybersécurité, où les menaces évoluent constamment et de nouveaux défis surgissent régulièrement. Des techniques telles que l’apprentissage par renforcement, les algorithmes génétiques et le contrôle adaptatif sont souvent utilisées pour améliorer l’adaptabilité des agents dans un SMA.

\subsection{Types de systèmes multi-agents}

Les SMA peuvent être classés en différents types en fonction de la nature des agents et des interactions entre eux. Vous trouverez ci-dessous quelques classifications courantes des SMA :

\paragraph{SMA homogène vs. hétérogène :}
Dans un SMA homogène, tous les agents sont identiques en termes de capacités, d'objectifs et de comportement. Cette simplicité peut faciliter la coordination, car tous les agents suivent les mêmes règles et peuvent se substituer les uns aux autres. En revanche, un SMA hétérogène se compose d'agents ayant des capacités, des rôles et des objectifs différents. Les SMA hétérogènes sont souvent plus puissants et plus flexibles, car ils peuvent exploiter les diverses compétences de différents agents pour résoudre des problèmes complexes. Cependant, la coordination dans un SMA hétérogène peut être plus difficile en raison des intérêts et des capacités variés des agents.

\paragraph{Systèmes d'approvisionnement en eau et de drainage coopératifs ou concurrentiels :}
Dans un SMA coopératif, les agents travaillent ensemble pour atteindre un objectif commun. La coopération est souvent facilitée par des mécanismes de communication et de coordination qui permettent aux agents de partager des informations et de synchroniser leurs actions \cite{falco2020tendencies}. Les SMA coopératifs sont couramment utilisés dans des applications telles que la robotique, où les agents doivent collaborer pour accomplir des tâches telles que l'exploration ou la manipulation d'objets. En revanche, un SMA compétitif implique des agents ayant des objectifs contradictoires. Dans ce contexte, les agents se font concurrence pour les ressources ou l'influence, et des stratégies de théorie des jeux sont souvent utilisées pour modéliser leurs interactions \cite{shoham2008multiagent}. Les SMA compétitifs sont répandus dans des domaines tels que l'économie, la sécurité et les jeux.

\paragraph{SMA statique vs. dynamique :}
Un SMA statique fonctionne dans un environnement où les agents et leurs relations restent constants au fil du temps. Cette stabilité simplifie la conception et l'analyse du système, car les interactions entre les agents peuvent être prédites et optimisées à l'avance. En revanche, un SMA dynamique fonctionne dans un environnement où les agents peuvent entrer, sortir ou modifier leurs rôles et leurs relations au fil du temps. Les SMA dynamiques nécessitent des stratégies plus flexibles et adaptatives, car les agents doivent continuellement ajuster leur comportement pour s'adapter aux changements de l'environnement. La cybersécurité est un excellent exemple d'environnement dynamique, où de nouvelles menaces émergent et les menaces existantes évoluent, obligeant les agents à s'adapter en temps réel.

\subsection{Organisation et coordination au sein du SMA}

\subsubsection{Organisation dans le SMA}

L'organisation des agents au sein d'un SMA est un facteur critique qui influence les performances, l'évolutivité et la résilience du système \cite{wooldridge2009introduction}. Le terme « organisation » fait référence à la structure et aux mécanismes de coordination qui régissent les interactions entre les agents \cite{weiss1999multiagent}. Différentes structures organisationnelles sont appropriées pour différents types de SMA, en fonction de la complexité de l'environnement, de la nature des agents et des objectifs du système \cite{durfee1999distributed}.

\paragraph{Organisation hiérarchique :}
Dans une organisation hiérarchique, les agents sont organisés selon une structure arborescente, les agents de niveau supérieur supervisant les activités des agents de niveau inférieur \cite{wooldridge2009introduction}. Ce type d'organisation est couramment utilisé dans les situations où une autorité centrale est nécessaire pour coordonner les actions des agents subordonnés \cite{durfee1999distributed}. Les organisations hiérarchiques sont relativement faciles à gérer et peuvent être efficaces dans les environnements statiques où les relations entre les agents ne changent pas fréquemment \cite{weiss1999multiagent}. Cependant, elles peuvent être moins efficaces dans les environnements dynamiques, où le contrôle centralisé peut devenir un goulot d'étranglement ou un point de défaillance unique \cite{durfee1999distributed}.

\paragraph{Organisation plate :}
Dans une organisation plate, tous les agents sont considérés comme égaux, sans structure hiérarchique \cite{weiss1999multiagent}. Les agents communiquent et se coordonnent directement entre eux, prenant des décisions basées sur des informations locales \cite{durfee1999distributed}. Les organisations plates sont hautement décentralisées et peuvent être plus résilientes que les structures hiérarchiques, car elles ne s'appuient pas sur une autorité centrale \cite{weiss1999multiagent}. Cependant, la coordination des actions des agents dans une organisation plate peut être plus difficile, en particulier dans les systèmes à grande échelle où la communication et la prise de décision doivent être synchronisées sur l'ensemble du réseau \cite{durfee1999distributed}.

\paragraph{Organisation Holonique :}
L'organisation holonique est une structure hybride qui combine des éléments d'organisations hiérarchiques et plates \cite{durfee1999distributed}. Dans un SMA holonique, les agents sont regroupés en clusters, appelés holons, qui fonctionnent comme des unités autonomes \cite{weiss1999multiagent}. Chaque holon peut être organisé hiérarchiquement en interne, mais les interactions entre les holons sont plus décentralisées \cite{durfee1999distributed}. Cette approche permet une plus grande flexibilité et évolutivité, car les holons peuvent fonctionner indépendamment tout en contribuant aux objectifs généraux du système \cite{weiss1999multiagent}. Les organisations holoniques sont particulièrement utiles dans les environnements dynamiques, où différentes parties du système peuvent avoir besoin de s'adapter à des conditions changeantes à des rythmes différents \cite{durfee1999distributed}.

\subsubsection{Coordination dans SMA}

La coordination est un aspect fondamental des SMA, permettant aux agents d'aligner leurs actions pour atteindre des objectifs communs. Plusieurs mécanismes de coordination ont été développés pour faciliter ce processus, chacun avec ses points forts et ses limites. Ci-dessous, nous discutons de certains des mécanismes de coordination les plus couramment utilisés dans les SMA :

\paragraph{Coordination centralisée :}
Dans la coordination centralisée, un seul agent (ou un petit groupe d'agents) est responsable de la prise de décisions et de l'attribution de tâches à d'autres agents \cite{jennings1998applications}. Cette approche peut être efficace dans les environnements où les connaissances globales sont disponibles et où les décisions doivent être optimisées pour l'ensemble du système. Cependant, la coordination centralisée peut devenir un goulot d'étranglement dans les environnements à grande échelle ou dynamiques, où l'agent central peut avoir du mal à traiter toutes les informations nécessaires en temps opportun \cite{weiss1999multiagent}.

\paragraph{Coordination distribuée :}
La coordination distribuée repose sur une prise de décision décentralisée, où chaque agent prend des décisions en fonction des informations locales et interagit avec d'autres agents pour atteindre les objectifs globaux du système. Cette approche est plus évolutive et résiliente que la coordination centralisée, car elle ne repose pas sur un point de contrôle unique. Cependant, la coordination distribuée peut être plus complexe à mettre en œuvre, car elle nécessite des protocoles de communication et de négociation sophistiqués pour garantir que les agents travaillent ensemble efficacement.

\paragraph{Coordination basée sur la négociation :}
Dans la coordination basée sur la négociation, les agents communiquent entre eux pour parvenir à des accords sur la manière d'allouer les tâches et les ressources. La négociation peut prendre de nombreuses formes, notamment des mécanismes basés sur des enchères, des protocoles de réseau de contrats et des stratégies de négociation. Cette approche est particulièrement utile dans les environnements où les agents ont des intérêts concurrents ou lorsque l'allocation des ressources doit être optimisée de manière dynamique. La coordination basée sur la négociation est souvent utilisée dans les SMA compétitifs, où les agents doivent équilibrer la coopération avec la concurrence.

\paragraph{Coordination émergente :}
La coordination émergente se produit lorsque le comportement du SMA dans son ensemble émerge des interactions entre les agents individuels, sans qu'il soit nécessaire de recourir à des mécanismes de coordination explicites. Cette approche s'appuie sur les principes d'auto-organisation, où les agents suivent des règles locales simples qui conduisent à un comportement global complexe. La coordination émergente est souvent observée dans les systèmes en essaim, où les agents présentent un comportement collectif tel que le regroupement, la recherche de nourriture ou le contrôle de formation. Si la coordination émergente peut être très efficace dans certains environnements, elle peut également être imprévisible et difficile à contrôler, en particulier dans les systèmes où des résultats précis sont requis.

\subsection{Conclusion et défis SMA}

Malgré les nombreux avantages des SMA, plusieurs défis doivent être relevés pour exploiter pleinement leur potentiel en matière de cyberdéfense. Ces défis incluent :

\paragraph{Évolutivité :}
Les SMA étant déployés dans des environnements plus vastes et plus complexes, l'évolutivité devient une préoccupation essentielle. Pour garantir que les agents puissent fonctionner efficacement à grande échelle, il faut des mécanismes efficaces de communication, de coordination et de gestion des ressources. Les recherches futures dans le domaine des SMA devraient se concentrer sur le développement d'architectures évolutives capables de prendre en charge des milliers, voire des millions d'agents, sans dégrader les performances du système.

\paragraph{Sécurité du SMA :}
Bien que les SMA soient utilisés pour améliorer la sécurité, ils sont également vulnérables aux attaques. Les agents compromis peuvent être utilisés pour perturber le système, manipuler des informations ou coordonner des attaques contre d'autres agents. Assurer la sécurité des SMA eux-mêmes est un défi de taille, nécessitant des mécanismes d'authentification, de chiffrement et de confiance robustes. De plus, la nature décentralisée des SMA peut rendre difficile la détection et la réponse aux menaces internes, où les agents compromis se comportent de manière malveillante tout en restant dans les paramètres de fonctionnement normaux du système.

\paragraph{Coordination dans les environnements dynamiques :}
Les environnements dynamiques, tels que ceux rencontrés en cyberdéfense, présentent des défis uniques en matière de coordination dans les SMA. Les agents doivent être capables de s'adapter à des conditions changeantes, telles que de nouvelles menaces, des stratégies d'attaque en constante évolution et des topologies de réseau changeantes. Le développement de mécanismes de coordination à la fois flexibles et efficaces dans des environnements dynamiques est un défi de recherche permanent.

\paragraph{Interaction homme-agent :}
Les SMA étant de plus en plus intégrés aux opérations de cybersécurité, l’interaction entre les opérateurs humains et les agents autonomes devient de plus en plus importante. Il est essentiel de veiller à ce que les agents puissent fournir des explications pertinentes de leurs actions, décisions et recommandations pour instaurer la confiance et permettre une collaboration efficace entre l’homme et l’agent. Les recherches futures devraient explorer les méthodes permettant d’améliorer la transparence, l’interprétabilité et la convivialité des SMA dans les contextes de cyberdéfense.



\section{Un état de l'art en matière de cyberdéfense distribuée ou décentralisée}

En réponse à la complexité et à la sophistication croissantes des cybermenaces, le domaine de la cyberdéfense a connu une évolution des architectures de sécurité centralisées vers des modèles distribués et décentralisés \cite{Chen2021, Misra2023}. Les approches centralisées traditionnelles, qui s'appuient sur un point de contrôle unique pour la prise de décision et l'atténuation des menaces, se sont révélées inadéquates face aux attaques par déni de service distribué (DDoS), aux ransomwares et à d'autres menaces multi-vecteurs qui peuvent cibler simultanément plusieurs parties d'un système \cite{Munsing2018}.

Les systèmes de cyberdéfense distribués et décentralisés, quant à eux, offrent une plus grande résilience en répartissant les capacités défensives sur l'ensemble du réseau \cite{Oliynykov2022}. Cette approche réduit la dépendance à une autorité centrale, minimise les points de défaillance uniques et permet des réponses plus rapides et localisées aux menaces \cite{Kott2021}. Dans cette section, nous explorons l'état actuel de la recherche en cyberdéfense distribuée et décentralisée, en nous concentrant sur les approches, technologies et applications clés.


\subsection{Architectures de cyberdéfense : des architectures de cyberdéfense centralisées aux architectures de cyberdéfense décentralisées}

Les architectures de cyberdéfense centralisées sont le choix traditionnel pour sécuriser les réseaux et les systèmes. Dans ces architectures, un nœud ou un serveur de sécurité central gère tous les aspects de la sécurité, y compris la détection des menaces, la réponse aux incidents et l'application des politiques. Bien que cette approche offre simplicité et facilité de gestion, elle souffre de plusieurs limitations :

\begin{itemize}
    \item \textbf{Point de défaillance unique :} le nœud de sécurité central devient un point de défaillance critique. S'il est compromis, l'ensemble du réseau est en danger \cite{Mell2011}.
    \item \textbf{Problèmes d'évolutivité :} à mesure que les réseaux se développent, les systèmes centralisés ont du mal à gérer le volume croissant de données et d'événements de sécurité \cite{Xu2019}.
    \item \textbf{Latence et temps de réponse :} les systèmes centralisés peuvent introduire une latence dans la détection et la réponse aux menaces, en particulier dans les réseaux géographiquement dispersés \cite{Benet2014}.
\end{itemize}

En revanche, les architectures décentralisées répartissent les fonctions de sécurité sur plusieurs nœuds du réseau. Chaque nœud est chargé de surveiller son propre environnement, de détecter les menaces et d'agir de manière indépendante ou en collaboration avec d'autres nœuds. La cyberdéfense décentralisée offre plusieurs avantages :

\begin{itemize}
    \item \textbf{Résilience améliorée :} L'absence d'un point de défaillance unique améliore la résilience globale du système \cite{Zohar2015}.
    \item \textbf{Évolutivité :} les systèmes décentralisés peuvent évoluer plus efficacement à mesure que de nouveaux nœuds sont ajoutés au réseau \cite{Xu2019}.
    \item \textbf{Temps de réponse plus rapides :} la détection et la réponse localisées aux menaces permettent une atténuation plus rapide des attaques \cite{Benet2014}.
\end{itemize}


% \subsection{Approches clés en cyberdéfense distribuée et décentralisée}

% Le développement de stratégies de cyberdéfense distribuées et décentralisées a conduit à l'émergence de plusieurs approches clés, notamment les systèmes collaboratifs de détection d'intrusion, la sécurité basée sur la blockchain, les réseaux peer-to-peer et l'utilisation de systèmes multi-agents (SMA). Ci-dessous, nous explorons ces approches plus en détail.

% \subsubsection{Systèmes collaboratifs de détection d'intrusion (CIDS)}

% Les systèmes collaboratifs de détection d'intrusion (CIDS) représentent une approche décentralisée de la sécurité réseau, où plusieurs systèmes de détection d'intrusion (IDS) fonctionnent ensemble pour détecter et atténuer les menaces \cite{Zhou2010}. Dans un CIDS, chaque IDS surveille un segment spécifique du réseau et partage des informations avec d'autres nœuds IDS \cite{Vasilomanolakis2015}. Cette collaboration permet au système d'identifier les attaques distribuées qui peuvent ne pas être détectables par un seul IDS fonctionnant de manière isolée.

% \subsubsection{Sécurité basée sur la blockchain}

% La technologie Blockchain, initialement développée pour la cryptomonnaie, a suscité une attention considérable en tant que solution potentielle pour la cyberdéfense décentralisée \cite{Kshetri2017}. La nature décentralisée et immuable de la blockchain en fait une option intéressante pour sécuriser les systèmes distribués et garantir l'intégrité des transactions, des données et des communications \cite{Taylor2020}.

% \subsubsection{Réseaux peer-to-peer (P2P)}

% Les réseaux peer-to-peer (P2P) sont intrinsèquement décentralisés, chaque nœud du réseau agissant à la fois comme client et comme serveur \cite{Wallach2003}. Les réseaux P2P ont été utilisés dans diverses applications de cyberdéfense, en particulier dans le partage de fichiers distribués, la distribution de contenu et la communication sécurisée \cite{Seedorf2009}.

% \subsubsection{Systèmes multi-agents (SMA)}

% Les systèmes multi-agents (SMA) sont apparus comme une approche puissante de la cyberdéfense distribuée. Dans un SMA, plusieurs agents autonomes travaillent ensemble pour surveiller, détecter et répondre aux menaces de manière décentralisée \cite{Herrero2009}. Chaque agent fonctionne de manière indépendante, mais peut collaborer avec d'autres agents pour atteindre un objectif commun, comme la protection d'un réseau contre les cyberattaques \cite{Jahanbin2013}.


\subsection{Technologies permettant la cyberdéfense distribuée}

Plusieurs technologies ont joué un rôle clé dans la mise en place de systèmes de cyberdéfense distribués et décentralisés. Ces technologies fournissent l'infrastructure et les outils de base nécessaires pour prendre en charge les architectures décentralisées et permettre une communication, une coordination et une prise de décision efficaces entre les nœuds distribués. Ci-dessous, nous mettons en évidence certaines des technologies habilitantes les plus importantes.

\paragraph{Informatique de pointe}

L'informatique de pointe rapproche le calcul et le stockage des données de l'endroit où ils sont nécessaires, réduisant ainsi la latence et l'utilisation de la bande passante \cite{Shi2016}. Dans un système de cyberdéfense décentralisé, l'informatique de pointe permet de distribuer les fonctions de sécurité sur l'ensemble du réseau, chaque nœud de pointe étant responsable de la surveillance et de la défense de son environnement local \cite{Roman2018}.

\paragraph{Informatique de brouillard}

Le fog computing étend le concept de edge computing en créant une architecture hiérarchique qui inclut des périphériques de périphérie, des nœuds de fog et des serveurs cloud \cite{Bonomi2012}. Dans un environnement de fog computing, les nœuds de fog agissent comme des intermédiaires entre les périphériques de périphérie et le cloud, fournissant une puissance de traitement et un stockage supplémentaires plus proches de la source de données \cite{Mukherjee2017}.

\paragraph{Réseau défini par logiciel (SDN)}

Le réseau défini par logiciel (SDN) est une approche architecturale qui sépare le plan de contrôle du plan de données dans un réseau. Cette séparation permet aux administrateurs réseau de gérer et d'optimiser le réseau par programmation, le rendant ainsi plus flexible et réactif aux conditions changeantes. \cite{Kreutz2015}

\paragraph{Intelligence artificielle et apprentissage automatique}

L'intelligence artificielle (IA) et l'apprentissage automatique (ML) font désormais partie intégrante des systèmes de cyberdéfense modernes, en particulier dans les environnements distribués et décentralisés \cite{Buczak2016}. L'IA et le ML peuvent être utilisés pour analyser de vastes quantités de données, identifier des modèles et détecter des anomalies pouvant indiquer une cyberattaque \cite{Vinayakumar2019}.


\subsection{Applications du SMA en cyberdéfense}

Les systèmes multi-agents ont été largement utilisés dans le domaine de la cybersécurité, où leur nature décentralisée et leur adaptabilité les rendent particulièrement adaptés à la défense contre des menaces complexes et évolutives. Ci-dessous, nous mettons en évidence certaines applications clés des SMA en cyberdéfense :

\paragraph{Détection et réponse aux intrusions :}
Les SMA sont souvent utilisés dans les systèmes de détection d'intrusion (IDS) pour surveiller le trafic réseau et détecter les comportements anormaux pouvant indiquer une cyberattaque. Les agents de l'IDS peuvent être déployés à différents points du réseau, chacun étant chargé de surveiller un segment spécifique du trafic. Lorsqu'une intrusion est détectée, les agents peuvent collaborer pour identifier la source de l'attaque et mettre en œuvre des contre-mesures appropriées. La nature décentralisée des SMA permet une détection d'intrusion plus robuste et évolutive, car chaque agent peut fonctionner de manière indépendante tout en contribuant à la sécurité globale du réseau.

\paragraph{Détection et atténuation des logiciels malveillants :}
SMA peut également être utilisé pour détecter et atténuer les infections par des logiciels malveillants dans les systèmes distribués. Les agents peuvent surveiller le comportement des appareils et des applications, à la recherche de signes d'activité de logiciels malveillants tels qu'une utilisation inhabituelle des ressources, des tentatives d'accès non autorisées ou des communications réseau suspectes. Lorsqu'un logiciel malveillant est détecté, les agents peuvent prendre des mesures pour isoler l'appareil infecté, supprimer le logiciel malveillant ou bloquer toute communication ultérieure de la source malveillante. La nature distribuée de SMA garantit que la détection et l'atténuation des logiciels malveillants peuvent se produire sur l'ensemble du réseau, même en présence de menaces sophistiquées et furtives.

\paragraph{Protection contre les dénis de service distribués (DDoS) :}
Les attaques DDoS, où plusieurs appareils compromis sont utilisés pour submerger une cible de trafic, constituent une menace importante pour les réseaux modernes. Le SMA peut être utilisé pour se protéger contre les attaques DDoS en répartissant des agents défensifs sur le réseau. Ces agents peuvent détecter le début d'une attaque DDoS en surveillant les schéSMA de trafic et en identifiant les pics anormaux de trafic. Une fois qu'une attaque est détectée, les agents peuvent collaborer pour atténuer l'impact en bloquant ou en redirigeant le trafic malveillant, en ajustant les allocations de ressources et en se coordonnant avec d'autres agents pour garantir la disponibilité continue des services critiques.

\paragraph{Sécurité des systèmes cyberphysiques :}
Les systèmes cyberphysiques (CPS), tels que les réseaux intelligents, les systèmes de contrôle industriel et les véhicules autonomes, sont de plus en plus ciblés par les cyberattaques. Les SMA peuvent jouer un rôle essentiel dans la sécurisation de ces systèmes en fournissant des capacités de surveillance, de détection des menaces et de réponse en temps réel. Dans un environnement CPS, des agents peuvent être déployés à différents points du système pour surveiller à la fois les aspects physiques et cybernétiques du fonctionnement du système. Lorsqu'une menace est détectée, les agents peuvent coordonner leurs réponses pour garantir que les composants physiques et cybernétiques du système restent sécurisés et opérationnels.


% \subsection{Études de cas et applications}

Pour mieux comprendre l'état de l'art en matière de cyberdéfense distribuée et décentralisée, il est utile d'examiner des études de cas et des applications concrètes. Ci-dessous, nous mettons en évidence plusieurs exemples notables de systèmes de cyberdéfense distribués en pratique.

% \subsubsection{Étude de cas 1 : atténuation des botnets à l'aide de la détection collaborative des intrusions}

% Les botnets constituent une menace majeure pour les réseaux modernes, car ils peuvent être utilisés pour lancer des attaques DDoS à grande échelle, voler des données et diffuser des logiciels malveillants. Un système collaboratif de détection d'intrusion (CIDS) a été déployé dans un grand réseau d'entreprise pour atténuer l'activité des botnets \cite{Vasilomanolakis2015}.

% \subsubsection{Étude de cas 2 : Sécurité basée sur la blockchain pour les réseaux IoT}

% Un réseau IoT décentralisé a été sécurisé à l'aide de la technologie blockchain pour authentifier les appareils et conserver un enregistrement immuable des événements de sécurité \cite{Dorri2017}.

% \subsubsection{Étude de cas 3 : Protection contre les attaques par déni de service distribué (DDoS) à l'aide de systèmes multi-agents}

% Un grand fournisseur de services cloud a déployé un système multi-agent (SMA) pour se protéger contre les attaques DDoS \cite{Kotenko2007}.

\subsection{Conclusion : Défis et orientations futures de la cyberdéfense distribuée et décentralisée}

\subsubsection{Défis de la cyberdéfense distribuée et décentralisée}

Malgré les avantages de la cyberdéfense distribuée et décentralisée, plusieurs défis doivent être relevés pour exploiter pleinement le potentiel de ces systèmes. Ces défis comprennent :

\paragraph{Coordination et communication :}
Dans un système décentralisé, la coordination et la communication entre les nœuds sont essentielles pour assurer une défense efficace. Cependant, la gestion des communications dans un réseau vaste et dynamique peut s'avérer difficile, en particulier lorsque les nœuds sont dispersés géographiquement ou fonctionnent dans des environnements à bande passante limitée.

\paragraph{Gestion de la confiance :}
Dans les systèmes décentralisés, établir la confiance entre les nœuds est un défi de taille. Il est essentiel pour l'efficacité du système de s'assurer que les nœuds peuvent se faire confiance pour partager des informations précises et fiables.

\paragraph{Évolutivité :}
À mesure que les systèmes de cyberdéfense décentralisés augmentent en taille, l'évolutivité devient une préoccupation majeure \cite{Bera2017}.

\paragraph{Sécurité du système lui-même :}
Les systèmes décentralisés ne sont pas à l’abri des attaques, et assurer la sécurité du système lui-même est un défi crucial \cite{Roman2013}.

\subsubsection{Orientations futures}

Les systèmes de cyberdéfense distribués et décentralisés représentent une avancée significative dans le domaine de la cybersécurité \cite{Christidis2016}. En répartissant les fonctions de sécurité sur le réseau et en permettant une détection et une réponse localisées aux menaces, ces systèmes offrent une plus grande résilience, une plus grande évolutivité et une plus grande adaptabilité face aux cybermenaces modernes \cite{Roman2013}. Cependant, des défis tels que la coordination, la gestion de la confiance et l'évolutivité doivent être relevés pour exploiter pleinement le potentiel de la cyberdéfense décentralisée \cite{Xu2019}.
Certains des domaines clés pour la recherche et le développement futurs comprennent :

\begin{itemize}
    \item \textbf{Cyberdéfense pilotée par l'IA :} L'intégration de l'intelligence artificielle et de l'apprentissage automatique dans les systèmes de cyberdéfense décentralisés continuera d'être un domaine d'intérêt majeur \cite{Kaur2023}.

    \item \textbf{Sécurité résistante aux attaques quantiques :} à mesure que l'informatique quantique progresse, il devient de plus en plus nécessaire de développer des systèmes de cyberdéfense décentralisés qui résistent aux attaques quantiques \cite{Bernstein2017}.

    \item \textbf{Protocoles de communication sécurisés :} Assurer une communication sécurisée entre les nœuds dans les systèmes décentralisés restera une priorité \cite{Granjal2015}.

    \item \textbf{Prise de décision autonome :} Améliorer l'autonomie des systèmes de cyberdéfense décentralisés sera essentiel pour améliorer leur capacité à répondre aux menaces émergentes \cite{Nguyen2019}.
\end{itemize}



\section{Synthèse et identification des lacunes de recherche}

Le domaine de la cyberdéfense décentralisée et distribuée a connu des avancées significatives au cours de la dernière décennie. Avec la complexité croissante des cybermenaces et les limites des mécanismes de défense centralisés traditionnels, les chercheurs se sont tournés vers des solutions plus résilientes et évolutives. Dans cette section, nous synthétisons les connaissances actuelles sur les approches, technologies et méthodologies clés de la cyberdéfense décentralisée, en nous concentrant sur les systèmes multi-agents (SMA), les systèmes collaboratifs de détection d'intrusion (CIDS), la sécurité basée sur la blockchain et d'autres technologies habilitantes.

\subsubsection{Systèmes multi-agents (SMA) en cyberdéfense}

Les systèmes multi-agents (SMA) sont apparus comme l'une des approches les plus prometteuses pour la cyberdéfense distribuée. Comme indiqué dans les sections précédentes, les SMA exploitent l'autonomie et la collaboration de plusieurs agents pour surveiller, détecter et répondre aux cybermenaces de manière décentralisée. Les principaux avantages des SMA sont les suivants :

\begin{itemize}
    \item \textbf{Évolutivité :} SMA peut évoluer efficacement en ajoutant davantage d'agents au réseau selon les besoins, sans avoir besoin d'une autorité centrale pour gérer l'ensemble du système.
    \item \textbf{Résilience :} La nature décentralisée du SMA réduit le risque d'un point de défaillance unique, améliorant ainsi la résilience globale du système.
    \item \textbf{Adaptabilité :} les SMA sont très adaptables, capables d’apprendre de leur environnement et d’ajuster leurs stratégies pour contrer les menaces nouvelles et émergentes.
\end{itemize}

Malgré ces avantages, les SMA sont confrontés à des défis liés à la coordination, à la communication et à la gestion de la confiance entre les agents. Les recherches actuelles se sont concentrées sur l'amélioration de l'efficacité des mécanismes de coordination, le développement de protocoles de communication robustes et la résolution des problèmes de confiance et de sécurité au sein des SMA. Cependant, plusieurs questions restent ouvertes concernant l'optimisation de la collaboration des agents dans des environnements hautement dynamiques et le développement d'algorithmes d'apprentissage plus sophistiqués pour améliorer l'autonomie des agents.

\subsubsection{Systèmes collaboratifs de détection d'intrusion (CIDS)}

Les systèmes collaboratifs de détection d'intrusion (CIDS) représentent une approche décentralisée de la détection des menaces, où plusieurs nœuds IDS travaillent ensemble pour surveiller le trafic réseau et détecter les anomalies. La principale force des CIDS réside dans leur capacité à regrouper des informations provenant de différentes parties du réseau, ce qui permet de détecter des attaques distribuées qui peuvent ne pas être visibles par un seul nœud IDS.

Les avancées récentes dans le domaine des CIDS se sont concentrées sur l'amélioration de la précision de la détection des menaces grâce à l'utilisation de l'apprentissage automatique et de techniques de détection des anomalies basées sur l'IA. En outre, les recherches ont exploré l'utilisation de topologies hiérarchiques et peer-to-peer (P2P) pour améliorer l'évolutivité et la résilience des CIDS.

Cependant, il reste encore des défis à relever pour garantir le partage rapide et précis des informations entre les nœuds CIDS, en particulier dans les réseaux de grande taille et dynamiques. La gestion de la confiance est également un enjeu crucial, car l'efficacité d'un CIDS dépend de la fiabilité des informations partagées entre les nœuds. Pour relever ces défis, il faut poursuivre les recherches sur les modèles de confiance, les protocoles de communication sécurisés et l'intégration de techniques d'IA avancées.

\subsubsection{Sécurité basée sur la blockchain}

La technologie blockchain a suscité un intérêt croissant en tant que solution potentielle pour sécuriser les systèmes décentralisés, notamment dans les réseaux IoT et la sécurité de la chaîne d'approvisionnement. La nature décentralisée et immuable de la blockchain en fait une option intéressante pour garantir l'intégrité des données et des transactions dans des environnements distribués.

Des recherches récentes ont exploré l’utilisation de la blockchain pour sécuriser les communications entre les appareils, authentifier les utilisateurs et conserver une piste d’audit des événements de sécurité. L’un des principaux avantages de la blockchain est sa résistance à la falsification, car toute tentative de modification des données nécessite le consensus de la majorité des nœuds du réseau.

Cependant, la sécurité basée sur la blockchain est confrontée à des défis liés à l'évolutivité, à la consommation d'énergie et à la latence. À mesure que les réseaux blockchain se développent, le mécanisme de consensus peut devenir un goulot d'étranglement, ralentissant les délais de traitement des transactions. En outre, la nature énergivore de l'exploitation minière de blockchain constitue un obstacle à une adoption généralisée, en particulier dans les environnements aux ressources limitées tels que les réseaux IoT. Les recherches en cours se concentrent sur le développement d'algorithmes de consensus plus efficaces et sur l'exploration d'architectures de blockchain alternatives, telles que les blockchains autorisées, pour relever ces défis.

\subsubsection{Edge et Fog Computing en cyberdéfense}

L'edge computing et le fog computing sont devenus des technologies clés pour permettre la cyberdéfense distribuée, en particulier dans les réseaux IoT et autres environnements décentralisés. En rapprochant le calcul et le stockage des données de l'endroit où ils sont nécessaires, l'edge computing et le fog computing réduisent la latence et l'utilisation de la bande passante, permettant ainsi la détection et la réponse aux menaces en temps réel.

L'informatique de périphérie permet de répartir les fonctions de sécurité sur l'ensemble du réseau, chaque nœud périphérique étant responsable de la surveillance et de la défense de son environnement local. Le fog computing étend ce concept en créant une architecture hiérarchique qui comprend des périphériques périphériques, des nœuds de fog et des serveurs cloud. Les nœuds de fog agissent comme des intermédiaires, fournissant une puissance de traitement et un stockage supplémentaires plus proches de la source de données.

Les recherches récentes sur le edge computing et le fog computing se sont concentrées sur l'optimisation du déploiement des fonctions de sécurité, l'amélioration de l'efficacité du traitement des données et le renforcement de la résilience du système. Cependant, des défis subsistent pour gérer la complexité des réseaux edge et fog à grande échelle, assurer une communication sécurisée entre les nœuds et développer des mécanismes de sécurité adaptatifs capables de répondre aux menaces émergentes en temps réel.

\subsection{Les lacunes vers un SMA de cyberdéfense}

Bien que des progrès significatifs aient été réalisés dans le développement de systèmes de cyberdéfense distribués et décentralisés, plusieurs lacunes subsistent dans la recherche. Ces lacunes représentent des domaines critiques où des recherches plus approfondies sont nécessaires pour exploiter pleinement le potentiel des architectures décentralisées en matière de cybersécurité. Ci-dessous, nous identifions et analysons les principales lacunes de la recherche dans ce domaine.

% \subsubsection{Lacune 1 : Confiance et sécurité dans les systèmes décentralisés}

% L'un des défis les plus importants de la cyberdéfense décentralisée est de garantir la confiance et la sécurité entre les nœuds distribués. Contrairement aux systèmes centralisés, où une seule autorité peut appliquer les politiques de sécurité et vérifier l'intégrité du système, les systèmes décentralisés s'appuient sur des nœuds individuels pour prendre leurs propres décisions de sécurité.

% Cette décentralisation introduit plusieurs défis liés à la gestion de la confiance, en particulier dans les environnements ouverts et dynamiques où les nœuds peuvent ne pas avoir de relations de confiance préétablies. Les modèles de confiance existants, tels que les systèmes basés sur la réputation, sont vulnérables aux attaques telles que la collusion, les faux rapports et les attaques Sybil, où un attaquant crée plusieurs fausses identités pour manipuler le système.

% Des recherches sont nécessaires pour développer des cadres de gestion de la confiance plus robustes, capables de fonctionner efficacement dans des environnements décentralisés. Cela comprend l'exploration de nouveaux modèles de confiance, le développement de protocoles de communication sécurisés et la mise en œuvre de mécanismes de détection et d'atténuation des menaces internes.

\subsubsection{Lacune 1 : Coordination et communication entre les nœuds distribués}

Une coordination et une communication efficaces sont essentielles au succès des systèmes de cyberdéfense décentralisés. Cependant, la gestion des communications dans un réseau vaste et dynamique peut s'avérer difficile, en particulier lorsque les nœuds sont dispersés géographiquement ou fonctionnent dans des environnements à bande passante limitée.

Des recherches sont nécessaires pour développer des protocoles de communication plus efficaces, capables de fonctionner dans des environnements aux ressources limitées et de garantir le partage rapide et précis des informations entre les nœuds. De plus, de nouveaux mécanismes de coordination sont nécessaires pour optimiser la collaboration entre les nœuds et améliorer l'efficacité globale du système.

\subsubsection{Lacune 2 : Adaptabilité et apprentissage dans des environnements dynamiques}

L’un des principaux avantages des systèmes de cyberdéfense décentralisés est leur capacité à s’adapter à des environnements changeants. Cependant, les systèmes actuels ont souvent du mal à répondre efficacement aux menaces nouvelles et émergentes, en particulier dans des environnements très dynamiques où les stratégies d’attaque évoluent constamment.

Des recherches sont nécessaires pour développer des algorithmes d’apprentissage plus sophistiqués qui peuvent améliorer l’adaptabilité des systèmes décentralisés. Cela comprend l’exploration de nouvelles approches de l’apprentissage automatique, de l’apprentissage par renforcement et de l’IA, ainsi que le développement de mécanismes d’apprentissage continu et d’adaptation en temps réel.


\subsubsection{Lacune 3 : Intégration des technologies émergentes}

Les technologies émergentes, telles que l’informatique quantique, l’IA et la blockchain, ont le potentiel de transformer les systèmes de cyberdéfense décentralisés. Cependant, l’intégration de ces technologies dans les architectures existantes présente plusieurs défis.

Par exemple, l’informatique quantique offre la possibilité d’améliorer considérablement la sécurité des systèmes décentralisés, mais elle introduit également de nouvelles vulnérabilités qu’il convient de corriger. De même, l’IA et l’apprentissage automatique peuvent améliorer la détection et la réponse aux menaces, mais ils introduisent également des défis liés à la transparence, à l’explicabilité et à la confiance.

Des recherches sont nécessaires pour étudier l’intégration des technologies émergentes dans les systèmes de cyberdéfense décentralisés et pour relever les nouveaux défis qu’elles posent. Il s’agit notamment de développer de nouveaux cadres d’évaluation de l’impact de ces technologies sur la sécurité et la résilience, ainsi que d’explorer de nouvelles possibilités d’exploiter ces technologies pour améliorer l’efficacité des architectures décentralisées.


\subsubsection{Lacune 4 : Évolutivité des architectures de cyberdéfense distribuées}

L'évolutivité est une préoccupation majeure dans les systèmes de cyberdéfense décentralisés, en particulier lorsque les réseaux deviennent plus grands et plus complexes. Garantir que le système puisse gérer un nombre croissant de nœuds, d'appareils et d'événements de sécurité sans dégrader les performances constitue un défi de taille.

Les recherches actuelles ont exploré diverses approches pour améliorer l'évolutivité des systèmes décentralisés, notamment l'utilisation d'architectures hiérarchiques, de blockchain et de bases de données distribuées. Cependant, ces solutions introduisent souvent de nouveaux défis, tels qu'une consommation accrue de ressources, une latence et une complexité accrues.

Des recherches supplémentaires sont nécessaires pour développer des architectures évolutives capables de prendre en charge des systèmes de cyberdéfense décentralisés à grande échelle. Cela comprend l'exploration de nouvelles technologies, telles que l'informatique quantique, qui peuvent offrir de nouvelles opportunités d'amélioration de l'évolutivité, ainsi que l'optimisation des architectures existantes pour réduire la consommation de ressources et améliorer les performances.



% \sous-section{Conclusion}

% La synthèse des connaissances actuelles en matière de cyberdéfense distribuée et décentralisée met en évidence les progrès significatifs réalisés ces dernières années. Les systèmes multi-agents, les systèmes collaboratifs de détection d'intrusion, la blockchain et l'edge computing ont tous contribué au développement d'architectures de cyberdéfense plus résilientes et évolutives. Cependant, plusieurs lacunes de recherche subsistent, notamment dans les domaines de la gestion de la confiance, de l'évolutivité, de la coordination, de l'adaptabilité et de l'intégration des technologies émergentes.

Pour combler ces lacunes, il faudra poursuivre les recherches et l’innovation dans ce domaine. En explorant de nouvelles approches en matière de confiance et de sécurité, en améliorant l’évolutivité des systèmes décentralisés et en développant des algorithmes d’apprentissage plus sophistiqués, les chercheurs peuvent contribuer à garantir que les architectures de cyberdéfense décentralisées sont capables de relever les défis de la cybersécurité moderne. En outre, l’intégration des technologies émergentes offre de nouvelles opportunités pour améliorer l’efficacité et la résilience des systèmes décentralisés, ce qui en fait un domaine essentiel pour les recherches futures.



% \section{Questions et objectifs de recherche}

% La synthèse des connaissances actuelles en matière de cyberdéfense distribuée et décentralisée, ainsi que l’identification des lacunes de la recherche, ont mis en évidence plusieurs défis cruciaux qui doivent être relevés pour faire progresser le domaine. Ces défis tournent autour de la gestion de la confiance, de l’évolutivité, de la coordination, de l’adaptabilité et de l’intégration des technologies émergentes dans les architectures décentralisées. Pour relever ces défis, les questions de recherche suivantes ont été formulées. Ces questions guideront l’orientation de cette recherche et façonneront le développement de nouvelles méthodologies, de nouveaux modèles et de nouveaux systèmes pour la cyberdéfense distribuée et décentralisée.

% \

% \subsection{Question de recherche 1 : Comment la confiance peut-elle être gérée efficacement dans les systèmes de cyberdéfense décentralisés ?}

% L’un des défis les plus importants de la cyberdéfense décentralisée est de garantir la confiance entre les nœuds distribués. Contrairement aux systèmes centralisés, où une seule autorité peut appliquer les politiques de sécurité et vérifier l’intégrité du système, les systèmes décentralisés s’appuient sur des nœuds individuels pour prendre des décisions de sécurité. Cette décentralisation introduit des vulnérabilités, telles que la possibilité que des nœuds malveillants perturbent le système, les attaques de collusion et les faux rapports. Pourtant, nous n’envisageons pas de relever ce défi directement dans cette thèse. En effet, nous nous concentrons sur le développement d’un SMA de cyberdéfense, qui implique intrinsèquement la gestion de la confiance dès son processus de conception.

% \textbf{Question de recherche 1 (Q1) :} \textit{Comment la confiance peut-elle être gérée efficacement dans les systèmes de cyberdéfense décentralisés pour garantir une communication sécurisée et fiable entre les nœuds, tout en atténuant les risques posés par les acteurs malveillants et en garantissant l'intégrité globale du système ?}

% Cette question vise à explorer de nouveaux modèles et cadres de gestion de la confiance dans les environnements décentralisés. Elle vise à relever les défis de l'établissement, du maintien et du rétablissement de la confiance dans des environnements dynamiques et potentiellement conflictuels. L'accent sera mis sur le développement de mécanismes permettant aux nœuds d'évaluer la fiabilité de leurs pairs et de prendre des décisions éclairées sur la collaboration et le partage d'informations.


% \subsection{Question de recherche 1 : Quels mécanismes de coordination sont les plus efficaces dans la cyberdéfense distribuée ?}

% La coordination est essentielle au succès des systèmes de cyberdéfense décentralisés, en particulier dans les environnements dynamiques où les menaces peuvent émerger à plusieurs endroits simultanément. Cependant, la gestion de la coordination dans un réseau étendu et distribué peut s'avérer difficile, en particulier lorsque les nœuds sont dispersés géographiquement ou fonctionnent dans des environnements à bande passante limitée.

% \textbf{Question de recherche 1 (Q1) :} \textit{Quels mécanismes de coordination et protocoles de communication sont les plus efficaces dans les systèmes de cyberdéfense distribués, et comment peuvent-ils être optimisés pour garantir des réponses rapides et précises aux menaces émergentes dans des environnements dynamiques et aux ressources limitées ?}

% Cette question porte sur l'identification et le développement de stratégies de coordination qui améliorent l'efficacité des systèmes de cyberdéfense décentralisés. La recherche explorera à la fois les mécanismes de coordination centralisés et décentralisés, ainsi que les approches hybrides qui équilibrent les compromis entre flexibilité et efficacité. Une attention particulière sera accordée aux défis de la coordination dans les environnements aux ressources limitées, tels que les réseaux IoT et les réseaux mobiles ad hoc (MANET).


% \subsection{Question de recherche 2 : Comment les systèmes de cyberdéfense décentralisés peuvent-ils être rendus plus adaptatifs aux menaces émergentes ?}

% L’un des principaux avantages des systèmes de cyberdéfense décentralisés est leur potentiel d’adaptation dans des environnements dynamiques. Cependant, les systèmes actuels ont souvent du mal à répondre efficacement aux menaces nouvelles et émergentes, en particulier lorsque les stratégies d’attaque évoluent et deviennent plus sophistiquées. Les systèmes décentralisés doivent apprendre et adapter en permanence leurs stratégies de défense pour suivre le rythme de l’évolution des menaces.

% \textbf{Question de recherche 2 (Q2) :} \textit{Comment les systèmes de cyberdéfense décentralisés peuvent-ils être rendus plus adaptables aux menaces émergentes, et quelles techniques d'apprentissage automatique et d'IA peuvent être exploitées pour améliorer la capacité du système à apprendre de son environnement et à faire évoluer ses stratégies de défense au fil du temps ?}

% Cette question vise à explorer de nouvelles approches de l'adaptabilité dans les systèmes décentralisés, en se concentrant sur l'intégration de l'apprentissage automatique, de l'apprentissage par renforcement et des techniques d'IA. La recherche examinera comment ces technologies peuvent être appliquées pour améliorer la capacité du système à détecter et à répondre à de nouveaux modèles d'attaque, tout en améliorant continuellement ses performances au fil du temps.


% \subsection{Question de recherche 3 : Comment les technologies émergentes peuvent-elles être intégrées dans des architectures de cyberdéfense décentralisées ?}

% Les technologies émergentes telles que l’informatique quantique, la blockchain et l’IA avancée ont le potentiel de transformer les systèmes de cyberdéfense décentralisés. Cependant, l’intégration de ces technologies dans les architectures existantes présente des défis importants, notamment en termes de sécurité, d’évolutivité et de performances.

% \textbf{Question de recherche 3 (Q3) :} \textit{Quelles sont les méthodes les plus efficaces pour intégrer les technologies émergentes, telles que l'informatique quantique et la blockchain, dans les architectures de cyberdéfense décentralisées, et comment ces technologies peuvent-elles être exploitées pour améliorer la sécurité et la résilience du système ?}

% Cette question vise à explorer le potentiel des technologies émergentes pour répondre aux défis auxquels sont confrontés les systèmes de cyberdéfense décentralisés. La recherche se concentrera sur l'identification des opportunités et des limites de ces technologies, ainsi que sur le développement de cadres pour leur intégration dans les architectures existantes et futures.



% \subsection{Question de recherche 2 : Comment les systèmes de cyberdéfense décentralisés peuvent-ils être mis à l'échelle efficacement ?}

% L'évolutivité est une préoccupation majeure dans les systèmes de cyberdéfense décentralisés, en particulier lorsque les réseaux deviennent plus grands et plus complexes. Garantir que le système puisse gérer un nombre croissant de nœuds, d'appareils et d'événements de sécurité sans dégrader les performances constitue un défi de taille. Les recherches actuelles ont exploré diverses approches, telles que les architectures hiérarchiques et les solutions basées sur la blockchain, mais celles-ci introduisent souvent de nouveaux problèmes liés à la consommation de ressources et à la latence.

% \textbf{Question de recherche 4 (Q4) :} \textit{Quelles innovations architecturales et algorithmiques sont nécessaires pour garantir l'évolutivité efficace des systèmes de cyberdéfense décentralisés, tout en maintenant les performances, la sécurité et l'efficacité des ressources dans les réseaux à grande échelle ?}

% Cette question vise à étudier de nouvelles approches pour la mise à l'échelle de systèmes décentralisés, en mettant l'accent sur l'optimisation de la communication, de la coordination et de l'allocation des ressources dans des environnements à grande échelle. La recherche explorera à la fois les aspects techniques et théoriques de l'évolutivité, dans le but de développer des solutions pouvant être appliquées à diverses architectures décentralisées.

% \

% Les questions de recherche présentées dans cette section visent à répondre aux principaux défis identifiés dans le domaine de la cyberdéfense distribuée et décentralisée. En explorant de nouvelles approches de gestion de la confiance, d'évolutivité, de coordination, d'adaptabilité et d'intégration des technologies émergentes, cette recherche vise à contribuer au développement de systèmes de cyberdéfense plus résilients et plus efficaces. Les réponses à ces questions constitueront la base des chapitres suivants, guidant à la fois les contributions théoriques et pratiques de cette thèse.

\section{Positionnement et apport de cette thèse}

Cette thèse se situe à l'intersection de la cyberdéfense, des systèmes multi-agents et de la prise de décision autonome via MARL. Bien que les recherches existantes aient exploré divers aspects des SMA en cyberdéfense, des lacunes importantes subsistent, notamment dans les domaines de la coordination, de l'adaptabilité et de l'évolutivité au sein des systèmes décentralisés. Les travaux antérieurs se sont principalement concentrés sur des modèles centralisés ou des applications spécifiques des SMA dans des contextes limités, tels que la détection d'intrusions ou l'atténuation des logiciels malveillants. Cependant, il manque des cadres complets qui abordent systématiquement la conception, la mise en œuvre et l'évaluation des SMA dans des environnements de cyberdéfense distribués.

\

La principale contribution de cette thèse est le développement d'un cadre global pour la conception et la mise en œuvre de systèmes multi-agents de cyberdéfense (SMA). Cette contribution se concrétise par les innovations clés suivantes, chacune d'entre elles répondant à des questions de recherche spécifiques posées dans cette thèse :

\begin{enumerate}
    \item \textbf{Cadre de modélisation formelle (CybSMAFM) :} La thèse présente un modèle formel qui intègre la théorie organisationnelle au sein du SMA en mettant l'accent sur la cyberdéfense. Ce modèle est basé sur le cadre du processus de décision de Markov partiellement observable décentralisé (Dec-POMDP), permettant l'exploration systématique des structures organisationnelles et des processus de prise de décision face à l'incertitude.

          \begin{itemize}
              \item \textbf{Question de recherche 1 (Mécanismes de coordination) :} Le modèle formel répond à cette question en fournissant un cadre structuré pour analyser et optimiser les mécanismes de coordination au sein des systèmes de cyberdéfense distribués. En modélisant l'environnement et les interactions entre agents via Dec-POMDP, cette contribution permet d'identifier les stratégies de coordination les plus efficaces dans diverses conditions.
          \end{itemize}

    \item \textbf{Approche de conception assistée (CybSMADA) :} En s'appuyant sur le modèle formel, cette thèse présente une méthodologie de conception assistée qui s'appuie sur l'apprentissage par renforcement multi-agents (MARL) pour automatiser la génération et l'optimisation des organisations SMA. Cette approche facilite la création de SMA adaptatifs et résilients capables de répondre efficacement aux menaces émergentes en temps réel.

          \begin{itemize}
              \item \textbf{Question de recherche 2 (Adaptabilité aux menaces émergentes) :} L'approche de conception assistée répond directement à cette question en permettant la création de SMA capables d'apprendre de leur environnement et de faire évoluer leurs stratégies de défense au fil du temps. L'utilisation de MARL garantit que le SMA peut s'adapter de manière dynamique à des vecteurs d'attaque nouveaux et imprévus.
          \end{itemize}

    \item \textbf{Mise en œuvre pratique et validation :} Le cadre et la méthodologie proposés sont validés par trois études de cas détaillées, chacune abordant différents aspects de la cyberdéfense, allant de la sécurité des infrastructures d'entreprise à la gestion des essaims de drones et à l'orchestration de Kubernetes. Ces études de cas démontrent l'efficacité du SMA proposé pour améliorer l'adaptabilité, l'autonomie et la résilience par rapport aux systèmes centralisés traditionnels.

          \begin{itemize}
              \item \textbf{Question de recherche 4 (Évolutivité des systèmes décentralisés) :} La validation par diverses études de cas fournit des preuves empiriques de l'évolutivité du cadre SMA proposé. En appliquant le cadre à différents scénarios et échelles de réseau, cette contribution montre comment le système peut maintenir ses performances et son efficacité même lorsque la complexité et la taille de l'environnement augmentent.
          \end{itemize}

    \item \textbf{Intégration des technologies émergentes :} Cette thèse explore également l'intégration de techniques d'IA avancées, telles que l'IA explicable (XAI) et MARL, dans le cadre SMA. Cette intégration améliore non seulement les capacités de prise de décision des agents, mais offre également une plus grande transparence et une plus grande fiabilité dans les opérations de cyberdéfense autonomes.

          \begin{itemize}
              \item \textbf{Question de recherche 3 (Intégration des technologies émergentes) :} L'exploration et la mise en œuvre des technologies émergentes dans le cadre du SMA répondent à cette question en démontrant comment les technologies de pointe comme XAI et l'identification du système peuvent être efficacement intégrées pour améliorer la sécurité et la résilience des systèmes de cyberdéfense décentralisés.
          \end{itemize}

\end{enumerate}


% \section{Définitions et propriétés}
% \begin{itemize}

% \item Définitions \& propriétés fondammentales pour la suite
% \item cyberdéfense, RL et SMA (+IA hybride éventuellement)
% \item ex : ouverture, dynamique, auto/réorganisation, explicabilité, etc.
% \end{itemize}

% \section{Travaux connexes}
% \begin{itemize}
% \item Travaux liés à l'AICA et autres SMA de Cyberdéfense
% \begin{itemize}
% \item SMA : organisation, modèle organisationnel...
% \item Travaux de l'Autonomous Cyber Operation
% \article ...
% \end{itemize}
% \end{itemize}

% \section{Lacunes théoriques et techniques}
% \begin{itemize}
% \item Manque de généricité, consistance, pas/peu objectif, peu formel, etc.
% \item Besoin d'un cadre théorique cohérent et générique si possible
% \end{itemize}

% \section{Installation du problème dans un cadre théorique}
% \begin{itemize}
% \item Motivation pour : Équipes vertes, bleues, rouges + réseau de noeud avec système d'attaque/contre-attaques d'après les standards de cyberdéfense, reste générique, etc.
% \end{itemize}

% \section{Répondre au problème grâce à une méthode de conception}
% \begin{itemize}
% \item Motiver les choix préliminaires pour la méthode en mettant en relation les travaux liés de façon cohérente
% \item Modèle organisationnel et MARL, Cyberdéfense avec MITRE...
% \item Cela doit permettre au lecteur de comprendre où je veux en venir par la suite...
% \end{itemize}