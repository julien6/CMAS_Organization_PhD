%*****************************************
\chapter{An Overview of the Domain and the Problem}\label{ch:domain_problem}
%*****************************************
%\setcounter{figure}{10}
% \NoCaseChange{Homo Sapiens}

\section{Definitions and Properties}
\begin{itemize}

    \item Définitions \& propriétés fondammentales pour la suite
    \item cyberdéfense, RL et SMA (+IA hybride éventuellement)
    \item ex : ouverture, dynamique, auto/réorganisation, explicabilité, etc.
\end{itemize}

\section{Related Works}
\begin{itemize}
    \item Travaux liés à l’AICA et autres SMA de Cyberdéfense
          \begin{itemize}
              \item SMA : organisation, modèle organisationel...
              \item Travaux de l'Autonomous Cyber Operation
              \item ...
          \end{itemize}
\end{itemize}

\section{Theoretical and Technical Gaps}
\begin{itemize}
    \item Manque de généricité, consistance, pas/peu objectif, peu formel, etc.
    \item Besoin d’un cadre théorique consistant et générique si possible
\end{itemize}

\section{Setting the Problem within a Theoretical Framework}
\begin{itemize}
    \item Motivation pour : Green, blue, red teams + réseau de noeud avec système d’ataque/contre-attaques d’après les standards de cyberdéfense, reste générique, etc.
\end{itemize}

\section{Answering the Problem through a Design Method}
\begin{itemize}
    \item Motiver les choix préliminaires pour la méthode en mettant en relation les travaux liés de façon cohérente
    \item Modèle organisationel et MARL, Cyberdéfense avec MITRE...
    \item Cela doit permettre au lecteur de comprendre où je veux en venir par la suite...
\end{itemize}
