\documentclass[11pt,a4paper]{article}

% =======================================
% PACKAGES
% =======================================
\usepackage[utf8]{inputenc}
\usepackage[T1]{fontenc}
\usepackage{lmodern}
\usepackage[french]{babel}
\usepackage{geometry}
\geometry{margin=2.2cm}
\usepackage{setspace}
\setstretch{1.15}
\usepackage{enumitem}
\usepackage{hyperref}
\hypersetup{colorlinks=true,linkcolor=blue,urlcolor=blue}
\usepackage{booktabs}
\usepackage{array}
\usepackage{csquotes}
\usepackage{multicol}
\usepackage{longtable}
\usepackage{pdflscape}
\usepackage{caption}
\captionsetup{width=\textwidth}
\usepackage{xcolor}
\definecolor{SectionColor}{RGB}{140,53,53}
\definecolor{SubSectionColor}{RGB}{140,53,53}
\usepackage{titlesec}
\usepackage{ulem}

% \titleformat{\section}
%   {\normalfont\Large\bfseries\color{SectionColor}}
%   {}{0pt}{#1\par\vspace{0.2em}\titlerule\vspace{0.8em}}



\titleformat{\section}{\normalfont\Large\bfseries\color{SectionColor}}{\arabic{section}.}{}{}[\vspace{-1em}\textcolor{black}{\titlerule}]

\titleformat{\subsection}
  {\normalfont\large\bfseries\color{SubSectionColor}}{}{0pt}{}
  
\titleformat{\subsubsection}
  {\normalfont\normalsize\bfseries\color{SubSectionColor}}{}{0pt}{}

\setlength{\parskip}{6pt}

\usepackage{fancyhdr}
\usepackage{lastpage}

\pagestyle{fancy}
\fancyhf{} % Efface les entêtes/pieds par défaut

% Texte du header
\lhead{\small{Julien Soulé}}
\rhead{\small{\today}}

\fancyfoot[C]{\textit{Page \thepage{} sur 15}}

% Optionnel : ligne en haut ou bas
\renewcommand{\headrulewidth}{0pt}
\renewcommand{\footrulewidth}{0pt}

% =======================================
\begin{document}

\begin{center}
    {\Large\textbf{Rapport d'activités}}\\[4pt]
    {\large Julien Soulé}\\[4pt]
    Candidature à la qualification 2025/2026 au coprs de Maître de conférences, section 27\\[4pt]
    \vspace{0.5cm}
\end{center}


\section*{Identité}
% =======================================================

\begin{itemize}
    \item \textbf{Nom} : Soulé
    \item \textbf{Prénom} : Julien
    \item \textbf{Adresse personnelle} : 35 Rue Mathieu de la Drôme, 26000 Valence, France
    \item \textbf{Adresse professionnelle} : Laboratoire LCIS, Grenoble INP -- UGA, 50 rue Barthélémy de Laffemas, CS 10054, 26902 Valence Cedex 9, France
    \item \textbf{Téléphone personnel} : +33 (0) 6 77 63 12 13
    \item \textbf{Téléphone professionnel} : +33 (0) 4 56 52 99 81
    \item \textbf{Courriel institutionnel} : \href{mailto:julien.soule@univ-grenoble-alpes.fr}{julien.soule@univ-grenoble-alpes.fr}
    \item \textbf{Courriel personnel} : \href{mailto:julien.soule@hotmail.fr}{julien.soule@hotmail.fr}
    \item \textbf{Pages scientifiques} :
          \begin{itemize}
              \item Portfolio : \url{https://julien6.github.io/home/}
              \item HAL : \url{https://hal.science/search?q=Julien+Soulé}
              \item ORCID : \url{https://orcid.org/0009-0002-3218-2614}
              \item DBLP : \url{https://dblp.org/pid/367/9947.html}
          \end{itemize}
    \item \textbf{Dépôts de code} :
          \begin{itemize}
              \item GitHub : \url{https://github.com/julien6}
          \end{itemize}
\end{itemize}

% =======================================================


% \section*{Présentation générale}
% % =======================================================

% Mon parcours s'est construit à l'interface entre l'ingénierie logicielle, les SMAs et la cybersécurité. Après ma formation d'ingénieur à l'INSA Rennes, j'ai rejoint le \textbf{Laboratoire de Conception et d'Intégration des Systèmes} (LCIS) pour une thèse CIFRE avec Thales, mêlant problématiques industrielles et recherche académique autour de l'automatisation de la conception de SMAs pour assurer la cyberdéfense.

% Mes travaux portent sur la conception et l'étude de SMAs pour la Cyberdéfense, combinant apprentissage par renforcement multi-agent, modélisation organisationnelle et analyse de comportements. J'ai développé plusieurs contributions méthodologiques, dont la méthode MAMAD et la plateforme CybMASDE, en conservant un lien direct avec les besoins opérationnels industriels.

% En parallèle, j'ai assuré des enseignements à l'Esisar et l'IUT de Valence sur les systèmes d'exploitation, la programmation système et les fondements de la cybersécurité, acquérant une expérience pédagogique variée auprès de publics de niveaux différents.


% =======================================================
\section*{Parcours universitaire}
% =======================================================

\subsection*{Doctorat}

\begin{itemize}
    \item \textbf{Intitulé} : Doctorat en informatique
    \item \textbf{Dates} : 1er octobre 2022 -- 17 novembre 2025
    \item \textbf{Nom du laboratoire} : Laboratoire de Conception et d'Intégration des Systèmes (LCIS)
    \item \textbf{Etablissement d'accueil} : Grenoble INP -- UGA
    \item \textbf{Spécialités} : Système Multi-Agents (thématique 86) et Cyberdéfense (thématique E4)
    \item \textbf{Encadrement} :
          \begin{itemize}
              \item \textbf{Jean-Paul Jamont} (Professeur, UGA, LCIS) -- Directeur de thèse
              \item \textbf{Michel Occello} (Professeur, UGA, LCIS) -- Co-directeur de thèse
              \item \textbf{Louis-Marie Traonouez} (Ingénieur IA, Thales LAS) -- Accompagnement industriel
              \item \textbf{Paul Théron} (Chercheur, AICA IWG) -- Accompagnement scientifique
          \end{itemize}
    \item \textbf{Rapporteurs} :
          \begin{itemize}
              \item \textbf{Laurent Vercouter} -- Professeur, INSA Rouen, LITIS
              \item \textbf{Gauthier Picard} -- Directeur de recherche, ONERA
          \end{itemize}
    \item \textbf{Examinateurs} :
          \begin{itemize}
              \item \textbf{Oum-El-Kheir Aktouf} -- Professeure, UGA, LCIS
              \item \textbf{Aurélie Beynier} -- Professeure, Sorbonne Université
              \item \textbf{Flavien Balbo} -- Professeur, École des Mines de Saint-Étienne
              \item \textbf{Laeticia Matignon} -- Maîtresse de Conférences, Université Claude Bernard Lyon
          \end{itemize}
\end{itemize}

\noindent
J'ai entrepris une thèse en informatique au sein du Laboratoire de Conception et d'Intégration des Systèmes (LCIS), rattaché à Grenoble INP -- UGA, entre 2022 et 2025. Ce travail s'inscrit dans une convention CIFRE avec Thales LAS, ce qui m'a permis d'évoluer à la fois dans un environnement académique et dans un contexte industriel fortement orienté vers les problématiques de Cyberdéfense.
%
Ma thèse, intitulée \textit{De l'Organisation d'un Système Multi-Agent de Cyberdéfense}, porte sur la conception SMAs dédiés à la détection et à la réponse à des comportements adversariaux dans des environnements distribués. Les travaux issus de cette thèse ont donné lieu à plusieurs publications composées de deux journaux, quatre conférences internationales et six conférences nationales.
La soutenance a eu lieu le 17~novembre 2025. Le jury était composé de :

\begin{itemize}
    \item \textbf{Laurent Vercouter} (Professeur, INSA Rouen, LITIS) -- Rapporteur
    \item \textbf{Gauthier Picard} (Directeur de recherche, ONERA) -- Rapporteur
    \item \textbf{Jean-Paul Jamont} (Professeur, UGA, LCIS) -- Président du jury
    \item \textbf{Michel Occello} (Professeur, UGA) -- Examinateur
    \item \textbf{Louis-Marie Traonouez} (Ingénieur de recherche, Thales LAS) -- Examinateur
\end{itemize}

\vspace{0.4cm}

\subsection*{Diplôme d'ingénieur}

\begin{itemize}
    \item \textbf{Intitulé} : Diplome d'ingénieur -- Grade de Master
    \item \textbf{Dates} : 1er septembre 2015 -- 8 janvier novembre 2021
    \item \textbf{Etablissement d'accueil} : Institut National des Sciences Appliquées de Rennes
    \item \textbf{Spécialités} : Intelligence artificielle (thématique 80) et Cyberdéfense (thématique E4)
\end{itemize}

Avant la thèse, j'ai obtenu le diplôme d'ingénieur en informatique de l'INSA Rennes (promotion 2020). Cette formation m'a permis d'acquérir des bases solides en génie logiciel, en programmation système, en réseaux et en sécurité, ainsi qu'une première exposition aux approches multi-agents.

Dans le cadre de mon cursus, j'ai également effectué un semestre d'études à l'École de Technologie Supérieure (Montréal, Canada), entre janvier et mai 2019. Cette expérience m'a offert l'occasion de suivre des enseignements complémentaires en informatique dans un environnement académique différent et de m'ouvrir à des pratiques pédagogiques et organisationnelles du système nord-américain.

Au-delà du cursus ingénieur, j'ai pu me former à divers outils et méthodes liés aux systèmes distribués, au développement logiciel et à la cybersécurité, principalement au cours de projets, de stages et d'expériences professionnelles préalables à la thèse.


\section*{Parcours professionnel}
% =======================================================

% Mon parcours professionnel s'est développé en parallèle de mon évolution académique, avec une continuité autour des systèmes distribués, de la programmation et des questions de sécurité.

\subsection*{Doctorant CIFRE / Ingénieur de recherche (2022--2025)}

\begin{itemize}
    \item \textbf{Intitulé} : Ingénieur de recherche doctorant en informatique
    \item \textbf{Dates} : Juillet 2022 -- Octobre 2025
    \item \textbf{Établissements} :
          \begin{itemize}
              \item Laboratoire LCIS, Université Grenoble Alpes, Valence, France
              \item \textit{Thales Land} \& \textit{Air Systems}, Rennes, France
          \end{itemize}
    \item \textbf{Statut} : Salarié en convention CIFRE (CDD de 3 ans)
\end{itemize}

\noindent
J'ai réalisé mon doctorat au sein du LCIS dans le cadre d'une convention CIFRE avec Thales LAS. Le temps était réparti avec 70\% du temps sur le site du LCIS pour la recherche et 30\% chez Thales LAS (sur le site de Rennes dit \textquote{La Ruche}) pour la collaboration industrielle. Cette position, à la fois académique et industrielle, m'a conduit à travailler sur des problématiques liées à la Cyberdéfense multi-agent, à l'apprentissage par renforcement et à la modélisation organisationnelle. Au sein de Thales, j'ai participé à la conception et à l'expérimentation de modules de simulation et d'analyse destinés à des scénarios opérationnels. % Cette collaboration m'a permis de confronter mes travaux de recherche aux contraintes d'environnements appliqués, en collaboration étroite avec les équipes de Thales pour adapter les outils aux besoins industriels, en particulier sur la robustesse des systèmes, la reproductibilité et la capacité à expliquer les comportements obtenus.

\subsection*{Vacataire d'enseignement en informatique}

\begin{itemize}
    \item \textbf{Intitulé} Enseignant vacataire en informatique
    \item \textbf{Dates} Années scolaires 2022--2023, 2023--2024, 2024--2025
    \item \textbf{Établissement}
          \begin{itemize}
              \item Grenoble INP -- Esisar, Valence, France
              \item IUT, Valence, France
          \end{itemize}
    \item \textbf{Statut} Vacataire
\end{itemize}

\noindent
Durant mon doctorat, j'ai également assuré des enseignements en tant que vacataire Esisar et à l'IUT de Valence. A l'Esisar, j'ai encadré pour un total de 135.25 hEqTD pour l'encadrement à l'Esisar (encadrement de séances de travaux pratiques, de séances de travaux dirigés et évaluations d'étudiants ingénieurs de niveau variant entre première et cinquième années). À l'IUT de Valence, pour un total de 48 hEqTD, j'ai encadré des séances de travaux dirigés et de travaux pratiques pour des étudiants en BUT de première et deuxième années, principalement sur des thématiques liées aux systèmes d'exploitation, à la programmation système et à la structuration de l'information.

\subsection*{Ingénieur R\&D -- Thales LAS (2021--2022)}

\begin{itemize}
    \item \textbf{Intitulé} Ingénieur R\&D en informatique
    \item \textbf{Dates} 1er Décembre 2021 -- Juin 2022
    \item \textbf{Établissement} Thales LAS, France
    \item \textbf{Statut} Salarié CDD (7 mois)
\end{itemize}

\noindent
Avant le début du doctorat, j'ai occupé un poste d'ingénieur R\&D au sein de Thales LAS, travaillant sur la modélisation de réseaux et la détection d'anomalies en cybersécurité. Cette expérience m'a permis d'acquérir une connaissance des technologies utilisées en défense et de contribuer au développement d'outils de simulation multi-agent pour la validation de stratégies de détection d'intrusions.

\subsection*{Ingénieur logiciel -- Atos / CNES (2020--2021)}

\begin{itemize}
    \item \textbf{Intitulé} Ingénieur développement
    \item \textbf{Dates} 17 août 2020 -- 1er octobre 2021
    \item \textbf{Établissement} Atos, Toulouse, France
    \item \textbf{Statut} Salarié CDI (1 an 8 mois)
\end{itemize}

\noindent
J'ai été salarié en CDI chez Atos sur la poursuite du projet \textbf{ISIS} (Initiative for Space Innovation Standard) du \textbf{CNES} (Centre national d'études spatiales), dédié au contrôle et à la surveillance des infrastructures au sol. Mon travail portait sur la conception d'outils de supervision, des scripts d'automatisation et des interfaces d'analyse.
Cette expérience m'a donné un aperçu du fonctionnement de systèmes critiques à grande échelle, ainsi que de l'importance des outils de diagnostic et de remontée d'information dans des environnements où la continuité de service est essentielle.

\subsection*{Stage Projet de fin d'études}

\begin{itemize}
    \item \textbf{Intitulé} Ingénieur de développement stagiaire
    \item \textbf{Dates} 17 février 2020 -- 14 août 2020
    \item \textbf{Établissement} Atos, Toulouse, France
    \item \textbf{Statut} Salarié CDD (6 mois)
\end{itemize}

\noindent
Dans le cadre de mon projet de fin d'études à l'INSA Rennes, j'ai effectué un stage de six mois chez Atos, travaillant sur le projet ISIS du CNES. J'ai contribué au développement d'outils de supervision et d'analyse pour les infrastructures au sol, en mettant l'accent sur la robustesse et la facilité d'utilisation des interfaces. Ce stage m'a permis de mettre en pratique mes compétences en programmation et en gestion de projets dans un contexte industriel exigeant.

\subsection*{Stage ingénieur}

\begin{itemize}
    \item \textbf{Intitulé} Ingénieur stagiaire
    \item \textbf{Dates} 1er mai 2019 -- 31 juillet
    \item \textbf{Établissement} SQLI, Toulouse, France
    \item \textbf{Statut} Salarié CDD (3 mois)
\end{itemize}

\noindent
Durant l'été 2019, j'ai effectué un stage de trois mois chez SQLI à Toulouse, où j'ai participé au développement d'une application web de gestion de contenu pour des projets \textit{Airbus Helicopters}. Ce stage m'a permis de renforcer mes compétences en développement web, en gestion de bases de données et en travail en équipe dans un contexte professionnel.


% \section*{Résumé de thèse}
% % =======================================================

% Ma thèse s'inscrit dans le domaine des SMAs appliqués à la Cyberdéfense, et plus particulièrement dans l'étude de mécanismes permettant de concevoir, d'entraîner et d'analyser des ensembles d'agents autonomes évoluant dans des environnements complexes, partiellement observables et soumis à des comportements hostiles. Les travaux ont été menés au sein du LCIS, en collaboration avec Thales LAS dans le cadre d'une convention CIFRE, ce qui a orienté une partie des réflexions vers des scénarios réalistes rencontrés dans les systèmes de défense et les environnements distribués de grande ampleur.

% Du point de vue scientifique, le travail se situe à l'intersection de deux approches : les modèles organisationnels issus des SMAs, qui permettent de structurer les rôles, missions et relations entre agents; et l'apprentissage par renforcement multi-agent, utilisé pour faire émerger des comportements coordonnés dans des situations qui ne se prêtent pas facilement à une modélisation analytique. L'objectif général de la thèse était de proposer une méthode permettant de combiner ces deux dimensions afin d'obtenir des comportements plus explicables, plus contrôlables et plus robustes face à l'incertitude.

% L'hypothèse de départ était qu'une contrainte organisationnelle, exprimée sous la forme d'un modèle de type MOISE+, pouvait guider ou structurer l'apprentissage multi-agent, et qu'une telle intégration favoriserait l'interprétation des politiques apprises. Plusieurs verrous ont rapidement émergé, notamment la difficulté d'articuler un modèle organisationnel statique avec des politiques apprises de manière distribuée, la question de la reproductibilité des comportements dans des environnements changeants, ainsi que l'absence d'outils permettant d'analyser finement ce que les agents apprennent réellement.

% Pour répondre à ces enjeux, la thèse s'articule autour de trois contributions principales.
% La première est une méthodologie, appelée \emph{MAMAD}, qui décrit un processus en plusieurs étapes : modélisation, apprentissage, analyse et transfert. Elle vise à structurer la conception et l'étude de SMAs destinés à des scénarios de Cyberdéfense. La seconde contribution concerne l'intégration d'un modèle organisationnel (MOISE+) dans l'apprentissage multi-agent, en influençant la récompense, la structure des interactions ou la façon dont les politiques sont contraintes. Enfin, la troisième contribution est un outil d'analyse, \emph{Auto-TEMM}, qui permet d'extraire automatiquement, à partir des traces d'exécution, une organisation émergente représentant les rôles et comportements que les agents ont effectivement adoptés.

% L'ensemble de ces contributions s'appuie sur des expérimentations réalisées dans plusieurs environnements de simulation, certains développés spécifiquement pour ce travail. Ces environnements mettent en scène des interactions entre agents attaquants et défenseurs, ou des tâches de surveillance distribuée. Les résultats montrent que l'organisation peut effectivement influencer l'apprentissage, et qu'elle constitue un levier intéressant pour améliorer la lisibilité des comportements appris. Ils soulignent également l'intérêt d'analyser non seulement les performances, mais aussi la structure des interactions résultantes.

% Comme tout travail de recherche, celui-ci présente des limites. Les environnements étudiés restent des abstractions de systèmes plus larges; les modèles organisationnels utilisés sont parfois simplifiés; et l'intégration entre organisation et apprentissage pourrait être approfondie sur le plan théorique. Ces limites ouvrent des perspectives naturelles, notamment l'étude d'organisations dynamiques, l'intégration de contraintes de sûreté issues de la Cyberdéfense réelle, ou l'application de ces approches à la robotique multi-agent ou à des systèmes cyber-physiques plus proches des cas d'usage industriels.

% \section*{Contexte CIFRE et partenariats industriels}
% % =======================================================

% Ma thèse a été réalisée dans le cadre d'une convention CIFRE associant le Laboratoire de Conception et d'Intégration des Systèmes (LCIS) et l'entreprise Thales LAS. Cette collaboration a joué un rôle important dans l'orientation de mes travaux, en particulier sur les aspects liés à la cybersécurité et à l'observation de comportements dans des environnements distribués.

% Thales LAS dispose d'une activité de recherche et développement importante dans les domaines de la cybersécurité, de la surveillance et des systèmes d'information critiques. Le groupe mène également plusieurs travaux liés à la défense autonome et aux architectures multi-agents. Dans ce contexte, les objectifs exprimés concernaient principalement la capacité à modéliser des situations complexes impliquant de nombreux acteurs, à simuler des interactions hostiles, et à explorer des mécanismes permettant d'automatiser une partie de la détection ou de la réponse.

% Le cadre CIFRE m'a conduit à travailler régulièrement avec les équipes de Thales, notamment sur la définition d'environnements de simulation adaptés aux problématiques de Cyberdéfense. Plusieurs modules développés au cours de la thèse ont été intégrés, à des degrés variés, dans les outils internes de prototypage. Cela concerne en particulier des éléments liés à l'analyse de traces multi-agents, la génération de scénarios ou l'extraction automatisée de comportements. Cette intégration s'est faite progressivement, selon les besoins exprimés par les ingénieurs et en tenant compte des contraintes propres aux environnements de développement internes.

% Sur le plan scientifique, la collaboration a fourni un accès privilégié à des retours d'expérience, à des modèles opérationnels et à des discussions régulières avec des ingénieurs spécialisés dans la défense. Ces échanges ont contribué à affiner certaines hypothèses de travail et à ancrer les contributions dans des problématiques concrètes. Ils ont également influencé la conception des environnements d'expérimentation, en orientant les choix vers des scénarios représentatifs des situations rencontrées dans les systèmes cyber.

% Enfin, le partenariat a facilité des interactions avec d'autres équipes, aussi bien du côté industriel que dans le milieu académique. Des échanges ponctuels ont eu lieu avec des équipes impliquées dans l'architecture des SMAs, dans l'analyse comportementale et dans l'apprentissage distribué, ce qui a enrichi la réflexion globale et apporté une diversité de points de vue.

% \section*{Compétences techniques}
% % =======================================================

% Au fil de mes expériences en entreprise et durant la thèse, j'ai été amené à travailler avec un ensemble d'outils et de technologies liés à la programmation, aux environnements distribués, à l'apprentissage automatique et aux pratiques de développement logiciel. Les éléments listés ci-dessous reflètent les outils les plus couramment utilisés dans mes travaux de recherche et dans les projets auxquels j'ai contribué.

% \subsection*{Langages}

% \begin{itemize}
%     \item Python : langage utilisé pour la majorité des développements liés aux environnements multi-agents, aux simulations et aux expérimentations en apprentissage par renforcement.
%     \item C et C++ : utilisés principalement pour des modules bas niveau, ou dans des contextes nécessitant un contrôle fin sur les ressources et les performances.
%     \item Java : pratiqué dans le cadre d'enseignements et de projets liés à la programmation orientée objet.
%     \item Bash et scripts Unix : pour l'automatisation de tâches, la gestion d'environnements et le déploiement de prototypes.
% \end{itemize}

% \subsection*{Frameworks et bibliothèques}

% \begin{itemize}
%     \item PyTorch : pour la mise en œuvre de modèles d'apprentissage, notamment les architectures utilisées dans les world models ou dans les algorithmes de MARL.
%     \item PettingZoo : utilisé pour la création et la manipulation d'environnements multi-agents.
%     \item RLlib : exploité ponctuellement pour comparer ou valider certains algorithmes d'apprentissage existants.
%     \item Outils d'organisation multi-agents (MOISE+) : intégrés à différents niveaux selon les besoins des simulations.
% \end{itemize}

% \subsection*{Outils scientifiques et DevOps}

% \begin{itemize}
%     \item Git : utilisé au quotidien, autant pour les développements de recherche que pour les projets collaboratifs.
%     \item Docker : pour la création d'environnements reproductibles, l'isolation de modules expérimentaux et les tests d'intégration.
%     \item Kubernetes : rencontré principalement dans le cadre des travaux portant sur l'autoscaling multi-agent.
%     \item Pipelines d'intégration continue : utilisés pour automatiser certains tests et s'assurer de la stabilité du code sur la durée.
% \end{itemize}

% \subsection*{Outils pédagogiques}

% \begin{itemize}
%     \item Moodle : utilisé pour la diffusion de supports de cours, la gestion d'évaluations et la communication avec les étudiants.
%     \item Notebooks Python : employés dans le cadre des TP ou pour illustrer certains concepts liés aux systèmes ou à l'IA.
%     \item Outils d'automatisation de correction et de distribution de matériel pédagogique, selon les besoins des enseignements.
% \end{itemize}

\section*{Activités d'enseignement}
% =======================================================

\noindent
Depuis le début de ma thèse, j'ai assuré différents enseignements au sein de l'Esisar et de l'IUT de Valence. Ces interventions ont concerné plusieurs niveaux, allant du premier cycle universitaire aux années de formation d'ingénieur. Elles m'ont permis d'acquérir une expérience variée de l'encadrement, en particulier dans des domaines liés aux systèmes, à la programmation et à la cybersécurité.

Le \autoref{tab:enseignement-esisar} et le \autoref{tab:enseignement-iut} reprennent les informations concernant les enseignements que j'ai réalisés respectivement à l'Esisar et à l'IUT de Valence. Les volumes horaires sont données en hEqTD et les effectifs en nombre d'étudiants, le produit $n \times x$ signifie la responsabilité de $x$ groupes de $n$ étudiants. Les années d'exercice correspondent aux années scolaires. La table 3 reprend le total des enseignements par \textbf{Nature}.

Les trois années de vacations durant mon doctorat (2022--2023, 2023--2024 et 2024--2025) m'ont permis de me former à la pédagogie et la didactique et de confronter les notions vues avec la réalité des séances de travaux pratiques et dirigés que j'encadrais. Par exemple, la difficulté d'aboutir à une juste quantité d'autonomie laissée à des élèves en projet pour qu'ils aient l'impression de pouvoir choisir la direction que prennent leurs travaux sans pour autant qu'ils ne se bloquent devant un trop grand nombre de choix.

\subsection*{Philosophie pédagogique}

Mon approche de l'enseignement repose principalement sur des activités pratiques et progressives, visant à rendre les étudiants autonomes dans l'analyse et la résolution de problèmes techniques. J'essaie autant que possible de relier les notions présentées à des situations réelles ou à des environnements proches du monde professionnel, notamment pour les cours liés aux systèmes ou à la cybersécurité.
L'évaluation continue occupe une place importante dans cette démarche, car elle permet aux étudiants d'identifier plus rapidement leurs points d'amélioration et d'ajuster leur méthode de travail au fil du semestre. Enfin, je m'efforce de proposer des supports clairs, accompagnés d'exemples concrets, afin de faciliter la compréhension de notions parfois abstraites pour des publics débutants.
% TODO:
%  - dire que pour les TDs, je fais un rappel de cours rapide au début en utilisant un support de cours que j'ai moi-même rédigé avec la collaboration des professeurs responsables de la matière
%  - dire que j'essaie de faire des liens avec le monde professionnel et les enjeux réels
%  - dire que j'essaie de faire des évaluations continues pour aider les étudiants à s'améliorer au fil du semestre
%  - dire que j'essaie de faire des supports clairs et avec des exemples concrets

\subsection*{Résumé chiffré des volumes d'enseignement}

\begin{itemize}
    \item \textbf{Travaux dirigés (TD)} : 41.5 h
    \item \textbf{Travaux pratiques (TP)} : 46.5 h
    \item \textbf{Projets encadrés} : interventions ponctuelles selon les besoins des équipes pédagogiques
\end{itemize}

Ces valeurs correspondent aux heures attestées sur la période 2022--2025. Elles incluent à la fois des enseignements en école d'ingénieur et en IUT.

\begin{landscape}

    \setlength{\LTcapwidth}{\linewidth}


    \begin{longtable}{%
            p{2cm}
            p{1.3cm}
            p{3.5cm}
            p{2cm}
            p{1.2cm}
            p{1.5cm}
            p{1.5cm}
            p{5cm}
            p{3.5cm}
        }
        \caption{\textbf{Enseignements en tant que \textit{vacataire}} (\textbf{Statut}) à l'Esisar (\textbf{Établissement}), devant un public d'étudiants ingénieurs (\textbf{Public}).}
        \label{tab:enseignement-esisar}                                  \\[0.3cm]
        \toprule
        \textbf{Année d'exercice}                                 &
        \textbf{Niveau}                                           &
        \textbf{Nom de la matière}                                &
        \textbf{Volume horaire}                                   &
        \textbf{Vol. eq TD}                                       &
        \textbf{Effectifs}                                        &
        \textbf{Nature}                                           &
        \textbf{Responsabilités}                                  &
        \textbf{Références}                                              \\
        \midrule
        \endfirsthead

        \multicolumn{8}{c}%
        {{\small\itshape Table \thetable{} -- suite}}                    \\
        \toprule
        \textbf{Année d'exercice}                                 &
        \textbf{Niveau}                                           &
        \textbf{Nom de la matière}                                &
        \textbf{Volume horaire}                                   &
        \textbf{Vol. eq TD}                                       &
        \textbf{Effectifs}                                        &
        \textbf{Nature}                                           &
        \textbf{Responsabilités}                                  &
        \textbf{Références}                                              \\
        \midrule
        \endhead

        \bottomrule
        \endfoot

        % --- Lignes de données (mêmes exemples que ci-dessus) ---
        2023--2024                                                & M1 &
        Introduction aux bases de données                         &
        18~h CM                                                   &
        18                                                        &
        45                                                        &
        CM                                                        &
        À partir de supports existants, préparation et réalisation
        des cours. Rédaction et correction d'un sujet de contrôle
        continu et d'un sujet d'examen.                           &
        Lien non disponible (changement de maquette)                     \\
        \midrule

        2023--2024                                                & M1 &
        Introduction aux bases de données                         &
        12~h TD                                                   &
        18                                                        &
        $\sim 22 \times 2$                                        &
        TD                                                        &
        Modification des sujets de TD, encadrement des étudiants. &
        (même référence)                                                 \\
        % ... etc. ...
    \end{longtable}

    \begin{longtable}{%
            p{2cm}
            p{1.3cm}
            p{3.5cm}
            p{2cm}
            p{1.2cm}
            p{1.5cm}
            p{1.5cm}
            p{5cm}
            p{3.5cm}
        }
        \caption{\textbf{Enseignements en tant que \textit{vacataire}} (Statut) à l'IUT de Valence (\textbf{Établissement}), devant un public d'étudiants en BUT (\textbf{Public}).}
        \label{tab:enseignement-iut}                                     \\[0.3cm]
        \toprule
        \textbf{Année d'exercice}                                 &
        \textbf{Niveau}                                           &
        \textbf{Nom de la matière}                                &
        \textbf{Volume horaire}                                   &
        \textbf{Vol. eq TD}                                       &
        \textbf{Effectifs}                                        &
        \textbf{Nature}                                           &
        \textbf{Responsabilités}                                  &
        \textbf{Références}                                              \\
        \midrule
        \endfirsthead

        \multicolumn{8}{c}%
        {{\small\itshape Table \thetable{} -- suite}}                    \\
        \toprule
        \textbf{Année d'exercice}                                 &
        \textbf{Niveau}                                           &
        \textbf{Nom de la matière}                                &
        \textbf{Volume horaire}                                   &
        \textbf{Vol. eq TD}                                       &
        \textbf{Effectifs}                                        &
        \textbf{Nature}                                           &
        \textbf{Responsabilités}                                  &
        \textbf{Références}                                              \\
        \midrule
        \endhead

        \bottomrule
        \endfoot

        % --- Lignes de données (mêmes exemples que ci-dessus) ---
        2023--2024                                                & M1 &
        Introduction aux bases de données                         &
        18~h CM                                                   &
        18                                                        &
        45                                                        &
        CM                                                        &
        À partir de supports existants, préparation et réalisation
        des cours. Rédaction et correction d'un sujet de contrôle
        continu et d'un sujet d'examen.                           &
        Lien non disponible (changement de maquette)                     \\
        \midrule

        2023--2024                                                & M1 &
        Introduction aux bases de données                         &
        12~h TD                                                   &
        18                                                        &
        $\sim 22 \times 2$                                        &
        TD                                                        &
        Modification des sujets de TD, encadrement des étudiants. &
        (même référence)                                                 \\
        % ... etc. ...
    \end{longtable}

\end{landscape}

\subsection*{Descriptions qualitatives}

Les enseignements assurés couvrent plusieurs blocs thématiques :

\subsubsection*{Systèmes d'exploitation}

Les cours et TP ont porté sur la gestion des processus, la mémoire, les appels système, les fichiers et les interactions avec le noyau. Les séances en travaux pratiques consistaient notamment à manipuler les primitives système, analyser le comportement de programmes concurrents et diagnostiquer des erreurs courantes. Ces activités permettaient aux étudiants d'acquérir les bases nécessaires à la programmation système ou aux enseignements avancés de cybersécurité.

\subsubsection*{Programmation système}

Les enseignements dispensés en BUT Informatique s'appuient sur des exercices progressifs autour de la manipulation de fichiers, des signaux, de la gestion des processus et de la synchronisation. L'objectif était d'amener les étudiants à comprendre les mécanismes sous-jacents aux applications qu'ils utilisent au quotidien, et de leur donner des outils pour analyser des comportements inattendus dans des contextes proches du réel.

\subsubsection*{Cyberdéfense}

Dans le cadre du module de veille technologique, j'ai participé à l'encadrement de travaux portant sur des thématiques de cybersécurité, en particulier l'analyse de scénarios d'attaque, la détection d'anomalies et la mise en place de stratégies défensives. Ces interventions prennent la forme de séances courtes mais interactives, orientées vers la compréhension de mécanismes concrets plutôt que vers la théorie pure.

\subsubsection*{Structuration et traitement de l'information}

Ce module destiné aux étudiants de BUT GEA vise à introduire les notions de base liées à l'organisation, la manipulation et l'analyse de données simples. Les séances comprenaient des exemples tirés de cas métiers, afin d'aider les étudiants à comprendre l'intérêt de ces techniques dans des contextes variés tels que la gestion administrative ou la logistique.

\subsection*{Encadrements étudiants}

J'ai eu l'occasion de participer à différents encadrements dans des contextes pédagogiques variés, notamment :

\begin{itemize}
    \item projets de première et deuxième année en école d'ingénieur, en particulier sur des thématiques liées à la programmation C et aux travaux pratiques de cybersécurité ;
    \item accompagnement d'étudiants dans la réalisation de projets d'analyse ou de simulation, notamment au début ou en fin de semestre selon les besoins des équipes pédagogiques ;
    \item suivi d'avancement, relecture de productions et participation aux évaluations orales.
\end{itemize}

Ces encadrements permettent d'aborder les difficultés rencontrées par les étudiants dans des situations moins formelles que les cours, et offrent un cadre plus adapté pour discuter des méthodes de travail ou des stratégies de résolution de problèmes.

\subsection*{Projet pédagogique}

À moyen terme, j'aimerais contribuer à la mise en place de modules autour des SMAs, de l'apprentissage par renforcement et de la cybersécurité. Un autre axe d'intérêt concerne l'intégration d'activités pratiques plus avancées dans les cours existants, par exemple sous la forme de mini-projets ou d'études de cas s'appuyant sur des environnements simulés.
La diversité des enseignements suivis durant ces dernières années constitue une base solide pour envisager la création d'enseignements transversaux reliant systèmes, IA distribuée et sécurité informatique.


\clearpage

\section*{Activités de recherche}
% =======================================================

L’ensemble de mes activités de recherche s’est déroulé au Laboratoire de Conception et d’Intégration des Systèmes de Grenoble INP -- Université Grenoble Alpes, situé à Valence dans la Drôme (\textbf{Lieu}) ainsi que dans l'entité dite \textquote{La Ruche} de Thales Land and Air Systems situé à Rennes dans l'Ille-et-Villaine (\textbf{Lieu}). Mes activités de recherche se sont développées dans le cadre de la thèse et s'articulent aujourd'hui autour de plusieurs axes complémentaires. Elles se situent à l'intersection des SMAs, de l'apprentissage automatique et de la cybersécurité. Les travaux réalisés s'appuient à la fois sur des problématiques théoriques liées à l'organisation des SMAs, leur explicabilité, et leur application pour des besoins industriels émanant de scénarios de cyberdéfense.
%
Les principaux axes qui structurent mes travaux sont les suivants :

\begin{itemize}
    \item \textbf{Modèles organisationnels pour SMAs} : utilisation de structures formelles pour décrire rôles, missions et relations entre agents, avec un intérêt particulier pour les modèles inspirés de MOISE+.
    \item \textbf{Apprentissage par renforcement multi-agent} : étude de mécanismes permettant d'apprendre des comportements coordonnés, en particulier dans des environnements partiellement observables ou soumis à des incertitudes.
    \item \textbf{Cyberdéfense autonome} : modélisation de scénarios d'attaque et de défense, conception de stratégies automatisées et analyse des interactions entre agents.
    \item \textbf{World Models et simulation} : construction de modèles dynamiques basés sur l'apprentissage permettant de simuler des environnements complexes à partir de traces ou d'observations.
\end{itemize}

Ces axes se recoupent régulièrement, et une partie du travail de thèse a consisté à les articuler dans un cadre cohérent.

\subsection*{Activités de recherche}

TODO
% Décrire les activités de recherche, notamment :
%  - thématique, lieu et nature de chaque activité ;
%  - résultats théoriques ou méthodologiques, synthèses, expériences, mesures, évaluations ;
%  - éléments de valorisation des productions logicielles et/ou matérielles (développement de logiciels ou de matériels, dépôts de logiciels à l’APP, etc.) ;
%  - visibilité :
%      - organisation de colloques, formations spécialisées, etc.,
%      - séjours scientifiques dans d’autres laboratoires/établissements, coopérations éventuelles, etc.,
%      - participation à des comités de programmes/éditoriaux
%  - encadrements éventuels, en précisant la participation du·de la candidat·e dans cet encadrement, le pourcentage d’encadrement, la date, la durée, la thématique, les résultats obtenus :
%      - projets de fin d’études d’ingénieur·e, mémoires CNAM, mémoires de Master
%      - thèses, postdoctorant·e·s, ingénieur·e·s (PR)
%  - participation à des jurys de thèse (voire d’HDR, pour les candidat·e·s PR) en précisant la fonction assurée (président.e, rapporteur.e) ;
%  - activités de médiation ;
%  - activités de transfert.


\subsection*{Publications}

% TODO
%  - l’identité de tous les auteur·e·s
%  - l’année de publication
%  - l’intitulé (complet) de la revue ou conférence
%  - le type d’article (long/court, présentation/poster…)
%  - les numéros de page (ou à défaut le nombre de pages)
%  - s’il s’agit d’une session annexe de la conférence, le préciser
%  - arguments (éventuellement, métriques) montrant l’importance/la sélectivité de la revue/conférence
%  - preuve de l’existence de la publication autre que l’article lui-même (lien vers les actes, une archive en ligne, le planning de la conférence, le sommaire de la revue, le DOI, lettre d’acceptation si l’article était à l’état “soumis”, etc.)
%  - distinguer les conférences nationales et internationales (comité de programme et audience internationales)
%  - indiquer les 3 (5 pour les PR) travaux fournis (les plus visibles) et les situer dans leur contexte de recherche. Pour ces publications, le·a candidat·e décrira sa contribution propre en informatique.
%  - pour les travaux pluridisciplinaires, préciser la contribution à la discipline informatique.

\noindent
La description de ma contribution pour chaque papier est basée sur la taxonomie CRediT\footnote{\url{https://credit.niso.org/}}:
%
\begin{multicols}{2}
    \begin{enumerate}[leftmargin=*, itemsep=0pt, topsep=0pt, label=\arabic*.]
        \item Conceptualisation
        \item Traitement des données
        \item Analyse formelle
        \item Acquisition des fonds
        \item Investigation
        \item Méthodologie
        \item Administration du projet
        \item Apport des ressources matérielles
        \item Développement logiciel
        \item Supervision
        \item Validation
        \item Visualisation
        \item Rédaction – première version
        \item Rédaction – après relecture \& modification
    \end{enumerate}
\end{multicols}

\subsubsection*{Revues nationales avec comité de lecture}

% TODO : Revue ROIA

\subsubsection*{Conférences internationales avec comité de lecture}

% TODO : AAMAS, CLOUD, SMC

\subsubsection*{Conférences nationales avec comité de lecture}

% TODO : JFSMA, RESSI, RJCIA

\subsubsection*{Productions logicielles et mise à disposition de données.}

% TODO : CybMASDE, environnements RL, World Models, MOISE+MARL, Auto-TEMM

\subsubsection*{Autres: présentations}


% ========================

J'ai également participé, à une échelle modeste mais régulière, à différentes activités d'animation scientifique :

\begin{itemize}
    \item relectures ponctuelles pour des conférences nationales ;
    \item participation à des séminaires du LCIS et à des échanges internes autour des SMAs et de la cybersécurité ;
    \item interventions lors d'ateliers ou de présentations internes, notamment auprès des équipes de Thales.
\end{itemize}

Ces activités ont contribué à maintenir un lien régulier avec la communauté et à confronter les contributions de la thèse à des points de vue variés.


\section*{Activités / responsabilités collectives, administratives}
% =======================================================

% Décrire les activités administratives, en détaillant complètement au moins les trois dernières années effectives :
%     - sujet, lieu et nature de chaque activité
%     - évaluation du temps consacré à ces activités et des éventuels résultats obtenus

Au cours de la thèse, j'ai participé à plusieurs activités collectives et administratives, aussi bien au sein du laboratoire que dans des groupes de travail extérieurs. Ces engagements ont été l'occasion de contribuer à des actions transverses, en lien soit avec la structuration des recherches en SMAs, soit avec l'organisation de la vie scientifique locale.

\subsection*{Implication dans le groupe international AICA}

Depuis 2023, je suis trésorier du groupe international AICA (Autonomous Intelligent Cyber Agents), qui rassemble des chercheurs et ingénieurs intéressés par les questions liées aux agents autonomes appliqués à la défense et à la cybersécurité. Mon rôle consiste principalement à assurer le suivi des aspects administratifs : gestion des adhésions, échanges avec les membres, mise à jour des informations financières et participation aux réunions de pilotage du groupe. Cette activité demande une coordination régulière avec les autres membres du bureau et offre un point de vue intéressant sur l'évolution des travaux menés dans ce domaine.

\subsection*{Participation à la vie du laboratoire}

J'ai également pris part, de manière régulière, aux activités organisées au LCIS. Cela inclut la participation aux séminaires internes, la présentation de l'avancement des travaux de thèse lors de réunions d'équipe, et l'implication ponctuelle dans des discussions autour de projets en cours. Ces échanges ont joué un rôle important dans la maturation des idées développées pendant la thèse, en permettant de confronter les approches à celles d'autres chercheurs du laboratoire.

\subsection*{Organisation ou soutien à des événements}

J'ai contribué à la préparation ou à l'animation de plusieurs activités en lien avec la diffusion des travaux du laboratoire ou avec des événements internes. Cela comprend :
\begin{itemize}
    \item la participation à des démonstrations lors de visites institutionnelles ou de présentations destinées à des partenaires ;
    \item la préparation de supports ou d'exemples pour des séminaires internes ;
    \item un soutien ponctuel à l'organisation de journées d'échange ou de réunions techniques avec les partenaires industriels.
\end{itemize}

Ces activités restent modestes, mais elles font partie du fonctionnement quotidien du laboratoire et constituent une bonne occasion d'échanger avec des collègues ou des visiteurs sur les travaux en cours.


\section*{Vulgarisation et diffusion}
% =======================================================

Au-delà des activités de recherche et d'enseignement, j'ai eu l'occasion de participer à différentes actions de diffusion, aussi bien au sein du laboratoire que dans des contextes plus larges. Ces interventions, souvent ponctuelles, ont surtout consisté à présenter l'avancement des travaux de thèse ou à contribuer à des échanges visant à rendre accessibles certains aspects techniques liés aux SMAs et à la cybersécurité.

\subsection*{Présentations et interventions}

J'ai présenté les travaux de la thèse à plusieurs occasions, notamment dans le cadre d'une intervention auprès de la chaire CybAIR de l'OTAN en 2023. Cette présentation visait à introduire les enjeux liés à l'automatisation de la défense dans des environnements distribués, ainsi que les approches multi-agents permettant de structurer ou d'analyser ces comportements.
Des présentations plus courtes ont également été réalisées au LCIS ou auprès de partenaires industriels, généralement pour exposer une méthode, clarifier un choix technique ou partager des résultats intermédiaires.

\subsection*{Posters et démonstrations}

Durant la thèse, j'ai participé à l'élaboration de posters pour différents événements scientifiques, notamment lors des JFSMA. Ces supports présentaient de manière synthétique certains éléments méthodologiques (comme l'analyse de comportements) ou des résultats obtenus dans des environnements simulés.
Dans certains cas, j'ai aussi contribué à des démonstrations simples destinées à illustrer le fonctionnement d'un environnement ou d'un module logiciel auprès de visiteurs du laboratoire ou de partenaires extérieurs.

\subsection*{Communications internes}

Enfin, j'ai pris part à plusieurs échanges internes liés à la diffusion des travaux au sein du laboratoire ou auprès de Thales. Cela comprend la rédaction de petits comptes rendus techniques, la mise à disposition d'exemples d'utilisation d'outils développés pendant la thèse, ainsi que la préparation de supports destinés à faciliter la prise en main de certains modules.
Ces activités restent modestes, mais elles contribuent au partage des connaissances et à la valorisation des outils produits au cours de la thèse.



\section*{Projet de recherche}
% =======================================================

Le travail réalisé pendant la thèse a ouvert plusieurs pistes que j'aimerais poursuivre dans les années qui viennent, en continuité avec les problématiques d'apprentissage multi-agent, de modélisation organisationnelle et de cybersécurité. Les objectifs présentés ci-dessous ne constituent pas un programme figé, mais plutôt une trajectoire cohérente qui prolonge les contributions développées au cours du doctorat et qui pourrait s'inscrire dans les activités d'un laboratoire travaillant sur les systèmes intelligents distribués.

\subsection*{Vision à court terme (1 an)}

À court terme, je souhaite approfondir certains aspects restés partiellement explorés pendant la thèse, en particulier la manière dont les contraintes organisationnelles influencent l'apprentissage multi-agent dans des environnements plus variés que ceux étudiés jusqu'ici. Il s'agirait notamment :
\begin{itemize}
    \item d'évaluer l'approche MOISE+ intégrée au MARL sur des scénarios moins structurés ou plus dynamiques ;
    \item de consolider les outils d'analyse existants, comme Auto-TEMM, afin d'améliorer la précision de l'extraction de comportements ;
    \item de poursuivre le développement d'environnements de simulation permettant de tester différentes formes de coordination ou de supervision.
\end{itemize}
Ce travail viserait avant tout à stabiliser la plateforme développée (CybMASDE) et à disposer d'un socle expérimental robuste pour des études plus ambitieuses.

\subsection*{Vision à moyen terme (3 ans)}

Sur un horizon de trois ans, j'aimerais explorer de manière plus systématique le lien entre organisation explicite et comportements appris. Plusieurs pistes me semblent particulièrement prometteuses :
\begin{itemize}
    \item l'étude de modèles organisationnels capables de s'adapter au cours de l'apprentissage, de manière à mieux refléter la dynamique des tâches ou des situations ;
    \item l'intégration de mécanismes de sûreté ou de contraintes opérationnelles directement dans les politiques apprises, afin d'éviter des comportements inutiles ou risqués ;
    \item l'analyse des comportements collectifs dans des environnements où les interactions sont moins symétriques (par exemple des équipes hybrides ou hétérogènes).
\end{itemize}
Ces perspectives permettraient de rapprocher davantage les approches multi-agents des problématiques rencontrées dans des systèmes opérationnels, notamment en matière d'explicabilité et de contrôle.

\subsection*{Vision à long terme (5 ans)}

À plus long terme, je souhaiterais étendre les travaux menés en Cyberdéfense à d'autres domaines où des agents doivent coopérer ou s'adapter dans des contextes incertains. La robotique multi-agent constitue une application naturelle, car elle partage de nombreux points communs avec la défense distribuée : coordination, observabilité partielle, nécessité de comportements sûrs et explicites.
Je souhaiterais également approfondir les liens entre apprentissage multi-agent et modélisation de systèmes cyber-physiques, en étudiant comment des représentations apprises (world models) peuvent aider à anticiper et comprendre les effets de certaines actions dans des environnements complexes.

Enfin, je m'intéresse aux questions de transfert du comportement appris, qu'il s'agisse de passer d'un environnement simulé à un système réel ou de transposer des comportements d'un domaine à un autre. Ces questions restent centrales pour la mise en œuvre concrète des approches multi-agents et constituent un axe de recherche important pour les années à venir.

\subsection*{Applications visées}

Les applications envisagées concernent principalement les systèmes autonomes opérant dans des environnements critiques :
\begin{itemize}
    \item \textbf{Cyberdéfense et systèmes d'information sensibles} : automatisation de la détection et de la réponse, analyse des comportements hostiles ;
    \item \textbf{robotique multi-agent} : coordination d'équipes hétérogènes, exécution de tâches collectives ;
    \item \textbf{systèmes cyber-physiques} : surveillance, gestion d'incidents, contrôle distribué ;
    \item \textbf{infrastructures critiques} : interactions entre agents opérateurs, modèles prédictifs pour l'anticipation de situations.
\end{itemize}

L'objectif global est de poursuivre une recherche située à la frontière entre modèles formels, apprentissage et simulation, avec un intérêt particulier pour l'explicabilité et la robustesse des comportements collectifs.

\section*{Annexes}
% =======================================================

Les documents listés ci-dessous complètent les informations présentées dans le rapport. Ils rassemblent les éléments administratifs, pédagogiques et techniques nécessaires à l'évaluation du dossier.

\begin{itemize}
    \item \textbf{Attestations d'enseignement} : copies des documents fournis par l'Esisar et l'IUT de Valence, indiquant les volumes horaires et les modules assurés au cours des différentes années universitaires.

    \item \textbf{Tableaux récapitulatifs} : synthèse des heures d'enseignement sous forme d'équivalents TD, accompagnée des détails par établissement et par type de séance (TD, TP).

    \item \textbf{Liens vers les dépôts logiciels} : accès aux principaux dépôts utilisés ou développés pendant la thèse, en particulier ceux contenant la plateforme CybMASDE, les environnements de simulation et les modules expérimentaux associés.

    \item \textbf{Éléments complémentaires de modélisation} : schémas et diagrammes utilisés dans le cadre de la thèse, incluant les représentations liées aux modèles organisationnels (MOISE+), à la méthodologie MAMAD et aux architectures logicielles mises en place.
\end{itemize}

\end{document}
