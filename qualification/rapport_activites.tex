\documentclass[11pt,a4paper]{article}

% =======================================
% PACKAGES
% =======================================
\usepackage[utf8]{inputenc}
\usepackage[T1]{fontenc}
\usepackage{lmodern}
\usepackage[french]{babel}
\usepackage{geometry}
\geometry{margin=2.2cm}
\usepackage{setspace}
\setstretch{1.15}
\usepackage{enumitem}
\usepackage{hyperref}
\hypersetup{colorlinks=true,linkcolor=blue,urlcolor=blue}
\usepackage{booktabs}
\usepackage{array}
\usepackage{csquotes}
\usepackage{multicol}
\usepackage{longtable}
\usepackage{pdflscape}
\usepackage{biblatex}
\usepackage{caption}
\captionsetup{width=\textwidth}
\usepackage{xcolor}
\definecolor{SectionColor}{RGB}{140,53,53}
\definecolor{SubSectionColor}{RGB}{140,53,53}
\usepackage{titlesec}
\usepackage[normalem]{ulem}

% \titleformat{\section}
%   {\normalfont\Large\bfseries\color{SectionColor}}
%   {}{0pt}{#1\par\vspace{0.2em}\titlerule\vspace{0.8em}}



\titleformat{\section}{\normalfont\Large\bfseries\color{SectionColor}}{\arabic{section}.}{}{}[\vspace{-1em}\textcolor{black}{\titlerule}]

\titleformat{\subsection}
  {\normalfont\large\bfseries\color{SubSectionColor}}{}{0pt}{}
  
\titleformat{\subsubsection}
  {\normalfont\normalsize\bfseries\color{SubSectionColor}}{}{0pt}{}

\setlength{\parskip}{6pt}

\usepackage{fancyhdr}
\usepackage{lastpage}

\pagestyle{fancy}
\fancyhf{} % Efface les entêtes/pieds par défaut

% Texte du header
\lhead{\small{Julien Soulé}}
\rhead{\small{\today}}

\fancyfoot[C]{\textit{Page \thepage{} sur 15}}

% Optionnel : ligne en haut ou bas
\renewcommand{\headrulewidth}{0pt}
\renewcommand{\footrulewidth}{0pt}

\addbibresource{references.bib}
% If you want to break on URL numbers
\setcounter{biburlnumpenalty}{9000}
% If you want to break on URL lower case letters
\setcounter{biburllcpenalty}{9000}
% If you want to break on URL UPPER CASE letters
\setcounter{biburlucpenalty}{9000}



% === Sorting by year (descending), then author ===
\ExecuteBibliographyOptions{sorting=ydnt}

% =======================================
\begin{document}

\begin{center}
    {\Large\textbf{Rapport d'activités}}\\[4pt]
    {\large Julien Soulé}\\[4pt]
    Candidature à la qualification 2025/2026 au corps de Maître de conférences, section 27\\[4pt]
    \vspace{0.5cm}
\end{center}


\section*{Identité}
% =======================================================

\begin{itemize}[leftmargin=0.1em, itemsep=-2pt, topsep=0pt]
    \item[] \textbf{Nom} : Soulé
    \item[] \textbf{Prénom} : Julien
    \item[] \textbf{Adresse personnelle} : 35 Rue Mathieu de la Drôme, 26000 Valence, France
    \item[] \textbf{Téléphone personnel} : +33 (0) 6 77 63 12 13
    \item[] \textbf{Courriel} : \href{mailto:julien.soule@hotmail.fr}{julien.soule@hotmail.fr}
    \item[] \textbf{Pages scientifiques} :
        \begin{itemize}[leftmargin=*, itemsep=-2pt, topsep=0pt]
            \item Portfolio : \url{https://julien6.github.io/home/}
            \item HAL : \url{https://arxiv.org/a/soule_j_1.html}
            \item ORCID : \url{https://orcid.org/0009-0002-3218-2614}
            \item DBLP : \url{https://dblp.org/pid/367/9947.html}
            \item GitHub : \url{https://github.com/julien6}
        \end{itemize}
\end{itemize}


% =======================================================
\section*{Parcours universitaire}
% =======================================================

\subsection*{Doctorat}

\begin{itemize}[leftmargin=0.1em, itemsep=-2pt, topsep=0pt]
    \item[] \textbf{Intitulé} : Doctorat en informatique
    \item[] \textbf{Dates} : 1er octobre 2022 -- 17 novembre 2025
    \item[] \textbf{Nom du laboratoire} : Laboratoire de Conception et d'Intégration des Systèmes (LCIS)
    \item[] \textbf{Etablissement d'accueil} : Grenoble-INP -- UGA
    \item[] \textbf{Spécialités} : Système Multi-Agents (thématique 86) et Cyberdéfense (thématique E4)
    \item[] \textbf{Encadrement} :
        \begin{itemize}[leftmargin=*, itemsep=-2pt, topsep=0pt]
            \item \textbf{Jean-Paul Jamont} (Professeur, UGA, LCIS) -- Directeur de thèse
            \item \textbf{Michel Occello} (Professeur, UGA, LCIS) -- Co-directeur de thèse
            \item \textbf{Louis-Marie Traonouez} (Ingénieur IA, Thales LAS) -- Accompagnement industriel
            \item \textbf{Paul Théron} (Chercheur, AICA IWG) -- Accompagnement scientifique
        \end{itemize}
    \item[] \textbf{Rapporteurs} :
        \begin{itemize}[leftmargin=*, itemsep=-2pt, topsep=0pt]
            \item \textbf{Laurent Vercouter} -- Professeur, INSA Rouen, LITIS
            \item \textbf{Gauthier Picard} -- Directeur de recherche, ONERA
        \end{itemize}
    \item[] \textbf{Examinateurs} :
        \begin{itemize}[leftmargin=*, itemsep=-2pt, topsep=0pt]
            \item \textbf{Oum-El-Kheir Aktouf} -- Professeure, UGA, LCIS
            \item \textbf{Aurélie Beynier} -- Professeure, Sorbonne Université
            \item \textbf{Flavien Balbo} -- Professeur, École des Mines de Saint-Étienne
            \item \textbf{Laeticia Matignon} -- Maîtresse de Conférences, Université Claude Bernard Lyon
        \end{itemize}
\end{itemize}

\noindent
J'ai entrepris une thèse en informatique au sein du LCIS, rattaché à Grenoble-INP -- UGA, entre 2022 et 2025. Ce travail s'inscrit dans une convention CIFRE (Convention Industrielle de Formation par la Recherche) avec Thales LAS, ce qui m'a permis d'évoluer à la fois dans un environnement académique et dans un contexte industriel fortement orienté vers les problématiques de Cyberdéfense.
%
Ma thèse, intitulée \textit{De l'Organisation d'un Système Multi-Agent de Cyberdéfense}, présentée le 17~novembre 2025, porte sur la conception de Systèmes Multi-Agents (SMA) dédiés à la détection et à la réponse à des comportements adversariaux dans des environnements distribués. Deux journaux, quatre conférences internationales et six conférences nationales ont été produits au cours de ces travaux.

% Le jury était composé de :

% \begin{itemize}
%     \item \textbf{Laurent Vercouter} (Professeur, INSA Rouen, LITIS) -- Rapporteur
%     \item \textbf{Gauthier Picard} (Directeur de recherche, ONERA) -- Rapporteur
%     \item \textbf{Jean-Paul Jamont} (Professeur, UGA, LCIS) -- Président du jury
%     \item \textbf{Michel Occello} (Professeur, UGA) -- Examinateur
%     \item \textbf{Louis-Marie Traonouez} (Ingénieur de recherche, Thales LAS) -- Examinateur
% \end{itemize}

% \vspace{0.4cm}

\subsection*{Diplôme d'ingénieur}

\begin{itemize}[leftmargin=0.1em, itemsep=-2pt, topsep=0pt]
    \item[] \textbf{Intitulé} : Diplome d'ingénieur -- Grade de Master
    \item[] \textbf{Dates} : 1er septembre 2015 -- 8 janvier novembre 2021
    \item[] \textbf{Etablissement d'accueil} : Institut National des Sciences Appliquées (INSA) de Rennes
    \item[] \textbf{Spécialités} : Intelligence artificielle (thématique 80) et Cyberdéfense (thématique E4)
\end{itemize}

Avant la thèse, j'ai obtenu le diplôme d'ingénieur en informatique de l'INSA Rennes (promotion 2020). Cette formation m'a permis d'acquérir des bases solides en génie logiciel, en programmation système, en réseaux et en sécurité, ainsi qu'une première exposition aux approches multi-agents.

Dans le cadre de mon cursus, j'ai également effectué un semestre d'études à l'École de Technologie Supérieure (Montréal, Canada), entre janvier et mai 2019. Cette expérience m'a offert l'occasion de suivre de m'ouvrir à des pratiques pédagogiques et organisationnelles du système nord-américain.

Au-delà du cursus ingénieur, j'ai pu me former à divers outils et méthodes liés aux systèmes distribués, au développement logiciel et à la Cybersécurité, principalement au cours de projets, de stages et d'expériences professionnelles préalables à la thèse.


\section*{Parcours professionnel}
% =======================================================

\subsection*{Doctorant CIFRE / Ingénieur de recherche (2022--2025)}

\begin{itemize}[leftmargin=0.1em, itemsep=-2pt, topsep=0pt]
    \item[] \textbf{Intitulé} : Ingénieur de recherche doctorant en informatique
    \item[] \textbf{Dates} : Juillet 2022 -- Octobre 2025
    \item[] \textbf{Établissements} :
        \begin{itemize}[leftmargin=*, itemsep=-2pt, topsep=0pt]
            \item Laboratoire LCIS, Université Grenoble Alpes, Valence, France
            \item \textit{Thales Land} \& \textit{Air Systems}, Rennes, France
        \end{itemize}
    \item[] \textbf{Statut} : Salarié en convention CIFRE (CDD de 3 ans)
\end{itemize}

\noindent
J'ai réalisé mon doctorat au sein du LCIS dans le cadre d'une convention CIFRE avec Thales LAS. Le temps était réparti avec 70\% du temps sur le site du LCIS pour la recherche et 30\% chez Thales LAS (sur le site de Rennes dit \textquote{La Ruche}) pour la collaboration industrielle. Cette position, à la fois académique et industrielle, m'a conduit à travailler sur des problématiques liées aux SMA, à la Cyberdéfense, à l'apprentissage par renforcement multi-agent (Multi-Agent Reinforcement Learning -- MARL), à l'apprentissage par renforcement (Reinforcement Learning -- RL) et la modélisation de systèmes complexes en simulation. Au sein de Thales, j'ai participé à la conception et à l'expérimentation de modules de simulation et d'analyse destinés à des scénarios opérationnels.

\subsection*{Vacataire d'enseignement en informatique}

\begin{itemize}[leftmargin=0.1em, itemsep=-2pt, topsep=0pt]
    \item[] \textbf{Intitulé} Enseignant vacataire en informatique
    \item[] \textbf{Dates} Années scolaires 2022--2023, 2023--2024, 2024--2025
    \item[] \textbf{Établissements}
        \begin{itemize}[leftmargin=*, itemsep=-2pt, topsep=0pt]
            \item Grenoble-INP -- Esisar, Valence, France
            \item IUT, Valence, France
        \end{itemize}
    \item[] \textbf{Statut} Vacataire
\end{itemize}

\noindent
Durant mon doctorat, j'ai également assuré des enseignements en tant que vacataire Esisar et à l'IUT de Valence. A l'école d'ingénieur Grenoble-INP -- Esisar, j'ai enseigné pour un total de \textbf{49.50 hEqTD} au cours de séances de Travaux Pratiques (TP), de séances de Travaux Dirigés (TD) pour des étudiants ingénieurs de niveau variant entre première et cinquième années. À l'IUT de Valence, pour un total de \textbf{48 hEqTD}, j'ai enseigné des séances de TP pour des étudiants en Bachelor Universitaire de Technologie (BUT) de première et deuxième années, principalement sur des thématiques liées aux systèmes d'exploitation (61), à la programmation système (75) et à la structuration de l'information (33).

\subsection*{Ingénieur R\&D -- Thales LAS (2021--2022)}

\begin{itemize}[leftmargin=0.1em, itemsep=-2pt, topsep=0pt]
    \item[] \textbf{Intitulé} Ingénieur R\&D en informatique
    \item[] \textbf{Dates} 1er Décembre 2021 -- Juin 2022
    \item[] \textbf{Établissement} Thales LAS, France
    \item[] \textbf{Statut} Salarié CDD (7 mois)
\end{itemize}

\noindent
Avant le début du doctorat, j'ai occupé un poste d'ingénieur R\&D au sein de Thales LAS, travaillant sur la modélisation de réseaux et la détection d'anomalies en Cybersécurité. Cette expérience m'a permis d'acquérir une connaissance des technologies utilisées en défense et de contribuer au développement d'outils de simulation multi-agent pour la validation de stratégies de détection d'intrusions.

\subsection*{Ingénieur logiciel -- Atos / CNES (2020--2021)}

\begin{itemize}[leftmargin=0.1em, itemsep=-2pt, topsep=0pt]
    \item[] \textbf{Intitulé} Ingénieur développement
    \item[] \textbf{Dates} 17 août 2020 -- 1er octobre 2021
    \item[] \textbf{Établissement} Atos, Toulouse, France
    \item[] \textbf{Statut} Salarié CDI (1 an 8 mois)
\end{itemize}

\noindent
J'ai été salarié en CDI chez Atos sur la poursuite du projet \textbf{ISIS} (\textit{Initiative for Space Innovation Standard}) du \textbf{CNES} (Centre national d'études spatiales), dédié au contrôle et à la surveillance des infrastructures au sol. Mon travail portait sur la conception d'outils de supervision, des scripts d'automatisation et des interfaces d'analyse.
Cette expérience m'a donné un aperçu du fonctionnement de systèmes critiques à grande échelle, ainsi que de l'importance des outils de diagnostic et de remontée d'information dans des environnements où la continuité de service est essentielle.

\subsection*{Stage Projet de fin d'études}

\begin{itemize}[leftmargin=0.1em, itemsep=-2pt, topsep=0pt]
    \item[] \textbf{Intitulé} Ingénieur de développement stagiaire
    \item[] \textbf{Dates} 17 février 2020 -- 14 août 2020
    \item[] \textbf{Établissement} Atos, Toulouse, France
    \item[] \textbf{Statut} Salarié CDD (6 mois)
\end{itemize}

\noindent
Dans le cadre de mon projet de fin d'études à l'INSA Rennes, j'ai effectué un stage de six mois chez Atos, travaillant sur le projet ISIS du CNES. J'ai contribué au développement d'outils de supervision et d'analyse pour les infrastructures au sol, en mettant l'accent sur la robustesse et la facilité d'utilisation des interfaces. Ce stage m'a permis de mettre en pratique mes compétences en programmation et en gestion de projets dans un contexte industriel exigeant.

\subsection*{Stage ingénieur}

\begin{itemize}[leftmargin=0.1em, itemsep=-2pt, topsep=0pt]
    \item[] \textbf{Intitulé} Ingénieur stagiaire
    \item[] \textbf{Dates} 1er mai 2019 -- 31 juillet
    \item[] \textbf{Établissement} SQLI, Toulouse, France
    \item[] \textbf{Statut} Salarié CDD (3 mois)
\end{itemize}

\noindent
Durant l'été 2019, j'ai effectué un stage de trois mois chez SQLI\footnote{\url{https://www.sqli.com}} à Toulouse, où j'ai participé au développement d'une application web de gestion de contenu pour des projets \textit{Airbus Helicopters}. Ce stage m'a permis de renforcer mes compétences en développement web, en gestion de bases de données et en travail en équipe dans un contexte professionnel.


\section*{Activités d'enseignement}
% =======================================================

\subsection*{Activités d'enseignement réalisées}

\noindent
Dès le début de ma thèse, j'ai assuré différents enseignements au sein de l'école d'ingénieur Grenoble-INP -- Esisar et de l'IUT de Valence. Ces interventions ont concerné plusieurs niveaux, allant du premier cycle universitaire aux années de formation d'ingénieur. Elles m'ont permis d'acquérir une expérience variée de l'enseignement, en particulier dans des domaines liés aux systèmes, à la programmation et à la Cybersécurité.

Le \autoref{tab:enseignement-esisar} et le \autoref{tab:enseignement-iut} reprennent les informations concernant les enseignements que j'ai réalisés respectivement à l'école d'ingénieur Grenoble-INP -- Esisar et à l'IUT de Valence. Les volumes horaires sont données en hEqTD et les effectifs en nombre d'étudiants, le produit $n \times x$ signifie la responsabilité de $x$ groupes de $n$ étudiants. Le volume horaire total effectué est présentée dans le \autoref{tab:enseignement-total}.
Les valeurs suivantes correspondent aux heures attestées sur la période 2022--2025. Elles incluent à la fois des enseignements en école d'ingénieur et en IUT par \textbf{Nature}. Les années d'exercice correspondent aux années scolaires: \textbf{TD} : 19,5h (Grenoble-INP -- Esisar) ; \textbf{TP} : 30h (Grenoble-INP -- Esisar) + 18h (IUT).

\subsubsection*{Philosophie pédagogique}

Les trois années de vacations m'ont permis de développer une approche pédagogique centrée sur l'autonomie et la pratique. Je privilégie les activités progressives reliant les concepts théoriques à des situations réelles, en particulier pour les systèmes et la Cybersécurité. En TD, je débute par un rappel structuré avant de passer aux exercices. L'évaluation continue permet aux étudiants d'identifier rapidement leurs lacunes et d'ajuster leur méthode de travail. Je fournis des supports synthétiques avec exemples concrets pour faciliter la compréhension de notions abstraites.

\subsubsection*{Descriptions qualitatives}

Les enseignements assurés couvrent plusieurs blocs thématiques : systèmes d'exploitation et programmation système (gestion des processus, mémoire, appels système et synchronisation), ainsi que Cyberdéfense et structuration de l'information (analyse de scénarios d'attaque, détection d'anomalies et stratégies défensives).

J'ai également participé à différents enseignements : projets de première et deuxième année en école d'ingénieur (programmation C, Cybersécurité), accompagnement d'étudiants en projets d'analyse ou simulation, suivi d'avancement et participation aux évaluations orales. Ces enseignements permettent d'aborder les difficultés dans un cadre moins formel que les cours et de discuter des méthodes de travail et stratégies de résolution de problèmes.

\subsection*{Projet d'enseignement}

Fort de mon parcours à l’interface entre SMA, MARL, et Cyberdéfense, je me sens prêt à assurer des enseignements dans ces domaines, auprès de publics variés allant du premier cycle universitaire aux élèves ingénieurs et masters. Je peux intervenir sur les thématiques suivantes :
\begin{itemize}
    \item \textbf{Fondamentaux} : algorithmique, structures de données, programmation (C, Java, Python), mathématiques (niveau licence) ;
    \item \textbf{SMA} : fondements, modélisation organisationnelle, applications (niveau licence, master, école d’ingénieur) ;
    \item \textbf{RL et MARL} : principes, algorithmes, applications à la robotique et à la cybersécurité ;
    \item \textbf{Cybersécurité} : introduction, analyse de scénarios, détection d’anomalies, sécurité des systèmes distribués ;
    \item \textbf{Programmation système et réseaux} : gestion des processus, mémoire, synchronisation, protocoles ;
    \item \textbf{Projets pratiques et études de cas} : conception et encadrement de mini-projets, TD et TP, intégrant des environnements simulés ou réels.
\end{itemize}

Mon expérience d’enseignement, acquise en école d’ingénieur, en IUT (ainsi qu'au sein de Thales), me permet d’adapter les contenus et méthodes pédagogiques à différents niveaux et profils d’étudiants. Par exemple, en IUT, j'insiste davantage sur les applications pratiques et les compétences techniques essentielles, tandis qu'en école d'ingénieur, je peux approfondir les concepts théoriques et encourager des projets plus complexes et innovants. Cette approche différenciée garantit que chaque étudiant, quel que soit son parcours, puisse tirer le meilleur parti de l'enseignement dispensé.

Je souhaite également contribuer à la création de modules transversaux reliant systèmes, IA distribuée et sécurité informatique, et à l’intégration d’activités pratiques avancées pour favoriser l’autonomie et l’acquisition de compétences appliquées. Par exemple, un module pourrait inclure un projet où les étudiants développent un système multi-agent capable de détecter et de répondre à des cybermenaces en temps réel, en utilisant des techniques d'apprentissage automatique pour améliorer la prise de décision des agents.

\begin{longtable}{p{6cm} p{4cm} p{4cm}}
    \caption{Résumé du volume total d'heures d'enseignement (2022--2025)}
    \label{tab:enseignement-total}                                  \\[0.3cm]
    \toprule
    \textbf{Nature}   & \textbf{Volume (h)} & \textbf{Volume eqTD}  \\
    \midrule
    \endfirsthead

    \multicolumn{3}{c}{{\small\itshape Table \thetable{} -- suite}} \\
    \toprule
    \textbf{Nature}   & \textbf{Volume (h)} & \textbf{Volume eqTD}  \\
    \midrule
    \endhead

    \bottomrule
    \endfoot

    \textbf{Total TD} & 41.5 h              & 41.5 hEqTD            \\
    \midrule
    \textbf{Total TP} & 46.5 h              & 46.5 hEqTD            \\
\end{longtable}

\begin{landscape}

    \setlength{\LTcapwidth}{\linewidth}


    \begin{longtable}{%
            p{2cm}
            p{1.3cm}
            p{3.5cm}
            p{2cm}
            p{1.2cm}
            p{1.5cm}
            p{1.5cm}
            p{5cm}
            p{3.5cm}
        }
        \caption{\textbf{Enseignements en tant que \textit{vacataire}} (\textbf{Statut}) à l'école d'ingénieur Grenoble-INP -- Esisar (\textbf{Établissement}), devant un public d'étudiants ingénieurs (\textbf{Public}).}
        \label{tab:enseignement-esisar}                                                                                                        \\[0.3cm]
        \toprule
        \textbf{Année d'exercice}                                                                                                            &
        \textbf{Niveau}                                                                                                                      &
        \textbf{Nom de la matière}                                                                                                           &
        \textbf{Volume horaire}                                                                                                              &
        \textbf{Vol. eq TD}                                                                                                                  &
        \textbf{Effectifs}                                                                                                                   &
        \textbf{Nature}                                                                                                                      &
        \textbf{Responsabilités}                                                                                                             &
        \textbf{Références}                                                                                                                    \\
        \midrule
        \endfirsthead

        \multicolumn{9}{c}{{\small\itshape Table \thetable{} -- suite}}                                                                        \\
        \toprule
        \textbf{Année d'exercice}                                                                                                            &
        \textbf{Niveau}                                                                                                                      &
        \textbf{Nom de la matière}                                                                                                           &
        \textbf{Volume horaire}                                                                                                              &
        \textbf{Vol. eq TD}                                                                                                                  &
        \textbf{Effectifs}                                                                                                                   &
        \textbf{Nature}                                                                                                                      &
        \textbf{Responsabilités}                                                                                                             &
        \textbf{Références}                                                                                                                    \\
        \midrule
        \endhead

        \bottomrule
        \endfoot

        2024--2025                                                                                                                           &
        1A                                                                                                                                   &
        3AMIN333 : Systèmes d’exploitation                                                                                                   &
        1.75 h TD                                                                                                                            &
        1.75                                                                                                                                 &
        $\sim 20$                                                                                                                            &
        TD                                                                                                                                   &
        Enseignement de séances de TD à partir de supports existants, accompagnement des étudiants et participation à l’évaluation.          &
        Attestation Esisar                                                                                                                     \\

        \midrule

        2024--2025                                                                                                                           &
        1A                                                                                                                                   &
        3AMIN333 : Systèmes d’exploitation                                                                                                   &
        15.75 h TD                                                                                                                           &
        15.75                                                                                                                                &
        $\sim 25$                                                                                                                            &
        TD                                                                                                                                   &
        Enseignement au cours de séances de TD pour étudiants ingénieurs, aide méthodologique et suivi des acquis.                           &
        Attestation Esisar                                                                                                                     \\

        \midrule

        2024--2025                                                                                                                           &
        1A                                                                                                                                   &
        3AMIN333 : Systèmes d’exploitation                                                                                                   &
        21 h TP                                                                                                                              &
        21                                                                                                                                   &
        $\sim 25$                                                                                                                            &
        TP                                                                                                                                   &
        Enseignement au cours de séances de TP (programmation système, processus, mémoire), accompagnement technique et évaluation continue. &
        Attestation Esisar                                                                                                                     \\

        \midrule

        2023--2024                                                                                                                           &
        3A                                                                                                                                   &
        3AMOS302 : Systèmes d’exploitation                                                                                                   &
        9 h TP                                                                                                                               &
        9                                                                                                                                    &
        $\sim 20$                                                                                                                            &
        TP                                                                                                                                   &
        Enseignement au cours de séances de TP pour étudiants ingénieurs de 3e année, assistance technique et validation des livrables.      &
        Attestation Esisar                                                                                                                     \\

        \midrule

        2023--2024                                                                                                                           &
        5A                                                                                                                                   &
        5AMPX511 : Conférences et veille technologique en cybersécurité                                                                      &
        2 h TD                                                                                                                               &
        2                                                                                                                                    &
        $\sim 15$                                                                                                                            &
        TD                                                                                                                                   &
        Animation de séances de veille technologique, accompagnement des présentations et discussions autour de thématiques cybersécurité.   &
        Attestation Esisar                                                                                                                     \\
    \end{longtable}


    \begin{longtable}{%
            p{2cm}
            p{1.3cm}
            p{3.5cm}
            p{2cm}
            p{1.2cm}
            p{1.5cm}
            p{1.5cm}
            p{5cm}
            p{3.5cm}
        }
        \caption{\textbf{Enseignements en tant que \textit{vacataire}} (Statut) à l'IUT de Valence (\textbf{Établissement}), devant un public d'étudiants en BUT (\textbf{Public}).}
        \label{tab:enseignement-iut}                                                                                                                                         \\[0.3cm]
        \toprule
        \textbf{Année d'exercice}                                                                                                                                          &
        \textbf{Niveau}                                                                                                                                                    &
        \textbf{Nom de la matière}                                                                                                                                         &
        \textbf{Volume horaire}                                                                                                                                            &
        \textbf{Vol. eq TD}                                                                                                                                                &
        \textbf{Effectifs}                                                                                                                                                 &
        \textbf{Nature}                                                                                                                                                    &
        \textbf{Responsabilités}                                                                                                                                           &
        \textbf{Références}                                                                                                                                                  \\
        \midrule
        \endfirsthead

        \multicolumn{9}{c}{{\small\itshape Table \thetable{} -- suite}}                                                                                                      \\
        \toprule
        \textbf{Année d'exercice}                                                                                                                                          &
        \textbf{Niveau}                                                                                                                                                    &
        \textbf{Nom de la matière}                                                                                                                                         &
        \textbf{Volume horaire}                                                                                                                                            &
        \textbf{Vol. eq TD}                                                                                                                                                &
        \textbf{Effectifs}                                                                                                                                                 &
        \textbf{Nature}                                                                                                                                                    &
        \textbf{Responsabilités}                                                                                                                                           &
        \textbf{Références}                                                                                                                                                  \\
        \midrule
        \endhead

        \bottomrule
        \endfoot

        2022--2023                                                                                                                                                         &
        BUT 1A                                                                                                                                                             &
        Administration système et fondamentaux de la virtualisation                                                                                                        &
        7.5 h eqTD                                                                                                                                                         &
        7.5                                                                                                                                                                &
        $\sim 20$                                                                                                                                                          &
        TD/TP                                                                                                                                                              &
        Enseignement au cours de séances de TP portant sur l’administration système et la virtualisation, accompagnement technique des étudiants et évaluation des acquis. &
        Attestation IUT Valence                                                                                                                                              \\

        \midrule

        2024--2025                                                                                                                                                         &
        BUT 1A                                                                                                                                                             &
        Structuration et traitement de l’information                                                                                                                       &
        24 h eqTD                                                                                                                                                          &
        24                                                                                                                                                                 &
        $\sim 30$                                                                                                                                                          &
        TD/TP                                                                                                                                                              &
        Enseignement au cours de séances de TD et TP sur la structuration de données et le traitement de l’information, suivi pédagogique et évaluation continue.          &
        Attestation IUT Valence                                                                                                                                              \\

        \midrule

        2024--2025                                                                                                                                                         &
        BUT 2A                                                                                                                                                             &
        Programmation système                                                                                                                                              &
        16.5 h eqTD                                                                                                                                                        &
        16.5                                                                                                                                                               &
        $\sim 25$                                                                                                                                                          &
        TD/TP                                                                                                                                                              &
        Enseignement au cours de séances de TP et de TD en programmation système (processus, mémoire, appels système), aide méthodologique et validation des travaux.      &
        Attestation IUT Valence                                                                                                                                              \\
    \end{longtable}

\end{landscape}

\section*{Activités de recherche}
% =======================================================

% Décrire les activités de recherche, notamment :
%  - thématique, lieu et nature de chaque activité ;
%  - résultats théoriques ou méthodologiques, synthèses, expériences, mesures, évaluations ;
%  - éléments de valorisation des productions logicielles et/ou matérielles (développement de logiciels ou de matériels, dépôts de logiciels à l’APP, etc.) ;
%  - visibilité :
%      - organisation de colloques, formations spécialisées, etc.,
%      - séjours scientifiques dans d’autres laboratoires/établissements, coopérations éventuelles, etc.,
%      - participation à des comités de programmes/éditoriaux
%  - encadrements éventuels, en précisant la participation du·de la candidat·e dans cet encadrement, le pourcentage d’encadrement, la date, la durée, la thématique, les résultats obtenus :
%      - projets de fin d’études d’ingénieur·e, mémoires CNAM, mémoires de Master
%      - thèses, postdoctorant·e·s, ingénieur·e·s (PR)
%  - participation à des jurys de thèse (voire d’HDR, pour les candidat·e·s PR) en précisant la fonction assurée (président.e, rapporteur.e) ;
%  - activités de médiation ;
%  - activités de transfert.

\noindent
Mes activités de recherche se sont développées principalement dans le cadre de ma thèse CIFRE (\textbf{Lieux}) au \textbf{LCIS} (Grenoble-INP -- UGA, Valence) et chez \textbf{Thales Land and Air Systems} (\textquote{La Ruche}, Rennes). Elles se situent à l’interface entre SMA, apprentissage automatique (en particulier le MARL) et Cybersécurité. La nature des travaux est à la fois méthodologique (modélisation organisationnelle, définition de cadres d’analyse) et expérimentale (simulation, évaluation de comportements dans des scénarios de Cyberdéfense).

\subsubsection*{Thématiques et nature des activités}

\noindent
Les activités menées s’organisent autour de trois axes principaux :
\begin{itemize}
    \item \textbf{RL et MARL avec une approche neuro-symbolique} : conception et étude d'architectures permettant d'injecter des contraintes issues du modèle $\mathcal{M}OISE^+$ dans des algorithmes RL/MARL. L'intégration d'une représentation sub-symbolique ou symbolique de l'organisation vise à guider l'apprentissage vers des comportements plus coordonnés et stables, permettant notamment de mitiger les problèmes inhérents au MARL, en particulier la non-stationarité et l'attribution de crédit.
    \item \textbf{Analyse de comportements et d'organisations émergentes au sein de SMA} : développement de méthodes basées sur l'apprentissage non-supervisé pour extraire, représenter et interpréter les organisations effectivement adoptées par les agents après apprentissage selon le modèle organisationnel $\mathcal{M}OISE^+$.
    \item \textbf{Modélisation d'environnements en simulation} : exploration d'approches automatisées pour la modélisation d'environnements en simulation fidèles à l'environnement réel cible via des techniques comme les \textit{World Models}~\cite{ha2018recurrent} issues du domaine du MARL. L'application d'une approche neuro-symbolique sur ce type de modèle constitue une piste de recherche pouvant renforcer la pertinence et la fiabilité des tests effectués en simulation.
    \item \textbf{Applications sur scénarios de Cyberdéfense cyber-physiques et logiciels} : application et développement du cadre neuro-symbolique à des systèmes cyber-physiques, infrastructures logicielles et scénarios IoT/IoBT, traitant la détection de menaces, la coordination des agents et l'explicabilité des décisions dans des contextes opérationnels complexes.
\end{itemize}
Ces activités combinent modélisation formelle, implémentation logicielle et validation expérimentale sur des cas d’étude.

\subsubsection*{Résultats théoriques, méthodologiques et expérimentaux}

Sur le plan méthodologique, les travaux ont conduit à la définition d’un cadre de conception et d’évaluation des SMA de Cyberdéfense s’appuyant sur la méthodologie \textit{MAMAD} (MOISE+MARL Assisted MAS Design), qui structure les étapes de modélisation, d’apprentissage, d’analyse et de transfert. L’intégration de $\mathcal{M}OISE^+$ dans le MARL a donné lieu à la proposition de mécanismes d’influence (récompenses modulées, filtrage d’actions, contraintes de coordination) et à l’étude de leur impact sur la stabilité et l’explicabilité des comportements.

Sur le plan expérimental, plusieurs campagnes de simulation ont été réalisées dans des environnements coopératifs (ramassage et déplacement de ressources, scénarios de type usine-entrepôt) et dans des scénarios inspirés de la Cyberdéfense. Ces expériences ont permis de mesurer l’effet des contraintes organisationnelles sur la performance, de comparer différentes variantes d’architecture (avec ou sans modèle organisationnel explicite) et de caractériser les organisations émergentes au moyen du module \textit{Trajectory-based Evaluation in MOISE+MARL} (TEMM). Les résultats sont synthétisés dans un article de journal, quatre publications en conférences internationales et plusieurs contributions à des conférences nationales.

\subsubsection*{Valorisation logicielle et matérielle}

Les travaux se sont appuyés sur le développement de plusieurs outils logiciels : \textbf{MOISE+MARL}, une bibliothèque intégrant $\mathcal{M}OISE^+$ au MARL avec le module \textit{TEMM} pour l'analyse des organisations émergentes (\url{https://github.com/julien6/MOISE-MARL}) ; \textbf{CybMASDE}, la plateforme principale mettant en œuvre la méthodologie \textit{MAMAD} et regroupant modélisation organisationnelle, environnements de simulation et algorithmes d'apprentissage (\url{https://github.com/julien6/CybMASDE}) ; et des environnements coopératifs en grille dédiés aux expériences (déménagement de caisses, usine-entrepôt) accessibles via \url{https://github.com/julien6/OMARLE}. Les traces et données expérimentales (trajectoires, organisations extraites, paramètres d'apprentissage) sont conservées sous forme structurée pour permettre des analyses ultérieures et, lorsque cela est possible, une diffusion encadrée, à l'exception des éléments liés à des cas industriels qui restent soumis aux contraintes de confidentialité de Thales.

\subsubsection*{Visibilité, collaborations et animation scientifique}

Les résultats de ces travaux ont été diffusés à travers des publications dans un journal national à comité de lecture (ROIA) et dans des conférences internationales (AAMAS, IEEE CLOUD, IEEE SMC) et nationales (JFSMA, RESSI, RJCIA), des présentations en séminaires internes au LCIS, en réunions techniques chez Thales et auprès de la chaire CybAIR de l'OTAN, ainsi qu'une implication dans l'évaluation scientifique via des relectures pour AAMAS, ICAART, SMC, ICDL, JFSMA et RESSI. La collaboration LCIS--Thales s'est structurée par la convention CIFRE et des échanges réguliers (réunions techniques, journées de présentation, organisation d'une rencontre \textquote{Thales La Ruche -- LCIS autour du cyber}), facilitant l'alignement entre les besoins industriels et les contributions académiques.

\subsubsection*{Encadrements, médiation et transfert}

Pendant la thèse, j'ai contribué à l'encadrement de projets étudiants en école d'ingénieur et en IUT, principalement en co-encadrements ponctuels portant sur la programmation, l'apprentissage automatique, les systèmes et la Cybersécurité, en soutien aux responsables de modules pour l'orientation technique, le suivi d'avancement et l'évaluation. Je n'ai pas encore encadré de thèse ni participé à des jurys de thèse à ce stade de ma carrière. Mes activités de médiation et de transfert incluent des participations à des événements de vulgarisation (CyberWeek, Kaléidoscope, PhD Days) où j'ai présenté les travaux de thèse et les enjeux de la Cyberdéfense à des publics variés, ainsi que le développement de la plateforme CybMASDE et des environnements associés, qui contribuent à rapprocher les aspects méthodologiques des systèmes étudiés chez Thales en préparant le terrain pour une éventuelle intégration dans des chaînes de simulation ou d'analyse existantes.




\subsection*{Projet de recherche}
% =======================================================

Les travaux menés pendant la thèse ouvrent plusieurs perspectives que je souhaite poursuivre autour du RL/MARL, de la modélisation d'environnements complexes en simulation et de la cyberdéfense. À court terme, il s’agira surtout de consolider les outils développés, en particulier l’intégration de $\mathcal{M}OISE^+$ au MARL et les modules d’analyse comme \textit{TEMM}, ainsi que d’étendre les environnements d’expérimentation afin d’étudier des situations plus variées et dynamiques. Ce travail vise à stabiliser un socle méthodologique et logiciel capable de soutenir des études plus approfondies.

À moyen et long terme, mon objectif est d’examiner plus finement le lien entre organisation explicite et comportements appris, notamment dans des contextes nécessitant sûreté, coordination et explicabilité. J’aimerais également étendre ces approches à des domaines connexes tels que la robotique multi-agent ou certains systèmes cyber-physiques, où la gestion de l’incertitude et la robustesse des comportements collectifs sont essentielles. Ces perspectives s’inscrivent dans une continuité naturelle avec la thèse et visent à explorer des solutions hybrides mêlant modèles formels, apprentissage et simulation pour des applications critiques.



\subsection*{Publications}

% TODO
%  - l’identité de tous les Auteurs
%  - l’année de publication
%  - l’intitulé (complet) de la revue ou conférence
%  - le type d’article (long/court, présentation/poster…)
%  - les numéros de page (ou à défaut le nombre de pages)
%  - s’il s’agit d’une session annexe de la conférence, le préciser
%  - arguments (éventuellement, métriques) montrant l’importance/la sélectivité de la revue/conférence
%  - preuve de l’existence de la publication autre que l’article lui-même (lien vers les actes, une archive en ligne, le planning de la conférence, le sommaire de la revue, le DOI, lettre d’acceptation si l’article était à l’état “soumis”, etc.)
%  - distinguer les conférences nationales et internationales (comité de programme et audience internationales)
%  - indiquer les 3 (5 pour les PR) travaux fournis (les plus visibles) et les situer dans leur contexte de recherche. Pour ces publications, le·a candidat·e décrira sa contribution propre en informatique.
%  - pour les travaux pluridisciplinaires, préciser la contribution à la discipline informatique.

\noindent
Les travaux numérotés \textcolor{SectionColor}{1}, \textcolor{SectionColor}{2} et \textcolor{SectionColor}{3} sont les plus significatifs. Pour ces travaux, la description de ma contribution pour chaque papier est basée sur la taxonomie \textit{CRediT}\footnote{\url{https://credit.niso.org/}}:
%
\begin{multicols}{2}
    \begin{enumerate}[leftmargin=*, itemsep=-4pt, topsep=0pt, label=\arabic*.]
        \item Conceptualisation
        \item Traitement des données
        \item Analyse formelle
        \item Acquisition des fonds
        \item Investigation
        \item Méthodologie
        \item Administration du projet
        \item Apport des ressources matérielles
        \item Développement logiciel
        \item Supervision
        \item Validation
        \item Visualisation
        \item Rédaction -- première version
        \item Rédaction -- après relecture \& modification
    \end{enumerate}
\end{multicols}

\subsubsection*{Revues nationales avec comité de lecture}

\begin{itemize}[leftmargin=1.5em]
    \item[\textbf{\textcolor{SectionColor}{1.}}] \begin{itemize}[leftmargin=0.1em, itemsep=-2pt, topsep=0pt]
            \item[] \textbf{Titre} : \textquote{Une approche organisationnelle pour améliorer l’explicabilité et le contrôle dans le MARL}
            \item[] \textbf{Auteurs} : Julien Soulé, Jean-Paul Jamont, Michel Occello, Louis-Marie Traonouez, Paul Théron
            \item[] \textbf{Année de publication} : 2025/2026
            \item[] \textbf{Intitulé de la revue} : Revue Ouverte d'Intelligence Artificielle (ROIA)
            \item[] \textbf{Type d'article} : Article de journal (sans limite de pages)
            \item[] \textbf{Numéros et nombre de page} : En cours de publication (28 pages)
            \item[] \textbf{Arguments d'importance} : ROIA est une revue scientifique à comité de lecture publiée sur la plateforme Centre Mersenne, soutenue par le CNRS, l’INRIA et l’Université Grenoble Alpes. La revue applique un processus rigoureux d’évaluation par les pairs et bénéficie d’une diffusion internationale, étant notamment indexée dans DBLP\footnote{\url{https://roia.centre-mersenne.org/}}.
            \item[] \textbf{Preuve existence publication} : En cours de publication (lettre d'acceptation et rapports des relecteurs fournis en annexe).
            \item[] \textbf{Ma contribution} : 1, 2, 3, 5, 6, 9, 11, 12, 13, 14
        \end{itemize}
\end{itemize}


\subsubsection*{Conférences internationales avec comité de lecture}

\begin{itemize}[leftmargin=1.5em]
    \item[\textbf{\textcolor{SectionColor}{2.}}] \begin{itemize}[leftmargin=0.1em, itemsep=-2pt, topsep=0pt]
            \item[] \textbf{Titre} : \textquote{An Organizationally-oriented Approach to Enhancing Explainability and Control in Multi-Agent Reinforcement Learning}
            \item[] \textbf{Auteurs} : Julien Soulé, Jean-Paul Jamont, Michel Occello, Louis-Marie Traonouez, Paul Théron
            \item[] \textbf{Année de publication} : 2025
            \item[] \textbf{Intitulé de la conférence} : \textit{24th International Conference on Autonomous Agents and Multiagent Systems (AAMAS)}
            \item[] \textbf{Type d'article} : Article de conférence internationale de type papier long
            \item[] \textbf{Numéros et nombre de page} : pages 1968 -- 1976 (9 pages)
            \item[] \textbf{Métrique d'importance} : Conférence de rang A* d'après le classement CORE 2023 (taux d'acceptation de 25\% pour les articles longs)\footnote{\url{https://portal.core.edu.au/conf-ranks/922/}}.
            \item[] \textbf{DOI} : \href{https://dl.acm.org/doi/10.5555/3709347.3743834}{10.5555/3709347.3743834}
            \item[] \textbf{Ma contribution} : 1, 2, 3, 5, 6, 9, 11, 12, 13, 14
        \end{itemize}

        \vspace{0.4cm}

    \item[\textbf{\textcolor{SectionColor}{3.}}] \begin{itemize}[leftmargin=0.1em, itemsep=-2pt, topsep=0pt]
            \item[] \textbf{Titre} : \textquote{Streamlining Resilient Kubernetes Autoscaling with Multi-Agent Systems via an Automated Online Design Framework}
            \item[] \textbf{Auteurs} : Julien Soulé, Jean-Paul Jamont, Michel Occello, Louis-Marie Traonouez, Paul Théron
            \item[] \textbf{Année de publication} : 2025
            \item[] \textbf{Intitulé de la conférence} : \textit{18th IEEE International Conference on Cloud Computing (CLOUD 2025)}
            \item[] \textbf{Type d'article} : Article de conférence internationale de type papier long
            \item[] \textbf{Numéros et nombre de page} : pages 43 -- 53 (9 pages)
            \item[] \textbf{Métrique d'importance} : Conférence de rang B d'après le classement CORE 2023\footnote{\url{https://portal.core.edu.au/conf-ranks/631/}}.
            \item[] \textbf{DOI} : \href{https://www.doi.org/10.1109/CLOUD67622.2025.00015}{10.1109/CLOUD67622.2025.00015}
            \item[] \textbf{Ma contribution} : 1, 2, 3, 5, 6, 9, 11, 12, 13, 14
        \end{itemize}

        \vspace{0.4cm}

    \item[4.] \begin{itemize}[leftmargin=0.1em, itemsep=-2pt, topsep=0pt]
            \item[] \textbf{Titre} : \textquote{Towards a Multi-Agent Simulation of Cyber-attackers and Cyber-defenders Battles}
            \item[] \textbf{Auteurs} : Julien Soulé, Jean-Paul Jamont, Michel Occello, Paul Théron, Louis-Marie Traonouez
            \item[] \textbf{Année de publication} : 2023
            \item[] \textbf{Intitulé de la conférence} : \textit{2023 IEEE International Conference on Systems, Man, and Cybernetics (SMC)}
            \item[] \textbf{Type d'article} : Article de conférence internationale de type papier long
            \item[] \textbf{Métrique d'importance} : Conférence de rang B d'après le classement CORE 2023\footnote{\url{https://portal.core.edu.au/conf-ranks/611/}}.
            \item[] \textbf{DOI} : \href{https://www.doi.org/10.1109/SMC53992.2023.10394564}{10.1109/SMC53992.2023.10394564}
            \item[] \textbf{Pages} : 3594 -- 3599
        \end{itemize}

\end{itemize}

\subsubsection*{Conférences nationales avec comité de lecture}

\noindent
J'ai produit plusieurs articles dans des conférences à comité de lecture, notamment lors des Journées Francophones sur les Systèmes Multi-Agents (JFSMA) \cite{soule2025jfsma}, \cite{soule2024approche}, \cite{soule2024outil}, \cite{soule2023jfsmathese} ; à la conférence Rendez-vous de la Recherche et de l'Enseignement de la Sécurité des Systèmes d'Information (RESSI) \cite{soule2023ressithese} et la conférence Rendez-vous des Jeunes Chercheurs en Intelligence Artificielle (RJCIA) \cite{soule2023rjciathese}. Ces contributions ont permis de présenter des avancées méthodologiques et des résultats expérimentaux.


\subsubsection*{Productions logicielles et mise à disposition de données.}

\noindent
Plusieurs outils logiciels ont été développés pour la modélisation, l'apprentissage et l'analyse de SMA, structurant les expérimentations et formant une part importante des contributions techniques.

\paragraph{MOISE+MARL.}
Un ensemble d'outils a été développé pour intégrer le modèle organisationnel $\mathcal{M}OISE^+$~\cite{Hubner2002} dans le MARL. Cet ensemble comprend l'implémentation des rôles, missions et normes de $\mathcal{M}OISE^+$, des mécanismes d'influence de l'apprentissage (récompenses basées sur les rôles, contraintes de coordination, filtrage d'actions), et le module \textit{\textbf{TEMM}} pour extraire les organisations émergentes des traces d'exécution. MOISE+MARL est disponible à \url{https://github.com/julien6/MOISE-MARL} avec documentation à \url{https://julien6.github.io/MOISE-MARL/}.

\paragraph{CybMASDE.}
La plateforme \textit{\textbf{CybMASDE}} constitue le logiciel principal utilisé pour structurer les étapes de la méthodologie \textit{MAMAD}. Elle regroupe la modélisation organisationnelle, la génération d'environnements, les algorithmes de MARL adaptés à la Cyberdéfense, ainsi que des modules d'analyse destinés à étudier les comportements acquis. La plateforme inclut différents environnements simulant des scénarios d'attaque/défense ou de surveillance distribuée, ainsi que des outils permettant de suivre l'évolution des politiques, des interactions et des rôles assumés par les agents. Le code source est accessible à l'adresse \url{https://github.com/julien6/CybMASDE} et accompagné d'une documentation de prise en main : \url{https://julien6.github.io/CybMASDE/}.

\paragraph{Environnements d'apprentissage.}
Deux environnements coopératifs en grille ont été développés pour évaluer les approches proposées : l'un simule le déménagement collectif de caisses, l'autre une usine-entrepôt où les agents collectent des ressources, fabriquent des produits et les livrent. Ces environnements sont disponibles à \url{https://github.com/julien6/OMARLE}.

\paragraph{Données expérimentales.}

Les expérimentations menées avec \textit{CybMASDE} ont généré des données structurées : trajectoires d'agents, \textit{World Models}~\cite{ha2018recurrent}, poids de réseaux neuronaux et analyses TEMM au format \textit{JSON}. Une partie est disponible dans \textit{CybMASDE} via l'environnement \textit{Overcooked-AI}. Les autres données sont accessibles sur demande, à l'exception de celles provenant d'applications industrielles, soumises à des restrictions de confidentialité.


\subsubsection*{Autres: présentations}

\noindent
J'ai présenté les travaux de la thèse à plusieurs occasions, notamment dans le cadre d'une intervention auprès de la chaire CybAIR\footnote{\url{https://gdr-securite.irisa.fr/wp-content/uploads/RESSI2019-Projet-ChaireCyberResilienceAerospatiale.pdf}} de l'OTAN en 2023. Cette présentation visait à introduire les enjeux liés à l'automatisation de la défense dans des environnements distribués, ainsi que les approches multi-agents permettant de structurer ou d'analyser ces comportements.

\subsubsection*{Références}

\setlength{\bibitemsep}{0.1\baselineskip}
\AtNextBibliography{\footnotesize}
\printbibliography[heading=none, sorting=ydnt]


\section*{Activités / responsabilités collectives, administratives}
% =======================================================

% Décrire les activités administratives, en détaillant complètement au moins les trois dernières années effectives :
%     - sujet, lieu et nature de chaque activité
%     - évaluation du temps consacré à ces activités et des éventuels résultats obtenus

\begin{enumerate}[leftmargin=1.5em]

    \item \begin{itemize}[leftmargin=0.1em, itemsep=-2pt, topsep=0pt]
              \item[] \textbf{Sujet} : Relectures pour les conférences AAMAS (3 revues), ICAART (3 revues), SMC (2 revues) et ICDL (1 revue) ainsi que JFSMA (2 revues) et RESSI (1 revue)
              \item[] \textbf{Lieu} : En ligne
              \item[] \textbf{Date} : 2023, 2024, 2025
              \item[] \textbf{Nature} : Relecteur de travaux scientifiques pour publication
              \item[] \textbf{Estimation} : Entre 3 et 5 heures par article
              \item[] \textbf{Résultat} : Relecture présentés sous la forme d'un rapport d'au moins une demi-page A4 pour chacun des articles des conférences.
          \end{itemize}

          \vspace{0.4cm}

    \item \begin{itemize}[leftmargin=0.1em, itemsep=-2pt, topsep=0pt]
              \item[] \textbf{Sujet} : Participations aux tables rondes de la semaine \textquote{Kaléidoscope}\footnote{\url{https://kaleidoscope.pages.ensimag.fr/}} pour le LCIS
              \item[] \textbf{Lieu} : Valence, France
              \item[] \textbf{Date} : Janvier et février 2024 et 2025
              \item[] \textbf{Nature} : Participation à un événement scientifique destiné à des élèves en étude supérieure
              \item[] \textbf{Estimation} : 8h
              \item[] \textbf{Résultat} : Le \textquote{Kaléidoscope} est un événement de Grenoble-INP visant à faire de la découverte de métiers. Dans le cas de ma participation, il s’agit de présenter le laboratoire (donc la recherche académique). Pour cela, j’ai aidé à coordonner les équipes de recherche pour réaliser des présentations, à communiquer avec les étudiants intéressés, et à assurer le bon déroulement lors des tables rondes.
          \end{itemize}

          \vspace{0.4cm}

    \item \begin{itemize}[leftmargin=0.1em, itemsep=-2pt, topsep=0pt]
              \item[] \textbf{Sujet} : Participation à la \textquote{CyberWeek} de Grenoble-INP -- Esisar\footnote{\url{https://esisar.grenoble-inp.fr/fr/l-ecole/retour-sur-la-cyberweek-de-lesisar}}
              \item[] \textbf{Lieu} : Valence, France
              \item[] \textbf{Date} : 2024
              \item[] \textbf{Nature} : Vulgarisation scientifique à un public d’étudiants ingénieurs
              \item[] \textbf{Estimation} : 8h
              \item[] \textbf{Résultat} : Lors d’un événement visant à présenter les opportunités après l’obtention d’un diplôme d’ingénieur en informatique avec une spécialité en Cybersécurité, présentation générale du déroulement d’une thèse académique, de mes travaux et réponse aux questions des étudiants.
          \end{itemize}

          \vspace{0.4cm}

    \item \begin{itemize}[leftmargin=0.1em, itemsep=-2pt, topsep=0pt]
              \item[] \textbf{Sujet} : Trésorier du groupe AICA IWG (Autonomous Intelligent Cyberdefense Agent International Work Group)\footnote{\url{https://www.aica-iwg.org/aica-iwg-vision-and-mission/}}
              \item[] \textbf{Lieu} : En ligne
              \item[] \textbf{Date} : 2023 -- 2025
              \item[] \textbf{Nature} : Responsabilité administrative au sein d'une organisation internationale
              \item[] \textbf{Estimation} : 2 à 3 heures par mois
              \item[] \textbf{Résultat} : Le groupe AICA IWG rassemble des chercheurs et ingénieurs intéressés par les questions liées aux agents autonomes appliqués à la Cyberdéfense et à la Cybersécurité. Mon rôle consiste principalement à assurer le suivi des aspects administratifs : gestion des adhésions, échanges avec les membres, mise à jour des informations financières et participation aux réunions de pilotage du groupe. Cette activité demande une coordination régulière avec les autres membres du bureau lors de réunions une fois toutes les deux semaines.
          \end{itemize}

          \vspace{0.4cm}

    \item \begin{itemize}[leftmargin=0.1em, itemsep=-2pt, topsep=0pt]
              \item[] \textbf{Sujet} : Organisation de la rencontre \textquote{Thales La Ruche -- LCIS autour du cyber}
              \item[] \textbf{Lieu} : LCIS, Valence (UGA)
              \item[] \textbf{Date} : 21 janvier 2024
              \item[] \textbf{Nature} : Coordination scientifique et logistique d’un événement de rencontre entre partenaires académiques et industriels
              \item[] \textbf{Estimation} : Environ 10 à 12 heures (préparation, coordination des intervenants, organisation des démonstrations)
              \item[] \textbf{Résultat} : L’événement a réuni des membres du LCIS et de Thales autour de thématiques liées à la cybersécurité. J’ai assuré la coordination de la journée, la planification des interventions et des démonstrations, ainsi que les échanges logistiques avec les partenaires. La rencontre a permis de présenter les travaux du laboratoire, de renforcer les liens avec Thales et d’identifier plusieurs pistes de collaboration.
          \end{itemize}

          \vspace{0.4cm}

    \item \begin{itemize}[leftmargin=0.1em, itemsep=-2pt, topsep=0pt]
              \item[] \textbf{Sujet} : Co-organisation (à 50~\%) d’une journée des doctorants à caractère scientifique et ludique
              \item[] \textbf{Lieu} : Valence, France
              \item[] \textbf{Date} : Janvier -- juin 2024
              \item[] \textbf{Nature} : Co-organisation d’un événement scientifique et de cohésion destiné aux doctorants
              \item[] \textbf{Estimation} : Environ 40 heures
              \item[] \textbf{Résultat} : La journée a réuni une cinquantaine de personnes (doctorants, stagiaires, enseignant-chercheurs) autour de présentations scientifiques (orales et posters) le matin, suivies d’activités sportives l’après-midi. En tant que co-organisateur, j’ai participé à la définition du programme scientifique, à la recherche et la réservation des lieux, à l’organisation logistique (traiteur, activités, budget), au choix de la date et à la communication auprès des participants.
          \end{itemize}

          \vspace{0.4cm}

    \item \begin{itemize}[leftmargin=0.1em, itemsep=-2pt, topsep=0pt]
              \item[] \textbf{Sujet} : Participation aux événements doctoraux Thales (\textit{PhD Days}, \textit{Time for Innovation}, journée doctorat \textit{KTD})
              \item[] \textbf{Lieu} : En ligne et sur site Thales (Rennes et Rungis)
              \item[] \textbf{Date} : 2023 -- 2025
              \item[] \textbf{Nature} : Présentations scientifiques et actions de diffusion interne au Groupe
              \item[] \textbf{Estimation} : Environ 35 à 40 heures (préparation de supports, répétitions, réalisation d’une vidéo, interventions)
              \item[] \textbf{Résultat} : Participation aux PhD Days via la réalisation d’une vidéo vulgarisée de type « Ma thèse en 180 secondes », présentation détaillée des travaux au niveau international dans le cadre du programme « Time for Innovation » (25-30 minutes), et intervention lors de la journée réunissant les doctorants Thales sur le site de Rungis. Ces différentes actions ont renforcé les échanges entre équipes techniques et la recherche.
          \end{itemize}

          \vspace{0.4cm}

    \item \begin{itemize}[leftmargin=0.1em, itemsep=-2pt, topsep=0pt]
              \item[] \textbf{Sujet} : Organisation de réunions techniques et échanges de coordination à \textquote{La Ruche} (Rennes)
              \item[] \textbf{Lieu} : Thales \textquote{La Ruche}, Rennes
              \item[] \textbf{Date} : 2023 -- 2024
              \item[] \textbf{Nature} : Coordination scientifique et animation de discussions techniques autour des thématiques cyber et multi-agents
              \item[] \textbf{Estimation} : Environ 10 heures par réunion (préparation, organisation, animation)
              \item[] \textbf{Résultat} : Mise en place et animation de plusieurs réunions de travail impliquant des ingénieurs, chercheurs et doctorants de \textquote{La Ruche} autour de problématiques liées à la simulation cyber, aux SMA et aux applications potentielles en environnement opérationnel. Ces rencontres ont facilité le partage d’avancées techniques, l’identification de besoins communs et l’exploration de pistes de collaboration entre le LCIS et les équipes Thales.
          \end{itemize}

\end{enumerate}

\end{document}
