\documentclass{article}
\usepackage[utf8]{inputenc}
\usepackage[T1]{fontenc}
\usepackage[french]{babel}
\usepackage{csquotes}
\begin{document}

\section*{Résumé de l'article \textquote{Streamlining Resilient Kubernetes Autoscaling with Multi-Agent Systems via an Automated Online Design Framework}}

Cet article se situe à l’intersection de l’intelligence artificielle distribuée, des systèmes multi-agents (SMA) et des systèmes cloud natifs, et s’intéresse au problème de l’autoscaling résilient de clusters Kubernetes dans des contextes dynamiques et potentiellement adversariaux (pannes, surcharge, attaques de type DDoS).

Les mécanismes classiques de \textit{Horizontal Pod Autoscaling} (HPA), principalement fondés sur des règles et des seuils, montrent des limites importantes face à des scénarios complexes impliquant des dépendances entre services, des défaillances en cascade ou des attaques ciblées. Les approches basées sur l’apprentissage par renforcement améliorent l’adaptabilité, mais restent majoritairement mono-agent et centrées sur un objectif unique, ce qui limite leur capacité à maintenir une résilience opérationnelle globale.

Pour répondre à ces limites, l’article propose \textbf{KARMA}, un cadre automatisé de conception et de déploiement de systèmes multi-agents pour l’autoscaling Kubernetes. KARMA repose sur la décomposition de l’objectif global de résilience opérationnelle en sous-objectifs spécialisés, pris en charge par des agents coopératifs dotés de rôles et de missions explicites, inspirés des modèles organisationnels des SMA et des architectures de cyberdéfense multi-agents.

Le cadre KARMA est structuré en quatre phases en ligne : (i) une phase de modélisation reposant sur la construction d’un jumeau numérique du cluster Kubernetes à partir de traces réelles, incluant un modèle de transition appris par réseau de neurones ; (ii) une phase d’entraînement en simulation par apprentissage par renforcement multi-agent, guidée par des rôles et des missions correspondant à des scénarios de défaillance (goulots d’étranglement, attaques DDoS, pannes de pods, contention de ressources) ; (iii) une phase d’analyse visant à améliorer l’explicabilité par l’étude des trajectoires d’agents et l’identification de comportements cohérents avec les rôles définis ; (iv) une phase de transfert des politiques apprises vers un cluster Kubernetes réel via l’API Kubernetes, avec des mécanismes de sécurité.

L’approche est évaluée sur un environnement Kubernetes complexe simulant des services chaînés, et comparée à plusieurs solutions d’autoscaling de l’état de l’art. Les résultats montrent que KARMA améliore significativement la résilience opérationnelle, la robustesse face aux attaques, la rapidité de récupération et la stabilité des décisions, tout en réduisant le temps de convergence grâce aux contraintes organisationnelles.

Cet article apporte une contribution méthodologique originale en proposant un cadre unifié, automatisé et explicable pour la conception de systèmes multi-agents d’autoscaling, combinant jumeaux numériques, apprentissage par renforcement multi-agent et modèles organisationnels, avec une validation sur infrastructure Kubernetes réaliste.

\end{document}
