% !TEX TS-program = xelatex
% Compile with: xelatex or lualatex

\documentclass[11pt,a4paper,sans]{moderncv}

% ModernCV style
\moderncvstyle{classic}    % casual, classic, banking, oldstyle, fancy
\moderncvcolor{blue}       % blue, orange, green, red, purple, grey, black

% Encoding and layout
\usepackage[utf8]{inputenc}
\usepackage[T1]{fontenc}
\usepackage[scale=0.85]{geometry}
\setlength{\hintscolumnwidth}{3cm}

%----------------------------------------------------------------------------------
% Personal data
%----------------------------------------------------------------------------------
\name{Julien}{Soulé}
\title{Doctor in Computer Science}
\address{35 Rue Mathieu-de-la-Drôme}{26000 Valence, France}{}
\phone[mobile]{+33~6~77~63~12~13}
\email{julien.soule@hotmail.fr}
\homepage{julien6.github.io/home/}
% \social[linkedin]{julien-soulé-6b2b27173}
% \social[github]{julien6}

\begin{document}

%----------------------------------------------------------------------------------
% Letter
%----------------------------------------------------------------------------------
\recipient{Helsing}{Recruitment Team}
\date{\today}
\opening{Dear Hiring Team,}
\closing{Sincerely,}
\enclosure[Attachment]{Curriculum Vitae}

\makelettertitle

I am writing to apply for the position of \textit{AI Research Engineer in Reinforcement Learning}. I recently completed a PhD in Computer Science at Université Grenoble Alpes, conducted in collaboration with Thales Land \& Air Systems. My work focused on designing Multi-Agent Systems (MASs) for Cyberdefense, Multi-Agent Reinforcement Learning (MARL), and the design of autonomous decision-making architectures in adversarial environments.

My academic background is rooted in computer engineering and cybersecurity, and my research progressively converged toward the question of how collections of learning agents can be structured, constrained, and evaluated as coherent systems rather than as isolated learners. During my doctoral work, I explored the use of organizational models to guide multi-agent reinforcement learning, with the objective of improving coordination, robustness, and interpretability. This research led to several peer-reviewed publications and to the development of research software platforms combining simulation, learning, and system-level analysis.

In parallel, my experience in industry, notably at Thales and Atos, allowed me to work on complex and safety-critical systems in operational contexts. These experiences strongly shaped my approach to research, with particular attention to system architecture, experimental reproducibility, and the practical constraints that arise when learning-based components are embedded in real-world systems. Over time, I have come to view research and engineering not as separate activities, but as two complementary aspects of building reliable autonomous systems.

At this stage of my career, I am looking to continue working on problems related to autonomous decision-making, learning under uncertainty, and the coordination of complex systems, within an environment where long-term technical depth and real-world impact matter. I am particularly interested in settings that combine advanced AI research with demanding operational constraints.

I would be glad to discuss my background and interests in more detail, and to explore whether my experience could be a good fit for your team.

\makeletterclosing

\end{document}
