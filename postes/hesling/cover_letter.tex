% !TEX TS-program = xelatex
% Compile with: xelatex or lualatex

\documentclass[11pt,a4paper,sans]{moderncv}

% ModernCV style
\moderncvstyle{classic}    % casual, classic, banking, oldstyle, fancy
\moderncvcolor{blue}       % blue, orange, green, red, purple, grey, black

% Encoding and layout
\usepackage[utf8]{inputenc}
\usepackage[T1]{fontenc}
\usepackage[scale=0.86]{geometry}
\setlength{\hintscolumnwidth}{3cm}

%----------------------------------------------------------------------------------
% Personal data
%----------------------------------------------------------------------------------
\name{Julien}{Soulé}
\title{Doctor in Computer Science}
\address{35 Rue Mathieu-de-la-Drôme}{26000 Valence, France}{}
\phone[mobile]{+33~6~77~63~12~13}
\email{julien.soule@hotmail.fr}
\homepage{julien6.github.io/home/}
% \social[linkedin]{julien-soulé-6b2b27173}
% \social[github]{julien6}

\AfterPreamble{
  \hypersetup{
    colorlinks=false,
    pdfborder={0 0 1},
    pdfborderstyle={/S/U/W 1},
    linkbordercolor={0 0 1},
    urlbordercolor={0 0 1},
    citebordercolor={0 0 1}
  }
}

\begin{document}

%----------------------------------------------------------------------------------
% Letter
%----------------------------------------------------------------------------------
\recipient{Helsing}{Recruitment Team}
\date{\today}
\opening{Dear Hiring Team,}
\closing{Sincerely,}
% \enclosure[Attachment]{Curriculum Vitae}

\makelettertitle

I am writing to apply for the position of \textit{AI Research Engineer in Reinforcement Learning}. I recently completed a PhD in Computer Science conducted jointly at \textit{Université Grenoble Alpes} and \textit{Thales Land \& Air Systems} within the \href{https://gdr-securite.irisa.fr/wp-content/uploads/RESSI2019-Projet-ChaireCyberResilienceAerospatiale.pdf}{\textbf{Cyb'Air}} research chair, which also brings together \textbf{Dassault} and the \textbf{\textit{Grand État-Major de l’Armée de l’Air et de l’Espace}}. \href{https://julien6.github.io/home/#thesis}{\textbf{My PhD}} was initially centered on the \href{https://link.springer.com/book/10.1007/978-3-031-29269-9}{\textbf{AICA}} (Autonomous Intelligent Cyberdefense Agent) initiative launched by the \textbf{NATO IST-152} working group and its successor, the \href{https://www.aica-iwg.org/}{\textbf{AICA IWG}} (AICA International Working Group), where I currently serve as treasurer. The AICA initiative aims to design and build a \textbf{Cyberdefense Multi-Agent System} deployable in any networked environment, particularly in highly critical and embedded systems, to defend against malicious agents and mitigate potential attacks and their consequences.

Throughout \href{https://julien6.github.io/home/#thesis}{\textbf{my PhD}}, I developed a \href{https://github.com/julien6/CybMASDE}{\textbf{methodological framework}} for automating or assisting the development of \textbf{Multi-Agent Systems}, framing their design as a constrained optimization problem. In this framework, the joint policy is learned automatically via \textbf{Multi-Agent Reinforcement Learning (MARL)}, while constraints are implemented through \textbf{organizational modeling} to guide or enforce agent behaviors in accordance with design requirements. One of the contributions of this framework is a method to explicitly characterize trained agent behaviors using \textbf{unsupervised machine learning} techniques. My goal was not merely to use \textbf{MARL}, but to make it more interpretable, controllable, and systematically applicable to AI-assisted system design.

In industry, my work focused on the cyber-resilience of safety-critical military telecommunication systems, managing heterogeneous network traffic under strict quality-of-service constraints. In academia, I explored \textbf{IoT}-inspired case studies ranging from enterprise network infrastructures to drone swarms and warehouse management systems. This work provided me with a solid technical background across the full stack of \textbf{MARL}, primarily developing in \textbf{Python} (as well as \textbf{Rust} and \textbf{Java} in Thales-related projects) using frameworks such as \textbf{PyTorch} and \textbf{TensorFlow}, together with RL/MARL libraries including \textbf{PettingZoo/Gym}, \textbf{MARLlib/RLlib}, \textbf{Jax/JaxMARL}, and \textbf{Optuna} for high-performance experimentation. I am accustomed to managing large-scale experimental workflows using high-performance computing infrastructures (such as the academic \href{https://gricad-doc.univ-grenoble-alpes.fr/en/}{\textbf{Gricad}} HPC platform), alongside software engineering practices including \textbf{Git}. Over time, I have come to view research and engineering not as separate activities, but as complementary aspects of building reliable autonomous systems.


Throughout my doctoral research, I was exposed to a rich international research environment, spanning both academia and industry (notably \textbf{Thales Germany}), and contributed to the dissemination of our research in top-tier conferences and journals such as \href{https://arxiv.org/abs/2503.23615}{\textbf{(AAMAS~2025)}} and \href{https://assets-eu.researchsquare.com/files/rs-7166037/v1_covered_908e23dd-6fb8-4efc-9ef3-a78c4d539bac.pdf?c=1753863562}{\textbf{(JAAMAS~2025, under revision)}}.
This experience has strengthened my ability to communicate and collaborate effectively within interdisciplinary teams.

At this stage of my career, I am looking to continue working on problems related to autonomous decision-making, joint-learning under uncertainty, and the coordination of agents, within an environment where long-term technical depth and real-world impact matter. I am particularly interested in settings that combine AI research with demanding operational constraints.

I would be glad to discuss my background and interests in more detail, and to explore whether my experience could be a good fit for your team.

\makeletterclosing

\end{document}
