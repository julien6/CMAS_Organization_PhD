
\begin{table}[htb]

  \caption{Un aperçu des fonctions de Cyberdéfense prises en charge par les SMA de Cyberdéfense étudiés}

  \begin{tabularx}{\textwidth}{
      >{\raggedright\arraybackslash\hsize=.9\hsize}X
      >{\raggedright\arraybackslash\hsize=.3\hsize}X}
    \toprule

    { {\textbf{Objectifs principaux}}}
     & {  \textbf{Travaux}}
    \\ \midrule

    {  \textbf{\textbf{R1}}: détection d'intrusion, surveillance du réseau, détection de menaces possibles}
     & {  \cite{vasilomanolakis2015taxonomy, gorodetski2003multi, de2017distributed, holloway2009self, lamont2009military, akandwanaho2018generic}}
    \\

    {  \textbf{\textbf{R2}}: application de contre-mesures, contrôles d'accès, correctifs de Cyberdéfense, stratégies de Cyberdéfense}
     & {  \cite{holloway2009self, lamont2009military, akandwanaho2018generic}}
    \\

    {  \textbf{\textbf{R3}}: investigations forensiques, élaboration de contre-mesures adaptées, apprentissage des cyber-attaques, adaptation aux cyber-attaques}
     & {  \cite{holloway2019self, haack2011ant, morteza2015method, demir2021adaptive}}
    \\
    \bottomrule
  \end{tabularx}
  \label{tab:reference-cyberdefense}
\end{table}


