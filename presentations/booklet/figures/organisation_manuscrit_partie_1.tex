\begin{tikzpicture}[
        chapter/.style={draw, fill=blue!10, thick, minimum width=8cm, minimum height=1.2cm, text centered, font=\bfseries},
        section/.style={draw, fill=blue!5, thick, minimum width=7cm, minimum height=1cm, text centered, font=\small},
        arrow/.style={-Latex, thick},
        node distance=0.4cm,
        annotated/.style={above,font=\small\itshape, inner sep=1pt, yshift=3mm, xshift=-5cm}
    ]

    % Chapitre 1
    \node[chapter] (ch1) {\parbox{10cm}{Chapitre 1 : Repenser la Cyberdéfense pour de nouveaux enjeux}};

    \node[section, below=1cm of ch1] (ch1s1) {\parbox{8cm}{Aperçu du domaine de la Cyberdéfense}};
    \node[section, below=1cm of ch1s1] (ch1s2) {\parbox{8cm}{Des menaces de plus en plus autonomes et distribuées}};
    \node[section, below=1cm of ch1s2] (ch1s3) {\parbox{8cm}{La piste d'une vision multi-agent}};
    \node[section, below=1cm of ch1s3] (ch1s4) {\parbox{8cm}{De la conception d'un SMA de Cyberdéfense}};

    \draw[arrow] (ch1.south) -- (ch1s1.north) node[annotated] {Poser les bases du domaine et ses spécificités};
    \draw[arrow] (ch1s1.south) -- (ch1s2.north) node[annotated] {Identifier les défis posés par les nouvelles menaces};
    \draw[arrow] (ch1s2.south) -- (ch1s3.north) node[annotated] {Discuter la nécessité de changer de paradigme};
    \draw[arrow] (ch1s3.south) -- (ch1s4.north) node[annotated] {Faire émerger la question de la conception organisationnelle};

    % Chapitre 2
    \node[chapter, below=1cm of ch1s4] (ch2) {\parbox{10cm}{Chapitre 2 : Vers des approches multi-agents de Cyberdéfense}};

    \node[section, below=1cm of ch2] (ch2s1) {\parbox{8cm}{Les approches d'agents autonomes}};
    \node[section, below=1cm of ch2s1] (ch2s2) {\parbox{8cm}{Les tentatives d'approche multi-agent}};
    \node[section, below=1cm of ch2s2] (ch2s3) {\parbox{8cm}{Une tension entre approche symbolique et connexioniste}};

    \draw[arrow] (ch2.south) -- (ch2s1.north) node[annotated] {Explorer les premières alternatives autonomes};
    \draw[arrow] (ch2s1.south) -- (ch2s2.north) node[annotated] {Passer du mono-agent au multi-agent pour mieux coordonner};
    \draw[arrow] (ch2s2.south) -- (ch2s3.north) node[annotated] {Mettre en lumière un défi d'intégration fondamental};

    % Chapitre 3
    \node[chapter, below=1cm of ch2s3] (ch3) {\parbox{10cm}{Chapitre 3 : D'un problème d'ingénierie à un problème d'optimisation}};

    \node[section, below=1cm of ch3] (ch3s1) {\parbox{8cm}{La question de recherche comme la modélisation et résolution d'un problème d'optimisation}};
    \node[section, below=1cm of ch3s1] (ch3s2) {\parbox{8cm}{Les sous-problèmes induits vers la conception}};
    \node[section, below=1cm of ch3s2] (ch3s3) {\parbox{8cm}{Vers une méthode de conception assistée}};

    \draw[arrow] (ch3.south) -- (ch3s1.north) node[annotated] {Donner une formulation rigoureuse du problème};
    \draw[arrow] (ch3s1.south) -- (ch3s2.north) node[annotated] {Décomposer le problème en sous-composantes manipulables};
    \draw[arrow] (ch3s2.south) -- (ch3s3.north) node[annotated] {Préparer le socle méthodologique de la thèse};

    % Liens inter-chapitres
    \draw[arrow] (ch1s4.south) -- (ch2.north) node[annotated] {Du besoin d'un SMA à l'analyse des solutions existantes};
    \draw[arrow] (ch2s3.south) -- (ch3.north) node[annotated] {Formaliser les besoins comme un problème d'optimisation};

\end{tikzpicture}
