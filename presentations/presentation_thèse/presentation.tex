\documentclass[9pt, aspectratio=169]{beamer}
% \documentclass[10pt]{beamer}
\usepackage[utf8]{inputenc}
\usepackage[T1]{fontenc}
\usepackage[english]{babel}
\usetheme{Frankfurt}

\usepackage[backend=biber, style=authoryear]{biblatex}
\addbibresource{local_references.bib}

%\usepackage{lmodern}
\usepackage{amsfonts,amssymb,amsmath}
\usepackage[english]{babel}
\usetheme{Frankfurt}

\usepackage{csquotes}
\usepackage{setspace}

\usepackage{colortbl}
\usepackage{tabularx}
\renewcommand\tabularxcolumn[1]{m{#1}}

% --- Tickz
\usepackage{physics}
\usepackage{amsmath}
\usepackage{tikz}
\usepackage{mathdots}
\usepackage{yhmath}
\usepackage{cancel}
\usepackage{color}
\usepackage{siunitx}
\usepackage{array}
\usepackage{multirow}
\usepackage{amssymb}
\usepackage{gensymb}
\usepackage{tabularx}
\usepackage{extarrows}
\usepackage{booktabs}
\usetikzlibrary{fadings}
\usetikzlibrary{patterns}
\usetikzlibrary{shadows.blur}
\usetikzlibrary{shapes}

% ---------

\usepackage{booktabs}
\usepackage{setspace}
\usepackage{amssymb}
\usepackage{adjustbox}
\usepackage{pifont}
\usepackage[inkscapeformat=png]{svg}
\usepackage{graphicx}
\usepackage{times}
\setbeamertemplate{caption}[numbered]
% % \setbeamertemplate{bibliography item}{[\theenumiv]}

\setbeamerfont{bibliography item}{size=\tiny}
\setbeamerfont{bibliography entry author}{size=\tiny}
\setbeamerfont{bibliography entry title}{size=\tiny}
\setbeamerfont{bibliography entry location}{size=\tiny}
\setbeamerfont{bibliography entry note}{size=\tiny}

\setbeamerfont{frametitle}{size=\large}

\usepackage{caption}
\usepackage{float}
\usepackage{xcolor}
\usepackage{listings}
\usepackage{animate}

\definecolor{codegreen}{rgb}{0,0.6,0}
\definecolor{codegray}{rgb}{0.5,0.5,0.5}
\definecolor{codepurple}{rgb}{0.58,0,0.82}
\definecolor{backcolour}{rgb}{0.95,0.95,0.92}
 
\lstdefinestyle{mystyle}{
    backgroundcolor=\color{backcolour},   
    commentstyle=\color{codegreen},
    keywordstyle=\color{magenta},
    numberstyle=\tiny\color{codegray},
    stringstyle=\color{codepurple},
    basicstyle=\footnotesize,
    breakatwhitespace=false,         
    breaklines=true,                 
    captionpos=b,                    
    keepspaces=true,                 
    numbers=left,                    
    numbersep=5pt,                  
    showspaces=false,                
    showstringspaces=false,
    showtabs=false,                  
    tabsize=2
}
 
\lstset{style=mystyle}

\usepackage{ragged2e}
\setbeamercolor{section in foot}{fg=white,bg=darkorange}
\setbeamercolor{subsection in foot}{fg=white,bg=darkorange}
\setbeamercolor{frametitle}{fg=white, bg=darkorange}
\setbeamercolor{title}{fg=white, bg=darkorange}
\setbeamercolor{frame}{bg=darkorange}
\setbeamercolor{block title}{bg=darkorange,fg=white}

\setbeamercolor{item}{fg=darkorange}

% \definecolor{darkorange}{rgb}{0.81, 0.52, 0.05}
\definecolor{darkorange}{rgb}{1,0.5,0}
\definecolor{darkorange2}{rgb}{1, 0.64, 0.2}
\definecolor{honeydew}{rgb}{1, 0.85, 0.45}


\newenvironment{variableblock}[3]{%
  \setbeamercolor{block body}{#2}
  \setbeamercolor{block title}{#3}
  \begin{block}{#1}}{\end{block}}

\newenvironment{prosblock}[1]{%
  % \setbeamercolor{block body}{bg=blue,fg=white}
  \setbeamercolor{block title}{bg=blue,fg=white}
  \begin{block}{#1}}{\end{block}}

\newenvironment{consblock}[1]{%
  % \setbeamercolor{block body}{bg=red,fg=white}
  \setbeamercolor{block title}{bg=red,fg=white}
  \begin{block}{#1}}{\end{block}}

\newcommand{\cmark}{\ding{51}}%
\newcommand{\xmark}{\ding{55}}%

\renewcommand{\arraystretch}{1.5}

% Please add the following required packages to your document preamble:
\usepackage{booktabs}
\usepackage{multirow}
\usepackage{colortbl}
% Beamer presentation requires \usepackage{colortbl} instead of \usepackage[table,xcdraw]{xcolor}

\usepackage{tabularray}\UseTblrLibrary{varwidth}
\usepackage{xcolor}
\def\BibTeX{{\rm B\kern-.05em{\sc i\kern-.025em b}\kern-.08em
    T\kern-.1667em\lower.7ex\hbox{E}\kern-.125emX}}
% \usepackage{cite}
\usepackage{amsmath}
\newcommand{\probP}{\text{I\kern-0.15em P}}
\usepackage{etoolbox}
\patchcmd{\thebibliography}{\section*{\refname}}{}{}{}

\setlength\tabcolsep{0.5pt}

\renewcommand{\arraystretch}{0.9}
\setlength{\tabcolsep}{2pt}

\usepackage{pgffor}
\usepackage[absolute,overlay]{textpos}
\setlength{\TPHorizModule}{1cm}
\setlength{\TPVertModule}{1cm}

\setbeamerfont{bibliography item}{size=\tiny}
\setbeamerfont{bibliography entry author}{size=\tiny}
\setbeamerfont{bibliography entry title}{size=\tiny}
\setbeamerfont{bibliography entry location}{size=\tiny}
\setbeamerfont{bibliography entry note}{size=\tiny}

\setbeamerfont{bibliography entry author}{shape=\upshape,series=\mdseries,size=\footnotesize}
\setbeamerfont{bibliography entry title}{shape=\slshape,series=\mdseries,size=\footnotesize}
\setbeamerfont{bibliography entry journal}{shape=\upshape,series=\mdseries,size=\footnotesize}
\setbeamerfont{bibliography entry note}{shape=\upshape,series=\mdseries,size=\footnotesize}

\renewcommand*{\bibfont}{\scriptsize}

\newenvironment<>{varblock}[2][.9\textwidth]{%
  \setlength{\textwidth}{#1}
  \begin{actionenv}#3%
    \def\insertblocktitle{#2}%
    \par%
    \usebeamertemplate{block begin}}
  {\par%
    \usebeamertemplate{block end}%
  \end{actionenv}}

% \setbeamertemplate{footline}[frame number]

\setbeamertemplate{footline}{
  \leavevmode%
  \hfill
  \usebeamercolor[fg]{page number in head/foot}%
  \scriptsize%
  \ifnum\value{framenumber}>23%
    Appendix \number\numexpr\value{framenumber}-32\relax/32%
  \else%
    \ifnum\value{framenumber}>20%
      %
    \else
      \number\numexpr\value{framenumber}\relax/20%
    \fi

  \fi%
  \hspace{1em}
}

\begin{document}

\author{\textbf{Julien Soulé$^{1,2}$}, Jean-Paul Jamont$^1$, Michel Occello$^1$, Louis-Marie Traonouez$^2$, Paul Théron$^3$}

\title{\textbf{Towards Assisted MAS Design: A Library for
Explainable MARL with Organizational Model}}

\subtitle{ECAI 2024 Demo Presentation}

% \logo{\includegraphics[scale=0.01]{figures/grenoble-inp_logo.png}}

\institute{\footnotesize \textit{University Grenoble Alpes, Grenoble INP, LCIS, 26000, Valence, France \\
$^1$\{julien.soule, jean-paul.jamont, michel.occello\}@lcis.grenoble-inp.fr \\ \phantom{U} \\
Thales Land and Air Systems, BL IAS, 35000, Rennes, France \\
$^2$\{julien.soule, louis-marie.traonouez\}@thalesgroup.com \\ \phantom{U} \\
AICA IWG, La Guillermie, France \\
$^3$paul.theron@orange.fr}}


\date{\textit{\footnotesize May 9, 2024}}

%\subject{}
\setbeamercovered{transparent}
%\setbeamertemplate{navigation symbols}{}
\begin{frame}[plain]
	\maketitle\vspace{-0.8cm}
	\begin{figure}[ht!]
		\centering
            \includegraphics[height=0.8cm]{figures/la-ruche_logo.png}
            \hspace{0.8cm}
            \includegraphics[height=0.8cm]{figures/lcis_logo.png}
            \hspace{0.8cm}
		\includegraphics[height=0.8cm]{figures/grenoble-inp_logo.png}
            \hspace{0.8cm}
            \includegraphics[height=0.8cm]{figures/uga_logo.jpg}
	\end{figure}
\end{frame}

% \begin{frame}{Content}
%   \tableofcontents
% \end{frame}

\addtocounter{framenumber}{-1}

\section{Introduction}

\begin{frame}{Contexte et problématique générale}

  \begin{itemize}
    \item \textbf{Augmentation surface attaque :} combinaison d'attaques, environnement distribué \& hétérogène, complexe, autonomisation des attaques\dots
    \item \textbf{Défis Cyberdéfense :} interruption communication, effectif opérateurs limité, coût\dots
  \end{itemize}

  \begin{exampleblock}{\textbf{AICA} : Autonomous Intelligent Cyberdefense Agent$^{1}$~\cite{AICAGuide2022}}
    \begin{itemize}
      \item Détecter anomalies, Etablir contremesures, Autonome, Capable d'apprendre, Discret\dots
      \item {\textbf{MASCARA} : Multi-Agent System Centric AICA Reference Architecture~\parencite{kott2018autonomous}}
            \begin{itemize}
              \item Une architecture modulaire fixe d'un AICA mono-agent
            \end{itemize}
    \end{itemize}
  \end{exampleblock}

  $\rightarrow$ Vers un AICA comme un \textbf{Système Multi-Agent de Cyberdéfense} :
  \begin{itemize}
    \item Flexibilité, Autonomie, Pas de SPOF, Auto-organisation/Re-organisation\dots
  \end{itemize}

  \medskip

  \begin{alertblock}{Problématique générale}
    Concevoir un SMA de Cyberdéfense auto-organisé capable de maximiser les capacités d'un AICA tout en s'adaptant continuellement aux :
    \begin{itemize}
      \item Contraintes dynamiques de l'environnement (incluant les attaques)
      \item Exigences architecturales de l'AICA
    \end{itemize}
  \end{alertblock}

  {\scriptsize \textit{Initiative OTAN IST-152 \& AICA IWG : \url{https://www.aica-iwg.org/}}}

\end{frame}

\begin{frame}{Approche de la problématique}

  Revue des travaux liés~\parencite{soule2023ressi}
  \begin{itemize}
    \item Approche multi-agent très peu explorée (prévalence mono-agent)
    \item Pas de travaux sur organisation des SMA de Cyberdéfense
    \item \textbf{Problèmes} : Coût conception, apprentissage, formalisation, généralisabilité, passage à l'échelle, autonomie\dots
    \item Travaux orientés \textbf{Machine Learning} (ML) plus avancés~\parencite{Hammar2022}
  \end{itemize}

  \medskip

  \textbf{Pivot :} Approche empirique / IA Symbolique $\rightarrow$ Approche mixte ML / IA Symbolique
  \begin{itemize}
    \item Permet d'addresser problèmes précédents mais problèmes \textbf{Explicabilité} et \textbf{Contrôle}
  \end{itemize}

  \medskip

  \begin{alertblock}{Cadre d'un problème optimisation sous-contraintes \& sous-problèmes}
    Quel mécanisme permet, à partir d'un environnement réel, d'optimiser les politiques des agents pour maximiser l'atteinte d'un objectif de Cyberdéfense sous les contraintes de l'environnement et de l'AICA.
    \begin{itemize}
      \item P1: Modélisation en simulation
      \item P2: Sûreté/Controle/Guidage
      \item P3: Explicabilité agents
      \item P4: Ecart simulation-réalité
    \end{itemize}
  \end{alertblock}

\end{frame}

\begin{frame}{Hypothèses et contribution principale}
  \textbf{Hypothèses :}
  \begin{itemize}
    \item H1 : La problématique peut être formalisée sur la base d'un modèle Markovien et résolue avec techniques \textbf{Multi-Agent Reinforcement Learning} (MARL) $\rightarrow$ (P1)
    \item H2 : Techniques \textbf{World Models} peut permettre modéliser l'environnement en simulation avec un écart faible entre simulation et réalité $\rightarrow$ (P1, P4)
    \item H3 : Contraintes peuvent être organisées dans un \textbf{modèle organisationel} et intgrées dans processus MARL $\rightarrow$ (P2)
    \item H4  : Comportements peuvent être analysées via techniques \textbf{"Unsupervised Learning" et un modèle organisationel} $\rightarrow$ (P3)
  \end{itemize}

  \medskip

  Intégration de ces hypothèses dans une méthode de conception d'un SMA de Cyberdéfense \\

  \

  \ \ \ $\rightarrow$ \textbf{MOISE+MARL Assisted MAS Design (MAMAD)}

\end{frame}


\section{Méthode de conception : MAMAD}


\begin{frame}{Aperçu général}

  \begin{columns}[c]

    \hspace{-1.5cm}

    \begin{column}{0.5\textwidth}
      \begin{enumerate}
        \item Modéliser l'environnement à partir de traces réelles ;
        \item Entraînement des agents à l'aide de MARL + Contraintes ;
        \item Analyse pour rôles et objectifs émergents des agents ;
        \item Transferer politiques pour piloter les actionneurs des agents + génère nouvelles traces $\rightarrow$ modèle simulé plus fidèle.
      \end{enumerate}

    \end{column}

    \hspace{-1.5cm}

    \begin{column}{0.5\textwidth}
      \begin{figure}[h!]
        \centering
        \includegraphics[width=1.2\linewidth]{figures/mamad_framework.png}
        \caption{Cycle de vie d'un SMA conçu avec MAMAD}
        \label{fig:cycle}
      \end{figure}
    \end{column}
  \end{columns}

\end{frame}

\begin{frame}{Modélisation en simulation}{Cadre}

  \textbf{Un environnement en réseau}
  \begin{itemize}
    \item Propriétés : ouvert, dynamique/statique, déterministe, accessible/inaccessible\dots
          \phantom{XXXXXXXXXXXXXXXXXXXXXXXXXXXXXXXXXXXXXXXXXXXXXXXXXXXXXXXXXXXXXXXX}
  \end{itemize}

  \includegraphics[width=\linewidth, trim=0cm 0cm 0cm 0.1cm, clip]{figures/0.png}

\end{frame}

\begin{frame}{Modélisation en simulation}{Cadre}

  \textbf{Avec une Green Team}
  \begin{itemize}
    \item Utilisateurs "normaux" : sessions, requêtes, scan des ports, envoi de données
          \phantom{XXXXXXXXXXXXXXXXXXXXXXXXXXXXXXXXXXXXXXXXXXXXXXXXXXXXXXXXXXXXXXXX}
  \end{itemize}

  \includegraphics[width=\linewidth, trim=0cm 0cm 0cm 0.1cm, clip]{figures/1.png}
\end{frame}

\begin{frame}{Modélisation en simulation}{Cadre}

  \textbf{Avec une Red Team}
  \begin{itemize}
    \item Cyber-attaquants : découverte de nœuds/services, exploitation des vulnérabilités, escalade de privilège, déplacement latéral, impact\dots
  \end{itemize}

  \includegraphics[width=\linewidth, trim=0cm 0cm 0cm 0.1cm, clip]{figures/2.png}
\end{frame}

\begin{frame}{Modélisation en simulation}{Cadre}

  \textbf{Avec une Blue Team}
  \begin{itemize}
    \item Cyber-défenseurs : analyse de menace, contrôle des accès, tuer des processus suspects, mise en place de "honey pots", re-imager un nœud\dots
  \end{itemize}

  \includegraphics[width=\linewidth, trim=0cm 0cm 0cm 0.1cm, clip]{figures/3.png}
\end{frame}

\begin{frame}{Modélisation en simulation}{Modèle Markovien}

  \begin{columns}[c]

    \hspace{-1.cm}

    \begin{column}{0.5\textwidth}

      \begin{itemize}
        \item Modèle Markovien (Dec-POMDP)
              \begin{itemize}
                \item Prise de décision collective
                \item Modélisation de incertitude des actions / observations
                \item \textbf{T : Fonction transition état}
              \end{itemize}

        \item[\phantom{X}] \phantom{Modélisation manuelle de T}
          \begin{itemize}
            \item[\phantom{X}] \phantom{CybORG}
          \end{itemize}
        \item[\phantom{X}] \phantom{Modélisation automatisée de T}
          \begin{enumerate}
            \item[\phantom{X}] \phantom{Collecte traces réelles}
            \item[\phantom{X}] \phantom{Entrainement via un RNN (LSTM)}
          \end{enumerate}
      \end{itemize}

    \end{column}

    \hspace{-1.5cm}

    \begin{column}{0.6\textwidth}
      \centering
      \includegraphics[width=\linewidth]{figures/marl_framework.png}
    \end{column}
  \end{columns}

  \medskip

  {\tiny \textit{Soulé, J., Jamont, J.-P., Occello, M., Théron, P., \& Traonouez, L.-M. Towards a Multi-Agent Simulation of Cyber-attackers and Cyber-defenders Battles. IEEE SMC 2023.}}

\end{frame}


\begin{frame}{Modélisation en simulation}{Modèle Markovien}

  \begin{columns}[c]

    \hspace{-1.cm}

    \begin{column}{0.5\textwidth}

      \begin{itemize}
        \item Modèle Markovien (Dec-POMDP)
              \begin{itemize}
                \item Prise de décision collective
                \item Modélisation de incertitude des actions / observations
                \item \textbf{T : Fonction transition état}
              \end{itemize}

        \item Modélisation manuelle de T
              \begin{itemize}
                \item CybORG
              \end{itemize}
        \item[\phantom{X}] \phantom{Modélisation automatisée de T}
          \begin{enumerate}
            \item[\phantom{X}] \phantom{Collecte traces réelles}
            \item[\phantom{X}] \phantom{Entrainement via un RNN (LSTM)}
          \end{enumerate}
      \end{itemize}

    \end{column}

    \hspace{-1.5cm}

    \begin{column}{0.6\textwidth}
      \centering
      \includegraphics[width=\linewidth]{figures/marl_framework.png}
    \end{column}
  \end{columns}

  \medskip
  {\tiny \textit{Soulé, J., Jamont, J.-P., Occello, M., Théron, P., \& Traonouez, L.-M. Towards a Multi-Agent Simulation of Cyber-attackers and Cyber-defenders Battles. IEEE SMC 2023.}}

\end{frame}


\begin{frame}{Modélisation en simulation}{Modèle Markovien}

  \begin{columns}[c]

    \hspace{-1.cm}

    \begin{column}{0.5\textwidth}

      \begin{itemize}
        \item Modèle Markovien (Dec-POMDP)
              \begin{itemize}
                \item Prise de décision collective
                \item Modélisation de incertitude des actions / observations
                \item \textbf{T : Fonction transition état}
              \end{itemize}

        \item Modélisation manuelle de T
              \begin{itemize}
                \item CybORG
              \end{itemize}
        \item Modélisation automatisée de T
              \begin{enumerate}
                \item Collecte traces réelles
                \item Entrainement via un RNN (LSTM)
              \end{enumerate}
      \end{itemize}

    \end{column}

    \hspace{-1.5cm}

    \begin{column}{0.6\textwidth}
      \centering
      \includegraphics[width=\linewidth]{figures/marl_framework.png}
    \end{column}
  \end{columns}

  \medskip

  {\tiny \textit{Soulé, J., Jamont, J.-P., Occello, M., Théron, P., \& Traonouez, L.-M. Towards a Multi-Agent Simulation of Cyber-attackers and Cyber-defenders Battles. IEEE SMC 2023.}}

\end{frame}



\begin{frame}{Apprentissage guidé/contraint}

  \textbf{MARL "Vanilla"}
  \begin{itemize}
    \item Choisir les meilleures actions pour maximiser récompense cumulée
  \end{itemize}

  \begin{center}
    \includegraphics[width=0.5\linewidth]{figures/vanilla_marl.png}
  \end{center}

  \vfill

  {\tiny \textit{Soulé, J., Jamont, J.-P., Occello, M., Traonouez, L.-M., \& Théron, P. An Organizationally-Oriented Approach to Enhancing Explainability and Control in Multi-Agent Reinforcement Learning. Proceedings of the 24th International Conference on Autonomous Agents and Multiagent Systems (AAMAS 2025), Detroit, Michigan, USA, 2025.}}
\end{frame}

\begin{frame}{Apprentissage guidé/contraint}

  \textbf{MARL + Organisation ($\mathcal{M}OISE^+$)}
  \begin{itemize}
    \item Rôle : imposer/refuser actions $\rightarrow$ garantie de sûreté
  \end{itemize}

  \begin{center}
    \includegraphics[width=0.5\linewidth]{figures/role_marl.png}
  \end{center}

  \vfill

  {\tiny \textit{Soulé, J., Jamont, J.-P., Occello, M., Traonouez, L.-M., \& Théron, P. An Organizationally-Oriented Approach to Enhancing Explainability and Control in Multi-Agent Reinforcement Learning. Proceedings of the 24th International Conference on Autonomous Agents and Multiagent Systems (AAMAS 2025), Detroit, Michigan, USA, 2025.}}
\end{frame}

\begin{frame}{Apprentissage guidé/contraint}

  \textbf{MARL + Organisation ($\mathcal{M}OISE^+$)}
  \begin{itemize}
    \item Objectif : inciter à atteindre un objectif intermédiaire
  \end{itemize}

  \begin{center}
    \includegraphics[width=0.5\linewidth]{figures/goal_marl.png}
  \end{center}

  \vfill

  {\tiny \textit{Soulé, J., Jamont, J.-P., Occello, M., Traonouez, L.-M., \& Théron, P. An Organizationally-Oriented Approach to Enhancing Explainability and Control in Multi-Agent Reinforcement Learning. Proceedings of the 24th International Conference on Autonomous Agents and Multiagent Systems (AAMAS 2025), Detroit, Michigan, USA, 2025.}}
\end{frame}


\begin{frame}{Analyse post-entrainement}

  \textbf{Trajectory-based Evaluation in MOISE+MARL (TEMM)}
  \begin{itemize}
    \item \textbf{Objectif} : Fournir une interprétation a posteriori du comportement des agents à un niveau organisationnel.
  \end{itemize}

  \vspace{1em}
  \textbf{Hypothèses sous-jacentes :}
  \begin{itemize}
    \item \textbf{Rôles} $\sim$ motifs fréquents de transitions \emph{(observation, action)} dans les trajectoires des agents.
    \item \textbf{Objectifs} $\sim$ observations reçues fréquemment au sein des trajectoires des agents.
  \end{itemize}

  \vspace{0.8cm}
  \begin{center}
    \begin{columns}[c]

      \begin{column}{0.4\textwidth}
        \centering
        


\tikzset{every picture/.style={line width=0.75pt}} %set default line width to 0.75pt        

\begin{tikzpicture}[x=0.75pt,y=0.75pt,yscale=-1,xscale=1]
    %uncomment if require: \path (0,1974); %set diagram left start at 0, and has height of 1974

    %Shape: Rectangle [id:dp9996076613305621] 
    \draw  [fill={rgb, 255:red, 255; green, 255; blue, 255 }  ,fill opacity=1 ] (24,1558.11) -- (176.1,1558.11) -- (176.1,1644) -- (24,1644) -- cycle ;
    %Straight Lines [id:da05824332013205091] 
    \draw [color={rgb, 255:red, 208; green, 2; blue, 27 }  ,draw opacity=1 ]   (142.67,1570.84) -- (124.28,1577.49) -- (87.26,1592.41) -- (108.68,1604.12) -- (93.53,1601.36) -- (86.58,1603.01) -- (86.58,1612.77) -- (82.05,1616.07) -- (81.22,1616.67) -- (78.65,1614.8) -- (70.51,1608.86) -- (54.44,1608.86) -- (57,1610.73) -- (49.09,1612.77) -- (51.85,1616.79) -- (38.38,1628.38) ;
    \draw [shift={(145.49,1569.82)}, rotate = 160.12] [fill={rgb, 255:red, 208; green, 2; blue, 27 }  ,fill opacity=1 ][line width=0.08]  [draw opacity=0] (3.57,-1.72) -- (0,0) -- (3.57,1.72) -- cycle    ;
    %Straight Lines [id:da9249559779542824] 
    \draw [color={rgb, 255:red, 80; green, 227; blue, 194 }  ,draw opacity=1 ]   (143.47,1568.13) -- (134.78,1577.63) -- (113.36,1577.63) -- (113.36,1585.44) -- (86.58,1593.25) -- (91.93,1597.15) -- (97.29,1604.96) -- (81.22,1601.06) -- (86.58,1608.86) -- (81.22,1608.86) -- (86.58,1616.67) -- (75.87,1616.67) -- (67.94,1608.94) -- (65.16,1614.72) -- (43.73,1603.01) -- (59.8,1616.67) -- (43.73,1608.86) -- (49.09,1616.67) -- (43.73,1632.29) ;
    \draw [shift={(145.49,1565.92)}, rotate = 132.45] [fill={rgb, 255:red, 80; green, 227; blue, 194 }  ,fill opacity=1 ][line width=0.08]  [draw opacity=0] (3.57,-1.72) -- (0,0) -- (3.57,1.72) -- cycle    ;
    %Straight Lines [id:da17118391857757054] 
    \draw [color={rgb, 255:red, 248; green, 231; blue, 28 }  ,draw opacity=1 ]   (153.23,1574.14) -- (126.83,1577.75) -- (124.07,1581.54) -- (105.41,1585.56) -- (91.93,1593.25) -- (93.53,1601.36) -- (89.34,1605.08) -- (81.22,1597.15) -- (91.93,1612.77) -- (91.93,1616.67) -- (81.22,1616.67) -- (59.99,1609.06) -- (57.21,1614.84) -- (41.14,1616.79) -- (35.78,1632.41) ;
    \draw [shift={(156.2,1573.73)}, rotate = 172.2] [fill={rgb, 255:red, 248; green, 231; blue, 28 }  ,fill opacity=1 ][line width=0.08]  [draw opacity=0] (3.57,-1.72) -- (0,0) -- (3.57,1.72) -- cycle    ;
    %Straight Lines [id:da6427777277243145] 
    \draw [color={rgb, 255:red, 144; green, 19; blue, 254 }  ,draw opacity=1 ]   (160.23,1577.39) -- (165.84,1578.41) -- (161.56,1573.73) -- (157.27,1570.61) -- (164.86,1573.73) -- (170.13,1578.41) -- (161.56,1583.1) -- (166.91,1589.34) -- (161.56,1597.15) -- (166.91,1601.06) -- (161.56,1612.77) -- (161.56,1628.38) -- (145.49,1624.48) -- (134.78,1624.48) -- (128.99,1623.07) -- (125.92,1622.33) -- (121.57,1621.27) -- (118.71,1620.58) -- (107.96,1619.71) -- (99.43,1619.01) -- (95.23,1618.25) -- (86.58,1616.67) -- (75.87,1616.67) -- (70.51,1620.58) -- (59.8,1624.48) -- (59.8,1632.29) ;
    \draw [shift={(157.27,1576.85)}, rotate = 10.33] [fill={rgb, 255:red, 144; green, 19; blue, 254 }  ,fill opacity=1 ][line width=0.08]  [draw opacity=0] (3.57,-1.72) -- (0,0) -- (3.57,1.72) -- cycle    ;
    %Straight Lines [id:da7390021320622445] 
    \draw [color={rgb, 255:red, 65; green, 117; blue, 5 }  ,draw opacity=1 ]   (159.15,1578.17) -- (164.77,1579.19) -- (160.49,1574.51) -- (156.2,1571.39) -- (163.79,1574.51) -- (169.06,1579.19) -- (161.56,1587.78) -- (165.84,1590.13) -- (163.7,1601.84) -- (150.85,1597.15) -- (161.56,1603.4) -- (174.41,1615.89) -- (157.27,1606.52) -- (160.49,1613.55) -- (163.7,1625.26) -- (152.99,1628.38) -- (135.85,1622.14) -- (123,1622.14) -- (116.57,1622.14) -- (110.14,1620.58) -- (108,1623.7) -- (103.72,1620.58) -- (105.86,1625.26) -- (94.16,1619.03) -- (85.51,1617.45) -- (74.8,1617.45) -- (69.44,1621.36) -- (58.73,1625.26) -- (58.73,1633.07) ;
    \draw [shift={(156.2,1577.63)}, rotate = 10.33] [fill={rgb, 255:red, 65; green, 117; blue, 5 }  ,fill opacity=1 ][line width=0.08]  [draw opacity=0] (3.57,-1.72) -- (0,0) -- (3.57,1.72) -- cycle    ;
    %Shape: Ellipse [id:dp30050508180239144] 
    \draw  [draw opacity=0][fill={rgb, 255:red, 208; green, 2; blue, 27 }  ,fill opacity=0.62 ] (46.49,1615.89) .. controls (46.49,1614.6) and (47.93,1613.55) .. (49.71,1613.55) .. controls (51.48,1613.55) and (52.92,1614.6) .. (52.92,1615.89) .. controls (52.92,1617.19) and (51.48,1618.23) .. (49.71,1618.23) .. controls (47.93,1618.23) and (46.49,1617.19) .. (46.49,1615.89) -- cycle ;
    %Shape: Ellipse [id:dp15311501498248647] 
    \draw  [draw opacity=0][fill={rgb, 255:red, 208; green, 2; blue, 27 }  ,fill opacity=0.62 ] (90.49,1619.03) .. controls (90.49,1617.74) and (91.93,1616.69) .. (93.71,1616.69) .. controls (95.48,1616.69) and (96.92,1617.74) .. (96.92,1619.03) .. controls (96.92,1620.32) and (95.48,1621.37) .. (93.71,1621.37) .. controls (91.93,1621.37) and (90.49,1620.32) .. (90.49,1619.03) -- cycle ;
    %Shape: Ellipse [id:dp19167487081496637] 
    \draw  [draw opacity=0][fill={rgb, 255:red, 208; green, 2; blue, 27 }  ,fill opacity=0.62 ] (161.11,1606.52) .. controls (161.11,1605.23) and (162.54,1604.18) .. (164.32,1604.18) .. controls (166.09,1604.18) and (167.53,1605.23) .. (167.53,1606.52) .. controls (167.53,1607.82) and (166.09,1608.86) .. (164.32,1608.86) .. controls (162.54,1608.86) and (161.11,1607.82) .. (161.11,1606.52) -- cycle ;
    %Shape: Ellipse [id:dp9201279867822619] 
    \draw  [draw opacity=0][fill={rgb, 255:red, 208; green, 2; blue, 27 }  ,fill opacity=0.62 ] (120.4,1622.92) .. controls (120.4,1621.62) and (121.84,1620.58) .. (123.62,1620.58) .. controls (125.39,1620.58) and (126.83,1621.62) .. (126.83,1622.92) .. controls (126.83,1624.21) and (125.39,1625.26) .. (123.62,1625.26) .. controls (121.84,1625.26) and (120.4,1624.21) .. (120.4,1622.92) -- cycle ;
    %Shape: Ellipse [id:dp3048334813609519] 
    \draw  [draw opacity=0][fill={rgb, 255:red, 208; green, 2; blue, 27 }  ,fill opacity=0.62 ] (161.11,1590.91) .. controls (161.11,1589.61) and (162.54,1588.56) .. (164.32,1588.56) .. controls (166.09,1588.56) and (167.53,1589.61) .. (167.53,1590.91) .. controls (167.53,1592.2) and (166.09,1593.25) .. (164.32,1593.25) .. controls (162.54,1593.25) and (161.11,1592.2) .. (161.11,1590.91) -- cycle ;
    %Shape: Ellipse [id:dp7290465976812913] 
    \draw  [draw opacity=0][fill={rgb, 255:red, 208; green, 2; blue, 27 }  ,fill opacity=0.62 ] (86.13,1603.4) .. controls (86.13,1602.11) and (87.56,1601.06) .. (89.34,1601.06) .. controls (91.11,1601.06) and (92.55,1602.11) .. (92.55,1603.4) .. controls (92.55,1604.69) and (91.11,1605.74) .. (89.34,1605.74) .. controls (87.56,1605.74) and (86.13,1604.69) .. (86.13,1603.4) -- cycle ;
    %Shape: Ellipse [id:dp6154487622646608] 
    \draw  [draw opacity=0][fill={rgb, 255:red, 208; green, 2; blue, 27 }  ,fill opacity=0.62 ] (109.69,1583.1) .. controls (109.69,1581.8) and (111.13,1580.76) .. (112.9,1580.76) .. controls (114.68,1580.76) and (116.12,1581.8) .. (116.12,1583.1) .. controls (116.12,1584.39) and (114.68,1585.44) .. (112.9,1585.44) .. controls (111.13,1585.44) and (109.69,1584.39) .. (109.69,1583.1) -- cycle ;
    %Shape: Ellipse [id:dp6108483574180856] 
    \draw  [draw opacity=0][fill={rgb, 255:red, 189; green, 16; blue, 224 }  ,fill opacity=0.8 ] (77.56,1615.89) .. controls (77.56,1614.6) and (79,1613.55) .. (80.77,1613.55) .. controls (82.55,1613.55) and (83.98,1614.6) .. (83.98,1615.89) .. controls (83.98,1617.19) and (82.55,1618.23) .. (80.77,1618.23) .. controls (79,1618.23) and (77.56,1617.19) .. (77.56,1615.89) -- cycle ;
    %Shape: Ellipse [id:dp08863924891219843] 
    \draw  [draw opacity=0][fill={rgb, 255:red, 208; green, 2; blue, 27 }  ,fill opacity=0.62 ] (84.52,1609.06) .. controls (84.52,1607.77) and (85.96,1606.72) .. (87.73,1606.72) .. controls (89.51,1606.72) and (90.95,1607.77) .. (90.95,1609.06) .. controls (90.95,1610.35) and (89.51,1611.4) .. (87.73,1611.4) .. controls (85.96,1611.4) and (84.52,1610.35) .. (84.52,1609.06) -- cycle ;
    %Shape: Ellipse [id:dp49807154634681794] 
    \draw  [draw opacity=0][fill={rgb, 255:red, 208; green, 2; blue, 27 }  ,fill opacity=0.62 ] (91.21,1601.64) .. controls (91.21,1600.35) and (92.65,1599.3) .. (94.43,1599.3) .. controls (96.2,1599.3) and (97.64,1600.35) .. (97.64,1601.64) .. controls (97.64,1602.94) and (96.2,1603.98) .. (94.43,1603.98) .. controls (92.65,1603.98) and (91.21,1602.94) .. (91.21,1601.64) -- cycle ;
    %Shape: Ellipse [id:dp17062416794692736] 
    \draw  [draw opacity=0][fill={rgb, 255:red, 208; green, 2; blue, 27 }  ,fill opacity=0.62 ] (100.59,1619.41) .. controls (100.59,1618.11) and (102.03,1617.06) .. (103.8,1617.06) .. controls (105.57,1617.06) and (107.01,1618.11) .. (107.01,1619.41) .. controls (107.01,1620.7) and (105.57,1621.75) .. (103.8,1621.75) .. controls (102.03,1621.75) and (100.59,1620.7) .. (100.59,1619.41) -- cycle ;
    %Shape: Ellipse [id:dp05427293190477478] 
    \draw  [draw opacity=0][fill={rgb, 255:red, 189; green, 16; blue, 224 }  ,fill opacity=0.8 ] (152.54,1626.82) .. controls (152.54,1625.53) and (153.98,1624.48) .. (155.75,1624.48) .. controls (157.52,1624.48) and (158.96,1625.53) .. (158.96,1626.82) .. controls (158.96,1628.12) and (157.52,1629.16) .. (155.75,1629.16) .. controls (153.98,1629.16) and (152.54,1628.12) .. (152.54,1626.82) -- cycle ;
    %Shape: Ellipse [id:dp0565658994925915] 
    \draw  [draw opacity=0][fill={rgb, 255:red, 208; green, 2; blue, 27 }  ,fill opacity=0.62 ] (158.96,1616.67) .. controls (158.96,1615.38) and (160.4,1614.33) .. (162.18,1614.33) .. controls (163.95,1614.33) and (165.39,1615.38) .. (165.39,1616.67) .. controls (165.39,1617.97) and (163.95,1619.01) .. (162.18,1619.01) .. controls (160.4,1619.01) and (158.96,1617.97) .. (158.96,1616.67) -- cycle ;
    %Shape: Ellipse [id:dp5007110255270828] 
    \draw  [draw opacity=0][fill={rgb, 255:red, 189; green, 16; blue, 224 }  ,fill opacity=0.8 ] (57,1610.73) .. controls (57,1609.43) and (58.44,1608.38) .. (60.21,1608.38) .. controls (61.99,1608.38) and (63.43,1609.43) .. (63.43,1610.73) .. controls (63.43,1612.02) and (61.99,1613.07) .. (60.21,1613.07) .. controls (58.44,1613.07) and (57,1612.02) .. (57,1610.73) -- cycle ;
    %Shape: Ellipse [id:dp22598728144573377] 
    \draw  [draw opacity=0][fill={rgb, 255:red, 208; green, 2; blue, 27 }  ,fill opacity=0.62 ] (88.72,1595.59) .. controls (88.72,1594.3) and (90.16,1593.25) .. (91.93,1593.25) .. controls (93.71,1593.25) and (95.15,1594.3) .. (95.15,1595.59) .. controls (95.15,1596.88) and (93.71,1597.93) .. (91.93,1597.93) .. controls (90.16,1597.93) and (88.72,1596.88) .. (88.72,1595.59) -- cycle ;
    %Shape: Ellipse [id:dp14749486568088088] 
    \draw  [draw opacity=0][fill={rgb, 255:red, 189; green, 16; blue, 224 }  ,fill opacity=0.8 ] (93.54,1589.34) .. controls (93.54,1588.05) and (94.98,1587) .. (96.75,1587) .. controls (98.53,1587) and (99.97,1588.05) .. (99.97,1589.34) .. controls (99.97,1590.64) and (98.53,1591.69) .. (96.75,1591.69) .. controls (94.98,1591.69) and (93.54,1590.64) .. (93.54,1589.34) -- cycle ;
    %Shape: Polygon Curved [id:ds6643267525526769] 
    \draw  [color={rgb, 255:red, 74; green, 144; blue, 226 }  ,draw opacity=1 ][fill={rgb, 255:red, 74; green, 144; blue, 226 }  ,fill opacity=0.5 ] (27.21,1628.38) .. controls (30.38,1623.39) and (36.63,1621.99) .. (42.56,1622.18) .. controls (47.05,1622.33) and (51.36,1623.39) .. (53.99,1624.48) .. controls (60.1,1627.02) and (65.56,1626.63) .. (64.7,1632.29) .. controls (63.85,1637.95) and (56.88,1637.17) .. (48.64,1636.19) .. controls (40.39,1635.22) and (21.64,1637.17) .. (27.21,1628.38) -- cycle ;
    %Shape: Polygon Curved [id:ds9461514343962948] 
    \draw  [color={rgb, 255:red, 208; green, 2; blue, 27 }  ,draw opacity=1 ][fill={rgb, 255:red, 208; green, 2; blue, 27 }  ,fill opacity=0.5 ] (139.68,1569.82) .. controls (145.25,1561.04) and (148.01,1561.59) .. (145.04,1565.92) .. controls (142.07,1570.25) and (154.68,1563.87) .. (153.82,1569.53) .. controls (152.97,1575.19) and (169.35,1578.61) .. (161.11,1577.63) .. controls (152.86,1576.66) and (134.11,1578.61) .. (139.68,1569.82) -- cycle ;
    %Shape: Ellipse [id:dp06406072166611776] 
    \draw  [draw opacity=0][fill={rgb, 255:red, 208; green, 2; blue, 27 }  ,fill opacity=0.62 ] (71.13,1617.45) .. controls (71.13,1616.16) and (72.57,1615.11) .. (74.34,1615.11) .. controls (76.12,1615.11) and (77.56,1616.16) .. (77.56,1617.45) .. controls (77.56,1618.75) and (76.12,1619.8) .. (74.34,1619.8) .. controls (72.57,1619.8) and (71.13,1618.75) .. (71.13,1617.45) -- cycle ;
    %Shape: Ellipse [id:dp049221150011381054] 
    \draw  [draw opacity=0][fill={rgb, 255:red, 189; green, 16; blue, 224 }  ,fill opacity=0.8 ] (56.59,1624.48) .. controls (56.59,1623.19) and (58.03,1622.14) .. (59.8,1622.14) .. controls (61.58,1622.14) and (63.01,1623.19) .. (63.01,1624.48) .. controls (63.01,1625.77) and (61.58,1626.82) .. (59.8,1626.82) .. controls (58.03,1626.82) and (56.59,1625.77) .. (56.59,1624.48) -- cycle ;


    % Text Node
    \draw (68.95,1583.75) node  [font=\tiny,color={rgb, 255:red, 189; green, 16; blue, 224 }  ,opacity=1 ] [align=left] {$\displaystyle g_{5} =\{\omega _{21} \dotsc \}$};
    % Text Node
    \draw (82.63,1627.35) node  [font=\tiny,color={rgb, 255:red, 189; green, 16; blue, 224 }  ,opacity=1 ] [align=left] {$\displaystyle g_{2} =\{\omega _{5}\}$};
    % Text Node
    \draw (152.91,1586.86) node  [font=\tiny,color={rgb, 255:red, 189; green, 16; blue, 224 }  ,opacity=1 ] [align=left] {$\displaystyle ...$};
    % Text Node
    \draw (101.5,1575.93) node  [font=\tiny,color={rgb, 255:red, 189; green, 16; blue, 224 }  ,opacity=1 ] [align=left] {$\displaystyle ...$};
    % Text Node
    \draw (136.45,1636.35) node  [font=\tiny,color={rgb, 255:red, 189; green, 16; blue, 224 }  ,opacity=1 ] [align=left] {$\displaystyle g_{4} =\{\omega _{301} ,\omega _{302}\}$};
    % Text Node
    \draw (113.58,1612.35) node  [font=\tiny,color={rgb, 255:red, 189; green, 16; blue, 224 }  ,opacity=1 ] [align=left] {$\displaystyle g_{3} =\{\omega _{10}\}$};
    % Text Node
    \draw (55.11,1600.35) node  [font=\tiny,color={rgb, 255:red, 189; green, 16; blue, 224 }  ,opacity=1 ] [align=left] {$\displaystyle g_{1} =\{\omega _{1}\}$};
    % Text Node
    \draw (105.58,1568.35) node  [font=\tiny,color={rgb, 255:red, 202; green, 52; blue, 69 }  ,opacity=1 ] [align=left] {$\displaystyle g_{*} =\Omega _{goal}$};
    % Text Node
    \draw (73.43,1637.35) node  [font=\tiny,color={rgb, 255:red, 74; green, 144; blue, 226 }  ,opacity=1 ] [align=left] {$\displaystyle \Omega _{init}$};
    % Text Node
    \draw (32.91,1567.84) node  [font=\scriptsize] [align=left] {$\displaystyle \Omega $};
    % Text Node
    \draw (93.61,1653) node   [align=left] {{\tiny \textit{Une visualisation abstraite des}}};
    \draw (93.61,1662) node   [align=left] {{\tiny \textit{observations dans les trajectoires}}};

\end{tikzpicture}
      \end{column}

      \begin{column}{0.1\textwidth}
      \end{column}

      \begin{column}{0.4\textwidth}
        \centering
        


\tikzset{every picture/.style={line width=0.75pt}} %set default line width to 0.75pt        

\begin{tikzpicture}[x=0.75pt,y=0.75pt,yscale=-1,xscale=1]
%uncomment if require: \path (0,1974); %set diagram left start at 0, and has height of 1974

%Shape: Rectangle [id:dp5335676631264512] 
\draw  [fill={rgb, 255:red, 255; green, 255; blue, 255 }  ,fill opacity=1 ] (190,1560.11) -- (342.1,1560.11) -- (342.1,1646) -- (190,1646) -- cycle ;
%Straight Lines [id:da6623576988919416] 
\draw [color={rgb, 255:red, 208; green, 2; blue, 27 }  ,draw opacity=1 ]   (308.67,1572.84) -- (290.28,1579.49) -- (253.26,1594.41) -- (292.1,1604) -- (274.1,1612) -- (262.1,1616) -- (252.58,1614.77) -- (246.1,1604) -- (240.1,1604) -- (236.1,1606) -- (236.51,1610.86) -- (220.44,1610.86) -- (223,1612.73) -- (215.09,1614.77) -- (217.85,1618.79) -- (204.38,1630.38) ;
\draw [shift={(311.49,1571.82)}, rotate = 160.12] [fill={rgb, 255:red, 208; green, 2; blue, 27 }  ,fill opacity=1 ][line width=0.08]  [draw opacity=0] (3.57,-1.72) -- (0,0) -- (3.57,1.72) -- cycle    ;
%Straight Lines [id:da5424854363807742] 
\draw [color={rgb, 255:red, 80; green, 227; blue, 194 }  ,draw opacity=1 ]   (309.47,1570.13) -- (300.78,1579.63) -- (279.36,1579.63) -- (279.36,1587.44) -- (252.58,1595.25) -- (250.1,1598) -- (248.1,1600) -- (247.22,1603.06) -- (252.58,1610.86) -- (247.22,1610.86) -- (242.1,1608) -- (242.1,1610) -- (233.94,1610.94) -- (231.16,1616.72) -- (209.73,1605.01) -- (225.8,1618.67) -- (209.73,1610.86) -- (215.09,1618.67) -- (209.73,1634.29) ;
\draw [shift={(311.49,1567.92)}, rotate = 132.45] [fill={rgb, 255:red, 80; green, 227; blue, 194 }  ,fill opacity=1 ][line width=0.08]  [draw opacity=0] (3.57,-1.72) -- (0,0) -- (3.57,1.72) -- cycle    ;
%Straight Lines [id:da21186841526109945] 
\draw [color={rgb, 255:red, 248; green, 231; blue, 28 }  ,draw opacity=1 ]   (319.23,1576.14) -- (292.83,1579.75) -- (290.07,1583.54) -- (271.41,1587.56) -- (257.93,1595.25) -- (280.1,1592) -- (284.1,1594) -- (290.1,1604) -- (257.93,1614.77) -- (257.93,1618.67) -- (247.22,1618.67) -- (225.99,1611.06) -- (223.21,1616.84) -- (207.14,1618.79) -- (201.78,1634.41) ;
\draw [shift={(322.2,1575.73)}, rotate = 172.2] [fill={rgb, 255:red, 248; green, 231; blue, 28 }  ,fill opacity=1 ][line width=0.08]  [draw opacity=0] (3.57,-1.72) -- (0,0) -- (3.57,1.72) -- cycle    ;
%Straight Lines [id:da6313290732282947] 
\draw [color={rgb, 255:red, 144; green, 19; blue, 254 }  ,draw opacity=1 ]   (326.23,1579.39) -- (331.84,1580.41) -- (327.56,1575.73) -- (323.27,1572.61) -- (330.86,1575.73) -- (336.13,1580.41) -- (327.56,1585.1) -- (332.91,1591.34) -- (324.1,1600) -- (330.1,1606) -- (312.1,1636) -- (320.1,1638) -- (306.1,1644) -- (306.1,1606) -- (300.1,1614) -- (311.49,1626.48) -- (300.78,1626.48) -- (294.99,1625.07) -- (291.92,1624.33) -- (287.57,1623.27) -- (284.71,1622.58) -- (273.96,1621.71) -- (265.43,1621.01) -- (261.23,1620.25) -- (250.1,1622) -- (241.87,1618.67) -- (236.51,1622.58) -- (225.8,1626.48) -- (225.8,1634.29) ;
\draw [shift={(323.27,1578.85)}, rotate = 10.33] [fill={rgb, 255:red, 144; green, 19; blue, 254 }  ,fill opacity=1 ][line width=0.08]  [draw opacity=0] (3.57,-1.72) -- (0,0) -- (3.57,1.72) -- cycle    ;
%Straight Lines [id:da1305524961942589] 
\draw [color={rgb, 255:red, 65; green, 117; blue, 5 }  ,draw opacity=1 ]   (325.15,1580.17) -- (330.77,1581.19) -- (326.49,1576.51) -- (322.2,1573.39) -- (329.79,1576.51) -- (335.06,1581.19) -- (327.56,1589.78) -- (331.84,1592.13) -- (329.7,1603.84) -- (316.85,1599.15) -- (327.56,1605.4) -- (314.1,1636) -- (318.1,1640) -- (310.1,1642) -- (304.1,1608) -- (298.1,1612) -- (301.85,1624.14) -- (289,1624.14) -- (282.57,1624.14) -- (276.14,1622.58) -- (274,1625.7) -- (269.72,1622.58) -- (271.86,1627.26) -- (260.16,1621.03) -- (251.51,1619.45) -- (240.8,1619.45) -- (235.44,1623.36) -- (224.73,1627.26) -- (224.73,1635.07) ;
\draw [shift={(322.2,1579.63)}, rotate = 10.33] [fill={rgb, 255:red, 65; green, 117; blue, 5 }  ,fill opacity=1 ][line width=0.08]  [draw opacity=0] (3.57,-1.72) -- (0,0) -- (3.57,1.72) -- cycle    ;
%Shape: Polygon Curved [id:ds29559681347985167] 
\draw  [color={rgb, 255:red, 184; green, 233; blue, 134 }  ,draw opacity=0 ][fill={rgb, 255:red, 74; green, 144; blue, 226 }  ,fill opacity=0.75 ] (203.31,1627.26) .. controls (208.88,1618.48) and (203.46,1612.19) .. (210,1608) .. controls (216.54,1603.81) and (229.5,1600.53) .. (248,1604) .. controls (266.5,1607.47) and (269.83,1605.44) .. (280,1604) .. controls (290.17,1602.56) and (250.44,1601.87) .. (252,1594) .. controls (253.56,1586.13) and (258.5,1591.77) .. (262,1588) .. controls (265.5,1584.23) and (314.2,1570.33) .. (316,1570) .. controls (317.8,1569.67) and (320.91,1572.43) .. (314,1576) .. controls (307.09,1579.57) and (294.2,1581.95) .. (288,1584) .. controls (281.8,1586.05) and (277.13,1589.32) .. (276,1590) .. controls (274.87,1590.68) and (280.1,1589.85) .. (284,1592) .. controls (287.9,1594.15) and (295.82,1601.53) .. (296,1602) .. controls (296.18,1602.47) and (264.78,1615.69) .. (264,1616) .. controls (263.22,1616.31) and (249.54,1612.38) .. (244,1612) .. controls (238.46,1611.62) and (217.32,1621.29) .. (214,1626) .. controls (210.68,1630.71) and (210.87,1632.35) .. (210,1636) .. controls (209.13,1639.65) and (197.74,1636.04) .. (203.31,1627.26) -- cycle ;
%Shape: Polygon Curved [id:ds2928272635642186] 
\draw  [color={rgb, 255:red, 208; green, 2; blue, 27 }  ,draw opacity=0 ][fill={rgb, 255:red, 208; green, 2; blue, 27 }  ,fill opacity=0.5 ] (304,1644) .. controls (302.69,1641.5) and (306.85,1644.5) .. (304,1634) .. controls (301.15,1623.5) and (270.08,1628.81) .. (266,1628) .. controls (261.92,1627.19) and (254.9,1623.42) .. (244,1624) .. controls (233.1,1624.58) and (230.1,1635.78) .. (226,1638) .. controls (221.9,1640.22) and (223.11,1630.38) .. (222,1630) .. controls (220.89,1629.62) and (222.67,1626.54) .. (226,1624) .. controls (229.33,1621.46) and (236.24,1618.61) .. (236,1618) .. controls (235.76,1617.39) and (243.31,1617.27) .. (250,1618) .. controls (256.69,1618.73) and (262.53,1620.62) .. (264,1620) .. controls (265.47,1619.38) and (296.11,1616.11) .. (296,1614) .. controls (295.89,1611.89) and (301.99,1602.86) .. (304,1602) .. controls (306.01,1601.14) and (317.99,1622.88) .. (320,1616) .. controls (322.01,1609.12) and (321.73,1568.73) .. (326,1570) .. controls (330.27,1571.27) and (339.87,1563.9) .. (338,1584) .. controls (336.13,1604.1) and (324.06,1639.29) .. (320,1642) .. controls (315.94,1644.71) and (305.31,1646.5) .. (304,1644) -- cycle ;


% Text Node
\draw (268.2,1637.75) node  [font=\tiny,color={rgb, 255:red, 189; green, 16; blue, 224 }  ,opacity=1 ] [align=left] {$\displaystyle \rho _{2} =\{( \omega _{11} ,a_{11}) \dotsc \}$};
% Text Node
\draw (231.2,1581.75) node  [font=\tiny,color={rgb, 255:red, 189; green, 16; blue, 224 }  ,opacity=1 ] [align=left] {$\displaystyle \rho _{1} =\{( \omega _{21} ,a_{21}) \dotsc \}$};
% Text Node
\draw (209.5,1570) node  [font=\scriptsize] [align=left] {$\displaystyle \Omega \times A$};
% Text Node
\draw (267.61,1655) node   [align=left] {{\tiny \textit{An abstract visualization of}}};
\draw (267.61,1665) node   [align=left] {{\tiny \textit{transitions in trajectories}}};

\end{tikzpicture}
      \end{column}
    \end{columns}
  \end{center}

  \begin{tikzpicture}[remember picture, overlay]
    \node[anchor=north west, text=black]
    at ([xshift=5.8cm,yshift=-5cm]current page.north west) {\small
      \begin{minipage}{0.3\linewidth}
        {\small \hspace{1.3cm} \textit{\textbf{Ideas\dots}}
          \begin{itemize}
            \item  \textit{Trajectories comme vecteurs;}
            \item  \textit{Distance: Smith-Waterman, LCS, Euclidean\dots;}
            \item  \textit{Clustering + Centroides \\ \ \ \ $\rightarrow$ roles/objectifs}
          \end{itemize}}
      \end{minipage}

    };
  \end{tikzpicture}

\end{frame}


\section{Cadre expérimental et discussion des résultats}


\begin{frame}{Environements \& Spécifications organisationnelles}

  \vspace{-0cm}

  \begin{columns}[c]

    \hspace{-1cm}

    \begin{column}{0.5\textwidth}
      {\scriptsize
        \textbf{Environements jouets académiques :}
        \begin{itemize}
          \item \textbf{Predator-Prey}~\autocite{lowe2017multi}.
                \begin{itemize}
                  \item \textit{\scriptsize Roles: chaser, blocker}
                  \item \textit{\scriptsize Goals: surround prey, prevent escape}
                \end{itemize}
          \item \textbf{Overcooked-AI}~\autocite{overcookedai}.
                \begin{itemize}
                  \item \textit{\scriptsize Roles: chef, assistant, server}
                  \item \textit{\scriptsize Goals: deliver dishes, avoid collisions}
                \end{itemize}
          \item \textbf{Warehouse Management (custom)}
                \begin{itemize}
                  \item \textit{\scriptsize Roles: picker, transporter, stocker}
                  \item \textit{\scriptsize Goals: move items, restock shelves}
                \end{itemize}
        \end{itemize}

        \medskip

        \textbf{Environements Cyberdéfense :}
        \begin{itemize}
          \item \textbf{Network Infrastructure - CybORG}~\autocite{Maxwell2021}.
                \begin{itemize}
                  \item \textit{\scriptsize Roles: IDS, responder, firewall operator}
                  \item \textit{\scriptsize Goals: detect intrusions, recover hosts}
                \end{itemize}
          \item \textbf{Drone Swarm - CybORG}~\autocite{Maxwell2021}.
                \begin{itemize}
                  \item \textit{\scriptsize Roles: detector, active defender}
                  \item \textit{\scriptsize Goals: detect intrusions, reimage host}
                \end{itemize}
          \item \textbf{K8s Attack/Defense}~\autocite{soule2025cloud}.
                \begin{itemize}
                  \item \textit{\scriptsize Roles: DDoS mitigator, contention manager, crash manager}
                  \item \textit{\scriptsize Goals: detect DDoS, minimize crash}
                \end{itemize}
        \end{itemize}}


    \end{column}

    \hspace{-1.5cm}

    \begin{column}{0.5\textwidth}
      \begin{tabular}{@{}c@{\hspace{1cm}}c@{}}
        \makebox[.48\textwidth][c]{\animategraphics[loop,autoplay,scale=0.15]{8}{figures/wm/frame}{0}{33}}   &
        \vspace{0.1cm} \makebox[.48\textwidth][c]{\animategraphics[loop,autoplay,scale=0.18]{8}{figures/overcooked_asymmetric_advantage/frame}{0}{66}} \\
        \small{Warehouse Management}                                                                         & \vspace{0.1cm} \small{Overcooked-AI}    \\
        \makebox[.48\textwidth][c]{\animategraphics[loop,autoplay,scale=0.135]{8}{figures/mpe/frame}{0}{25}} &
        \makebox[.48\textwidth][c]{\animategraphics[loop,autoplay,scale=0.135]{8}{figures/cyborg/frame}{0}{33}}                                        \\
        \small{Predator-Prey}                                                                                & \small{CybORG}                          \\
      \end{tabular}
    \end{column}
  \end{columns}
\end{frame}

\begin{frame}{Cas d'application}{\textbf{Drone Swarm Scenario (CAGE Challenge)}}

  \textbf{Trois modèles d'organisation :}
  \begin{itemize}
    \item "Suspect Isolation"
    \item "Active Defense"
    \item "Manual"
  \end{itemize}

  \makebox[0.4\textwidth][c]{\animategraphics[loop,autoplay,scale=0.19]{8}{figures/cyborg/frame}{0}{33}}

  \begin{textblock*}{20cm}(6cm,1.8cm)
    \includegraphics[width=0.5\linewidth]{figures/cage_own_results.png}
  \end{textblock*}

  \begin{textblock*}{14cm}(7.5cm,4.8cm)
    \includegraphics[width=0.5\linewidth]{figures/cage_leader_results.png}
  \end{textblock*}

\end{frame}

\begin{frame}{Cas d'application}{\textbf{K8s Attack/Defense}}

  \centering
  \includegraphics[width=0.8\linewidth]{figures/scenario_introduction.pdf}

\end{frame}

\begin{frame}{Cas d'application}{\textbf{K8s Attack/Defense}}


  \begin{columns}[c]

    \begin{column}{0.4\textwidth}

      \textbf{Approche SMA}
      \begin{itemize}
        \item Un agent par problème
        \item Cibler les problèmes par priorité
        \item Passage à l'échelle facilité
      \end{itemize}

      \

      \textbf{Mise en œuvre dans KARMA}
      \begin{itemize}
        \item PoC fonctionnel sur cluster simple
        \item Amélioration résilience opérationnelle
        \item Convergence plus rapide
      \end{itemize}

    \end{column}

    \begin{column}{0.7\textwidth}

      \includegraphics[width=\linewidth]{figures/KARMA_architecture.png}

    \end{column}
  \end{columns}

  \vfill

  {\tiny \textit{J. Soule, J.-P. Jamont, M. Occello, L.-M. Traonouez, and P. Théron. Streamlining Resilient Kubernetes Autoscaling with Multi-Agent Systems via an Automated Online Design Framework. Proceedings of the 18th IEEE International Conference on Cloud Computing (CLOUD), Helsinki, Finland, July 2025. (Accepted).}}

\end{frame}

\section{Conclusion}

\begin{frame}{Conclusion et perspectives}

  \begin{itemize}
    \item Une méthode de conception d'un SMA de Cyberdéfense
    \item Application sur études de cas
    \item[ ] \phantom{X}
    \item \textbf{Travaux futurs}
          \begin{itemize}
            \item Consolidation travaux existants
            \item Rédaction manuscrit, finalisation articles
          \end{itemize}
    \item \textbf{Perspectives}
          \begin{itemize}
            \item Diminution écart "Simulation - Réalité"
            \item Spécifications organisationnelles dynamiques
            \item Amélioration post-analyse
            \item Ouverture industrielle pour AICA
          \end{itemize}
  \end{itemize}

\end{frame}

\appendix
%\setbeamertemplate{headline}{}
\setbeamertemplate{mini frames}{}

% \AtBeginSection[]{
% 	\begin{frame}
% 		\frametitle{}
% 		\tableofcontents[currentsection]
% 	\end{frame}
% }

% %%%%%%%%%%%%%%%%%%%%%%%%%%%%%%%%%%%%

\section*{\phantom{Thanks}}

\begin{frame}{}

  \vspace{6ex}

  \centering
  {
    \Huge
    \emph{Thank You}
  }

  \vspace{6ex}

  \begin{columns}

    \hspace{-27ex}

    \begin{column}{0.5\textwidth}
      \raggedleft
      {\Large Demo video $\Longrightarrow$}
    \end{column}

    \hspace{-12ex}

    \begin{column}{0.5\textwidth}
      \includegraphics[width=0.5\linewidth]{figures/demo_qr_code.png}
    \end{column}

  \end{columns}

  \vspace{3ex}

  \centering
  {\Large
    \url{https://t.ly/4JBxr}
  }

\end{frame}


\section*{\phantom{References}}
\begin{frame}[allowframebreaks]{References}{}
  \printbibliography
\end{frame}

\newcounter{mainframenumber}
\setcounter{mainframenumber}{\value{framenumber}}

% % \begin{frame}{Annexes}
    {Context}

    \begin{block}{Multi-Agent Systems (MAS) paradigm for complex \& distributed problems}
        \begin{itemize}
            \item \textbf{task decomposition}: missions delegated to agents achieved through cooperation~\cite{Raileanu2023};
            \item \textbf{benefits}: handle conflicting goals, parallel computation, system robustness, scalability\dots
        \end{itemize}
    \end{block}

    \begin{block}{\textbf{Organization}: key for MAS designing}
        \begin{itemize}
            \item \textbf{coordination}: how to collaboratively achieve a common goal~\cite{Hubner2007};
            \item \textbf{dynamic \& uncertain environments}: flexible runtime behavior to adapt~\cite{Kathleen2020};
        \end{itemize}
    \end{block}

    \begin{block}{Methods and practice for MAS design}
        \begin{itemize}
            \item \textbf{approach + organizational model}: methods rely on designers' experience to hand-craft agents' \textbf{policies} so resulting MAS achieve goals;
                  %   \begin{itemize}
                  %       \item Examples: \emph{GAIA}~\cite{Wooldridge2000,Cernuzzi2014}, \emph{ADELFE}~\cite{Mefteh2015}, or \emph{DIAMOND}~\cite{Jamont2015}, \emph{KB-ORG}~\cite{Sims2008}
                  %   \end{itemize}
            \item \textbf{simulation to reality}: 1) safe \& efficient MAS design in high fidelity simulated environment; \quad 2) transfer to real environment to perform adequately~\cite{Schon2021}.
        \end{itemize}
        \vspace{1ex}
        \quad $\Longrightarrow$ \textbf{Iterative process proceeding by trial and error}

    \end{block}

\end{frame}

\begin{frame}{Annexes}
    {MAS basics}

    \begin{block}{Keywords}
        \begin{itemize}
            \item \textbf{Agent}: entity immersed in an environment perceiving observation and making decision autonomously to achieve some goals;
            \item \textbf{MAS}: a set of agents collaborating with self/re-organizing mechanisms to achieve their goal;
            \item \textbf{Organization}: the agents' interactions even though it may be implicit;
            \item \textbf{Organizational Model (OM)}: medium to formally describe an explicit/implicit organization;
            \item \textbf{Organizational Specifications (OS)}: components of an OM to characterize an organization
        \end{itemize}
    \end{block}

    \begin{block}{Organizational model: $\mathcal{M}OISE^+$}
        \begin{itemize}
            \item more complex than \emph{Agent Group Roles} (integration of standards);
            \item takes into account the social aspects between agents explicitly;
            \item possible to link agents' policies to organizational specifications.
        \end{itemize}
    \end{block}

\end{frame}

\begin{frame}{Annexes}
    {MARL basics}

    \begin{block}{Keywords}
        \begin{itemize}
            \item \textbf{Policy}: the \textquote{logic} to choose next action according to observation for an agent;
            \item \textbf{History/trajectory}: the tuple of (observation, action) couples over an episode;
            \item \textbf{Joint-policy / Joint-history}: all of the agents' policies / histories as tuples;
            \item \textbf{Reinforcement learning}: an agent updates its policy to maximize a cumulative reward;
            \item \textbf{Multi-Agent Reinforcement Learning (MARL)}: extends to multiple agents that learn while considering the actions of other agents;
        \end{itemize}
    \end{block}

\end{frame}



\end{document}
