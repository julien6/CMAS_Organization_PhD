\documentclass[9pt, aspectratio=169]{beamer}
% \documentclass[10pt]{beamer}
\usepackage[utf8]{inputenc}
\usepackage[T1]{fontenc}
\usepackage[english]{babel}
\usetheme{Frankfurt}

\usepackage[backend=biber, style=authoryear]{biblatex}
\addbibresource{local_references.bib}

%\usepackage{lmodern}
\usepackage{amsfonts,amssymb,amsmath}
\usepackage[english]{babel}
\usetheme{Frankfurt}

\usepackage{csquotes}
\usepackage{setspace}

\usepackage{colortbl}
\usepackage{tabularx}
\renewcommand\tabularxcolumn[1]{m{#1}}

% --- Tickz
\usepackage{physics}
\usepackage{amsmath}
\usepackage{tikz}
\usepackage{mathdots}
\usepackage{yhmath}
\usepackage{cancel}
\usepackage{color}
\usepackage{siunitx}
\usepackage{array}
\usepackage{multirow}
\usepackage{amssymb}
\usepackage{gensymb}
\usepackage{tabularx}
\usepackage{extarrows}
\usepackage{booktabs}
\usetikzlibrary{fadings}
\usetikzlibrary{patterns}
\usetikzlibrary{shadows.blur}
\usetikzlibrary{shapes}

% ---------

\usepackage{booktabs}
\usepackage{setspace}
\usepackage{amssymb}
\usepackage{adjustbox}
\usepackage{pifont}
\usepackage[inkscapeformat=png]{svg}
\usepackage{graphicx}
\usepackage{times}
\setbeamertemplate{caption}[numbered]
% % \setbeamertemplate{bibliography item}{[\theenumiv]}

\setbeamerfont{bibliography item}{size=\tiny}
\setbeamerfont{bibliography entry author}{size=\tiny}
\setbeamerfont{bibliography entry title}{size=\tiny}
\setbeamerfont{bibliography entry location}{size=\tiny}
\setbeamerfont{bibliography entry note}{size=\tiny}

\setbeamerfont{frametitle}{size=\large}

\usepackage{caption}
\usepackage{float}
\usepackage{xcolor}
\usepackage{listings}
\usepackage{animate}

\definecolor{codegreen}{rgb}{0,0.6,0}
\definecolor{codegray}{rgb}{0.5,0.5,0.5}
\definecolor{codepurple}{rgb}{0.58,0,0.82}
\definecolor{backcolour}{rgb}{0.95,0.95,0.92}
 
\lstdefinestyle{mystyle}{
    backgroundcolor=\color{backcolour},   
    commentstyle=\color{codegreen},
    keywordstyle=\color{magenta},
    numberstyle=\tiny\color{codegray},
    stringstyle=\color{codepurple},
    basicstyle=\footnotesize,
    breakatwhitespace=false,         
    breaklines=true,                 
    captionpos=b,                    
    keepspaces=true,                 
    numbers=left,                    
    numbersep=5pt,                  
    showspaces=false,                
    showstringspaces=false,
    showtabs=false,                  
    tabsize=2
}
 
\lstset{style=mystyle}

\usepackage{ragged2e}
\setbeamercolor{section in foot}{fg=white,bg=darkorange}
\setbeamercolor{subsection in foot}{fg=white,bg=darkorange}
\setbeamercolor{frametitle}{fg=white, bg=darkorange}
\setbeamercolor{title}{fg=white, bg=darkorange}
\setbeamercolor{frame}{bg=darkorange}
\setbeamercolor{block title}{bg=darkorange,fg=white}

\setbeamercolor{item}{fg=darkorange}

% \definecolor{darkorange}{rgb}{0.81, 0.52, 0.05}
\definecolor{darkorange}{rgb}{1,0.5,0}
\definecolor{darkorange2}{rgb}{1, 0.64, 0.2}
\definecolor{honeydew}{rgb}{1, 0.85, 0.45}


\newenvironment{variableblock}[3]{%
  \setbeamercolor{block body}{#2}
  \setbeamercolor{block title}{#3}
  \begin{block}{#1}}{\end{block}}

\newenvironment{prosblock}[1]{%
  % \setbeamercolor{block body}{bg=blue,fg=white}
  \setbeamercolor{block title}{bg=blue,fg=white}
  \begin{block}{#1}}{\end{block}}

\newenvironment{consblock}[1]{%
  % \setbeamercolor{block body}{bg=red,fg=white}
  \setbeamercolor{block title}{bg=red,fg=white}
  \begin{block}{#1}}{\end{block}}

\newcommand{\cmark}{\ding{51}}%
\newcommand{\xmark}{\ding{55}}%

\renewcommand{\arraystretch}{1.5}

% Please add the following required packages to your document preamble:
\usepackage{booktabs}
\usepackage{multirow}
\usepackage{colortbl}
% Beamer presentation requires \usepackage{colortbl} instead of \usepackage[table,xcdraw]{xcolor}

\usepackage{tabularray}\UseTblrLibrary{varwidth}
\usepackage{xcolor}
\def\BibTeX{{\rm B\kern-.05em{\sc i\kern-.025em b}\kern-.08em
    T\kern-.1667em\lower.7ex\hbox{E}\kern-.125emX}}
% \usepackage{cite}
\usepackage{amsmath}
\newcommand{\probP}{\text{I\kern-0.15em P}}
\usepackage{etoolbox}
\patchcmd{\thebibliography}{\section*{\refname}}{}{}{}

\setlength\tabcolsep{0.5pt}

\renewcommand{\arraystretch}{0.9}
\setlength{\tabcolsep}{2pt}

\usepackage{pgffor}
\usepackage[absolute,overlay]{textpos}
\setlength{\TPHorizModule}{1cm}
\setlength{\TPVertModule}{1cm}

\setbeamerfont{bibliography item}{size=\tiny}
\setbeamerfont{bibliography entry author}{size=\tiny}
\setbeamerfont{bibliography entry title}{size=\tiny}
\setbeamerfont{bibliography entry location}{size=\tiny}
\setbeamerfont{bibliography entry note}{size=\tiny}

\setbeamerfont{bibliography entry author}{shape=\upshape,series=\mdseries,size=\footnotesize}
\setbeamerfont{bibliography entry title}{shape=\slshape,series=\mdseries,size=\footnotesize}
\setbeamerfont{bibliography entry journal}{shape=\upshape,series=\mdseries,size=\footnotesize}
\setbeamerfont{bibliography entry note}{shape=\upshape,series=\mdseries,size=\footnotesize}

\renewcommand*{\bibfont}{\scriptsize}

\newenvironment<>{varblock}[2][.9\textwidth]{%
  \setlength{\textwidth}{#1}
  \begin{actionenv}#3%
    \def\insertblocktitle{#2}%
    \par%
    \usebeamertemplate{block begin}}
  {\par%
    \usebeamertemplate{block end}%
  \end{actionenv}}

\newenvironment<>{alertvarblock}[2][.9\textwidth]{%
  \setlength{\textwidth}{#1}%
  \begin{actionenv}#3%
    \begin{alertblock}{#2}%
}{%
    \end{alertblock}%
  \end{actionenv}%
}

% \setbeamertemplate{footline}[frame number]

\setbeamertemplate{footline}{
  \leavevmode%
  \hfill
  \usebeamercolor[fg]{page number in head/foot}%
  \scriptsize%
  \ifnum\value{framenumber}>46%
    Appendix \number\numexpr\value{framenumber}-55\relax/55%
  \else%
    \ifnum\value{framenumber}>43%
      %
    \else
      \number\numexpr\value{framenumber}\relax/43%
    \fi

  \fi%
  \hspace{1em}
}


\begin{document}

\author{\textbf{Julien Soulé$^{1,2}$}, Jean-Paul Jamont$^1$, Michel Occello$^1$, Louis-Marie Traonouez$^2$, Paul Théron$^3$}

\title{\textbf{Towards Assisted MAS Design: A Library for
Explainable MARL with Organizational Model}}

\subtitle{ECAI 2024 Demo Presentation}

% \logo{\includegraphics[scale=0.01]{figures/grenoble-inp_logo.png}}

\institute{\footnotesize \textit{University Grenoble Alpes, Grenoble INP, LCIS, 26000, Valence, France \\
$^1$\{julien.soule, jean-paul.jamont, michel.occello\}@lcis.grenoble-inp.fr \\ \phantom{U} \\
Thales Land and Air Systems, BL IAS, 35000, Rennes, France \\
$^2$\{julien.soule, louis-marie.traonouez\}@thalesgroup.com \\ \phantom{U} \\
AICA IWG, La Guillermie, France \\
$^3$paul.theron@orange.fr}}


\date{\textit{\footnotesize May 9, 2024}}

%\subject{}
\setbeamercovered{transparent}
%\setbeamertemplate{navigation symbols}{}
\begin{frame}[plain]
	\maketitle\vspace{-0.8cm}
	\begin{figure}[ht!]
		\centering
            \includegraphics[height=0.8cm]{figures/la-ruche_logo.png}
            \hspace{0.8cm}
            \includegraphics[height=0.8cm]{figures/lcis_logo.png}
            \hspace{0.8cm}
		\includegraphics[height=0.8cm]{figures/grenoble-inp_logo.png}
            \hspace{0.8cm}
            \includegraphics[height=0.8cm]{figures/uga_logo.jpg}
	\end{figure}
\end{frame}

\begin{frame}{Sommaire}
  \tableofcontents
\end{frame}

\addtocounter{framenumber}{-2}


\section{Introduction}

\AtBeginSection[]{
  \begin{frame}
    \frametitle{}
    \tableofcontents[currentsection]
  \end{frame}
}

\begin{frame}{Introduction}{Contexte}

  \begin{columns}

    \begin{column}{0.7\textwidth}

      \begin{itemize}
        \item \textbf{« Agents Autonomes Intelligents de Cyberdéfense » (AICA) théorisé par 'IST-152 NATO' (2016-2019) \parencite{Kott2023}}
              \begin{itemize}
                \item Détecter, identifier et caractériser les anomalies/attaques
                \item Planifier et exécuter des contre-mesures
                \item Être autonome, discret, interopérable, capable d'apprendre
              \end{itemize}
      \end{itemize}

      \ \\

      $\Longrightarrow$ Besoin de: \textbf{réactivité, flexibilité, autonomie}\dots

      \ \\

      \begin{itemize}
        \item \textbf{« Multi-Agent System Centric AICA Reference Architecture »}
              \begin{itemize}
                \item Limites des opérateurs : contraintes temporelles, charge de travail, complexité\dots
                \item Limites de l'environnement : brouillage, interruption de communication\dots
              \end{itemize}
      \end{itemize}

      \ \\

      $\Longrightarrow$ \textbf{Système Multi-Agent de Cyberdéfense} pour \textbf{adaptation, passage à l'échelle, délégation de sous-tâches\dots}\dots

      \ \\

      $\Longrightarrow$ \textbf{Organisation} des agents cruciale pour \textbf{performance, robustesse, explicabilité\dots}

    \end{column}

    \begin{column}{0.4\textwidth}
      \begin{figure}
        \includegraphics[width=\linewidth]{figures/casino.jpg}
        \caption*{\vspace{-0.5cm}\tiny\url{https://hackread.com/hackers-casinos-fish-tank-smart-thermometer-hack/}}
      \end{figure}

      \vspace{0.cm}
      \animategraphics[autoplay,loop,width=\linewidth]{1}{figures/cyberdefense_mas_frames/frame}{0}{8}

    \end{column}

  \end{columns}

\end{frame}

\begin{frame}{Introduction}{Problème}



  \begin{columns}

    \begin{column}{0.5\textwidth}
      \centering

      \begin{alertvarblock}[9cm]{Problème général}
        \textbf{Quels \underline{mécanismes organisationnels} du SMA de cyberdéfense permettent d'\underline{optimiser son fonctionnement} en tenant compte de ses \underline{contraintes} ?}
      \end{alertvarblock}

      \begin{itemize}
        \item \textbf{Idée d'un SMA de Cyberdéfense}
        \item \textbf{Problème recherche organisation adaptée}
              \begin{itemize}
                \item Contraintes environnementales
                \item Exigences de conception
              \end{itemize}
        \item \textbf{Automatisation recherche organisation de SMA de Cyberdéfense selon :}
              \begin{itemize}
                \item[(C1)] Autonomie
                \item [(C2)] Performance
                \item [(C3)] Adaptation
                \item [(C4)] Contrôle
                \item [(C5)] Explicabilité
              \end{itemize}
      \end{itemize}

      \ \\[-0.2cm]

      $\Longrightarrow$ \textbf{Etude de littérature\dots}
    \end{column}

    \begin{column}{0.4\textwidth}
      % \resizebox{0.5\textwidth}{!}{
      \centering
      % \includegraphics[width=0.95\linewidth]{figures/general_problem_illustration.png}
      \resizebox{\linewidth}{!}{


\tikzset{every picture/.style={line width=0.75pt}} %set default line width to 0.75pt        

\begin{tikzpicture}[x=0.75pt,y=0.75pt,yscale=-1,xscale=1]
    %uncomment if require: \path (0,5804); %set diagram left start at 0, and has height of 5804

    %Shape: Rectangle [id:dp37204586795845374] 
    \draw   (15,4410) -- (660,4410) -- (660,5270) -- (15,5270) -- cycle ;
    %Shape: Rectangle [id:dp6528644209370597] 
    \draw   (40,4795) -- (271.14,4795) -- (271.14,4935) -- (40,4935) -- cycle ;
    %Shape: Rectangle [id:dp04838573396960488] 
    \draw  [fill={rgb, 255:red, 245; green, 166; blue, 35 }  ,fill opacity=1 ] (45,4804) .. controls (45,4801.79) and (46.79,4800) .. (49,4800) -- (101,4800) .. controls (103.21,4800) and (105,4801.79) .. (105,4804) -- (105,4826) .. controls (105,4828.21) and (103.21,4830) .. (101,4830) -- (49,4830) .. controls (46.79,4830) and (45,4828.21) .. (45,4826) -- cycle ;
    %Straight Lines [id:da28779803352593725] 
    % \draw    (105,4780) -- (125,4780) ;
    %Straight Lines [id:da9267919848430177] 
    \draw    (105,4865) -- (125,4865) ;
    %Straight Lines [id:da6747003832540825] 
    \draw    (185,4915) -- (205,4915) ;
    %Straight Lines [id:da42065693917376556] 
    \draw    (75,4880) -- (75,4900) ;
    %Straight Lines [id:da15025959978456238] 
    \draw    (155,4830) -- (155,4850) ;
    %Shape: Rectangle [id:dp04109612445272104] 
    \draw  [fill={rgb, 255:red, 126; green, 211; blue, 33 }  ,fill opacity=1 ] (125,4804) .. controls (125,4801.79) and (126.79,4800) .. (129,4800) -- (181,4800) .. controls (183.21,4800) and (185,4801.79) .. (185,4804) -- (185,4826) .. controls (185,4828.21) and (183.21,4830) .. (181,4830) -- (129,4830) .. controls (126.79,4830) and (125,4828.21) .. (125,4826) -- cycle ;
    %Shape: Rectangle [id:dp8218797438686123] 
    \draw  [fill={rgb, 255:red, 189; green, 16; blue, 224 }  ,fill opacity=1 ] (125,4854) .. controls (125,4851.79) and (126.79,4850) .. (129,4850) -- (181,4850) .. controls (183.21,4850) and (185,4851.79) .. (185,4854) -- (185,4876) .. controls (185,4878.21) and (183.21,4880) .. (181,4880) -- (129,4880) .. controls (126.79,4880) and (125,4878.21) .. (125,4876) -- cycle ;
    %Shape: Rectangle [id:dp37193948395169385] 
    \draw  [fill={rgb, 255:red, 184; green, 233; blue, 134 }  ,fill opacity=1 ] (45,4854) .. controls (45,4851.79) and (46.79,4850) .. (49,4850) -- (101,4850) .. controls (103.21,4850) and (105,4851.79) .. (105,4854) -- (105,4876) .. controls (105,4878.21) and (103.21,4880) .. (101,4880) -- (49,4880) .. controls (46.79,4880) and (45,4878.21) .. (45,4876) -- cycle ;
    %Straight Lines [id:da699052721095542] 
    \draw    (75,4830) -- (75,4850) ;
    %Shape: Rectangle [id:dp5056775522132636] 
    \draw  [fill={rgb, 255:red, 144; green, 19; blue, 254 }  ,fill opacity=1 ] (205,4804) .. controls (205,4801.79) and (206.79,4800) .. (209,4800) -- (261,4800) .. controls (263.21,4800) and (265,4801.79) .. (265,4804) -- (265,4876) .. controls (265,4878.21) and (263.21,4880) .. (261,4880) -- (209,4880) .. controls (206.79,4880) and (205,4878.21) .. (205,4876) -- cycle ;

    %Shape: Rectangle [id:dp5493685328045492] 
    \draw  [fill={rgb, 255:red, 208; green, 2; blue, 27 }  ,fill opacity=1 ] (45,4904) .. controls (45,4901.79) and (46.79,4900) .. (49,4900) -- (181,4900) .. controls (183.21,4900) and (185,4901.79) .. (185,4904) -- (185,4926) .. controls (185,4928.21) and (183.21,4930) .. (181,4930) -- (49,4930) .. controls (46.79,4930) and (45,4928.21) .. (45,4926) -- cycle ;
    %Straight Lines [id:da7074117972544659] 
    \draw    (155,4880) -- (155,4900) ;
    %Shape: Rectangle [id:dp32171767014290364] 
    \draw  [fill={rgb, 255:red, 80; green, 227; blue, 194 }  ,fill opacity=1 ] (205,4904) .. controls (205,4901.79) and (206.79,4900) .. (209,4900) -- (261,4900) .. controls (263.21,4900) and (265,4901.79) .. (265,4904) -- (265,4926) .. controls (265,4928.21) and (263.21,4930) .. (261,4930) -- (209,4930) .. controls (206.79,4930) and (205,4928.21) .. (205,4926) -- cycle ;
    %Straight Lines [id:da5336821600138906] 
    \draw    (235,4880) -- (235,4900) ;
    %Image [id:dp01351702105019581] 
    \draw (322.3,4872.5) node  {\includegraphics[width=18.44pt,height=18.75pt]{figures/cloud.png}};
    %Image [id:dp2893008276787382] 
    \draw (423.11,4905) node  {\includegraphics[width=14.75pt,height=15pt]{figures/server.png}};
    %Image [id:dp33084000544395364] 
    \draw (423.11,4870) node  {\includegraphics[width=14.75pt,height=15pt]{figures/server.png}};
    %Image [id:dp4045173494088059] 
    \draw (423.11,4835) node  {\includegraphics[width=14.75pt,height=15pt]{figures/server.png}};
    %Image [id:dp21670294007047108] 
    \draw (354.26,4870) node  {\includegraphics[width=14.75pt,height=15pt]{figures/11468565_brick_block_construction_build_wall_icon.png}};
    %Image [id:dp339375141118186] 
    \draw (457.54,4870) node  {\includegraphics[width=14.75pt,height=15pt]{figures/11468565_brick_block_construction_build_wall_icon.png}};
    %Image [id:dp8954911849692438] 
    \draw (388.69,4870) node  {\includegraphics[width=14.75pt,height=15pt]{figures/router.png}};
    %Image [id:dp6043253137812208] 
    \draw (491.97,4870) node  {\includegraphics[width=14.75pt,height=15pt]{figures/router.png}};
    %Image [id:dp5884883297493204] 
    \draw (565.74,4850) node  {\includegraphics[width=14.75pt,height=15pt]{figures/laptop.png}};
    %Image [id:dp08417059183927911] 
    \draw (600.16,4850) node  {\includegraphics[width=14.75pt,height=15pt]{figures/laptop.png}};
    %Image [id:dp7156849150132849] 
    \draw (600.16,4885) node  {\includegraphics[width=14.75pt,height=15pt]{figures/workstation.png}};
    %Image [id:dp6237848089278466] 
    \draw (565.74,4885) node  {\includegraphics[width=14.75pt,height=15pt]{figures/server.png}};
    %Image [id:dp7449851735440329] 
    \draw (531.31,4885) node  {\includegraphics[width=14.75pt,height=15pt]{figures/server.png}};
    %Image [id:dp005667224738301391] 
    \draw (531.31,4920) node  {\includegraphics[width=14.75pt,height=15pt]{figures/workstation.png}};
    %Image [id:dp8146427195897723] 
    \draw (565.74,4920) node  {\includegraphics[width=14.75pt,height=15pt]{figures/database.png}};
    %Image [id:dp49310370523635805] 
    \draw (600.16,4920) node  {\includegraphics[width=14.75pt,height=15pt]{figures/database.png}};
    %Image [id:dp020292600387318505] 
    \draw (531.31,4850) node  {\includegraphics[width=14.75pt,height=15pt]{figures/computer.png}};
    %Straight Lines [id:da2500380552659036] 
    \draw    (398.52,4870) -- (413.28,4870) ;
    %Straight Lines [id:da353449120783108] 
    \draw    (433.2,4870) -- (447.95,4870) ;
    %Straight Lines [id:da5004469875213738] 
    \draw    (467.38,4870) -- (482.13,4870) ;
    %Straight Lines [id:da6503894906953667] 
    \draw    (501.8,4870) -- (506.39,4870) -- (511.64,4870) ;
    %Straight Lines [id:da37401288462201976] 
    \draw    (511.64,4835) -- (511.64,4905) ;
    %Straight Lines [id:da5414293349526964] 
    \draw    (511.64,4870) -- (600.16,4870) ;
    %Straight Lines [id:da882160807467605] 
    \draw    (511.64,4835) -- (600.16,4835) ;
    %Straight Lines [id:da17424002654409498] 
    \draw    (511.64,4905) -- (600.16,4905) ;
    %Straight Lines [id:da974757219094287] 
    \draw    (531.31,4835) -- (531.31,4840) ;
    %Straight Lines [id:da12326774495407566] 
    \draw    (565.74,4835) -- (565.74,4840) ;
    %Straight Lines [id:da8825714625264635] 
    \draw    (600.16,4835) -- (600.16,4840) ;
    %Straight Lines [id:da8730507055269275] 
    \draw    (531.31,4870) -- (531.31,4875) ;
    %Straight Lines [id:da9685106023683034] 
    \draw    (565.74,4870) -- (565.74,4875) ;
    %Straight Lines [id:da6139098058213984] 
    \draw    (600.16,4870) -- (600.16,4875) ;
    %Straight Lines [id:da8488931299803314] 
    \draw    (531.31,4910) -- (531.31,4905) ;
    %Straight Lines [id:da48290744419465503] 
    \draw    (565.74,4910) -- (565.74,4905) ;
    %Straight Lines [id:da8004477536972611] 
    \draw    (600.16,4910) -- (600.16,4905) ;
    %Straight Lines [id:da25143467968820776] 
    \draw    (403.44,4835) -- (403.44,4905) ;
    %Straight Lines [id:da14951928183264374] 
    \draw    (442.79,4835) -- (442.79,4905) ;
    %Straight Lines [id:da9023759580794825] 
    \draw    (403.44,4835) -- (418.2,4835) ;
    %Straight Lines [id:da7547540666765368] 
    \draw    (432.95,4835) -- (442.79,4835) ;
    %Straight Lines [id:da050461431438784565] 
    \draw    (403.44,4905) -- (418.2,4905) ;
    %Straight Lines [id:da3825444784935419] 
    \draw    (432.95,4905) -- (442.79,4905) ;
    %Straight Lines [id:da8273733550666897] 
    \draw    (364.1,4870) -- (378.85,4870) ;
    %Straight Lines [id:da28397571787748854] 
    \draw    (329.67,4870) -- (344.43,4870) ;

    %Shape: Arc [id:dp5159031208901032] 
    \draw  [draw opacity=0] (296.48,5209.48) .. controls (296.48,5209.48) and (296.48,5209.48) .. (296.48,5209.48) .. controls (296.48,5209.48) and (296.48,5209.48) .. (296.48,5209.48) .. controls (298.29,5205.57) and (292.91,5199.13) .. (284.45,5195.11) .. controls (275.99,5191.08) and (267.66,5190.99) .. (265.84,5194.9) -- (281.16,5202.19) -- cycle ; \draw   (296.48,5209.48) .. controls (296.48,5209.48) and (296.48,5209.48) .. (296.48,5209.48) .. controls (296.48,5209.48) and (296.48,5209.48) .. (296.48,5209.48) .. controls (298.29,5205.57) and (292.91,5199.13) .. (284.45,5195.11) .. controls (275.99,5191.08) and (267.66,5190.99) .. (265.84,5194.9) ;
    %Shape: Arc [id:dp14312770217862503] 
    \draw  [draw opacity=0] (292.1,5213.16) .. controls (292.1,5213.16) and (292.1,5213.16) .. (292.1,5213.16) .. controls (293.91,5209.25) and (289.51,5203.28) .. (282.26,5199.83) .. controls (275.01,5196.38) and (267.66,5196.75) .. (265.84,5200.66) -- (278.97,5206.91) -- cycle ; \draw   (292.1,5213.16) .. controls (292.1,5213.16) and (292.1,5213.16) .. (292.1,5213.16) .. controls (293.91,5209.25) and (289.51,5203.28) .. (282.26,5199.83) .. controls (275.01,5196.38) and (267.66,5196.75) .. (265.84,5200.66) ;
    %Shape: Arc [id:dp10841807511888646] 
    \draw  [draw opacity=0] (287.71,5216.84) .. controls (289.53,5212.93) and (286.11,5207.43) .. (280.06,5204.55) .. controls (274.02,5201.68) and (267.65,5202.52) .. (265.83,5206.43) -- (276.77,5211.64) -- cycle ; \draw   (287.71,5216.84) .. controls (289.53,5212.93) and (286.11,5207.43) .. (280.06,5204.55) .. controls (274.02,5201.68) and (267.65,5202.52) .. (265.83,5206.43) ;
    %Shape: Arc [id:dp5497029665203269] 
    \draw  [draw opacity=0] (283.33,5220.52) .. controls (283.33,5220.52) and (283.33,5220.52) .. (283.33,5220.52) .. controls (285.15,5216.61) and (282.7,5211.58) .. (277.87,5209.28) .. controls (273.04,5206.97) and (267.64,5208.28) .. (265.83,5212.19) -- (274.58,5216.36) -- cycle ; \draw   (283.33,5220.52) .. controls (283.33,5220.52) and (283.33,5220.52) .. (283.33,5220.52) .. controls (285.15,5216.61) and (282.7,5211.58) .. (277.87,5209.28) .. controls (273.04,5206.97) and (267.64,5208.28) .. (265.83,5212.19) ;
    %Shape: Arc [id:dp9043594701708034] 
    \draw  [draw opacity=0] (278.95,5224.21) .. controls (278.95,5224.21) and (278.95,5224.21) .. (278.95,5224.21) .. controls (280.77,5220.29) and (279.3,5215.72) .. (275.68,5214) .. controls (272.05,5212.27) and (267.64,5214.05) .. (265.82,5217.96) -- (272.39,5221.08) -- cycle ; \draw   (278.95,5224.21) .. controls (278.95,5224.21) and (278.95,5224.21) .. (278.95,5224.21) .. controls (280.77,5220.29) and (279.3,5215.72) .. (275.68,5214) .. controls (272.05,5212.27) and (267.64,5214.05) .. (265.82,5217.96) ;

    %Shape: Smiley Face [id:dp410186171361024] 
    \draw  [line width=1.5]  (274.77,5238.14) .. controls (274.77,5231.59) and (269.23,5226.28) .. (262.39,5226.28) .. controls (255.55,5226.28) and (250,5231.59) .. (250,5238.14) .. controls (250,5244.69) and (255.55,5250) .. (262.39,5250) .. controls (269.23,5250) and (274.77,5244.69) .. (274.77,5238.14) -- cycle ; \draw  [line width=1.5]  (267.84,5234.11) .. controls (267.84,5233.45) and (267.28,5232.92) .. (266.6,5232.92) .. controls (265.91,5232.92) and (265.36,5233.45) .. (265.36,5234.11) .. controls (265.36,5234.76) and (265.91,5235.29) .. (266.6,5235.29) .. controls (267.28,5235.29) and (267.84,5234.76) .. (267.84,5234.11) -- cycle ; \draw  [line width=1.5]  (259.41,5234.11) .. controls (259.41,5233.45) and (258.86,5232.92) .. (258.17,5232.92) .. controls (257.49,5232.92) and (256.94,5233.45) .. (256.94,5234.11) .. controls (256.94,5234.76) and (257.49,5235.29) .. (258.17,5235.29) .. controls (258.86,5235.29) and (259.41,5234.76) .. (259.41,5234.11) -- cycle ; \draw  [line width=1.5]  (268.58,5242.88) .. controls (264.45,5246.05) and (260.32,5246.05) .. (256.19,5242.88) ;
    %Shape: Smiley Face [id:dp6297513983610752] 
    \draw  [line width=1.5]  (244.99,5126.86) .. controls (244.99,5120.31) and (250.59,5115) .. (257.5,5115) .. controls (264.4,5115) and (270,5120.31) .. (270,5126.86) .. controls (270,5133.41) and (264.4,5138.72) .. (257.5,5138.72) .. controls (250.59,5138.72) and (244.99,5133.41) .. (244.99,5126.86) -- cycle ; \draw  [line width=1.5]  (252,5122.83) .. controls (252,5122.17) and (252.56,5121.64) .. (253.25,5121.64) .. controls (253.94,5121.64) and (254.5,5122.17) .. (254.5,5122.83) .. controls (254.5,5123.48) and (253.94,5124.01) .. (253.25,5124.01) .. controls (252.56,5124.01) and (252,5123.48) .. (252,5122.83) -- cycle ; \draw  [line width=1.5]  (260.5,5122.83) .. controls (260.5,5122.17) and (261.06,5121.64) .. (261.75,5121.64) .. controls (262.44,5121.64) and (263,5122.17) .. (263,5122.83) .. controls (263,5123.48) and (262.44,5124.01) .. (261.75,5124.01) .. controls (261.06,5124.01) and (260.5,5123.48) .. (260.5,5122.83) -- cycle ; \draw  [line width=1.5]  (251.25,5131.6) .. controls (255.41,5134.77) and (259.58,5134.77) .. (263.75,5131.6) ;
    %Shape: Smiley Face [id:dp4180730149094508] 
    \draw  [line width=1.5]  (426.61,5136.15) .. controls (426.61,5129.6) and (432.2,5124.29) .. (439.11,5124.29) .. controls (446.01,5124.29) and (451.61,5129.6) .. (451.61,5136.15) .. controls (451.61,5142.69) and (446.01,5148) .. (439.11,5148) .. controls (432.2,5148) and (426.61,5142.69) .. (426.61,5136.15) -- cycle ; \draw  [line width=1.5]  (433.61,5132.11) .. controls (433.61,5131.46) and (434.17,5130.93) .. (434.86,5130.93) .. controls (435.55,5130.93) and (436.11,5131.46) .. (436.11,5132.11) .. controls (436.11,5132.77) and (435.55,5133.3) .. (434.86,5133.3) .. controls (434.17,5133.3) and (433.61,5132.77) .. (433.61,5132.11) -- cycle ; \draw  [line width=1.5]  (442.11,5132.11) .. controls (442.11,5131.46) and (442.67,5130.93) .. (443.36,5130.93) .. controls (444.05,5130.93) and (444.61,5131.46) .. (444.61,5132.11) .. controls (444.61,5132.77) and (444.05,5133.3) .. (443.36,5133.3) .. controls (442.67,5133.3) and (442.11,5132.77) .. (442.11,5132.11) -- cycle ; \draw  [line width=1.5]  (432.86,5140.89) .. controls (437.02,5144.05) and (441.19,5144.05) .. (445.36,5140.89) ;
    %Shape: Arc [id:dp026164586189738936] 
    \draw  [draw opacity=0] (298.04,5128.6) .. controls (298.04,5128.6) and (298.04,5128.6) .. (298.04,5128.6) .. controls (298.04,5128.6) and (298.04,5128.6) .. (298.04,5128.6) .. controls (301.88,5130.63) and (301.47,5139.01) .. (297.12,5147.32) .. controls (292.77,5155.63) and (286.13,5160.72) .. (282.28,5158.69) -- (290.16,5143.64) -- cycle ; \draw   (298.04,5128.6) .. controls (298.04,5128.6) and (298.04,5128.6) .. (298.04,5128.6) .. controls (298.04,5128.6) and (298.04,5128.6) .. (298.04,5128.6) .. controls (301.88,5130.63) and (301.47,5139.01) .. (297.12,5147.32) .. controls (292.77,5155.63) and (286.13,5160.72) .. (282.28,5158.69) ;
    %Shape: Arc [id:dp21730922944992026] 
    \draw  [draw opacity=0] (292.27,5128.3) .. controls (292.27,5128.3) and (292.27,5128.3) .. (292.27,5128.3) .. controls (296.11,5130.33) and (296.21,5137.75) .. (292.48,5144.87) .. controls (288.75,5151.99) and (282.61,5156.12) .. (278.77,5154.09) -- (285.52,5141.19) -- cycle ; \draw   (292.27,5128.3) .. controls (292.27,5128.3) and (292.27,5128.3) .. (292.27,5128.3) .. controls (296.11,5130.33) and (296.21,5137.75) .. (292.48,5144.87) .. controls (288.75,5151.99) and (282.61,5156.12) .. (278.77,5154.09) ;
    %Shape: Arc [id:dp8745570103378975] 
    \draw  [draw opacity=0] (286.5,5128) .. controls (290.35,5130.03) and (290.95,5136.48) .. (287.84,5142.42) .. controls (284.73,5148.35) and (279.1,5151.52) .. (275.25,5149.49) -- (280.87,5138.74) -- cycle ; \draw   (286.5,5128) .. controls (290.35,5130.03) and (290.95,5136.48) .. (287.84,5142.42) .. controls (284.73,5148.35) and (279.1,5151.52) .. (275.25,5149.49) ;
    %Shape: Arc [id:dp32428917068964835] 
    \draw  [draw opacity=0] (280.73,5127.7) .. controls (280.73,5127.7) and (280.73,5127.7) .. (280.73,5127.7) .. controls (284.58,5129.73) and (285.68,5135.22) .. (283.2,5139.97) .. controls (280.71,5144.72) and (275.58,5146.92) .. (271.73,5144.89) -- (276.23,5136.29) -- cycle ; \draw   (280.73,5127.7) .. controls (280.73,5127.7) and (280.73,5127.7) .. (280.73,5127.7) .. controls (284.58,5129.73) and (285.68,5135.22) .. (283.2,5139.97) .. controls (280.71,5144.72) and (275.58,5146.92) .. (271.73,5144.89) ;
    %Shape: Arc [id:dp1721423421834739] 
    \draw  [draw opacity=0] (274.96,5127.39) .. controls (278.81,5129.42) and (280.42,5133.96) .. (278.55,5137.52) .. controls (276.69,5141.08) and (272.06,5142.32) .. (268.21,5140.29) .. controls (268.21,5140.29) and (268.21,5140.29) .. (268.21,5140.29) -- (271.59,5133.84) -- cycle ; \draw   (274.96,5127.39) .. controls (278.81,5129.42) and (280.42,5133.96) .. (278.55,5137.52) .. controls (276.69,5141.08) and (272.06,5142.32) .. (268.21,5140.29) .. controls (268.21,5140.29) and (268.21,5140.29) .. (268.21,5140.29) ;

    %Shape: Arc [id:dp5844771861273294] 
    \draw  [draw opacity=0] (401.71,5157.34) .. controls (401.71,5157.34) and (401.71,5157.34) .. (401.71,5157.34) .. controls (401.71,5157.34) and (401.71,5157.34) .. (401.71,5157.34) .. controls (397.38,5157.78) and (393,5150.6) .. (391.93,5141.29) .. controls (390.87,5131.98) and (393.51,5124.07) .. (397.84,5123.63) -- (399.78,5140.48) -- cycle ; \draw   (401.71,5157.34) .. controls (401.71,5157.34) and (401.71,5157.34) .. (401.71,5157.34) .. controls (401.71,5157.34) and (401.71,5157.34) .. (401.71,5157.34) .. controls (397.38,5157.78) and (393,5150.6) .. (391.93,5141.29) .. controls (390.87,5131.98) and (393.51,5124.07) .. (397.84,5123.63) ;
    %Shape: Arc [id:dp15304922541241428] 
    \draw  [draw opacity=0] (406.66,5154.39) .. controls (402.33,5154.83) and (398.07,5148.73) .. (397.16,5140.75) .. controls (396.25,5132.77) and (399.02,5125.94) .. (403.35,5125.49) -- (405,5139.94) -- cycle ; \draw   (406.66,5154.39) .. controls (402.33,5154.83) and (398.07,5148.73) .. (397.16,5140.75) .. controls (396.25,5132.77) and (399.02,5125.94) .. (403.35,5125.49) ;
    %Shape: Arc [id:dp5873695783447137] 
    \draw  [draw opacity=0] (411.61,5151.44) .. controls (407.28,5151.89) and (403.15,5146.86) .. (402.39,5140.21) .. controls (401.63,5133.56) and (404.52,5127.81) .. (408.85,5127.36) -- (410.23,5139.4) -- cycle ; \draw   (411.61,5151.44) .. controls (407.28,5151.89) and (403.15,5146.86) .. (402.39,5140.21) .. controls (401.63,5133.56) and (404.52,5127.81) .. (408.85,5127.36) ;
    %Shape: Arc [id:dp3340412252291639] 
    \draw  [draw opacity=0] (416.56,5148.49) .. controls (416.56,5148.49) and (416.56,5148.49) .. (416.56,5148.49) .. controls (412.23,5148.94) and (408.23,5144.99) .. (407.62,5139.67) .. controls (407.01,5134.35) and (410.02,5129.68) .. (414.35,5129.23) -- (415.46,5138.86) -- cycle ; \draw   (416.56,5148.49) .. controls (416.56,5148.49) and (416.56,5148.49) .. (416.56,5148.49) .. controls (412.23,5148.94) and (408.23,5144.99) .. (407.62,5139.67) .. controls (407.01,5134.35) and (410.02,5129.68) .. (414.35,5129.23) ;
    %Shape: Arc [id:dp8319335091648151] 
    \draw  [draw opacity=0] (421.51,5145.54) .. controls (421.51,5145.54) and (421.51,5145.54) .. (421.51,5145.54) .. controls (417.18,5145.99) and (413.3,5143.12) .. (412.84,5139.13) .. controls (412.39,5135.14) and (415.53,5131.54) .. (419.86,5131.1) -- (420.69,5138.32) -- cycle ; \draw   (421.51,5145.54) .. controls (421.51,5145.54) and (421.51,5145.54) .. (421.51,5145.54) .. controls (417.18,5145.99) and (413.3,5143.12) .. (412.84,5139.13) .. controls (412.39,5135.14) and (415.53,5131.54) .. (419.86,5131.1) ;

    %Shape: Arc [id:dp8902921027407593] 
    \draw  [draw opacity=0] (443.09,5184.48) .. controls (441.25,5180.57) and (446.69,5174.13) .. (455.23,5170.11) .. controls (463.76,5166.08) and (472.17,5165.99) .. (474.01,5169.9) .. controls (474.01,5169.9) and (474.01,5169.9) .. (474.01,5169.9) -- (458.55,5177.19) -- cycle ; \draw   (443.09,5184.48) .. controls (441.25,5180.57) and (446.69,5174.13) .. (455.23,5170.11) .. controls (463.76,5166.08) and (472.17,5165.99) .. (474.01,5169.9) .. controls (474.01,5169.9) and (474.01,5169.9) .. (474.01,5169.9) ;
    %Shape: Arc [id:dp27157954351388136] 
    \draw  [draw opacity=0] (447.51,5188.17) .. controls (447.51,5188.17) and (447.51,5188.17) .. (447.51,5188.17) .. controls (445.68,5184.25) and (450.12,5178.28) .. (457.44,5174.83) .. controls (464.76,5171.38) and (472.18,5171.75) .. (474.01,5175.67) -- (460.76,5181.92) -- cycle ; \draw   (447.51,5188.17) .. controls (447.51,5188.17) and (447.51,5188.17) .. (447.51,5188.17) .. controls (445.68,5184.25) and (450.12,5178.28) .. (457.44,5174.83) .. controls (464.76,5171.38) and (472.18,5171.75) .. (474.01,5175.67) ;
    %Shape: Arc [id:dp24024767983904605] 
    \draw  [draw opacity=0] (451.93,5191.85) .. controls (451.93,5191.85) and (451.93,5191.85) .. (451.93,5191.85) .. controls (450.1,5187.93) and (453.55,5182.43) .. (459.65,5179.55) .. controls (465.75,5176.68) and (472.18,5177.52) .. (474.02,5181.43) -- (462.97,5186.64) -- cycle ; \draw   (451.93,5191.85) .. controls (451.93,5191.85) and (451.93,5191.85) .. (451.93,5191.85) .. controls (450.1,5187.93) and (453.55,5182.43) .. (459.65,5179.55) .. controls (465.75,5176.68) and (472.18,5177.52) .. (474.02,5181.43) ;
    %Shape: Arc [id:dp20091537172617546] 
    \draw  [draw opacity=0] (456.35,5195.53) .. controls (456.35,5195.53) and (456.35,5195.53) .. (456.35,5195.53) .. controls (454.52,5191.62) and (456.99,5186.58) .. (461.87,5184.28) .. controls (466.75,5181.98) and (472.19,5183.28) .. (474.02,5187.2) -- (465.19,5191.36) -- cycle ; \draw   (456.35,5195.53) .. controls (456.35,5195.53) and (456.35,5195.53) .. (456.35,5195.53) .. controls (454.52,5191.62) and (456.99,5186.58) .. (461.87,5184.28) .. controls (466.75,5181.98) and (472.19,5183.28) .. (474.02,5187.2) ;
    %Shape: Arc [id:dp5171323467750554] 
    \draw  [draw opacity=0] (460.78,5199.21) .. controls (458.94,5195.3) and (460.42,5190.73) .. (464.08,5189) .. controls (467.74,5187.28) and (472.19,5189.05) .. (474.03,5192.96) -- (467.4,5196.09) -- cycle ; \draw   (460.78,5199.21) .. controls (458.94,5195.3) and (460.42,5190.73) .. (464.08,5189) .. controls (467.74,5187.28) and (472.19,5189.05) .. (474.03,5192.96) ;

    %Shape: Smiley Face [id:dp8410574975235354] 
    \draw  [line width=1.5]  (464.99,5213.14) .. controls (464.99,5206.59) and (470.59,5201.29) .. (477.5,5201.29) .. controls (484.4,5201.29) and (490,5206.59) .. (490,5213.14) .. controls (490,5219.69) and (484.4,5225) .. (477.5,5225) .. controls (470.59,5225) and (464.99,5219.69) .. (464.99,5213.14) -- cycle ; \draw  [line width=1.5]  (472,5209.11) .. controls (472,5208.46) and (472.56,5207.93) .. (473.25,5207.93) .. controls (473.94,5207.93) and (474.5,5208.46) .. (474.5,5209.11) .. controls (474.5,5209.77) and (473.94,5210.3) .. (473.25,5210.3) .. controls (472.56,5210.3) and (472,5209.77) .. (472,5209.11) -- cycle ; \draw  [line width=1.5]  (480.5,5209.11) .. controls (480.5,5208.46) and (481.06,5207.93) .. (481.75,5207.93) .. controls (482.44,5207.93) and (483,5208.46) .. (483,5209.11) .. controls (483,5209.77) and (482.44,5210.3) .. (481.75,5210.3) .. controls (481.06,5210.3) and (480.5,5209.77) .. (480.5,5209.11) -- cycle ; \draw  [line width=1.5]  (471.25,5217.89) .. controls (475.41,5221.05) and (479.58,5221.05) .. (483.75,5217.89) ;
    %Shape: Smiley Face [id:dp9549925896715118] 
    \draw  [line width=1.5]  (374.99,5215.96) .. controls (374.99,5209.41) and (380.59,5204.1) .. (387.5,5204.1) .. controls (394.4,5204.1) and (400,5209.41) .. (400,5215.96) .. controls (400,5222.51) and (394.4,5227.82) .. (387.5,5227.82) .. controls (380.59,5227.82) and (374.99,5222.51) .. (374.99,5215.96) -- cycle ; \draw  [line width=1.5]  (382,5211.93) .. controls (382,5211.27) and (382.56,5210.74) .. (383.25,5210.74) .. controls (383.94,5210.74) and (384.5,5211.27) .. (384.5,5211.93) .. controls (384.5,5212.58) and (383.94,5213.11) .. (383.25,5213.11) .. controls (382.56,5213.11) and (382,5212.58) .. (382,5211.93) -- cycle ; \draw  [line width=1.5]  (390.5,5211.93) .. controls (390.5,5211.27) and (391.06,5210.74) .. (391.75,5210.74) .. controls (392.44,5210.74) and (393,5211.27) .. (393,5211.93) .. controls (393,5212.58) and (392.44,5213.11) .. (391.75,5213.11) .. controls (391.06,5213.11) and (390.5,5212.58) .. (390.5,5211.93) -- cycle ; \draw  [line width=1.5]  (381.25,5220.7) .. controls (385.41,5223.86) and (389.58,5223.86) .. (393.75,5220.7) ;
    %Shape: Smiley Face [id:dp2085578545555946] 
    \draw  [line width=1.5]  (190.31,5173) .. controls (190.31,5166.45) and (185,5161.14) .. (178.46,5161.14) .. controls (171.92,5161.14) and (166.61,5166.45) .. (166.61,5173) .. controls (166.61,5179.55) and (171.92,5184.86) .. (178.46,5184.86) .. controls (185,5184.86) and (190.31,5179.55) .. (190.31,5173) -- cycle ; \draw  [line width=1.5]  (183.67,5168.97) .. controls (183.67,5168.31) and (183.14,5167.78) .. (182.49,5167.78) .. controls (181.83,5167.78) and (181.3,5168.31) .. (181.3,5168.97) .. controls (181.3,5169.62) and (181.83,5170.15) .. (182.49,5170.15) .. controls (183.14,5170.15) and (183.67,5169.62) .. (183.67,5168.97) -- cycle ; \draw  [line width=1.5]  (175.62,5168.97) .. controls (175.62,5168.31) and (175.09,5167.78) .. (174.43,5167.78) .. controls (173.78,5167.78) and (173.25,5168.31) .. (173.25,5168.97) .. controls (173.25,5169.62) and (173.78,5170.15) .. (174.43,5170.15) .. controls (175.09,5170.15) and (175.62,5169.62) .. (175.62,5168.97) -- cycle ; \draw  [line width=1.5]  (184.38,5177.74) .. controls (180.43,5180.91) and (176.49,5180.91) .. (172.54,5177.74) ;
    %Shape: Arc [id:dp5937201326362099] 
    \draw  [draw opacity=0] (340.3,5212.7) .. controls (340.3,5212.7) and (340.3,5212.7) .. (340.3,5212.7) .. controls (340.3,5212.7) and (340.3,5212.7) .. (340.3,5212.7) .. controls (336.24,5211.16) and (335.6,5202.79) .. (338.88,5194.01) .. controls (342.16,5185.22) and (348.12,5179.34) .. (352.19,5180.88) -- (346.25,5196.79) -- cycle ; \draw   (340.3,5212.7) .. controls (340.3,5212.7) and (340.3,5212.7) .. (340.3,5212.7) .. controls (340.3,5212.7) and (340.3,5212.7) .. (340.3,5212.7) .. controls (336.24,5211.16) and (335.6,5202.79) .. (338.88,5194.01) .. controls (342.16,5185.22) and (348.12,5179.34) .. (352.19,5180.88) ;
    %Shape: Arc [id:dp6639911291514243] 
    \draw  [draw opacity=0] (346.07,5212.28) .. controls (346.07,5212.28) and (346.07,5212.28) .. (346.07,5212.28) .. controls (342,5210.74) and (340.98,5203.39) .. (343.79,5195.86) .. controls (346.6,5188.33) and (352.18,5183.47) .. (356.25,5185.01) -- (351.16,5198.64) -- cycle ; \draw   (346.07,5212.28) .. controls (346.07,5212.28) and (346.07,5212.28) .. (346.07,5212.28) .. controls (342,5210.74) and (340.98,5203.39) .. (343.79,5195.86) .. controls (346.6,5188.33) and (352.18,5183.47) .. (356.25,5185.01) ;
    %Shape: Arc [id:dp2509905687869437] 
    \draw  [draw opacity=0] (351.83,5211.86) .. controls (351.83,5211.86) and (351.83,5211.86) .. (351.83,5211.86) .. controls (347.76,5210.32) and (346.36,5203.99) .. (348.7,5197.72) .. controls (351.05,5191.44) and (356.24,5187.6) .. (360.31,5189.13) -- (356.07,5200.5) -- cycle ; \draw   (351.83,5211.86) .. controls (351.83,5211.86) and (351.83,5211.86) .. (351.83,5211.86) .. controls (347.76,5210.32) and (346.36,5203.99) .. (348.7,5197.72) .. controls (351.05,5191.44) and (356.24,5187.6) .. (360.31,5189.13) ;
    %Shape: Arc [id:dp1612107460365959] 
    \draw  [draw opacity=0] (357.59,5211.44) .. controls (357.59,5211.44) and (357.59,5211.44) .. (357.59,5211.44) .. controls (353.52,5209.9) and (351.74,5204.59) .. (353.61,5199.57) .. controls (355.49,5194.55) and (360.31,5191.72) .. (364.38,5193.26) -- (360.98,5202.35) -- cycle ; \draw   (357.59,5211.44) .. controls (357.59,5211.44) and (357.59,5211.44) .. (357.59,5211.44) .. controls (353.52,5209.9) and (351.74,5204.59) .. (353.61,5199.57) .. controls (355.49,5194.55) and (360.31,5191.72) .. (364.38,5193.26) ;
    %Shape: Arc [id:dp8336468620330247] 
    \draw  [draw opacity=0] (363.35,5211.02) .. controls (359.28,5209.49) and (357.12,5205.19) .. (358.53,5201.42) .. controls (359.93,5197.66) and (364.37,5195.85) .. (368.44,5197.39) .. controls (368.44,5197.39) and (368.44,5197.39) .. (368.44,5197.39) -- (365.89,5204.2) -- cycle ; \draw   (363.35,5211.02) .. controls (359.28,5209.49) and (357.12,5205.19) .. (358.53,5201.42) .. controls (359.93,5197.66) and (364.37,5195.85) .. (368.44,5197.39) .. controls (368.44,5197.39) and (368.44,5197.39) .. (368.44,5197.39) ;

    %Shape: Arc [id:dp4042411603143603] 
    \draw  [draw opacity=0] (213.9,5194.19) .. controls (213.9,5194.19) and (213.9,5194.19) .. (213.9,5194.19) .. controls (213.9,5194.19) and (213.9,5194.19) .. (213.9,5194.19) .. controls (218.01,5194.64) and (222.16,5187.45) .. (223.17,5178.15) .. controls (224.18,5168.84) and (221.67,5160.93) .. (217.57,5160.48) -- (215.73,5177.34) -- cycle ; \draw   (213.9,5194.19) .. controls (213.9,5194.19) and (213.9,5194.19) .. (213.9,5194.19) .. controls (213.9,5194.19) and (213.9,5194.19) .. (213.9,5194.19) .. controls (218.01,5194.64) and (222.16,5187.45) .. (223.17,5178.15) .. controls (224.18,5168.84) and (221.67,5160.93) .. (217.57,5160.48) ;
    %Shape: Arc [id:dp5065568635976311] 
    \draw  [draw opacity=0] (209.21,5191.24) .. controls (209.21,5191.24) and (209.21,5191.24) .. (209.21,5191.24) .. controls (213.32,5191.69) and (217.35,5185.58) .. (218.21,5177.6) .. controls (219.08,5169.63) and (216.46,5162.8) .. (212.35,5162.35) -- (210.78,5176.8) -- cycle ; \draw   (209.21,5191.24) .. controls (209.21,5191.24) and (209.21,5191.24) .. (209.21,5191.24) .. controls (213.32,5191.69) and (217.35,5185.58) .. (218.21,5177.6) .. controls (219.08,5169.63) and (216.46,5162.8) .. (212.35,5162.35) ;
    %Shape: Arc [id:dp6862390383553492] 
    \draw  [draw opacity=0] (204.52,5188.3) .. controls (204.52,5188.3) and (204.52,5188.3) .. (204.52,5188.3) .. controls (208.62,5188.74) and (212.54,5183.71) .. (213.26,5177.06) .. controls (213.98,5170.42) and (211.24,5164.66) .. (207.13,5164.22) -- (205.83,5176.26) -- cycle ; \draw   (204.52,5188.3) .. controls (204.52,5188.3) and (204.52,5188.3) .. (204.52,5188.3) .. controls (208.62,5188.74) and (212.54,5183.71) .. (213.26,5177.06) .. controls (213.98,5170.42) and (211.24,5164.66) .. (207.13,5164.22) ;
    %Shape: Arc [id:dp525263387926934] 
    \draw  [draw opacity=0] (199.83,5185.35) .. controls (199.83,5185.35) and (199.83,5185.35) .. (199.83,5185.35) .. controls (199.83,5185.35) and (199.83,5185.35) .. (199.83,5185.35) .. controls (203.93,5185.79) and (207.73,5181.84) .. (208.31,5176.52) .. controls (208.88,5171.21) and (206.02,5166.53) .. (201.92,5166.09) -- (200.87,5175.72) -- cycle ; \draw   (199.83,5185.35) .. controls (199.83,5185.35) and (199.83,5185.35) .. (199.83,5185.35) .. controls (199.83,5185.35) and (199.83,5185.35) .. (199.83,5185.35) .. controls (203.93,5185.79) and (207.73,5181.84) .. (208.31,5176.52) .. controls (208.88,5171.21) and (206.02,5166.53) .. (201.92,5166.09) ;
    %Shape: Arc [id:dp7742397888460077] 
    \draw  [draw opacity=0] (195.13,5182.4) .. controls (195.13,5182.4) and (195.13,5182.4) .. (195.13,5182.4) .. controls (199.24,5182.85) and (202.92,5179.97) .. (203.35,5175.98) .. controls (203.78,5172) and (200.81,5168.4) .. (196.7,5167.95) .. controls (196.7,5167.95) and (196.7,5167.95) .. (196.7,5167.95) -- (195.92,5175.18) -- cycle ; \draw   (195.13,5182.4) .. controls (195.13,5182.4) and (195.13,5182.4) .. (195.13,5182.4) .. controls (199.24,5182.85) and (202.92,5179.97) .. (203.35,5175.98) .. controls (203.78,5172) and (200.81,5168.4) .. (196.7,5167.95) .. controls (196.7,5167.95) and (196.7,5167.95) .. (196.7,5167.95) ;

    %Right Arrow [id:dp8983534975244674] 
    \draw   (342.5,5060) -- (342.5,5110.26) -- (350,5110.26) -- (335,5129.35) -- (320,5110.26) -- (327.5,5110.26) -- (327.5,5060) -- cycle ;
    %Right Arrow [id:dp7985368465895043] 
    \draw   (255.66,4936.27) -- (291.2,4971.8) -- (296.5,4966.5) -- (299.39,4990.61) -- (275.29,4987.71) -- (280.59,4982.41) -- (245.05,4946.87) -- cycle ;
    %Right Arrow [id:dp2336720313313474] 
    \draw   (414.95,4946.87) -- (379.41,4982.41) -- (384.71,4987.71) -- (360.61,4990.61) -- (363.5,4966.5) -- (368.8,4971.8) -- (404.34,4936.27) -- cycle ;
    %Shape: Arc [id:dp47517974435288335] 
    \draw  [draw opacity=0] (298.48,4587.43) .. controls (298.48,4587.43) and (298.48,4587.43) .. (298.48,4587.43) .. controls (298.48,4587.43) and (298.48,4587.43) .. (298.48,4587.43) .. controls (300.29,4583.52) and (294.91,4577.08) .. (286.45,4573.06) .. controls (277.99,4569.03) and (269.66,4568.94) .. (267.84,4572.85) -- (283.16,4580.14) -- cycle ; \draw   (298.48,4587.43) .. controls (298.48,4587.43) and (298.48,4587.43) .. (298.48,4587.43) .. controls (298.48,4587.43) and (298.48,4587.43) .. (298.48,4587.43) .. controls (300.29,4583.52) and (294.91,4577.08) .. (286.45,4573.06) .. controls (277.99,4569.03) and (269.66,4568.94) .. (267.84,4572.85) ;
    %Shape: Arc [id:dp5873107211523902] 
    \draw  [draw opacity=0] (294.1,4591.11) .. controls (294.1,4591.11) and (294.1,4591.11) .. (294.1,4591.11) .. controls (295.91,4587.2) and (291.51,4581.23) .. (284.26,4577.78) .. controls (277.01,4574.33) and (269.66,4574.71) .. (267.84,4578.62) -- (280.97,4584.87) -- cycle ; \draw   (294.1,4591.11) .. controls (294.1,4591.11) and (294.1,4591.11) .. (294.1,4591.11) .. controls (295.91,4587.2) and (291.51,4581.23) .. (284.26,4577.78) .. controls (277.01,4574.33) and (269.66,4574.71) .. (267.84,4578.62) ;
    %Shape: Arc [id:dp5846911046296042] 
    \draw  [draw opacity=0] (289.71,4594.8) .. controls (291.53,4590.88) and (288.11,4585.38) .. (282.06,4582.51) .. controls (276.02,4579.63) and (269.65,4580.47) .. (267.83,4584.38) -- (278.77,4589.59) -- cycle ; \draw   (289.71,4594.8) .. controls (291.53,4590.88) and (288.11,4585.38) .. (282.06,4582.51) .. controls (276.02,4579.63) and (269.65,4580.47) .. (267.83,4584.38) ;
    %Shape: Arc [id:dp25372532150421623] 
    \draw  [draw opacity=0] (285.33,4598.48) .. controls (285.33,4598.48) and (285.33,4598.48) .. (285.33,4598.48) .. controls (287.15,4594.57) and (284.7,4589.53) .. (279.87,4587.23) .. controls (275.04,4584.93) and (269.64,4586.23) .. (267.83,4590.15) -- (276.58,4594.31) -- cycle ; \draw   (285.33,4598.48) .. controls (285.33,4598.48) and (285.33,4598.48) .. (285.33,4598.48) .. controls (287.15,4594.57) and (284.7,4589.53) .. (279.87,4587.23) .. controls (275.04,4584.93) and (269.64,4586.23) .. (267.83,4590.15) ;
    %Shape: Arc [id:dp8262344630748313] 
    \draw  [draw opacity=0] (280.95,4602.16) .. controls (280.95,4602.16) and (280.95,4602.16) .. (280.95,4602.16) .. controls (282.77,4598.25) and (281.3,4593.68) .. (277.68,4591.95) .. controls (274.05,4590.23) and (269.64,4592) .. (267.82,4595.91) -- (274.39,4599.04) -- cycle ; \draw   (280.95,4602.16) .. controls (280.95,4602.16) and (280.95,4602.16) .. (280.95,4602.16) .. controls (282.77,4598.25) and (281.3,4593.68) .. (277.68,4591.95) .. controls (274.05,4590.23) and (269.64,4592) .. (267.82,4595.91) ;

    %Shape: Smiley Face [id:dp3125315195686723] 
    \draw  [line width=1.5]  (276.77,4616.09) .. controls (276.77,4609.54) and (271.23,4604.23) .. (264.39,4604.23) .. controls (257.55,4604.23) and (252,4609.54) .. (252,4616.09) .. controls (252,4622.64) and (257.55,4627.95) .. (264.39,4627.95) .. controls (271.23,4627.95) and (276.77,4622.64) .. (276.77,4616.09) -- cycle ; \draw  [line width=1.5]  (269.84,4612.06) .. controls (269.84,4611.41) and (269.28,4610.88) .. (268.6,4610.88) .. controls (267.91,4610.88) and (267.36,4611.41) .. (267.36,4612.06) .. controls (267.36,4612.72) and (267.91,4613.25) .. (268.6,4613.25) .. controls (269.28,4613.25) and (269.84,4612.72) .. (269.84,4612.06) -- cycle ; \draw  [line width=1.5]  (261.41,4612.06) .. controls (261.41,4611.41) and (260.86,4610.88) .. (260.17,4610.88) .. controls (259.49,4610.88) and (258.94,4611.41) .. (258.94,4612.06) .. controls (258.94,4612.72) and (259.49,4613.25) .. (260.17,4613.25) .. controls (260.86,4613.25) and (261.41,4612.72) .. (261.41,4612.06) -- cycle ; \draw  [line width=1.5]  (270.58,4620.84) .. controls (266.45,4624) and (262.32,4624) .. (258.19,4620.84) ;
    %Shape: Smiley Face [id:dp5486818330347384] 
    \draw  [line width=1.5]  (246.99,4504.81) .. controls (246.99,4498.26) and (252.59,4492.95) .. (259.5,4492.95) .. controls (266.4,4492.95) and (272,4498.26) .. (272,4504.81) .. controls (272,4511.36) and (266.4,4516.67) .. (259.5,4516.67) .. controls (252.59,4516.67) and (246.99,4511.36) .. (246.99,4504.81) -- cycle ; \draw  [line width=1.5]  (254,4500.78) .. controls (254,4500.13) and (254.56,4499.59) .. (255.25,4499.59) .. controls (255.94,4499.59) and (256.5,4500.13) .. (256.5,4500.78) .. controls (256.5,4501.43) and (255.94,4501.97) .. (255.25,4501.97) .. controls (254.56,4501.97) and (254,4501.43) .. (254,4500.78) -- cycle ; \draw  [line width=1.5]  (262.5,4500.78) .. controls (262.5,4500.13) and (263.06,4499.59) .. (263.75,4499.59) .. controls (264.44,4499.59) and (265,4500.13) .. (265,4500.78) .. controls (265,4501.43) and (264.44,4501.97) .. (263.75,4501.97) .. controls (263.06,4501.97) and (262.5,4501.43) .. (262.5,4500.78) -- cycle ; \draw  [line width=1.5]  (253.25,4509.56) .. controls (257.41,4512.72) and (261.58,4512.72) .. (265.75,4509.56) ;
    %Shape: Smiley Face [id:dp043685298112697835] 
    \draw  [line width=1.5]  (428.61,4514.1) .. controls (428.61,4507.55) and (434.2,4502.24) .. (441.11,4502.24) .. controls (448.01,4502.24) and (453.61,4507.55) .. (453.61,4514.1) .. controls (453.61,4520.65) and (448.01,4525.96) .. (441.11,4525.96) .. controls (434.2,4525.96) and (428.61,4520.65) .. (428.61,4514.1) -- cycle ; \draw  [line width=1.5]  (435.61,4510.07) .. controls (435.61,4509.41) and (436.17,4508.88) .. (436.86,4508.88) .. controls (437.55,4508.88) and (438.11,4509.41) .. (438.11,4510.07) .. controls (438.11,4510.72) and (437.55,4511.25) .. (436.86,4511.25) .. controls (436.17,4511.25) and (435.61,4510.72) .. (435.61,4510.07) -- cycle ; \draw  [line width=1.5]  (444.11,4510.07) .. controls (444.11,4509.41) and (444.67,4508.88) .. (445.36,4508.88) .. controls (446.05,4508.88) and (446.61,4509.41) .. (446.61,4510.07) .. controls (446.61,4510.72) and (446.05,4511.25) .. (445.36,4511.25) .. controls (444.67,4511.25) and (444.11,4510.72) .. (444.11,4510.07) -- cycle ; \draw  [line width=1.5]  (434.86,4518.84) .. controls (439.02,4522) and (443.19,4522) .. (447.36,4518.84) ;
    %Shape: Arc [id:dp7415390794067207] 
    \draw  [draw opacity=0] (300.04,4506.55) .. controls (300.04,4506.55) and (300.04,4506.55) .. (300.04,4506.55) .. controls (300.04,4506.55) and (300.04,4506.55) .. (300.04,4506.55) .. controls (303.88,4508.58) and (303.47,4516.96) .. (299.12,4525.27) .. controls (294.77,4533.58) and (288.13,4538.67) .. (284.28,4536.64) -- (292.16,4521.6) -- cycle ; \draw   (300.04,4506.55) .. controls (300.04,4506.55) and (300.04,4506.55) .. (300.04,4506.55) .. controls (300.04,4506.55) and (300.04,4506.55) .. (300.04,4506.55) .. controls (303.88,4508.58) and (303.47,4516.96) .. (299.12,4525.27) .. controls (294.77,4533.58) and (288.13,4538.67) .. (284.28,4536.64) ;
    %Shape: Arc [id:dp7225483025795462] 
    \draw  [draw opacity=0] (294.27,4506.25) .. controls (294.27,4506.25) and (294.27,4506.25) .. (294.27,4506.25) .. controls (298.11,4508.28) and (298.21,4515.7) .. (294.48,4522.82) .. controls (290.75,4529.95) and (284.61,4534.07) .. (280.77,4532.04) -- (287.52,4519.15) -- cycle ; \draw   (294.27,4506.25) .. controls (294.27,4506.25) and (294.27,4506.25) .. (294.27,4506.25) .. controls (298.11,4508.28) and (298.21,4515.7) .. (294.48,4522.82) .. controls (290.75,4529.95) and (284.61,4534.07) .. (280.77,4532.04) ;
    %Shape: Arc [id:dp8508292273083863] 
    \draw  [draw opacity=0] (288.5,4505.95) .. controls (292.35,4507.98) and (292.95,4514.44) .. (289.84,4520.37) .. controls (286.73,4526.31) and (281.1,4529.47) .. (277.25,4527.44) -- (282.87,4516.7) -- cycle ; \draw   (288.5,4505.95) .. controls (292.35,4507.98) and (292.95,4514.44) .. (289.84,4520.37) .. controls (286.73,4526.31) and (281.1,4529.47) .. (277.25,4527.44) ;
    %Shape: Arc [id:dp041354587129001086] 
    \draw  [draw opacity=0] (282.73,4505.65) .. controls (282.73,4505.65) and (282.73,4505.65) .. (282.73,4505.65) .. controls (286.58,4507.68) and (287.68,4513.17) .. (285.2,4517.92) .. controls (282.71,4522.67) and (277.58,4524.87) .. (273.73,4522.84) -- (278.23,4514.25) -- cycle ; \draw   (282.73,4505.65) .. controls (282.73,4505.65) and (282.73,4505.65) .. (282.73,4505.65) .. controls (286.58,4507.68) and (287.68,4513.17) .. (285.2,4517.92) .. controls (282.71,4522.67) and (277.58,4524.87) .. (273.73,4522.84) ;
    %Shape: Arc [id:dp025846342447859216] 
    \draw  [draw opacity=0] (276.96,4505.35) .. controls (280.81,4507.38) and (282.42,4511.91) .. (280.55,4515.47) .. controls (278.69,4519.03) and (274.06,4520.27) .. (270.21,4518.24) .. controls (270.21,4518.24) and (270.21,4518.24) .. (270.21,4518.24) -- (273.59,4511.8) -- cycle ; \draw   (276.96,4505.35) .. controls (280.81,4507.38) and (282.42,4511.91) .. (280.55,4515.47) .. controls (278.69,4519.03) and (274.06,4520.27) .. (270.21,4518.24) .. controls (270.21,4518.24) and (270.21,4518.24) .. (270.21,4518.24) ;

    %Shape: Arc [id:dp930200188188434] 
    \draw  [draw opacity=0] (403.71,4535.29) .. controls (403.71,4535.29) and (403.71,4535.29) .. (403.71,4535.29) .. controls (403.71,4535.29) and (403.71,4535.29) .. (403.71,4535.29) .. controls (399.38,4535.74) and (395,4528.55) .. (393.93,4519.24) .. controls (392.87,4509.93) and (395.51,4502.03) .. (399.84,4501.58) -- (401.78,4518.43) -- cycle ; \draw   (403.71,4535.29) .. controls (403.71,4535.29) and (403.71,4535.29) .. (403.71,4535.29) .. controls (403.71,4535.29) and (403.71,4535.29) .. (403.71,4535.29) .. controls (399.38,4535.74) and (395,4528.55) .. (393.93,4519.24) .. controls (392.87,4509.93) and (395.51,4502.03) .. (399.84,4501.58) ;
    %Shape: Arc [id:dp6517224450556985] 
    \draw  [draw opacity=0] (408.66,4532.34) .. controls (404.33,4532.79) and (400.07,4526.68) .. (399.16,4518.7) .. controls (398.25,4510.72) and (401.02,4503.89) .. (405.35,4503.45) -- (407,4517.89) -- cycle ; \draw   (408.66,4532.34) .. controls (404.33,4532.79) and (400.07,4526.68) .. (399.16,4518.7) .. controls (398.25,4510.72) and (401.02,4503.89) .. (405.35,4503.45) ;
    %Shape: Arc [id:dp9738880031041819] 
    \draw  [draw opacity=0] (413.61,4529.39) .. controls (409.28,4529.84) and (405.15,4524.81) .. (404.39,4518.16) .. controls (403.63,4511.51) and (406.52,4505.76) .. (410.85,4505.31) -- (412.23,4517.35) -- cycle ; \draw   (413.61,4529.39) .. controls (409.28,4529.84) and (405.15,4524.81) .. (404.39,4518.16) .. controls (403.63,4511.51) and (406.52,4505.76) .. (410.85,4505.31) ;
    %Shape: Arc [id:dp17234426159324023] 
    \draw  [draw opacity=0] (418.56,4526.44) .. controls (418.56,4526.44) and (418.56,4526.44) .. (418.56,4526.44) .. controls (414.23,4526.89) and (410.23,4522.94) .. (409.62,4517.62) .. controls (409.01,4512.3) and (412.02,4507.63) .. (416.35,4507.18) -- (417.46,4516.81) -- cycle ; \draw   (418.56,4526.44) .. controls (418.56,4526.44) and (418.56,4526.44) .. (418.56,4526.44) .. controls (414.23,4526.89) and (410.23,4522.94) .. (409.62,4517.62) .. controls (409.01,4512.3) and (412.02,4507.63) .. (416.35,4507.18) ;
    %Shape: Arc [id:dp34652823860233317] 
    \draw  [draw opacity=0] (423.51,4523.5) .. controls (423.51,4523.5) and (423.51,4523.5) .. (423.51,4523.5) .. controls (419.18,4523.94) and (415.3,4521.07) .. (414.84,4517.08) .. controls (414.39,4513.09) and (417.53,4509.5) .. (421.86,4509.05) -- (422.69,4516.27) -- cycle ; \draw   (423.51,4523.5) .. controls (423.51,4523.5) and (423.51,4523.5) .. (423.51,4523.5) .. controls (419.18,4523.94) and (415.3,4521.07) .. (414.84,4517.08) .. controls (414.39,4513.09) and (417.53,4509.5) .. (421.86,4509.05) ;

    %Shape: Arc [id:dp6957485207604929] 
    \draw  [draw opacity=0] (445.09,4562.44) .. controls (443.25,4558.52) and (448.69,4552.09) .. (457.23,4548.06) .. controls (465.76,4544.03) and (474.17,4543.94) .. (476.01,4547.85) .. controls (476.01,4547.85) and (476.01,4547.85) .. (476.01,4547.85) -- (460.55,4555.15) -- cycle ; \draw   (445.09,4562.44) .. controls (443.25,4558.52) and (448.69,4552.09) .. (457.23,4548.06) .. controls (465.76,4544.03) and (474.17,4543.94) .. (476.01,4547.85) .. controls (476.01,4547.85) and (476.01,4547.85) .. (476.01,4547.85) ;
    %Shape: Arc [id:dp13816808192600538] 
    \draw  [draw opacity=0] (449.51,4566.12) .. controls (449.51,4566.12) and (449.51,4566.12) .. (449.51,4566.12) .. controls (447.68,4562.21) and (452.12,4556.24) .. (459.44,4552.78) .. controls (466.76,4549.33) and (474.18,4549.71) .. (476.01,4553.62) -- (462.76,4559.87) -- cycle ; \draw   (449.51,4566.12) .. controls (449.51,4566.12) and (449.51,4566.12) .. (449.51,4566.12) .. controls (447.68,4562.21) and (452.12,4556.24) .. (459.44,4552.78) .. controls (466.76,4549.33) and (474.18,4549.71) .. (476.01,4553.62) ;
    %Shape: Arc [id:dp12179882005890796] 
    \draw  [draw opacity=0] (453.93,4569.8) .. controls (453.93,4569.8) and (453.93,4569.8) .. (453.93,4569.8) .. controls (452.1,4565.89) and (455.55,4560.38) .. (461.65,4557.51) .. controls (467.75,4554.63) and (474.18,4555.47) .. (476.02,4559.38) -- (464.97,4564.59) -- cycle ; \draw   (453.93,4569.8) .. controls (453.93,4569.8) and (453.93,4569.8) .. (453.93,4569.8) .. controls (452.1,4565.89) and (455.55,4560.38) .. (461.65,4557.51) .. controls (467.75,4554.63) and (474.18,4555.47) .. (476.02,4559.38) ;
    %Shape: Arc [id:dp42923215849981666] 
    \draw  [draw opacity=0] (458.35,4573.48) .. controls (458.35,4573.48) and (458.35,4573.48) .. (458.35,4573.48) .. controls (456.52,4569.57) and (458.99,4564.53) .. (463.87,4562.23) .. controls (468.75,4559.93) and (474.19,4561.24) .. (476.02,4565.15) -- (467.19,4569.32) -- cycle ; \draw   (458.35,4573.48) .. controls (458.35,4573.48) and (458.35,4573.48) .. (458.35,4573.48) .. controls (456.52,4569.57) and (458.99,4564.53) .. (463.87,4562.23) .. controls (468.75,4559.93) and (474.19,4561.24) .. (476.02,4565.15) ;
    %Shape: Arc [id:dp29112603901575573] 
    \draw  [draw opacity=0] (462.78,4577.16) .. controls (460.94,4573.25) and (462.42,4568.68) .. (466.08,4566.95) .. controls (469.74,4565.23) and (474.19,4567) .. (476.03,4570.91) -- (469.4,4574.04) -- cycle ; \draw   (462.78,4577.16) .. controls (460.94,4573.25) and (462.42,4568.68) .. (466.08,4566.95) .. controls (469.74,4565.23) and (474.19,4567) .. (476.03,4570.91) ;

    %Shape: Smiley Face [id:dp20793953727316972] 
    \draw  [line width=1.5]  (466.99,4591.1) .. controls (466.99,4584.55) and (472.59,4579.24) .. (479.5,4579.24) .. controls (486.4,4579.24) and (492,4584.55) .. (492,4591.1) .. controls (492,4597.65) and (486.4,4602.96) .. (479.5,4602.96) .. controls (472.59,4602.96) and (466.99,4597.65) .. (466.99,4591.1) -- cycle ; \draw  [line width=1.5]  (474,4587.07) .. controls (474,4586.41) and (474.56,4585.88) .. (475.25,4585.88) .. controls (475.94,4585.88) and (476.5,4586.41) .. (476.5,4587.07) .. controls (476.5,4587.72) and (475.94,4588.25) .. (475.25,4588.25) .. controls (474.56,4588.25) and (474,4587.72) .. (474,4587.07) -- cycle ; \draw  [line width=1.5]  (482.5,4587.07) .. controls (482.5,4586.41) and (483.06,4585.88) .. (483.75,4585.88) .. controls (484.44,4585.88) and (485,4586.41) .. (485,4587.07) .. controls (485,4587.72) and (484.44,4588.25) .. (483.75,4588.25) .. controls (483.06,4588.25) and (482.5,4587.72) .. (482.5,4587.07) -- cycle ; \draw  [line width=1.5]  (473.25,4595.84) .. controls (477.41,4599) and (481.58,4599) .. (485.75,4595.84) ;
    %Shape: Smiley Face [id:dp26956258976076386] 
    \draw  [line width=1.5]  (376.99,4593.91) .. controls (376.99,4587.36) and (382.59,4582.05) .. (389.5,4582.05) .. controls (396.4,4582.05) and (402,4587.36) .. (402,4593.91) .. controls (402,4600.46) and (396.4,4605.77) .. (389.5,4605.77) .. controls (382.59,4605.77) and (376.99,4600.46) .. (376.99,4593.91) -- cycle ; \draw  [line width=1.5]  (384,4589.88) .. controls (384,4589.22) and (384.56,4588.69) .. (385.25,4588.69) .. controls (385.94,4588.69) and (386.5,4589.22) .. (386.5,4589.88) .. controls (386.5,4590.53) and (385.94,4591.06) .. (385.25,4591.06) .. controls (384.56,4591.06) and (384,4590.53) .. (384,4589.88) -- cycle ; \draw  [line width=1.5]  (392.5,4589.88) .. controls (392.5,4589.22) and (393.06,4588.69) .. (393.75,4588.69) .. controls (394.44,4588.69) and (395,4589.22) .. (395,4589.88) .. controls (395,4590.53) and (394.44,4591.06) .. (393.75,4591.06) .. controls (393.06,4591.06) and (392.5,4590.53) .. (392.5,4589.88) -- cycle ; \draw  [line width=1.5]  (383.25,4598.65) .. controls (387.41,4601.82) and (391.58,4601.82) .. (395.75,4598.65) ;
    %Shape: Smiley Face [id:dp5651334377856084] 
    \draw  [line width=1.5]  (192.31,4550.95) .. controls (192.31,4544.4) and (187,4539.09) .. (180.46,4539.09) .. controls (173.92,4539.09) and (168.61,4544.4) .. (168.61,4550.95) .. controls (168.61,4557.5) and (173.92,4562.81) .. (180.46,4562.81) .. controls (187,4562.81) and (192.31,4557.5) .. (192.31,4550.95) -- cycle ; \draw  [line width=1.5]  (185.67,4546.92) .. controls (185.67,4546.27) and (185.14,4545.74) .. (184.49,4545.74) .. controls (183.83,4545.74) and (183.3,4546.27) .. (183.3,4546.92) .. controls (183.3,4547.58) and (183.83,4548.11) .. (184.49,4548.11) .. controls (185.14,4548.11) and (185.67,4547.58) .. (185.67,4546.92) -- cycle ; \draw  [line width=1.5]  (177.62,4546.92) .. controls (177.62,4546.27) and (177.09,4545.74) .. (176.43,4545.74) .. controls (175.78,4545.74) and (175.25,4546.27) .. (175.25,4546.92) .. controls (175.25,4547.58) and (175.78,4548.11) .. (176.43,4548.11) .. controls (177.09,4548.11) and (177.62,4547.58) .. (177.62,4546.92) -- cycle ; \draw  [line width=1.5]  (186.38,4555.7) .. controls (182.43,4558.86) and (178.49,4558.86) .. (174.54,4555.7) ;
    %Shape: Arc [id:dp003791797532442631] 
    \draw  [draw opacity=0] (342.3,4590.65) .. controls (342.3,4590.65) and (342.3,4590.65) .. (342.3,4590.65) .. controls (342.3,4590.65) and (342.3,4590.65) .. (342.3,4590.65) .. controls (338.24,4589.12) and (337.6,4580.75) .. (340.88,4571.96) .. controls (344.16,4563.18) and (350.12,4557.3) .. (354.19,4558.83) -- (348.25,4574.74) -- cycle ; \draw   (342.3,4590.65) .. controls (342.3,4590.65) and (342.3,4590.65) .. (342.3,4590.65) .. controls (342.3,4590.65) and (342.3,4590.65) .. (342.3,4590.65) .. controls (338.24,4589.12) and (337.6,4580.75) .. (340.88,4571.96) .. controls (344.16,4563.18) and (350.12,4557.3) .. (354.19,4558.83) ;
    %Shape: Arc [id:dp9215892916401076] 
    \draw  [draw opacity=0] (348.07,4590.23) .. controls (348.07,4590.23) and (348.07,4590.23) .. (348.07,4590.23) .. controls (344,4588.7) and (342.98,4581.35) .. (345.79,4573.82) .. controls (348.6,4566.28) and (354.18,4561.42) .. (358.25,4562.96) -- (353.16,4576.6) -- cycle ; \draw   (348.07,4590.23) .. controls (348.07,4590.23) and (348.07,4590.23) .. (348.07,4590.23) .. controls (344,4588.7) and (342.98,4581.35) .. (345.79,4573.82) .. controls (348.6,4566.28) and (354.18,4561.42) .. (358.25,4562.96) ;
    %Shape: Arc [id:dp9291574986564861] 
    \draw  [draw opacity=0] (353.83,4589.81) .. controls (353.83,4589.81) and (353.83,4589.81) .. (353.83,4589.81) .. controls (349.76,4588.28) and (348.36,4581.94) .. (350.7,4575.67) .. controls (353.05,4569.39) and (358.24,4565.55) .. (362.31,4567.09) -- (358.07,4578.45) -- cycle ; \draw   (353.83,4589.81) .. controls (353.83,4589.81) and (353.83,4589.81) .. (353.83,4589.81) .. controls (349.76,4588.28) and (348.36,4581.94) .. (350.7,4575.67) .. controls (353.05,4569.39) and (358.24,4565.55) .. (362.31,4567.09) ;
    %Shape: Arc [id:dp10824765149686433] 
    \draw  [draw opacity=0] (359.59,4589.39) .. controls (359.59,4589.39) and (359.59,4589.39) .. (359.59,4589.39) .. controls (355.52,4587.86) and (353.74,4582.54) .. (355.61,4577.52) .. controls (357.49,4572.5) and (362.31,4569.68) .. (366.38,4571.21) -- (362.98,4580.3) -- cycle ; \draw   (359.59,4589.39) .. controls (359.59,4589.39) and (359.59,4589.39) .. (359.59,4589.39) .. controls (355.52,4587.86) and (353.74,4582.54) .. (355.61,4577.52) .. controls (357.49,4572.5) and (362.31,4569.68) .. (366.38,4571.21) ;
    %Shape: Arc [id:dp594481757967215] 
    \draw  [draw opacity=0] (365.35,4588.97) .. controls (361.28,4587.44) and (359.12,4583.14) .. (360.53,4579.38) .. controls (361.93,4575.61) and (366.37,4573.8) .. (370.44,4575.34) .. controls (370.44,4575.34) and (370.44,4575.34) .. (370.44,4575.34) -- (367.89,4582.16) -- cycle ; \draw   (365.35,4588.97) .. controls (361.28,4587.44) and (359.12,4583.14) .. (360.53,4579.38) .. controls (361.93,4575.61) and (366.37,4573.8) .. (370.44,4575.34) .. controls (370.44,4575.34) and (370.44,4575.34) .. (370.44,4575.34) ;

    %Shape: Arc [id:dp8020389812394149] 
    \draw  [draw opacity=0] (215.9,4572.14) .. controls (215.9,4572.14) and (215.9,4572.14) .. (215.9,4572.14) .. controls (215.9,4572.14) and (215.9,4572.14) .. (215.9,4572.14) .. controls (220.01,4572.59) and (224.16,4565.41) .. (225.17,4556.1) .. controls (226.18,4546.79) and (223.67,4538.88) .. (219.57,4538.43) -- (217.73,4555.29) -- cycle ; \draw   (215.9,4572.14) .. controls (215.9,4572.14) and (215.9,4572.14) .. (215.9,4572.14) .. controls (215.9,4572.14) and (215.9,4572.14) .. (215.9,4572.14) .. controls (220.01,4572.59) and (224.16,4565.41) .. (225.17,4556.1) .. controls (226.18,4546.79) and (223.67,4538.88) .. (219.57,4538.43) ;
    %Shape: Arc [id:dp33954898432123704] 
    \draw  [draw opacity=0] (211.21,4569.2) .. controls (211.21,4569.2) and (211.21,4569.2) .. (211.21,4569.2) .. controls (215.32,4569.64) and (219.35,4563.54) .. (220.21,4555.56) .. controls (221.08,4547.58) and (218.46,4540.75) .. (214.35,4540.3) -- (212.78,4554.75) -- cycle ; \draw   (211.21,4569.2) .. controls (211.21,4569.2) and (211.21,4569.2) .. (211.21,4569.2) .. controls (215.32,4569.64) and (219.35,4563.54) .. (220.21,4555.56) .. controls (221.08,4547.58) and (218.46,4540.75) .. (214.35,4540.3) ;
    %Shape: Arc [id:dp8405188678638126] 
    \draw  [draw opacity=0] (206.52,4566.25) .. controls (206.52,4566.25) and (206.52,4566.25) .. (206.52,4566.25) .. controls (210.62,4566.69) and (214.54,4561.67) .. (215.26,4555.02) .. controls (215.98,4548.37) and (213.24,4542.62) .. (209.13,4542.17) -- (207.83,4554.21) -- cycle ; \draw   (206.52,4566.25) .. controls (206.52,4566.25) and (206.52,4566.25) .. (206.52,4566.25) .. controls (210.62,4566.69) and (214.54,4561.67) .. (215.26,4555.02) .. controls (215.98,4548.37) and (213.24,4542.62) .. (209.13,4542.17) ;
    %Shape: Arc [id:dp8031120655797426] 
    \draw  [draw opacity=0] (201.83,4563.3) .. controls (201.83,4563.3) and (201.83,4563.3) .. (201.83,4563.3) .. controls (201.83,4563.3) and (201.83,4563.3) .. (201.83,4563.3) .. controls (205.93,4563.75) and (209.73,4559.8) .. (210.31,4554.48) .. controls (210.88,4549.16) and (208.02,4544.48) .. (203.92,4544.04) -- (202.87,4553.67) -- cycle ; \draw   (201.83,4563.3) .. controls (201.83,4563.3) and (201.83,4563.3) .. (201.83,4563.3) .. controls (201.83,4563.3) and (201.83,4563.3) .. (201.83,4563.3) .. controls (205.93,4563.75) and (209.73,4559.8) .. (210.31,4554.48) .. controls (210.88,4549.16) and (208.02,4544.48) .. (203.92,4544.04) ;
    %Shape: Arc [id:dp908455961450063] 
    \draw  [draw opacity=0] (197.13,4560.35) .. controls (197.13,4560.35) and (197.13,4560.35) .. (197.13,4560.35) .. controls (201.24,4560.8) and (204.92,4557.93) .. (205.35,4553.94) .. controls (205.78,4549.95) and (202.81,4546.35) .. (198.7,4545.91) .. controls (198.7,4545.91) and (198.7,4545.91) .. (198.7,4545.91) -- (197.92,4553.13) -- cycle ; \draw   (197.13,4560.35) .. controls (197.13,4560.35) and (197.13,4560.35) .. (197.13,4560.35) .. controls (201.24,4560.8) and (204.92,4557.93) .. (205.35,4553.94) .. controls (205.78,4549.95) and (202.81,4546.35) .. (198.7,4545.91) .. controls (198.7,4545.91) and (198.7,4545.91) .. (198.7,4545.91) ;

    %Right Arrow [id:dp23976616065108836] 
    \draw   (342.5,4640) -- (342.5,4690.26) -- (350,4690.26) -- (335,4709.35) -- (320,4690.26) -- (327.5,4690.26) -- (327.5,4640) -- cycle ;


    % Text Node
    \draw (330,4935) node  [font=\normalsize] [align=left] {\begin{minipage}[lt]{68pt}\setlength\topsep{0pt}
            \begin{center}
                {\Large \textbf{... via un ...}}
            \end{center}

        \end{minipage}};
    % Text Node
    \draw (420,4665) node  [font=\normalsize] [align=left] {\begin{minipage}[lt]{81.6pt}\setlength\topsep{0pt}
            \begin{center}
                {\Large \textbf{... sous des...}}
            \end{center}

        \end{minipage}};
    % Text Node
    \draw (340,4435) node  [font=\normalsize] [align=left] {\begin{minipage}[lt]{299.2pt}\setlength\topsep{0pt}
            \begin{center}
                {\Large \textbf{Un SMA de Cyberdéfense à designer...}}
            \end{center}

        \end{minipage}};
    % Text Node
    \draw (239,4626.5) node  [font=\LARGE] [align=left] {\begin{minipage}[lt]{28.35pt}\setlength\topsep{0pt}
            \begin{center}
                {\footnotesize $\displaystyle ?$}
            \end{center}

        \end{minipage}};
    % Text Node
    \draw (421,4613.5) node  [font=\LARGE] [align=left] {\begin{minipage}[lt]{28.35pt}\setlength\topsep{0pt}
            \begin{center}
                {\footnotesize $\displaystyle ?$}
            \end{center}

        \end{minipage}};
    % Text Node
    \draw (519,4563.5) node  [font=\LARGE] [align=left] {\begin{minipage}[lt]{28.35pt}\setlength\topsep{0pt}
            \begin{center}
                {\footnotesize $\displaystyle ?$}
            \end{center}

        \end{minipage}};
    % Text Node
    \draw (461,4473.5) node  [font=\LARGE] [align=left] {\begin{minipage}[lt]{28.35pt}\setlength\topsep{0pt}
            \begin{center}
                {\footnotesize $\displaystyle ?$}
            \end{center}

        \end{minipage}};
    % Text Node
    \draw (299,4473.5) node  [font=\LARGE] [align=left] {\begin{minipage}[lt]{28.35pt}\setlength\topsep{0pt}
            \begin{center}
                {\footnotesize $\displaystyle ?$}
            \end{center}

        \end{minipage}};
    % Text Node
    \draw (161,4523.5) node  [font=\LARGE] [align=left] {\begin{minipage}[lt]{28.35pt}\setlength\topsep{0pt}
            \begin{center}
                {\footnotesize $\displaystyle ?$}
            \end{center}

        \end{minipage}};
    % Text Node
    \draw (481,5103.5) node  [font=\LARGE] [align=left] {\begin{minipage}[lt]{28.35pt}\setlength\topsep{0pt}
            \begin{center}
                {\footnotesize $\displaystyle \mathbf{\textcolor[rgb]{0.31,0.89,0.76}{\pi _{6}}}$}
            \end{center}

        \end{minipage}};
    % Text Node
    \draw (519,5176.5) node  [font=\LARGE] [align=left] {\begin{minipage}[lt]{28.35pt}\setlength\topsep{0pt}
            \begin{center}
                {\footnotesize $\displaystyle \mathbf{\textcolor[rgb]{0.56,0.07,1}{\pi _{5}}}$}
            \end{center}

        \end{minipage}};
    % Text Node
    \draw (429,5226.5) node  [font=\LARGE] [align=left] {\begin{minipage}[lt]{28.35pt}\setlength\topsep{0pt}
            \begin{center}
                {\footnotesize $\displaystyle \mathbf{\textcolor[rgb]{0.74,0.06,0.88}{\pi _{4}}}$}
            \end{center}

        \end{minipage}};
    % Text Node
    \draw (221,5233.5) node  [font=\LARGE] [align=left] {\begin{minipage}[lt]{28.35pt}\setlength\topsep{0pt}
            \begin{center}
                {\footnotesize $\displaystyle \mathbf{\textcolor[rgb]{0.96,0.65,0.14}{\pi _{3}}}$}
            \end{center}

        \end{minipage}};
    % Text Node
    \draw (299,5093.5) node  [font=\LARGE] [align=left] {\begin{minipage}[lt]{28.35pt}\setlength\topsep{0pt}
            \begin{center}
                {\footnotesize $\displaystyle \mathbf{\textcolor[rgb]{0.82,0.01,0.11}{\pi }\textcolor[rgb]{0.82,0.01,0.11}{_{1}}}$}
            \end{center}

        \end{minipage}};
    % Text Node
    \draw (169,5123.5) node  [font=\LARGE] [align=left] {\begin{minipage}[lt]{28.35pt}\setlength\topsep{0pt}
            \begin{center}
                {\footnotesize $\displaystyle \mathbf{\textcolor[rgb]{0.49,0.83,0.13}{\pi _{2}}}$}
            \end{center}

        \end{minipage}};
    % Text Node
    \draw  [fill={rgb, 255:red, 245; green, 166; blue, 35 }  ,fill opacity=1 ]  (88,4998) -- (562,4998) -- (562,5052) -- (88,5052) -- cycle  ;
    \draw (325,5025) node  [font=\LARGE] [align=left] {\begin{minipage}[lt]{319.6pt}\setlength\topsep{0pt}
            \begin{center}
                \textbf{Processus de conception proposé...}
            \end{center}

        \end{minipage}};
    % Text Node
    \draw (197.5,4760) node   [align=left] {\begin{minipage}[lt]{248.4pt}\setlength\topsep{0pt}
            \begin{center}
                {\Large \textbf{Contraintes coonception SMA de Cyberdéfense}}\\\textit{(architectures, modèles, exigences...)}
            \end{center}

        \end{minipage}};
    % Text Node
    \draw (517.5,4765) node   [align=left] {\begin{minipage}[lt]{187pt}\setlength\topsep{0pt}
            \begin{center}
                {\Large \textbf{Contraintes environnement}}\\\textit{(contraintes déploiement, cyber-attaques, brouillage communications...)}
            \end{center}

        \end{minipage}};
    % Text Node
    \draw (234.5,4914.5) node   [align=left] {{\tiny \textit{\textbf{component 6}}}};
    % Text Node
    \draw (113.83,4914.5) node   [align=left] {{\tiny \textit{\textbf{component 5}}}};
    % Text Node
    \draw (74.5,4864.5) node   [align=left] {{\tiny \textit{\textbf{component 4}}}};
    % Text Node
    \draw (154.5,4864.5) node   [align=left] {{\tiny \textit{\textbf{component 3}}}};
    % Text Node
    \draw (154.5,4814.5) node   [align=left] {{\tiny \textit{\textbf{component 2}}}};
    % Text Node
    \draw (74.5,4814.5) node   [align=left] {{\tiny \textit{\textbf{component 1}}}};
    % Text Node
    \draw (234.5,4838.67) node   [align=left] {{\tiny \textit{\textbf{component 7}}}};


\end{tikzpicture}}
      % }
    \end{column}

  \end{columns}

\end{frame}


\section{État de l'art}

\AtBeginSection[]{
  \begin{frame}
    \frametitle{}
    \tableofcontents[currentsection]
  \end{frame}
}


\begin{frame}{État de l'art}{Synthèse des travaux liés}


  \begin{columns}

    \begin{column}{0.5\textwidth}
      \begin{itemize}
        \item \textbf{Autonomie (C1)} : cycle de vie encore très dépendant de l'humain ; autonomie organisationnelle peu étudiée.
        \item \textbf{Performance (C2)} : critère le plus traité (récompense, succès), robustesse limitée.
        \item \textbf{Adaptation (C3)} : quelques approches co-évolutives, peu généralisables.
        \item \textbf{Contrôle (C4)} : quasi-absent (spécification/vérification de contraintes organisationnelles).
        \item \textbf{Explicabilité (C5)} : outils rares pour relier comportements et organisation.
      \end{itemize}
    \end{column}

    \begin{column}{0.6\textwidth}
      % Scale the whole table to the column width (adjust width as needed, e.g. 0.9\linewidth)
      \begin{center}
        \begin{adjustbox}{width=\linewidth}
          \setlength{\tabcolsep}{6pt}
\renewcommand{\arraystretch}{1.8}

\begin{tabular}{
  >{\centering\arraybackslash}p{3.5cm}|   % nouvelle colonne regroupement
  p{7cm}
  >{\centering\arraybackslash}p{3cm}
  >{\centering\arraybackslash}p{1.8cm}
  >{\centering\arraybackslash}p{1.8cm}
  >{\centering\arraybackslash}p{1.8cm}
  >{\centering\arraybackslash}p{1.8cm}
  }
  \rowcolor{black!10}
  \textbf{Tendance Famille IA} & \textbf{Catégorie de travaux}                        & \textbf{C1 Autonomie} & \textbf{C2 Perf.} & \textbf{C3 Adapt.} & \textbf{C4 Contrôle} & \textbf{C5 Explic.} \\
  \hline

  % ------------------------- SYMBOLIQUE -------------------------
  \multirow{3}{*}{\vspace{0.8cm}\textbf{$\sim$ \textit{Symbolique}}}
                               & \textbf{Systèmes experts $\sim$ SMA de Cyberdéfense}
  (CIDS, Ant-defense, MTD...)
                               & \xmark~--~$\sim$                                     & $\sim$                & $\sim$            & \cmark             & \cmark                                     \\

                               & \textbf{Modélisation d’environnement}
  (ADT, Petri Nets, Attack Graphs...)
                               & \xmark                                               & $\sim$                & $\sim$            & \cmark             & \cmark                                     \\

                               & \textbf{Cadres de conception SMA}
  (CSLE, CALDERA, CybORG...)
                               & \cmark                                               & \cmark                & $\sim$~--~\cmark  & \xmark             & \xmark~--~$\sim$                           \\

  \vspace{-2.5cm}

  \multirow{1}{*}{
    \begin{tikzpicture}
      \draw[<->, thick] (0,1)--(0,5);
      \vspace{1cm}
    \end{tikzpicture}
  }



  % ---------------------- CONNEXIONNISTE ------------------------
  \multirow{3}{*}{\vspace{-6.5cm}\textbf{$\sim$ \textit{Connexionniste}}}
  \vspace{-0.35cm}
                               & \textbf{Maintien cohérence simulation/réel}
  (Sim2Real, Domain Rand.)
                               & $\sim$~--~\cmark                                     & n/a                   & \cmark            & $\sim$             & $\sim$                                     \\

                               & \textbf{Contraintes organisationnelles en MARL}
  (Shielding, Shaping, Constrained-RL)
                               & \xmark~--~$\sim$                                     & $\sim$~--~\cmark      & $\sim$            & \cmark             & $\sim$~--~\cmark                           \\

                               & \textbf{Extraction organisationnelle émergente}
  (ROMA, unsupervised ML...)
                               & \xmark~--~$\sim$                                     & $\sim$                & $\sim$            & \xmark~--~$\sim$   & \cmark                                     \\
  \hline
\end{tabular}

        \end{adjustbox}
      \end{center}
    \end{column}

  \end{columns}

  \vspace{0.5cm}

  $\Longrightarrow$ \textbf{Approches : « IA Symbolique » // « IA connexioniste »} \\ \ \\

  \noindent {\tiny \begin{spacing}{0.8}
      \textit{Soulé, J. (2023). De l'organisation d'un système de cyberdéfense multi-agents. Dans les Actes des Rencontres sur la Recherche et l'Enseignement de la Sécurité des Systèmes d'Information (RESSI 2023). France.}
    \end{spacing}}

\end{frame}



\begin{frame}{État de l'art}{Recommendations : décomposition problème \& hypothèses}
  \centering
  \setlength{\tabcolsep}{8pt}
  \renewcommand{\arraystretch}{1.5}
  \setlength{\extrarowheight}{2pt}

  \resizebox{\linewidth}{!}{
    \begin{tabular}{lp{6cm}p{5cm}}
      \hline
      \textbf{Sous-problème} & \textbf{Hypothèse}                                                                                                   & \textbf{Critères C1–C5}       \\
      \hline
      \textbf{MOD}           & \textbf{H-MOD} : Obtention d’un environnement simulé réaliste (manuellement ou via apprentissage).                   & C1 Autonomie, C3 Adaptation   \\

      \textbf{TRN}           & \textbf{H-TRN} : Intégration des contraintes organisationnelles dans le MARL pour des politiques sûres et conformes. & C2 Performance, C4 Contrôle   \\

      \textbf{ANL}           & \textbf{H-ANL} : Inférence automatique des rôles et objectifs organisationnels à partir des trajectoires d’agents.   & C4 Contrôle, C5 Explicabilité \\

      \textbf{TRF}           & \textbf{H-TRF} : Couplage simulation–réel (jumeau numérique) assurant un transfert sûr et adaptatif.                 & C1 Autonomie, C3 Adaptation   \\
    \end{tabular}}
\end{frame}



\section{Méthode}

\AtBeginSection[]{
  \begin{frame}
    \frametitle{}
    \begin{columns}[T]
      \begin{column}{0.62\textwidth}
        \tableofcontents[currentsection]
      \end{column}
      \begin{column}{0.38\textwidth}
        \centering
        \vspace{0.5cm}
        % Remplacez le chemin ci-dessous par votre image
        % \includegraphics[width=\linewidth]{figures/transition_figure.png}
      \end{column}
    \end{columns}
  \end{frame}
}

\subsection{Aperçu général}

\begin{frame}{Méthode}{Aperçu général}

  \begin{columns}[c]

    \hspace{-1.5cm}

    \begin{column}{0.5\textwidth}
      \begin{enumerate}
        \item Modéliser l'environnement à partir de traces réelles ;
        \item Entraînement des agents à l'aide de MARL + Contraintes ;
        \item Analyse pour rôles et objectifs émergents des agents ;
        \item Transferer politiques pour piloter les actionneurs des agents + génère nouvelles traces $\rightarrow$ modèle simulé plus fidèle.
      \end{enumerate}

    \end{column}

    \hspace{-1.5cm}

    \begin{column}{0.5\textwidth}
      \begin{figure}[h!]
        \centering
        \includegraphics[width=1.2\linewidth]{figures/mamad_framework.png}
        \label{fig:cycle}
      \end{figure}
    \end{column}
  \end{columns}

\end{frame}



\subsection{Modélisation}

\begin{frame}{}
  \begin{columns}[T]
    \begin{column}{0.5\textwidth}
      \tableofcontents[currentsubsection]
    \end{column}
    \begin{column}{0.5\textwidth}
      \centering
      \vspace{0.5cm}
      \begin{figure}
        \includegraphics[trim={0cm 7.5cm 0 0},clip, width=\linewidth]{figures/mamad_framework.png}
      \end{figure}
    \end{column}
  \end{columns}
\end{frame}

\begin{frame}{Modélisation en simulation}{Cadre}

  \textbf{Un environnement en réseau}
  \begin{itemize}
    \item Propriétés : ouvert, dynamique/statique, déterministe, accessible/inaccessible\dots
          \phantom{XXXXXXXXXXXXXXXXXXXXXXXXXXXXXXXXXXXXXXXXXXXXXXXXXXXXXXXXXXXXXXXX}
  \end{itemize}

  \includegraphics[width=\linewidth, trim=0cm 0cm 0cm 0.1cm, clip]{figures/0.png}

\end{frame}

\begin{frame}{Modélisation en simulation}{Cadre}

  \textbf{Avec une Green Team}
  \begin{itemize}
    \item Utilisateurs \textquote{normaux} : sessions, requêtes, scan des ports, envoi de données
          \phantom{XXXXXXXXXXXXXXXXXXXXXXXXXXXXXXXXXXXXXXXXXXXXXXXXXXXXXXXXXXXXXXXX}
  \end{itemize}

  \includegraphics[width=\linewidth, trim=0cm 0cm 0cm 0.1cm, clip]{figures/1.png}
\end{frame}

\begin{frame}{Modélisation en simulation}{Cadre}

  \textbf{Avec une Red Team}
  \begin{itemize}
    \item Cyber-attaquants : découverte de nœuds/services, exploitation des vulnérabilités, escalade de privilège, déplacement latéral, impact\dots
  \end{itemize}

  \includegraphics[width=\linewidth, trim=0cm 0cm 0cm 0.1cm, clip]{figures/2.png}
\end{frame}

\begin{frame}{Modélisation en simulation}{Cadre}

  \textbf{Avec une Blue Team}
  \begin{itemize}
    \item Cyber-défenseurs : analyse de menace, contrôle des accès, tuer des processus suspects, mise en place de \textquote{honey pots}, re-imager un nœud\dots
  \end{itemize}

  \includegraphics[width=\linewidth, trim=0cm 0cm 0cm 0.1cm, clip]{figures/3.png}
\end{frame}

\begin{frame}{Modélisation en simulation}{Modèle Markovien}

  \begin{columns}[c]

    \hspace{-1.cm}

    \begin{column}{0.5\textwidth}

      \begin{itemize}
        \item Modèle Markovien (Dec-POMDP) \\ \parencite{Oliehoek2016}
              \begin{itemize}
                \item Prise de décision collective
                \item Modélisation de incertitude des actions / observations
                \item \textbf{T : Fonction transition état}
              \end{itemize}

        \item[\phantom{X}] \phantom{Modélisation manuelle de T}
          \begin{itemize}
            \item[\phantom{X}] \phantom{CybORG}
          \end{itemize}
        \item[\phantom{X}] \phantom{Modélisation automatisée de T}
          \begin{enumerate}
            \item[\phantom{X}] \phantom{Collecte traces réelles}
            \item[\phantom{X}] \phantom{Entrainement via un RNN (LSTM)}
          \end{enumerate}
      \end{itemize}

    \end{column}

    \hspace{-1.5cm}

    \begin{column}{0.6\textwidth}
      \centering
      \includegraphics[width=\linewidth]{figures/marl_framework.png}
    \end{column}
  \end{columns}

  \medskip

  {\tiny \begin{spacing}{0.8}
      \textit{Soulé, J., Jamont, J.-P., Occello, M., Théron, P., \& Traonouez, L.-M. Towards a Multi-Agent Simulation of Cyber-attackers and Cyber-defenders Battles. IEEE SMC 2023.}
    \end{spacing}}

\end{frame}


\begin{frame}{Modélisation en simulation}{Modèle Markovien}

  \begin{columns}[c]

    \hspace{-1.cm}

    \begin{column}{0.5\textwidth}

      \begin{itemize}
        \item Modèle Markovien (Dec-POMDP) \\ \parencite{Oliehoek2016}
              \begin{itemize}
                \item Prise de décision collective
                \item Modélisation de incertitude des actions / observations
                \item \textbf{T : Fonction transition état}
              \end{itemize}

        \item Modélisation manuelle de T
              \begin{itemize}
                \item Simulateurs / Emulateurs : CybORG \allowbreak \parencite{Maxwell2021}
              \end{itemize}
        \item[\phantom{X}] \phantom{Modélisation automatisée de T}
          \begin{enumerate}
            \item[\phantom{X}] \phantom{Collecte traces réelles}
            \item[\phantom{X}] \phantom{Entrainement via un RNN (LSTM)}
          \end{enumerate}
      \end{itemize}

    \end{column}

    \hspace{-1.5cm}

    \begin{column}{0.6\textwidth}
      \centering
      \includegraphics[width=\linewidth]{figures/literature_aco_envs.png}
    \end{column}
  \end{columns}

  \medskip

  {\tiny \begin{spacing}{0.8}
      \textit{Soulé, J., Jamont, J.-P., Occello, M., Théron, P., \& Traonouez, L.-M. Towards a Multi-Agent Simulation of Cyber-attackers and Cyber-defenders Battles. IEEE SMC 2023.}
    \end{spacing}}

\end{frame}


\begin{frame}{Modélisation en simulation}{Modèle Markovien}

  \vspace{1cm}

  \begin{columns}[c]

    \hspace{-1.cm}

    \begin{column}{0.5\textwidth}

      \begin{itemize}
        \item Modèle Markovien (Dec-POMDP) \\ \parencite{Oliehoek2016}
              \begin{itemize}
                \item Prise de décision collective
                \item Modélisation de incertitude des actions / observations
                \item \textbf{T : Fonction transition état}
              \end{itemize}

        \item Modélisation automatisée de T
              \begin{itemize}
                \item Dataset de trajectoires
                \item World Models~\parencite{Ha2018}
                \item Joint-Observation Prediction Model
                      \begin{itemize}
                        \item Prédit prochaine observation conjointe
                      \end{itemize}
              \end{itemize}
      \end{itemize}

    \end{column}

    \hspace{-1cm}

    \begin{column}{0.6\textwidth}
      \raggedright
      \includegraphics[width=1\linewidth]{figures/jopm.png}
    \end{column}
  \end{columns}

  \vspace{1cm}

  {\tiny \begin{spacing}{0.8}
      \textit{Soulé, J., Jamont, J.-P., Occello, M., Théron, P., \& Traonouez, L.-M. Towards a Multi-Agent Simulation of Cyber-attackers and Cyber-defenders Battles. IEEE SMC 2023.}
    \end{spacing}}

\end{frame}



\subsection{Entraînement}

\begin{frame}{}
  \begin{columns}[T]
    \begin{column}{0.5\textwidth}
      \tableofcontents[currentsubsection]
    \end{column}
    \begin{column}{0.5\textwidth}
      \centering
      \vspace{0.5cm}
      \begin{figure}
        \includegraphics[trim={0cm 0cm 14.5cm 0},clip, width=0.5\linewidth]{figures/mamad_framework.png}
      \end{figure}
    \end{column}
  \end{columns}
\end{frame}


\begin{frame}{Entraînement}{MARL \textquote{Vanilla}}

  \begin{itemize}
    \item Choisir les meilleures actions pour maximiser récompense cumulée
  \end{itemize}

  \begin{center}
    \includegraphics[width=0.5\linewidth]{figures/vanilla_marl.png}
  \end{center}

  \vfill

  {\tiny \begin{spacing}{0.8}
      \textit{Soulé, J., Jamont, J.-P., Occello, M., Traonouez, L.-M., \& Théron, P. An Organizationally-Oriented Approach to Enhancing Explainability and Control in Multi-Agent Reinforcement Learning. Proceedings of the 24th International Conference on Autonomous Agents and Multiagent Systems (AAMAS 2025), Detroit, Michigan, USA, 2025.}
    \end{spacing}}

\end{frame}

\begin{frame}{Entraînement}{Apprentissage guidé/contraint}

  \textbf{MARL + Organisation ($\mathcal{M}OISE^+$)}
  \begin{itemize}
    \item Rôle : imposer/refuser actions $\rightarrow$ garantie de sûreté
  \end{itemize}

  \begin{center}
    \includegraphics[width=0.5\linewidth]{figures/role_marl.png}
  \end{center}

  \vfill

  {\tiny \begin{spacing}{0.8}
      \textit{Soulé, J., Jamont, J.-P., Occello, M., Traonouez, L.-M., \& Théron, P. An Organizationally-Oriented Approach to Enhancing Explainability and Control in Multi-Agent Reinforcement Learning. Proceedings of the 24th International Conference on Autonomous Agents and Multiagent Systems (AAMAS 2025), Detroit, Michigan, USA, 2025.}
    \end{spacing}}

\end{frame}

\begin{frame}{Entraînement}{Apprentissage guidé/contraint}

  \textbf{MARL + Organisation ($\mathcal{M}OISE^+$)}
  \begin{itemize}
    \item Objectif : inciter à atteindre un objectif intermédiaire
  \end{itemize}

  \begin{center}
    \includegraphics[width=0.5\linewidth]{figures/goal_marl.png}
  \end{center}

  \vfill

  {\tiny \begin{spacing}{0.8}
      \textit{Soulé, J., Jamont, J.-P., Occello, M., Traonouez, L.-M., \& Théron, P. An Organizationally-Oriented Approach to Enhancing Explainability and Control in Multi-Agent Reinforcement Learning. Proceedings of the 24th International Conference on Autonomous Agents and Multiagent Systems (AAMAS 2025), Detroit, Michigan, USA, 2025.}
    \end{spacing}}

\end{frame}

\begin{frame}{Entraînement}{Le framework MOISE+MARL}

  \begin{columns}[c] % c = vertically center content
    \begin{column}{0.4\textwidth}
      \begin{itemize}
        \item Combiner le Dec-POMDP avec le modèle organisationnel MOISE+.
        \item Les agents se voient attribuer des rôles et des missions sous forme de contraintes.
        \item Utiliser des guides de contraintes pour ajuster :
              \begin{itemize}
                \item \textbf{Actions} via RoleActionGuides (RAG)
                \item \textbf{Récompenses} via RoleRewardGuides et GoalRewardGuides
              \end{itemize}
      \end{itemize}
    \end{column}
    \begin{column}{0.6\textwidth}
      \begin{figure}
        \centering
        \includegraphics[width=1.\linewidth]{figures/mm_simple_representation.png}
      \end{figure}
    \end{column}
  \end{columns}

  \vfill

  {\tiny \begin{spacing}{0.8}
      \textit{Soulé, J., Jamont, J.-P., Occello, M., Traonouez, L.-M., \& Théron, P. An Organizationally-Oriented Approach to Enhancing Explainability and Control in Multi-Agent Reinforcement Learning. Proceedings of the 24th International Conference on Autonomous Agents and Multiagent Systems (AAMAS 2025), Detroit, Michigan, USA, 2025.}
    \end{spacing}}

\end{frame}



\subsection{Analyse}

\begin{frame}{}
  \begin{columns}[T]
    \begin{column}{0.5\textwidth}
      \tableofcontents[currentsubsection]
    \end{column}
    \begin{column}{0.5\textwidth}
      \centering
      \vspace{0.5cm}
      \begin{figure}
        \includegraphics[trim={0cm 0cm 0cm 7.5cm},clip, width=\linewidth]{figures/mamad_framework.png}
      \end{figure}
    \end{column}
  \end{columns}
\end{frame}

\begin{frame}{Analyse}{Aperçu général de la méthode}

  \textbf{Trajectory-based Evaluation in MOISE+MARL (TEMM)}
  \begin{itemize}
    \item \textbf{Objectif} : Fournir une interprétation a posteriori du comportement des agents à un niveau organisationnel.
  \end{itemize}

  \vspace{1em}
  \textbf{Hypothèses sous-jacentes :}
  \begin{itemize}
    \item \textbf{Rôles} $\sim$ motifs fréquents de transitions \emph{(observation, action)} dans les trajectoires des agents.
    \item \textbf{Objectifs} $\sim$ observations reçues fréquemment au sein des trajectoires des agents.
  \end{itemize}

  \vspace{0.8cm}
  \begin{center}
    \begin{columns}[c]

      \begin{column}{0.4\textwidth}
        \centering
        


\tikzset{every picture/.style={line width=0.75pt}} %set default line width to 0.75pt        

\begin{tikzpicture}[x=0.75pt,y=0.75pt,yscale=-1,xscale=1]
    %uncomment if require: \path (0,1974); %set diagram left start at 0, and has height of 1974

    %Shape: Rectangle [id:dp9996076613305621] 
    \draw  [fill={rgb, 255:red, 255; green, 255; blue, 255 }  ,fill opacity=1 ] (24,1558.11) -- (176.1,1558.11) -- (176.1,1644) -- (24,1644) -- cycle ;
    %Straight Lines [id:da05824332013205091] 
    \draw [color={rgb, 255:red, 208; green, 2; blue, 27 }  ,draw opacity=1 ]   (142.67,1570.84) -- (124.28,1577.49) -- (87.26,1592.41) -- (108.68,1604.12) -- (93.53,1601.36) -- (86.58,1603.01) -- (86.58,1612.77) -- (82.05,1616.07) -- (81.22,1616.67) -- (78.65,1614.8) -- (70.51,1608.86) -- (54.44,1608.86) -- (57,1610.73) -- (49.09,1612.77) -- (51.85,1616.79) -- (38.38,1628.38) ;
    \draw [shift={(145.49,1569.82)}, rotate = 160.12] [fill={rgb, 255:red, 208; green, 2; blue, 27 }  ,fill opacity=1 ][line width=0.08]  [draw opacity=0] (3.57,-1.72) -- (0,0) -- (3.57,1.72) -- cycle    ;
    %Straight Lines [id:da9249559779542824] 
    \draw [color={rgb, 255:red, 80; green, 227; blue, 194 }  ,draw opacity=1 ]   (143.47,1568.13) -- (134.78,1577.63) -- (113.36,1577.63) -- (113.36,1585.44) -- (86.58,1593.25) -- (91.93,1597.15) -- (97.29,1604.96) -- (81.22,1601.06) -- (86.58,1608.86) -- (81.22,1608.86) -- (86.58,1616.67) -- (75.87,1616.67) -- (67.94,1608.94) -- (65.16,1614.72) -- (43.73,1603.01) -- (59.8,1616.67) -- (43.73,1608.86) -- (49.09,1616.67) -- (43.73,1632.29) ;
    \draw [shift={(145.49,1565.92)}, rotate = 132.45] [fill={rgb, 255:red, 80; green, 227; blue, 194 }  ,fill opacity=1 ][line width=0.08]  [draw opacity=0] (3.57,-1.72) -- (0,0) -- (3.57,1.72) -- cycle    ;
    %Straight Lines [id:da17118391857757054] 
    \draw [color={rgb, 255:red, 248; green, 231; blue, 28 }  ,draw opacity=1 ]   (153.23,1574.14) -- (126.83,1577.75) -- (124.07,1581.54) -- (105.41,1585.56) -- (91.93,1593.25) -- (93.53,1601.36) -- (89.34,1605.08) -- (81.22,1597.15) -- (91.93,1612.77) -- (91.93,1616.67) -- (81.22,1616.67) -- (59.99,1609.06) -- (57.21,1614.84) -- (41.14,1616.79) -- (35.78,1632.41) ;
    \draw [shift={(156.2,1573.73)}, rotate = 172.2] [fill={rgb, 255:red, 248; green, 231; blue, 28 }  ,fill opacity=1 ][line width=0.08]  [draw opacity=0] (3.57,-1.72) -- (0,0) -- (3.57,1.72) -- cycle    ;
    %Straight Lines [id:da6427777277243145] 
    \draw [color={rgb, 255:red, 144; green, 19; blue, 254 }  ,draw opacity=1 ]   (160.23,1577.39) -- (165.84,1578.41) -- (161.56,1573.73) -- (157.27,1570.61) -- (164.86,1573.73) -- (170.13,1578.41) -- (161.56,1583.1) -- (166.91,1589.34) -- (161.56,1597.15) -- (166.91,1601.06) -- (161.56,1612.77) -- (161.56,1628.38) -- (145.49,1624.48) -- (134.78,1624.48) -- (128.99,1623.07) -- (125.92,1622.33) -- (121.57,1621.27) -- (118.71,1620.58) -- (107.96,1619.71) -- (99.43,1619.01) -- (95.23,1618.25) -- (86.58,1616.67) -- (75.87,1616.67) -- (70.51,1620.58) -- (59.8,1624.48) -- (59.8,1632.29) ;
    \draw [shift={(157.27,1576.85)}, rotate = 10.33] [fill={rgb, 255:red, 144; green, 19; blue, 254 }  ,fill opacity=1 ][line width=0.08]  [draw opacity=0] (3.57,-1.72) -- (0,0) -- (3.57,1.72) -- cycle    ;
    %Straight Lines [id:da7390021320622445] 
    \draw [color={rgb, 255:red, 65; green, 117; blue, 5 }  ,draw opacity=1 ]   (159.15,1578.17) -- (164.77,1579.19) -- (160.49,1574.51) -- (156.2,1571.39) -- (163.79,1574.51) -- (169.06,1579.19) -- (161.56,1587.78) -- (165.84,1590.13) -- (163.7,1601.84) -- (150.85,1597.15) -- (161.56,1603.4) -- (174.41,1615.89) -- (157.27,1606.52) -- (160.49,1613.55) -- (163.7,1625.26) -- (152.99,1628.38) -- (135.85,1622.14) -- (123,1622.14) -- (116.57,1622.14) -- (110.14,1620.58) -- (108,1623.7) -- (103.72,1620.58) -- (105.86,1625.26) -- (94.16,1619.03) -- (85.51,1617.45) -- (74.8,1617.45) -- (69.44,1621.36) -- (58.73,1625.26) -- (58.73,1633.07) ;
    \draw [shift={(156.2,1577.63)}, rotate = 10.33] [fill={rgb, 255:red, 65; green, 117; blue, 5 }  ,fill opacity=1 ][line width=0.08]  [draw opacity=0] (3.57,-1.72) -- (0,0) -- (3.57,1.72) -- cycle    ;
    %Shape: Ellipse [id:dp30050508180239144] 
    \draw  [draw opacity=0][fill={rgb, 255:red, 208; green, 2; blue, 27 }  ,fill opacity=0.62 ] (46.49,1615.89) .. controls (46.49,1614.6) and (47.93,1613.55) .. (49.71,1613.55) .. controls (51.48,1613.55) and (52.92,1614.6) .. (52.92,1615.89) .. controls (52.92,1617.19) and (51.48,1618.23) .. (49.71,1618.23) .. controls (47.93,1618.23) and (46.49,1617.19) .. (46.49,1615.89) -- cycle ;
    %Shape: Ellipse [id:dp15311501498248647] 
    \draw  [draw opacity=0][fill={rgb, 255:red, 208; green, 2; blue, 27 }  ,fill opacity=0.62 ] (90.49,1619.03) .. controls (90.49,1617.74) and (91.93,1616.69) .. (93.71,1616.69) .. controls (95.48,1616.69) and (96.92,1617.74) .. (96.92,1619.03) .. controls (96.92,1620.32) and (95.48,1621.37) .. (93.71,1621.37) .. controls (91.93,1621.37) and (90.49,1620.32) .. (90.49,1619.03) -- cycle ;
    %Shape: Ellipse [id:dp19167487081496637] 
    \draw  [draw opacity=0][fill={rgb, 255:red, 208; green, 2; blue, 27 }  ,fill opacity=0.62 ] (161.11,1606.52) .. controls (161.11,1605.23) and (162.54,1604.18) .. (164.32,1604.18) .. controls (166.09,1604.18) and (167.53,1605.23) .. (167.53,1606.52) .. controls (167.53,1607.82) and (166.09,1608.86) .. (164.32,1608.86) .. controls (162.54,1608.86) and (161.11,1607.82) .. (161.11,1606.52) -- cycle ;
    %Shape: Ellipse [id:dp9201279867822619] 
    \draw  [draw opacity=0][fill={rgb, 255:red, 208; green, 2; blue, 27 }  ,fill opacity=0.62 ] (120.4,1622.92) .. controls (120.4,1621.62) and (121.84,1620.58) .. (123.62,1620.58) .. controls (125.39,1620.58) and (126.83,1621.62) .. (126.83,1622.92) .. controls (126.83,1624.21) and (125.39,1625.26) .. (123.62,1625.26) .. controls (121.84,1625.26) and (120.4,1624.21) .. (120.4,1622.92) -- cycle ;
    %Shape: Ellipse [id:dp3048334813609519] 
    \draw  [draw opacity=0][fill={rgb, 255:red, 208; green, 2; blue, 27 }  ,fill opacity=0.62 ] (161.11,1590.91) .. controls (161.11,1589.61) and (162.54,1588.56) .. (164.32,1588.56) .. controls (166.09,1588.56) and (167.53,1589.61) .. (167.53,1590.91) .. controls (167.53,1592.2) and (166.09,1593.25) .. (164.32,1593.25) .. controls (162.54,1593.25) and (161.11,1592.2) .. (161.11,1590.91) -- cycle ;
    %Shape: Ellipse [id:dp7290465976812913] 
    \draw  [draw opacity=0][fill={rgb, 255:red, 208; green, 2; blue, 27 }  ,fill opacity=0.62 ] (86.13,1603.4) .. controls (86.13,1602.11) and (87.56,1601.06) .. (89.34,1601.06) .. controls (91.11,1601.06) and (92.55,1602.11) .. (92.55,1603.4) .. controls (92.55,1604.69) and (91.11,1605.74) .. (89.34,1605.74) .. controls (87.56,1605.74) and (86.13,1604.69) .. (86.13,1603.4) -- cycle ;
    %Shape: Ellipse [id:dp6154487622646608] 
    \draw  [draw opacity=0][fill={rgb, 255:red, 208; green, 2; blue, 27 }  ,fill opacity=0.62 ] (109.69,1583.1) .. controls (109.69,1581.8) and (111.13,1580.76) .. (112.9,1580.76) .. controls (114.68,1580.76) and (116.12,1581.8) .. (116.12,1583.1) .. controls (116.12,1584.39) and (114.68,1585.44) .. (112.9,1585.44) .. controls (111.13,1585.44) and (109.69,1584.39) .. (109.69,1583.1) -- cycle ;
    %Shape: Ellipse [id:dp6108483574180856] 
    \draw  [draw opacity=0][fill={rgb, 255:red, 189; green, 16; blue, 224 }  ,fill opacity=0.8 ] (77.56,1615.89) .. controls (77.56,1614.6) and (79,1613.55) .. (80.77,1613.55) .. controls (82.55,1613.55) and (83.98,1614.6) .. (83.98,1615.89) .. controls (83.98,1617.19) and (82.55,1618.23) .. (80.77,1618.23) .. controls (79,1618.23) and (77.56,1617.19) .. (77.56,1615.89) -- cycle ;
    %Shape: Ellipse [id:dp08863924891219843] 
    \draw  [draw opacity=0][fill={rgb, 255:red, 208; green, 2; blue, 27 }  ,fill opacity=0.62 ] (84.52,1609.06) .. controls (84.52,1607.77) and (85.96,1606.72) .. (87.73,1606.72) .. controls (89.51,1606.72) and (90.95,1607.77) .. (90.95,1609.06) .. controls (90.95,1610.35) and (89.51,1611.4) .. (87.73,1611.4) .. controls (85.96,1611.4) and (84.52,1610.35) .. (84.52,1609.06) -- cycle ;
    %Shape: Ellipse [id:dp49807154634681794] 
    \draw  [draw opacity=0][fill={rgb, 255:red, 208; green, 2; blue, 27 }  ,fill opacity=0.62 ] (91.21,1601.64) .. controls (91.21,1600.35) and (92.65,1599.3) .. (94.43,1599.3) .. controls (96.2,1599.3) and (97.64,1600.35) .. (97.64,1601.64) .. controls (97.64,1602.94) and (96.2,1603.98) .. (94.43,1603.98) .. controls (92.65,1603.98) and (91.21,1602.94) .. (91.21,1601.64) -- cycle ;
    %Shape: Ellipse [id:dp17062416794692736] 
    \draw  [draw opacity=0][fill={rgb, 255:red, 208; green, 2; blue, 27 }  ,fill opacity=0.62 ] (100.59,1619.41) .. controls (100.59,1618.11) and (102.03,1617.06) .. (103.8,1617.06) .. controls (105.57,1617.06) and (107.01,1618.11) .. (107.01,1619.41) .. controls (107.01,1620.7) and (105.57,1621.75) .. (103.8,1621.75) .. controls (102.03,1621.75) and (100.59,1620.7) .. (100.59,1619.41) -- cycle ;
    %Shape: Ellipse [id:dp05427293190477478] 
    \draw  [draw opacity=0][fill={rgb, 255:red, 189; green, 16; blue, 224 }  ,fill opacity=0.8 ] (152.54,1626.82) .. controls (152.54,1625.53) and (153.98,1624.48) .. (155.75,1624.48) .. controls (157.52,1624.48) and (158.96,1625.53) .. (158.96,1626.82) .. controls (158.96,1628.12) and (157.52,1629.16) .. (155.75,1629.16) .. controls (153.98,1629.16) and (152.54,1628.12) .. (152.54,1626.82) -- cycle ;
    %Shape: Ellipse [id:dp0565658994925915] 
    \draw  [draw opacity=0][fill={rgb, 255:red, 208; green, 2; blue, 27 }  ,fill opacity=0.62 ] (158.96,1616.67) .. controls (158.96,1615.38) and (160.4,1614.33) .. (162.18,1614.33) .. controls (163.95,1614.33) and (165.39,1615.38) .. (165.39,1616.67) .. controls (165.39,1617.97) and (163.95,1619.01) .. (162.18,1619.01) .. controls (160.4,1619.01) and (158.96,1617.97) .. (158.96,1616.67) -- cycle ;
    %Shape: Ellipse [id:dp5007110255270828] 
    \draw  [draw opacity=0][fill={rgb, 255:red, 189; green, 16; blue, 224 }  ,fill opacity=0.8 ] (57,1610.73) .. controls (57,1609.43) and (58.44,1608.38) .. (60.21,1608.38) .. controls (61.99,1608.38) and (63.43,1609.43) .. (63.43,1610.73) .. controls (63.43,1612.02) and (61.99,1613.07) .. (60.21,1613.07) .. controls (58.44,1613.07) and (57,1612.02) .. (57,1610.73) -- cycle ;
    %Shape: Ellipse [id:dp22598728144573377] 
    \draw  [draw opacity=0][fill={rgb, 255:red, 208; green, 2; blue, 27 }  ,fill opacity=0.62 ] (88.72,1595.59) .. controls (88.72,1594.3) and (90.16,1593.25) .. (91.93,1593.25) .. controls (93.71,1593.25) and (95.15,1594.3) .. (95.15,1595.59) .. controls (95.15,1596.88) and (93.71,1597.93) .. (91.93,1597.93) .. controls (90.16,1597.93) and (88.72,1596.88) .. (88.72,1595.59) -- cycle ;
    %Shape: Ellipse [id:dp14749486568088088] 
    \draw  [draw opacity=0][fill={rgb, 255:red, 189; green, 16; blue, 224 }  ,fill opacity=0.8 ] (93.54,1589.34) .. controls (93.54,1588.05) and (94.98,1587) .. (96.75,1587) .. controls (98.53,1587) and (99.97,1588.05) .. (99.97,1589.34) .. controls (99.97,1590.64) and (98.53,1591.69) .. (96.75,1591.69) .. controls (94.98,1591.69) and (93.54,1590.64) .. (93.54,1589.34) -- cycle ;
    %Shape: Polygon Curved [id:ds6643267525526769] 
    \draw  [color={rgb, 255:red, 74; green, 144; blue, 226 }  ,draw opacity=1 ][fill={rgb, 255:red, 74; green, 144; blue, 226 }  ,fill opacity=0.5 ] (27.21,1628.38) .. controls (30.38,1623.39) and (36.63,1621.99) .. (42.56,1622.18) .. controls (47.05,1622.33) and (51.36,1623.39) .. (53.99,1624.48) .. controls (60.1,1627.02) and (65.56,1626.63) .. (64.7,1632.29) .. controls (63.85,1637.95) and (56.88,1637.17) .. (48.64,1636.19) .. controls (40.39,1635.22) and (21.64,1637.17) .. (27.21,1628.38) -- cycle ;
    %Shape: Polygon Curved [id:ds9461514343962948] 
    \draw  [color={rgb, 255:red, 208; green, 2; blue, 27 }  ,draw opacity=1 ][fill={rgb, 255:red, 208; green, 2; blue, 27 }  ,fill opacity=0.5 ] (139.68,1569.82) .. controls (145.25,1561.04) and (148.01,1561.59) .. (145.04,1565.92) .. controls (142.07,1570.25) and (154.68,1563.87) .. (153.82,1569.53) .. controls (152.97,1575.19) and (169.35,1578.61) .. (161.11,1577.63) .. controls (152.86,1576.66) and (134.11,1578.61) .. (139.68,1569.82) -- cycle ;
    %Shape: Ellipse [id:dp06406072166611776] 
    \draw  [draw opacity=0][fill={rgb, 255:red, 208; green, 2; blue, 27 }  ,fill opacity=0.62 ] (71.13,1617.45) .. controls (71.13,1616.16) and (72.57,1615.11) .. (74.34,1615.11) .. controls (76.12,1615.11) and (77.56,1616.16) .. (77.56,1617.45) .. controls (77.56,1618.75) and (76.12,1619.8) .. (74.34,1619.8) .. controls (72.57,1619.8) and (71.13,1618.75) .. (71.13,1617.45) -- cycle ;
    %Shape: Ellipse [id:dp049221150011381054] 
    \draw  [draw opacity=0][fill={rgb, 255:red, 189; green, 16; blue, 224 }  ,fill opacity=0.8 ] (56.59,1624.48) .. controls (56.59,1623.19) and (58.03,1622.14) .. (59.8,1622.14) .. controls (61.58,1622.14) and (63.01,1623.19) .. (63.01,1624.48) .. controls (63.01,1625.77) and (61.58,1626.82) .. (59.8,1626.82) .. controls (58.03,1626.82) and (56.59,1625.77) .. (56.59,1624.48) -- cycle ;


    % Text Node
    \draw (68.95,1583.75) node  [font=\tiny,color={rgb, 255:red, 189; green, 16; blue, 224 }  ,opacity=1 ] [align=left] {$\displaystyle g_{5} =\{\omega _{21} \dotsc \}$};
    % Text Node
    \draw (82.63,1627.35) node  [font=\tiny,color={rgb, 255:red, 189; green, 16; blue, 224 }  ,opacity=1 ] [align=left] {$\displaystyle g_{2} =\{\omega _{5}\}$};
    % Text Node
    \draw (152.91,1586.86) node  [font=\tiny,color={rgb, 255:red, 189; green, 16; blue, 224 }  ,opacity=1 ] [align=left] {$\displaystyle ...$};
    % Text Node
    \draw (101.5,1575.93) node  [font=\tiny,color={rgb, 255:red, 189; green, 16; blue, 224 }  ,opacity=1 ] [align=left] {$\displaystyle ...$};
    % Text Node
    \draw (136.45,1636.35) node  [font=\tiny,color={rgb, 255:red, 189; green, 16; blue, 224 }  ,opacity=1 ] [align=left] {$\displaystyle g_{4} =\{\omega _{301} ,\omega _{302}\}$};
    % Text Node
    \draw (113.58,1612.35) node  [font=\tiny,color={rgb, 255:red, 189; green, 16; blue, 224 }  ,opacity=1 ] [align=left] {$\displaystyle g_{3} =\{\omega _{10}\}$};
    % Text Node
    \draw (55.11,1600.35) node  [font=\tiny,color={rgb, 255:red, 189; green, 16; blue, 224 }  ,opacity=1 ] [align=left] {$\displaystyle g_{1} =\{\omega _{1}\}$};
    % Text Node
    \draw (105.58,1568.35) node  [font=\tiny,color={rgb, 255:red, 202; green, 52; blue, 69 }  ,opacity=1 ] [align=left] {$\displaystyle g_{*} =\Omega _{goal}$};
    % Text Node
    \draw (73.43,1637.35) node  [font=\tiny,color={rgb, 255:red, 74; green, 144; blue, 226 }  ,opacity=1 ] [align=left] {$\displaystyle \Omega _{init}$};
    % Text Node
    \draw (32.91,1567.84) node  [font=\scriptsize] [align=left] {$\displaystyle \Omega $};
    % Text Node
    \draw (93.61,1653) node   [align=left] {{\tiny \textit{Une visualisation abstraite des}}};
    \draw (93.61,1662) node   [align=left] {{\tiny \textit{observations dans les trajectoires}}};

\end{tikzpicture}
      \end{column}

      \begin{column}{0.1\textwidth}
      \end{column}

      \begin{column}{0.4\textwidth}
        \centering
        


\tikzset{every picture/.style={line width=0.75pt}} %set default line width to 0.75pt        

\begin{tikzpicture}[x=0.75pt,y=0.75pt,yscale=-1,xscale=1]
%uncomment if require: \path (0,1974); %set diagram left start at 0, and has height of 1974

%Shape: Rectangle [id:dp5335676631264512] 
\draw  [fill={rgb, 255:red, 255; green, 255; blue, 255 }  ,fill opacity=1 ] (190,1560.11) -- (342.1,1560.11) -- (342.1,1646) -- (190,1646) -- cycle ;
%Straight Lines [id:da6623576988919416] 
\draw [color={rgb, 255:red, 208; green, 2; blue, 27 }  ,draw opacity=1 ]   (308.67,1572.84) -- (290.28,1579.49) -- (253.26,1594.41) -- (292.1,1604) -- (274.1,1612) -- (262.1,1616) -- (252.58,1614.77) -- (246.1,1604) -- (240.1,1604) -- (236.1,1606) -- (236.51,1610.86) -- (220.44,1610.86) -- (223,1612.73) -- (215.09,1614.77) -- (217.85,1618.79) -- (204.38,1630.38) ;
\draw [shift={(311.49,1571.82)}, rotate = 160.12] [fill={rgb, 255:red, 208; green, 2; blue, 27 }  ,fill opacity=1 ][line width=0.08]  [draw opacity=0] (3.57,-1.72) -- (0,0) -- (3.57,1.72) -- cycle    ;
%Straight Lines [id:da5424854363807742] 
\draw [color={rgb, 255:red, 80; green, 227; blue, 194 }  ,draw opacity=1 ]   (309.47,1570.13) -- (300.78,1579.63) -- (279.36,1579.63) -- (279.36,1587.44) -- (252.58,1595.25) -- (250.1,1598) -- (248.1,1600) -- (247.22,1603.06) -- (252.58,1610.86) -- (247.22,1610.86) -- (242.1,1608) -- (242.1,1610) -- (233.94,1610.94) -- (231.16,1616.72) -- (209.73,1605.01) -- (225.8,1618.67) -- (209.73,1610.86) -- (215.09,1618.67) -- (209.73,1634.29) ;
\draw [shift={(311.49,1567.92)}, rotate = 132.45] [fill={rgb, 255:red, 80; green, 227; blue, 194 }  ,fill opacity=1 ][line width=0.08]  [draw opacity=0] (3.57,-1.72) -- (0,0) -- (3.57,1.72) -- cycle    ;
%Straight Lines [id:da21186841526109945] 
\draw [color={rgb, 255:red, 248; green, 231; blue, 28 }  ,draw opacity=1 ]   (319.23,1576.14) -- (292.83,1579.75) -- (290.07,1583.54) -- (271.41,1587.56) -- (257.93,1595.25) -- (280.1,1592) -- (284.1,1594) -- (290.1,1604) -- (257.93,1614.77) -- (257.93,1618.67) -- (247.22,1618.67) -- (225.99,1611.06) -- (223.21,1616.84) -- (207.14,1618.79) -- (201.78,1634.41) ;
\draw [shift={(322.2,1575.73)}, rotate = 172.2] [fill={rgb, 255:red, 248; green, 231; blue, 28 }  ,fill opacity=1 ][line width=0.08]  [draw opacity=0] (3.57,-1.72) -- (0,0) -- (3.57,1.72) -- cycle    ;
%Straight Lines [id:da6313290732282947] 
\draw [color={rgb, 255:red, 144; green, 19; blue, 254 }  ,draw opacity=1 ]   (326.23,1579.39) -- (331.84,1580.41) -- (327.56,1575.73) -- (323.27,1572.61) -- (330.86,1575.73) -- (336.13,1580.41) -- (327.56,1585.1) -- (332.91,1591.34) -- (324.1,1600) -- (330.1,1606) -- (312.1,1636) -- (320.1,1638) -- (306.1,1644) -- (306.1,1606) -- (300.1,1614) -- (311.49,1626.48) -- (300.78,1626.48) -- (294.99,1625.07) -- (291.92,1624.33) -- (287.57,1623.27) -- (284.71,1622.58) -- (273.96,1621.71) -- (265.43,1621.01) -- (261.23,1620.25) -- (250.1,1622) -- (241.87,1618.67) -- (236.51,1622.58) -- (225.8,1626.48) -- (225.8,1634.29) ;
\draw [shift={(323.27,1578.85)}, rotate = 10.33] [fill={rgb, 255:red, 144; green, 19; blue, 254 }  ,fill opacity=1 ][line width=0.08]  [draw opacity=0] (3.57,-1.72) -- (0,0) -- (3.57,1.72) -- cycle    ;
%Straight Lines [id:da1305524961942589] 
\draw [color={rgb, 255:red, 65; green, 117; blue, 5 }  ,draw opacity=1 ]   (325.15,1580.17) -- (330.77,1581.19) -- (326.49,1576.51) -- (322.2,1573.39) -- (329.79,1576.51) -- (335.06,1581.19) -- (327.56,1589.78) -- (331.84,1592.13) -- (329.7,1603.84) -- (316.85,1599.15) -- (327.56,1605.4) -- (314.1,1636) -- (318.1,1640) -- (310.1,1642) -- (304.1,1608) -- (298.1,1612) -- (301.85,1624.14) -- (289,1624.14) -- (282.57,1624.14) -- (276.14,1622.58) -- (274,1625.7) -- (269.72,1622.58) -- (271.86,1627.26) -- (260.16,1621.03) -- (251.51,1619.45) -- (240.8,1619.45) -- (235.44,1623.36) -- (224.73,1627.26) -- (224.73,1635.07) ;
\draw [shift={(322.2,1579.63)}, rotate = 10.33] [fill={rgb, 255:red, 65; green, 117; blue, 5 }  ,fill opacity=1 ][line width=0.08]  [draw opacity=0] (3.57,-1.72) -- (0,0) -- (3.57,1.72) -- cycle    ;
%Shape: Polygon Curved [id:ds29559681347985167] 
\draw  [color={rgb, 255:red, 184; green, 233; blue, 134 }  ,draw opacity=0 ][fill={rgb, 255:red, 74; green, 144; blue, 226 }  ,fill opacity=0.75 ] (203.31,1627.26) .. controls (208.88,1618.48) and (203.46,1612.19) .. (210,1608) .. controls (216.54,1603.81) and (229.5,1600.53) .. (248,1604) .. controls (266.5,1607.47) and (269.83,1605.44) .. (280,1604) .. controls (290.17,1602.56) and (250.44,1601.87) .. (252,1594) .. controls (253.56,1586.13) and (258.5,1591.77) .. (262,1588) .. controls (265.5,1584.23) and (314.2,1570.33) .. (316,1570) .. controls (317.8,1569.67) and (320.91,1572.43) .. (314,1576) .. controls (307.09,1579.57) and (294.2,1581.95) .. (288,1584) .. controls (281.8,1586.05) and (277.13,1589.32) .. (276,1590) .. controls (274.87,1590.68) and (280.1,1589.85) .. (284,1592) .. controls (287.9,1594.15) and (295.82,1601.53) .. (296,1602) .. controls (296.18,1602.47) and (264.78,1615.69) .. (264,1616) .. controls (263.22,1616.31) and (249.54,1612.38) .. (244,1612) .. controls (238.46,1611.62) and (217.32,1621.29) .. (214,1626) .. controls (210.68,1630.71) and (210.87,1632.35) .. (210,1636) .. controls (209.13,1639.65) and (197.74,1636.04) .. (203.31,1627.26) -- cycle ;
%Shape: Polygon Curved [id:ds2928272635642186] 
\draw  [color={rgb, 255:red, 208; green, 2; blue, 27 }  ,draw opacity=0 ][fill={rgb, 255:red, 208; green, 2; blue, 27 }  ,fill opacity=0.5 ] (304,1644) .. controls (302.69,1641.5) and (306.85,1644.5) .. (304,1634) .. controls (301.15,1623.5) and (270.08,1628.81) .. (266,1628) .. controls (261.92,1627.19) and (254.9,1623.42) .. (244,1624) .. controls (233.1,1624.58) and (230.1,1635.78) .. (226,1638) .. controls (221.9,1640.22) and (223.11,1630.38) .. (222,1630) .. controls (220.89,1629.62) and (222.67,1626.54) .. (226,1624) .. controls (229.33,1621.46) and (236.24,1618.61) .. (236,1618) .. controls (235.76,1617.39) and (243.31,1617.27) .. (250,1618) .. controls (256.69,1618.73) and (262.53,1620.62) .. (264,1620) .. controls (265.47,1619.38) and (296.11,1616.11) .. (296,1614) .. controls (295.89,1611.89) and (301.99,1602.86) .. (304,1602) .. controls (306.01,1601.14) and (317.99,1622.88) .. (320,1616) .. controls (322.01,1609.12) and (321.73,1568.73) .. (326,1570) .. controls (330.27,1571.27) and (339.87,1563.9) .. (338,1584) .. controls (336.13,1604.1) and (324.06,1639.29) .. (320,1642) .. controls (315.94,1644.71) and (305.31,1646.5) .. (304,1644) -- cycle ;


% Text Node
\draw (268.2,1637.75) node  [font=\tiny,color={rgb, 255:red, 189; green, 16; blue, 224 }  ,opacity=1 ] [align=left] {$\displaystyle \rho _{2} =\{( \omega _{11} ,a_{11}) \dotsc \}$};
% Text Node
\draw (231.2,1581.75) node  [font=\tiny,color={rgb, 255:red, 189; green, 16; blue, 224 }  ,opacity=1 ] [align=left] {$\displaystyle \rho _{1} =\{( \omega _{21} ,a_{21}) \dotsc \}$};
% Text Node
\draw (209.5,1570) node  [font=\scriptsize] [align=left] {$\displaystyle \Omega \times A$};
% Text Node
\draw (267.61,1655) node   [align=left] {{\tiny \textit{An abstract visualization of}}};
\draw (267.61,1665) node   [align=left] {{\tiny \textit{transitions in trajectories}}};

\end{tikzpicture}
      \end{column}
    \end{columns}
  \end{center}

  \begin{tikzpicture}[remember picture, overlay]
    \node[anchor=north west, text=black]
    at ([xshift=5.8cm,yshift=-5cm]current page.north west) {\small
      \begin{minipage}{0.3\linewidth}
        {\small \hspace{1.3cm} \textit{\textbf{Ideas\dots}}
          \begin{itemize}
            \item  \textit{Trajectories comme vecteurs;}
            \item  \textit{Distance: Smith-Waterman, LCS, Euclidean\dots;}
            \item  \textit{Clustering + Centroides \\ \ \ \ $\rightarrow$ roles/objectifs}
          \end{itemize}}
      \end{minipage}

    };
  \end{tikzpicture}

  \vspace{-0.5cm}

  {\tiny \begin{spacing}{0.8}
      \textit{Soulé, J., Jamont, J.-P., Occello, M., Traonouez, L.-M., \& Théron, P. An Organizationally-Oriented Approach to Enhancing Explainability and Control in Multi-Agent Reinforcement Learning. Proceedings of the 24th International Conference on Autonomous Agents and Multiagent Systems (AAMAS 2025), Detroit, Michigan, USA, 2025.}
    \end{spacing}}

\end{frame}


\subsection{Transfert}

\begin{frame}{}
  \begin{columns}[T]
    \begin{column}{0.5\textwidth}
      \tableofcontents[currentsubsection]
    \end{column}
    \begin{column}{0.5\textwidth}
      \centering
      \vspace{0.5cm}
      \begin{figure}
        \includegraphics[trim={14.5cm 0cm 0cm 0},clip, width=0.5\linewidth]{figures/mamad_framework.png}
      \end{figure}
    \end{column}
  \end{columns}
\end{frame}

\begin{frame}{Transfert / Déploiement}

  \begin{columns}
    \begin{column}{0.4\textwidth}
      \begin{itemize}
        \item Déploiement \textit{remote} ou \textit{direct} ; collecte de nouvelles transitions.
        \item Boucle fermée : amélioration du modèle simulé par itération.
      \end{itemize}
    \end{column}

    \begin{column}{0.6\textwidth}
      \begin{figure}
        \centering
        \includegraphics[width=.9\linewidth]{figures/transferring_illustration.pdf}
      \end{figure}
    \end{column}
  \end{columns}

\end{frame}

\section{Implémentation : CybMASDE}

\begin{frame}{Cyber(defense) Multi-Agent System Development Environment}

  \begin{columns}

    \begin{column}{0.5\textwidth}
      \begin{itemize}
        \item Application Web (Electron) :
              \begin{itemize}
                \item \textbf{backend} Python
                \item \textbf{frontend} Angular
              \end{itemize}
        \item Projet :
              \begin{itemize}
                \item \textbf{structure} : un dossier par activité (modèles entraînés, bases, configs).
                \item \textbf{fichier JSON central} :
                      \begin{itemize}
                        \item Modélisation : HP JOPM, Autoencodeur…
                        \item Entrainement : HP MARL…
                        \item Analyse : HP Clustering…
                        \item Transfert : Fréquence collecte transition…
                      \end{itemize}
              \end{itemize}

        \item Utilisation :
              \begin{itemize}
                \item \textbf{CLI}
                      \begin{itemize}
                        \item \texttt{init}, \texttt{validate}, \texttt{model}, \texttt{train}, \texttt{analyze}, \texttt{refine}, \texttt{deploy}
                      \end{itemize}
                \item \textbf{GUI}
              \end{itemize}
      \end{itemize}
    \end{column}

    \begin{column}{0.5\textwidth}
      \includegraphics[width=\textwidth]{figures/seq_diag_1.png}
    \end{column}

  \end{columns}

\end{frame}


\begin{frame}{Cyber(defense) Multi-Agent System Development Environment}

  \begin{columns}

    \begin{column}{0.5\textwidth}
      \begin{itemize}
        \item Application Web (Electron) :
              \begin{itemize}
                \item \textbf{backend} Python
                \item \textbf{frontend} Angular
              \end{itemize}
        \item Projet :
              \begin{itemize}
                \item \textbf{structure} : un dossier par activité (modèles entraînés, bases, configs).
                \item \textbf{fichier JSON central} :
                      \begin{itemize}
                        \item Modélisation : HP JOPM, Autoencodeur…
                        \item Entrainement : HP MARL…
                        \item Analyse : HP Clustering…
                        \item Transfert : Fréquence collecte transition…
                      \end{itemize}
              \end{itemize}

        \item Utilisation :
              \begin{itemize}
                \item \textbf{CLI}
                      \begin{itemize}
                        \item \texttt{init}, \texttt{validate}, \texttt{model}, \texttt{train}, \texttt{analyze}, \texttt{refine}, \texttt{deploy}
                      \end{itemize}
                \item \textbf{GUI}
              \end{itemize}
      \end{itemize}
    \end{column}

    \begin{column}{0.5\textwidth}
      \includegraphics[width=\textwidth]{figures/seq_diag_2.png}
    \end{column}

  \end{columns}

\end{frame}


\begin{frame}{Cyber(defense) Multi-Agent System Development Environment}

  \vfill
  \begin{center}
    {\Huge \textbf{Démonstration}}
  \end{center}
  \vfill

  \href{https://www.youtube.com/watch?v=b3wqFpfXZi0}{MOISE+MARL Presentation -- YouTube}

  \href{https://www.youtube.com/watch?v=oKQ_a4C0Wxw}{Simple CLI use of CybMASDE -- YouTube}

  \href{https://www.youtube.com/watch?v=WKj8pFgOKEU}{Simple GUI use of CybMASDE -- YouTube}

\end{frame}



\section{Études de cas}



\begin{frame}{Etude de cas}{Environnements non-cyberdéfense}

  \vspace{-0cm}

  \begin{columns}[c]

    \hspace{-1cm}

    \begin{column}{0.5\textwidth}

      {\scriptsize
        \textbf{Application de la méthode :}
        \begin{itemize}

          \item \textbf{Modélisation}
                \begin{itemize}
                  \item Jeu comme « environnement réel »
                \end{itemize}

          \item \textbf{Entrainement}
                \begin{itemize}
                  \item \textbf{Predator-prey}~\parencite{Lowe2017}
                  \item \textbf{Warehouse Management}~\parencite{warehouse_management}
                  \item \textbf{Overcooked-AI}~\parencite{Carroll2019}
                \end{itemize}

          \item \textbf{Analyse}
                \begin{itemize}
                  \item TEMM avec distance euclidienne
                \end{itemize}

          \item \textbf{Transfert}
                \begin{itemize}
                  \item Remote
                \end{itemize}

        \end{itemize}
      }

    \end{column}

    \hspace{-1.5cm}

    \begin{column}{0.5\textwidth}
      \begin{tabular}{@{}c@{\hspace{1cm}}c@{}}
        \makebox[.48\textwidth][c]{\animategraphics[loop,autoplay,scale=0.15]{8}{figures/wm/frame}{0}{33}} &
        \vspace{0.1cm} \makebox[.48\textwidth][c]{\animategraphics[loop,autoplay,scale=0.18]{8}{figures/overcooked_asymmetric_advantage/frame}{0}{66}} \\
        \small{Warehouse Management}                                                                       & \vspace{0.1cm} \small{Overcooked-AI}
      \end{tabular}

      \hspace{3cm}
      \makebox[.48\textwidth][c]{\animategraphics[loop,autoplay,scale=0.135]{8}{figures/mpe/frame}{0}{25}} \\
      \vspace{0.1cm} \hspace{4cm} \small{Predator-prey}

    \end{column}

  \end{columns}

\end{frame}

\begin{frame}{Etude de cas}{Environnements non-cyberdéfense}

  \vspace{-0.5cm}

  \begin{columns}[c]

    \hspace{-0cm}

    \begin{column}{0.5\textwidth}
      {\scriptsize
        \textbf{Principaux résultats :}
        \begin{itemize}

          \item Modélisation
                \begin{itemize}
                  \item Validé quand dataset de trajectoires « assez important » (>5 Go)
                \end{itemize}

          \item Entrainement
                \begin{itemize}
                  \item Validé vis-à-vis du respect des contraintes dures et douces
                \end{itemize}

          \item Analyse
                \begin{itemize}
                  \item Validé rôles et objectifs inférées correspondent aux attentes
                \end{itemize}

          \item Transfert
                \begin{itemize}
                  \item Validé prise en compte des modifications environnementales en temps réel
                \end{itemize}

        \end{itemize}
      }
    \end{column}

    \begin{column}{0.5\textwidth}
      \centering
      \ \\ \ \\
      \includegraphics[width=0.65\linewidth]{figures/pca_overcooked.png}
      \includegraphics[width=0.87\linewidth]{figures/non_cyber_envs.png}
      \vspace{-1cm}
    \end{column}

  \end{columns}

  \vspace{-0.25cm}
  \includegraphics[width=0.5\linewidth]{figures/non_cyber_results.png}

\end{frame}


\begin{frame}{Etude de cas}{Infrastructure d'entreprise}

  \vspace{-0cm}

  \begin{columns}[c]

    \hspace{-1cm}

    \begin{column}{0.5\textwidth}

      {\scriptsize
        \textbf{Application de la méthode :}
        \begin{itemize}
          \item Trois modèles organisationnels :
                \begin{itemize}
                  \item Attaquant / défenseur aléatoire
                  \item Attaquant / défenseur manuel
                  \item Apprendre attaquant / défenseur
                \end{itemize}
        \end{itemize}
      }

      \vspace{1cm}

      \includegraphics[width=0.8\linewidth]{figures/network_infra.png}

    \end{column}

    \hspace{-2cm}

    \begin{column}{0.5\textwidth}
      \hspace{1cm}
      \includegraphics[width=0.8\linewidth]{figures/attack_defense_tree.jpg}
    \end{column}

  \end{columns}


\end{frame}

% \begin{frame}{Etude de cas}{Infrastructure d'entreprise}

%   \vspace{-0.5cm}

%   \begin{columns}[c]

%     \hspace{-0cm}

%     \begin{column}{0.5\textwidth}
%       {\scriptsize
%         \textbf{Principaux résultats :}
%         \begin{itemize}

%           \item Modélisation
%                 \begin{itemize}
%                   \item Validé quand dataset de trajectoires « assez important » (>5 Go)
%                 \end{itemize}

%           \item Entrainement
%                 \begin{itemize}
%                   \item Validé vis-à-vis du respect des contraintes dures et douces (taux de violation 0 pour dureté de contraintes forte)
%                 \end{itemize}

%           \item Analyse
%                 \begin{itemize}
%                   \item Validé rôles et objectifs inférées correspondent aux attentes (vérification manuelle des trajectoires produites)
%                 \end{itemize}

%           \item Transfert
%                 \begin{itemize}
%                   \item Validé prise en compte des modifications environnementales en temps réel (ajout/retrait d'un nœud dans le réseau)
%                 \end{itemize}

%         \end{itemize}
%       }
%     \end{column}

%     \begin{column}{0.5\textwidth}
%       \centering
%       \includegraphics[width=1\linewidth]{figures/company_infra_learning_curve.png}
%       \includegraphics[width=0.9\linewidth]{figures/infra_results.png}
%       \vspace{-1cm}
%     \end{column}

%   \end{columns}

% \end{frame}


\begin{frame}{Etude de cas}{Scénario essaim de drones}

  \begin{columns}
    \begin{column}{0.5\textwidth}

      \textbf{Trois modèles d'organisation :}
      \begin{itemize}
        \item \textquote{Suspect Isolation}
        \item \textquote{Active Defense}
        \item \textquote{Manual}
      \end{itemize}

      \hspace{1.5cm}
      \makebox[0.4\textwidth][c]{\animategraphics[loop,autoplay,scale=0.19]{8}{figures/cyborg/frame}{0}{33}}

    \end{column}


    \begin{column}{0.5\textwidth}

      \centering

      \includegraphics[width=0.55\linewidth]{figures/markov_mdp_drones.png}

      \vspace{0.1cm}

      \includegraphics[width=0.85\linewidth]{figures/drones_illustration.png}

    \end{column}
  \end{columns}




\end{frame}

% \begin{frame}{Etude de cas}{Scénario essaim de drones}

%   \vspace{-0.5cm}

%   \begin{columns}[c]

%     \hspace{-0cm}

%     \begin{column}{0.5\textwidth}
%       {\scriptsize
%         \textbf{Principaux résultats :}
%         \begin{itemize}

%           \item Modélisation
%                 \begin{itemize}
%                   \item Validé quand dataset de trajectoires « assez important » (>5 Go)
%                 \end{itemize}

%           \item Entrainement
%                 \begin{itemize}
%                   \item Validé comme plus stable et plus grande convergence
%                 \end{itemize}

%           \item Analyse
%                 \begin{itemize}
%                   \item Validé rôles et objectifs inférées correspondent aux attentes (vérification manuelle des trajectoires produites)
%                 \end{itemize}

%           \item Transfert
%                 \begin{itemize}
%                   \item Validé prise en compte des modifications environnementales en temps réel (forcer un drone à être infecté pour déstabiliser)
%                 \end{itemize}

%         \end{itemize}
%       }
%     \end{column}

%     \begin{column}{0.5\textwidth}
%       \ \\ \ \\
%       \centering
%       \includegraphics[width=0.87\linewidth]{figures/drone_swarm_results.png}
%       \ \\ \ \medskip
%       \includegraphics[width=0.87\linewidth]{figures/cage_own_results.png}
%       \ \\ \ \medskip
%       \includegraphics[width=0.5\linewidth]{figures/cage_leader_results.png}
%     \end{column}

%   \end{columns}

% \end{frame}


\begin{frame}{Étude de cas}{Architecture des microservices Kubernetes}

  \centering
  \includegraphics[width=0.7\linewidth]{figures/scenario_introduction.pdf}

  \vfill

  {\tiny \begin{spacing}{0.8}
      \textit{J. Soule, J.-P. Jamont, M. Occello, L.-M. Traonouez, and P. Théron. Streamlining Resilient Kubernetes Autoscaling with Multi-Agent Systems via an Automated Online Design Framework. Proceedings of the 18th IEEE International Conference on Cloud Computing (CLOUD), Helsinki, Finland, July 2025. (Accepted).}
    \end{spacing}}

\end{frame}

\begin{frame}{Étude de cas}{Architecture des microservices Kubernetes}


  \begin{columns}[c]

    \begin{column}{0.4\textwidth}

      \textbf{Approche SMA}
      \begin{itemize}
        \item Un agent par problème
        \item Cibler les problèmes par priorité
        \item Changement d'échelle facilité
      \end{itemize}

      \

      \textbf{Mise en œuvre dans KARMA}
      \begin{itemize}
        \item PoC fonctionnel sur cluster simple
        \item Amélioration résilience opérationnelle
        \item Convergence plus rapide
      \end{itemize}

    \end{column}

    \begin{column}{0.7\textwidth}

      \includegraphics[width=\linewidth]{figures/KARMA_architecture.png}

    \end{column}
  \end{columns}

  \vfill

  {\tiny \begin{spacing}{0.8}
      \textit{J. Soule, J.-P. Jamont, M. Occello, L.-M. Traonouez, and P. Théron. Streamlining Resilient Kubernetes Autoscaling with Multi-Agent Systems via an Automated Online Design Framework. Proceedings of the 18th IEEE International Conference on Cloud Computing (CLOUD), Helsinki, Finland, July 2025. (Accepted).}
    \end{spacing}}

\end{frame}

% \begin{frame}{Étude de cas}{Architecture des microservices Kubernetes}

%   \vspace{-0cm}

%   \begin{columns}[c]

%     \hspace{-0cm}

%     \begin{column}{0.5\textwidth}
%       {\scriptsize
%         \textbf{Principaux résultats :}
%         \begin{itemize}

%           \item Modélisation
%                 \begin{itemize}
%                   \item Validé quand dataset de trajectoires « assez important » (>7 Go)
%                 \end{itemize}

%           \item Entrainement
%                 \begin{itemize}
%                   \item Validé comme plus stable, plus grande convergence, résilience opérationnelle plus importante que HPA
%                 \end{itemize}

%           \item Analyse
%                 \begin{itemize}
%                   \item Validé rôles et objectifs inférées correspondent aux attentes (vérification manuelle des trajectoires produites, visualisation PCA…)
%                 \end{itemize}

%           \item Transfert
%                 \begin{itemize}
%                   \item Validé prise en compte des modifications environnementales en temps réel (changement brusque du volume d'entrée, changement politique de l'attaquant…)
%                 \end{itemize}

%         \end{itemize}
%       }
%     \end{column}

%     \begin{column}{0.5\textwidth}
%       \centering
%       \includegraphics[width=0.87\linewidth]{figures/k8s_learning_curve.png}
%       \ \\ \ \medskip
%       \includegraphics[width=0.87\linewidth]{figures/k8s_results.png}
%     \end{column}

%   \end{columns}

%   \vfill

%   {\tiny \begin{spacing}{0.8}
%       \textit{J. Soule, J.-P. Jamont, M. Occello, L.-M. Traonouez, and P. Théron. Streamlining Resilient Kubernetes Autoscaling with Multi-Agent Systems via an Automated Online Design Framework. Proceedings of the 18th IEEE International Conference on Cloud Computing (CLOUD), Helsinki, Finland, July 2025. (Accepted).}
%     \end{spacing}}

% \end{frame}

\begin{frame}{Quelques résultats saillant pour environnements de cyberdéfense}

  \begin{columns}[c]

    \hspace{-0cm}

    \begin{column}{0.5\textwidth}
      \centering
      \includegraphics[width=1\linewidth]{figures/company_infra_learning_curve.png}
      \ \\ \ \medskip
      \includegraphics[width=0.9\linewidth]{figures/infra_results.png}
    \end{column}

    \begin{column}{0.5\textwidth}
      \centering
      \includegraphics[width=0.87\linewidth]{figures/k8s_results.png}
      \ \\ \ \medskip
      \includegraphics[width=0.87\linewidth]{figures/cage_own_results.png}
      \ \\ \ \medskip
      \includegraphics[width=0.5\linewidth]{figures/cage_leader_results.png}
    \end{column}

  \end{columns}

\end{frame}

\begin{frame}{Couverture globale des critères C1–C5 — Apports de la méthode MAMAD}
  \scriptsize
  \setlength{\tabcolsep}{8pt}
  \renewcommand{\arraystretch}{1.5}
  \setlength{\extrarowheight}{2pt}
  \centering
  \begin{tabular}{|l|c|c|c|c|c|}
    \hline
    \textbf{Environnement} & \textbf{C1 Autonomie} & \textbf{C2 Performance} & \textbf{C3 Adaptation} & \textbf{C4 Contrôle} & \textbf{C5 Explicabilité} \\
    \hline
    Overcooked-AI          & 0.20                  & 0.82                    & 0.80                   & 0.75                 & 0.72                      \\
    Predator-Prey          & 0.20                  & 0.79                    & 0.77                   & 0.73                 & 0.69                      \\
    Warehouse Management   & 0.20                  & 0.85                    & 0.82                   & 0.77                 & 0.76                      \\
    Company Infrastructure & 0.25                  & 0.88                    & 0.83                   & 0.85                 & 0.81                      \\
    Microservices K8s      & 0.25                  & 0.91                    & 0.86                   & 0.88                 & 0.83                      \\
    Drone Swarm            & 0.20                  & 0.89                    & 0.84                   & 0.86                 & 0.82                      \\
    \hline
    \textbf{Moyenne}       & \textbf{0.21}         & \textbf{0.86}           & \textbf{0.82}          & \textbf{0.81}        & \textbf{0.77}             \\
    \hline
  \end{tabular}

  \vspace{0.6em}
  \begin{itemize}\scriptsize
    \item \textbf{Stabilité accrue} : réduction notable de la variance inter-épisodes $\Rightarrow$ apprentissage plus stable et comportements plus lisibles.
    \item \textbf{Contrôle organisationnel effectif} : taux de violation des contraintes $<1\,\%$, maintien de la cohérence des rôles et missions dans tous les environnements.
    \item \textbf{Explicabilité satisfaisante} : alignement moyen $\approx 0.8$ entre rôles inférés et spécifications MOISE+ $\Rightarrow$ compréhension post-hoc automatisée.
    \item \textbf{Adaptation généralisée} : robustesse confirmée face à la défaillance d’agents, l’ajout/retrait de nœuds ou les changements de topologie.
    \item \textbf{Performance globale élevée} : convergence plus rapide, politiques stables et meilleures récompenses cumulées sur l’ensemble des cas d’usage.
  \end{itemize}
\end{frame}



\section{Conclusion}

\begin{frame}{Synthèse et contributions majeures}
  \small
  \textbf{Une méthode complète de conception de SMA pour la Cyberdéfense}
  \begin{itemize}
    \item \textbf{Pipeline intégré} :
          \textit{Modélisation → Entraînement sous contraintes → Analyse organisationnelle → Transfert au réel}
    \item \textbf{Contributions clés :}
          \begin{itemize}
            \item Extension \textbf{multi-agent des world models}
            \item Intégration $\mathcal{M}OISE^+$ \textbf{/ MARL} pour un apprentissage sous contraintes
            \item \textbf{Analyse explicable} des comportements émergents (rôles, missions)
            \item \textbf{Implémentation unifiée} (CybMASDE) pour la reproductibilité
          \end{itemize}
          \vspace{0.3em}
    \item \textbf{Résultats expérimentaux : un compromis optimal entre}
          \begin{itemize}
            \item \textbf{Résilience + Adaptation + Autonomie}
            \item \textbf{Performance élevée / Contrôle effectif}
            \item \textbf{Lisibilité accrue des comportements émergents}
          \end{itemize}
  \end{itemize}

\end{frame}

\begin{frame}{Perspectives et prolongements}
  \small
  \textbf{À court et moyen terme}
  \begin{itemize}
    \item \textbf{Modélisation enrichie} : hybridation données / connaissances expertes (\textit{World Model} neuro-symbolique)
    \item \textbf{Apprentissage plus robuste} : environnements dynamiques, adversaires adaptatifs, Sim2Real continu
    \item \textbf{Analyse organisationnelle avancée} : inférence automatique de rôles et structures
  \end{itemize}

  \vspace{0.4em}
  \textbf{À plus long terme}
  \begin{itemize}
    \item \textbf{Passage à l’échelle} : SMA distribués sur réseaux réels
    \item \textbf{Interopérabilité opérationnelle} : intégration SOC / CSIRT
    \item \textbf{Acceptabilité et explicabilité humaine} : transparence, audit, confiance
  \end{itemize}

  \vspace{0.3em}
  \centering
  \textit{\textbf{→ Vers des SMA autonomes, sûrs et compréhensibles dans des environnements réels.}}
\end{frame}

\begin{frame}{Publications et communications}
  \scriptsize
  \textbf{Journal international}
  \begin{itemize}
    \item J. Soulé, J.-P. Jamont, M. Occello, L.-M. Traonouez, P. Théron.
          \textit{Assisting Multi-Agent System Design with MOISE+ and MARL: The MAMAD Method},
          \textbf{JAAMAS}, 2025. (sous révision majeure)
  \end{itemize}

  \vspace{0.4em}
  \textbf{Conférences internationales}
  \begin{itemize}
    \item \textit{Streamlining Resilient Kubernetes Autoscaling with Multi-Agent Systems via an Automated Online Design Framework},
          IEEE CLOUD 2025, Helsinki.
    \item \textit{An Organizationally-Oriented Approach to Enhancing Explainability and Control in MARL},
          AAMAS 2025.
    \item \textit{A MARL-based Approach for Easing MAS Organization Engineering},
          AIAI 2024.
    \item \textit{Towards a Multi-Agent Simulation of Cyber-Attackers and Cyber-Defenders Battles},
          IEEE SMC 2023.
  \end{itemize}

  \vspace{0.4em}
  \textbf{Conférences nationales}
  \begin{itemize}
    \item \textit{Une approche organisationnelle pour améliorer l’explicabilité et le contrôle dans l’AR multi-agent},
          JFSMA 2025 — \textbf{Best Paper Award}.
    \item \textit{Approche par Renforcement pour l’Ingénierie Organisationnelle d’un SMA}, JFSMA 2024.
    \item \textit{Outil pour la Conception de SMA par AR et Modélisation Organisationnelle}, JFSMA 2024.
    \item \textit{De l’Organisation des SMA de Cyberdéfense}, RJCIA 2023 et RESSI 2023.
  \end{itemize}

  \vspace{0.4em}
  \textbf{Autres communications}
  \begin{itemize}
    \item Poster — \textit{De l’Organisation des SMA de Cyberdéfense}, JFSMA 2023.
    \item Conférence invitée — \textit{CybAIR NATO Chair, École de l’air et de l’espace}, mars 2023.
  \end{itemize}
\end{frame}


\appendix
%\setbeamertemplate{headline}{}
\setbeamertemplate{mini frames}{}

% \AtBeginSection[]{
% 	\begin{frame}
% 		\frametitle{}
% 		\tableofcontents[currentsection]
% 	\end{frame}
% }

% %%%%%%%%%%%%%%%%%%%%%%%%%%%%%%%%%%%%

\section*{\phantom{Thanks}}

\begin{frame}{}

  \vspace{6ex}

  \centering
  {
    \Huge
    \emph{Thank You}
  }

  \vspace{6ex}

  \begin{columns}

    \hspace{-27ex}

    \begin{column}{0.5\textwidth}
      \raggedleft
      {\Large Demo video $\Longrightarrow$}
    \end{column}

    \hspace{-12ex}

    \begin{column}{0.5\textwidth}
      \includegraphics[width=0.5\linewidth]{figures/demo_qr_code.png}
    \end{column}

  \end{columns}

  \vspace{3ex}

  \centering
  {\Large
    \url{https://t.ly/4JBxr}
  }

\end{frame}


\section*{\phantom{References}}
\begin{frame}[allowframebreaks]{References}{}
  \printbibliography
\end{frame}

\newcounter{mainframenumber}
\setcounter{mainframenumber}{\value{framenumber}}

% % \begin{frame}{Annexes}
    {Context}

    \begin{block}{Multi-Agent Systems (MAS) paradigm for complex \& distributed problems}
        \begin{itemize}
            \item \textbf{task decomposition}: missions delegated to agents achieved through cooperation~\cite{Raileanu2023};
            \item \textbf{benefits}: handle conflicting goals, parallel computation, system robustness, scalability\dots
        \end{itemize}
    \end{block}

    \begin{block}{\textbf{Organization}: key for MAS designing}
        \begin{itemize}
            \item \textbf{coordination}: how to collaboratively achieve a common goal~\cite{Hubner2007};
            \item \textbf{dynamic \& uncertain environments}: flexible runtime behavior to adapt~\cite{Kathleen2020};
        \end{itemize}
    \end{block}

    \begin{block}{Methods and practice for MAS design}
        \begin{itemize}
            \item \textbf{approach + organizational model}: methods rely on designers' experience to hand-craft agents' \textbf{policies} so resulting MAS achieve goals;
                  %   \begin{itemize}
                  %       \item Examples: \emph{GAIA}~\cite{Wooldridge2000,Cernuzzi2014}, \emph{ADELFE}~\cite{Mefteh2015}, or \emph{DIAMOND}~\cite{Jamont2015}, \emph{KB-ORG}~\cite{Sims2008}
                  %   \end{itemize}
            \item \textbf{simulation to reality}: 1) safe \& efficient MAS design in high fidelity simulated environment; \quad 2) transfer to real environment to perform adequately~\cite{Schon2021}.
        \end{itemize}
        \vspace{1ex}
        \quad $\Longrightarrow$ \textbf{Iterative process proceeding by trial and error}

    \end{block}

\end{frame}

\begin{frame}{Annexes}
    {MAS basics}

    \begin{block}{Keywords}
        \begin{itemize}
            \item \textbf{Agent}: entity immersed in an environment perceiving observation and making decision autonomously to achieve some goals;
            \item \textbf{MAS}: a set of agents collaborating with self/re-organizing mechanisms to achieve their goal;
            \item \textbf{Organization}: the agents' interactions even though it may be implicit;
            \item \textbf{Organizational Model (OM)}: medium to formally describe an explicit/implicit organization;
            \item \textbf{Organizational Specifications (OS)}: components of an OM to characterize an organization
        \end{itemize}
    \end{block}

    \begin{block}{Organizational model: $\mathcal{M}OISE^+$}
        \begin{itemize}
            \item more complex than \emph{Agent Group Roles} (integration of standards);
            \item takes into account the social aspects between agents explicitly;
            \item possible to link agents' policies to organizational specifications.
        \end{itemize}
    \end{block}

\end{frame}

\begin{frame}{Annexes}
    {MARL basics}

    \begin{block}{Keywords}
        \begin{itemize}
            \item \textbf{Policy}: the \textquote{logic} to choose next action according to observation for an agent;
            \item \textbf{History/trajectory}: the tuple of (observation, action) couples over an episode;
            \item \textbf{Joint-policy / Joint-history}: all of the agents' policies / histories as tuples;
            \item \textbf{Reinforcement learning}: an agent updates its policy to maximize a cumulative reward;
            \item \textbf{Multi-Agent Reinforcement Learning (MARL)}: extends to multiple agents that learn while considering the actions of other agents;
        \end{itemize}
    \end{block}

\end{frame}



\end{document}
