\documentclass[9pt, aspectratio=169]{beamer}
% \documentclass[10pt]{beamer}
\usepackage[utf8]{inputenc}
\usepackage[T1]{fontenc}
\usepackage[english]{babel}
\usetheme{Frankfurt}

\usepackage[backend=biber, style=authoryear]{biblatex}
\addbibresource{local_references.bib}

%\usepackage{lmodern}
\usepackage{amsfonts,amssymb,amsmath}
\usepackage[english]{babel}
\usetheme{Frankfurt}

\usepackage{csquotes}
\usepackage{setspace}

\usepackage{colortbl}
\usepackage{tabularx}
\renewcommand\tabularxcolumn[1]{m{#1}}

% --- Tickz
\usepackage{physics}
\usepackage{amsmath}
\usepackage{tikz}
\usepackage{mathdots}
\usepackage{yhmath}
\usepackage{cancel}
\usepackage{color}
\usepackage{siunitx}
\usepackage{array}
\usepackage{multirow}
\usepackage{amssymb}
\usepackage{gensymb}
\usepackage{tabularx}
\usepackage{extarrows}
\usepackage{booktabs}
\usetikzlibrary{fadings}
\usetikzlibrary{patterns}
\usetikzlibrary{shadows.blur}
\usetikzlibrary{shapes}

% ---------

\usepackage{booktabs}
\usepackage{setspace}
\usepackage{amssymb}
\usepackage{adjustbox}
\usepackage{pifont}
\usepackage[inkscapeformat=png]{svg}
\usepackage{graphicx}
\usepackage{times}
\setbeamertemplate{caption}[numbered]
% % \setbeamertemplate{bibliography item}{[\theenumiv]}

\setbeamerfont{bibliography item}{size=\tiny}
\setbeamerfont{bibliography entry author}{size=\tiny}
\setbeamerfont{bibliography entry title}{size=\tiny}
\setbeamerfont{bibliography entry location}{size=\tiny}
\setbeamerfont{bibliography entry note}{size=\tiny}

\setbeamerfont{frametitle}{size=\large}

\usepackage{caption}
\usepackage{float}
\usepackage{xcolor}
\usepackage{listings}
\usepackage{animate}

\definecolor{codegreen}{rgb}{0,0.6,0}
\definecolor{codegray}{rgb}{0.5,0.5,0.5}
\definecolor{codepurple}{rgb}{0.58,0,0.82}
\definecolor{backcolour}{rgb}{0.95,0.95,0.92}
 
\lstdefinestyle{mystyle}{
    backgroundcolor=\color{backcolour},   
    commentstyle=\color{codegreen},
    keywordstyle=\color{magenta},
    numberstyle=\tiny\color{codegray},
    stringstyle=\color{codepurple},
    basicstyle=\footnotesize,
    breakatwhitespace=false,         
    breaklines=true,                 
    captionpos=b,                    
    keepspaces=true,                 
    numbers=left,                    
    numbersep=5pt,                  
    showspaces=false,                
    showstringspaces=false,
    showtabs=false,                  
    tabsize=2
}
 
\lstset{style=mystyle}

\usepackage{ragged2e}
\setbeamercolor{section in foot}{fg=white,bg=darkorange}
\setbeamercolor{subsection in foot}{fg=white,bg=darkorange}
\setbeamercolor{frametitle}{fg=white, bg=darkorange}
\setbeamercolor{title}{fg=white, bg=darkorange}
\setbeamercolor{frame}{bg=darkorange}
\setbeamercolor{block title}{bg=darkorange,fg=white}

\setbeamercolor{item}{fg=darkorange}

% \definecolor{darkorange}{rgb}{0.81, 0.52, 0.05}
\definecolor{darkorange}{rgb}{1,0.5,0}
\definecolor{darkorange2}{rgb}{1, 0.64, 0.2}
\definecolor{honeydew}{rgb}{1, 0.85, 0.45}


\newenvironment{variableblock}[3]{%
  \setbeamercolor{block body}{#2}
  \setbeamercolor{block title}{#3}
  \begin{block}{#1}}{\end{block}}

\newenvironment{prosblock}[1]{%
  % \setbeamercolor{block body}{bg=blue,fg=white}
  \setbeamercolor{block title}{bg=blue,fg=white}
  \begin{block}{#1}}{\end{block}}

\newenvironment{consblock}[1]{%
  % \setbeamercolor{block body}{bg=red,fg=white}
  \setbeamercolor{block title}{bg=red,fg=white}
  \begin{block}{#1}}{\end{block}}

\newcommand{\cmark}{\ding{51}}%
\newcommand{\xmark}{\ding{55}}%

\renewcommand{\arraystretch}{1.5}

% Please add the following required packages to your document preamble:
\usepackage{booktabs}
\usepackage{multirow}
\usepackage{colortbl}
% Beamer presentation requires \usepackage{colortbl} instead of \usepackage[table,xcdraw]{xcolor}

\usepackage{tabularray}\UseTblrLibrary{varwidth}
\usepackage{xcolor}
\def\BibTeX{{\rm B\kern-.05em{\sc i\kern-.025em b}\kern-.08em
    T\kern-.1667em\lower.7ex\hbox{E}\kern-.125emX}}
% \usepackage{cite}
\usepackage{amsmath}
\newcommand{\probP}{\text{I\kern-0.15em P}}
\usepackage{etoolbox}
\patchcmd{\thebibliography}{\section*{\refname}}{}{}{}

\setlength\tabcolsep{0.5pt}

\renewcommand{\arraystretch}{0.9}
\setlength{\tabcolsep}{2pt}

\usepackage{pgffor}
\usepackage[absolute,overlay]{textpos}
\setlength{\TPHorizModule}{1cm}
\setlength{\TPVertModule}{1cm}

\setbeamerfont{bibliography item}{size=\tiny}
\setbeamerfont{bibliography entry author}{size=\tiny}
\setbeamerfont{bibliography entry title}{size=\tiny}
\setbeamerfont{bibliography entry location}{size=\tiny}
\setbeamerfont{bibliography entry note}{size=\tiny}

\setbeamerfont{bibliography entry author}{shape=\upshape,series=\mdseries,size=\footnotesize}
\setbeamerfont{bibliography entry title}{shape=\slshape,series=\mdseries,size=\footnotesize}
\setbeamerfont{bibliography entry journal}{shape=\upshape,series=\mdseries,size=\footnotesize}
\setbeamerfont{bibliography entry note}{shape=\upshape,series=\mdseries,size=\footnotesize}

\renewcommand*{\bibfont}{\scriptsize}

\newenvironment<>{varblock}[2][.9\textwidth]{%
  \setlength{\textwidth}{#1}
  \begin{actionenv}#3%
    \def\insertblocktitle{#2}%
    \par%
    \usebeamertemplate{block begin}}
  {\par%
    \usebeamertemplate{block end}%
  \end{actionenv}}

% \setbeamertemplate{footline}[frame number]

\setbeamertemplate{footline}{
  \leavevmode%
  \hfill
  \usebeamercolor[fg]{page number in head/foot}%
  \scriptsize%
  \ifnum\value{framenumber}>23%
    Appendix \number\numexpr\value{framenumber}-32\relax/32%
  \else%
    \ifnum\value{framenumber}>20%
      %
    \else
      \number\numexpr\value{framenumber}\relax/20%
    \fi

  \fi%
  \hspace{1em}
}

\begin{document}

\author{\textbf{Julien Soulé$^{1,2}$}, Jean-Paul Jamont$^1$, Michel Occello$^1$, Louis-Marie Traonouez$^2$, Paul Théron$^3$}

\title{\textbf{Towards Assisted MAS Design: A Library for
Explainable MARL with Organizational Model}}

\subtitle{ECAI 2024 Demo Presentation}

% \logo{\includegraphics[scale=0.01]{figures/grenoble-inp_logo.png}}

\institute{\footnotesize \textit{University Grenoble Alpes, Grenoble INP, LCIS, 26000, Valence, France \\
$^1$\{julien.soule, jean-paul.jamont, michel.occello\}@lcis.grenoble-inp.fr \\ \phantom{U} \\
Thales Land and Air Systems, BL IAS, 35000, Rennes, France \\
$^2$\{julien.soule, louis-marie.traonouez\}@thalesgroup.com \\ \phantom{U} \\
AICA IWG, La Guillermie, France \\
$^3$paul.theron@orange.fr}}


\date{\textit{\footnotesize May 9, 2024}}

%\subject{}
\setbeamercovered{transparent}
%\setbeamertemplate{navigation symbols}{}
\begin{frame}[plain]
	\maketitle\vspace{-0.8cm}
	\begin{figure}[ht!]
		\centering
            \includegraphics[height=0.8cm]{figures/la-ruche_logo.png}
            \hspace{0.8cm}
            \includegraphics[height=0.8cm]{figures/lcis_logo.png}
            \hspace{0.8cm}
		\includegraphics[height=0.8cm]{figures/grenoble-inp_logo.png}
            \hspace{0.8cm}
            \includegraphics[height=0.8cm]{figures/uga_logo.jpg}
	\end{figure}
\end{frame}

\begin{frame}{Sommaire}
  \tableofcontents
\end{frame}

\addtocounter{framenumber}{-2}

\section{Résumé}
\begin{frame}{Résumé}
  \begin{itemize}
    \item \textbf{Contexte AICA (IST-152 NATO, 2016--2019)} : détecter/identifier des anomalies, planifier et exécuter des contre-mesures, autonomie, discrétion, interopérabilité, capacité d’apprentissage.
    \item \textbf{Motivation} : vers un \textit{Système Multi-Agent Centric AICA} face à l’augmentation de la surface d’attaque et aux défis pendant l’attaque.
    \item \textbf{Problème} : rechercher automatiquement une organisation de SMA de cyberdéfense satisfaisant (C1) Autonomie, (C2) Performance, (C3) Adaptation, (C4) Contrôle, (C5) Explicabilité.
  \end{itemize}
\end{frame}

\section{Introduction}
\begin{frame}{Introduction}{General context}

  \begin{columns}

    \begin{column}{0.7\textwidth}

      \begin{itemize}
        \item \textbf{Increasing attack surface}
              \begin{itemize}
                \item IoT/IoBT: drones, autonomous vehicles
              \end{itemize}
        \item \textbf{Challenges during attack}
              \begin{itemize}
                \item Operators' limitations: time constraints, workload, complexity\dots
                \item Environment's limitations: jamming, communication interruption\dots
              \end{itemize}
      \end{itemize}

      \ \\

      $\Longrightarrow$ Need for: \textbf{reactivity, flexibiity, autonomy}\dots

      \begin{itemize}
        \item A Multi-Agent approach for Cyberdefense
              \begin{itemize}
                \item An agent\dots
                \item A Multi-Agent System (MAS)\dots
              \end{itemize}
      \end{itemize}

      \ \\

      $\Longrightarrow$ Promising for: \textbf{adaptation, scalability, sub-task delegation}\dots

    \end{column}

    \begin{column}{0.4\textwidth}
      \begin{figure}
        \includegraphics[width=\linewidth]{figures/casino.jpg}
        \caption*{\tiny\url{https://hackread.com/hackers-casinos-fish-tank-smart-thermometer-hack/}}
      \end{figure}

      \vspace{0.cm}
      \animategraphics[autoplay,loop,width=\linewidth]{1}{figures/cyberdefense_mas_frames/frame}{0}{8}

    \end{column}

  \end{columns}

\end{frame}

\begin{frame}{Introduction}{General context}

  \begin{columns}

    \begin{column}{0.5\textwidth}

      \begin{itemize}
        \item MASCARA (Multi Agent Centric AICA Reference Architecture)
              \begin{itemize}
                \item AICA theorized by “IST-152 NATO” (2016-2019)
              \end{itemize}

        \item Detect, identify and characterize anomalies/attacks

              \begin{itemize}
                \item Plan and execute countermeasures
              \end{itemize}

        \item Communicate with C2 / operators…

              \begin{itemize}
                \item Be autonomous, stealthy, interoperable, capable of learning
              \end{itemize}

        \item MASCARA: A Multi-Agent vision of AICA
              \begin{itemize}
                \item An implicit organization
              \end{itemize}
      \end{itemize}

    \end{column}

    \hspace{-2ex}
    \begin{column}{0.6\textwidth}
      \includegraphics[width=\linewidth]{figures/mascara.png}
    \end{column}

  \end{columns}

\end{frame}

\begin{frame}{Introduction}{Problem}

  \begin{alertblock}{General problem}
    \textbf{What organizational mechanisms of the Cyberdefense MAS (AICA) to optimize its operation taking into account its constraints?}
  \end{alertblock}

  \begin{columns}

    \begin{column}{0.6\textwidth}
      \begin{figure}
        \centering
        \includegraphics[width=0.95\linewidth]{figures/general_problem_illustration.png}
      \end{figure}
    \end{column}

    \begin{column}{0.5\textwidth}
      \textbf{Literature study of available Cyberdefense MAS (CMAS) organizations}

      \begin{itemize}
        \item Few works dealing with a Multi-Agent approach to Cyberdefense
        \item Difficult to have a general vision of the effects of the organizations involved depending on the deployment environment
      \end{itemize}

      \ \\

      $\Longrightarrow$ \textbf{Need a study framework to address the problem…}
    \end{column}
  \end{columns}

\end{frame}

\begin{frame}{Introduction}{Approach for addressing the problem}

  A methodological contribution:
  \begin{itemize}

    % \item Review of work for the development of Cyberdefense MAS
    %       \begin{itemize}
    %           \item Expectations of a methodology for the development of Cyberdefense MAS
    %           \item Overview of available work versus expectations
    %           \item Discussion on methodological obstacles
    %       \end{itemize}

    \item \textbf{Need for modeling problem \& design foundation}
          \begin{itemize}
            \item[$\rightarrow$] \textbf{CybMASFM}: Markovian framework + Digital Twins (simulation/emulation coupling)
          \end{itemize}

    \item \textbf{Need for automated safe design}
          \begin{itemize}
            \item[$\rightarrow$] \textbf{CybMASDA}: Comprehensive design process
              \begin{itemize}
                \item[$\rightarrow$] OMARL: MARL + Organizational model
              \end{itemize}
          \end{itemize}

    \item \textbf{Need for practical design means}
          \begin{itemize}
            \item[$\rightarrow$] \textbf{CybMASDE}: implemented CybMASDA as an API + GUI
          \end{itemize}

  \end{itemize}

  \ \\
  \begin{itemize}

    \item Academic \& industrial \textbf{case studies} for AICA\dots
          \begin{itemize}
            \item Drone swarm
            \item Company Infrastructure
            \item Kubernetes/Drones environment
          \end{itemize}

  \end{itemize}

\end{frame}


\section{Contexte et Problématique}
\begin{frame}{Contexte}
  \begin{itemize}
    \item AICA : cadre capable de détecter, caractériser et répondre aux attaques en continu.
    \item Référence \textbf{MASCARA} : \textit{Multi-Agent System Centric AICA Reference Architecture}.
    \item Exemple d’attaque IoT : capteurs/thermomètres connectés compromis (vulnérabilités et latéralisation).
  \end{itemize}
\end{frame}

\begin{frame}{Motivations et Objectifs}
  \begin{itemize}
    \item Besoin d’une \textbf{organisation} adaptée du SMA sous contraintes environnementales et d’ingénierie.
    \item \textbf{Objectif} : automatiser la recherche d’organisation optimisant C1--C5.
  \end{itemize}
  \begin{block}{Hypothèse directrice}
    La combinaison \textbf{MARL} + \textbf{modèle organisationnel} permet d’améliorer \textit{contrôle}, \textit{sûreté} et \textit{explicabilité} sans dégrader les performances.
  \end{block}
\end{frame}

\section{État de l’art}
\begin{frame}{Synthèse}
  \begin{itemize}
    \item \textbf{Autonomie (C1)} : cycle de vie encore très dépendant de l’humain ; autonomie organisationnelle peu étudiée.
    \item \textbf{Performance (C2)} : critère le plus traité (récompense, succès), robustesse limitée.
    \item \textbf{Adaptation (C3)} : quelques approches co-évolutives, peu généralisables.
    \item \textbf{Contrôle (C4)} : quasi-absent (spécification/vérification de contraintes organisationnelles).
    \item \textbf{Explicabilité (C5)} : outils rares pour relier comportements et organisation.
  \end{itemize}
\end{frame}

\section{Approche proposée}
\begin{frame}{Principe général}
  \begin{itemize}
    \item Concevoir un SMA de cyberdéfense mariant \textbf{ML/MARL} et \textbf{contrôle organisationnel}.
    \item \textbf{Méthode en 4 activités} : (1) Modélisation, (2) Entraînement, (3) Analyse, (4) Transfert.
  \end{itemize}
\end{frame}

\section{Méthode}
\subsection{1. Modélisation}
\begin{frame}{Modélisation : environnement réseau}
  \begin{itemize}
    \item Propriétés : ouvert, dynamique/statique, déterministe, (in)accessible.
    \item \textbf{Équipes} : verte (utilisateurs/benign), rouge (attaquants), bleue (défenseurs).
    \item Observations : interfaces réseau, processus, sessions, OS, etc.
    \item Évaluation : temps de détection/réponse, nœuds affectés, messages, ressources.
  \end{itemize}
  % \includegraphics[width=\linewidth]{figures/network_context.pdf}
\end{frame}

\begin{frame}{Modélisation : cadre Markovien}
  \begin{itemize}
    \item Décision collective sous \textbf{(Dec-)POMDP}.
    \item Modélisation des incertitudes (actions / observations).
    \item \textbf{T} : fonction de transition d’état.
  \end{itemize}
  \begin{block}{Deux voies pour T}
    \begin{itemize}
      \item \textit{Manuelle} : simulateurs/émulateurs (p.ex. CybORG).
      \item \textit{Automatisée} : \textit{world models} (trajectoires $\rightarrow$ apprentissage prédictif).
    \end{itemize}
  \end{block}
\end{frame}

\subsection{2. Entraînement}
\begin{frame}{Entraînement : MARL + Organisation}
  \begin{itemize}
    \item \textbf{Vanilla MARL} : maximiser la récompense cumulée par agent.
    \item \textbf{MOISE+MARL} : contraintes \textit{dures/douces} via rôles et objectifs (imposer/refuser des actions ; guider des sous-objectifs).
  \end{itemize}
  % \includegraphics[width=.6\linewidth]{figures/marl_org.pdf}
\end{frame}

\subsection{3. Analyse}
\begin{frame}{Analyse post-entraînement (TEMM)}
  \begin{itemize}
    \item Hypothèses : \textbf{rôles} $\sim$ motifs fréquents (obs, action) ; \textbf{objectifs} $\sim$ obs. critiques fréquentes.
    \item Méthode empirique : trajectoires $\rightarrow$ vecteurs ; distances (Smith–Waterman, LCS, euclidienne) ; \textit{clustering} et centroïdes ; échantillonnage pour extraire rôles/objectifs.
  \end{itemize}
  % 


\tikzset{every picture/.style={line width=0.75pt}} %set default line width to 0.75pt        

\begin{tikzpicture}[x=0.75pt,y=0.75pt,yscale=-1,xscale=1]
    %uncomment if require: \path (0,1974); %set diagram left start at 0, and has height of 1974

    %Shape: Rectangle [id:dp9996076613305621] 
    \draw  [fill={rgb, 255:red, 255; green, 255; blue, 255 }  ,fill opacity=1 ] (24,1558.11) -- (176.1,1558.11) -- (176.1,1644) -- (24,1644) -- cycle ;
    %Straight Lines [id:da05824332013205091] 
    \draw [color={rgb, 255:red, 208; green, 2; blue, 27 }  ,draw opacity=1 ]   (142.67,1570.84) -- (124.28,1577.49) -- (87.26,1592.41) -- (108.68,1604.12) -- (93.53,1601.36) -- (86.58,1603.01) -- (86.58,1612.77) -- (82.05,1616.07) -- (81.22,1616.67) -- (78.65,1614.8) -- (70.51,1608.86) -- (54.44,1608.86) -- (57,1610.73) -- (49.09,1612.77) -- (51.85,1616.79) -- (38.38,1628.38) ;
    \draw [shift={(145.49,1569.82)}, rotate = 160.12] [fill={rgb, 255:red, 208; green, 2; blue, 27 }  ,fill opacity=1 ][line width=0.08]  [draw opacity=0] (3.57,-1.72) -- (0,0) -- (3.57,1.72) -- cycle    ;
    %Straight Lines [id:da9249559779542824] 
    \draw [color={rgb, 255:red, 80; green, 227; blue, 194 }  ,draw opacity=1 ]   (143.47,1568.13) -- (134.78,1577.63) -- (113.36,1577.63) -- (113.36,1585.44) -- (86.58,1593.25) -- (91.93,1597.15) -- (97.29,1604.96) -- (81.22,1601.06) -- (86.58,1608.86) -- (81.22,1608.86) -- (86.58,1616.67) -- (75.87,1616.67) -- (67.94,1608.94) -- (65.16,1614.72) -- (43.73,1603.01) -- (59.8,1616.67) -- (43.73,1608.86) -- (49.09,1616.67) -- (43.73,1632.29) ;
    \draw [shift={(145.49,1565.92)}, rotate = 132.45] [fill={rgb, 255:red, 80; green, 227; blue, 194 }  ,fill opacity=1 ][line width=0.08]  [draw opacity=0] (3.57,-1.72) -- (0,0) -- (3.57,1.72) -- cycle    ;
    %Straight Lines [id:da17118391857757054] 
    \draw [color={rgb, 255:red, 248; green, 231; blue, 28 }  ,draw opacity=1 ]   (153.23,1574.14) -- (126.83,1577.75) -- (124.07,1581.54) -- (105.41,1585.56) -- (91.93,1593.25) -- (93.53,1601.36) -- (89.34,1605.08) -- (81.22,1597.15) -- (91.93,1612.77) -- (91.93,1616.67) -- (81.22,1616.67) -- (59.99,1609.06) -- (57.21,1614.84) -- (41.14,1616.79) -- (35.78,1632.41) ;
    \draw [shift={(156.2,1573.73)}, rotate = 172.2] [fill={rgb, 255:red, 248; green, 231; blue, 28 }  ,fill opacity=1 ][line width=0.08]  [draw opacity=0] (3.57,-1.72) -- (0,0) -- (3.57,1.72) -- cycle    ;
    %Straight Lines [id:da6427777277243145] 
    \draw [color={rgb, 255:red, 144; green, 19; blue, 254 }  ,draw opacity=1 ]   (160.23,1577.39) -- (165.84,1578.41) -- (161.56,1573.73) -- (157.27,1570.61) -- (164.86,1573.73) -- (170.13,1578.41) -- (161.56,1583.1) -- (166.91,1589.34) -- (161.56,1597.15) -- (166.91,1601.06) -- (161.56,1612.77) -- (161.56,1628.38) -- (145.49,1624.48) -- (134.78,1624.48) -- (128.99,1623.07) -- (125.92,1622.33) -- (121.57,1621.27) -- (118.71,1620.58) -- (107.96,1619.71) -- (99.43,1619.01) -- (95.23,1618.25) -- (86.58,1616.67) -- (75.87,1616.67) -- (70.51,1620.58) -- (59.8,1624.48) -- (59.8,1632.29) ;
    \draw [shift={(157.27,1576.85)}, rotate = 10.33] [fill={rgb, 255:red, 144; green, 19; blue, 254 }  ,fill opacity=1 ][line width=0.08]  [draw opacity=0] (3.57,-1.72) -- (0,0) -- (3.57,1.72) -- cycle    ;
    %Straight Lines [id:da7390021320622445] 
    \draw [color={rgb, 255:red, 65; green, 117; blue, 5 }  ,draw opacity=1 ]   (159.15,1578.17) -- (164.77,1579.19) -- (160.49,1574.51) -- (156.2,1571.39) -- (163.79,1574.51) -- (169.06,1579.19) -- (161.56,1587.78) -- (165.84,1590.13) -- (163.7,1601.84) -- (150.85,1597.15) -- (161.56,1603.4) -- (174.41,1615.89) -- (157.27,1606.52) -- (160.49,1613.55) -- (163.7,1625.26) -- (152.99,1628.38) -- (135.85,1622.14) -- (123,1622.14) -- (116.57,1622.14) -- (110.14,1620.58) -- (108,1623.7) -- (103.72,1620.58) -- (105.86,1625.26) -- (94.16,1619.03) -- (85.51,1617.45) -- (74.8,1617.45) -- (69.44,1621.36) -- (58.73,1625.26) -- (58.73,1633.07) ;
    \draw [shift={(156.2,1577.63)}, rotate = 10.33] [fill={rgb, 255:red, 65; green, 117; blue, 5 }  ,fill opacity=1 ][line width=0.08]  [draw opacity=0] (3.57,-1.72) -- (0,0) -- (3.57,1.72) -- cycle    ;
    %Shape: Ellipse [id:dp30050508180239144] 
    \draw  [draw opacity=0][fill={rgb, 255:red, 208; green, 2; blue, 27 }  ,fill opacity=0.62 ] (46.49,1615.89) .. controls (46.49,1614.6) and (47.93,1613.55) .. (49.71,1613.55) .. controls (51.48,1613.55) and (52.92,1614.6) .. (52.92,1615.89) .. controls (52.92,1617.19) and (51.48,1618.23) .. (49.71,1618.23) .. controls (47.93,1618.23) and (46.49,1617.19) .. (46.49,1615.89) -- cycle ;
    %Shape: Ellipse [id:dp15311501498248647] 
    \draw  [draw opacity=0][fill={rgb, 255:red, 208; green, 2; blue, 27 }  ,fill opacity=0.62 ] (90.49,1619.03) .. controls (90.49,1617.74) and (91.93,1616.69) .. (93.71,1616.69) .. controls (95.48,1616.69) and (96.92,1617.74) .. (96.92,1619.03) .. controls (96.92,1620.32) and (95.48,1621.37) .. (93.71,1621.37) .. controls (91.93,1621.37) and (90.49,1620.32) .. (90.49,1619.03) -- cycle ;
    %Shape: Ellipse [id:dp19167487081496637] 
    \draw  [draw opacity=0][fill={rgb, 255:red, 208; green, 2; blue, 27 }  ,fill opacity=0.62 ] (161.11,1606.52) .. controls (161.11,1605.23) and (162.54,1604.18) .. (164.32,1604.18) .. controls (166.09,1604.18) and (167.53,1605.23) .. (167.53,1606.52) .. controls (167.53,1607.82) and (166.09,1608.86) .. (164.32,1608.86) .. controls (162.54,1608.86) and (161.11,1607.82) .. (161.11,1606.52) -- cycle ;
    %Shape: Ellipse [id:dp9201279867822619] 
    \draw  [draw opacity=0][fill={rgb, 255:red, 208; green, 2; blue, 27 }  ,fill opacity=0.62 ] (120.4,1622.92) .. controls (120.4,1621.62) and (121.84,1620.58) .. (123.62,1620.58) .. controls (125.39,1620.58) and (126.83,1621.62) .. (126.83,1622.92) .. controls (126.83,1624.21) and (125.39,1625.26) .. (123.62,1625.26) .. controls (121.84,1625.26) and (120.4,1624.21) .. (120.4,1622.92) -- cycle ;
    %Shape: Ellipse [id:dp3048334813609519] 
    \draw  [draw opacity=0][fill={rgb, 255:red, 208; green, 2; blue, 27 }  ,fill opacity=0.62 ] (161.11,1590.91) .. controls (161.11,1589.61) and (162.54,1588.56) .. (164.32,1588.56) .. controls (166.09,1588.56) and (167.53,1589.61) .. (167.53,1590.91) .. controls (167.53,1592.2) and (166.09,1593.25) .. (164.32,1593.25) .. controls (162.54,1593.25) and (161.11,1592.2) .. (161.11,1590.91) -- cycle ;
    %Shape: Ellipse [id:dp7290465976812913] 
    \draw  [draw opacity=0][fill={rgb, 255:red, 208; green, 2; blue, 27 }  ,fill opacity=0.62 ] (86.13,1603.4) .. controls (86.13,1602.11) and (87.56,1601.06) .. (89.34,1601.06) .. controls (91.11,1601.06) and (92.55,1602.11) .. (92.55,1603.4) .. controls (92.55,1604.69) and (91.11,1605.74) .. (89.34,1605.74) .. controls (87.56,1605.74) and (86.13,1604.69) .. (86.13,1603.4) -- cycle ;
    %Shape: Ellipse [id:dp6154487622646608] 
    \draw  [draw opacity=0][fill={rgb, 255:red, 208; green, 2; blue, 27 }  ,fill opacity=0.62 ] (109.69,1583.1) .. controls (109.69,1581.8) and (111.13,1580.76) .. (112.9,1580.76) .. controls (114.68,1580.76) and (116.12,1581.8) .. (116.12,1583.1) .. controls (116.12,1584.39) and (114.68,1585.44) .. (112.9,1585.44) .. controls (111.13,1585.44) and (109.69,1584.39) .. (109.69,1583.1) -- cycle ;
    %Shape: Ellipse [id:dp6108483574180856] 
    \draw  [draw opacity=0][fill={rgb, 255:red, 189; green, 16; blue, 224 }  ,fill opacity=0.8 ] (77.56,1615.89) .. controls (77.56,1614.6) and (79,1613.55) .. (80.77,1613.55) .. controls (82.55,1613.55) and (83.98,1614.6) .. (83.98,1615.89) .. controls (83.98,1617.19) and (82.55,1618.23) .. (80.77,1618.23) .. controls (79,1618.23) and (77.56,1617.19) .. (77.56,1615.89) -- cycle ;
    %Shape: Ellipse [id:dp08863924891219843] 
    \draw  [draw opacity=0][fill={rgb, 255:red, 208; green, 2; blue, 27 }  ,fill opacity=0.62 ] (84.52,1609.06) .. controls (84.52,1607.77) and (85.96,1606.72) .. (87.73,1606.72) .. controls (89.51,1606.72) and (90.95,1607.77) .. (90.95,1609.06) .. controls (90.95,1610.35) and (89.51,1611.4) .. (87.73,1611.4) .. controls (85.96,1611.4) and (84.52,1610.35) .. (84.52,1609.06) -- cycle ;
    %Shape: Ellipse [id:dp49807154634681794] 
    \draw  [draw opacity=0][fill={rgb, 255:red, 208; green, 2; blue, 27 }  ,fill opacity=0.62 ] (91.21,1601.64) .. controls (91.21,1600.35) and (92.65,1599.3) .. (94.43,1599.3) .. controls (96.2,1599.3) and (97.64,1600.35) .. (97.64,1601.64) .. controls (97.64,1602.94) and (96.2,1603.98) .. (94.43,1603.98) .. controls (92.65,1603.98) and (91.21,1602.94) .. (91.21,1601.64) -- cycle ;
    %Shape: Ellipse [id:dp17062416794692736] 
    \draw  [draw opacity=0][fill={rgb, 255:red, 208; green, 2; blue, 27 }  ,fill opacity=0.62 ] (100.59,1619.41) .. controls (100.59,1618.11) and (102.03,1617.06) .. (103.8,1617.06) .. controls (105.57,1617.06) and (107.01,1618.11) .. (107.01,1619.41) .. controls (107.01,1620.7) and (105.57,1621.75) .. (103.8,1621.75) .. controls (102.03,1621.75) and (100.59,1620.7) .. (100.59,1619.41) -- cycle ;
    %Shape: Ellipse [id:dp05427293190477478] 
    \draw  [draw opacity=0][fill={rgb, 255:red, 189; green, 16; blue, 224 }  ,fill opacity=0.8 ] (152.54,1626.82) .. controls (152.54,1625.53) and (153.98,1624.48) .. (155.75,1624.48) .. controls (157.52,1624.48) and (158.96,1625.53) .. (158.96,1626.82) .. controls (158.96,1628.12) and (157.52,1629.16) .. (155.75,1629.16) .. controls (153.98,1629.16) and (152.54,1628.12) .. (152.54,1626.82) -- cycle ;
    %Shape: Ellipse [id:dp0565658994925915] 
    \draw  [draw opacity=0][fill={rgb, 255:red, 208; green, 2; blue, 27 }  ,fill opacity=0.62 ] (158.96,1616.67) .. controls (158.96,1615.38) and (160.4,1614.33) .. (162.18,1614.33) .. controls (163.95,1614.33) and (165.39,1615.38) .. (165.39,1616.67) .. controls (165.39,1617.97) and (163.95,1619.01) .. (162.18,1619.01) .. controls (160.4,1619.01) and (158.96,1617.97) .. (158.96,1616.67) -- cycle ;
    %Shape: Ellipse [id:dp5007110255270828] 
    \draw  [draw opacity=0][fill={rgb, 255:red, 189; green, 16; blue, 224 }  ,fill opacity=0.8 ] (57,1610.73) .. controls (57,1609.43) and (58.44,1608.38) .. (60.21,1608.38) .. controls (61.99,1608.38) and (63.43,1609.43) .. (63.43,1610.73) .. controls (63.43,1612.02) and (61.99,1613.07) .. (60.21,1613.07) .. controls (58.44,1613.07) and (57,1612.02) .. (57,1610.73) -- cycle ;
    %Shape: Ellipse [id:dp22598728144573377] 
    \draw  [draw opacity=0][fill={rgb, 255:red, 208; green, 2; blue, 27 }  ,fill opacity=0.62 ] (88.72,1595.59) .. controls (88.72,1594.3) and (90.16,1593.25) .. (91.93,1593.25) .. controls (93.71,1593.25) and (95.15,1594.3) .. (95.15,1595.59) .. controls (95.15,1596.88) and (93.71,1597.93) .. (91.93,1597.93) .. controls (90.16,1597.93) and (88.72,1596.88) .. (88.72,1595.59) -- cycle ;
    %Shape: Ellipse [id:dp14749486568088088] 
    \draw  [draw opacity=0][fill={rgb, 255:red, 189; green, 16; blue, 224 }  ,fill opacity=0.8 ] (93.54,1589.34) .. controls (93.54,1588.05) and (94.98,1587) .. (96.75,1587) .. controls (98.53,1587) and (99.97,1588.05) .. (99.97,1589.34) .. controls (99.97,1590.64) and (98.53,1591.69) .. (96.75,1591.69) .. controls (94.98,1591.69) and (93.54,1590.64) .. (93.54,1589.34) -- cycle ;
    %Shape: Polygon Curved [id:ds6643267525526769] 
    \draw  [color={rgb, 255:red, 74; green, 144; blue, 226 }  ,draw opacity=1 ][fill={rgb, 255:red, 74; green, 144; blue, 226 }  ,fill opacity=0.5 ] (27.21,1628.38) .. controls (30.38,1623.39) and (36.63,1621.99) .. (42.56,1622.18) .. controls (47.05,1622.33) and (51.36,1623.39) .. (53.99,1624.48) .. controls (60.1,1627.02) and (65.56,1626.63) .. (64.7,1632.29) .. controls (63.85,1637.95) and (56.88,1637.17) .. (48.64,1636.19) .. controls (40.39,1635.22) and (21.64,1637.17) .. (27.21,1628.38) -- cycle ;
    %Shape: Polygon Curved [id:ds9461514343962948] 
    \draw  [color={rgb, 255:red, 208; green, 2; blue, 27 }  ,draw opacity=1 ][fill={rgb, 255:red, 208; green, 2; blue, 27 }  ,fill opacity=0.5 ] (139.68,1569.82) .. controls (145.25,1561.04) and (148.01,1561.59) .. (145.04,1565.92) .. controls (142.07,1570.25) and (154.68,1563.87) .. (153.82,1569.53) .. controls (152.97,1575.19) and (169.35,1578.61) .. (161.11,1577.63) .. controls (152.86,1576.66) and (134.11,1578.61) .. (139.68,1569.82) -- cycle ;
    %Shape: Ellipse [id:dp06406072166611776] 
    \draw  [draw opacity=0][fill={rgb, 255:red, 208; green, 2; blue, 27 }  ,fill opacity=0.62 ] (71.13,1617.45) .. controls (71.13,1616.16) and (72.57,1615.11) .. (74.34,1615.11) .. controls (76.12,1615.11) and (77.56,1616.16) .. (77.56,1617.45) .. controls (77.56,1618.75) and (76.12,1619.8) .. (74.34,1619.8) .. controls (72.57,1619.8) and (71.13,1618.75) .. (71.13,1617.45) -- cycle ;
    %Shape: Ellipse [id:dp049221150011381054] 
    \draw  [draw opacity=0][fill={rgb, 255:red, 189; green, 16; blue, 224 }  ,fill opacity=0.8 ] (56.59,1624.48) .. controls (56.59,1623.19) and (58.03,1622.14) .. (59.8,1622.14) .. controls (61.58,1622.14) and (63.01,1623.19) .. (63.01,1624.48) .. controls (63.01,1625.77) and (61.58,1626.82) .. (59.8,1626.82) .. controls (58.03,1626.82) and (56.59,1625.77) .. (56.59,1624.48) -- cycle ;


    % Text Node
    \draw (68.95,1583.75) node  [font=\tiny,color={rgb, 255:red, 189; green, 16; blue, 224 }  ,opacity=1 ] [align=left] {$\displaystyle g_{5} =\{\omega _{21} \dotsc \}$};
    % Text Node
    \draw (82.63,1627.35) node  [font=\tiny,color={rgb, 255:red, 189; green, 16; blue, 224 }  ,opacity=1 ] [align=left] {$\displaystyle g_{2} =\{\omega _{5}\}$};
    % Text Node
    \draw (152.91,1586.86) node  [font=\tiny,color={rgb, 255:red, 189; green, 16; blue, 224 }  ,opacity=1 ] [align=left] {$\displaystyle ...$};
    % Text Node
    \draw (101.5,1575.93) node  [font=\tiny,color={rgb, 255:red, 189; green, 16; blue, 224 }  ,opacity=1 ] [align=left] {$\displaystyle ...$};
    % Text Node
    \draw (136.45,1636.35) node  [font=\tiny,color={rgb, 255:red, 189; green, 16; blue, 224 }  ,opacity=1 ] [align=left] {$\displaystyle g_{4} =\{\omega _{301} ,\omega _{302}\}$};
    % Text Node
    \draw (113.58,1612.35) node  [font=\tiny,color={rgb, 255:red, 189; green, 16; blue, 224 }  ,opacity=1 ] [align=left] {$\displaystyle g_{3} =\{\omega _{10}\}$};
    % Text Node
    \draw (55.11,1600.35) node  [font=\tiny,color={rgb, 255:red, 189; green, 16; blue, 224 }  ,opacity=1 ] [align=left] {$\displaystyle g_{1} =\{\omega _{1}\}$};
    % Text Node
    \draw (105.58,1568.35) node  [font=\tiny,color={rgb, 255:red, 202; green, 52; blue, 69 }  ,opacity=1 ] [align=left] {$\displaystyle g_{*} =\Omega _{goal}$};
    % Text Node
    \draw (73.43,1637.35) node  [font=\tiny,color={rgb, 255:red, 74; green, 144; blue, 226 }  ,opacity=1 ] [align=left] {$\displaystyle \Omega _{init}$};
    % Text Node
    \draw (32.91,1567.84) node  [font=\scriptsize] [align=left] {$\displaystyle \Omega $};
    % Text Node
    \draw (93.61,1653) node   [align=left] {{\tiny \textit{Une visualisation abstraite des}}};
    \draw (93.61,1662) node   [align=left] {{\tiny \textit{observations dans les trajectoires}}};

\end{tikzpicture}
\end{frame}

\subsection{4. Transfert}
\begin{frame}{Transfert / Déploiement}
  \begin{itemize}
    \item Déploiement \textit{remote} ou \textit{direct} ; collecte de nouvelles transitions.
    \item Boucle fermée : amélioration du modèle simulé par itération.
  \end{itemize}
\end{frame}

\section{Implémentation : CybMASDE}
\begin{frame}{CybMASDE : outil}
  \begin{itemize}
    \item Application Web (Electron) : \textbf{backend} Python, \textbf{frontend} Angular.
    \item Projet : dossiers par activité (modèles entraînés, bases, configs), fichier JSON central.
    \item Interfaces : \texttt{CLI} (\texttt{init}, \texttt{validate}, \texttt{model}, \texttt{train}, \texttt{analyze}, \texttt{refine}, \texttt{deploy}) et \textit{GUI}.
  \end{itemize}
\end{frame}

\section{Études de cas}
\begin{frame}{Environnements non-cyberdéfense}
  \begin{itemize}
    \item \textbf{Predator–Prey} : rôles « leader chasseur », « follower hunter », « proie » ; objectifs « encercler la proie », etc.
    \item \textbf{Warehouse Management} : rôles « artisan principal/secondaire » ; objectif « maximiser l’artisanat ».
    \item \textbf{Overcooked-AI} : rôles « polyvalent/serveur/cuisinier » ; objectif « maximiser le bol non vide ».
  \end{itemize}
  \begin{block}{Résultats (extraits)}
    Modélisation validée avec datasets $\gtrsim$ 5 Go ; contraintes respectées ; rôles/objectifs inférés conformes aux attentes ; transfert réactif aux modifications en temps réel.
  \end{block}
\end{frame}

\begin{frame}{Infrastructure d’entreprise}
  \begin{itemize}
    \item Trois modèles organisationnels : attaquant/défenseur aléatoire, manuel, et apprentissage attaquant/défenseur.
  \end{itemize}
  \begin{block}{Résultats (extraits)}
    Zéro violation sous contraintes dures ; vérifications manuelles des trajectoires ; adaptation à l’ajout/retrait de nœud réseau en temps réel.
  \end{block}
\end{frame}

\begin{frame}{Essaim de drones (CAGE Challenge)}
  \begin{itemize}
    \item Organisations : « Isolation du suspect », « Défense Active », « Manuel ».
  \end{itemize}
  \begin{block}{Résultats (extraits)}
    Stabilité accrue, meilleure convergence ; validation des rôles/objectifs ; robustesse face aux perturbations (p.ex., forcer l’infection d’un drone).
  \end{block}
\end{frame}

\begin{frame}{Microservices Kubernetes}
  \begin{itemize}
    \item \textbf{Approche SMA} : un agent par problème ; priorisation ; passage à l’échelle.
    \item \textbf{KARMA (PoC)} : amélioration de la résilience ; convergence plus rapide qu’HPA.
  \end{itemize}
  \begin{block}{Résultats (extraits)}
    Datasets $\gtrsim$ 7 Go ; stabilité et convergence accrues ; visualisations (PCA) confirmant les rôles/objectifs ; adaptation aux changements brusques (charge, politique d’attaquant).
  \end{block}
\end{frame}

\section{Conclusion}
\begin{frame}{Synthèse et contributions}
  \begin{itemize}
    \item Méthode de conception SMA pour la cyberdéfense : \textbf{modélisation} (simu \& world models), \textbf{entraînement sous contraintes} (MOISE+MARL), \textbf{analyse organisationnelle} (TEMM), \textbf{transfert}.
    \item Contributions : extension multi-agent des world models ; intégration MOISE+ dans MARL ; méthode d’analyse des comportements (rôles, missions) ; implémentation unifiée ; expérimentations multi-environnements.
    \item Bénéfices : \textbf{Résilience} + \textbf{Adaptation} + \textbf{Autonomie} ; meilleurs compromis performance/contrôle ; meilleure lisibilité des comportements.
  \end{itemize}
\end{frame}

\begin{frame}{Outils utilisables}
  \begin{itemize}
    \item \textbf{MOISE+MARL} : lib Python RL/MARL pour scénarios multi-agents/simulation ; rôles/objets JSON ; outils d’explication ; export JSON.
    \item \textbf{CybMASDE} : app Electron pour configurer/lancer la méthode (\textit{model}, \textit{train}, \textit{analyze}, \textit{transfer}); \texttt{install.sh} ; \texttt{CLI} et \textit{GUI}.
  \end{itemize}
\end{frame}

\begin{frame}{Perspectives}
  \begin{itemize}
    \item \textbf{Court/moyen terme} : modélisation enrichie (hybridation data+symbolique) ; apprentissage plus robuste (dynamiques/adversaires ; sim2real continu) ; analyse orga avancée (inférence structurelle).
    \item \textbf{Long terme} : passage à l’échelle (clusters, réseaux réels) ; interopérabilité (SOC, CSIRT) ; explicabilité opérationnelle ; déploiement hybride simu/réel.
  \end{itemize}
\end{frame}

\appendix
%\setbeamertemplate{headline}{}
\setbeamertemplate{mini frames}{}

% \AtBeginSection[]{
% 	\begin{frame}
% 		\frametitle{}
% 		\tableofcontents[currentsection]
% 	\end{frame}
% }

% %%%%%%%%%%%%%%%%%%%%%%%%%%%%%%%%%%%%

\section*{\phantom{Thanks}}

\begin{frame}{}

  \vspace{6ex}

  \centering
  {
    \Huge
    \emph{Thank You}
  }

  \vspace{6ex}

  \begin{columns}

    \hspace{-27ex}

    \begin{column}{0.5\textwidth}
      \raggedleft
      {\Large Demo video $\Longrightarrow$}
    \end{column}

    \hspace{-12ex}

    \begin{column}{0.5\textwidth}
      \includegraphics[width=0.5\linewidth]{figures/demo_qr_code.png}
    \end{column}

  \end{columns}

  \vspace{3ex}

  \centering
  {\Large
    \url{https://t.ly/4JBxr}
  }

\end{frame}


\section*{\phantom{References}}
\begin{frame}[allowframebreaks]{References}{}
  \printbibliography
\end{frame}

\newcounter{mainframenumber}
\setcounter{mainframenumber}{\value{framenumber}}

% % \begin{frame}{Annexes}
    {Context}

    \begin{block}{Multi-Agent Systems (MAS) paradigm for complex \& distributed problems}
        \begin{itemize}
            \item \textbf{task decomposition}: missions delegated to agents achieved through cooperation~\cite{Raileanu2023};
            \item \textbf{benefits}: handle conflicting goals, parallel computation, system robustness, scalability\dots
        \end{itemize}
    \end{block}

    \begin{block}{\textbf{Organization}: key for MAS designing}
        \begin{itemize}
            \item \textbf{coordination}: how to collaboratively achieve a common goal~\cite{Hubner2007};
            \item \textbf{dynamic \& uncertain environments}: flexible runtime behavior to adapt~\cite{Kathleen2020};
        \end{itemize}
    \end{block}

    \begin{block}{Methods and practice for MAS design}
        \begin{itemize}
            \item \textbf{approach + organizational model}: methods rely on designers' experience to hand-craft agents' \textbf{policies} so resulting MAS achieve goals;
                  %   \begin{itemize}
                  %       \item Examples: \emph{GAIA}~\cite{Wooldridge2000,Cernuzzi2014}, \emph{ADELFE}~\cite{Mefteh2015}, or \emph{DIAMOND}~\cite{Jamont2015}, \emph{KB-ORG}~\cite{Sims2008}
                  %   \end{itemize}
            \item \textbf{simulation to reality}: 1) safe \& efficient MAS design in high fidelity simulated environment; \quad 2) transfer to real environment to perform adequately~\cite{Schon2021}.
        \end{itemize}
        \vspace{1ex}
        \quad $\Longrightarrow$ \textbf{Iterative process proceeding by trial and error}

    \end{block}

\end{frame}

\begin{frame}{Annexes}
    {MAS basics}

    \begin{block}{Keywords}
        \begin{itemize}
            \item \textbf{Agent}: entity immersed in an environment perceiving observation and making decision autonomously to achieve some goals;
            \item \textbf{MAS}: a set of agents collaborating with self/re-organizing mechanisms to achieve their goal;
            \item \textbf{Organization}: the agents' interactions even though it may be implicit;
            \item \textbf{Organizational Model (OM)}: medium to formally describe an explicit/implicit organization;
            \item \textbf{Organizational Specifications (OS)}: components of an OM to characterize an organization
        \end{itemize}
    \end{block}

    \begin{block}{Organizational model: $\mathcal{M}OISE^+$}
        \begin{itemize}
            \item more complex than \emph{Agent Group Roles} (integration of standards);
            \item takes into account the social aspects between agents explicitly;
            \item possible to link agents' policies to organizational specifications.
        \end{itemize}
    \end{block}

\end{frame}

\begin{frame}{Annexes}
    {MARL basics}

    \begin{block}{Keywords}
        \begin{itemize}
            \item \textbf{Policy}: the \textquote{logic} to choose next action according to observation for an agent;
            \item \textbf{History/trajectory}: the tuple of (observation, action) couples over an episode;
            \item \textbf{Joint-policy / Joint-history}: all of the agents' policies / histories as tuples;
            \item \textbf{Reinforcement learning}: an agent updates its policy to maximize a cumulative reward;
            \item \textbf{Multi-Agent Reinforcement Learning (MARL)}: extends to multiple agents that learn while considering the actions of other agents;
        \end{itemize}
    \end{block}

\end{frame}



\end{document}
