
\setlength{\tabcolsep}{8pt}       % horizontal space between column text and vertical rules (default ~6pt)
\renewcommand{\arraystretch}{1.8} % vertical stretch for row height
\setlength{\extrarowheight}{2pt}  % requires \usepackage{array}; adds a little extra row height

\begin{tabular}{p{4cm}p{3cm}p{3cm}p{3cm}p{3cm}p{3cm}}
    \hline
    \textbf{Catégorie de travaux} & \textbf{C1 Autonomie}                                                             & \textbf{C2 Performance} & \textbf{C3 Adaptation} & \textbf{C4 Contrôle} & \textbf{C5 Explicabilité} \\
    \hline
    {\textbf{Organisations de SMA de Cyberdéfense} \textit{(CIDS, Ant-defense, moving target, etc.)}}
                                  & Partielle (autonomie locale dans essaims, mais dépendance hiérarchique fréquente)
                                  & Bonne (agrégation efficace, mais fragilité sous charge)
                                  & Moyenne (auto-organisation présente mais difficile à stabiliser)
                                  & Variable (contrôle fort en centralisé, diffus en décentralisé)
                                  & Moyenne (explicabilité locale bonne, globale limitée)                                                                                                                                   \\

    \textbf{Cadres de conception de SMA} \textit{(CSLE, CALDERA, NaSimEmu, CybORG, CyberBattleSim, etc.)}
                                  & Partielle (agents simulés, supervision humaine requise)
                                  & Bonne (mesure de récompenses et convergence)
                                  & Partielle (co-évolution et variantes topologiques limitées)
                                  & Faible à partielle (peu de contraintes explicites)
                                  & Faible à partielle (visualisation sans formalisation organisationnelle)                                                                                                                 \\

    \textbf{Modélisation d’environnement de Cyberdéfense} \textit{(Attack\&Defense Trees, Petri Nets, Attack graphs, World Models, etc.)}
                                  & Partielle (fidélité simulée perfectible)
                                  & Bonne (évaluation réaliste des politiques)
                                  & Bonne (test de scénarios variés)
                                  & Faible (peu de gouvernance organisationnelle)
                                  & Faible (manque d’explicabilité de la dynamique simulée)                                                                                                                                 \\

    \textbf{Intégration de contraintes / guidages organisationnels en MARL} \textit{(Shielding, Reward Shaping, Constrained-RL, etc.)}
                                  & Partielle (influence limitée sur autonomie individuelle)
                                  & Bonne (meilleure convergence sous contraintes)
                                  & Bonne (robustesse accrue)
                                  & Moyenne à bonne (cohérence organisationnelle assurée)
                                  & Moyenne (amélioration de la traçabilité)                                                                                                                                                \\

    \textbf{Extraction organisationnelle émergente} \textit{(ROMA, Unsupervised ML for time-series, etc.)}
                                  & Bonne (auto-organisation complète)
                                  & Moyenne (stabilité difficile à garantir)
                                  & Bonne (rôles et dépendances émergents)
                                  & Partielle (pas de vérification formelle)
                                  & Bonne (explicabilité post-hoc ou via rôles émergents)                                                                                                                                   \\

    \textbf{Maintien de cohérence simulation/réel} \textit{(Sim2Real, Transfer learning, Domain Randomization, etc.)}
                                  & Moyenne (boucles de transfert semi-automatisées)
                                  & Bonne (amélioration de la robustesse inter-domaines)
                                  & Bonne (réentraînement adaptatif possible)
                                  & Moyenne (cohérence opérationnelle non garantie)
                                  & Moyenne (traçabilité améliorée, mais encore limitée)                                                                                                                                    \\
    \hline
\end{tabular}
