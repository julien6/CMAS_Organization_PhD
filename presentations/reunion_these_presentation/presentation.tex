\documentclass[9pt, aspectratio=169]{beamer}
% \documentclass[10pt]{beamer}
\usepackage[utf8]{inputenc}
\usepackage[T1]{fontenc}
\usepackage[english]{babel}
\usetheme{Frankfurt}

\usepackage[backend=biber, style=authoryear]{biblatex}
\addbibresource{local_references.bib}

%\usepackage{lmodern}
\usepackage{amsfonts,amssymb,amsmath}
\usepackage[english]{babel}
\usetheme{Frankfurt}

\usepackage{csquotes}
\usepackage{setspace}

\usepackage{colortbl}
\usepackage{tabularx}
\renewcommand\tabularxcolumn[1]{m{#1}}

% --- Tickz
\usepackage{physics}
\usepackage{amsmath}
\usepackage{tikz}
\usepackage{mathdots}
\usepackage{yhmath}
\usepackage{cancel}
\usepackage{color}
\usepackage{siunitx}
\usepackage{array}
\usepackage{multirow}
\usepackage{amssymb}
\usepackage{gensymb}
\usepackage{tabularx}
\usepackage{extarrows}
\usepackage{booktabs}
\usetikzlibrary{fadings}
\usetikzlibrary{patterns}
\usetikzlibrary{shadows.blur}
\usetikzlibrary{shapes}

% ---------

\usepackage{booktabs}
\usepackage{setspace}
\usepackage{amssymb}
\usepackage{adjustbox}
\usepackage{pifont}
\usepackage[inkscapeformat=png]{svg}
\usepackage{graphicx}
\usepackage{times}
\setbeamertemplate{caption}[numbered]
% % \setbeamertemplate{bibliography item}{[\theenumiv]}

\setbeamerfont{bibliography item}{size=\tiny}
\setbeamerfont{bibliography entry author}{size=\tiny}
\setbeamerfont{bibliography entry title}{size=\tiny}
\setbeamerfont{bibliography entry location}{size=\tiny}
\setbeamerfont{bibliography entry note}{size=\tiny}

\setbeamerfont{frametitle}{size=\large}

\usepackage{caption}
\usepackage{float}
\usepackage{xcolor}
\usepackage{listings}
\usepackage{animate}

\definecolor{codegreen}{rgb}{0,0.6,0}
\definecolor{codegray}{rgb}{0.5,0.5,0.5}
\definecolor{codepurple}{rgb}{0.58,0,0.82}
\definecolor{backcolour}{rgb}{0.95,0.95,0.92}
 
\lstdefinestyle{mystyle}{
    backgroundcolor=\color{backcolour},   
    commentstyle=\color{codegreen},
    keywordstyle=\color{magenta},
    numberstyle=\tiny\color{codegray},
    stringstyle=\color{codepurple},
    basicstyle=\footnotesize,
    breakatwhitespace=false,         
    breaklines=true,                 
    captionpos=b,                    
    keepspaces=true,                 
    numbers=left,                    
    numbersep=5pt,                  
    showspaces=false,                
    showstringspaces=false,
    showtabs=false,                  
    tabsize=2
}
 
\lstset{style=mystyle}

\usepackage{ragged2e}
\setbeamercolor{section in foot}{fg=white,bg=darkorange}
\setbeamercolor{subsection in foot}{fg=white,bg=darkorange}
\setbeamercolor{frametitle}{fg=white, bg=darkorange}
\setbeamercolor{title}{fg=white, bg=darkorange}
\setbeamercolor{frame}{bg=darkorange}
\setbeamercolor{block title}{bg=darkorange,fg=white}

\setbeamercolor{item}{fg=darkorange}

% \definecolor{darkorange}{rgb}{0.81, 0.52, 0.05}
\definecolor{darkorange}{rgb}{1,0.5,0}
\definecolor{darkorange2}{rgb}{1, 0.64, 0.2}
\definecolor{honeydew}{rgb}{1, 0.85, 0.45}


\newenvironment{variableblock}[3]{%
  \setbeamercolor{block body}{#2}
  \setbeamercolor{block title}{#3}
  \begin{block}{#1}}{\end{block}}

\newenvironment{prosblock}[1]{%
  % \setbeamercolor{block body}{bg=blue,fg=white}
  \setbeamercolor{block title}{bg=blue,fg=white}
  \begin{block}{#1}}{\end{block}}

\newenvironment{consblock}[1]{%
  % \setbeamercolor{block body}{bg=red,fg=white}
  \setbeamercolor{block title}{bg=red,fg=white}
  \begin{block}{#1}}{\end{block}}

\newcommand{\cmark}{\ding{51}}%
\newcommand{\xmark}{\ding{55}}%

\renewcommand{\arraystretch}{1.5}

% Please add the following required packages to your document preamble:
\usepackage{booktabs}
\usepackage{multirow}
\usepackage{colortbl}
% Beamer presentation requires \usepackage{colortbl} instead of \usepackage[table,xcdraw]{xcolor}

\usepackage{tabularray}\UseTblrLibrary{varwidth}
\usepackage{xcolor}
\def\BibTeX{{\rm B\kern-.05em{\sc i\kern-.025em b}\kern-.08em
    T\kern-.1667em\lower.7ex\hbox{E}\kern-.125emX}}
% \usepackage{cite}
\usepackage{amsmath}
\newcommand{\probP}{\text{I\kern-0.15em P}}
\usepackage{etoolbox}
\patchcmd{\thebibliography}{\section*{\refname}}{}{}{}

\setlength\tabcolsep{0.5pt}

\renewcommand{\arraystretch}{0.9}
\setlength{\tabcolsep}{2pt}

\usepackage{pgffor}

\setbeamerfont{bibliography item}{size=\tiny}
\setbeamerfont{bibliography entry author}{size=\tiny}
\setbeamerfont{bibliography entry title}{size=\tiny}
\setbeamerfont{bibliography entry location}{size=\tiny}
\setbeamerfont{bibliography entry note}{size=\tiny}

\setbeamerfont{bibliography entry author}{shape=\upshape,series=\mdseries,size=\footnotesize}
\setbeamerfont{bibliography entry title}{shape=\slshape,series=\mdseries,size=\footnotesize}
\setbeamerfont{bibliography entry journal}{shape=\upshape,series=\mdseries,size=\footnotesize}
\setbeamerfont{bibliography entry note}{shape=\upshape,series=\mdseries,size=\footnotesize}

\renewcommand*{\bibfont}{\scriptsize}

\newenvironment<>{varblock}[2][.9\textwidth]{%
  \setlength{\textwidth}{#1}
  \begin{actionenv}#3%
    \def\insertblocktitle{#2}%
    \par%
    \usebeamertemplate{block begin}}
  {\par%
    \usebeamertemplate{block end}%
  \end{actionenv}}

% \setbeamertemplate{footline}[frame number]

\setbeamertemplate{footline}{
  \leavevmode%
  \hfill
  \usebeamercolor[fg]{page number in head/foot}%
  \scriptsize%
  \ifnum\value{framenumber}>19%
    Appendix \number\numexpr\value{framenumber}-19\relax/32%
  \else%
    \ifnum\value{framenumber}>16%
      %
    \else
      \number\numexpr\value{framenumber}\relax/16%
    \fi

  \fi%
  \hspace{1em}
}


\begin{document}

\author{\textbf{Julien Soulé$^{1,2}$}, Jean-Paul Jamont$^1$, Michel Occello$^1$, Louis-Marie Traonouez$^2$, Paul Théron$^3$}

\title{\textbf{Towards Assisted MAS Design: A Library for
Explainable MARL with Organizational Model}}

\subtitle{ECAI 2024 Demo Presentation}

% \logo{\includegraphics[scale=0.01]{figures/grenoble-inp_logo.png}}

\institute{\footnotesize \textit{University Grenoble Alpes, Grenoble INP, LCIS, 26000, Valence, France \\
$^1$\{julien.soule, jean-paul.jamont, michel.occello\}@lcis.grenoble-inp.fr \\ \phantom{U} \\
Thales Land and Air Systems, BL IAS, 35000, Rennes, France \\
$^2$\{julien.soule, louis-marie.traonouez\}@thalesgroup.com \\ \phantom{U} \\
AICA IWG, La Guillermie, France \\
$^3$paul.theron@orange.fr}}


\date{\textit{\footnotesize May 9, 2024}}

%\subject{}
\setbeamercovered{transparent}
%\setbeamertemplate{navigation symbols}{}
\begin{frame}[plain]
	\maketitle\vspace{-0.8cm}
	\begin{figure}[ht!]
		\centering
            \includegraphics[height=0.8cm]{figures/la-ruche_logo.png}
            \hspace{0.8cm}
            \includegraphics[height=0.8cm]{figures/lcis_logo.png}
            \hspace{0.8cm}
		\includegraphics[height=0.8cm]{figures/grenoble-inp_logo.png}
            \hspace{0.8cm}
            \includegraphics[height=0.8cm]{figures/uga_logo.jpg}
	\end{figure}
\end{frame}

% \begin{frame}{Content}
%   \tableofcontents
% \end{frame}

\addtocounter{framenumber}{-1}

\begin{frame}{Objectifs de la journée}

  \begin{itemize}
    \item Faire un point sur le sujet de la thèse et les travaux effectués
    \item Faire le point sur l'état d'avancement du manuscrit
    \item Faire le point sur le planning jusqu'à la fin de la thèse
  \end{itemize}

\end{frame}

\begin{frame}{Plan de la journée}
  \tableofcontents
\end{frame}

\section{Aperçu / Retour sur Thèse \& Travaux}

\begin{frame}{Contexte}

  \begin{itemize}
    \item Explosion de l'IoT, hétérogenité \& faible protection des objets connectés \\ $\rightarrow$ augmentation de la surface d'attaque difficilement pris en compte
    \item Attaques décentralisé et stratégisé en combinant différents vecteurs d'attaques par différentes entités \\ $\rightarrow$ peu couverts par systèmes de cyberdéfense actuels
    \item Montée en autonomie des différents attaquants par l'utilisation de bots ou assistance reposant sur l'IA \\ $\rightarrow$ renforce difficulté à assurer cyberdéfense
  \end{itemize}
\end{frame}

\begin{frame}{Motivation}
  \begin{itemize}
    \item Besoin assurer cyberdéfense adaptée au contexte de déploiement, des attaques/menaces cyber de façon dynamique \\ $\rightarrow$ apprentissage \& adaptation
    \item Besoin de  prendre en compte le systèmes distribués et décentralisés \\ $\rightarrow$ exclure approche centralisée
    \item Besoin de prendre en compte exigences de conception particulières \\ $\rightarrow$ interaction/controle opérateurs
    \item Besoin prise en compte décentralisation et distribution des cyber-attaques \\ $\rightarrow$ privilegier cyber-défense de multiple vecteurs d'action collaboratifs
    \item Travaux initiés par l'IST-152 puis l'AICA IWG sur les agents AICA \\
          $\rightarrow$ agent AICA couvre la plupart des besoins mais manque aspect distribué et décentralisé \\
          $\rightarrow$ Approche d'un SMA flexible et auto-organisé d'un agent AICA
  \end{itemize}
\end{frame}

\begin{frame}{Question de recherche}

  \begin{block}{Question de recherche}
    \begin{itemize}
      \item[Q1] Comment automatiser la conception d’un SMA de cyberdéfense auto-organisé capable d’atteindre de manière optimale et continue dans le temps ses objectifs de cyberdéfense dans un environnement dynamique, en tenant compte :
            \begin{itemize}
              \item des contraintes dynamiques liées à l'environnement incluant les cyber-attaques ;
              \item et des exigences architecturales définies par le cadre AICA ?
            \end{itemize}
    \end{itemize}
  \end{block}

\end{frame}

\begin{frame}{Verrous théoriques et techniques}

  \begin{itemize}
    \item[G1] Pas de formalisation du problème de conception d'un SMA de Cyberdéfense
    \item[G2] Faible généralisabilité et automatisation des approches de conception de SMA, et inexistence pour les SMA de Cyberdéfense
    \item[G3] Peu d'adaptivité dans la réponse cyber vis à vis des contraintes dynamiques de l'environnement et des cyber-attaques
    \item[G4] Faible explicabilité des SMA émergents (peu explicite, pas de vision organisationnelle)
    \item[G5] Impossibilité ou risques liés à l'experimentation directe sur l'environnement réel durant la conception du SMA
  \end{itemize}

\end{frame}

\begin{frame}{Hypothèses}

  \begin{itemize}
    \item[H1] Question de recherche peut être formalisé comme problème d'optimisation des politiques des agents sous contraintes de l'environnement (incluant attaquants) et exigences de conception pour maximiser objectif ;
    \item[H2] Techniques "World Models" peut permettre modéliser environnement déploiement comme simulation en plus de modélisation manuelle ;
    \item[H3] Techniques "MARL" peut permettre de résoudre le problème formalisé automatiquement ;
    \item[H4] Contraintes des exigences de conception, peuvent être intégrées dans le processus MARL pour guider/contraindre l'apprentissage des agents en utilisant un modèle organisationnel explicite
    \item[H5] Comportements des agents peuvent être compris dans un modèle organisationnel explicite
  \end{itemize}

\end{frame}

\begin{frame}{Aperçu de la réponse générale proposée}

  \begin{itemize}
    \item En général : Passer d'une approche \textquote{IA Symbolique} à une approche plus orientée \textquote{IA Machine Learning - MARL/RL} pour implémenter l'auto-organisation ;
    \item \textbf{MOISE+MARL Assisted MAS Design (MAMAD)} : Une méthode de conception en \textbf{4 phases séquentielle} sur un \textbf{cycle fermé}
    \item Chaque phase se base sur des contributions reposant sur les hypothèses précédentes
          \begin{itemize}
            \item \textbf{Modélisation} : utilisation des "World Models" pour créer un modèle de simulation réaliste
            \item \textbf{Entrainement} : utilisation de MARL + modèle organisationnel $\mathcal{M}OISE^+$ $\rightarrow$ entrainer les agents dans le cadre d'une organisation
            \item \textbf{Analyse} : utilisation de "Unsupervised Learning" sur trajectoires + modèle organisationnel $\mathcal{M}OISE^+$ $\rightarrow$ voir les agents dans le cadre d'une organisation
            \item \textbf{Transfert} : utilisation d'un framework Kubernetes pour le déploiement / mise à jour des politiques des agents dans l'environnement de déploiement ; et amélioration fidélité du modèle simulé à la réalité
          \end{itemize}
  \end{itemize}

\end{frame}

\begin{frame}{Aperçu de la réponse générale proposée}

  \begin{columns}[c]

    \hspace{-1.5cm}

    \begin{column}{0.5\textwidth}
      \begin{enumerate}
        \item Modéliser l’environnement à partir de traces réelles, de l’objectif global et des exigences de conception, sous forme de rôles et d’objectifs ;
        \item Entraînement des agents à l’aide de MOISE+MARL ;
        \item Analyse pour inférrer des rôles et objectifs émergents des agents $\rightarrow$ amélioration des spécifications organisationnelles appliquées ($\times$ itérations 2-3) ;
        \item Transferer les politiques pour piloter les actionneurs des agents (déploiement direct ou à distance) + génère nouvelles traces $\rightarrow$ modèle simulé plus fidèle.
      \end{enumerate}

    \end{column}

    \hspace{-1.5cm}

    \begin{column}{0.5\textwidth}
      \begin{figure}[h!]
        \centering
        


\tikzset{every picture/.style={line width=0.75pt}} %set default line width to 0.75pt        

\begin{tikzpicture}[x=0.75pt,y=0.75pt,yscale=-1,xscale=1]
%uncomment if require: \path (0,3307); %set diagram left start at 0, and has height of 3307

%Shape: Smiley Face [id:dp29065495216725257] 
\draw  [line width=1.5]  (85.38,2800.11) .. controls (85.38,2797.7) and (87.16,2795.75) .. (89.36,2795.75) .. controls (91.55,2795.75) and (93.34,2797.7) .. (93.34,2800.11) .. controls (93.34,2802.52) and (91.55,2804.48) .. (89.36,2804.48) .. controls (87.16,2804.48) and (85.38,2802.52) .. (85.38,2800.11) -- cycle ; \draw  [line width=1.5]  (87.61,2798.63) .. controls (87.61,2798.39) and (87.78,2798.19) .. (88,2798.19) .. controls (88.22,2798.19) and (88.4,2798.39) .. (88.4,2798.63) .. controls (88.4,2798.87) and (88.22,2799.07) .. (88,2799.07) .. controls (87.78,2799.07) and (87.61,2798.87) .. (87.61,2798.63) -- cycle ; \draw  [line width=1.5]  (90.31,2798.63) .. controls (90.31,2798.39) and (90.49,2798.19) .. (90.71,2798.19) .. controls (90.93,2798.19) and (91.11,2798.39) .. (91.11,2798.63) .. controls (91.11,2798.87) and (90.93,2799.07) .. (90.71,2799.07) .. controls (90.49,2799.07) and (90.31,2798.87) .. (90.31,2798.63) -- cycle ; \draw  [line width=1.5]  (87.37,2801.86) .. controls (88.69,2803.02) and (90.02,2803.02) .. (91.35,2801.86) ;
%Shape: Rectangle [id:dp42672371521059915] 
\draw  [dash pattern={on 5.63pt off 4.5pt}][line width=1.5]  (74.03,2763.75) -- (192,2763.75) -- (192,2813.93) -- (74.03,2813.93) -- cycle ;
%Shape: Smiley Face [id:dp9817389082285293] 
\draw  [line width=1.5]  (144.45,2803.6) .. controls (144.45,2801.19) and (146.24,2799.24) .. (148.43,2799.24) .. controls (150.63,2799.24) and (152.41,2801.19) .. (152.41,2803.6) .. controls (152.41,2806.01) and (150.63,2807.97) .. (148.43,2807.97) .. controls (146.24,2807.97) and (144.45,2806.01) .. (144.45,2803.6) -- cycle ; \draw  [line width=1.5]  (146.68,2802.12) .. controls (146.68,2801.88) and (146.86,2801.68) .. (147.08,2801.68) .. controls (147.3,2801.68) and (147.48,2801.88) .. (147.48,2802.12) .. controls (147.48,2802.36) and (147.3,2802.56) .. (147.08,2802.56) .. controls (146.86,2802.56) and (146.68,2802.36) .. (146.68,2802.12) -- cycle ; \draw  [line width=1.5]  (149.39,2802.12) .. controls (149.39,2801.88) and (149.57,2801.68) .. (149.79,2801.68) .. controls (150.01,2801.68) and (150.18,2801.88) .. (150.18,2802.12) .. controls (150.18,2802.36) and (150.01,2802.56) .. (149.79,2802.56) .. controls (149.57,2802.56) and (149.39,2802.36) .. (149.39,2802.12) -- cycle ; \draw  [line width=1.5]  (146.44,2805.35) .. controls (147.77,2806.51) and (149.1,2806.51) .. (150.42,2805.35) ;
%Shape: Smiley Face [id:dp49419175504212776] 
\draw  [line width=1.5]  (179.09,2781.5) .. controls (179.09,2779.09) and (180.87,2777.13) .. (183.06,2777.13) .. controls (185.26,2777.13) and (187.04,2779.09) .. (187.04,2781.5) .. controls (187.04,2783.91) and (185.26,2785.86) .. (183.06,2785.86) .. controls (180.87,2785.86) and (179.09,2783.91) .. (179.09,2781.5) -- cycle ; \draw  [line width=1.5]  (181.31,2780.01) .. controls (181.31,2779.77) and (181.49,2779.58) .. (181.71,2779.58) .. controls (181.93,2779.58) and (182.11,2779.77) .. (182.11,2780.01) .. controls (182.11,2780.25) and (181.93,2780.45) .. (181.71,2780.45) .. controls (181.49,2780.45) and (181.31,2780.25) .. (181.31,2780.01) -- cycle ; \draw  [line width=1.5]  (184.02,2780.01) .. controls (184.02,2779.77) and (184.2,2779.58) .. (184.42,2779.58) .. controls (184.64,2779.58) and (184.81,2779.77) .. (184.81,2780.01) .. controls (184.81,2780.25) and (184.64,2780.45) .. (184.42,2780.45) .. controls (184.2,2780.45) and (184.02,2780.25) .. (184.02,2780.01) -- cycle ; \draw  [line width=1.5]  (181.07,2783.24) .. controls (182.4,2784.4) and (183.73,2784.4) .. (185.05,2783.24) ;
%Flowchart: Punched Tape [id:dp3565745198144521] 
\draw  [fill={rgb, 255:red, 255; green, 255; blue, 255 }  ,fill opacity=1 ] (291.67,2877.34) .. controls (291.67,2880.23) and (301.36,2882.58) .. (313.31,2882.58) .. controls (325.26,2882.58) and (334.95,2880.23) .. (334.95,2877.34) .. controls (334.95,2874.45) and (344.64,2872.11) .. (356.6,2872.11) .. controls (368.55,2872.11) and (378.24,2874.45) .. (378.24,2877.34) -- (378.24,2919.23) .. controls (378.24,2916.34) and (368.55,2913.99) .. (356.6,2913.99) .. controls (344.64,2913.99) and (334.95,2916.34) .. (334.95,2919.23) .. controls (334.95,2922.12) and (325.26,2924.46) .. (313.31,2924.46) .. controls (301.36,2924.46) and (291.67,2922.12) .. (291.67,2919.23) -- cycle ;
%Straight Lines [id:da23451091058783402] 
\draw [line width=1.5]    (320.63,2891.89) -- (349.47,2889.91) ;
\draw [shift={(352.46,2889.7)}, rotate = 176.08] [color={rgb, 255:red, 0; green, 0; blue, 0 }  ][line width=1.5]    (8.53,-2.57) .. controls (5.42,-1.09) and (2.58,-0.23) .. (0,0) .. controls (2.58,0.23) and (5.42,1.09) .. (8.53,2.57)   ;
%Straight Lines [id:da05993633349010663] 
\draw [line width=1.5]    (320.63,2894.07) -- (335.84,2901.48) ;
\draw [shift={(338.53,2902.79)}, rotate = 205.98] [color={rgb, 255:red, 0; green, 0; blue, 0 }  ][line width=1.5]    (8.53,-2.57) .. controls (5.42,-1.09) and (2.58,-0.23) .. (0,0) .. controls (2.58,0.23) and (5.42,1.09) .. (8.53,2.57)   ;
%Shape: Smiley Face [id:dp5316832937595011] 
\draw  [line width=1.5]  (312.91,2893.34) .. controls (312.91,2890.93) and (314.69,2888.98) .. (316.89,2888.98) .. controls (319.09,2888.98) and (320.87,2890.93) .. (320.87,2893.34) .. controls (320.87,2895.75) and (319.09,2897.7) .. (316.89,2897.7) .. controls (314.69,2897.7) and (312.91,2895.75) .. (312.91,2893.34) -- cycle ; \draw  [line width=1.5]  (315.14,2891.86) .. controls (315.14,2891.61) and (315.32,2891.42) .. (315.54,2891.42) .. controls (315.76,2891.42) and (315.94,2891.61) .. (315.94,2891.86) .. controls (315.94,2892.1) and (315.76,2892.29) .. (315.54,2892.29) .. controls (315.32,2892.29) and (315.14,2892.1) .. (315.14,2891.86) -- cycle ; \draw  [line width=1.5]  (317.85,2891.86) .. controls (317.85,2891.61) and (318.02,2891.42) .. (318.24,2891.42) .. controls (318.46,2891.42) and (318.64,2891.61) .. (318.64,2891.86) .. controls (318.64,2892.1) and (318.46,2892.29) .. (318.24,2892.29) .. controls (318.02,2892.29) and (317.85,2892.1) .. (317.85,2891.86) -- cycle ; \draw  [line width=1.5]  (314.9,2895.08) .. controls (316.23,2896.25) and (317.55,2896.25) .. (318.88,2895.08) ;
%Shape: Smiley Face [id:dp5491508300746957] 
\draw  [line width=1.5]  (338.38,2904.97) .. controls (338.38,2902.56) and (340.16,2900.61) .. (342.35,2900.61) .. controls (344.55,2900.61) and (346.33,2902.56) .. (346.33,2904.97) .. controls (346.33,2907.38) and (344.55,2909.34) .. (342.35,2909.34) .. controls (340.16,2909.34) and (338.38,2907.38) .. (338.38,2904.97) -- cycle ; \draw  [line width=1.5]  (340.6,2903.49) .. controls (340.6,2903.25) and (340.78,2903.05) .. (341,2903.05) .. controls (341.22,2903.05) and (341.4,2903.25) .. (341.4,2903.49) .. controls (341.4,2903.73) and (341.22,2903.93) .. (341,2903.93) .. controls (340.78,2903.93) and (340.6,2903.73) .. (340.6,2903.49) -- cycle ; \draw  [line width=1.5]  (343.31,2903.49) .. controls (343.31,2903.25) and (343.49,2903.05) .. (343.71,2903.05) .. controls (343.93,2903.05) and (344.1,2903.25) .. (344.1,2903.49) .. controls (344.1,2903.73) and (343.93,2903.93) .. (343.71,2903.93) .. controls (343.49,2903.93) and (343.31,2903.73) .. (343.31,2903.49) -- cycle ; \draw  [line width=1.5]  (340.36,2906.72) .. controls (341.69,2907.88) and (343.02,2907.88) .. (344.34,2906.72) ;
%Shape: Smiley Face [id:dp21362593128550156] 
\draw  [line width=1.5]  (352.64,2888.69) .. controls (352.64,2886.28) and (354.42,2884.32) .. (356.61,2884.32) .. controls (358.81,2884.32) and (360.59,2886.28) .. (360.59,2888.69) .. controls (360.59,2891.1) and (358.81,2893.05) .. (356.61,2893.05) .. controls (354.42,2893.05) and (352.64,2891.1) .. (352.64,2888.69) -- cycle ; \draw  [line width=1.5]  (354.86,2887.2) .. controls (354.86,2886.96) and (355.04,2886.77) .. (355.26,2886.77) .. controls (355.48,2886.77) and (355.66,2886.96) .. (355.66,2887.2) .. controls (355.66,2887.44) and (355.48,2887.64) .. (355.26,2887.64) .. controls (355.04,2887.64) and (354.86,2887.44) .. (354.86,2887.2) -- cycle ; \draw  [line width=1.5]  (357.57,2887.2) .. controls (357.57,2886.96) and (357.75,2886.77) .. (357.97,2886.77) .. controls (358.19,2886.77) and (358.36,2886.96) .. (358.36,2887.2) .. controls (358.36,2887.44) and (358.19,2887.64) .. (357.97,2887.64) .. controls (357.75,2887.64) and (357.57,2887.44) .. (357.57,2887.2) -- cycle ; \draw  [line width=1.5]  (354.62,2890.43) .. controls (355.95,2891.59) and (357.28,2891.59) .. (358.6,2890.43) ;
%Left Arrow [id:dp22187584774212898] 
\draw   (215,2804.55) -- (220.28,2802) -- (220.28,2803.27) -- (263.54,2803.27) -- (263.54,2805.82) -- (220.28,2805.82) -- (220.28,2807.09) -- cycle ;
%Left Arrow [id:dp1861077704673879] 
\draw   (315.35,2834) -- (317.89,2837.8) -- (316.62,2837.8) -- (316.62,2868.91) -- (314.07,2868.91) -- (314.07,2837.8) -- (312.8,2837.8) -- cycle ;
%Left Arrow [id:dp2590948740182193] 
\draw   (130.55,2868.91) -- (128,2865.11) -- (129.27,2865.11) -- (129.27,2834) -- (131.82,2834) -- (131.82,2865.11) -- (133.09,2865.11) -- cycle ;
%Left Arrow [id:dp7631269314674067] 
\draw   (262.54,2900.55) -- (257.26,2903.09) -- (257.26,2901.82) -- (214,2901.82) -- (214,2899.27) -- (257.26,2899.27) -- (257.26,2898) -- cycle ;
%Shape: Arc [id:dp8010751146858193] 
\draw  [draw opacity=0] (78.55,2898.86) .. controls (77.97,2897.43) and (79.7,2895.07) .. (82.41,2893.59) .. controls (85.13,2892.11) and (87.81,2892.08) .. (88.39,2893.51) -- (83.47,2896.19) -- cycle ; \draw   (78.55,2898.86) .. controls (77.97,2897.43) and (79.7,2895.07) .. (82.41,2893.59) .. controls (85.13,2892.11) and (87.81,2892.08) .. (88.39,2893.51) ;  
%Shape: Arc [id:dp2168479262754166] 
\draw  [draw opacity=0] (79.96,2900.21) .. controls (79.37,2898.78) and (80.79,2896.59) .. (83.12,2895.32) .. controls (85.45,2894.06) and (87.81,2894.19) .. (88.39,2895.63) -- (84.17,2897.92) -- cycle ; \draw   (79.96,2900.21) .. controls (79.37,2898.78) and (80.79,2896.59) .. (83.12,2895.32) .. controls (85.45,2894.06) and (87.81,2894.19) .. (88.39,2895.63) ;  
%Shape: Arc [id:dp1657064934185728] 
\draw  [draw opacity=0] (81.36,2901.56) .. controls (81.36,2901.56) and (81.36,2901.56) .. (81.36,2901.56) .. controls (80.78,2900.13) and (81.88,2898.11) .. (83.82,2897.06) .. controls (85.76,2896) and (87.81,2896.31) .. (88.39,2897.74) -- (84.88,2899.65) -- cycle ; \draw   (81.36,2901.56) .. controls (81.36,2901.56) and (81.36,2901.56) .. (81.36,2901.56) .. controls (80.78,2900.13) and (81.88,2898.11) .. (83.82,2897.06) .. controls (85.76,2896) and (87.81,2896.31) .. (88.39,2897.74) ;  
%Shape: Arc [id:dp6696163073703636] 
\draw  [draw opacity=0] (82.77,2902.92) .. controls (82.77,2902.92) and (82.77,2902.92) .. (82.77,2902.92) .. controls (82.77,2902.92) and (82.77,2902.92) .. (82.77,2902.92) .. controls (82.19,2901.48) and (82.97,2899.63) .. (84.53,2898.79) .. controls (86.08,2897.94) and (87.81,2898.42) .. (88.39,2899.86) -- (85.58,2901.39) -- cycle ; \draw   (82.77,2902.92) .. controls (82.77,2902.92) and (82.77,2902.92) .. (82.77,2902.92) .. controls (82.77,2902.92) and (82.77,2902.92) .. (82.77,2902.92) .. controls (82.19,2901.48) and (82.97,2899.63) .. (84.53,2898.79) .. controls (86.08,2897.94) and (87.81,2898.42) .. (88.39,2899.86) ;  
%Shape: Arc [id:dp5914598807756752] 
\draw  [draw opacity=0] (84.18,2904.27) .. controls (83.6,2902.83) and (84.07,2901.15) .. (85.23,2900.52) .. controls (86.4,2899.89) and (87.81,2900.54) .. (88.4,2901.97) -- (86.29,2903.12) -- cycle ; \draw   (84.18,2904.27) .. controls (83.6,2902.83) and (84.07,2901.15) .. (85.23,2900.52) .. controls (86.4,2899.89) and (87.81,2900.54) .. (88.4,2901.97) ;  

%Image [id:dp3722282424817167] 
\draw (291.67,2795.75) node  {\includegraphics[width=7.64pt,height=13.09pt]{figures/robot.png}};
%Shape: Rectangle [id:dp9197785817800539] 
\draw  [line width=1.5]  (275.37,2763.75) -- (390.8,2763.75) -- (390.8,2813.93) -- (275.37,2813.93) -- cycle ;
%Image [id:dp9715658782589778] 
\draw (382.32,2779.46) node  {\includegraphics[width=7.64pt,height=13.09pt]{figures/robot.png}};
%Image [id:dp635616861971029] 
\draw (352.78,2801.57) node  {\includegraphics[width=7.64pt,height=13.09pt]{figures/robot.png}};
%Shape: Rectangle [id:dp647928357040308] 
\draw  [fill={rgb, 255:red, 0; green, 0; blue, 0 }  ,fill opacity=1 ] (291.67,2769.57) -- (301.85,2769.57) -- (301.85,2781.21) -- (291.67,2781.21) -- cycle ;
%Shape: Rectangle [id:dp9626828362725837] 
\draw  [fill={rgb, 255:red, 0; green, 0; blue, 0 }  ,fill opacity=1 ] (373.15,2792.84) -- (383.33,2792.84) -- (383.33,2804.48) -- (373.15,2804.48) -- cycle ;
%Shape: Ellipse [id:dp6171740062199291] 
\draw  [fill={rgb, 255:red, 0; green, 0; blue, 0 }  ,fill opacity=1 ] (347.69,2775.39) .. controls (347.69,2772.17) and (349.97,2769.57) .. (352.78,2769.57) .. controls (355.59,2769.57) and (357.87,2772.17) .. (357.87,2775.39) .. controls (357.87,2778.6) and (355.59,2781.21) .. (352.78,2781.21) .. controls (349.97,2781.21) and (347.69,2778.6) .. (347.69,2775.39) -- cycle ;
%Shape: Triangle [id:dp8145134127966778] 
\draw  [fill={rgb, 255:red, 0; green, 0; blue, 0 }  ,fill opacity=1 ] (322.22,2792.84) -- (327.31,2804.48) -- (317.13,2804.48) -- cycle ;
%Shape: Rectangle [id:dp07981685971419106] 
\draw  [fill={rgb, 255:red, 0; green, 0; blue, 0 }  ,fill opacity=1 ] (89.45,2769.57) -- (99.64,2769.57) -- (99.64,2781.21) -- (89.45,2781.21) -- cycle ;
%Shape: Rectangle [id:dp9786998324005067] 
\draw  [fill={rgb, 255:red, 0; green, 0; blue, 0 }  ,fill opacity=1 ] (170.94,2792.84) -- (181.12,2792.84) -- (181.12,2804.48) -- (170.94,2804.48) -- cycle ;
%Shape: Ellipse [id:dp6465785854464419] 
\draw  [fill={rgb, 255:red, 0; green, 0; blue, 0 }  ,fill opacity=1 ] (145.47,2775.39) .. controls (145.47,2772.17) and (147.75,2769.57) .. (150.57,2769.57) .. controls (153.38,2769.57) and (155.66,2772.17) .. (155.66,2775.39) .. controls (155.66,2778.6) and (153.38,2781.21) .. (150.57,2781.21) .. controls (147.75,2781.21) and (145.47,2778.6) .. (145.47,2775.39) -- cycle ;
%Shape: Triangle [id:dp5909890868954251] 
\draw  [fill={rgb, 255:red, 0; green, 0; blue, 0 }  ,fill opacity=1 ] (120.01,2792.84) -- (125.1,2804.48) -- (114.92,2804.48) -- cycle ;
%Shape: Smiley Face [id:dp661163164093121] 
\draw  [line width=1.5]  (85.52,2909.38) .. controls (85.52,2906.98) and (87.3,2905.03) .. (89.5,2905.03) .. controls (91.7,2905.03) and (93.48,2906.98) .. (93.48,2909.38) .. controls (93.48,2911.78) and (91.7,2913.73) .. (89.5,2913.73) .. controls (87.3,2913.73) and (85.52,2911.78) .. (85.52,2909.38) -- cycle ; \draw  [line width=1.5]  (87.75,2907.9) .. controls (87.75,2907.66) and (87.93,2907.46) .. (88.15,2907.46) .. controls (88.37,2907.46) and (88.55,2907.66) .. (88.55,2907.9) .. controls (88.55,2908.14) and (88.37,2908.33) .. (88.15,2908.33) .. controls (87.93,2908.33) and (87.75,2908.14) .. (87.75,2907.9) -- cycle ; \draw  [line width=1.5]  (90.46,2907.9) .. controls (90.46,2907.66) and (90.63,2907.46) .. (90.85,2907.46) .. controls (91.07,2907.46) and (91.25,2907.66) .. (91.25,2907.9) .. controls (91.25,2908.14) and (91.07,2908.33) .. (90.85,2908.33) .. controls (90.63,2908.33) and (90.46,2908.14) .. (90.46,2907.9) -- cycle ; \draw  [line width=1.5]  (87.51,2911.12) .. controls (88.84,2912.28) and (90.16,2912.28) .. (91.49,2911.12) ;
%Shape: Rectangle [id:dp9256921796782376] 
\draw  [dash pattern={on 5.63pt off 4.5pt}][line width=1.5]  (74.17,2873.12) -- (192,2873.12) -- (192,2923.15) -- (74.17,2923.15) -- cycle ;
%Shape: Smiley Face [id:dp12230401154700177] 
\draw  [line width=1.5]  (144.6,2912.86) .. controls (144.6,2910.46) and (146.38,2908.51) .. (148.58,2908.51) .. controls (150.77,2908.51) and (152.56,2910.46) .. (152.56,2912.86) .. controls (152.56,2915.26) and (150.77,2917.21) .. (148.58,2917.21) .. controls (146.38,2917.21) and (144.6,2915.26) .. (144.6,2912.86) -- cycle ; \draw  [line width=1.5]  (146.83,2911.38) .. controls (146.83,2911.14) and (147,2910.94) .. (147.22,2910.94) .. controls (147.44,2910.94) and (147.62,2911.14) .. (147.62,2911.38) .. controls (147.62,2911.62) and (147.44,2911.81) .. (147.22,2911.81) .. controls (147,2911.81) and (146.83,2911.62) .. (146.83,2911.38) -- cycle ; \draw  [line width=1.5]  (149.53,2911.38) .. controls (149.53,2911.14) and (149.71,2910.94) .. (149.93,2910.94) .. controls (150.15,2910.94) and (150.33,2911.14) .. (150.33,2911.38) .. controls (150.33,2911.62) and (150.15,2911.81) .. (149.93,2911.81) .. controls (149.71,2911.81) and (149.53,2911.62) .. (149.53,2911.38) -- cycle ; \draw  [line width=1.5]  (146.59,2914.6) .. controls (147.91,2915.76) and (149.24,2915.76) .. (150.57,2914.6) ;
%Shape: Smiley Face [id:dp8847243900502049] 
\draw  [line width=1.5]  (179.23,2890.23) .. controls (179.23,2887.83) and (181.01,2885.88) .. (183.21,2885.88) .. controls (185.4,2885.88) and (187.19,2887.83) .. (187.19,2890.23) .. controls (187.19,2892.63) and (185.4,2894.58) .. (183.21,2894.58) .. controls (181.01,2894.58) and (179.23,2892.63) .. (179.23,2890.23) -- cycle ; \draw  [line width=1.5]  (181.46,2888.75) .. controls (181.46,2888.51) and (181.63,2888.32) .. (181.85,2888.32) .. controls (182.07,2888.32) and (182.25,2888.51) .. (182.25,2888.75) .. controls (182.25,2888.99) and (182.07,2889.19) .. (181.85,2889.19) .. controls (181.63,2889.19) and (181.46,2888.99) .. (181.46,2888.75) -- cycle ; \draw  [line width=1.5]  (184.16,2888.75) .. controls (184.16,2888.51) and (184.34,2888.32) .. (184.56,2888.32) .. controls (184.78,2888.32) and (184.96,2888.51) .. (184.96,2888.75) .. controls (184.96,2888.99) and (184.78,2889.19) .. (184.56,2889.19) .. controls (184.34,2889.19) and (184.16,2888.99) .. (184.16,2888.75) -- cycle ; \draw  [line width=1.5]  (181.22,2891.97) .. controls (182.54,2893.13) and (183.87,2893.13) .. (185.2,2891.97) ;
%Shape: Rectangle [id:dp5525291488755686] 
\draw  [fill={rgb, 255:red, 0; green, 0; blue, 0 }  ,fill opacity=1 ] (89.6,2878.92) -- (99.78,2878.92) -- (99.78,2890.53) -- (89.6,2890.53) -- cycle ;
%Shape: Rectangle [id:dp35042622253694655] 
\draw  [fill={rgb, 255:red, 0; green, 0; blue, 0 }  ,fill opacity=1 ] (171.08,2902.13) -- (181.27,2902.13) -- (181.27,2913.73) -- (171.08,2913.73) -- cycle ;
%Shape: Ellipse [id:dp9658079314838142] 
\draw  [fill={rgb, 255:red, 0; green, 0; blue, 0 }  ,fill opacity=1 ] (145.62,2884.72) .. controls (145.62,2881.52) and (147.9,2878.92) .. (150.71,2878.92) .. controls (153.52,2878.92) and (155.8,2881.52) .. (155.8,2884.72) .. controls (155.8,2887.93) and (153.52,2890.53) .. (150.71,2890.53) .. controls (147.9,2890.53) and (145.62,2887.93) .. (145.62,2884.72) -- cycle ;
%Shape: Triangle [id:dp5926435260290868] 
\draw  [fill={rgb, 255:red, 0; green, 0; blue, 0 }  ,fill opacity=1 ] (120.15,2902.13) -- (125.25,2913.73) -- (115.06,2913.73) -- cycle ;
%Shape: Arc [id:dp2058241396036773] 
\draw  [draw opacity=0] (133.56,2911.66) .. controls (132.26,2911.1) and (132.06,2908.03) .. (133.1,2904.81) .. controls (134.15,2901.58) and (136.04,2899.43) .. (137.34,2899.99) -- (135.45,2905.83) -- cycle ; \draw   (133.56,2911.66) .. controls (132.26,2911.1) and (132.06,2908.03) .. (133.1,2904.81) .. controls (134.15,2901.58) and (136.04,2899.43) .. (137.34,2899.99) ;  
%Shape: Arc [id:dp9303770446429336] 
\draw  [draw opacity=0] (135.39,2911.51) .. controls (135.39,2911.51) and (135.39,2911.51) .. (135.39,2911.51) .. controls (135.39,2911.51) and (135.39,2911.51) .. (135.39,2911.51) .. controls (134.1,2910.95) and (133.77,2908.25) .. (134.67,2905.49) .. controls (135.56,2902.72) and (137.34,2900.94) .. (138.63,2901.5) -- (137.01,2906.51) -- cycle ; \draw   (135.39,2911.51) .. controls (135.39,2911.51) and (135.39,2911.51) .. (135.39,2911.51) .. controls (135.39,2911.51) and (135.39,2911.51) .. (135.39,2911.51) .. controls (134.1,2910.95) and (133.77,2908.25) .. (134.67,2905.49) .. controls (135.56,2902.72) and (137.34,2900.94) .. (138.63,2901.5) ;  
%Shape: Arc [id:dp23450230368676772] 
\draw  [draw opacity=0] (137.22,2911.35) .. controls (137.22,2911.35) and (137.22,2911.35) .. (137.22,2911.35) .. controls (137.22,2911.35) and (137.22,2911.35) .. (137.22,2911.35) .. controls (135.93,2910.79) and (135.48,2908.47) .. (136.23,2906.17) .. controls (136.98,2903.86) and (138.63,2902.45) .. (139.93,2903.02) -- (138.58,2907.19) -- cycle ; \draw   (137.22,2911.35) .. controls (137.22,2911.35) and (137.22,2911.35) .. (137.22,2911.35) .. controls (137.22,2911.35) and (137.22,2911.35) .. (137.22,2911.35) .. controls (135.93,2910.79) and (135.48,2908.47) .. (136.23,2906.17) .. controls (136.98,2903.86) and (138.63,2902.45) .. (139.93,2903.02) ;  
%Shape: Arc [id:dp32480365085094887] 
\draw  [draw opacity=0] (139.06,2911.2) .. controls (139.06,2911.2) and (139.06,2911.2) .. (139.06,2911.2) .. controls (137.76,2910.64) and (137.2,2908.69) .. (137.79,2906.85) .. controls (138.39,2905) and (139.92,2903.97) .. (141.22,2904.53) -- (140.14,2907.87) -- cycle ; \draw   (139.06,2911.2) .. controls (139.06,2911.2) and (139.06,2911.2) .. (139.06,2911.2) .. controls (137.76,2910.64) and (137.2,2908.69) .. (137.79,2906.85) .. controls (138.39,2905) and (139.92,2903.97) .. (141.22,2904.53) ;  
%Shape: Arc [id:dp7108436649867611] 
\draw  [draw opacity=0] (140.89,2911.05) .. controls (139.6,2910.48) and (138.91,2908.91) .. (139.36,2907.53) .. controls (139.8,2906.14) and (141.22,2905.48) .. (142.51,2906.05) -- (141.7,2908.55) -- cycle ; \draw   (140.89,2911.05) .. controls (139.6,2910.48) and (138.91,2908.91) .. (139.36,2907.53) .. controls (139.8,2906.14) and (141.22,2905.48) .. (142.51,2906.05) ;  

%Shape: Arc [id:dp7234198948762418] 
\draw  [draw opacity=0] (171.31,2898) .. controls (171.31,2898) and (171.31,2898) .. (171.31,2898) .. controls (169.93,2898.17) and (168.53,2895.53) .. (168.19,2892.12) .. controls (167.85,2888.7) and (168.7,2885.8) .. (170.08,2885.64) .. controls (170.08,2885.64) and (170.08,2885.64) .. (170.08,2885.64) -- (170.69,2891.82) -- cycle ; \draw   (171.31,2898) .. controls (171.31,2898) and (171.31,2898) .. (171.31,2898) .. controls (169.93,2898.17) and (168.53,2895.53) .. (168.19,2892.12) .. controls (167.85,2888.7) and (168.7,2885.8) .. (170.08,2885.64) .. controls (170.08,2885.64) and (170.08,2885.64) .. (170.08,2885.64) ;  
%Shape: Arc [id:dp05399242918401237] 
\draw  [draw opacity=0] (172.88,2896.92) .. controls (172.88,2896.92) and (172.88,2896.92) .. (172.88,2896.92) .. controls (171.5,2897.09) and (170.15,2894.85) .. (169.86,2891.92) .. controls (169.57,2888.99) and (170.45,2886.49) .. (171.83,2886.32) -- (172.35,2891.62) -- cycle ; \draw   (172.88,2896.92) .. controls (172.88,2896.92) and (172.88,2896.92) .. (172.88,2896.92) .. controls (171.5,2897.09) and (170.15,2894.85) .. (169.86,2891.92) .. controls (169.57,2888.99) and (170.45,2886.49) .. (171.83,2886.32) ;  
%Shape: Arc [id:dp7827225311205266] 
\draw  [draw opacity=0] (174.46,2895.84) .. controls (174.46,2895.84) and (174.46,2895.84) .. (174.46,2895.84) .. controls (173.08,2896) and (171.76,2894.16) .. (171.52,2891.72) .. controls (171.28,2889.28) and (172.2,2887.17) .. (173.58,2887.01) -- (174.02,2891.42) -- cycle ; \draw   (174.46,2895.84) .. controls (174.46,2895.84) and (174.46,2895.84) .. (174.46,2895.84) .. controls (173.08,2896) and (171.76,2894.16) .. (171.52,2891.72) .. controls (171.28,2889.28) and (172.2,2887.17) .. (173.58,2887.01) ;  
%Shape: Arc [id:dp9906438850599013] 
\draw  [draw opacity=0] (176.03,2894.76) .. controls (174.65,2894.92) and (173.38,2893.47) .. (173.19,2891.52) .. controls (172.99,2889.57) and (173.95,2887.86) .. (175.33,2887.69) -- (175.68,2891.23) -- cycle ; \draw   (176.03,2894.76) .. controls (174.65,2894.92) and (173.38,2893.47) .. (173.19,2891.52) .. controls (172.99,2889.57) and (173.95,2887.86) .. (175.33,2887.69) ;  
%Shape: Arc [id:dp545106976508444] 
\draw  [draw opacity=0] (177.61,2893.68) .. controls (177.61,2893.68) and (177.61,2893.68) .. (177.61,2893.68) .. controls (176.23,2893.84) and (174.99,2892.79) .. (174.85,2891.33) .. controls (174.7,2889.86) and (175.7,2888.54) .. (177.08,2888.38) -- (177.34,2891.03) -- cycle ; \draw   (177.61,2893.68) .. controls (177.61,2893.68) and (177.61,2893.68) .. (177.61,2893.68) .. controls (176.23,2893.84) and (174.99,2892.79) .. (174.85,2891.33) .. controls (174.7,2889.86) and (175.7,2888.54) .. (177.08,2888.38) ;  

%Down Arrow [id:dp8971518008111754] 
\draw   (230,2776) -- (232.5,2776) -- (232.5,2764) -- (237.5,2764) -- (237.5,2776) -- (240,2776) -- (235,2784) -- cycle ;


% Text Node
\draw (187.23,2773.35) node  [font=\scriptsize] [align=left] {\begin{minipage}[lt]{8.67pt}\setlength\topsep{0pt}
\begin{center}
{\footnotesize \textbf{\textcolor[rgb]{0.82,0.01,0.11}{?}}}
\end{center}

\end{minipage}};
% Text Node
\draw (152.6,2795.46) node  [font=\scriptsize] [align=left] {\begin{minipage}[lt]{8.67pt}\setlength\topsep{0pt}
\begin{center}
{\footnotesize \textbf{\textcolor[rgb]{0.82,0.01,0.11}{?}}}
\end{center}

\end{minipage}};
% Text Node
\draw (93.53,2791.97) node  [font=\scriptsize] [align=left] {\begin{minipage}[lt]{8.67pt}\setlength\topsep{0pt}
\begin{center}
{\footnotesize \textbf{\textcolor[rgb]{0.82,0.01,0.11}{?}}}
\end{center}

\end{minipage}};
% Text Node
\draw (182.5,2877.5) node  [font=\scriptsize] [align=left] {\begin{minipage}[lt]{8.67pt}\setlength\topsep{0pt}
\begin{center}
{\footnotesize $\displaystyle \mathbf{\textcolor[rgb]{0.82,0.01,0.11}{\pi }\textcolor[rgb]{0.82,0.01,0.11}{_{3}}}$}
\end{center}

\end{minipage}};
% Text Node
\draw (97.6,2901.34) node  [font=\scriptsize] [align=left] {\begin{minipage}[lt]{8.67pt}\setlength\topsep{0pt}
\begin{center}
{\footnotesize $\displaystyle \mathbf{\textcolor[rgb]{0.82,0.01,0.11}{\pi }\textcolor[rgb]{0.82,0.01,0.11}{_{1}}}$}
\end{center}

\end{minipage}};
% Text Node
\draw (358.3,2787.5) node  [font=\scriptsize] [align=left] {\begin{minipage}[lt]{8.67pt}\setlength\topsep{0pt}
\begin{center}
{\footnotesize $\displaystyle \textcolor[rgb]{0.82,0.01,0.11}{(}\mathbf{\textcolor[rgb]{0.82,0.01,0.11}{\pi }\textcolor[rgb]{0.82,0.01,0.11}{_{2}}}\textcolor[rgb]{0.82,0.01,0.11}{)}$}
\end{center}

\end{minipage}};
% Text Node
\draw (368.3,2770.5) node  [font=\scriptsize] [align=left] {\begin{minipage}[lt]{8.67pt}\setlength\topsep{0pt}
\begin{center}
{\footnotesize $\displaystyle \textcolor[rgb]{0.82,0.01,0.11}{(}\mathbf{\textcolor[rgb]{0.82,0.01,0.11}{\pi }\textcolor[rgb]{0.82,0.01,0.11}{_{3}}}\textcolor[rgb]{0.82,0.01,0.11}{)}$}
\end{center}

\end{minipage}};
% Text Node
\draw (299.81,2787.31) node  [font=\scriptsize] [align=left] {\begin{minipage}[lt]{8.67pt}\setlength\topsep{0pt}
\begin{center}
{\footnotesize $\displaystyle \textcolor[rgb]{0.82,0.01,0.11}{(}\mathbf{\textcolor[rgb]{0.82,0.01,0.11}{\pi }\textcolor[rgb]{0.82,0.01,0.11}{_{1}}}\textcolor[rgb]{0.82,0.01,0.11}{)}$}
\end{center}

\end{minipage}};
% Text Node
\draw (154.64,2902.5) node  [font=\scriptsize] [align=left] {\begin{minipage}[lt]{8.67pt}\setlength\topsep{0pt}
\begin{center}
{\footnotesize $\displaystyle \mathbf{\textcolor[rgb]{0.82,0.01,0.11}{\pi }\textcolor[rgb]{0.82,0.01,0.11}{_{2}}}$}
\end{center}

\end{minipage}};
% Text Node
\draw (336.27,2821.93) node  [font=\footnotesize] [align=left] {\begin{minipage}[lt]{83.6pt}\setlength\topsep{0pt}
\begin{center}
\textit{Target environment}
\end{center}

\end{minipage}};
% Text Node
\draw (336,2948) node   [align=left] {\begin{minipage}[lt]{62.53pt}\setlength\topsep{0pt}
\begin{center}
{\footnotesize "blueprints" of}\\{\footnotesize suggested MAS}
\end{center}

\end{minipage}};
% Text Node
\draw (139.36,2933.19) node   [align=left] {\begin{minipage}[lt]{96.97pt}\setlength\topsep{0pt}
\begin{center}
\textit{{\footnotesize Simulated environment + Trained agents}}
\end{center}

\end{minipage}};
% Text Node
\draw (138.5,2824.5) node  [font=\footnotesize] [align=left] {\begin{minipage}[lt]{93.02pt}\setlength\topsep{0pt}
\begin{center}
\textit{Simulated environment}
\end{center}

\end{minipage}};
% Text Node
\draw (171.5,2848.5) node  [font=\footnotesize] [align=left] {\textbf{2) Training}};
% Text Node
\draw (357.3,2848.5) node  [font=\footnotesize] [align=left] {\textbf{4) Transfer}};
% Text Node
\draw (348.3,2877.5) node  [font=\scriptsize] [align=left] {\begin{minipage}[lt]{8.67pt}\setlength\topsep{0pt}
\begin{center}
{\footnotesize $\displaystyle \mathbf{\textcolor[rgb]{0.82,0.01,0.11}{\pi }\textcolor[rgb]{0.82,0.01,0.11}{_{3}}}$}
\end{center}

\end{minipage}};
% Text Node
\draw (354.3,2900.5) node  [font=\scriptsize] [align=left] {\begin{minipage}[lt]{8.67pt}\setlength\topsep{0pt}
\begin{center}
{\footnotesize $\displaystyle \mathbf{\textcolor[rgb]{0.82,0.01,0.11}{\pi }\textcolor[rgb]{0.82,0.01,0.11}{_{2}}}$}
\end{center}

\end{minipage}};
% Text Node
\draw (305.3,2886.5) node  [font=\scriptsize] [align=left] {\begin{minipage}[lt]{8.67pt}\setlength\topsep{0pt}
\begin{center}
{\footnotesize $\displaystyle \mathbf{\textcolor[rgb]{0.82,0.01,0.11}{\pi }\textcolor[rgb]{0.82,0.01,0.11}{_{1}}}$}
\end{center}

\end{minipage}};
% Text Node
\draw (236.43,2889.27) node  [font=\footnotesize] [align=left] {\textbf{3) Analyze}};
% Text Node
\draw (233.5,2792.5) node  [font=\footnotesize] [align=left] {\textbf{1) Modeling}};


\end{tikzpicture}
        \caption{Cycle de vie d’un SMA conçu avec MAMAD}
        \label{fig:cycle}
      \end{figure}
    \end{column}
  \end{columns}


\end{frame}

\section{Détails des contributions par article}

\begin{frame}{Organisation des contributions}

  \begin{itemize}
    \item Chaque article de conférence présente une ou plusieurs des contributions s'intégrant dans la réponse générale proposée et peuvent se superposer en partie entre elles ;
    \item Chaque article de conférence évalue ces contributions sur un ou plusieurs cas d'application spécifique ;
    \item L'article de journal présente la méthode qui orchestre toute ces contributions comme réponse, voulue comme complète, de la question de recherche
  \end{itemize}

  \

  \begin{center}
    \begin{table}[h!]
      \begin{tabular}{ c c c c c c c }
        \textbf{Article}                   & \multicolumn{2}{c}{\textbf{Modélisation}} & \multicolumn{2}{c}{\textbf{Entrainement}} & \textbf{Analyse} & \textbf{Transfert}                         \\
        \cmidrule(lr){2-3} \cmidrule(lr){4-5}
                                           & Manuel                                    & Automatisé                                & Vanilla          & Contraint          &           &           \\
        \midrule                                                                                                                                                                                   \\
        SMC                                & \ding{51}                                 & \ding{55}                                 & \ding{51}        & \ding{55}          & \ding{55} & \ding{55} \\
        AIAI                               & \ding{51}                                 & \ding{55}                                 & \ding{55}        & $\sim$             & $\sim$    & \ding{55} \\
        $\text{CAID}_{\lnot \text{pub.}}$  & \ding{51}                                 & \ding{55}                                 & \ding{55}        & \ding{51}          & \ding{55} & \ding{55} \\
        $\text{PRIMA}_{\lnot \text{pub.}}$ & \ding{51}                                 & \ding{55}                                 & \ding{55}        & \ding{51}          & $\sim$    & $\sim$    \\
        AAMAS                              & \ding{55}                                 & \ding{55}                                 & \ding{55}        & \ding{51}          & \ding{51} & \ding{55} \\
        CLOUD                              & \ding{55}                                 & \ding{51}                                 & \ding{55}        & \ding{51}          & \ding{51} & \ding{51} \\
        Journal                            & \ding{51}                                 & \ding{51}                                 & \ding{51}        & \ding{51}          & \ding{51} & \ding{51}
      \end{tabular}
      \caption{Couverture des contributions de la méthode MAMAD (journal) par les articles de conférence}
    \end{table}
  \end{center}

\end{frame}

\begin{frame}{Assisting Multi-Agent System Design with
    MOISE+ and MARL: The MAMAD Method (\textbf{Journal})}
  Voir journal + corrections
\end{frame}

\begin{frame}{Towards a Multi-Agent Simulation of Cyber-attackers and Cyber-defenders Battles (\textbf{SMC})}
  Voir papier + présentation
\end{frame}

\begin{frame}{A MARL-Based Approach for Easing MAS Organization Engineering (\textbf{AIAI})}
  Voir papier + présentation
\end{frame}

\begin{frame}{An organizationally-oriented approach to enhancing explainability and control in multi-agent reinforcement learning (\textbf{AAMAS})}
  Voir papier + présentation
\end{frame}


\begin{frame}{An Organization-oriented MARL Approach for Cyberdefense in a Drone Swarm Scenario (\textbf{CAID})}
  Voir papier
\end{frame}

\begin{frame}{Streamlining Resilient Kubernetes Autoscaling with Multi-Agent Systems via an Automated Online Design Framework (\textbf{CLOUD})}
  Voir papier
\end{frame}

\section{Planning thèse}
\begin{frame}{Planning thèse}

  \begin{itemize}
    \item Juin
          \begin{itemize}
            \item Finaliser le manuscrit (fin Juin)
          \end{itemize}
    \item Juillet
          \begin{itemize}
            \item Envoyer première version
          \end{itemize}
    \item Août
          \begin{itemize}
            \item Rien
          \end{itemize}
    \item Septembre
          \begin{itemize}
            \item Retours, prise en compte remarques
            \item Entrainement oral pour défense (fin Septembre)
          \end{itemize}
    \item Octobre
          \begin{itemize}
            \item Entrainement oral pour défense (début Octobre)
            \item Défense (mi-Octobre)
          \end{itemize}
  \end{itemize}

\end{frame}

\section{Manuscrit}

\begin{frame}{Des articles au manuscrit}
  \begin{itemize}
    \item En général : Le journal sur la méthode MAMAD donne la structure générale de la contribution pour répondre à la question de recherche $\rightarrow$ Chapitre \textbf{Méthode de conception d'un SMA de Cyberdéfense}
    \item Les articles de conf sont des cas d'application particuliers $\rightarrow$ Chapitre \textbf{Experimentation et évaluation}
  \end{itemize}
\end{frame}

\begin{frame}[allowframebreaks]{Plan}

  \begin{enumerate}

    \item Introduction
          \begin{enumerate}
            \item[1.1] Un contexte de Cyberdéfense avec des défis futurs et nouveaux
            \item[1.2] L’idée d’un système multi-agents de Cyberdéfense
            \item[1.3] Question de recherche et objectifs de la thèse
            \item[1.4] Organisation du manuscrit
          \end{enumerate}

    \item vers un système multi-agents de cyberdéfense
          \begin{enumerate}
            \item[2.1] Concepts dans les systèmes multi-agents et l’organisation
            \item[2.2] Un état de l’art en matière de Cyberdéfense distribuée ou décentralisée
            \item[2.3] Synthèse et identification des verrous théoriques
          \end{enumerate}

    \item 3 problèmes détaillées et hypothèses de travail
          \begin{enumerate}
            \item[3.1] Problèmes à considérer
            \item[3.2] Hypothèses et contributions proposées
          \end{enumerate}

    \item 4 méthode de conception de systèmes multi-agents de cyberdéfense
          \begin{enumerate}
            \item[4.1] Vers un modèle formel des systèmes multi-agents de Cyberdéfense
            \item[4.2] Vers une approche de développement de systèmes multiagents de Cyberdéfense
            \item[4.3] The MAMAD method
            \item[4.4] CybSMADE : Environnement de développement de systèmes multi-agents de Cyberdéfense
          \end{enumerate}

    \item Experimentation et évaluation
          \begin{enumerate}
            \item[5.1] Intégration d’un agent AICA dans CybSMADE
            \item[5.2] Expériences à travers trois études de cas
            \item[5.3] Un scénario d’infrastructure d’entreprise
            \item[5.4] Un scénario d’essaim de drones
            \item[5.5] Un scénario d’orchestration Kubernetes
            \item[5.6] Résultats et discussion
            \item[5.7] Tendances générales qui ressortent des résultats et de la synthèse
          \end{enumerate}

    \item Conclusion et travaux futurs
          \begin{enumerate}
            \item[6.1] Vers une application industrielle pour la prise de décision en matière de Cyberdéfense
            \item[6.2] Limitations et perspectives
          \end{enumerate}

  \end{enumerate}

\end{frame}

\begin{frame}{Etat d'avancement}

  Voir pdf actuel

\end{frame}

\section{Démo : MOISE+MARL}


\appendix
%\setbeamertemplate{headline}{}
\setbeamertemplate{mini frames}{}

% \AtBeginSection[]{
% 	\begin{frame}
% 		\frametitle{}
% 		\tableofcontents[currentsection]
% 	\end{frame}
% }

% %%%%%%%%%%%%%%%%%%%%%%%%%%%%%%%%%%%%

\section*{\phantom{Thanks}}

\begin{frame}{}

  \vspace{6ex}

  \centering
  {
    \Huge
    \emph{Thank You}
  }

  \vspace{6ex}

  \begin{columns}

    \hspace{-27ex}

    \begin{column}{0.5\textwidth}
      \raggedleft
      {\Large Demo video $\Longrightarrow$}
    \end{column}

    \hspace{-12ex}

    \begin{column}{0.5\textwidth}
      \includegraphics[width=0.5\linewidth]{figures/demo_qr_code.png}
    \end{column}

  \end{columns}

  \vspace{3ex}

  \centering
  {\Large
    \url{https://t.ly/4JBxr}
  }

\end{frame}

% \AtBeginSection[]{
% 	\begin{frame}
% 		\frametitle{}
% 		\tableofcontents[currentsection]
% 	\end{frame}
% }

% %%%%%%%%%%%%%%%%%%%%%%%%%%%%%%%%%%%%

\section*{\phantom{References}}

\begin{frame}[allowframebreaks]{References}{}

    % \bibliographystyle{plain}
    % \bibliography{local_references}
    \printbibliography

\end{frame}

\newcounter{mainframenumber}
\setcounter{mainframenumber}{\value{framenumber}}

% \begin{frame}{Annexes}
    {Context}

    \begin{block}{Multi-Agent Systems (MAS) paradigm for complex \& distributed problems}
        \begin{itemize}
            \item \textbf{task decomposition}: missions delegated to agents achieved through cooperation~\cite{Raileanu2023};
            \item \textbf{benefits}: handle conflicting goals, parallel computation, system robustness, scalability\dots
        \end{itemize}
    \end{block}

    \begin{block}{\textbf{Organization}: key for MAS designing}
        \begin{itemize}
            \item \textbf{coordination}: how to collaboratively achieve a common goal~\cite{Hubner2007};
            \item \textbf{dynamic \& uncertain environments}: flexible runtime behavior to adapt~\cite{Kathleen2020};
        \end{itemize}
    \end{block}

    \begin{block}{Methods and practice for MAS design}
        \begin{itemize}
            \item \textbf{approach + organizational model}: methods rely on designers' experience to hand-craft agents' \textbf{policies} so resulting MAS achieve goals;
                  %   \begin{itemize}
                  %       \item Examples: \emph{GAIA}~\cite{Wooldridge2000,Cernuzzi2014}, \emph{ADELFE}~\cite{Mefteh2015}, or \emph{DIAMOND}~\cite{Jamont2015}, \emph{KB-ORG}~\cite{Sims2008}
                  %   \end{itemize}
            \item \textbf{simulation to reality}: 1) safe \& efficient MAS design in high fidelity simulated environment; \quad 2) transfer to real environment to perform adequately~\cite{Schon2021}.
        \end{itemize}
        \vspace{1ex}
        \quad $\Longrightarrow$ \textbf{Iterative process proceeding by trial and error}

    \end{block}

\end{frame}

\begin{frame}{Annexes}
    {MAS basics}

    \begin{block}{Keywords}
        \begin{itemize}
            \item \textbf{Agent}: entity immersed in an environment perceiving observation and making decision autonomously to achieve some goals;
            \item \textbf{MAS}: a set of agents collaborating with self/re-organizing mechanisms to achieve their goal;
            \item \textbf{Organization}: the agents' interactions even though it may be implicit;
            \item \textbf{Organizational Model (OM)}: medium to formally describe an explicit/implicit organization;
            \item \textbf{Organizational Specifications (OS)}: components of an OM to characterize an organization
        \end{itemize}
    \end{block}

    \begin{block}{Organizational model: $\mathcal{M}OISE^+$}
        \begin{itemize}
            \item more complex than \emph{Agent Group Roles} (integration of standards);
            \item takes into account the social aspects between agents explicitly;
            \item possible to link agents' policies to organizational specifications.
        \end{itemize}
    \end{block}

\end{frame}

\begin{frame}{Annexes}
    {MARL basics}

    \begin{block}{Keywords}
        \begin{itemize}
            \item \textbf{Policy}: the \textquote{logic} to choose next action according to observation for an agent;
            \item \textbf{History/trajectory}: the tuple of (observation, action) couples over an episode;
            \item \textbf{Joint-policy / Joint-history}: all of the agents' policies / histories as tuples;
            \item \textbf{Reinforcement learning}: an agent updates its policy to maximize a cumulative reward;
            \item \textbf{Multi-Agent Reinforcement Learning (MARL)}: extends to multiple agents that learn while considering the actions of other agents;
        \end{itemize}
    \end{block}

\end{frame}



\end{document}
